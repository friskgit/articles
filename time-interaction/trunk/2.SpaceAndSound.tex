\section{Space and Sound}
\label{sec:spaceandsound}

Aural and embodied modes of perception obviously play an important role in all phases of musical creation, from conception (and composition) through performance to reception. Furthermore, these modes appear to be different from the otherwise so dominating visual modes of communication where the breach between the \emph{observer} and \emph{the observed}---between the seing subject and the observed object\footcite[For an artistic investigation on the topic, see][]{leiderstam06}---appears to be different from the activities and modalities of the listening ear. Where the eye explores and reaches out, the ear is the passive receptor that ``perceives the whole as one''\footnote{\cite{berendt90} as cited in \cite[139]{cobussen08}}.
But this is not to say that listening transcend or move away from spatial properties. On the contrary, they interact with them in ways that may be said to simulate some aspects of visual perception. The maelstrom of sonic stimuli that we are constantly exposed to help us understand the spaces that surround us, they help us stratify them, discern distances and identify movements within them entirely without looking. With eyes closed we can get an aural `image' of the room we currently inhabit---its size and the facing of its walls, the height of the ceiling and the type of floor---just by listening to its vibrancies.  Sounds whose trajectories intercede those of a particular space pick up some of the properties of the space, almost as if the room rubs off itself unto the sound, and carries some information about the nature of the space with it. Space affects and adds colour to any sound whether produced within it or merely travelling through it. The sound of footsteps is different if the boot steps on tiles or on a carpet but also if it does so in a church or in a small room. 
%But in either case the \emph{sound} signifies both an action (`walking') and a materiality (`tiles in a church') and listening to it, perceiving the sound, (re)creates the spaces through which it has travelled. 
This sound, the sound of footsteps, tells something about the environment in which it emmanted but also something about the action that took place to produce it. Listening to it, perceiving the sound, (re)creates the spaces through which it has travelled. 

Eric Clarke, in his book \citetitle{clarke05}, similarily points to how a given sound object refers, not to one, but to many different symbols. In an ecological sense the perceptual meaning of a sound is a result of a mutual relation between that which is specified by the environment, and the knowledge and experience of the perceiver.\footnote{Clarke describes musical ecology as an approach to perception that ``offers an alternative view that gives a coherent account of the directness of listeners' perceptual responses to a variety of environmental attributes, ranging from the spatial location and physical source of musical sounds, to their structural function and cultural and ideological value.'' \cite[46]{clarke05}} The sound specifies objects that are particular to the origin of the sound, real or virtual. A musical sound for example specifies the instrument on which it was played but also loudness---relative as well as absolute---its quality and relative change---dynamic or static. Hence, given that the perceiver has some understanding of Western music and knows something about the violin, the perceptual meaning of a violin sound, apart from the instrument itself, may be how the sound is produced (bowed or plucked) and the kind of event the instrument (or the musician more correctly) is performing (playing a song, practicing a scale, etc.). But it also specifies \emph{where} it was played (concert hall, practice room, outdoors, etc) which would imply that the sound carries with it information pointing to itself but also information pointing outside itself, that points to the space in which it was created.\footnote{As noted above, an ecological system rests on the fact that the perceiver is able to resonate with the received object.} The sound as an object perceived by a subject: A description of listening that approaches the \emph{voyeur}, of the detached eye that can see without being seen? But the word `object' is misguiding here. A sound is not first and foremost only an object (though it may well be understood as one). A sound exist in time and as such it can only be an object in memory, never in real life. It's minute changes over time is what specifies all the information carried by it: Sound by its very definition is a function of time, a continuous series of fluctuations. And, though it is true that these audio fluctuations of a given `object' may vary to quite some degree before the perception is distorted, if events are rearranged in time the perceptual meaning may change rather rapidly. Or, put differently, a well known melody played with a drastic and unidiomatic vibrato will not as easily alter the perception of the melody as long as the order of the notes and the rhythm remains the same.

If we return to the foot steps in the church, if the rhythm is changed, if the slow, steady sound of a man walking is replaced by a random distribution of foot steps, though the timbral properties of each individual step sound has not changed, the object (the object being `footsteps in church') has been distorted. And yet the way the object signifies the space remains the same: Even though the audible image of a man walking is lost, the properties (or invariants) of the space in which the sounds were produced reamins the same and, ecologically speaking, the better attuned we are and the more refined our inner resonances are, the more precise our `aural vision' is. The subject understands the world through sound in time and is able to \emph{also} understand the world of the sound through listening. 
As a musician, I am listening not only to `external' sounds\footnote{This is an extremely important aspect of being musician.} but also to `internal' sounds, sounds that I produce myself and that, for the most part, I have some kind of understanding of \emph{before} they exist, they have been played.
The musician as a listener emitting sounds that return back to its producer, carrying with it information about what processes were applied to it, what surfaces it encountered and how these surfaces are layed out. The delay between production and reception tells something about the size of the space, the loudness and dispersion something about the quality of the space. Play to hear the space. Play to explore the space. (Jazz)musicians are often talking about the importance of learning to ``play the room'': I.e. to adopt their style of playing to the space. A drummer can not play the same way in a church as he or she can in a club and a trumpeter can use a different range of dynamics playing outdoors as compared to playing in a tight chamber music hall. The players perceive the space through sound but are also creating and altering the space through their playing.

Space as heard through sound. Sound creates space. But, as John Cage proved when he entered the anechoic chamber at Harvard University in the 50's, space also creates sound. An anechoic chamber is a room void of any resonance whatsoever, no echoes at all: A silent room. Already the fact that such a space needs to be constructed may be taken as a proof that space also creates sound, that space interferes with the true sonic qulaity of a give sound source. But Cage, when entering the room discovered it was no silent at all. He went out and told the engineer he had heard two sounds while inside; one high and one low: ``When I described them to the engineer in charge, he informed me that the high one was my nervous system in operation and the low one my blood in circulation.''\footcite[8]{cage61} The room made it possible to experience sounds otherwise masked and these sounds revealed that behind what is referred to as silence hides a new sonic universe. ``There is no such thing as an empty space or an empty time''\footcite[8]{cage61} and there is no space without a sound.

The more sensible our perception, the more accurate is our understanding when we look with our ears. Wether we are fully aware of it or not, spatial apperception and hearing are closely coupled. Space is received through sound and sound is received through space.




%%% Local Variables: 
%%% mode: latex
%%% TeX-master: "InteractivityTimeSpace"
%%% End: 
