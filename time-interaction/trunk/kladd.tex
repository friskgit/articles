%\footnote{The MP3 compression algorithm changes the fluctuations of an audio signal just enough to reduce its storage, but not more than for the listener to refer original audio is }

%And, although there are many minute varitations in the sounds performed on the violin, there are also invariant properties that do not change over time. 

%However, if any of the invariants of a sound start changing or gets altered---the notes are plucked rather than bowed---the resulting event (i.e. the perceived `object') will also change. Analogically, i

%In the case of the foot steps in the church, these too have invariants that, if changed, distorts the object (the object being `footsteps in church'). 


That sound may carry spatial information is well known by sound engineers and, above all, by \emph{sound designers}. Sound designers create illussions of space through sound by either manufacturing the sound and the ambience desired, or by capturing the sound in the environmnent aimed at.
Once the sound is created it is injected into the virtual world where it, at its best, creates the desired illusion. Sound design is used in a vide variety of businesses and everything from the sound of a car door closing, the click of a camera shot to gun shot sounds or car crash noises in movies is likely to have been carefully fabricated with the primary purpose of creating an illusion of spatial properties in the perceived object. 

Sound design relies on an ecological approach to listening that allows us to see musical perception as the subject's resonance to features immanent to the sound rather than to the subject listening. 

And this kind of perception is of a different order, distinct from visual modes of perception. 

There is a certain ontological breach between the seeing subject and the seen object. 



%%%%%%%%%%%%%%%%%%%%%%%%%%%%%%%%%%%%%%%%%%%%%%%%%%

Det otidsenliga: något virtuellt i förhållande till den vanliga tiden, det som agerar ``\emph{mot} tiden, därigenom på tiden, och förhppningsvis till förmån för en kommande tid'' (Nietsche i \emph{Otdisenliga betraktelser}). (Om historiens nytta och skada: en otidsenlig betraktelse. Övers Alf W Johansson. Stockholm: Raben Prisma, 1998)


individen kan fattas som en koposit av pre-individuella singulariteter, nämligen som en lokal effekt av det ontologiska sållet.

Sven-Olov Wallenstein, förord till Deleuze, Gilles, Vecket: Leibniz och barocken. Bokförlaget Daidalos (2004)
%%%%%%%%%%%%%%%%%%%%%%%%%%%%%%%%%%%%%%%%%%%%%%%%%%


the timbre of the sound in order for the listener to `hear the space'. The sound carries information about the space and this information resonates within the listener for whom the audible image of the environment appears. But in Clarke's examples the primary purpose is to investigate the listeners perception of the sound when the source of the sound is distinct from the listener and they are separate in space. The musician emitting sounds into the space is however similarly exploring the space by perceiving the difference in timbre between sounds emitted and sounds returned, constantly adopting to minute changes in reverberance and resonance. 

As a musician I am also always a listener and as a listener, although I may be surprised by the sounds I produce, I also listen to the space but part of the carrier for the information is my own sound as emitted by me. 

 meaning that their playing have to adjust according to the layout of the room; The space is created by the sound also subconsciously: We receive space through sound. The maelstrom of sonic stimuli that we are constantly exposed to help us understand the spaces that surround us, they help us stratify them, discern distances and identify movements within them---all this without falling back on visual stimuli. Composers and artists that have shown particular interest in the sound-space relation are, to only mention a few, Richard Wagner, Iannis Xenakis and sound artist Michael Brewster. Wagner spoke of ``the weight of sound'' and the gravity of harmony. Xenakis devoted much of his artistic life, as an architect as well as a composer, to issues concerning not only sound and space, but also light and movement, and defined the composer as someone who is a ``thinker and plastic artist who expresses himself through sound beings.''\footcite[][255]{xenakis71} Brewster, with his sound sculptures, more consciously introduces interactivity and the user/listener in his creation of space altering works.\footcite[][Ch. 11]{labelle06}
% (Cage story about high an low sounds?)

%As an improviser I'm always existing in a multitemporality whichoffers 

John Cage proved tho opposite

But the opposite is also true as was proved by John Cage in 

As stated by John Cage ``There is no such thing as an empty space or an empty time.''\footcite[8]{cage61} If there is no sound 

A more extreme example of the space--sound relation is John Cage's famous recollection of how he entered 

In music time \emph{creates} space.

%%% Local Variables: 
%%% mode: latex
%%% TeX-master: "InteractivityTimeSpace"
%%% End: 
