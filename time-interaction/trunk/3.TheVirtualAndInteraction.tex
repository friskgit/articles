
\section*{Interacting with the Virtual}
\label{sec:inter-with-virt}

If (musical) time is a (non)contiguous chain of (causal) events (notes, spaces, variations, etc.) these may be gathered together causing a perceptual entity such as `a melody'. The memory of such an entity may further be seen as the spatial (and virtual) translation of this chain. In other words, a virtual representation of a chain of musical stimuli may be held in memory and, once stored, it may be understood as a single unit. This is similar to how, in the previous section, it was discussed how the perceptual meaning of a sound may refer to the kind of event produced. If a saxophone is playing a well known tune, the sound of the saxophone signifies several objects of which one is the tune played. If I as a listener knows it, I can keep a virtual representation of it in memory and understand it as a single unit as well as a continuous series of events. It is then possible to move through, or navigate, a known piece of music as if it constituted a space, a space fed with the sounds of our memories. 

The almost mystical sensation of simultaneously being able to be in time, `now' and in memory, in the recollection of a previous now, is an important and powerful aspect of time-based arts in general and music in particular. In his book \citetitle{ricoeur04} Paul Ric{\oe}ur critically discusses Husserl's concept of \emph{retention} and \emph{reproduction} (the full meaning of which however is far beyond the scope of this chapter) in his survey of memory and imagination. The sounding note makes the now perceptible, but as it continues to sound it is an ever new now wheras previous ones arew sinking back into memory, into \emph{retention}.\footcite[Part 1, Ch. 1:4]{ricoeur04} Perhaps we can imagine an interaction between retention and recollection in which new notes, held on to and kept in retnetion, are compared to recollected notes from previous experiences. While the now continously begins and recedes the recollection is brought to the surface constructing a `new' now.

May memory be the link between time and space? According to Bergson we can look at the present-past axis as an inverted cone where the tip is the present and the base represents the oldest unconscious memories. Each segment of this cone, where a segment is a line trisecting the cone somewhere between its point and its base, represents a virtual plane.\footcite[][Ch.3]{bergson91} According to Deleuze each such segment contains the totality of past \emph{and} the present in different levels of contraction or relaxation.\footcite[][60]{deleuze88} Interesting in this context is the notion that the plane on which the cone stands is moving, which would suggest that the level of contraction at any other virtual plane would be constantly changing. In other words, the virtual plane of memory is not a static \emph{image} (space) of a time in the past (a sound), but a constantly changing one: The relation between time and space is in all regards a dynamic one.

% the listener knows it, once it is recognized it constitutes a space % for the listener to navigate.

These virtual planes of memory are likely to be linked to the virtuality present in all musical creation, and common to both the virtual planes of the memory and to the virtuality of music is that, although they generate visual elements and images, they do not depend on them.
But in real-time, in performance, in improvisation---in the spur of the moment---the leap from `now' to a virtual plane of the past may create a breach. An ontological difference that, in the best case, fuels a musical performance and takes it to new heights, but at its worst, detaches the performance from the logic of the `now', and the virtual plane fails to actualize itself and remains trapped in the memory of the performer(s). These leaps may be carried out as a result of affectivity within the performer or simply by aesthetic choices. The subject-object continuum, particularly so in group improvisation, may play an important role. 
All of these cases, studied from the performers point of view, will be important areas of inquiry for this chapter. % inspired by Deleuze's analysis of Bergson.

But we approach the question of the sometimes overly simplified subject-object continuum by way of the already mentioned real-virtual divide. Human-Computer Interaction (HCI) is a field of research informed primarily by how users (humans) can come to interact (use) computers and other kinds of technology as effortlessly as possible. For some reason the human voyage into the virtual worlds of present day technology and information processing should take place without strenuous endeavor. As if the otherwise fairly well established relation between ease of use and level of user influence did not appertain to the world of the virtual (or, perhaps, precisely \emph{becauses} of it). Technology, it is argued, has entered our lives to make things easier and must, in order to fulfil its own purpose be easy to use. This is as troublesome in the real (virtual) world as it is in the musical virtual world: The (electronic) musical instrument that may be learned and mastered in ten minutes is not likly to be able to sustain the users' interest for much more than 20 minutes.\footcite[An example of an instrument that does not obey this rule is the CrackleBox. See][]{waisvisz75} Unfortunately, the opposite is as troublesome: The virtual musical instrument that allows for such minute control and variation as the violin offers its trained practitioner will quickly become so complicated that no one will bother to learn it. Which is however not the same as to assert that an interface must be simple in order to be useful (the violin is certainly not simple to learn, nor to use). One of the reasons the violin is a succesful controller of musical intention is the way it interfaces with the human body of the player in a very efficient way and the very reason, perhaps the only reason, our hypothetical virtual instrument becomes a usability disaster is to do with the body-machine boundary.\footnote{There is much interesting research going on in this field and lots of progress is being done. See in particular the research performed at Canadian McGill university (\cite{wanderley09,wanderley00}) and recent PhD dissertation \cite{jensenius08}}

%%% Local Variables: 
%%% mode: latex
%%% TeX-master: "InteractivityTimeSpace"
%%% End: 
