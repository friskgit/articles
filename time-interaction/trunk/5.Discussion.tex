
\section*{Discussion}
\label{sec:discussion}

The `real' virtualities---those that belong to the digital domains---are in a way, at least for the musician, a means for bypassing these leaps into memory. Machines are phenomenal individual forgetting devices\footcite[See][]{miller04} partly due to their lack of an embodied relation to their operators (envision the four members of Kraftwerk in front of their keyboards/laptops) further diffusing the subject-object divide. These digital virtualities are truly non-ocular in their complete absence of a consented visual component. (After all, this is one of their great qualities; one is able to mold one's own virtual visuality) But herein lies also our problem: 
How can these (non-)spaces, the aural and virtual, be truthfully represented outside their own domains? How may they be documented and communicated in the context of AbR?

These questions become even more acute when dealing with real-time technologies in music, because the real-time `object', the music that is a result of real-time processes such as improvisation, live coding, interpretation, etc. is made up of a volatile substance that is not easily transformed to a researchable `object'. While investigating how the virtual sound worlds of computer instruments, created and edited in real-time, may interact with (or fail at interacting with) the real world, we will address these and other questions, all of which are intimately linked to issues of time and space. Although the Western tradition has developed powerful musicological methods to represent and document music visually,\footcite[See]{bregman94} independent of time, are there methods that retain the temporal identity of the object rather than do away with it? Such means of representation may have to include also questions rarely explored in AbR such as the impact of contextuality and listener reception in musical performances. Also more multi disciplinary research projects will here be of great interest, in particular those that actively deal with visuality in different forms such as computer games and film, in particular considering the great impact video works has had on the electro acoustic music community over the last decade.

%%% Local Variables: 
%%% mode: latex
%%% TeX-master: "InteractivityTimeSpace"
%%% End: 
