% Created 2020-10-19 mån 11:00
% Intended LaTeX compiler: pdflatex
\documentclass[11pt]{article}
\usepackage[utf8]{inputenc}
\usepackage[T1]{fontenc}
\usepackage{graphicx}
\usepackage{grffile}
\usepackage{longtable}
\usepackage{wrapfig}
\usepackage{rotating}
\usepackage[normalem]{ulem}
\usepackage{amsmath}
\usepackage{textcomp}
\usepackage{amssymb}
\usepackage{capt-of}
\usepackage{hyperref}
\usepackage[english]{babel}
\usepackage[style=authoryear-ibid,natbib=true,backend=biber,hyperref=false]{biblatex}
\bibliography{/Users/henrik_frisk/Dropbox/Documents/articles/biblio/bibliography.bib}
\author{Henrik Frisk}
\date{ }
\title{About \emph{pvm}}
\hypersetup{
 pdfauthor={Henrik Frisk},
 pdftitle={Description},
 pdfkeywords={},
 pdfsubject={},
 pdfcreator={Emacs 26.3 (Org mode 9.4)}, 
 pdflang={English}}
\begin{document}

\maketitle
\noindent
\emph{pvm} is an improvisation based on interactions with the Vietnamese master musician Pham Van Mon. These interactions were carried out in Vietnam on numerous trips to the southern parts of the country, on line in virtual presence interaction, and in a manner that involved sending material back and forth. The material has been further developed in online performances in concerts in Sweden and in Hanoi. This piece is part of the Transformations project, an artistic research project that investigates the impact of musical traditions in transformation, and which involves the Vietnamese-Swedish group The Six Tones and several other collaborators, including Pham Van Mon.\\

This structured improvisation was made using software that I have composed myself in PureData, Max, Supercollider and Faust. I play on a QuNeo physical controller and a laptop following ideas concerning laptop performance that I have developed over a number of years. The general philosophy is to layer tools and software in a complex system that dismantles some of the idiomatic traits of each piece of software by itself. In addition, the general complexity of the system adds a certain indeterminacy to it that encourages improvisatory strategies. Furthermore, some of the material is performed by the software KimAuto (KA), developed by Norwegian center Notam within the artistic research project Goodbye Intuition.\footnote{Frisk, H. (2020). Aesthetics, interaction and machine improvisation. \emph{Organised Sound}, 25(1), 33–40. \url{http://dx.doi.org/10.1017/S135577181900044X}\\} KA is an automated improvisation machine that collects audio in real time, reorganizes it and plays it back following, as well as questioning, common free improvisation aesthetics. In this piece I fed it with material from previous pieces along with material from the recordings of Pham Van Mon.\\

This work relates perhaps primarily to the general theme of computational creativity techniques for the reasons explained above. To some extent it can be said to loosely deal with the reproduction of musical style, though not directly using machine learning. This however only slightly warped, in the ways that musical tradition, i.e. Vitenamese traditional music, is filtered through the process of electronic manipulation, and thereby distorted by it. However, I will argue that this is an important aspect of better understanding the advantages of techniques such as machine learning. The way that this project has developed both online and in physical meetings, and how it is performed, is to a high degree an example of emerging self-organization, though not only in an automated manner.\\

I find that one of the challenges in artistic practice in music is to find a proper balance between a systematic approach to the treatment of the musical and non musical material, and some kind of intuitive method with which the system can be manipulated and mastered. The system, which can be almost any conceptual structure, including a tonal system, a sound source and its possible variations, a data set to sonify, a synthesis model, or whatever is of relevance to the artistic idea. The manipulation part of it can also be any one technique, in a range of different methods. The purpose is to allow oneself to get acquainted with the system to the point where it becomes second nature when explored, examined and scrutinized, to the point that allows one to truly \emph{play} or simply improvise with it.\\

The mastering of the systems, or concepts, is not a question of interpreting them. The main purpose is not to find out what the concept means and dwell into it, but rather to understand what it does.\footnote{Sontag, S., \emph{Against interpretation: and other essays} (1986), : Octagon Books.\\} Any one system by itself is rarely what makes a piece work. Instead it is how it is set in motion, how it operates; to understand what it can do under different conditions. A concept as a point of departure for an artistic expression is like a part without a whole. Before it can be made useful it has to be conceptualized within the frame of the expression, and instantiated together with the practice with which it can unite. This can be achieved through play and improvisation. Considering my exercise with the software KA in this piece I started with a concept, a set of sounds, that by themselves were not really interesting. They have a certain meaning to me, because I have gone through the process of playing with them in a different context. But now, as I feed them to KA, they are empty. It is in the playing process that I interact with KA and where the concept is set into motion. For some of the sounds I fed it I could not reach a point that felt meaningful, but through the process of improvisation I managed to not only feel that \emph{we}, KA and me, were playing the sounds in a way that I could sympathize with. More importantly, however, I also reached a position in which the boundary between myself and KA was dismantled and I could understand the larger system consisting of me and KA from the \emph{inside}.\\

For a long time I have been interested in the ways in which technology affects how and what I play. One of the defining ideas has been to understand the interaction with technology as an encounter with another, and understand it less as a tool and more as an agent that I influence as much as it influences me. This process has led to questioning what interaction actually means in artistic practice and has taken a stance against the defining character of interaction as simply a mode of control. The main thread is the idea that all kinds of encounters with tools of some kind in the artistic process is an opportunity to enter into a larger sphere of interaction that creates difference, and that the prevailing idea of \emph{controlling} the technology is a drive that may have a negative effect on the possibilities of the system. This reasoning has its roots in an aesthetic choice more concerned with the human capacity for collaboration than with personally and individually rooted attempts at artistic expression.
\end{document}