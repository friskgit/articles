% Created 2021-12-06 mån 10:10
% Intended LaTeX compiler: pdflatex
\documentclass[11pt]{article}
\usepackage[utf8]{inputenc}
\usepackage[T1]{fontenc}
\usepackage{graphicx}
\usepackage{longtable}
\usepackage{wrapfig}
\usepackage{rotating}
\usepackage[normalem]{ulem}
\usepackage{amsmath}
\usepackage{amssymb}
\usepackage{capt-of}
\usepackage{hyperref}
\usepackage[english]{babel}
\usepackage[style=authoryear-ibid,natbib=true,backend=biber,hyperref=false]{biblatex}
\addbibresource[]{/home/henrikfr/Dropbox/Documents/articles/biblio/bibliography.bib}
\author{Henrik Frisk}
\date{}
\title{The structurality of artistic research}
\hypersetup{
 pdfauthor={Henrik Frisk},
 pdftitle={The structurality of artistic research},
 pdfkeywords={},
 pdfsubject={},
 pdfcreator={Emacs 27.2 (Org mode 9.5)}, 
 pdflang={English}}
\begin{document}

\maketitle
In the Swedish Higher Education Ordinance there are two distinct and equal disciplinary foundations: scientific or artistic. This is a cornerstone in both education and research in Sweden. It is a great advantage in many respects and has been a driving force behind the strong development of artistic research in Sweden. It gives artistically merited teachers a solid foundation upon which to build and develop their competences on par with scientifically merited teachers. However, and this should come as no surprise, it also creates a structural and epistemological dividing line between arts and science which sometimes appears as an unfortunate obstacle in interdisciplinary projects.

But why is this separation a problem? The idea that science is the
sole valid producer of knowledge is obviously debatable and as a
consequence the strict dividing line between the sciences and all
other possible kinds of knowledge production is unfortunate. In an
artistic practice or research project that in one way or another uses
scientific methods or modes of thinking, or that attempts to
critically confront these, the separation may become an issue. And in music, subjects such as musicology and music psychology are structurally separated from artistic higher education in music. 

It is not always self evident how the gap between arts and science should be approached or understood, and in explicitly interdisciplinary projects it may be difficult to achieve a solid ground for the research. Due to the organizational distance between the artistic and scientific modes of thinking there is a risk that whatever interdisciplinary work is produced is absorbed by the gap and useful knowledge that was produced in the project remains hidden to either of the two disciplines. By the sheer difference in the size between art and science, the risk of loss is commonly greater from the point of view of the arts, but the consequences may be equally problematic.

The role that science has played in the development of modernity and
creation of wealth (in the West) is undeniable. "Science and modernity
have become inseparable" \citep{nowotny2013} as put in the book
\emph{Re-Thinking Science: Knowledge and the Public in an Age of
    Uncertainty}. However, they go on to suggest that this is continuously being challenged. Bruno Latour makes a similar claim in his 1998 \emph{Essays on Science and Society} pointing to the breathtaking scientific development characterized by a movement from the "culture of 'science' to the culture of 'research'" \citep{Latour1998}. That art has had an equally undeniable influence on the development of society for much longer than what science has had is clear--after all, art is much older than science. Nevertheless, art has had obvious difficulties in competing with the attention given to science during the post-war era of capitalist-driven development.

Latour continues:
\begin{quote}
Science is certainty; research is uncertainty. Science is supposed to be cold, straight, and detached; research is warm, involving, and risky. Science puts an end to the vagaries of human disputes; research creates controversies.  Science produces objectivity by escaping as much as possible from the shackles of ideology, passions, and emotions; research feeds on all of those to render objects of inquiry familiar. \citep{Latour1998}
\end{quote}

In this quote, it is possible to replace 'research' with 'art' and
come up with a partly relevant and preliminary characterization of
some of the contemporary art fields. Characteristics such as
uncertainty, involving, risky, and controversial could be seen as a
reasonable way to depict art practices of the 21st century: an
activity that feeds on passion and emotion in various
constellations. Following from this, it is possible to draw the
conclusion that it is \emph{science}, not
\emph{research} that differs from \emph{art} and the activity of
\emph{research} may be seen as the common ground.

At any rate, the way Latour represents research is not at odds with
\emph{artistic} research, at least not in my view: the issue here is
again structural, not practical. I wish to claim that regarding
research as the common activity between art and science is a fruitful
way in which interdisciplinarity may be addressed. What appears to
sometimes be a steep difference between \emph{artistic research} and
\emph{research}, as two distinct disciplines, can be overcome by
regarding the act of \emph{research} as the common denominator between
science and art. That one is scientific and the other artistic is
perhaps less important (and the difference between art and science may
even be less compared to, say gender studies and physics).

In addition, arts and science have a shared interlocutor
\emph{society} that contextualizes and participates in the validation
of the research. Because, just as "science and society cannot be
separated, they depend on the same foundation" \citep{Latour1998},
neither can art and society. In other words, if research is the common
aspect of arts and sciences, their respective relation to society is
another. In this view the prefix \emph{artistic} is merely a
contextualization of the activity, an indication that practice, method,
and theory may differ from scientific research practices. Regardless
of the type of activity, the validity of the results is measured by
its usefulness for other research, for the field of practice, or
society. The freedom that the discipline of \emph{artistic
    research} provides will remain untouched. This view, merely allows
us to disregard the obstacles of the structurality of research
disciplines and instead focus on the search for knowledge, which in
all cases should be the driving force.

\label{sec:org4513767}
\printbibliography

\end{document}