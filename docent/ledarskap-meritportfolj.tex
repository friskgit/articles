\documentclass[a4paper]{article}
\usepackage[english]{babel}
\usepackage[T1]{fontenc}
\usepackage[authoryear,round]{natbib}
\usepackage{url}
\usepackage[utf8]{inputenc}
\usepackage{enumitem}
\renewcommand{\encodingdefault}{T1}
\usepackage{fancyhdr}
\pagestyle{fancy}
\renewcommand{\rmdefault}{pad}
\renewcommand{\sfdefault}{pfr}
%\lhead{\small{\textit{Henrik Frisk}}}
%\chead{}
%\rhead{\small{\textit{title}}}

\title{F: Ledarskap och administration}
%\author{Henrik Frisk}
\date{}

\begin{document}
\selectlanguage{english}
%\maketitle

\thispagestyle{empty}

\section*{\textsf{4F. Ledarskap och administration}}

\section*{\textsf{Sammanfattning}}

Ledarskap är en naturlig del av nära nog varje konstnärskap. Min främsta ledarskapserfarenhet kommer således från mitt arbete som bandledare och dirigent samt som administrativ ledare av grupper och turnéer. De viktiga områdena kan sammanfattas som:

\begin{enumerate}
\item \textbf{Orkesterledning}. I drygt två decennier har jag initierat och lett konstnärliga projekt. Ofta underfinansierade och med små praktiska marginaler har jag lärt mig tänka ekonomiskt, pragmatiskt och praktiskt, dock utan att någonsin tumma på kvaliteten. 
\item \textbf{Utbildningsledning}. När jag tog ansvaret för musikerutbildningen för improvisations och jazzmusiker så fick jag möjlighet att se hur min erfarenhet från det konstnärliga området kunde användas i praktiken inom högskolan. Senare använde jag dessa erfarenheter och byggde på dem vid mitt jobb som ansvarig för forskarutbildningen vid KMH.
\item \textbf{Initiativ till samverkan mellan institutioner}. Jag brinner för att skapa synergieffekter och utveckla verksamheter genom samarbete med andra. Min konstnärliga och filosofiska gärning kretsar kring att samverka för att nå resultat. I KU-samverkansgruppen i Stockholm går arbetet ut på precis detta, på att skapa kontakter där dessa inte tidigare funnits och effektivisera genom att inte göra samma sak på flera ställen.
\end{enumerate}

\section*{\textsf{Självreflektion}}

Min erfarenhet av ledarskap inom högskola är begränsad. Detta beror till stor del på att Musikhögskolan i Malmö, liksom ofta många andra konstnärliga högskolor och fakulteter, inte har haft konstnärlig forskning som ett eget ämnesområde med en ämnesstruktur. Under min tid som doktorand var det utbildningsledaren som ledde ämnet, därefter dekanen. Det är först 2014 som vi har fått en ämnesansvarig. På Musikhögskolan i Stockholm har jag haft en ledningsposition för ämnet men inte ett formellt ämnesansvar varför möjligheten att utöva ledarskapet har varit ytterst begränsat. Den stora diskrepansen mellan min befattning där (forskarassistent) och mina arbetsuppgifter (uppbyggnad av forskarutbildning och huvudhandledning av doktorander samt administration av forskarskolan) gör yttermera att en jämförelse mellan situationen inom andra fakulteter blir svår.

Under tiden som ansvarig för jazz och improvisationsmusikerutbildningen så jobbade jag intensivt med att knyta band mellan vår utbildning och andra utbildningar på Musikhögskolan, andra utbildningar i regionen, andra skolor i närheten (Danmark) och andra skolor i Skandinavien och resten av Europa. Vi hade ett studentutbyte med Frankrike, flera projektveckor med Rytmisk Musikkonservatorium i Köpenhamn och täta samarbeten med folkhögskolorna i regionen. Den kanske viktigaste förändringen som jag tog initiativ till, och tillsammans med en kollega drev igenom, var att knyta den konstnärliga och den pedagogiska varianten av jazzutbildning närmare varandra. När jag började hade utbildningarna inga gemensamma moment och efter förändringarna, som först mötte stort motstånd från flera håll, så hade studenterna flera gemensamma moment och ett i allmänhet mer varierat kursutbud till följd av fler studenter. Fortfarande nu, mer än tio år efter förändringen, fungerar de två utbildningarna på samma sätt, enligt samma princip.

Denna typ av förändringar som kräver diplomati och envishet tror jag är en av mina administrativa och ledarmässiga styrkor. Jag har lätt för att knyta allianser med folk och har möjligheten att höja blicken och skapa strukturer som är bra för alla parter. En av mina negativa sidor är att jag ibland kan sakna tålamod om förändringarna tar för lång tid. Möjligheten att se synergieffekter och sträva mot förändringar och förbättringar som tar till vara dessa effekter är en annan av mina styrkor, där jag inte backar för om det initialt blir mer arbete för mig själv. Ser jag att det kan bli en förbättring till en låg kostnad så ser jag inte mitt merarbete som ett hinder.

Ledarskapserfarenhet har jag även fått från mina sju år som ordförande i föräldrakooperativet Barnhagen i Uppsala. Under sju år hade jag arbetsgivaransvaret för 5-6 anställda förskollärare och barnskötare, arbetsmiljöfrågor, säkerhetsfrågor, försäkringsfrågor, frågor rörande tystnadsplikt, förhandlingar med facket i arbetsrättsliga frågor, med mera. Under min tid gjorde vi en uppsägning med MBL-förhandling samt fyra nyanställningar och totalt åtta löneförhandlingar. Som ordförande för Föreningen Sveriges Jazzmusiker hade jag kontakt med den andra FSJ-organisationerna och musikinstitutioner i Sverige samt med våra medlemmar naturligtvis.

\newpage

\section*{\textsf{Akademiskt ledarskap och administration}}

\section*{\textsf{Meritförteckning}}



\subsection*{\textsf{Formell ledarskaps- respektive administrativ utbildning }}

Ingen formell ledarskaps- eller administrativ utbildning.
 
\subsection*{\textsf{Ledarskapsbefattningar inom akademin}}

Inga formella Ledarskapsbefattningar.

\subsection*{\textsf{Ledarskapsbefattningar utanför akademin}}

\begin{enumerate}
\item Ordförande för föräldrakooperativet Barnhagen, Uppsala 2008-2014.
\item Ordförande för Sveriges Jazzmusiker Syd, Malmö 2006-2008
\item Ledare för storbandet The Orchestra, Köpenhamn 1992-1994; 1995-1998 
\end{enumerate}

\subsection*{\textsf{Uppdrag inom nämnder och kommittéer}}

\begin{enumerate}
\item Vice ordförande i Nämnden för konstnärligt utvecklingsarbete, Musikhögskolan i Malmö, Lunds Universitet 2004-
\end{enumerate}

\subsection*{\textsf{Uppdrag rörande etik, jämställdhet, arbetsmiljö och miljöfrågor}}

Inga uppdrag.

\subsection*{\textsf{Lednings- och samarbetskompetens inom andra organisationer utanför universitetet såsom vetenskapliga eller fackliga organisationer}}

Se ovan, ordförande för Sveriges Jazzmusiker Syd, Malmö 2006-2008.

\section*{\textsf{Bilagor}}

\begin{enumerate}
\item Intyg Barnhagen 2013.
\item Intyg KU-nämnden 2014.
\end{enumerate}



% \bibliographystyle{apalike}
% \bibliography{/run/media/henrikfr/Homer/Home/Documents/svn/admin/conf/biblio/bibliography}
\end{document}