\documentclass[a4paper]{article}
\usepackage[english]{babel}
\usepackage[T1]{fontenc}
%\usepackage[authoryear,round]{natbib}
\usepackage[style=authoryear,natbib=true,backend=biber]{biblatex}
\bibliography{./../biblio.bib}
\usepackage{url}
\usepackage[utf8]{inputenc}
\usepackage{enumitem}
\renewcommand{\encodingdefault}{T1}
\usepackage{fancyhdr}
\pagestyle{fancy}
\renewcommand{\rmdefault}{pad}
\renewcommand{\sfdefault}{pfr}
%\lhead{\small{\textit{Henrik Frisk}}}
%\chead{}
%\rhead{\small{\textit{title}}}

\title{I. Konstnärligt forskningsarbete sedan disputationen}
%\author{Henrik Frisk}
\date{}

\begin{document}
\selectlanguage{english}
\maketitle

\thispagestyle{empty}

\section*{\textsf{1. Summary}}

The main threads of my current work and my work since my dissertation can be summarized as in the following:

\begin{enumerate}
\item \textbf{Self and otherness}: re-configuring the binary relation between self and other: Allowing the Self to ``be informed by what is now going on in the process rather than by what has traditionally shaped it'' \citep{frisk2013}. Obviously a huge field of investigation however made manageable by keeping the interrogations within my artistic practice. It also forms the context for all of the following three areas.
\item \textbf{Artistic research as an actor in social and political thinking}:  ``placing the artistic work in the light of a particular social, theoretical, cultural, or philosophical framework [\ldots] causes the political dimension to surface'' \citep{frisk-ost13}. Concretely, this field has benefited largely through the many interactions I have had with Vietnam in recent years. My current plan is to include other cultures in my research and look at close interactions with North Africa.
\item \textbf{Expanding the field of improvisation} reconsidering the impact of individuality, habit and freedom: ``The impact of freedom, being such an essential concept in the understanding of improvisation, is closely related to some of the more social and political topics [\ldots] and can be understood in a number of ways, such as freedom \emph{of the self} and freedom \emph{from the self}'' \citep{frisk12-improv}. The continuation in this area is to both expand the perimeter, looking at other expressions of improvised music, as well as narrowing and confine the study to look at the multitude of expressions that exist within jazz improvisation.
\item \textbf{Artistic research methodology}: In \citep{frisk-ost13} we claim that it is necessary to question replacement terms such as `silent knowledge, `narration' and `new knowledge' ``and trust the power and efficiency of the artistic practice to be solid enough to withstand the impact of established and hybrid qualitative research methods without losing its qualities as art while displaying its potential as research.'' In this area, however, I am also working with more tangible tools for documenting artistic practice. The aspect of documentation is in many cases an integral part of artistic research methodology.
\end{enumerate}

\section*{\textsf{2. Research activities}}

\subsection*{\textsf{Completed research projects since dissertation, and current}}

One of the sections taken out from my 2008 PhD dissertation ``Improvisation, computers and interaction'' \citep{frisk08} was the one concerning the Self. From the beginning the idea of the Self as the defining difference between man and machine seemed to me one of the more important aspects of the investigation I was carrying out. Due to the way the project had developed, however, in the end it proved to be less important. The topic of human-computer interaction and improvisation had taken new and unexpected turns and cutting this section out was part of the often common process of narrowing down a thesis to give it more focus. After the dissertation, however, I wanted to approach this vast and difficult area of the Self and its constitution in art and improvisation and it became the focus of my post doc research within the project \emph{(Re)Thinking Improvisation}. 

The self influences so many aspects that I have already explored, such as the concept of \emph{control}, \emph{autonomy}, and of the \emph{work in motion}. In almost all my works during the five years that has passed since my PhD defense, the notion of the Self and its relation to the work, the collaborators, the audience, and the field has played a central role. I have been aiming for an expanded context for artistic production and research in which the 'author' is distributed among several agents.
% (approaching Berkeley's question if the tree makes a sound if noone hears it, rephrased as: does the musical work exist unless it is heard?). 
What I am discussing here is not to outsource parts of the work creation, nor to remove myself from the creative process. Rather, it is about taking responsibility for a relation with the other (listener, co-musician, co-artist, etc.) where \emph{listening} is one of the key components. It is about the acceptance that control and result is less interesting than process and about giving up ownership and authority in an attempt to open-source the musical work. Henry David Thoreau, a main character of the development of the objectivist view, whose work became an important inspiration for John Cage, speaks of the ``transparent eye-ball'' and the ``objective I'' \citep{thoreau2004}. However, objectivity is not the main matter here. Rather, it is the space for subjectivity that holds the key to the success of important aspects of collaborative practices:``Validity then is fundamentally a matter of making the subjectivity of the artist visible in the research design. The need for creating a multi-layered understanding of subject-positions does come out clearly in studies of collaborative creativity'' \citep{frisk-ost13}. In collaborative practices negotiation and sharing are at the center, as is the general philosophy of being open to the perspective of the other, a generally phenomenological approach, and giving priority of eye over I. 

The ethical dimension grew out of the artistic projects, in particular out of the Vietnamese-Swedish group The Six Tones and the different projects we engaged in. Its intercultural context made it necessary to probe the questions of identity, belonging, difference, otherness, ontology and epistemology in relation to our work, to music and to the inter cultural context we are situated in. In this dimension the important question of identity has to be investigated. The psychologically charged relationship of identity and Self has a special meaning in postcolonial theory as many of the common components for constructing identity are rooted in us-and-them binary relations. The hypothesis that I have been developing through this work is that the identity of the Self as described in a binary relation to the other (as in soloist-accompanist, composer-performer, performer-listener, art music-traditional music, Westerner-foreigner, etc.) may be deconstructed in artistic practices through the tools I developed in my thesis. These are primarily the ideas concerning the distribution of the creative process. The main artistic contributions with reference to the Self are the double CD Signal in Noise \citep{sixtones13} and the composition \emph{The Transparent I} \citep{frisk-transparent} and in \emph{Improvisation and the self} \citep{frisk12-improv} the social and political aspects are discussed:
%%%%% POLITICAL AND SOCIAL

\begin{quote}
  We must also include and consider the fact that the self is
  continuously constructing the other, and similarly, how the dynamics
  of the self is influenced by social and political powers. Finally,
  it is important to remember that the social and political domains
  themselves may be influenced and even altered by how the self and
  the other is constituted.
\end{quote}

In a complex interplay between music, improvisation, consciousness through reflection, experimentation, methodological rigor, practice, the social and the political I have begun to deconstruct the Self in a way that makes possible an alternative understanding of relations to the other.
%%%%% SELF AND OTHER -END

The work Stefan and I did in The Six Tones while preparing for the already mentioned recording of \emph{Signal in Noise} turned out to be the beginning of a new direction for the group we started in 2006. The Six Tones had done a few projects already but the way we engaged with different aspects of the diversified musical life of Hanoi was an idea the artistic influence of which we had not anticipated. We boldly emerged into quite radical improvisations with the traditional flute playing of Le Pho and we brought our own acoustic expressions to the studio with the contemporary electronic playing of Tri Minh and Vu Na Than. These sessions exemplify the potential complexity of artistic work. In music it is often spoken of that which is natural or comes naturally, as something that is effortlessness is meant to be. Our playing in these sessions was effortless but at the same time there was a considerable resistance involved, a resistance that did not necessarily had to be overcome but that had to be dealt with. Hence, though we quickly arrived at a satisfying result the process was anything but effortless to me. Partly, the issue was related to the political and social aspects of me as a white man abroad making use of the traditions and knowledge of people economically much less fortunate than me. Though the CD project made these ideas surface, asking these questions has been at the very heart of the Six Tones since the beginning:

\begin{quote}
  In order to truly be able to encounter the new and the unforeseen,
  challenging different aspects of the notion of ``center'' and
  ``periphery'' was necessary: is Western art music the norm and
  traditional Vietnamese music an exotic other? Are Stefan and I
  `visiting' a music outside of our own sphere, or is it rather Thuy
  and My that are forced to approach us. Is it at all possible to
  communicate on equal terms? The social impact of the Eurocentric
  view of the world, however, should not be underestimated. Stefan and
  I belong to what Mark \citet{slobin1987} labels ``the superculture''
  (p. 31), and the complex political and economic asymmetry between
  east and west plays an important role in our understanding of the
  other in our multi layered work with traditional Vietnamese music in
  general, and with The Six Tones in particular. \citep{frisk12-improv}
\end{quote}

In the project The Six Tones I place myself in a context where I \emph{have} to consider the political impact of my artistic activities even before they themselves become political. After all, if we would not be able to avoid becoming part of cultural appropriation the ambition of the project would fail.

Jacques \citet{attali85} points to the merchandising prospect of most music and how that implicitly adds a political facet to it. But music, however, also heralds a subversion and a possibility for {\textquotedblleft}a radically, new organization{\textquotedblright} that is yet unimagined (ibid. p. 5). Artistic research in general, and the project discussed above in particular may be considered an arena where such new organizations may advance and flourish. On that arena the political and social aspects of the activities must be carefully considered which is not, however, the same as the art emanating from that process will be political.

As has already been mentioned above, there is a need to re-contextualize the artistic practice on many different levels. It was one of the aims of the project (Re)Thinking Improvisation, essentially trying to expand the field of improvisation and look at the improvisatory properties of composition, interpretation and many other musical activities:

\begin{quote}
  Through these multiple perspectives on improvisation it may be
  concluded that interpretation, improvisation, composition and the
  musical work are fluid but closely interrelated concepts. While
  definitions may then become more of local, and often narrowly
  political, statements, a study of the how these concepts interact
  and bleed into one another appears to us as a way to begin
  reconsider some of the fundamentals of Western art music which can
  be thought of as a beginning towards what could become a rethinking
  of improvisation. \citep{frisk-re_improv13}
\end{quote}

The Integra project for which the initial aim was to fuse technology and music, contributed to a shift in my artistic activities. Music technology, as much technology, has long been focused on a rigid epistemology firmly based on the scientific field rather than the artistic. Though many of the artworks that emanate from this field are certainly not scientific in nature (such as Stockhausen's earlier work, Alvin Lucier's brilliant conceptual pieces or John Cage's low-tech electronic pieces), the discipline has since long been situated in an academic context where music technology specialists are distinct from artists. The electronic music studio in Paris, IRCAM, introduced the role of the \emph{musical assistant} as someone who would contribute technical knowledge to composers. Essentially a modernist idea of labor division where composers could not, or should not, deal with the technical issues in works with a strong technological component. This may seem a logical and pragmatic solution, similar to the labor division commonly found in theaters, opera houses, film productions, and so forth, with the effect that the technology easily becomes the auxiliary tool to the main artistic work. In reality, the \emph{musical assistant} became as much of an artist as the composer having to come up with creative solutions to impossible problems, albeit always a second grade artist never recognized for his or her contribution. Historically the artistic impetus of the composer makes the authorship remain with him or her.

The goal of the Integra project however was not primarily to counteract the authority of the creator but to facilitate for musicians and new music ensembles to work with electronics. Part of the task was to build tools for musicians and composers that were sophisticated, yet easy to use. Along with the work Stefan Östersjö and I did through \emph{Repetition Repeats all other Repetitions} and the studies entitled \emph{Negotiating the Musical Work} (all part of my dissertation) Integra further fueled my ideas concerning the \emph{work in motion} as a distributed work with no beginning and no end, an open sourced work of art that almost anyone can fork and continue to work on. In the end, there is much work yet to be done, and, obviously, not all artistic projects are suitable for this approach. Instead, the model can be seen as a guide, a method to consciously counteract the tendency for artistic expressions to fall back on modernist and romantic structures, closed for listener participation \citep[A collaborate paper on Repetition is in print along with a recording of a new version of the piece.][]{friskcoessens2013}. Towards the end of the Integra project the Swedish composer Kent Olofsson and I, both of us involved in the scientific aspect of the Integra project, joined forces with chamber group Ars Nova, part of the artistic side of Integra, and created the digital chamber group \emph{Switched-On}. Five musicians all playing electronic, or electronically enhanced versions of their instruments. Both Kent and I wrote music for this ensemble and the setup, technically incredibly advanced, was inspired by the ideas developed in the Integra project. 

My contribution to this ensemble was the composition \emph{The Mystic Writing Pad}: 

\begin{quote}
\emph{This piece is an improvisation based on the structure of the American composer Harry Partch's 43-tone Just Intonation scale; a division of the octave in 43 unequal steps. The 43 notes of the scale have been distributed among the five instruments, and it is only together that they can explore the full potential of the scale. The collaborative aspect of this piece is further explored by its meta-instruments: instruments that are hidden under the surface and for which the players need to join forces in order to control.}

\emph{The title, The Mystic Writing Pad, refers to Freud's 1924 paper in which he lays out a hypothesis about the inner functionality of human perception. Though much can be said about his hypothesis (and much has been said about it, not the least by French philosopher Jacques Derrida) my reasons for choosing this title is much more practical and metaphorical. The functionality of the technology upon which Switched-On relies can often be very mystical, but the ease with which it can be used to register the phrases played by the musicians is truly akin to a writing pad: great at quickly taking notes (in two senses of the word), but terrible at making thoroughly thought through statements.} \citep[][(Program note)]{frisk-mystic}
\end{quote}

\emph{The Mystic Writing Pad} is in three parts where the first part is a structured improvisation, the second a purely electronic fixed media part and the third is an ensemble passage for midi-saxophone solo. All three movements use a microtonal scale by Harry Partch. Working with alternate tuning systems in combination with improvisation creates a backdrop for the investigation of questions concerning the Self and the \emph{work-in-motion}. The microtonal scale and the very technically complex ensemble allowed me to work with the music in a conceptual way much more than a structured way. Furthermore, for all the musicians, and myself, the odd harmony of the tuning system made it necessary for us to question ourselves in a way that we might have hesitated to do in other contexts.

The Integra project was also the beginning of my main methodological project at this point. A dynamic and open ended database for documentation, assessment, evaluation and discussion of artistic works. Fully implemented the system will also be a good way to document the \emph{work-in-motion}.

To summarize, the four fields sketched out in the beginning have influenced all of my research during the last six years. A period during which I have been very active both as performer and researcher with great opportunities to discuss and try the experimental projects I have been engaged in. These are, however, quite wide areas and the projects I am engaging in during the fall of 2014 are part of an attempt to extend and converge these fields into more focused areas of research.


\subsection*{\textsf{Current and future research projects}}

The projects below are all in the planning and will start up in the second half of 2014 or early 2015. These projects will in essence be my main research activity for 2015 and 2016.

\begin{enumerate}[label=(\emph{\alph*})]
\item \textbf{Text as sound and sound as text} 

Building on (1), \emph{Self and otherness}, above, this project is expanding on the work that I have done within the SixTones. The main idea is to investigate how (i) speech as sound with syntactical meaning compares to (ii) speech without syntactical meaning (as in a language one doesn't understand) and (iii) sound as music. What may the musical response be to text as sound as compared to text as language with or without a semantic meaning. This clearly puts the focus on the identity and community creating aspects of language -- attaching to my previous work on the self, the social and the political -- and provides a set of data that we intend to feed into the artistic processes in various ways. This project is a collaboration between Malmö Academy of Music and Humlabbet at Lund University, as well as the Music department and CCRMA at Stanford University, California.

\item \textbf{Localizing nature and composition} 

Also following up on the intercultural work done within the Six Tones but expanding this into new countries, this project is an extension and further focus of (2) \emph{Artistic research as an actor in social and political thinking} above. Together with long term associate Stefan Östersjö and ethnographer Robert Willim of Lund University I will make excursions, according to current planning into North Africa, and work locally attempting to use our own artistic practice and adapt and intersect with that within the country we are visiting through a few artists/musicians. Nature, local and composition should be thought of as keywords in the widest sense and in many combinations. This project, along with the prior, continues the political thread that has become very important to me. The prospect of using artistic research to shed light on, and offer alternate descriptions of problems and social challenges that lie ahead of us, is very promising in my experience, and provides a stimulating context for artistic research.

\item \textbf{Contemporary methods for improvisation} 

Following (3) \emph{Expanding the field of improvisation}, at the surface this project may be seen as a means to narrow the field of improvisation. Its goal is to approach some of the leading jazz improvisers in the world and meet with them in practice, in a rehearsal space as well as in concert, and attempt to understand (on the level of performance) and document their creative modes. As such the project is an experiment in terms of group documentation and reflection and an honest attempt to map knowledge which is in every regard missing from the common channels of knowledge.

\item \textbf{Documentation of artistic research projects} To begin with not an artistic project but rather an attempt to follow up a thread in my thesis as well as an attempt at providing useful methodological tools to artistic researchers. This project will form an essential part of both (b) and (c) and will be a platform for documenting, discussing and sharing artistic research results. I have funding to start up an ethnographically oriented artistic and artistic research project in free jazz improvisation the aim of which it is to attempt to map aesthetic threads and tacit knowledge within the practice of guest musicians with a strong individual style.
\end{enumerate}

% \newpage

% 
%\section*{\textsf{3. Research experience}}

\subsubsection*{\textsf{Research environment}}

I am or have been involved in the following research environments and networks:

\begin{enumerate}
\item \textbf{CATS}: An initiative to bring together KMH, KKH and Stockholm University to initiate and stimulate cross discipline projects. 
\item \textbf{IAC}: Inter Arts Center at Lund University is a network as well as a location for artistic research. It has been the home of several of my projects since it opened a few years ago.
\item \textbf{Integra}: Integra was a research project financed by the EU that I worked intensively with during the years it was active. Not only was it an active environment at the time but it also led to sustained contact with Birmingham, Oslo and McGill, Montreal, Canada.
\item \textbf{(Re)Thinking Improvisation}: A research project and a research environment active between 2008 through 2013.
\item \textbf{ICMC}: Up until 2013 I was quite active in ICMC and the conferences it produced and several people in ICMC are still part of my network.
\item \textbf{EPARM (AEC)}: This loosely built network, organized around a small working group of which I am a part, has proven very useful to my own projects in the way it gives me insight in the state of affairs concerning artistic research in Europe in general.
\end{enumerate}

\subsubsection*{\textsf{Supervision experience}}

\subsubsection*{\textsf{As main supervisor}}
  \begin{enumerate}[label=\arabic*]
  \item Anna Einarsson

    \begin{tabular}[c]{p{3.3cm} p{7cm}}
      \textbf{Exam}: & 2016 \\
      \textbf{School}: & KMH \\
      \textbf{Project}: & \emph{A study in voice and composition. New means of performance and computer assisted composition by singing voice feature recognition} \\
      \textbf{Secondary supervisor}: & Susan Kozel \\
    \end{tabular}

  \item Kent Olofsson

    \begin{tabular}[c]{p{3.3cm} p{7cm}}
      \textbf{Exam}: & 2016 \\
      \textbf{School}: & KMH \\
      \textbf{Project}: & \emph{Rethinking Music Drama: Composing Sonic Art Theatre} \\
      \textbf{Secondary supervisor}: & Erik Rynell \\
    \end{tabular}

  \end{enumerate}
\subsubsection*{\textsf{As secondary supervisor}}

  \begin{enumerate}[start=3]
  \item Susanne Rosenberg

    \begin{tabular}[c]{p{3.3cm} p{7cm}}
      \textbf{Exam}: & 2014 \\
      \textbf{School}: & KMH \\
      \textbf{Project}: & \emph{Kurbits-ReBoot, svensk folksång i ny scenisk gestaltning} \\
      \textbf{Main supervisor}: & Hannu Tolvanen \\
    \end{tabular}
  \item Sara Wilén

    \begin{tabular}[c]{p{3.3cm} p{7cm}}
      \textbf{Exam}: & 2016 \\
      \textbf{School}: & KMH \\
      \textbf{Project}: & \emph{Opera improvisation} \\
      \textbf{Main supervisor}: & Karin Johansson \\
    \end{tabular}

  \end{enumerate}


\subsubsection*{\textsf{Conferences and symposia organized}}

\begin{enumerate}
\item \textbf{EPARM Stockholm}

  \begin{tabular}[l]{p{0.33\linewidth} p{0.33\linewidth} p{0.33\linewidth}}
    Working group member & AEC & April 2014    
  \end{tabular}

\item \textbf{SMAC/SMC}

  \begin{tabular}[c]{p{0.33\linewidth} p{0.33\linewidth} p{0.33\linewidth}}
    Music Co-chair & KMH/KTH & August 2013    
  \end{tabular}

\item \textbf{Att utveckla och kommunicera musikalisk kunskap }

  \begin{tabular}[c]{p{0.33\linewidth} p{0.33\linewidth} p{0.33\linewidth}}
    Co-chair & Kungl. Mus. Akademien & Februari 2013    
  \end{tabular}

\item \textbf{EPARM Lyon}

  \begin{tabular}[c]{p{0.33\linewidth} p{0.33\linewidth} p{0.33\linewidth}}
    Working group member & AEC & April 2013    
  \end{tabular}

\item \textbf{KMH Music research days}

  \begin{tabular}[c]{p{0.33\linewidth} p{0.33\linewidth} p{0.33\linewidth}}
    Co-chair & KMH & January 2013    
  \end{tabular}

\item \textbf{EPARM Rome}

  \begin{tabular}[c]{p{0.33\linewidth} p{0.33\linewidth} p{0.33\linewidth}}
    Working group member & AEC & April 2012    
  \end{tabular}

\item \textbf{(Re)Thinking Improvisation: International sessions of artistic research}

  \begin{tabular}[c]{p{0.33\linewidth} p{0.33\linewidth} p{0.33\linewidth}}
    Co-chair & MHM & November 2011    
  \end{tabular}

\item \textbf{Eye of the Needle}

  \begin{tabular}[c]{p{0.33\linewidth} p{0.33\linewidth} p{0.33\linewidth}}
    Co-chair & MHM & October 2008
  \end{tabular}
\item \textbf{Connect}

  \begin{tabular}[c]{p{0.33\linewidth} p{0.33\linewidth} p{0.33\linewidth}}
    Associate & MHM & 2006    
  \end{tabular}

\end{enumerate}

\subsubsection*{\textsf{Editor experience}}

\emph{(Re)thinking Improvisation: Artistic explorations and conceptual writing} \citep{frisk-re_improv13}: A two hundred page book, two audio CDs and one DVD.

\subsubsection*{\textsf{Research collaborations}}

\subsubsection*{\textsf{Integra}}

Integra brought together research centers and new music ensembles and was led by Birmingham Conservatoire at Birmingham City University in the UK. The activities of the project included developing new software to make music with live electronics, modernizing works that use old technology, commissioning composers and overseeing performances. The project was funded by the EU started in 2006 and went on for six years. 

My main duties in the project was working with software development, participating in research meetings and collaborating with the other partners of the project.

\subsubsection*{\textsf{(Re)Thinking Improvisation}}

(Re)Thinking Improvisation was a research project started by the Malmö Academy of Music and financed by the Swedish Research Council and involved artists and researchers from Sweden, the Netherlands and Asia. The project started in 2009 and ended 2013. 

My role in the project was to coordinate it together with my colleague Stefan Östersjö. 

\subsubsection*{\textsf{Review experience}}

\subsubsection*{\textsf{As opponent}}
%% p{0.23\linewidth} p{0.23\linewidth} p{0.23\linewidth} p{0.23\linewidth}
\begin{tabular}[c]{llll}
  Kim Hedås & Dissertation & September 2013 & Göteborg University \\
  Peter Spissky & 25\% seminar & February 2012 & Malmö Academy of Music \\
  Sten Sandell & 25\% seminar & February 2010 & Göteborg University \\
\end{tabular}

\subsubsection*{\textsf{As reviewer}}

\begin{itemize}
\item KU - artistic development projects in Gothenburg 2011
\item Review of the publication \emph{Art Monitor} for the Artistic Faculty, Göteborgs Universitet.
\item Music submissions for SMC 2013, KMH/KTH
\item Paper sumbissions for EMS 2012, KMH
\item Peer reviews for JAR publications
\item Reviews for Critical Studies in Improvisation publications
\item Reviews for NIME
\item Reviews for ICMC
\item Reviews for (Re)Thinking Improvisation
\item Reviews of project proposals for Framtidens Kultur 2006-2010
\item Reviews of artistic development projects at Malmö Academy of Music, 2002-2013.

\end{itemize}

\subsubsection*{\textsf{Prices and rewards}}

No particular scientific prices.

\newpage

%%% Local Variables: 
%%% mode: latex
%%% TeX-master: "vetenskaplig-meritportfolj"
%%% End: 


% Enligt Lunds Univsersitets mall för Vetenskaplig meritportfölj finns inte konstnärliga arbeten med. För att göra denna uppställning tydlig, och så att den ändå ska följa mallen, har jag lagt de textbaserade forskningsarbeten jag gjort först enligt den efterfrågade uppställningen (4.1) och därefter listat de konstnärliga och de konstnärliga forskningsarbetena (4.2). Detta trots att de senare av hävd är själva grunden för de tidigare och att uppställningen därför borde vara i omvänd ordning.

\subsubsection*{\textsf{a) Publicerade originalartiklar i referee-bedömda internationella tidskrifter}}


\begin{enumerate}
\item \fullcite{frisk-ost13}

\item  \fullcite{frisk-cobussen09}

\item  \fullcite{frisk09:improv}

\item  \fullcite{frisk07}

\item  \fullcite{frisk05}
\end{enumerate}


\subsubsection*{\textsf{b) Översiktsartiklar och andra inviterade artiklar i internationella tidskrifter}}
\subsubsection*{\textsf{c) Böcker, bokkapitel}}


\begin{enumerate}
\item \fullcite{frisk12-improv}

\item   \fullcite{frisk-re_improv13}

\item   \fullcite{frisk2013}

\item   \fullcite{friskostersjo2013}

\item   \fullcite{friskostersjo2013b}

\item   \fullcite{frisk11:ELIA}

 \item  \fullcite{frisk10}
\end{enumerate}


\subsubsection*{\textsf{d) Övriga artiklar och rapporter publicerade i internationella tidskrifter}}
\subsubsection*{\textsf{e) Vetenskapliga artiklar och rapporter publicerade på svenska}}

\begin{enumerate}
\item \fullcite{frisk13-vr}
\end{enumerate}

\subsubsection*{\textsf{f) Populärvetenskapliga artiklar / presentationer}}


\begin{enumerate}
\item \fullcite{frisk14-ab}

\item  \fullcite{frisk2013-2}

 \item \fullcite{frisk2013c}

 \item \fullcite{frisk-better}

 \item \fullcite{frisk11}

 \item \fullcite{friskDiary}
\end{enumerate}


\subsubsection*{\textsf{g) Konferensbidrag }}


\begin{enumerate}
\item \fullcite{frisK-bull11}

\item  \fullcite{frisk09}

\item  \fullcite{frisk-bullock08}

\item  \fullcite{frisk-bull07}

\item  \fullcite{frisk-ost06}

\item  \fullcite{frisk-ost06-2}

\item  \fullcite{frisk1}
\end{enumerate}



\subsubsection*{\textsf{h) Manuskript (inskickade manuskript ska listas först följda av de under bearbetning)}}

\begin{enumerate}
\item \fullcite{frisk2015} (Under bearbetning, utgivning 2015)
\end{enumerate}

\newpage
\subsection*{\textsf{iii. Publikationer - konstnärliga}}

\subsubsection*{\textsf{Fonogram - urval}}

\begin{enumerate}
\item [2014] lim, \emph{No title}, Kopasetic (under production)
\item [2013] The Six Tones, \emph{Signal in Noise}, dB
  Productions (double CD)
\item [2013] (Re)Thinking Improvisation, \emph{Artistic explorations},
  Lund University Press (double CD)
\item [2011] Stefan Ostersjo, \emph{Strandlines}, dB Productions
\item [2011] lim, \emph{lim with Marc Ducret}, Kopasetic Prod.
\item [2011] Scandinavian Electroacoustic Music, \emph{Spanning},
  ChamberSound
\item [2008] Henrik Frisk \& Peter Nilsson, \emph{etherSound},
  Kopasetic Prod.
\item [2007] lim and Marc Ducret, \emph{KOPAlectric}, Kopasetic Prod.
\item[2006] lim, \emph{SuperLim}, Kopasetic Prod.
\item[2006] New Century Series, \emph{Volume 17}, MMC.
\item[2005] David Liebman Big Band, \emph{Beyond the line}, OmniTone
\item[2004] Viola con Forza, \emph{Henrik Frendin}, Phono Suecia.
\item [2003] The Orchestra plays Jakob Riis, \emph{In Absence of
    Mind}, Cope Records
\item[2003] lim, \emph{lim}, dB Productions
\item[2002] Fixerad Anarki, \emph{Fixerad Anarki}, dB Productions
\item [1999] NoBass, \emph{Hello World}, LJ Records
\item[1999] Expressions, \emph{Henrik Frisk/Richie Beirach},
  Hornblower Recordings
\item [1998] The Orchestra, \emph{Noxx}, dacapo
\item[1998] The Orchestra, \emph{SmokeOut}, dB Productions
\item[1997] Blue Pages, \emph{Blue Pages}, Caprice Records
\item [1996] Beijbom/Kroner Big Band, \emph{Live in Copenhagen}, Four
  Leaf Clover
\item[1995] Henrik Frisk, \emph{Inventions of Solitude}, Hornblower
  Recordings
\item [1993] The Orchestra, \emph{Not as softly as\ldots}, MusicMecca
\end{enumerate}


\subsubsection*{\textsf{Beställningar (urval)}}
\begin{enumerate}
\item [2013] Ars Nova: work for video and electronic music: \emph{Orpheus in motion}
\item [2011] G\"{o}teborgs Biennalen: music for Better Life by Isaac Julien: \emph{Better Life}
\item [2011] Ensemble Midt Vest: interactive production for chamber group, interactive electronics and video.
\item[2011] Kopasetic, Konstn\"arsn\"amnden: work for jazz ensemble and electronics: \emph{M\r{a}nens gr\r{a}a \"{o}gon}
\item [2010] The Trembling Aeroplanes: work for digital chamber group: \emph{The Mystic Writing Pad}
\item [2009] Ensemble Ars Nova, Swedish Arts Council: work for chamber group and electronics: \emph{The Transparent I}
\item[2008] Marcel Cobussen, music for a web site: \emph{The Field of Musical Improvisation}.
 \item[2006] Copenhagen Art Ensemble, small big band and computer: \emph{Continuous Breach}.
 \item[2005] Stefan \"Ostersj\"o for \emph{Repetition Repeats all other
 Repetitions}, 10-stringed guitar and computer.
 \item[2005] Mercedes Gomez (\emph{work in progress}), harp and
 computer.
 \item[2003] Miya Yoshida for \emph{etherSound}, mobile phone sound
 installation.
 \item[2001] Stockholm Saxophone Quartet for \emph{Perspicio}, Saxophone
 quartet and computer.
 \item[2002] Henrik Frendin for \emph{Drive}, Electric viola grande and
 computer. 
\end{enumerate}

\subsubsection*{\textsf{Konserter (urval)}}

\begin{enumerate}
\item [2014] \emph{Saxophone and computer} Concerts in California.
\item [2014] \emph{Saxophone} Concerts with lim in Sweden and Denmark.
\item [2013] \emph{Saxophone and computer} Hanoi Sound stuff and Tour in Vietnam.
\item [2013] \emph{Saxophone} Lennart \r{A}berg Group featuring Peter Erskine, UKK, Uppsala.
\item [2012] \emph{Soxophone and computer} Frisk/Frendin, Musik i Halland.
\item [2012] \emph{Soxophone} Stroman/J\"{o}nsson group, London, UK, and Sweden.
\item [2012] \emph{Saxophone} KOPAorchestra on tour, Sweden and Denmark
\item [2011] \emph{Saxophone and computer} International session on artistic research, Malm\"o.
\item [2011] \emph{Saxophone} KOPAorchestra on tour, Sweden
\item [2011] \emph{Saxophone} Stroman/J\"onsson Vocal project feat. Lena Willemark, KOPAfestival, Lund, Sweden
\item [2011] \emph{Saxophone and computer} Collaboration with Isaac Julien, G\"oteborg International Biennial for Contemporary Art, Sweden
\item [2011] \emph{Saxophone and computer} Kalvfestivalen, Kalv, Sweden
\item [2011] \emph{Computer} Summerfestival, Ensebmle Midt Vest, Herning, Denmark
\item [2011] \emph{Computer} Integra Festival, Copenhagen, Denmark
\item [2011] \emph{Saxophone and computer} re:New festival, Copenhagen
\item [2010] \emph{Composition} Repetition Repeats\ldots in Ghent, Bruxelles
\item [2010] \emph{Saxophone} Frisk / Beirach / Mogensen trio, tour in Sweden and Denmark.
\item [2010] \emph{Saxophone and computer} EarZoom festival, Lubljana, Slovenia.
\item [2010] \emph{Saxophone and computer} The Six Tones concert at Green Space Festival, Hanoi, Vietnam.
\item [2009] \emph{Computer} - Performance of Anne LeBaron's cyberopera \emph{Sucktion}
\item [2009] \emph{Saxophone and bandleader} - Premiere of the 10-piece band Henrik Frisk's Pli at Kopasetic Festival, Malm\"o, Sweden.
\item [2009] \emph{Computer} - Performances of \emph{Continuous Breach} at Copenhagen and Aarhus jazz festivals.
\item [2009] \emph{Computer} - etherSound in Stockholm at Nybrokajen 11.
\item [2009] \emph{Saxophone} - performances in Malm\"o and Lund with lim.
\item [2009] \emph{Computer} - Tour in Sweden and Denmark with The Six Tones
\item [2007] \emph{Saxophone} - Performances with lim and Marc Ducret. 
\item [2007] \emph{Saxophone and computer} - ICMC Copenhagen
 \item[Since 2006] \emph{Computer} - Performances with Stefan \"Ostersj\"o (guitar) in Vietnam, China, USA, UK and Sweden.
 \item[Since 2003] \emph{Saxophone} - Performances in Sweden and USA in duo with Per-Anders
 Nilsson (laptop).
 \item[Since 2002] \emph{Computer} - Performances in Sweden, Iceland,
 USA, Canada and Germany in duo with Henrik Frendin (viola).
 \item[2002] \emph{Saxophone} - Tour in Mexico.
 \item[2002] \emph{Saxophone and computer} - ICMC, Gothenburg, Sweden
 \item[2001] \emph{Saxophone and computer} - ICMC, Habana, Cuba
 \item[2001] \emph{Saxophone} - Bell Atlantic Jazz Festival, New York. \emph{Michael Formanek
 Northern Exposure}.
 \item[2000] \emph{Saxophone} - Scandinavian tour as well as New York
 performances. \emph{Michael Formanek Northern Exposure}.
 \item[1996-2005] \emph{Saxophone and computer} - Several installations and performances in Scandinavia,
 Germany and Bellarus with Swedish visual artist Stefan Lundgren.
 \item[1997] \emph{Saxophone} - Tour and recordings with Blue Pages, nominated Swedish Jazz
 Group of the Year.
 \item[1996-98] \emph{Saxophone} - Tours and recordings with pianist Richie Beirach.
 \item[1995] \emph{Saxophone} - Swedish tour with David Liebman.
 \item[1992] \emph{Conductor, saxophone} - Montreux International Jazz Festival. \emph{The Orchestra}
\end{enumerate}

%%% Local Variables: 
%%% mode: latex
%%% TeX-master: "vetenskaplig-meritportfolj"
%%% End: 


% \vspace{1cm}
% 

\begin{tabular}[c]{| p{3cm}|l|l|l|l|l|p{2cm}|}
\hline
  \emph{Projektetnamn} & \emph{2009} & \emph{2010} & \emph{2011} & \emph{2012} & \emph{2013} & \emph{Finansiär} \\ [0.2cm] \hline
  
  (Re)Thinking Improvisation (MHM) & ~1.300 & ~1.300 & ~1.300 & 0 & 0 & VR \\ [0.8cm] \hline
  Virtuality and precence: a seamless performance space (KMH) & 0 & 0 & 20.000 & 0 & 0 & Knut och Alice Wallenbergs fond \\ [0.3cm] \hline
\end{tabular}
\vspace{0.1cm}

\noindent
\emph{All amounts in 1.000 SEK.}\\

\vspace{0.3cm}
\noindent
Note: I cannot easily tell how much of the received funds was given to me specifically. The funds from KAW was for infrastrcture for creating an artistic research hub at KMH which is now in progress of being built.

%%% Local Variables: 
%%% mode: latex
%%% TeX-master: "vetenskaplig-meritportfolj"
%%% End: 


% \newpage

\printbibliography
\end{document}