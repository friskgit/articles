%%%%%%%%%%%%%%%%%%%%%%%%%%%%%%%%%%%%%%%%%
% Beamer Presentation
% LaTeX Template
% Version 1.0 (10/11/12)
%
% This template has been downloaded from:
% http://www.LaTeXTemplates.com
%
% License:
% CC BY-NC-SA 3.0 (http://creativecommons.org/licenses/by-nc-sa/3.0/)
%
%%%%%%%%%%%%%%%%%%%%%%%%%%%%%%%%%%%%%%%%%

%----------------------------------------------------------------------------------------
%	PACKAGES AND THEMES
%----------------------------------------------------------------------------------------

\documentclass{beamer}

\mode<presentation> {

% The Beamer class comes with a number of default slide themes
% which change the colors and layouts of slides. Below this is a list
% of all the themes, uncomment each in turn to see what they look like.

%\usetheme{default}
%\usetheme{AnnArbor}
%\usetheme{Antibes}
%\usetheme{Bergen}
%\usetheme{Berkeley}
%\usetheme{Berlin}
%\usetheme{Boadilla}
%\usetheme{CambridgeUS}
%\usetheme{Copenhagen}
%\usetheme{Darmstadt}
%\usetheme{Dresden}
%\usetheme{Frankfurt}
%\usetheme{Goettingen}
%\usetheme{Hannover}
%\usetheme{Ilmenau}
%\usetheme{JuanLesPins}
%\usetheme{Luebeck}
\usetheme{Madrid}
%\usetheme{Malmoe}
%\usetheme{Marburg}
%\usetheme{Montpellier}
%\usetheme{PaloAlto}
%\usetheme{Pittsburgh}
%\usetheme{Rochester}
%\usetheme{Singapore}
%\usetheme{Szeged}
%\usetheme{Warsaw}

% As well as themes, the Beamer class has a number of color themes
% for any slide theme. Uncomment each of these in turn to see how it
% changes the colors of your current slide theme.

%\usecolortheme{albatross}
%\usecolortheme{beaver}
%\usecolortheme{beetle}
%\usecolortheme{crane}
%\usecolortheme{dolphin}
%\usecolortheme{dove}
%\usecolortheme{fly}
%\usecolortheme{lily}
%\usecolortheme{orchid}
%\usecolortheme{rose}
%\usecolortheme{seagull}
%\usecolortheme{seahorse}
%\usecolortheme{whale}
%\usecolortheme{wolverine}

%\setbeamertemplate{footline} % To remove the footer line in all slides uncomment this line
%\setbeamertemplate{footline}[page number] % To replace the footer line in all slides with a simple slide count uncomment this line

%\setbeamertemplate{navigation symbols}{} % To remove the navigation symbols from the bottom of all slides uncomment this line
}

\usepackage{graphicx} % Allows including images
\usepackage{booktabs} % Allows the use of \toprule, \midrule and \bottomrule in tables
\usepackage[utf8]{inputenc}
\usepackage[style=authoryear,natbib=true,backend=biber]{biblatex}
\bibliography{/run/media/henrikfr/Homer/Home/Documents/svn/admin/conf/biblio/bibliography.bib}
%\usepackage{enumitem}

%----------------------------------------------------------------------------------------
%	TITLE PAGE
%----------------------------------------------------------------------------------------

\title[Docentansökan]{Docentansökan} % The short title appears at the bottom of every slide, the full title is only on the title page

\author{Henrik Frisk} % Your name
\institute[UCLA] % Your institution as it will appear on the bottom of every slide, may be shorthand to save space
{
Musikhögskolan i Malmö \\ % Your institution for the title page
}
\date{\today} % Date, can be changed to a custom date

\begin{document}

\begin{frame}
\titlepage % Print the title page as the first slide
\end{frame}

% \begin{frame}
% \frametitle{Overview} % Table of contents slide, comment this block out to remove it
% \tableofcontents % Throughout your presentation, if you choose to use \section{} and \subsection{} commands, these will automatically be printed on this slide as an overview of your presentation
% \end{frame}

%----------------------------------------------------------------------------------------
%	PRESENTATION SLIDES
%----------------------------------------------------------------------------------------

%------------------------------------------------
\section{First Section} % Sections can be created in order to organize your presentation into discrete blocks, all sections and subsections are automatically printed in the table of contents as an overview of the talk
%------------------------------------------------

\subsection{Subsection Example} % A subsection can be created just before a set of slides with a common theme to further break down your presentation into chunks

\begin{frame}
\frametitle{Innehåll}

\begin{enumerate}
\item Konstnärligt forskningsarbete sedan disputationen.
\item Åberopade arbeten.
\item Kortfattad CV
\item Merit och tjänsteförteckning enligt LUs meritportfölj
\item Förteckning över bilagor
\end{enumerate}

\end{frame}

%------------------------------------------------

\begin{frame}
\frametitle{Innehåll meritförteckning}

Merit och tjänsteförteckning enligt LUs meritportfölj:
\begin{enumerate}
  \item[D.] Vetenskaplig meritförteckning
    \begin{enumerate}
      \item Forskningserfarenhet
      \item Publikationer vetenskapliga
      \item Publikationer konstnärliga
      \item Forskningsanslag
    \end{enumerate}
  \item[E.] Pedagogisk meritförteckning
  \item[F.] Ledarskap och administration
  \item[G.] Samverkan, innovation och entrepenörskap
  \end{enumerate}

\end{frame}

%------------------------------------------------

\begin{frame}
\frametitle{Main threads since dissertation}

\begin{itemize}
\item \textbf{Self and otherness}
\item \textbf{Artistic research as an actor in social and political thinking}
\item \textbf{Expanding the field of improvisation}
\item \textbf{Artistic research methodology}
\end{itemize}

\end{frame}

%------------------------------------------------

\begin{frame}
\frametitle{Self and otherness}

Reconfiguring the binary relation between self and other: Allowing the Self to ``be informed by what is now going on in the process rather than by what has traditionally shaped it'' 

Obviously a huge field of investigation however made manageable by keeping the interrogations within my artistic practice. It also forms the context for all of the following three areas.

\vspace{0.5cm}
\fullcite{frisk2013}.

\end{frame}

%------------------------------------------------

\begin{frame}
\frametitle{Artistic research as an actor in social and political thinking}

``Placing the artistic work in the light of a particular social, theoretical, cultural, or philosophical framework [\ldots] causes the political dimension to surface.'' 

Concretely, this field has benefited largely through the many interactions I have had with Vietnam in recent years. My current plan is to include other cultures in my research and look at close interactions with North Africa.

\vspace{0.5cm}
\fullcite{frisk-ost13}. 

\end{frame}

%------------------------------------------------

\begin{frame}
\frametitle{Expanding the field of improvisation}

Reconsidering the impact of individuality, habit and freedom: ``The impact of freedom, being such an essential concept in the understanding of improvisation, is closely related to some of the more social and political topics [\ldots] and can be understood in a number of ways, such as freedom \emph{of the self} and freedom \emph{from the self}'' 

\vspace{0.5cm}
\fullcite{frisk12-improv}

\end{frame}

%------------------------------------------------

\begin{frame}
\frametitle{Artistic research methodology}

In \emph{Beyond Validity: claiming the legacy of the artist-researcher} we claim that it is necessary to question replacement terms such as `silent knowledge', `narration' and `new knowledge', ``and trust the power and efficiency of the artistic practice to be solid enough to withstand the impact of established and hybrid qualitative research methods without losing its qualities as art while displaying its potential as research.'' In this area, however, I am also working with more tangible tools for documenting artistic practice. The aspect of documentation is in many cases an integral part of artistic research methodology.

\vspace{0.5cm}
\fullcite{frisk-ost13}
\end{frame}

%------------------------------------------------

\begin{frame}
\frametitle{Current and future research projects}

\begin{itemize}
\item Text as sound and sound as text
\item Localizing nature and composition
\item Contemporary methods for improvisation
\item Documentation of artistic research projects
\end{itemize}
\end{frame}

%------------------------------------------------

\begin{frame}
\frametitle{Åberopade arbeten}
  \fullcite{frisk-re_improv13} %\vspace{0.1cm}
  \begin{enumerate}
  \item \fullcite{frisk-transparent}
  \item \fullcite{frisk2013}
  \item \fullcite{friskostersjo2013b}
  \end{enumerate}

% Dessa fyra arbeten är delar av det postdoc projekt jag gjorde med fokus på improvisation. Det första är den av mig och Stefan Östersjö editerade publikation som de följande tre ingår i. Den tredje delen är en gemensam text som jag och Östersjö har författat tillsammans till lika del.

\end{frame}

%------------------------------------------------

\begin{frame}
\frametitle{Åberopade arbeten}
\fullcite{frisk12-improv}

Detta bokkapitel är också kopplat till postdoc-projektet ovan (se \emph{2-ImprovisationAndSelf})

\vspace{0.4cm}

\fullcite{sixtones13}

En CD som är ett resultat av ett då femårigt samarbete och kring vilket många av texterna här kretsar.

\vspace{0.4cm}

\fullcite{frisk-better}

Ett större projekt till Göteborgsbiennalen 2011 beställt av curator Sarat Maharaj och Gertrud Sandqvist. 

\end{frame}

%------------------------------------------------

\begin{frame}
\frametitle{Åberopade arbeten}

  \fullcite{frisk-ost13}

En artikel författad till lika delar av mig och Stefan Östersjö som tar ett grepp om den fortsatta utvecklingen av konstnärlig forskning.

\vspace{0.4cm}

  \fullcite{frisk-cobussen09}

I detta arbete har vi konceptuellt jobbat gemensamt, men praktiskt har jag gjort all musik.

\vspace{0.4cm}

  \fullcite{frisk08}

\end{frame}

%------------------------------------------------

\begin{frame}
\frametitle{Åberopade arbeten}

  \fullcite{frisk10}

Huvudsakligen författad av mig med kommentarer och inpass från Henrik Karlsson, också redaktör för boken.

\vspace{0.4cm}

  \fullcite{frisk-mystic}

Detta verk är en konstnärlig förstudie till serien av arbeten i 1-2 ovan.

\vspace{0.4cm}

\end{frame}


\end{document} 