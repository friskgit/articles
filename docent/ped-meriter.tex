
\section*{\textsf{Pedagogiska meriter}}

\subsection*{\textsf{Formell högskolepedagogisk utbildning}}

Handledarutbildning i handledning av konstnärliga doktorander, CED. Se bilaga.

\subsection*{\textsf{Ämnesrelevant pedagogisk utbildning eller annan pedagogisk utbildning}}

Ingen formell.

\subsection*{\textsf{Andra erfarenheter av pedagogisk natur som motsvarar eller kompletterar}}

Jag har lång erfarenhet av undervisning på flera nivåer och på flera olika skolor i flera länder. Min undervisning har uppskattats av elever och såväl i Köpenhamn som i Malmö och Stockholm var det studenternas önskemål att jag skulle ge undervisning. Jag vill också påpeka att det inte är ovanligt att lärare vid konstnärliga högskolor saknar pedagogisk utbildning. Jag tar mitt pedagogiska arbete på största allvar och är intresserad av att utveckla metoder för min undervisning och min handledning. Vad gäller den mer allmänna högskolepedagogiska utbildningen, som jag även föreläst i vid ett tillfälle på musikhögskolan, ämnar jag att ta den distanskursen vid CED nästa gång tillfälle ges.

\subsection*{\textsf{Undervisningserfarenhet eller motsvarande}}

\begin{enumerate}
\item Fridhems folkhögskola: saxofon och ensemble
\item Skurups folkhögskola: saxofon och ensemble
\item Rytmisk Musikkonservatorium, Köpenhamn: musikteori, komposition, dirigering och storbandsledning.
\item Odense Musikkonservatorium, Odense: ensemble
\item Musikhögskolan i Malmö: Pedagogisk ledning av utbildningen för jazz och improvisationsmusik, musikhistoria, musikteori och praktik, ensemble, saxofon.
\item Kungliga Musikhögskolan i Stockholm: handledning av doktorander, handledning av masterstudenter, pedagogisk ledning.
\item Musikhögskolan i Malmö: Undervisning i forskarutbildningskurser.
\item Gästlärare och/eller föreläsare vid 
  \begin{enumerate}
  \item UC Berkeley, CA, USA
  \item University of Maine, USA
  \item Berklee College, Boston, USA
  \item UC San Diego, USA
  \item Det Fynske Musikkonservatorium, Odense
  \item Esbjerg Musikkonservatorium, Esbjerg
  \item Sibelius Akademien, Helsingfors
  \item Hanoi Music Conservatory, Hanoi, Vietnam
  \end{enumerate}
\end{enumerate}

\subsection*{\textsf{Handledning på grund- och avancerad nivå}}

\begin{enumerate}
\item Kungliga Musikhögskolan i Stockholm: handledning av
  doktorander, handledning av masterstudenter, pedagogisk ledning.
\item Musikhögskolan i Malmö: Handledning av kandidat och
  masterstudenter, handledning av doktorander.
\end{enumerate}


\subsection*{\textsf{Pedagogiskt ledarskap}}

\begin{enumerate}
\item Musikhögskolan i Malmö: Utbildningen för jazz och improviserad musik, 1999-2004
\item Kungliga musikhögskolan i Stockholm: Konstnärliga forskarutbildningen, 2011-
\end{enumerate}

\subsection*{\textsf{Pedagogiskt utvecklingsarbete}}

\begin{enumerate}
\item Samarbetet inom närverket NordPuls för jazzutbildningar i
  Skandinavien handlade till stor del om pedagogiskt utvecklingsarbete
  och kursplansutveckling.
\item Arbetet som initierades och stöddes av KU-nämnden, där jag sitter som vice ordförande, är i sin helhet att betrakta som pedagogiskt utvecklingsarbete.
\end{enumerate}


\subsection*{\textsf{Läromedelsproduktion och publikationer}}

\begin{enumerate}
\item Kapitel i \emph{Antologi för handledning av konstnärliga doktorander} (arbetstitel), red. Karin Johansson.
\item Bistått och tagit initiativ till läromedelsframtagning genom mitt vice ordförandeskap i KU-nämnden sedan 2005.
\item En artikel om det konstnärliga forskningsseminariet under produktion. Medförfattare: Karin Johansson.
\end{enumerate}

\subsection*{\textsf{Nationellt och internationellt pedagogiskt arbete}}

Se Undervisningserfarenhet ovan.

\subsection*{\textsf{Internationaliseringsarbete inom den pedagogiska praktiken}}

Det skandinaviska ERASMUS-nätverket NordPuls.

\subsection*{\textsf{Rapporteringsuppdrag och utvärderingsuppdrag}}

Ett utvärderingsuppdrag som jag kommer arbeta med löpande under 2014/2015 är rapporten och utvärderingen av Konstnärliga forskarskolans handledarutbildning 2011-2015.

\subsection*{\textsf{Symposier, konferenser, workshops och samarbeten}}

\begin{enumerate}
\item Ansvar för en konferens i det skandinaviska
  ERASMUS-nätverket NordPuls i Malmö 2002.
\item Ansvar för de återkommande symposierna ForMuLär (Forum för musikaliskt lärande) som under flera år producerades av KU-nämnden och som jag höll i.
\end{enumerate}


\subsection*{\textsf{Utmärkelser och priser inom pedagogisk verksamhet}}

Inga utmärkelser eller priser.


%%% Local Variables: 
%%% mode: latex
%%% TeX-master: "pedagogisk-meritportfolj"
%%% End: 
