\documentclass{article}
\usepackage[english]{babel}
\usepackage[T1]{fontenc}
\usepackage{url}
\usepackage[utf8]{inputenc}
\usepackage{enumitem}
\usepackage{csquotes}
\usepackage{fixltx2e}
\renewcommand{\encodingdefault}{T1}
%\usepackage{hyperref}
\usepackage[pdftex]{graphicx}
%\usepackage[authoryear,round]{natbib}
\usepackage[style=authoryear,natbib=true,backend=biber]{biblatex}
\bibliography{~/shared/Documents/svn/admin/conf/biblio/bibliography}
\renewcommand{\rmdefault}{pad}
\renewcommand{\sfdefault}{pfr}

\begin{document}

\title{Self and otherness, text and sound: the political dimension of artistic research}
\author{Henrik Frisk\\Docentföreläsning - Musikhögskolan i Malmö\\{\small henrik.frisk@mhm.lu.se}}
\date{}

\maketitle

\section*{Abstract}

In this lecture I will discuss some of the implications of working as an artistic researcher. First by examining the conceptualization of the artist and the ways in which it may stand in the way of the research effort. Then, I will use a recent work as an example of method development and how information flows between both practice and research. Finally, I will present how the political dimension of artistic research may surface and how it may be developed in future projects.

\section*{Self and Otherness}

One of the sections taken out from my 2008 PhD dissertation ``Improvisation, computers and interaction'' \cite{frisk08} was the one concerning the Self. From the beginning the idea of the Self as the defining difference between man and machine seemed to me one of the more important aspects of the investigation I was carrying out. Due to the way the project had developed, however, in the end it proved to be less important. The topic of human-computer interaction and improvisation had taken new and unexpected turns and cutting this section out was part of the often common process of narrowing down a thesis to give it more focus. After the dissertation, however, I wanted to approach this vast and difficult area of the Self and begin to look into its constitution in art and improvisation. It became the focus of my post doc research within the project \emph{(Re)Thinking Improvisation}. 

The self influences so many aspects that I have already explored, such as the concept of \emph{control}, \emph{autonomy}, and of the \emph{work in motion}. In almost all my works during the five years that has passed since my PhD defense, the notion of the Self and its relation to the work, the collaborators, the audience, and the field, has played a central role. I have been aiming for an expanded context for artistic production and research in which the 'author' is distributed among several agents.

After all, the discussion about the ontology of the musical work, in particular the idea that the score can constitute a work independent of its sonic realization, is old and not particularly interesting. What is interesting however, is to regard 'works' in a constant state of flux. As becoming-work. The abstract idea of becoming-work as compared to the actualized work is not only a battle between fixture and motion but also between the perceived and the imagined. Actualized in the question, often attributed to Berkeley, whether the tree makes a sound if no one is around to hear it, can perhaps be rephrased as: does the musical work exist unless it is heard?.

What I am discussing here is not to outsource parts of the work creation, nor to remove myself from the creative process. Rather, it is about taking responsibility for a relation with the other (listener, co-musician, co-artist, etc.) where \emph{listening} is one of the key components. It is about the acceptance that control and result is less interesting than process and about giving up ownership and authority in an attempt to open-source the musical work. The prevailing image since the romantic era is one where the artist is at the nucleus of artist production and communication and in the context of artistic research that image has to be dismantled. But even outside the narrow field of practice based research is this conception of the artist quite unproductive if the goal is communication and development. 

Henry David Thoreau, a main character of the development of the objectivist view, whose work became an important inspiration for John Cage, speaks of the ``transparent eye-ball'' and the ``objective I'' \citep{thoreau2004}. However, objectivity is not the main matter here. Rather, it is the space for subjectivity that holds the key to the success of important aspects of collaborative practices. In our joint paper \emph{Beyond Validity} Stefan and I write: ``Validity then is fundamentally a matter of making the subjectivity of the artist visible in the research design. The need for creating a multi-layered understanding of subject-positions does come out clearly in studies of collaborative creativity'' \citep{frisk-ost13}. In collaborative practices negotiation and sharing are at the center, as is the general philosophy of being open to the perspective of the other and giving priority of eye over I. 

This altered perspective opens up a number of interesting ethical dimensions. In my work they grew out of the artistic projects, in particular out of the work with Vietnamese-Swedish group The Six Tones and the different projects we engaged in. Its intercultural context made it necessary to probe the questions of identity, belonging, difference, otherness, ontology and epistemology in relation to our work, to music and to the inter cultural context we are situated in. In this dimension the important question of identity has to be investigated. The psychologically charged relationship of identity and Self has a special meaning in postcolonial theory of Gayatri Chakravorty Spivak (and others) as many of the common components for constructing identity are rooted in us-and-them relations. The hypothesis that I have been developing through this work is that the excluding identity of the Self as a binary relation to the other (as in soloist-accompanist, composer-performer, performer-listener, art music-traditional music, Westerner-foreigner, etc.) may be deconstructed in artistic practices through the tools I developed in my thesis. These are primarily the ideas concerning the distribution of the creative process. Some of the artistic output of this process of the Self can be heard on the double CD \emph{Signal in Noise} \citep{sixtones13} and the composition \emph{The Transparent I} \citep{frisk-transparent}. In the book chapter \emph{Improvisation and the self} \citep{frisk12-improv} from the book \emph{Soundweaving: Writings on Improvisation} the social and political aspects are discussed:
%%%%% POLITICAL AND SOCIAL

\begin{quote}
  We must also include and consider the fact that the self is
  continuously constructing the other, and similarly, how the dynamics
  of the self is influenced by social and political powers. Finally,
  it is important to remember that the social and political domains
  themselves may be influenced and even altered by how the self and
  the other is constituted.
\end{quote}

In a complex interplay between music, improvisation, consciousness through reflection, experimentation, methodological rigor, practice, the social and the political, I have begun to deconstruct the Self in a way that makes possible an alternative understanding of relations to the other. It is in this respect that I see art as an inherently political act. Because of the communicative aspect of performative art it is also a practice and theory of influencing other people. It is, however, also a practice that may easily deceive us to believ that the artist is merely a medium rather than an actor. But we will return to this soon and focus on a related bearing of political and social structure.

\section*{Drinking and working with text}

Language. Language is the ultimate political tool. It is constructed in a power relation to both the speaker and the listener. The logocentric attitude in the West is probably obvious to most and artistic research exists not outside of it but neither fully within it. Personally I am leaning on an assumption that the epistomology of art is to be sought in the expression of the artistic language rather than in its constitution. This is opposed to de Saussure's claim that the structure of \emph{parole} is revealed through the structure of \emph{langue} and alligns with Derrida's deconstruction and the idea of a center being outside of its centre. 

By related reasons the Self cannot be the centre of the artistic research practice while it is the very centre. It is the centre outside of centre that allows for the subjective stance. These relations that would be obscured by the Self as centre similarly to how the work as art would risk to cloud the work-in-motion. Escaping the central point of reference and negating the hidden structure within is for me what allows for an act of research that retains its openness to change. \emph{Il n'y a pas de hors-texte}. 

%Which turns our attention back to the politics of artistic research.

In a recent project we are turning to text. Text as a carrier of non-semantical meaning, as sound. We attempt to break down the power structures of language as meaning and turn to other readings, if you wish. The background and starting point for this project was a piece we did with a Thai poet Zakariya Amataya whose recording of a reading of one of his poems we used as a basis for a structured improvisation. The method, listening to the sounds of his voice and building new relations between the utterances was stimulating and inspirational.

In a new piece, not yet recorded, I worked with \emph{A Drinking Song} by W. B. Yeats in a translation to Vietnamese, read and recorded by Nguyen Thanh Thuy. Using the recording I made several layers of analysis based on timbre, velocity, loudness, pitch, etc. Then I mapped this data to different parameters of the graphic layouts in five different layers. It is difficult to say anything about the result yet but based on a first rehearsal the score conveys information to the performers valid to the attempt to make musical meaning out of an abstract text.

This method of working I find interesting in the light of the discussion of the political aspect of artistic research that we will soon returns to. Dismantling language as the dominating vehicle for distributing knowledge, and one of the major oppressive factors, is in itself a goal worth aiming for. But the more interesting relations lies in the attempt of this project to use the errors in translation as a creative power. 

One general notion that may be applicable to the entire field of artistic research is that artistic practice carries a hidden and hitherto largely unexplored potential for knowledge and learning. My personal reflection on this is that I am convinced that artistic practice has the potential for influencing many other fields of knowledge in ways that would result in many new perspectives, in addition to the interdisciplinary expansions and transformations it may offer. This potential is not always revealed within the practice, which is not to say that it is not there, but that it requires an additional force to surface. This force may reveal itself in different configurations, out of which artistic research is one. The research configuration is actualized in relation to the artistic practice through the method and the theory, and either of these may be brought about entirely within the artistic domain, or outside of it, or anywhere in between. In other words, an artistic research process may take place entirely within the practice of the researcher, without including any obvious material apart from the artistic expression.\footnote{This is not, however, to say that all artistic expression are artistic research.} In reality the relations between the art work, the research, the practice, the theory and the method are extremely complex and probably not useful to attempt to generalize. It is, however, worth noting the impact the research most often has on both the practice and the expected output. The borders between the art practice and the artistic research may not be easy to detect but in my own experience almost any operation on, or scrutiny of, the artistic practice changes it and ones own impression of it. The detached observer that unobrusively explores the art object is an impossibility. \citet{heisenberg1958} similarily concludes that while trying to determine what happens with a quantum particle ``the term `happens' is restricted to the observation.  Now, this is a very strange result, since it seems to indicate that the observation plays a decisive role in the event and that the reality varies, depending upon whether we observe it or not.'' (p. 21)


The work done while preparing for the recording of \emph{Signal in Noise} by The Six Tones may also be seen as a process that explored errors in translation. It did however turn out to be the beginning of a new direction for the group. The way we engaged with different aspects of the diversified musical life of Hanoi was an idea the artistic influence of which we had not anticipated. We boldly emerged into quite radical improvisations with traditional Vietnamese musicians, and we brought our own acoustic expressions to the studio with the contemporary electronic playing of Tri Minh and Vu Na Than. These sessions exemplify the potential complexity of artistic work. In music we often speak of the ``natural'' or ``what comes naturally'', as something that is effortless or that is meant to be. Our playing in these sessions was effortless but at the same time there was a considerable resistance involved, a resistance that did not necessarily had to be overcome--it would in fact have been a mistake to try to overcome it--but that had to be dealt with. Hence, though we quickly arrived at a satisfying result, the process was anything but effortless to me. Partly, the issue was related to the political and social aspects of me as a white man abroad making use of the traditions and knowledge of people economically much less fortunate than me. The CD project made these thoughts surface and I make an attempt to discuss them in the chapter \emph{Improvisation and the self: to listen to the other} in the book \emph{Soundweaving}:

\begin{quote}
  In order to truly be able to encounter the new and the unforeseen,
  challenging different aspects of the notion of ``center'' and
  ``periphery'' was necessary: is Western art music the norm and
  traditional Vietnamese music an exotic other? Are Stefan and I
  `visiting' a music outside of our own sphere, or is it rather Thuy
  and My that are forced to approach us. Is it at all possible to
  communicate on equal terms? The social impact of the Eurocentric
  view of the world, however, should not be underestimated. Stefan and
  I belong to what Mark \citet{slobin1987} labels ``the superculture''
  (p. 31), and the complex political and economic asymmetry between
  east and west plays an important role in our understanding of the
  other in our multi layered work with traditional Vietnamese music in
  general, and with The Six Tones in particular. \citep{frisk12-improv}
\end{quote}

In the project The Six Tones I place myself in a context where I \emph{have} to consider the political impact of my artistic activities even before they themselves become political. After all, if we were not be able to avoid becoming part of an act of cultural appropriation the ambition of the project would fail.

Jacques \citet{attali85} points to the merchandising prospect of most music and how that implicitly adds a political facet to it. But music, however, also heralds a subversion and a possibility for {\textquotedblleft}a radically, new organization{\textquotedblright} that is yet unimagined (ibid. p. 5). Artistic research in general, and the projects discussed above, may be considered an arena where such new organizations may advance and flourish.



\printbibliography
\end{document}
