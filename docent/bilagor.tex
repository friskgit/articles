\documentclass[a4paper]{article}
\usepackage[english]{babel}
\usepackage[T1]{fontenc}
\usepackage[authoryear,round]{natbib}
\usepackage{url}
\usepackage[utf8]{inputenc}
\usepackage{enumitem}
\renewcommand{\encodingdefault}{T1}
\usepackage{fancyhdr}
\pagestyle{fancy}
\renewcommand{\rmdefault}{pad}
\renewcommand{\sfdefault}{pfr}

\title{Bilagor}
%\author{Henrik Frisk}
\date{}

\begin{document}
\selectlanguage{english}
%\maketitle

\thispagestyle{empty}

\section*{\textsf{V. Bilagor}}


\vspace{2cm}

Textbilagor och partitur ligger i mappen ``Bilagor''. Klingande material skickar jag så snart jag fått besked om vem som ska ha det och hur många kopior det behövs.

\section*{\textsf{Bilagor B}}

\begin{enumerate}
\item[1.] Intyg om doktorsexamen (Intyg-101011.pdf)
\item[2.] Kursintyg, komposition (MHM-komposition.pdf)
\item [3.] Intyg och rekommendation, Köpenhamn (Rytkons.pdf)
\end{enumerate}

\section*{\textsf{Bilagor D}}

\begin{enumerate}
\item [4.] ReThinking Improvisation, publikation (RethinkingImprovisation-Publication.pdf)
\item [5.] Recension av Kim Hedås avhandling (Hedås-review.pdf)
\item [6.] Jeremy Cox, rekommendation till SUA (Cox-rek.pdf)
\item [7.] Cecilia Hultberg, rekommendation till SUA (Hultberg-rek.pdf)
\end{enumerate}

\section*{\textsf{Bilagor E}}

\begin{enumerate}
\item [8.] Intyg CED (Intyg-CED.jpg)
\item [9.] Intyg Musikhögskolan i Malmö (MHM.pdf)
\item [10.] Intyg Landskrona kulturskola (Landskrona.pdf)
\end{enumerate}

\section*{\textsf{Bilagor F}}

\begin{enumerate}
\item [11.] Rekommendation, ideellt arbete (Collen-rek.pdf)
\end{enumerate}

\section*{\textsf{Bilagor G}}

\begin{enumerate}
\item [12.] Intyg, KU-nämndsarbete (Hellsten-KU.pdf)
\end{enumerate}


\end{document}