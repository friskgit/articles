\documentclass[a4paper]{article}
\usepackage[english]{babel}
\usepackage[T1]{fontenc}
\usepackage[authoryear,round]{natbib}
\usepackage{bibentry}
\usepackage{url}
\usepackage[utf8]{inputenc}
\renewcommand{\encodingdefault}{T1}
\usepackage{fancyhdr}
\pagestyle{fancy}

%%\usepackage[style=authoryear,natbib=true,backend=bibtex]{biblatex}
\bibliography{/run/media/henrikfr/Homer/Home/Documents/svn/admin/conf/biblio/bibliography.bib}

%\lhead{\small{\textit{Henrik Frisk}}}
%\chead{}
%\rhead{\small{\textit{title}}}

\title{Docentansökan}
\author{Henrik Frisk}
\date{\today}

\begin{document}
\selectlanguage{english}
\maketitle

\thispagestyle{empty}

\section{Introduction}

Themes: Interaction, Improvisation, The self, Ethics, Technology, Collaboration, Politics

\bibentry{frisk2013}

The main threads of my work since my dissertation include:

\begin{itemize}
\item \textbf{Self and otherness}: reconfiguring the binary relation between self and other: Allowing the Self to ``be informed by what is now going on in the process rather than by what has traditionally shaped it'' \citep{frisk2013}.
\item \textbf{Artistic research as an actor in social and political thinking}:  ``placing the artistic work in the light of a particular social, theoretical, cultural, or philosophical framework [\ldots] causes the political dimension to surface'' \cite{frisk-ost13}.
\item \textbf{Expanding the field of improvisation} reconsidering the impact of individuality, habit and freedom: ``The impact of freedom, being such an essential concept in the understanding of improvisation, is closely related to some of the more social and political topics [\ldots] and can be understood in a number of ways, such as freedom \emph{of the self} and freedom \emph{from the self}'' \cite{frisk12-improv}
\item \textbf{Artistic research methodology}: In \cite{frisk-ost13} we claim that it is necessary to question replacement terms such as `silent knowledge', `narration' and `new knowledge' ``and trust the power and efficiency of the artistic practice to be solid enough to withstand the impact of established and hybrid qualitative research methods without losing its qualities as art while displaying its potential as research.''
\end{itemize}

\section{Work since dissertation}

One of the sections taken out from my 2008 PhD dissertation ``Improvisation, computers and interaction'' \cite{frisk08} was the one concerning the Self. From the beginning the idea of the Self as the defining difference between man and machine seemed to me one of the more important aspects of the investigation I was carrying out. Due to the way the project had developed, however, in the end it proved to be less important. The topic of human-computer interaction and improvisation had taken new and unexpected turns and cutting this section out was part of the often common process of narrowing down a thesis to give it more focus. After the dissertation, however, I wanted to approach this vast and difficult area of the Self and its constitution in art and improvisation and it became the focus of my post doc research within the project \emph{(Re)Thinking Improvisation}. 

The self influences so many aspects that I have already explored, such as the concept of \emph{control}, \emph{autonomy}, and of the \emph{work in motion}. In almost all my works during the five years that has passed since my PhD defence, the notion of the Self and its relation to the work, the collaborators, the audience, and the field has been a central aspect. I have been aiming for an expanded context for artistic production and research in which the 'author' is distributed among several agents.
% (approaching Berkeley's question if the tree makes a sound if noone hears it, rephrased as: does the musical work exist unless it is heard?). 
What I am discussing here is not to outsource parts of the work creation, nor to remove myself from the creative process. Rather, it is about taking responsibility for a relation with the other (listener, co-musician, co-artist, etc.) where \emph{listening} is one of the key components. It is about the acceptence that control and result is less interesting than process and about giving up ownership and authority in an attempt to open-source the musical work. Henry David Thoreau, a main character of the development of the objectivist view, whose work became an important inspiration for John Cage, speaks of the transparent eye-ball and the objective I \cite{thoreau2004}. However, objectivity is not the main matter here. Rather, it is the space for subjectivity that holds the key to the success of important aspects of collaborative practicies:``Validity then is fundamentally a matter of making the subjectivity of the artist visible in the research design. The need for creating a multi-layered understanding of subject-positions does come out clearly in studies of collaborative creativity'' \cite{frisk-ost13}. In collaborative practicies negotiation and sharing are at the center, as is the general philosophy of being open to the perspective of the other, a generally phenomenological approach, and giving priority of eye over I. 

The ethical dimension grew out of the artistic projects, in particular out of the Vietnamese-Swedish group The Six Tones and the different projects we engaged in. Its inter-cultural context made it necessary to probe the questions of identity, belonging, difference, otherness, ontology and epistemology in relation to our work, to music and to the inter cultural context we are situated in. In this dimension the important question of identity has to be investigated. The psychologically charged relationship of identity and Self has a special meaning in post-colonial theory as many of the common components for constructing identity are rooted in us-and-them binary relations. The hypthesis that I have been developing throught this work is that the identity of the Self as described in a binary relation to the other (as in soloist-accompanist, composer-performer, performer-listener, art music-traditional music, Westerner-foreigner, etc.) may be deconstructed in artistic practicies through the tools I developed in my thesis. These are primarily the ideas concerning the distribution of the creative process. The primary artistic contributions with reference to the self are the double CD Signal in Noise \cite{sixtones13} and the composition \emph{The Transparent I} \cite{frisk-transparent} and in \emph{Improvisation and the self} \cite{frisk12-improv} the social and political aspects are discussed:

\begin{quote}
  We must also include and consider the fact that the self is
  continuously constructing the other, and similarly, how the dynamics
  of the self is influenced by social and political powers. Finally,
  it is important to remember that the social and political domains
  themselves may be influenced and even altered by how the self and
  the other is constituted.
\end{quote}

In a complex interplay between music, improvisation, consciousness through reflection, experimentation, methodological rigour, practice, the social and the political I have begun to deconstruct the Self in a way that makes an alternative understanding of the other possible.

The work Stefan and I did in The Six Tones while preparing for the already mentioned recording of \emph{Signal in Noise} turned out to be the beginning of a new direction for the group we started in 2006. The Six Tones had done a few projects but the way we engaged with different aspects of the diversified musical life of Hanoi was an idea whose artistic influence we had not anticipated. We boldly emerged into quite radical improvisations with the traditional flute playing of xxx and we brought our own acoustic expressions to the studio with the contemporary electronic playing of Tri Minh and Vu Na Than. These sessions exemplify the potential complexity of artistic work. In music it is often speak of that which is natural, or that comes naturally, as something that is effortlessness is meant to be. Our playing in these sessions was effortless but to me there was a considerable resistance involved, a resistance that had to be overcome. Hence, though we quickly arrived at a satisfying result the process was anythin but effortless to me. This double experience of easyness and resistance was 

Even the Integra project whose initial aim was to fuse technology and music contributed to the shift in my artistic activities. Music technology, as much technology, has long been focussed on a rigid epistemology firmly based on the scientific field rather than the artistic. Though many of the artworks that emmanate from this field certainly are not scientific in their nature (such as Stockhausen's earlier work, Alvin Lucier's brilliant conceptual pieces or John Cage's low-tech electronic pieces) the discipline has since long been situated in an academic context where music technology specialistts are distinct from artists. The electronic music studio in Paris, IRCAM, developed the term \emph{musical assistant} who would contribute technical knowledge to composers. An essentially modernist idea of labour division where composers could not, or should not, deal with the technical issues in works containing technology. This may seem as a logical and pragmatic solution, similar to those made in theaters, opera houses, film productions, e.g. However, it is also one in which the technology is treated as an auxillary tool as compared to the main artistic work. In reality the \emph{musical assistant} became as much of an artist as the composer having to come up with creative solutions to impossible problems, albeit always a second grade artist never recognized for his or her contribution. The artistic impetus of the composer made the auteurship remain with him or her.

The goal of the integra project however was not primarily to counteract the authority of the creator but to facilitate for musicians and new music ensembles to work with electronics. Part of the task was to build tools for musicians and composers that were sophisticated yet easy to use. As a consequence, along with the work Stefan Östersjö and I did with \emph{Repetition Repeats all other Repetitions} and the studies entitled \emph{Negotiating the Musical Work} (all part of my dissetation) Integra further fueled my ideas concerning the \emph{work in motion}, a distributed work with no beginning and no end, an open sourced work of art that almost anyone can fork and continue to work on. In the end, there is much work yet to be done, and, obviously, not all projects are suitable for this method. Instead, the model can be seen as a guide, a method to consciously counteract the tendency for artistic expressions to fall back on modernist structures, closed for listener participation \cite[A collaborate paper on Repetition is in print along with a recording of a new version of the piece.][]{friskcoessens2013}. Towards the end of the project the Swedish composer Kent Olofsson and I, both of us involved in the scientific aspect of the Integra project, joined forces with chamber group Ars Nova, part of the artistic side of Integra, and created the digital chanber group \emph{Switched-On}. Five musicians all playing electronic, or electronically enhanced versions of their instruments. Both Kent and I wrote music for this ensemble and the setup, technically incredibly advanced, was inspired by the ideas developed in the Integra project. 

My contribution to this ensemble was the composition \emph{The Mystic Writing Pad}: 

\begin{quote}
\emph{This piece is an improvisation based on the structure of the American composer Harry Partch's 43-tone Just Intonation scale; a division of the octave in 43 unequal steps. The 43 notes of the scale have been distributed among the five instruments, and it is only together that they can explore the full potential of the scale. The collaborative aspect of this piece is further explored by its meta-instrumnts: instruments that are hidden under the surface and for which the players need to join forces in order to control.}

\emph{The title, The Mystic Writing Pad, refers to Freud's 1924 paper in which he lays out a hypothesis about the inner functionlity of human perception. Though much can be said about his hypothesis (and much has been said about it, not the least by French philosopher Jacques Derrida) my resons for choosing this title is much more practical and metaphorcal. The functionality of the technology upon which Switched-On relies can often be very mystical, but the ease with which it can be used to register the phrases played by the musicians is truly akin to a writing pad: great at quickly taking notes (in two senses of the word), but terrible at making thoroughly thought through statements.} \cite[][(Program note)]{frisk-mystic}
\end{quote}

The piece is in three parts where the first part is a structured improvisation, the second a purely electronic fixed media part and the third is an ensemble passage for midi-saxophone solo. 

%The projects listed as part of this project all explore the concept of \emph{work-in-motion} in different ways.

\clearpage


\section{Bibliografi}
%\printbibliography

\bibliographystyle{apalike}
\end{document}