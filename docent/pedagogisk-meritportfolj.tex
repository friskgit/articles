\documentclass[a4paper]{article}
\usepackage[english]{babel}
\usepackage[T1]{fontenc}
\usepackage[authoryear,round]{natbib}
\usepackage{url}
\usepackage[utf8]{inputenc}
\usepackage{enumitem}
\renewcommand{\encodingdefault}{T1}
\usepackage{fancyhdr}
\pagestyle{fancy}
\renewcommand{\rmdefault}{pad}
\renewcommand{\sfdefault}{pfr}
%\lhead{\small{\textit{Henrik Frisk}}}
%\chead{}
%\rhead{\small{\textit{title}}}

\title{E: Pedagogical merits}
%\author{Henrik Frisk}
\date{}

\begin{document}
\selectlanguage{english}
%\maketitle

\thispagestyle{empty}

\section*{\textsf{4E. Pedagogical merits}}
\section*{\textsf{Summary of pedagogical experience}}

The key components of my pedagogical occupation is to be found in my accumulated experience as a teacher. For the last three years my main preoccupation has been doctorate supervision and courses for supervisor's training. But the following is a summary of my experience since I started teaching:

\begin{enumerate}
\item \textbf{1992-1996: Pre-conservatory training}. Teaching ensemble and saxophone at pre-conservatory schools was an important starting point for me as a teacher. I used my own practice methods to teach my students technique and independent development.
\item \textbf{1994-2000: Copenhagen Rhythmic Conservatory}. I taught music theory, composition and arranging as well as conducted the big band and had individual classes in big band conducting. It was at CRC that I started working with Curriculum development of the music theory subject.
\item \textbf{1999-2004: Malmö Academy of Music}. Leading the performance program for jazz and improvised music as well as teaching saxophone, integrated theory/ear training/ensemble, music history and ensemble I developed the Curriculum, worked on international relations and pedagogic exchanges and integration with the rest of the Academy.
\item \textbf{2010--: Royal college of Music and Malmö Academy of Music}. Primarily working with doctoral supervision but also with supervision of master students. Also involved in the development of the examination process of the Master and Bachelor programs.
\item \textbf{2012--: Nationella Forskarskolan}. Primarily coordinating the supervisor's education program but also involved in the development of the structure for artistic supervisor's training.
\end{enumerate}

% saxophone and conducting (individual teaching) and classes in music theory, composition, arranging, big band, integrated theory/ear training/ensemble, music history. I have supervised master students

\section*{\textsf{Pedagogical self reflection}}

Sine the beginning of my university teaching career in 1997 I have taught first, second and third cycle students in several different subjects in both performance and composition and arranging programs. Prior to 1997 I also taught conservatory preparation programs and for a short time beginners in music school. In 1999 I assumed responsibility for the program for jazz and improvised music that I led for four years, developed the curriculum substantially and expanded from eight students to sixteen. For the last four years my teaching has mainly been focused on supervision of master students and doctoral students. However, I do not see my pedagogical challenge to be limited to teaching within the Academy. In my artistic practice there is a great need for thinking and working pedagogically in order to communicate with my audiences, press and promoters. Anything from presenting concerts and pieces on stage to writing press releases and proposals involves the translation from concept to statement. Even writing grant proposals involves the same kind of shaping of a pedagogically oriented message.

I see my teaching as an integral part of my artistic practice. It is as rewarding for myself as I am hoping it is for my students on all levels (though it is worth noting the great difference between the first two and the third cycle, something I will return to later). I would also like to further stress that the one of the potentially great assets for an institution having an artistic doctoral education is the way in which experience and skill can flow from tutors to doctorate students, and the other way around, and, most importantly, how the first and second cycles can benefit from the doctoral education. I believe that the definition of a successful university education structure is one where knowledge is allowed to pass freely between the different layers of the school. If the aim to integrate all three levels of studies at the academy while maintaining individual space for the students becomes succesful, I believe that our students are better prepared to meet their career challenges and embrace lifelong learning from peers and collegues.


\subsection*{\textsf{Curriculum development}}
\label{sec:curr-devel}

%As an interdisciplinary field artistic research will benefit from a close collaboration with many different 

So far artistic research has to a large degree depended on the social sciences for its theory courses, but I believe it is now time to develop a theory specific to the field of artistic research in the performing arts. Although I see no conflict as such between traditional theory and artistic theory--they can and should coexist--it is time to claim the theoretical competence of art researchers in their own right. I have written several course plans, most recently for artistic method development based on these notions, as well as a recent paper relating to these topics \citep{frisk-ost13}.

Developing courses for artistic research is a process balancing on a quite thin line at times. The particular needs of the individual doctorates need to be at equilibrium with the group of PhD students at large, expectations of the supervisors and the expectations of the Academy. What also needs to be weighed in is what should be considered individual study courses and what should be a regular course. These questions are an interesting part of the individual design of the artistic doctorate program and in my experience from Konstnärliga Forskarskolan these are important decisions that needs to carefully considered.

At the research seminar in Malmö during the last three semesters we have worked deliberately trying to find a dynamic structure while at the same time supporting the need for structure and forward planning. Similarly connected to the balance between individual and collective needs, in the case of the seminar we have allowed themes and questions that arise from the evaluation of previous courses, project or seminars to influence the planning of our future workshops. In a sense, the seminar is both the course evaluation and the course event. This process must of course be closely supervised or it could easily get out of hand. An other thing we introduced is that we invite other researchers at the Academy to join the seminar and we encourage them, as well as ourselves to carry out the tasks assigned to the students. This latter point is to inspire the exchange of knowledge throughout the department.

\subsection*{\textsf{Supervision}}

While supervising doctoral students in general has been a practice widely discussed and taught, supervision in the arts lacks that history of experience. It is a field that in some respects has to invent itself. Supervising bachelor and master students is very different in the arts from supervising doctoral students but is also in the need of further development.\footnote{One of the main reasons the bachelor and masters levels have been easier to deal with is that the requirements has hitherto been different. A written paper on any subject related to the field of study in combination with a concert is accepted. In artistic research education there is an expectation that the artistic and the written outputs should be highly integrated. It is my own opinion that this should also be required at first and second cycle.}

At the first and second cycles supervision in performance major in music is a counterpart to the individual instrument teaching to the student. Ultimately the supervisor and the teacher should work closely together in preparing for the final exam and to assure that the exam paper is written on a topic relevant to the major subject of the student. During several years, in both Stockholm and Malmö I have participated in the development of the structures for these works. Both schools did very well in a recent evaluation by Universitetskanslerämbetet of the exam program. My own experience with supervising bachelor and master students is that the process is closer to that of teaching than what supervision in the doctoral education is. Bachelor and master students are in a continuous learning process that makes the supervision part also closer to teaching. It is more concerned with what to do and what not to do.

The supervision in the third cycle however, is more concerned with the discussion of what might be done and what might not be done on a more subtle level. The supervisor is more of a peer in some respects, someone that can guide the doctoral student through the challenges encountered. I see my role as a supervisor as divided in several sub-tasks:

\begin{enumerate}
\item \emph{Protect}.\footnote{This is inspired from a talk by Mika Hannula in Gothenburg, 2002} To do a PhD in any subject can be very stressful, to do it in the arts may be even more so as any artistic reflection may also involve self reflection. At the same time many artistic research projects rely on this self reflection. Hence, I believe that I as supervisor should attempt to create a safe zone for the doctoral student where experiments can be carried out and where failure is perceived an asset rather than a threat.
\item \emph{File and focus}. Keeping track of what the student accomplishes is evidently necessary, but I also try to organize the material and file the projects independently of the students organization in order to be able to wisely supervise the structuring and focusing of the thesis. This process also allows me to find my own blind spots and identify what kind of work I have to do to stay in phase with the project.
\item \emph{Support and critique}. I also see as my role to carefully balance the support and the critique of the work, to allow the research student to feel safe enough to experiment and critiqued enough to make the experiment worthwile.
\end{enumerate}

From my experience supervising doctoral students the questions concerning theory and method along with documentation (which can be seen as part of the methodology) are currently the greatest obstacles. The lack of form for representation, the lack of methods for documentation and the lack of clear requisites in terms of written representation and text formats are troublesome to artistic PhD students today. This is the reason I have devoted my chapter in the upcoming Anthology on supervision in the arts to these questions \citep{frisk2015}.

Through my position as coordinator of the courses for supervisors at Konstnärliga Forskarskolan I have gained insight into the challenges of artistic research supervision, but also into research supervision in general. It is my belief that designing courses for supervisors in the field of artistic research is a key component for the healthy development of the field. There are unique challenges that needs to be tackled from the inside, so to speak, much like artistic research itself. An opportunity for supervisors to discuss their concerns and challenges has proven to be of great value to the supervisors in Konstnärliga Forskarskolan.

\subsection*{\textsf{Theory and method in artistic research}}

A theory represents a system of ideas that may shed light on a given phenomena or process. Sometimes we may expect the theory to generalize, or add a level of abstraction, to that which it discusses--to be independent of it, but theory may also be highly specific. Although many practices are based on theoretical principles, artistic practice is an activity that may develop completely independent of given theoretical frameworks. One aspect of using theory in artistic research is to comment on it based on experiences made in the artistic practice much in the same way that scientific research uses empirical data to feed back into the theory but we must bear in mind that in artistic research the theoretical trace left by it cannot by itself constitute the research.

Often theory is also used to situate the research within a field. By citing a given author I contextualize my research within the field of the work of that author and by criticizing the text I position myself against it. Although it is important to frame artistic research within a field one may ask oneself whether theory in terms of text is the best way to do it. Given that the nature of a theory is to generalize phenomena, it will always be difficult to look at an externally defined theory and use it to discuss an artistic project whose practice is the focal point. In a sense all artistic projects are singletons in the mathematical sense, sets with only one element, and attempts to generalize will always risk at distancing the explanation of the work from its core.

Going back to the general definition, that a theory represents a system of ideas, there is nothing to say that the theory needs to be defined in terms of text. In music serialism is a system of ideas, and so is postmodernism, abstract and generalized definitions that may be approached through their artistic exponents. When the object of research and the theory belongs to the same domain the aspect of generalization becomes less of a problem. (The category \emph{type-1-artworks} may generalize the object \emph{artwork} which may not as easily be generalized by \emph{type-1-theories}. In this sense, we are building a theoretical framework from a practice as a system of ideas.)

Even more important, however, is to approach the potential theory from \emph{within the practice}. Looking at my own practice I am, in a manner of speaking, a victim to my subjective aesthetics. No matter how much I try to control it, in the artistic process the work follows its own paths, or I subconsciously steers it in ``my'' direction\cite[See][for an extended discussion on this topic.]{frisk12-improv}. Hence, trying to theorize an artistic process by logical deduction is more likely to reveal what the researcher \emph{wants} to say, rather than what the artistic practice has to say. Instead, the research has to start from within it using a valid method.

\subsection*{\textsf{Developing research interest among music academy teachers}}
\label{sec:devel-rese-awar}

As artistic research is developing and expanding it is important to not disregard the competence among the teachers that have not yet engaged themselves in practice based research. At the Royal Academy of Music I initiated a competence development project for teachers at the academy. There was a great interest from other institutions and we discussed to give the course also at the music academy in Oslo, Norway and share the experiences. Due to practical and economical reasons the course has not yet been given but it is my conviction that such activities are necessary at the stage that we are in.

Typically a higher art education in Sweden has only a few teachers involved in artistic research. For the field to gain momentum there is a need for those professors not involved in research to expand their competences and engage in research like activities. This is not to say that all professors need to do research, but rather that we should explore different levels of research involvement. Thinking about one's own practice as a reflective practice may open the door to a different kind of practice. Furthermore, this may move some parts of the professors artistic activities closer to the academy whereas today they rarely take place at the school.

\subsection*{\textsf{Summary}}
\label{sec:teaching-experience}

In parallel to my artistic practice I have taught university level students for almost twenty years and I have been employed by three different institutions in two different countries. I enjoy teaching and I feel comfortable with my role as a teacher. In the last three years almost all of my teaching has been centered on supervision and on developing courses for training supervisors.

During 2013 I also taught electronic and experimental music in Hanoi, Vietnam in a democracy project involving Swedish SIDA. To have to rely on modes of communication other than the verbal/aural has been very rewarding and it has developed into an experience of communicating artistic knowledge through art.

To summarize, I believe that my strength in pedagogy, despite my lack in formal studies, is my ability to use all of my professional experience within the field of pedagogy and fit that within the spheres of Academy teaching and Curriculum development.

\bibliographystyle{apalike}
\bibliography{./../biblio}

\newpage


\section*{\textsf{Pedagogiska meriter}}

\subsection*{\textsf{Formell högskolepedagogisk utbildning}}

Handledarutbildning i handledning av konstnärliga doktorander, CED. Se bilaga.

\subsection*{\textsf{Ämnesrelevant pedagogisk utbildning eller annan pedagogisk utbildning}}

Ingen formell.

\subsection*{\textsf{Andra erfarenheter av pedagogisk natur som motsvarar eller kompletterar}}

Jag har lång erfarenhet av undervisning på flera nivåer och på flera olika skolor i flera länder. Min undervisning har uppskattats av elever och såväl i Köpenhamn som i Malmö och Stockholm var det studenternas önskemål att jag skulle ge undervisning. Jag vill också påpeka att det inte är ovanligt att lärare vid konstnärliga högskolor saknar pedagogisk utbildning. Jag tar mitt pedagogiska arbete på största allvar och är intresserad av att utveckla metoder för min undervisning och min handledning. Vad gäller den mer allmänna högskolepedagogiska utbildningen, som jag även föreläst i vid ett tillfälle på musikhögskolan, ämnar jag att ta den distanskursen vid CED nästa gång tillfälle ges.

\subsection*{\textsf{Undervisningserfarenhet eller motsvarande}}

\begin{enumerate}
\item Fridhems folkhögskola: saxofon och ensemble
\item Skurups folkhögskola: saxofon och ensemble
\item Rytmisk Musikkonservatorium, Köpenhamn: musikteori, komposition, dirigering och storbandsledning.
\item Odense Musikkonservatorium, Odense: ensemble
\item Musikhögskolan i Malmö: Pedagogisk ledning av utbildningen för jazz och improvisationsmusik, musikhistoria, musikteori och praktik, ensemble, saxofon.
\item Kungliga Musikhögskolan i Stockholm: handledning av doktorander, handledning av masterstudenter, pedagogisk ledning.
\item Musikhögskolan i Malmö: Undervisning i forskarutbildningskurser.
\item Gästlärare och/eller föreläsare vid 
  \begin{enumerate}
  \item UC Berkeley, CA, USA
  \item University of Maine, USA
  \item Berklee College, Boston, USA
  \item UC San Diego, USA
  \item Det Fynske Musikkonservatorium, Odense
  \item Esbjerg Musikkonservatorium, Esbjerg
  \item Sibelius Akademien, Helsingfors
  \item Hanoi Music Conservatory, Hanoi, Vietnam
  \end{enumerate}
\end{enumerate}

\subsection*{\textsf{Handledning på grund- och avancerad nivå}}

\begin{enumerate}
\item Kungliga Musikhögskolan i Stockholm: handledning av
  doktorander, handledning av masterstudenter, pedagogisk ledning.
\item Musikhögskolan i Malmö: Handledning av kandidat och
  masterstudenter, handledning av doktorander.
\end{enumerate}


\subsection*{\textsf{Pedagogiskt ledarskap}}

\begin{enumerate}
\item Musikhögskolan i Malmö: Utbildningen för jazz och improviserad musik, 1999-2004
\item Kungliga musikhögskolan i Stockholm: Konstnärliga forskarutbildningen, 2011-
\end{enumerate}

\subsection*{\textsf{Pedagogiskt utvecklingsarbete}}

\begin{enumerate}
\item Samarbetet inom närverket NordPuls för jazzutbildningar i
  Skandinavien handlade till stor del om pedagogiskt utvecklingsarbete
  och kursplansutveckling.
\item Arbetet som initierades och stöddes av KU-nämnden, där jag sitter som vice ordförande, är i sin helhet att betrakta som pedagogiskt utvecklingsarbete.
\end{enumerate}


\subsection*{\textsf{Läromedelsproduktion och publikationer}}

\begin{enumerate}
\item Kapitel i \emph{Antologi för handledning av konstnärliga doktorander} (arbetstitel), red. Karin Johansson.
\item Bistått och tagit initiativ till läromedelsframtagning genom mitt vice ordförandeskap i KU-nämnden sedan 2005.
\item En artikel om det konstnärliga forskningsseminariet under produktion. Medförfattare: Karin Johansson.
\end{enumerate}

\subsection*{\textsf{Nationellt och internationellt pedagogiskt arbete}}

Se Undervisningserfarenhet ovan.

\subsection*{\textsf{Internationaliseringsarbete inom den pedagogiska praktiken}}

Det skandinaviska ERASMUS-nätverket NordPuls.

\subsection*{\textsf{Rapporteringsuppdrag och utvärderingsuppdrag}}

Ett utvärderingsuppdrag som jag kommer arbeta med löpande under 2014/2015 är rapporten och utvärderingen av Konstnärliga forskarskolans handledarutbildning 2011-2015.

\subsection*{\textsf{Symposier, konferenser, workshops och samarbeten}}

\begin{enumerate}
\item Ansvar för en konferens i det skandinaviska
  ERASMUS-nätverket NordPuls i Malmö 2002.
\item Ansvar för de återkommande symposierna ForMuLär (Forum för musikaliskt lärande) som under flera år producerades av KU-nämnden och som jag höll i.
\end{enumerate}


\subsection*{\textsf{Utmärkelser och priser inom pedagogisk verksamhet}}

Inga utmärkelser eller priser.


%%% Local Variables: 
%%% mode: latex
%%% TeX-master: "pedagogisk-meritportfolj"
%%% End: 


\end{document}


