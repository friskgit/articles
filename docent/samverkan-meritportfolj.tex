\documentclass[a4paper]{article}
\usepackage[english]{babel}
\usepackage[T1]{fontenc}
\usepackage[authoryear,round]{natbib}
\usepackage{url}
\usepackage[utf8]{inputenc}
\usepackage{enumitem}
\renewcommand{\encodingdefault}{T1}
\usepackage{fancyhdr}
\pagestyle{fancy}
\renewcommand{\rmdefault}{pad}
\renewcommand{\sfdefault}{pfr}
%\lhead{\small{\textit{Henrik Frisk}}}
%\chead{}
%\rhead{\small{\textit{title}}}

\title{G: Samverkan, innovation och entreprenörskap}
%\author{Henrik Frisk}
\date{}

\begin{document}
\selectlanguage{english}
%\maketitle

\thispagestyle{empty}

\section*{\textsf{4G. Samverkan, innovation och entreprenörskap}}

\section*{\textsf{Sammanfattning}}

De viktigaset aspekterna av mitt samverkansarbete menar jag kan beskrivas utifrån följande punkter:

\begin{enumerate}
\item \textbf{Söker aktivt för att bredda fältet}. Konstnärlig forskning, även då det är fullt utbyggt, kommer alltid att vara ett litet och ganska smalt område. Därför är det avgörande att vi hittar allianser utanför vår egen disciplin med vilka vi kan samarbeta kring viktiga frågor och med vilka vi kan söka medel och förbättre förutsättningarna. I en tid då det fristående kulturlivet blir alltmer vingklippt måste vi se till att det finns tillräckligt med medel och möjligheter för produktioner.
\item \textbf{Söker flera gränsytor i vilka fältet kan reflekteras}. Det konstnärliga forskningsfältet behöver det fristående kulturlivet som gränsyta men det behöver också andra forskningsdiscipliner att reflektera sin kunskapsproduktion mot. Detta behöver inte vara regelrätta samarbeten utan kan vara möten, seminarier, föreläsningar; möjligheter till utvärdering och reflektion.
\item \textbf{Söker aktivt för att skapa förståelse för fältet}. Konstnärlig forskning är förhållandevis etablerad nu men det finns viktiga instanser som ännu inte har en tydlig bild av vad det är och hur det ser ut. Flera av de statliga myndigheterna har endast en rudimentär uppfattning om vad forskningsfältet består i och det skapar problem vid såväl finansiering som vid genomföranden av projekt. 
\end{enumerate}

\section*{\textsf{Självreflektion}}

Samverkan med det omgivande samhället är en helt central del av den konstnärliga forskningsprocessen. Konstnärlig forskning utan samverkan med det omgivande konstfältet är i närmast otänkbar eftersom detta måste vara med som en del av valideringen av forskningen. Således vill jag hänvisa till min samlade utgivning av CD samt min konsertverksamhet som en del av min samverkansgärning.

Mitt arbete i Kungliga Musikaliska Akademiens forskningsnämnd är ett exempel på ett försök att få olika forskningsområden i musik att närma sig varandra. Seminariet \emph{Att utveckla och kommunicera musikalisk kunskap} som jag på uppdrag av nämnden planerade 2012 och genomförde i januari 2013 var ett försök att bjuda in politiker, musikinstitutioner, musikutbildningar på alla nivåer, andra forskare samt allmänheten och själva fokus för seminariet, som titeln antyder, var att utveckla metoderna för hur vi kommunicerar den kunskap som konstnärlig forskning producerar.

Under min tid vid Kungliga Musikhögskolan i Stockholm (KMH) har jag suttit med i KU-samverkansgruppen; en grupp som från början hade till uppgift att uppdatera de ansvariga för konstnärligt utvecklingsarbete om arbetet på de sju konstnärliga högskolorna i Stockholm med varandra kring frågor som rörde KU. Gruppen kom dock att mer och mer intressera sig för konstnärlig forskning istället och det senaste året har vi främst arbetat med handledning och gemensamt handledarkollegium. I denna grupp har jag sett som min uppgift att verka för ett tätare samarbete mellan de olika skolorna vilket jag menar också har lyckats. Det har varit en värdefull erfarenhet, inte minst med tanke på att vi har varit en från högskoleledningarna relativt oberoende grupp.

Vid KMH har jag också arbetat för ett närmande mellan konstnärlig och musikpedagogisk forskning. Bland annat tog jag tillsammans med professor Cecilia Hultberg initiativ till en forskningsvecka där vi även bjöd in de konstnärliga och musikpedagogiska doktoranderna från Malmö. Jag har också arbetat för ett tätare samarbete mellan KMH och KTH beträffande kurser, handledning och forskning. I Malmö har jag sedan ett drygt år diskuterat möjligheten att samarbeta med forskarutbildningen på Malmö Högskola, K3.

Inom Lunds Universitet känns det nu som vi kommer kunna få igång ett stabilt tvärvetenskapligt samarbete efter att ha försökt i många år. IAC, i sig en samverkansresurs, menar jag kommer kunna spela en viktig roll i denna utveckling vilket även Pufendorfinstitutet kan göra. Ett av projekten som jag hoppas kunna dra igång redan till hösten 2014 är att starta en study group vid Pufendorf med syftet att skapa en förståelse för de olika språk de olika disciplinerna använder. Målet är att komma upp med en taxonomi, inte för att skapa allmängiltighet utan snarare för att skapa relationer mellan etablerade språk.

Sedan 1994 har jag publicerat texter i dagstidningar och radion (SR P2, Sydsvenska Dagbladet, Svenska Dagbladet och senast Aftonbladet) kring ämnen som rör musik och utveckling. Efter en period då det varit svårt att publicera texter ser det nu ut som det är möjligt att föra dialogen med media. Jag ser detta som en mycket viktig uppgift. Jag skriver även i två egna bloggar, en på svenska som rör politiska och konstnärliga diskussioner av ganska bred giltighet och en på engelska som är mer fokuserad på konstnärlig forskning och dess metoder.

Slutligen driver jag tillsammans med några musikerkolleger kulturföreningen Neo i Uppsala. Syftet med föreningen är att hitta nya platser och ny former för konserter med nutida musik. Jag är även med i musikerkollektivet, skivbolaget och produktionsbolaget Kopasetic Productions där jag gett ut de flesta av mina CD de senaste åren.\footnote{Se \url{http://www.kopasetic.se/}}

\newpage

\section*{\textsf{Samverkan, innovation och entreprenörskap\\Meritförteckning}}


\subsection*{\textsf{Formell utbildning inom media och kommunikation}}

Ingen formell utbildning men lång erfarenhet av att skriva till dagstidningar och radio (SR P2, Sydsvenska Dagbladet, Svenska Dagbladet och Aftonbladet).

\subsection*{\textsf{Information till näringsliv/kulturliv/föreningsliv/industri/offentlig sektor}}

Såsom beskrivits ovan är detta en helt avgörande del av att vara konstnärlig forskare. Den konstnärliga praktiken står i centrum för forskningsinsatsen och den praktiken måste formas med, mot eller genom det omgivande samhället. Det musikaliska omgivande samhället för den genre jag representerar är organiserat i föreningsformer och offentlig sektor. Jag arbetar även med en egen kulturförening som producerar konserter i Uppsala, omkring fem till sex om året. Dessa konserter arbetar vi hårt med att hitta en ny publik och vi arrangerar därför konserter på biblioteket och i konstmuseet i Uppsala.

\subsection*{\textsf{Rådgivning till näringsliv/kulturliv/föreningsliv/industri/offentlig sektor}}

Jag granskade ansökningar för Stiftelsen Framtidens Kultur i flera år.

\subsection*{\textsf{Utvecklande av informations- och utbildningsmaterial till allmänheten, andra yrkesgrupper etc.}}

Här kan igen nämnas PR-materialet som vi tar fram för Kulturföreningen Neo och som framgångsrikt lockat ny publik till konserterna.

\subsection*{\textsf{Medverkan i olika medier}}

Som tidigare nämnt har jag skrivit i flera svenska tidningar och till SR P2 Nyläst. Jag har även medverkat i TV (TV4, TV i Mexico, i Vietnam och i Kina).

\subsection*{\textsf{Exempel som visar innovation inom exempelvis utbildning, forskning eller annat område}}

Den implementation av en dokumentationsplattform som jag för närvarande arbetat med är ett exempel på en innovation som möjligtvis även kan få resonans även utanför det konstnärliga fältet.

\subsection*{\textsf{Exempel som visar entreprenörskap}}

Inget direkt entreprenörskap även om jag vill framhålla att utan en känsla för entreprenörskap är det svårt at klara sig som musik

\subsection*{\textsf{Lista över patent}}

Inga patent


%%% Local Variables: 
%%% mode: latex
%%% TeX-master: "samverkan-meritprotfolj"
%%% End: 



% \bibliographystyle{apalike}
% \bibliography{/run/media/henrikfr/Homer/Home/Documents/svn/admin/conf/biblio/bibliography}
\end{document}