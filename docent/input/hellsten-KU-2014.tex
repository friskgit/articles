\documentclass[mhm,color]{luletter}
\usepackage[utf8]{inputenc}

\begin{document}

\documentname{Intyg}
\diarynumber{}
\renewcommand{\today}{28 juli 2014}
\name{Professor Hans Hellsten}
\signature{Hans Hellsten}
\telephone{}
%\phone{2 49 84}
\email{}
\homepage{}

\begin{letter}{}

\opening{}
\small
\sloppy
Jag intygar härmed att Henrik Frisk har varit vice ordförande i Nämnden för konstnärligt utvecklingsarbete (KU-nämnden) sedan december 2005. Med undantag för ett års frånvaro 2005-2006 på grund av föräldraledighet har jag själv varit ordförande i nämnden sedan hösten 2004 och kan därför väl redogöra för Henrik Frisks deltagande i nämndarbetet.

Nämnden för konstnärligt utvecklingsarbete är liten, har sex ledamöter och inga suppleanter, arbetar både operativt och strategiskt, har kommunikationskrävande arbetsuppgifter, så ledamotskap tar mycket tid. Henrik Frisk har ägnat nämnden mycket uppmärksamhet, har varit en mycket aktiv vice ordförande och permanent deltagit i beredningen av både löpande, årligen återkommande och mera strategiska beslut. 

Bland arbetsuppgifterna kan nämnas beredningen och bedömningen av projektansökningar, fördelningen av resebidrag, arrangerandet av seminarier, samt viss handledning av projekt. Nämnden har också själv initierat ett antal projekt varav särskilt två skall nämnas: en databas för genomförda KU-arbeten och seminarie- och sedan publikationsserien ForMuLär. Henrik Frisk har varit en initierande, reflekterande och pådrivande kraft i genomförandet av alla dessa arbetsuppgifter.

Jag kan också nämna att Nämnden för konstnärligt utvecklingsarbete också fungerat som ett informellt nätverk i utvecklingsfrågor inom dess intresseområde. Ett exempel är den energi som dess medlemmar – och inte minst Henrik Frisk – ägnat utformningen av musiker- och kyrkomusikerutbildningarnas examensarbete.


\closing{Malmö den 28 juli 2014}

\end{letter}
\end{document}
