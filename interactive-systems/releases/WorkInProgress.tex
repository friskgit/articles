
\newif\ifpdf
\ifx\pdfoutput\undefined
\pdffalse % we are not running PDFLaTeX
\else
\pdfoutput=1 % we are running PDFLaTeX
\pdftrue
\fi

\documentclass[a4paper]{article}
\usepackage[swedish, english]{babel}
\usepackage[T1]{fontenc}

\ifpdf
\usepackage[pdftex]{graphicx}
\renewcommand{\encodingdefault}{T1}
\renewcommand{\rmdefault}{pad}
\pdfcompresslevel=9
\else
\usepackage{graphicx}
\fi

\usepackage{fancyhdr}
\pagestyle{fancy}
\usepackage{url}
%% Define a new 'leo' style for the package that will use a smaller font.
\makeatletter
\def\url@leostyle{%
  \@ifundefined{selectfont}{\def\UrlFont{\sf}}{\def\UrlFont{\small\ttfamily}}}
\makeatother
%% Now actually use the newly defined style.
\urlstyle{leo}

\lhead{\small{\textit{Henrik Frisk}}}
\chead{}
\rhead{\small{\textit{Interactive systems}}}

\title{Interactive systems in improvisation and composition. A work in progress}
\author{Henrik Frisk\\{\small PhD candidate}\\{\small Malm� Academy of Music - Lund University}\\{\small henrik.frisk@mhm.lu.se}}
\date{\today}

\begin{document}
\selectlanguage{english}
\maketitle
\thispagestyle{empty}

\begin{abstract}
In this article I will present my artistic research project entitled
 `Interactive systems in improvisation and composition' and discuss
 certain aspects of the project's development during the course of my
 work on it. I will briefly explain the theory that forms the framework
 for the research and discuss three selected projects that are all a
 part of my PhD studies. I will also touch upon the widely discussed
 issue of choosing a method for artistic research. The conclusion drawn
 from my experience is that an artistic research project is similar to
 the very nature of artistic practice; one needs to follow the paths
 opened up by the work. Finally, I comment on the potential validity of
 artistic research as a discipline.
\end{abstract}

\section{Introduction - the project}

The core of this project is to establish a set of artistic means through
which electronic instruments and timbres (created through computer
software) can be successfully integrated with acoustic instruments in
both improvisational and compositional contexts. The success of this
project may be measured in terms of how well I manage to describe and,
through the development of a series of works, implement models for
artistic creation (compositional and improvisational) where acoustic and
electronic instruments interact.

%In the broadest sense the ambition with my project was to leave the
%representation of the music (the notation) in favor of the object of
%representation (sound) - to return to the sound itself. Specifically
%this should be understood in the context of contemporary Western art
%music and the attempt is to improve the premises for a successful
%interaction between acoustic instruments and machines (mainly
%computers). 

At the outset the thesis was that if a seamless integration of digital
and analog instruments is the object, i.e. both on the level of sound
and performance, and we do not wish to compromise the 'human feel' of
the performance\footnote[1]{Remember that there are types of music in
which this is not desirable - I am sure that for Kraftwerk it would have
made no sense to attempt at making the instruments sound more 'human'.},
then we need to make the digital instruments approach the analog
instruments, both on the level of sound and performance. Hence, the
research question may in essence be articulated as: how does one
accomplish this seamless integration? Since this question is too broad
to handle in a PhD project it has been limited to a test of the
following case: {\textit {Can an interactive system that uses sound as
its object of interaction provide the necessary premises for an
integration of digital and analog sound sources on the level of both
sound (as it is perceived) and performance (as it is experienced)?}}

%I will
%further elaborate the meaning of 'succesful integration' in the text
%below,

%Can an interactive system that uses the sound as its
%object of interaction provide the necessary premises for a succesful
%integration of digital and analog sound sources on the level of both
%(perceived) sound and performance (as it is experienced)?

I will return to this research question in an attempt to untangle its
 meanings, contextualize the field in which this research is
 performed and discuss the environments in which the results of the
 research may have significance. But before that, I will present the
 tools with which I will make these descriptions and the method with
 which I will present the work in progress; i.e. the meta-method of this
 article.

\section{Method}

For this article I will make use of some of the methodological ideas for
artistic research introduced by Mika Hannula in his paper, ``The
Responsibility and Freedom of Interpretation'' \cite{hannula02}. In this
paper, Hannula uses a hermeneutic approach in order to arrive at ``a preliminary, and
[...] a merely tentative notion, of the distinctive character and
minimum requirements of artistic research'' (\textit{ibid} pp. 73).  It
should be noted that Hannula himself is mainly active in the field of
visual arts and his proposed method, although it takes a general approach,
is probably mainly geared towards that area. Though the two disciplines,
music and the field of visual arts, have many overlapping areas of interest, the
field of visual arts is in many respects very different from that of
contemporary Western art music.

\begin{quote}
 The researcher must explain what he or she is researching, why he or
 she is doing the research, why it is of interest, and what is the
 aim. The success or distortion of artistic research is largely
 dependent on how carefully and meticulously this first step is planned
 and then, of course, implemented. At this stage, the researcher should
 explain why the research is undertaken in the sphere of art and within
 the purview of contemporary art, and not in art history, for example,
 or sociology. (\textit{ibid} pp. 82)
\end{quote}

These ideas were further developed in a presentation by Hannula at a
seminar in Gothenburg, Sweden, in 2004\footnote[2]{The seminar was held
as a part of the series ``Fria seminarierna'', at Gothenburg University,
October 20, 2004. The title of Hannula's presentation was ``What are we
talking about when we we talk about artistic research?''}.  At this
seminar he pointed out a total of ten aspects of artistic research where
the first three aspects are:

\renewcommand{\labelenumi}{\Alph{enumi}}
\begin{enumerate}
\item{Before. (Where am I coming from?)}
\item{Now. (Where am I at?)}
\item{After. (Where am I going?)}
\end{enumerate}

In my interpretation of Hannula's method I look at the first item
in the list (Before) as an inquiry into the aesthetics of the artistic
work of the researcher and (which I believe is related) perhaps even
into trivial details of the personal and artistic background of the
researcher. The second aspect (Now) is described by Hannula as the
attempt at ``contextualising oneself'' and it presupposes the
'before'. The 'now' is obviously something that needs to be constantly
re-phrased or re-thought. In my project I have been trying to map these
`nows' in my online research diary \cite{friskDiary}. The third aspect
(After) presupposes that the first and the second aspects are absolutely
transparent, according to Hannula. As I am currently three quarters into
the project, it is difficult for me, at this point, to envision any
'after'. I am currently very focused on the 'now,' which is why this third
aspect will only be briefly touched upon here.

%%% !!!!!!!!!!!!!!!!!!!!!
%%% Needs more work!

The aesthetics or the aesthetic theory that leads one to, or forms the
context for, the research project is both interesting and, I believe,
fairly significant in the field of artistic research. It is not
necessarily an explicit aesthetic but rather the contextual background
of the artist that creates a platform on which the investigation can
take place. For my particular project it is probably difficult to even
begin to understand its relevance until it is seen in the context of my
own artistic work.

%Perhaps the open disclosure of the
%underlying aesthetics or contextual background in artistic practice is
%related to the complex issue of ethics in scientific research.

%An inter-field discussion could
%be instantiated in which the question of whether artistic research
%should generate results that are valid regardless of aesthetics would be
%debated. 

%What makes Hannula's method appealing to me is that the knowledge
%produced in the course of the research project can be measured in the
%difference between the item A and B and between B and C in the
%list. Would this be an entirely linear process - which it unlikely is -
%we could formulate the equation:

%\begin{displaymath}
%knowledge = C-A
%\end{displaymath}

\section{Before}

%Whether my background as a jazz musician and improvisor is the
%explanation for, or a consequence of my interest in self organisation
%and open forms is perhaps not the key issue. But this interest is the
%most important aspect influencing the choice of research area in this
%project. Specifically my attempts to combine non-static electronic music
%elements made it obvious to me that the available tools and existing
%solutions to the 

For more than ten years, there have been two significant areas of
interest in my musical practice:
\renewcommand{\labelenumi}{\arabic{enumi}}
\begin{enumerate}
 \item As an improvisor, I am interested in the dynamic relationship
       between stimuli and responses. Even when working with composition
       in a relatively traditional manner (i.e., using musical
       notation), it is the non-static, or that which changes or
       evolves, that is at the very core of my involvement.
 \item I have worked with the computer in almost all of my artistic
       work. As an artist I feel a responsibility to explore the world
       and its artifacts. The computer has become an important part of
       this world and hence, a part of our culture. In the Western world
       it is part of daily life and allows for our most basic as well as
       our most intimate communications. It cannot be placed outside of
       our culture, nor can it be regarded as merely a tool or a
       fashionable gadget with a limited import; rather it must be
       included in our understanding of the world as well as in our
       artistic explorations.
\end{enumerate}

The basis for my PhD project can be traced back to my difficulties in
successfully integrating two different types of music -
\emph{improvised} or \emph{open form} music and \emph{computer-music}.
For the sake of argument one might make some rough generalizations
and move to a more abstract level, calling this a dichotomy between the
\emph{continuous} or \emph{analog} on the one hand and the \emph{discrete}
or \emph{digital} on the other. The problems I experienced in my
practice occurred in the contexts of improvisation as well as
composition and resulted in growing artistic frustration. It was my
attempt at addressing this frustration that led me to form the current
research project.

The problem for me was that I could not achieve the merge between the
analog - i.e. the acoustic intruments - and the digital - the electronic
instruments. My aim was, and still is, to be able to alter, distort or
expand the traditional notion and pre-conception of the sound of the
musical instrument; i.e., to introduce a discontinuity between the
expected sound and the perceived sound. To achieve this result I use
digital sound processing of the acoustic instrument(s), sometimes in
combination with pre-recorded and synthesized sound sources. If the two
sound sources do not merge successfully, then the desired effect will
fail to appear and the perceptual result will be that of two discrete sound
sources. This undertaking, along with the wish to explore the computer
in the sphere of contemporary culture is in principle an attempt to move
past ``the 20th century's ambivalent relationship to the technology of
machines'' \cite{Garnett}. Having said that we could further generalize
the dichotomy above and talk about the \emph{human}/\emph{machine}
relationship; I believe the tension between man and machine is at the
very heart of my project.

To use Garnett's words, this is not only an ambivalent relationship, it
is also a complex one. He continues: 

\begin{quote}
The machine did not make the life of the factory worker better, at least
not at first. Rather, the worker had to learn to adapt to the pace and
consistency of the machine, with sometimes rather unpleasant
effects. Part of my contention here is that this view of technology is
now no longer relevant. Technology is beginning to empower
individuals. (\textit{ibid})
\end{quote}

Garnett brings up a practical aspect of the relationship between man and
machine, dating from the early ages of industrialism. I agree that there
is a need for an updated view of technology,\footnote[3]{The term
technology is used here to denote the hardware, the machines themselves,
and not how they are used - it should not be confused with the
Aristotelian notion of \emph{techne}} but even if it is true that
technology is empowering the individual, what is the nature of our
relationship with technology? Are we comfortable with the tools we have
been given? Skepticism towards the machine can be found in many
sources. In Derrida's reading of Freud in ``Freud and the scene of
writing'' \cite{derrida78-2} \nocite{der78} he discusses the Mystic
Writing-Pad, a construction that ``shows a remarkable agreement with my
hypothetical structure of our perceptual apparatus'':\footnote[4]{Freud
quoted in \cite[pp. 280]{derrida78-2}.}

\begin{quote}
That the machine does not run by itself means something else: a
 mechanism without its own energy. The machine is dead. It is death. Not
 that we risk death in playing with machines, but because the origin of
 machines is the relation to death. (\textit{ibid.} pp. 285)
\end{quote}

The machine according to Derrida is dangerous because it is the opposite
of life; it is ``pure representation'' and ``never runs by itself''
(\textit{ibid.} pp. 286). So far it is difficult to argue with
Derrida. But is there no way around this? Is the machine destined to be
``complexity without depth''? (\textit{ibid.}) Danish philosopher Peter
Kemp argues that Derrida's interpretation of the machine is problematic
because it posits a 'technique' (now in the sense of \emph{techne}): a
'technique' that is only problematic in the way we make use of
it. According to Kemp, Derrida argues as if the ``technological development of
our culture did not confront us with an existential inquiry of planning
and organisation'' \cite[pp. 139]{kemp81}. I find Kemp's idea that we
are in fact in control, or at least that we may gain control, of the
multiple, co-existing technological systems, to be a very useful one. We can
use them and inform them. The machine cannot run by itself but it can
run alongside us. It is not merely a tool and part of our 'technique',
but may also become a part of our culture.
%Not merely as tools, as part of our 'technique', but also as part of our culture. 
More specifically, if we move to the field
of music, our creative intentions and our wish to \emph{play} (in every sense
of the word) need to be communicated to the machine in order to
integrate it into the larger sphere of (musical) communication. This
does not make the machine autonomous in any way - the machine still does
not run by itself - it runs by means of parallel representation. This is
somewhat closer to Deleuze and Guattari's notion of the war machine
that, according to them, can potentially be ``an `ideological,'
scientific, or artistic movement'' \cite[pp. 466]{deleuze80}. But even
Deleuze and Guattari choose to call it a ``war machine'': a machine with
a potential power to kill, hence closely related to death.

What then are the issues that have to be overcome in order to move from
the idea of the machine as death to the machine as a potential
instrument to be included in artistic practice? Derrida's main point is
that the machine has no energy and by definition cannot express
anything, but can only represent. Could this be related to the reason
for my inability to merge the two elements in practice? Let us for
the time being move away from abstract philosophy and turn to the
natural sciences.
\begin{quote}
Digital computers are superb number crunchers. Ask them to predict a
rocket's trajectory or calculate the financial figures for a large
multinational corporation, and they can churn out the answers in
seconds. But seemingly simple actions that people routinely perform,
such as recognizing a face or reading handwriting, have been devilishy
tricky to program.\\ 
\cite{turing1}
\end{quote}
This is in essence the reason Derrida calls the machine 'dead'. It can
do things that are incomprehensible for a human being but yet it cannot
solve routine tasks that we perform on a daily basis. But Alan Turing,
who was the first to conceive of the abstract machine that we now refer
to as the digital computer, was already beginning to think about
connectionist networks in 1947: ``Perhaps the networks of neurons that
make up the brain have a natural facility for such tasks that standard
computers lack. Scientists have thus been investigating computers
modeled more closely on the human brain.'' (\textit{ibid}) And over the
last few decades, not only the way we think about computers, but the way
computer science thinks about programming, is moving away from an extreme
formalism with a focus on the program, towards an increasing attention
to the \emph{programmer};
\begin{quote}
[...] from the logical and computational structure of algorithms to the
 cognitive structures of the people who produce them [the
 programs]. Innovations such as interactive programming environments,
 object-oriented programming, and visual programming have not been
 driven by considerations of algorithm efficiency or formal program
 verification, but by the ongoing drive to increase the programmer's
 effectiveness in understanding, generating, and modifying
 code. \cite{winograd95}
\end{quote}

The interaction between machine and programmer and, eventually, betweeen
machine and end user, is the key issue, if the computer is to be fully
incorporated in our practice - in our culture and in our artistic
expressions. And this interaction must take place on multiple levels. It
cannot be simply the tapping of fingers on a keyboard, but also has to
move to yet unresolved spheres of action. We need to find ways to
navigate the 'de-territorialized,' smooth, nomadic space of the war
machine as Deleuze and Guattari refers to it, ``a tactile space, or
rather `haptic,' a sonorous much more than a visual space.''
\cite[pp.421]{deleuze80} They discuss this further in the chapter
following ``Treatise on Nomadology'':

\begin{quote}
[...] the reinvention of a machine in which the human beings are
 constituent parts, instead of subjective workers or users. If
 motorized machines constituted the second age of the technical machine,
 cybernetic and informational machines form a third age that
 reconstructs a generalized regime of subjection: [...] the relation
 between human and machine is based on internal, mutual communication,
 and no longer on usage or action. \cite[pp. 505-6]{deleuze80}
\end{quote}

The need is not only to move focus away from the program and towards the
programmer but further, towards the \emph{meaning} of the program in
relation to the intended context for the program, including the user,
and any possible output from it. In the context of the construction of
music, the programming of a machine is an artistic endeavour comparable
to the preparation of a score for a performer or a group of
performers. It is a prescriptive notation meant for the machine as
expert interpreter and performer. For this to even begin to be possible,
the fear of the machine as a representative of power and/or destruction
must be abandoned. Furthermore, great care must be taken to the way in
which the flow of information within the man/machine system is
understood, since we are no longer talking about a simple
sender/receiver information theory system. We are looking at a
potentially very complex and continuous system in which we need to allow
for concepts such as `interpretation,' `multiple meaning,' `sign' and
`signifier,' `cultural convention,' etc.

% - perhaps surprisingly so since one
%would expect musicians and composers that engage in musical work with
%computers to have a relaxed attitude towards techonolgy in
%general.
%
%In Jacques Derrida's discussions 
%
%Danish philosopher Peter Kemp brings up Captain Nemo of Jules
%Verne's book XXX and the way that he has abandoned man in favour of the
%perfect machine - his technologically marvelous submarine driven by
%the invisible force. 


%In a wider context this is a very active research field that roughly can
%be divided in two main categories:
%
%
%\begin{enumerate}
% \item Controllers. Development of the ways in which a computer based
%       instrument can be played.
% \item Interpretation. Development of the ways a score can be performed
%       by a computer.
%\end{enumerate}
%
%For a review of the aesthetics of interactive music see \cite{Garnett}
%and for an introduction to the field of interactive music in the
%contexts of performance, composition and improvisation see \cite{rowe}.
%
%In my preliminary work it was my understanding that it was the inherent
%nature of these two entities that made them difficult to combine. 
%
%I will attempt to further delimit the field and explain the object of
%research by providing a few examples from my work prior to the start of
%the PhD project. Basically we can look at the problem as 

\section{Now}

If I had a tendency, at the beginning of this PhD study to think of
computer/performer interaction as merely a technical problem, the
research involved in this study has made me reconsider the question
behind the project itself. Through the projects that I have completed,
as well as the ones that I am currently working on, I have focused on
investigating the experience of interaction between the different agents
involved in the production of musical content. In the case of
\emph{etherSound}, it was the interaction between the user (and/or
listener) and the sound; in the project \emph{Negotiating the Musical
Work}, it is between the performer and the composer; and in the
composition `Repetition Repeats all other Repetitions' (for which the
collaborative project \emph{Negotiating the Musical Work} is a
pre-study), it is between the performer, the score, the computer and the
sounds produced by means of the computer. I now believe that it is only
through a thorough understanding of the very nature of interaction in
the context of musical production, that an artistically relevant and
perceivable mapping mechanism between input and output in an interactive
system can be achieved. The updated research question now reads:
{\textit {Can an interactive system that uses the sound as its object of
interaction provide the necessary premises for an integration of digital
and analog sound sources on the level of both sound (as it is perceived)
and performance (as it is experienced), and, furthermore, how can
significant features of human/human and human/sound interaction in the
context of musical production inform such a system?}}

%what is the
%significance of human/human interaction in this context to human/machine
%interaction?'.

\subsection{Main Projects}

At first glance, the following selection of artistic works and projects,
belonging to the larger investigation of sound and interactivity, may
seem disparate or heterogeneous. However, if the focus is placed not so
much on the content of each of the projects, but rather on the larger
frame within which they exist, i.e., on the premises and outsets, then
the projects form a more homogeneous collection of studies. The
intention is to create a platform for the investigation of the limits
for interaction and sound, and the result is intended to be an artifact
of artistic output. Apart from the three projects presented below, a
number of improvisations and smaller scale works have been, and will be,
produced within my PhD project.

\subsubsection{etherSound - an interactive sound installation}

\emph{etherSound} was commissioned by curator Miya Yoshida for her
project \emph{The Invisible Landscapes} and was premiered in August 2003
at Malm� Art Museum in the city of Malm�, Sweden. The curatorial concept
for The Invisible Landscapes project was the use of cellular phones as a
means of experiencing and creating artistic expressions. The principle
idea behind \emph{etherSound} came to be an attempt at developing an
instrument that can be played by anyone who knows how to send an SMS
(Short Messages Service) from a cellular phone. In the version displayed
at \emph{The Invisible Landscapes}, all messages sent to a specified
phone number were received by an Internet server, parsed for its content
as well as the phone number it was sent from and the date and time it
was received. This information was written into a database which was
queried at regular intervals by a computer running a control as well as
a text analysis application (written in Java \cite{j2se, j2ee}), with
sound-synthesis software (Max/MSP \cite{max} running a Csound orchestra
\cite{csound}). For every new message, the data was downloaded,
processed and analyzed by the control program, and turned into control
signals, which were then sent to the sound-synthesis engine. Every
message generated one sonic object that would last for up to two
minutes. The response was very direct with a clear causality between the
input and the output of the system - a received SMS would result in an
immediate and perceivable change in the sound (see
\url{http://www.henrikfrisk.com/index.jsp?metaId=music&id=music&about=1&field=name&query=etherSound}
for some audio examples).

There are two states in which \emph{etherSound} may operate. One is as
a stand-alone, interactive sound installation and the other as a
vehicle for improvisation. In the latter, one or several performers
improvise along with the sounds of the installation while the audience
contribute actively to the performance by sending text
messages. \emph{etherSound} is an investigation of some of the aspects
of interaction between the listener, the sounds created and the
performing musicians, and also of the formal and temporal distribution of
the music that this interaction results in \cite{frisk05, yoshida06,
frisk1}.

\emph{etherSound} also has significance in relation to the discussion of
the man/machine relationship, described above. Perhaps the mobile phone
is one of the machines of recent years that humans have most
successfully and naturally integrated in their lives. Furthermore,
although the traditional roles of the agents involved in the production
of the music in \emph{etherSound} were shifted or distorted, there is no
doubt that the programming, i.e., the actual code that constitutes the
synthesis and the control program, is the score - if a score exists at
all. In any event, these programs contain the only apparent work identifying
instructions, to use the language of Stephen Davies \cite{davies}. When
technology is put in the center in this way, the \emph{technique} (in
the broad sense of the word) that is required of the user/listener is of
a different kind. Traditionally there is an intimate connection between
social class, level of education and cultural interests \cite{dimaggio,
bourdieu} which affects cultural consumption. Despite the fact that the
connection between social class and mobile phones is likely to be of a
different nature than that between social class and arts consumption,
interactivity and collaborative art in themselves may help to counteract the
exclusiveness of contemporary art and music. Perhaps it can contribute to
creating conditions for classless and unprejudiced participation in the
arts without compromising the content and the expression. Roy Ascott, in
addressing the issue of `content' in art involving computers and
telecommunications writes:
\begin{quote}
In telematic art, meaning  is  not  something  created  by  the  artist,
distributed through the network, and \emph{received} by the  observer. Meaning
is the product of interaction between the observer and the system, the
content of  which is in a state of flux, of endless change and transformation \cite{Ascott}.
\end{quote}
Following this line of thought, it may be concluded that the need for a
thorough insight into the history of art or electronic music is no
longer a prerequisite for understanding a collaborative, interactive
work of art. This limits the advantages of the educated listener and makes room
for new interpretations of the term `understanding' in the arts.

\subsubsection{Negotiating the musical work.}

I am undertaking this project, which is not yet completed, in
collaboration with guitarist Stefan \"{O}stersj\"{o}. It consists of three
distinct parts:
\begin{enumerate}
 \item Empirical analysis of composer-performer interaction.
 \item Application of the resulting data from the empirical analysis in
       the composition of a new work for guitar and computer for Stefan
       \"{O}stersj\"{o} (`Repetition Repeats all other Repetitions').
 \item Assessment of the research and comparison of the analysis and the
       different versions (performances) of `Repetition Repeats all
       other Repetitions'.
\end{enumerate}

Primarily, we discuss the musical work prior to its ultimate notation
and prior to its performance; we discuss the musical work within the
context of Western 'art music' tradition, in which musical notation has
an ontologically crucial function. The study deals exclusively with
music for solo instrument and live electronics. Our purpose is to
acquire a deeper understanding of the underlying processes in the
communication between the composer and the performer as well as their
respective roles. Through an improved understanding of the musical interaction
between the two parties involved in the creation of the work, we also
hope to better understand the necessary conditions for a successful
interaction between the performer and the electronics. We have developed a
hybrid method of investigation, which involves musical semiology,
qualitative method involving hermeneutics, and verbatim transcriptions
of the video documentation.

Musical semiology has been constructed with the intention of providing
tools for analytical understanding of the musical work in its entirety -
not only in terms of analyzing formal structures or details in the
construction of the work, but also examining its socio-cultural
context. Attempting to move to a more basic level of organization than
that of musical notation may help to further clarify the issue in
relation to a wider sphere of knowledge. For the analysis of the
composer/performer interaction we used the tripartite model suggested by
Nattiez and Molino:
\begin{quotation}
...recognizing, elaborating, and articulating the three relatively
autonomous levels (poietic, neutral and esthesic) facilitates knowledge
of all processes unleashed by the musical work, from the moment of the
work's conception, passing through its 'writing down', to its
performance. \cite{nattiez}
\end{quotation}
In short, according to Nattiez the \emph{poietic} phase is the complex
series of activities that are part of the construction of a musical
work, the \emph{esthesic} phase is the reconstruction of the message and
the \emph{neutral} level is the trace left by the poietic (or the esthesic)
processes.

The conclusions we draw from the first stage of the study, which has
many implications for both the second and the third phase, is that both
the creative and interpretative activities oscillate between poietic and
esthesic processes. Taken in this context, in \emph{etherSound} for
example, one may dispute whether one can talk about \emph{a single work}
as an ontological unit or even \emph{a single originator} in the role of
'composer.' We also found striking examples of creative
misunderstandings between the agents involved in the collaboration which
led us to the perhaps somewhat exaggerated conclusion that, in
communication noise is not a problem. As has already been mentioned, we
are used to thinking of a computer-based interactive system as a
cybernetic system in which information is transmitted from point A to
point B, and where great care is taken to avoid noise in the
transmission. In our joint project we will attempt to avoid the kind of
binary oppositions that require a clean control-signal path (such as the
pressing of a pedal) in the design of the interactive system. Obviously
this will also affect the way the instrumental part is written.

%
%In this context 'the work' is
%the conceptual vision, in whichever way it comes to the
%composer. However, as the following analysis shows, it makes little
%sense to isolate one certain point in this process claiming that it
%alone should make up 'the work'. Rather, it seems fair to assume that
%the musical work is the product of a complex chain of poietic and
%esthesic processes working together in an infinite loop.
%
%
%A musical work is the product of
%a complex chain of poietic and esthesic processes, which starts out with
%the conceptual vision, in whichever way it comes to a
%composer. 
%
%It seems
%to us that it makes no sense to isolate one certain point in this
%process claiming that it alone should make up 'the work'.
%
%Referring to the discussion above, the transformation that these two
%bars of music goes through would imply that the composition is neither
%the acoustic trace left by the performance nor is it the score. It is
%the conceptual idea, the vision, of the work as it is envisaged by the
%composer. Everything else, including the notation of the music, are
%creative interpretations of this first and original notion of the work.


\subsubsection{timbreMap - an audio analysis software for tracing relative timbre changes.}
%In short it is a software
%development project that has as its goal to allow for direct interaction
%with the sound itself rather than with an abstract classification of
%sound. 

\emph{timbreMap} is a software development project that attempts to
allow for direct interaction with sound itself rather than with an
abstract classification of sound. This is the part of my doctoral
project that I was initially inclined to look at as the central goal. I
have already referred to my growing frustration with the way in which
the tools available to me for letting a computer interact with a
performer in real time did not satisfy my needs as a composer. One
example of a widely-used tool in electroacoustic music with live
instruments is something that is referred to as ``pitch-tracking''. What
this process attempts to achieve is the transformation of a (monophonic)
audio signal into discrete pitches. Aside from the fact that this is a
difficult task, the information gleaned by this system is only
useful if the pitch class representation is a meaningful and substantial
parameter in the intended totality of the musical output. In much of my
music it is not.\footnote[5]{Please see
\url{http://www.henrikfrisk.com/index.jsp?metaId=music&id=music&about=1
&field=name&query=Insanity} for an example of an improvised piece in
which pitch has no significance from a structural point of view.}
With \emph{timbreMap} I have attempted to construct a system that uses
self-organizing feature maps and chained neural-networks to track the
relative change of timbre in an audio stream and make this information
available for interaction. As was mentioned above, 'connectionism' or
neuron-like computing is in itself a move away from the binary
representation of numbers, approaching what may be called an attempt at
modeling continuous processes. It is a special purpose machine that is
closely linked to the artistic enterprise that created the need for
it. In that sense, although in a more abstract way than the programming
of \emph{etherSound}, its code is part of the notation of the possible
pieces it may give rise to.

In the time that has passed since I started my doctoral project I have
come to realize that, although \emph{timbreMap} is a big part of the
totality of my project, both in terms of time invested and its
significance to the whole, its most intriguing aspect may be the mapping
of the output of the system and the musical stimuli to which it gives
rise. The mapping must be related to large-scale empirical studies, such
as those mentioned in this article, but also has to be tested in the
specific case-studies. From \emph{etherSound} I learned that successful
mapping involves a certain amount of pedagogy - knowledge creates
anticipation and expectation. The studies performed within the project
\emph{Negotiating the Musical Work} opened up the idea of the `creative
misunderstanding' and a semiological analysis of the communication
within an interactive system.

\section{After}
\subsection{Artistic research and its methods}

Artistic research is a much-disputed activity. Does research performed
within the realms of artistic practice fulfill the fundamental
requirements of research in general, whether within the natural or human
sciences? That question is discussed in detail by Henk Borgdorff in this
issue and I will not in this article attempt to defend the \emph{raison
d'\^{e}tre} of my own research project. However, for the sake of
argument and perspective, I would like to begin a short discussion on
the methods used in my artistic research project by referring to
Freud's dream of a scientific psychology succinctly described by Peter
Kemp \cite[pp. 29]{kemp81}. Freud argued that \emph{no}
science can be or become science by building on clear and sharply
defined fundamental concepts, but that any science has to begin with a
description of phenomena that may then be grouped, ordered and put into
context. The fundamental concepts of the science are developed afterwards.
%- ``Grundbegriffe der Wissenschaft''. 
Freud's line of thought in ``Grundbegriffe der Wissenschaft'' is
summarized by Kemp as a four-step process in the development of a new
science (\textit{ibid} pp. 30):
\begin{enumerate}
 \item Description of material.
 \item Use of abstract ideas based on this material.
 \item Creation of fundamental concepts.
 \item Definition of the fundamental concepts.
\end{enumerate}
Not only is there a lack of general terminology in the field of
electroacoustic music or computer music,\footnote[6]{Recently the EMS
2006 conference was dedicated to the subject of language and terminology
in the field of elctroacoustic music. There is no consensus even for the name of the
genre itself \cite{emsweb}.} there is also a lack of methodology and
terminology in the field of artistic and practice-based research. In my
own project, I am at the second step in the list above; I am using
abstract ideas based on the material, and I have begun to consider the
creation of the fundamental concepts for my work.

Further, the primary method I am using is that of artistic practice. But
it is not the only method. In the projects presented above, the practice
as the method was a point of departure. In \emph{etherSound}, the
investigation of the interaction between the user/listener and the sound
in the production of musical content was carried out in the form of a
sound installation that functioned as a vehicle for public
participation. As it turned out, the development and the design of the
software for the interface required a fairly standard scientific method
rooted in information theory. On the other hand, in \emph{Negotiating
the Musical Work}, although the intended result (i.e., the composition
`Repetition Repeats all other Repetitions') can in one sense be said
to be the method, we soon realized the project needed a much firmer
methodological framework. As has already been mentioned we developed a
hybrid method for the analysis of the case studies. In \emph{timbreMap},
both \emph{etherSound} and \emph{Negotiating the Musical Work} are part
of the method, as well as a hybrid of semiology and information theory.

In all three cases the research method was developed as a result
of initiating the artistic process - the artistic work led the way to
the method(s) with which the problems, as they appeared in this process,
could be resolved. Furthermore, in all three cases the method or
methods chosen were well known within closely related disciplines
(musicology, sociology, computer science, etc.). I would argue that this
is a relevant methodology for artistic research: to let the needs that
arise within the artistic practice yield the method.

%{\textit{Can an interactive system
%that uses the sound as its object of interaction provide the necessary
%premises for a successful integration of digital and analog sound sources
%on the level of both sound and performance?}}

Going back to the research question: {\textit{Can an interactive system
that uses sound as its object of interaction provide the necessary
premises for an integration of digital and analog sound sources on the
level of both sound (as it is perceived) and performance (as it is
experienced), and, furthermore, how can significant features of
human/human and human/sound interaction in the context of musical
production inform such a system?}} In what ways will I be able to
answer this question and what will be the significance of this
answer? In this article I have allowed myself to move rather freely
between generalized philosophical reasoning and specific cases. The
context and the musical sphere I am working within is, however, that of
contemporary Western improvised and composed music and it is in relation
to this field that the results of my work will primarily be of
interest. The study of different forms of interaction between musician
and computer is an active research field in the computer music community
and there has been a growing interest in the sound itself over the last
few years. I believe that the great strength of artistic research in
general (and I hope that this will be true of my project as well) is
that the research is informed by the artistic work. For my project this
means that subjects traditionally belonging to the realm of natural
sciences, such as sound analysis, can now be informed by the
values of an artistic practice, which in turn may lead to potentially
very different results as compared with more traditionally-oriented
research. At the risk of sounding evasive, I believe that newly posed questions
will be just as meaningful a response to the research question as a
clearly defined 'answer.' 

%This is not to say, however, that
%the research will be meaningless outside the confines of this very
%specific case. If we take the above-mentioned improvisational model as an
%example, the way it is structured and the way that it sounds will 
%create the context within which the research will obtain its full
%meaning.

%\section{skisser}
%
%
%If we take a CD player as an example one single motion or action - the
%pressing of a button - may result in hours and hours of music without
%any need for extra action or interaction from the user. But, at the same
%time the output will not change in any meaningful way in relation to
%\emph{how} the button is pressed. There is simply to little information
%at the source in relation to the requested output of the system.
%
%An open form aesthtetics puts certain requirements on the technoligies
%used. Obviously the otherwise revolutionary method of composing
%electronic music as was done in the 50's and forward - to record sounds
%and fixate them on tape - is not a working strategy. Having said that we
%have also encountered the first issue that has to be dealt with - an
%issue which is also at the core of the investigation described here: For
%an artistic work can some parts of the 'techne' stand in opposition to
%the artistic intentions? In other words, if the intention is to create a
%musical work in which open endednes in form and content is an important
%aspect, what happens if the technologies used in the production and
%performance of this work does not harmonize with these intentions? This
%is a question that we will return to.
%
%
%
%One may be tempted to look at any system involving computers or
%communication technology as a cybernetic system in which information is
%passed from point A to point B. According to information theory any
%distortion of the message at the level of transaction is regarded as
%noise. If the received message is not identical to the message sent the
%transmission is noisy. The musicologist Jean Molino reminds us that this
%hypothesis; that there is a 'single, well-defined item of information to
%be transmitted, all the rest being simply noise' is 'dangerously
%inaccurate and misleading as soon as we move from the artificial
%communication of information to a concrete act of human communication as
%a total social fact.' \cite{molino} \nocite{molino2} I will argue that,
%when we experience a musical work involving electronic sounds and
%computers we are expecting an 'act of human communication as a total
%social fact'.
%
%To instead try to understand and design the interactive system based on
%a semiotic model is for me starting to stand out as a much more feasible
%way to trace the processes that influence the different parts of the
%system. In particular semiology as it is proposed by Eco and Barthes, in
%which the cultural context for the commusication becomes a defining
%factor is an interesting model in this context. If I take
%\emph{etherSound} as an example it is very clear that there is not a
%clear and simple connection between input and output, between the sent
%message and the received sound object. The transformation of the message
%can only be understood in relation to the (sub)cultural context defined
%by me as the composer in relation to the audience as a totality. This
%would also hold true for an analysis on a lower level of communication.
%
%Mika Hannula has suggested hermenutics as a suitable method for artistic
%research. With its concept of Vorverstehen it is without doubt a valid
%theoretical approach to artistic work. The key issue with the artistic
%research project in general is the problematic issue of objectivity. As
%the artist is performing research on his or her own artistic output, any
%output that deals with complex object-subject relations will be of
%benefit to the artistic researcher. Gadamer's thourough work in the
%field of interpretation is certainly an asset and a possible method to
%be used by the artist.
%
%Finally, experimental phenomenology is a possible method for artistic
%research and a path that I am pursuing in my search for a stable
%theoretical and methodological framework for my work.

\bibliography{bibliography} \bibliographystyle{apalike}
\end{document}
