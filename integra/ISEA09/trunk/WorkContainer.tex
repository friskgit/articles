\documentclass[a4paper, onecolumn]{article}

\usepackage[style=verbose-ibid]{biblatex}

\bibliography{bibliography}
\DeclareBibliographyCategory{nobib}
%\addtolength{\textwidth}{1.75in}
\addtolength{\topmargin}{-.875in}
\addtolength{\textheight}{0.75in}

\begin{document}

\title{The work as a container of experiences: exploring the implications of the Integra storage model}
\date{}
% \author{Jamie Bullock \and Henrik Frisk}
\maketitle

\section*{Abstract}
\label{sec:abstract}

In this paper we discuss how new standards for multimedia module abstraction and storage---developed by the authors under the auspices of the Integra project---can be used as a platform for collaborative art practices. We bring forth questions relating to cross-platform environment-independent storage of all aspects of artistic works including visual representations of data, visual representations of interactive possibilities, and ways in which time based data can be stored and represented. We present an object-oriented model for information relating to artworks and performances involving live electronic processing. In this model objects, as well as their inherent relations and connections, are abstracted from implementation-specific details. The resulting graphs of interrelations may consist of objects describing performances, artists, organizations, DSP modules, time-based data and other data relating to the artistic work.

The aims of the original Integra project\footnote{See http://www.integralive.org} were primarily focused on \emph{work preservation} and \emph{sustainability}.
%\footcites[See][]{frisk-bull07}[See also][]{frisk-bullock08}
Both of these concepts are rooted in the wish to archive and preserve works of art in one particular state, with the ambition of recreating them in a form that is as close as possible to how they were originally conceived. However, due to the generic nature of the representations in the different parts of the proposed model we find it possible to also use the framework for purposes more closely related to participation, change, and difference.  Representation of a work in the Integra model is less centered on \emph{notation} and \emph{description}, and more focused on \emph{relations} and \emph{differences}. It is a distributed though interconnected array of containers of information that, by its nature of representation does not discriminate between the kinds of works it represents, and as a result the advantage of notational forms versus improvisatory and singular forms versus collaborative forms is likely to decrease. We believe that \emph{libIntegra}, the software library that implements the model briefly presented above, may provide for a framework in which the ontology of the art work may be described, and, eventually, visualized in a meaningful way. In the case of music this will be particularly true for the kind of works that do not lend themselves well to standard musical notation such as improvisatory and collaborative works.


% \bibliography{bibliography.bib}
% \bibliographystyle{jurabib}

\end{document}
