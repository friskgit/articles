% $Header: /cvsroot/latex-beamer/latex-beamer/solutions/generic-talks/generic-ornate-15min-45min.en.tex,v 1.4 2004/10/07 20:53:08 tantau Exp $

\documentclass[xcolor=table]{beamer}

\mode<presentation>
{
  \usetheme{Integra}
  \useinnertheme{circles}
  \setbeamercovered{transparent}
}

\usepackage[english]{babel}
% or whatever

\usepackage[latin1]{inputenc}
% or whatever

\usepackage{times}
\usepackage[T1]{fontenc}
% Or whatever. Note that the encoding and the font should match. If T1
% does not look nice, try deleting the line with the fontenc.

\usepackage{multimedia}

\usepackage{graphicx}

\usepackage{colortbl}

\title[Temporal data representation] % (optional, use only with long paper titles)
{An object oriented model for the representation of temporal data in the Integra framework}

\author[Bullock \& Frisk] % (optional, use only with lots of
				 % authors)
{James Bullock\inst{1} \and Henrik Frisk\inst{2}}

\institute[Birmingham, Lund] % (optional, but mostly needed)
{
\inst{1}Music Technology Department at Birmingham Conservatoire\\
Birmingham City University
\and
\inst{2}Composition Department at Malm\"{o} Academy of Music\\
Lund University
}
% - Use the \inst command only if there are several affiliations.
% - Keep it simple, no one is interested in your street address.

\date[ICMC 2009] % (optional)
{ICMC 2009}

\subject{Talks}
% This is only inserted into the PDF information catalog. Can be left
% out. 

% If you have a file called "university-logo-filename.xxx", where xxx
% is a graphic format that can be processed by latex or pdflatex,
% resp., then you can add a logo as follows:

 \pgfdeclareimage[height=1cm]{university-logo}{../img/integra_logo_colour}
 \logo{\pgfuseimage{university-logo}}

\beamerdefaultoverlayspecification{<+->}

\begin{document}

\begin{frame}
  \titlepage
\end{frame}

% \begin{frame}
%   \frametitle{Outline}
%   \tableofcontents[pausesections]
% \end{frame}

\section{Introduction}

\begin{frame}
  \frametitle{What is Integra?}
  % - A title should summarize the slide in an understandable fashion
  %   for anyone how does not follow everything on the slide itself.
  \begin{itemize}
  \item ``A European Composition and Performance Environment for
    Sharing Live Music Technologies''
   \item An EC financed project led by Birmingham Conservatoire in the UK
   \item Attempts to address the problems of persistent storage,
     portability and standardized intercommunication between systems
     for electronic music.
  \end{itemize}
\end{frame}

\begin{frame}
  \frametitle{Objective}

  \begin{block}{For this research the objectives are:}
    \begin{itemize}
    \item<1-> To complement the (synchronous) Integra module with the
      ability to store and edit time-based data.
    \item<2-> \alert{Integra module?}
      \begin{itemize}
      \item<3-> An abstract definition (and possible implementation)
        of a DSP process, a documentation item or a relation between
        modules.
      \end{itemize}
    \item<4-> It should be possible to use the same time data
      regardless of module implementation.
    \item<5-> It should be possible to extend and alter existing work.
    \item<6-> The work here is a proposal, and work is currently at the development stage.
    \end{itemize}
  \end{block}

\end{frame}

\begin{frame}
  \frametitle{Related work (music)}
  \begin{block}{MIDI}
    The (still?) dominant mechanism for time based information.
  \end{block}
  \begin{block}{SDIF and GDIF}
    Spectral and gestural data. May be incorporated in IXD.
  \end{block}
  \begin{block}{MetriXML}
    CLAM's XML based score file format. Similar to IXD sequences.
  \end{block}
\end{frame}

\begin{frame}
  \frametitle{Related work (general)}
  \begin{block}{SMIL}
    A W3C endorsed multimedia format for synchronizing multimedia.
  \end{block}
  \begin{block}{RDF}
    A language for describing resources (on the web).
  \end{block}
  \begin{block}{OWL}
    Exhaustive (RDF related) language for descriptive ontologies.
  \end{block}
\end{frame}

\section{The Player Module}

\begin{frame}
  \frametitle{The Integra framework}
  \centering
    \includegraphics<1>[width=30em]{../img/framework_illustration-1}
    \includegraphics<2>[width=30em]{../img/framework_illustration-2}
    \includegraphics<3>[width=30em]{../img/framework_illustration-3}
    \includegraphics<4>[width=30em]{../img/framework_illustration-4}
  % \begin{itemize}

%   \item 
%     An Integra module encapsulates a specific piece of message or
%     signal processing functionality (e.g. a waveform generator or
%     digital filter).
%   \item A module 
%     \begin{itemize}\item 
%       must have an interface definition
%     \item may have an implementation
%     \item may be associated with instance data.
%     \end{itemize}
% %  \item Only the implementation is platform specific.
%   \item A module may inherit the interface from any other module.
%   \item A module definition can be thought of as an abstract class.
%   \end{itemize}
%   \begin{block}{ }
%     Only the \alert{implementation} is platform specific.
%   \end{block}
\end{frame}

\begin{frame}
  \frametitle{The player module}

  \begin{block}{Schedule events}
    \begin{itemize}
    \item<1-> continuously in Sequences
    \item<1-> statically as in state changes in Presets
    \end{itemize}
  \end{block}
  
  \begin{block}{Player features}
    \begin{itemize}
    \item<2-> looped and reverse-looped playback of sequenced data
    \item<3-> random access to sequence data
    \item<4-> non-linear sequences
      % \begin{itemize}
      % \item<5-> sequences that contain other sequences
      % \item<6-> sequences that control the playback of other sequences
      % \end{itemize}
    \item<5-> relative representation of time (non-absolute)
    \item<6-> non-track based
    \end{itemize}
  \end{block}
\end{frame}

\begin{frame}
  \frametitle{The Event interface}
  \begin{block}{Events}
    `events' scheduled by the player are instances of the Event interface. Its attributes are:
    \begin{itemize}
    \item<2-> address
    \item<3-> value
    \item<4-> interpolation
    \end{itemize}
  \end{block}
  \begin{block}<5->{Presets}
    Presets inherit from the Event class: An event that contains events.
    \begin{itemize}
    \item<6-> No time information
    \item<7-> All addresses must be unique
    \end{itemize}
  \end{block}
\end{frame}

\begin{frame}
  \frametitle{Player module example}
  \hspace{0em} 

  \setbeamercolor{postit}{fg=black,bg=white}
  \begin{beamercolorbox}[sep=0em, wd=9.7cm,rounded=true, shadow=true]{postit}
    
    \includegraphics<1>[width=25em]{../img/player_module_graph}
    
  \end{beamercolorbox}
\end{frame}

\begin{frame}
  \frametitle{Storage }
  \begin{block}{The IXD file format}
    % An XML-based format already used for module definitions and collections of modules (patches). An extension to the current IXD format will provide means for storing and sharing time based data. 
    \begin{itemize}
    \item<2-> Already used for module definitions and collections of modules (patches).
    \item<3-> Time based data forms an extension to the existing formats.
    \end{itemize}
  \end{block}
  \begin{block}<4->{Sequence}
    Sequences of events in time.
  \end{block}
  \begin{block}<5->{Preset}
    A set of events describing a state.
  \end{block}
\end{frame}

\begin{frame}
  \frametitle{Storing Sequences}
   \begin{block}{Re-usability}
    A Sequence can link in other Sequences, or parts of other Sequences:
  \end{block}
  
  \only<2-3>{
  \setbeamercolor{postit}{fg=black,bg=white}
  \hspace{4.6em} \begin{beamercolorbox}[sep=0em,wd=6cm, rounded=true, shadow=true]{postit}

    \includegraphics<2>[width=10em]{../img/code_list_link-1}
    \includegraphics<3>[width=10em]{../img/code_list_link-2}

\end{beamercolorbox}
}
\end{frame}

\begin{frame}
  \frametitle{Storing Sequences}
   \begin{block}{Sequences}
    List of timed events. Sequences may trigger other Sequences.
  \end{block}
  
  \setbeamercolor{postit}{fg=black,bg=white}
  \hspace{4.6em} \begin{beamercolorbox}[sep=0em,wd=6cm, rounded=true, shadow=true]{postit}

    \includegraphics<1>[width=10em]{../img/code_list_seq-1}
    \includegraphics<2>[width=10em]{../img/code_list_seq-2}

\end{beamercolorbox}
\end{frame}

\begin{frame}
  \frametitle{Storing Presets}
   \begin{block}{Preset}
     Presets define events: address/value pairs with no time information
  \end{block}
  
  \setbeamercolor{postit}{fg=black,bg=white}
  \hspace{4.6em} \begin{beamercolorbox}[sep=0em,wd=6cm, rounded=true, shadow=true]{postit}

    \includegraphics<1>[width=14em]{../img/code_list_preset-1}
    \includegraphics<2>[width=14em]{../img/code_list_preset-2}
    \includegraphics<3>[width=14em]{../img/code_list_preset-3}

\end{beamercolorbox}
\end{frame}

\begin{frame}
  \frametitle{Inheritance}
   \begin{block}{Overriding properties}
     Existing data may be extended, dynamically or statically.
  \end{block}
  
  \setbeamercolor{postit}{fg=black,bg=white}
  \hspace{4.6em} \begin{beamercolorbox}[sep=0em,wd=8cm, rounded=true, shadow=true]{postit}

    \includegraphics<1>[width=17em]{../img/code_list_inh-1}
    \includegraphics<2>[width=17em]{../img/code_list_inh-2}

\end{beamercolorbox}
\end{frame}

\begin{frame}
  \frametitle{Meta-Data}
  \begin{block}{Tags}
    \begin{itemize}
    \item<2-> Sequences and Presets may be tagged with semantic information.
    \item<3-> Relations between entities may be created.
    \end{itemize}
  \end{block}
  \begin{block}<4->{Documentation}
    \begin{itemize}
    \item<5-> Documentation resources may be linked in with Sequence and Preset files.
    \end{itemize}
  \end{block}
\end{frame}

\begin{frame}
  \frametitle{Implementation}
  \begin{block}{Integra Environment (beta)}
    \only<1>{\includegraphics<1>[width=17em]{../img/designs5/integra_05_arrange}}
    \only<2>{\includegraphics<2>[width=17em]{../img/designs5/integra_05_arrange_module}}
    \only<3>{\includegraphics<3>[width=17em]{../img/designs5/integra_05_arrange_scripting}}
  \end{block}
\end{frame}

\begin{frame}
  \frametitle{Summary}
  \begin{block}{A proposed format for storing and sharing time based data.}
    \begin{itemize}
    \item<1-> XML-based drawing on MIDI, RDF and SMIL with the ability to include SDIF and GDIF.
    \item<2-> Extending the IXD Schemas for Module definitions, Collections and Integra documentation.
    \item<3-> Employ semantic richness and sustainability.
    \end{itemize}
  \end{block}
\end{frame}

\section{Final note}

\begin{frame}
  \frametitle{Thank you!}


  \begin{block}{Funding}
    The Integra project is funded by the European Commission and is a
    collaboration between Universities, research centers and New Music
    Ensembles in Europe.
  \end{block}

  \begin{block}<2->{Questions?}
    \ldots
  \end{block}

  \setbeamercolor{postit}{fg=black,bg=white}
  \vspace{1.5em}
  \hspace{5.6em} 
  \begin{beamercolorbox}[sep=0em,wd=5cm, rounded=true, shadow=true]{postit}
    \begin{figure}[h]
      \centering
      \includegraphics[width=10em]{../img/EU_logo}
    \end{figure}
  \end{beamercolorbox}
\end{frame}

\end{document}


