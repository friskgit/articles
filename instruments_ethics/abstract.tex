% Created 2023-02-15 ons 12:27
% Intended LaTeX compiler: pdflatex
\documentclass[11pt]{article}
\PassOptionsToPackage{hyphens}{url}
\usepackage[utf8]{inputenc}
\usepackage[T1]{fontenc}
\usepackage{graphicx}
\usepackage{longtable}
\usepackage{wrapfig}
\usepackage{rotating}
\usepackage[normalem]{ulem}
\usepackage{amsmath}
\usepackage{amssymb}
\usepackage{capt-of}
\usepackage{hyperref}
\usepackage[american]{babel}
\usepackage[x11names]{xcolor}
\hypersetup{linktoc = all, colorlinks = true, urlcolor = DodgerBlue4, citecolor = black, linkcolor = black}
\date{\today}
\title{On the ethics of instruments}
\makeatletter
\newcommand{\citeprocitem}[2]{\hyper@linkstart{cite}{citeproc_bib_item_#1}#2\hyper@linkend}
\makeatother

\usepackage[notquote]{hanging}
\begin{document}

\maketitle
\noindent
Proposal for \emph{Vintage Materialities in Music}, lecture-recital.
\section*{Abstract\footnote{The main topic of the call addressed in this study is how obsolete technologies for music production and performance can become desirable again in an era of high-tech, across different genres, and in both the professional and amateur spheres.}}
\label{sec:orga7b8591}

In this lecture-recital Tresch \& Dolan (\citeprocitem{2}{2013})'s notion that the material aspects, mediations and the telos of an instrument can provide grounds for an analysis of its  \emph{ethics} is discussed. It may first appear odd to speak of ethics in relation to objects such as a musical instruments, and this is an attempt to revisit the origin of this idea and critically examine it by briefly discussing its roots in Foucault's \emph{History of Sexuality part 2} (\citeprocitem{1}{Foucault, 2012}). In the presentation a performance on the \emph{Dataton 3000}, a modular synthesizer and audio mixer designed in Sweden in the 1970's (see ANONYMIZED), is used to illustrate how these ideas can be understood and critically assessed. To attempt to understand the qualities and the particularities of this instrument a wide range of parameters need to be considered, including those related to the context in which it was originally created. Yet, development of performance practices may also happen by simply disregarding such information and treat the instrument primarily as a vehicle for the creativity of its player. In the attempt to understand the dialectical relation between staying true to the instrument's origin (according to some principle) and allowing new practices to be formed on top of old, or by simply bypassing existing and/or forgotten practices, there is a need for a method. Though the notion of an ethics of instruments as sketched out by Tresch and Dolan appears to be useful, it also raises questions related to the agency of the instrument in a network af agents formed through performance and through the interfaces that emrge in the playing.

\vspace{1cm}


\section*{Bibliography}
\label{sec:org86e3f90}
\begin{hangparas}{1.5em}{1}
\hypertarget{citeproc_bib_item_1}{Foucault, M. (2012). \textit{The history of sexuality, vol. 2: The use of pleasure}. Vintage.}

\hypertarget{citeproc_bib_item_2}{Tresch, J., \& Dolan, E. I. (2013). Toward a new organology: Instruments of music and science. \textit{Osiris}, \textit{28}(1), 278–298. \url{https://doi.org/10.1086/671381}}
\end{hangparas}
\end{document}