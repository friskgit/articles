% Created 2023-01-31 tis 17:04
% Intended LaTeX compiler: pdflatex
\documentclass[11pt]{article}
\PassOptionsToPackage{hyphens}{url}
\usepackage[utf8]{inputenc}
\usepackage[T1]{fontenc}
\usepackage{graphicx}
\usepackage{longtable}
\usepackage{wrapfig}
\usepackage{rotating}
\usepackage[normalem]{ulem}
\usepackage{amsmath}
\usepackage{amssymb}
\usepackage{capt-of}
\usepackage{hyperref}
\usepackage[american]{babel}
\usepackage[x11names]{xcolor}
\hypersetup{linktoc = all, colorlinks = true, urlcolor = DodgerBlue4, citecolor = black, linkcolor = black}
\author{Henrik Frisk}
\date{\today}
\title{Abstract}
\makeatletter
\newcommand{\citeprocitem}[2]{\hyper@linkstart{cite}{citeproc_bib_item_#1}#2\hyper@linkend}
\makeatother

\usepackage[notquote]{hanging}
\begin{document}

\maketitle
\tableofcontents

In the research project \emph{Historically informed design of sound synthesis: A multidisciplinary, structured approach to the digitisation and exploration of electronic music heritage} (HID) one of the questions concerns how a set of obsolete electronic music instruments can be artistically explored as a cultural heritage without dismantling the uniqueness of the properties of the actual instruments. 

One of the challenges in this process is to identify and categorize the instrument in and of itself, not as an object of nostalgia or as an object with which contemporary ideas and trends are exploited, but instead attempt to do it as closely as possible to the cultural, social and political affordances at its time. Based on the heritage research methodology  (Lundberg, 2015) chosen the idea is to explore the possibility of a historically informed performance practice and explore this as a vehicle for artistic expression informed by possibly previously hidden features of the historic instrument. One of the instruments we have approached is the \emph{Dataton 3000}, a modular synthesizer and audio mixer designed in Sweden in the 1970's. To attempt to understand the qualities the particularity of this instrument a wide range of parameters need to be considered related to the context in which it was originally created. Should this analysis not be successful there is a risk that either the instrument's proper affordances are misunderstood, or that one ends up recreating what has already been done with it, or both.

The challenge found a preliminary solution in Tresch \& Dolan (2013)'s notion of an \emph{ethics of instruments} (see Holzer et al., 2021) leaning on the notion that the material aspects, mediations and the telos of an instrument can provide grounds for an analysis of its \emph{ethics}. It may appear odd to speak of ethics in relation to dead objects such as a musical instruments, and this paper is an attempt to revisit the origin of this idea and critically examine it by briefly discussing its roots in Foucault's \emph{History of Sexuality part 2} (Foucault, 2012) and related theories on ethics. The general idea that enables using ethics in this way is based on the fact that instruments have agency, and is further rooted in a relational ontology in which the instrument exists. The heritage process that we use implies that these objects have agency, and that it is of importance. The question approached in this paper is if the concept of "ethical specificities" (Tresch \& Dolan, 2013 p. 298) of instruments can contribute to knowledge about arrangements of historic electronic instruments and their players without blurring the conceptual differences between the actors involved.

There is an obvious risk that the instrument is anthropomorphized in the process of considering its ethics. In the flat ontology that this view is promoting there is further a number of problems that arises, the most urgent perhaps, is that even though the instrument has agency, on the surface level it still lacks the will and the freedom that a human possesses. This objection is clearly situated in a Western dualist view of the world. Traditional laws of causality enforce this view, but it may be useful to see the use of the instrument in (at least) two distinct but related articulations. First, there is the object in an of itself. As such it has certain observable properties and even in this stage it exposes its interface and through it its connections. The fact that it is a musical instrument, that it is playable, that it belonged to a historic area, and so forth. At this stage there is no obvious causality. Second, in the situation when a musician engages with it a different epistemic network emerges and a cybernetic connection emerges through the interface that the instrument exposes. The causal relations in this network are contributing to both the knowledge in the system and to its output.

Whether or not either of these contexts provides evidence for an ethics of the instrument is a philosophical question difficult to resolve from a prctical perspective. If instead the focus is put on the \emph{usefulness} of an ethical perspective of the instrument in the present investigation the query may be approached. In a network consisting of human and non-human objects such as a musician and a \emph{Dataton} module it would be fair to say that the ethics of the human extends to instrument in a way that alters the possibilities of both musician and instrument, making the ethics of the situation useful to consider, artistically as well as from a heritage point of view. The musician has to adhere to the materiality of the instrument, its mediations and telos. As (Dalton, 2018) concludes, "rejecting the freedom of material objects [\ldots] does not imply their moral neutrality" (Dalton, 2018). However, removing the instrument from this network neutralizes its ethical specificities making the player an ethical necessity. Returning to Foucault, however, this presents a possible conceptual problem and one that was always present. The modes of subjectification that he promotes are rooted in his aesthetics of existence and departs thereby necessarily from the private sphere and will not function as a code exposed on an object from the outside.



As a final 
\end{document}