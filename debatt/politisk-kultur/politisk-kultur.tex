\documentclass[a4paper]{article}
\usepackage[swedish, english]{babel}
\usepackage[T1]{fontenc}
\usepackage[utf8]{inputenc}
\usepackage{fancyhdr}
\pagestyle{fancy}

\title{Politisk kultur och akulturell politik}
\author{Henrik Frisk\\{\small henrik.frisk@mhm.lu.se}}
\date{\today}

\begin{document}
\selectlanguage{swedish}
\maketitle

\thispagestyle{empty}

Under en konsert i Hanoi som jag deltog i förra året bröt konserthuschefen plötsligt strömmen. Musiken och videon som spelades hade blivit för magstark och censuren fick gripa in. Ingen förklaring gavs, och de västerlänningar, också diplomater, verkade inte bry sig så mycket. Men för mig var det en omtumlande upplevelse, en påminnelse om vilken oerhört stor betydelse det har att få leva och verka i ett öppet samhälle.

Att kulturministern läxar upp partikollegor i Sverige eller att hennes motsvarighet i Norge hackar på kulturinstitutionsstyrelser går förstås inte att jämföra med den typen av fruktan för praktiska, politiska och ekonomiska repressalier som samtliga vietnameser lever under dagligen. Men här finns en obehgligt gemensam dramaturgi som knyter ihop våra skandinaviska kulturministrar med censuren i diktaturer i andra länder.

Det helt oreflekterade maktutövandet är en del av problemet. Bara en tillstymmelse till reflektion om betydelsen av den makt de besitter borde omedelbart ha lett till insikten att det inflytande man vill utöva ska utövas i ett helt andra sammanhang än genom personliga reprimander. Det är bl.a.genom de personliga beroendekedjorna som makten bygger upp folket hålls på mattan i Vietnam (om t.ex. konserthuschefen inte upprätthåller censuren i konsertsalen kan hans svåger förlora jobbet och hans systers barn kastas ut från den fina skola de går i).

En annan aspekt är oförståelsen för att den ideologiska partipolitiken inte kan styra konstens innehåll eller avsikt på detaljnivå utan att det för med sig stora problem. Att konsten är politisk eller tar ställning politiskt, socialt eller ekonomiskt ändrar inte på det. Vi behöver konsten som motvikt i det politiska samtalet. Förlorar vi den har vi förlorat mycket mer än en sida av konstens uttryck, hela kulturen kommer så småningom kantra.

Henrik Frisk\\
musiker\\
bitr lektor i konstnärlig forskning, Musikhögskolan i Malmö\\
forskarassistent i konstnärlig forskning, Kungl. Musikhögskolan i Stockholm
\end{document}
%%% Local Variables: 
%%% mode: latex
%%% TeX-master: t
%%% End: 
