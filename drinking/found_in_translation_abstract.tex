% Created 2019-09-02 Mån 18:14
% Intended LaTeX compiler: pdflatex
\documentclass[11pt]{article}
\usepackage[utf8]{inputenc}
\usepackage[T1]{fontenc}
\usepackage{graphicx}
\usepackage{grffile}
\usepackage{longtable}
\usepackage{wrapfig}
\usepackage{rotating}
\usepackage[normalem]{ulem}
\usepackage{amsmath}
\usepackage{textcomp}
\usepackage{amssymb}
\usepackage{capt-of}
\usepackage{hyperref}
\usepackage[lf]{ebgaramond}
\usepackage{sectsty}
\allsectionsfont{\sf}
\usepackage[style=authoryear-ibid,natbib=true,backend=biber,hyperref=false]{biblatex}
\bibliography{/Users/henrik_frisk/Dropbox/Documents/articles/biblio/bibliography.bib}
\renewcommand*{\nameyeardelim}{\space}%
\renewcommand{\postnotedelim}{: }%
\author{Henrik Frisk, Nguyen Thanh Thuy}
\date{}
\title{Found in translation}
\hypersetup{
 pdfauthor={Henrik Frisk, Nguyen Thanh Thuy},
 pdftitle={Found in translation},
 pdfkeywords={},
 pdfsubject={},
 pdfcreator={Emacs 26.1 (Org mode 9.1.9)}, 
 pdflang={English}}
\begin{document}

\maketitle

\section*{Keywords}
\label{sec:org5cf115d}
Translation, Authenticity, Interpretation, Emotional Response, Electronic music

\section*{Abstract}
\label{sec:org6778a59}
This audio paper is an exploration of part of the process of making Drinking (2014) - a piece for voice, Vietnamese zither đàn tranh, and electronics. The composition is developed from composer William Brook’s project \emph{After Yeats} in which Brooks provides an instruction describing a collaboration between a performer and a composer. \emph{After Yeats} is not itself a composition, but should rather be seen as a method with which a composer and a performer may realize a score \citep{Brooks2013}. At the outset the chosen poem--\emph{A Drinking Song} by William B. Yeats in a translation to Vietnamese, the native language of Nguyen Thanh Thuy--was recited, and it was the sonic trace of the reading that was the point of departure for the composition Drinking. In this case the meaning of the words is of less significance than the sound of the reading. \emph{After Yeats} is in some ways an attempt to revive the tradition of chanting poems to a lyre accompaniment common in the practice of Yeats and his contemporaries. Henrik Frisk, a Swedish composer, takes the recording of the reading and, according to the instructions in \emph{After Yeats}, works to compose the piece according to the implications of the declamation from Nguyen's reading. According to Marcel Cobussen, "listening also involves an opening of the senses that is not necessarily enfolded in conscious meaningfulness, by sense. It also takes place outside, before or beyond sense; it also refers to a sense that operates outside, before or beyond signification" \citep[p. 131]{cobussen08}. The emotional responses that arise from engaging in artistic practice operate in similar ways to how Cobussen situates listening in a space beyond signification. They may be communicated and sometimes brought to the fore, but they always have a tendency to appear mysterious, arcane and ephemeral. The influence by the many different kinds of agents that may be found in Brooks’ meta-composition makes the creative process daunting to understand but is also wrapped around the idea of what is to be found outside, before and beyond.

\printbibliography
\end{document}