\section{From Open Work to \textit{work-in-movement}}
\label{sec:from-open-work}

The way \emph{Repetition} was conceived of prior to its preparatory
studies and prior to the experiences gained from the two hitherto
realized version of it, it was an open work similar to other works by
Henrik (Det Goda/Det Onda for flute and computer and Drive for
Electric Viola Grande and computer). It was manifested as building
blocks notated and left to the performer to organize. Looking back at
Eco's previous quote from ``The Open Work'' he would probably have
called already this early conception of it a work-in-movement: ``the
auditor is required to do some of this organizing and structuring of
the musical discourse. He collaborates with the composer in making the
composition.'' However, the way the process has evolved, informed by
the knowledge produced in the study performed on Stefan and Love, the
emphasis in the term `work-in-movement' has travelled from `work' to
`movement'. And the emphasis on the `composer' in the quote from Eco
(the auditor is interchangable but not the composer) should in our
case instead be put on the collaboration: The action is more important
than the object.

The work-in-movement that we imply is a literal construction kit, an
IKEA music to be assembled and re-assembled in a recursive process
that should be allowed to continue outside of the collaboration that
gave rise to it. Stefan and Henrik set it in motion but its
authenticity can only be derived from the movement. An open source
music that may be dismantled and reconstructed, added to and altered,
according to its current conglomerate of participants.

\subsection{Documenting the \textit{work-in-movement}: The annotated score.}
\label{sec:docum-work-movem}

In order to fully realize the notion of a \emph{work-in-movement} the
process of documentation---or in musical notation terminology---the
work identifying instructions needs to be reconsidered. The way
\emph{Repetition} as \emph{a work} has evolved a traditional printed
score in musical notation, no matter how detailed the written
instructions are, would be misguiding. The operative word in
\emph{work-in-movement} is precisely \emph{the movement}: the
change. The difference that makes a difference. Therefore the score
will need to communicate the process rather than the result. The
musical notation will no doubt constitute an important part of the
documentation but it will not be sufficient in itself. 

As we have already discussed, apart from this more musicological
aspect on the notation of \emph{Repetition}, there are also practical
issues regarding the notation and the score that need to be resolved
in order to fully allow for the work to continue to evolve. If the
performer is to be able to alter the order of the sections according
to choices made in performance the score needs to be adopted. But also
if the performer chooses to do what Stefan did in the first version,
settlle on a form prior to performance, the 'final' score should
easily allow for this. Finally, adding the third version discussed
above---with a real time interactive computer part---to the list of
possible modes of performance, the score needs to fully document and
guide the performer in how to set up the computer part.

Design considerations for the annotated score:
\begin{itemize}
\item It should encourage the interpreter to engage in a
  collaborative process similar to the one we have gone through with
  the primary goal not to repeat or recreate our process but to find
  his or her own unique version.
\item It should allow the performer to look up any part of the notation at any
  instant.
\item It should allow for the interpreter to easily create a static
  form that he/she can use in performance.
\item It should contain all the software and all the soundfile needed
  to create an interactive as well as a non-interactive version of the
  computer part.
\item It should document important parts of our collaboration and
  allow for other performers to add important parts of their own
  collaborations with the work.
\end{itemize}

Realizing an annotated score fulfilling all of these design goals will
be possible making use of the libIntegra framework and the associated
IXD file format described in \citet{frisk-bull07}. In addition to the
existing and implemented functionalities the Integra 'bundle' format
(see \url{http://www.integralive.org/dokuwiki/doku.php/lib:bundles})
will need to be further elaborated and a IXD browser---a web
application used to access the information in a bundle---needs to be
developed. libIntegra, which is developed and maintained primarily by Henrik
Frisk and Jamie Bullock is open source software published under the
GPL license and the source code may be accessed through
\url{http://sourceforge.net/projects/integralive}.


%%% Local Variables: 
%%% mode: latex
%%% TeX-master: "../Repetition"
%%% End: 
