
\section{Introduction}
\label{sec:introduction-1}


In this text we give an account of the dynamics of the interaction in
a composer/performer collaboration. We also discuss the dynamic system
of a musical work conceived in a mode of radical open structure and of
the interaction between acoustic and electronic sound worlds and modes
of production. When we engaged in the actual work of producing a piece
of music for 10-stringed guitar and electronics we departed from the
point of view that the activities of either one of us could be
analyzed and understood as poietic (constructive) or esthesic
(interpretative) and that both of us could be oscillating back and
forth between these two modes. (This oscillation was something we had
observed in the collaboration between Stefan \"{O}stersj\"{o} and
composer Love Mangs described in \citet{frisk-ost06}.) I.e. already at
the outset had we consciously dismantled part of the notion of the
composer-performer dichotomy with the main purpose of engaging in a
collaboration in which we would allow ourselves to challenge our
respective roles. Challenge how these roles affects our choices but
also how they influence the collaboration. As we will see this
attitude resulted in some unexpected results as well as (obviously)
some expected.

In a previous article \citep{frisk-ost06-2} we outlined a theoretical
background for an altered use of the poietic-esthesic dynamics and a
semiotic approach to analyzing communication between agents involved
in the production of musical content. In Section 2 we will look at
other aspects of collaborations and collective work. We look further
into the consequences of reclaiming the musician at the expense of the
specialized roles of the the composer, the performer, the programmer,
etc. In Section 3 we return to Umberto Eco's classic writings on the
poetics of the open work and adopt his category of work-in-motion,
referring the work-kind that has become a feature of post-modern
aesthetics in the age of the internet. The unfolding of the different
versions of the composition Repetition Repeats all other Repetitions
(from here on referred to as Repetition) that we hitherto have
produced and that we plan to produce is described in Section 4. We
discuss several instances within this process from the point of
departure of video documentation of our work. In Section 5 we return
to the discussion of the work-in-motion and the documentation of works
of this kind. In the discussion in section 6, we evaluate the impact
of the double roles of artists and researchers that we have had in the
present project and discuss specifically the influence that the
research has had on the development of the guitar piece.

\section{Collaborative work: A background }
\label{sec:coll-work:-backgr-1}

Collaborative artistic work may be seen as a reaction against the
singularity of the modernist view of the artist as the inspired,
predominantly male, creator from whom great artworks emanate. It is
however very difficult to draw a distinction between collaborative
work as decentralized division of labor on the one hand and, let us
call it, less collaborative work that stems from centralized authority
on the other. To begin with, in music, it is difficult to imagine a
performance that does not involve collaboration between different
agencies at some level of the production. Perhaps the Study No. 21,
Canon X from ``Studies for Player Piano'' by Conlon Nancarrow can
serve as an example of non-collaborative music. Not only is it
composed for a mechanical instrument (the Ampico player piano
\url{http://en.wikipedia.org/wiki/Player_piano}) and punched into the
player roll rather than written out in traditional notation, it wasn't
performed in public (with one exception) until after his international
break through in 1982. Following is an excerpt from an interview with
Nancarrow that gives evidence of the seclusion and non-collaborative
environment in which this and the other Studies were composed:

\begin{quote}
  Duckworth: Between the early 1940s and 1960, when you were writing
  most of your Studies, was anyone hearing your music besides you?

  Nancarrow: No Duckworth: Were you just playing them for yourself?

  Nancarrow: Occasionally someone would hear it, but very
  occasionally. (p 49)
\end{quote}

At the other end of the collaborative/non-collaborative continuum,
among other things, we could place any example of free jazz group
improvisation in which nothing is pre-determined apart from the
instruments, the players and the physical (and usually social)
context. Evan Parker writes about a recording of a trio improvisation
featuring, apart from himself on saxophone, Derek Bailey and Han
Bennink: ``We operate without rules (pre-composed material) or
well-defined codes of behavior (fixed tempi, tonalities, serial
structures, etc.) and yet are able to distinguish success from
failure.'' (as cited in ``Reflections of surrealism in Postmodern
Musics'' by Ann LeBaron pp. 39). A lack or dismissal of ``rules'' of
operation or ``codes of behavior'' are not necessarily prerequisites
for collaboration but, they may be seen as a method with which the
participants are forced to collaborate in order to avoid cacaphony and
chaos. (Though, admittedly, chaos may also be a most viable means of
collaborative expression.) The above quote from Evan Parker may need a
further comment, since of course no musical collaboration takes place
outside of a musical discourse. When Parker claims that the
performance mentioned was independent of well defined codes of
behaviour one should also take into account the intertextuality at
play in any musical discourse, and the influence of regulative 'texts'
on a specific performance. Certainly, in the case of Parker, Bailey
and Bennink, we are talking of performers that can expected to have a
strong awareness of the development of the whole field of free
improvisation and hence that any performance is interconnected with an
entire tradition of new performance techniques, approaches to sound
production and musical form as well as to specific perfomances,
especially CD-recordings, that function as `text' in the musical
discourse. These two examples can be seen as demarcation points
indicating the relative amount of freedom and artistic choice left to
the perfomer. In the case of Nancarrow's piano works, the performer is
exchanged by the piano technician for whom the artistic freedom is
limited to the binary choice between mounting the roll or not mounting
the roll. On the other hand, the by themselves perceived freedom of
Parker, Bailey and Bennink is unlimited.
 
Vera John-Steiner (John-Steiner, 2000) has provided some of the most
important theoretical work on artistic collaboration so
far. Emphasising the amount of groundbreaking research and artistic
work that has been performed in different collaborative contexts, she
argues that the possibilities for stretching the individual's
potential makes up a strong argument to reconsider the individualistic
approach to knowledge construction that is so typical of Western
thinking, artistic endeavour and research. She refers to the work of
Lev Vygotsky and finds that in artistic collaboration, zones of
`proximal development' can be created, by which the participators can
stretch their artistic abilities in a mode similar to how Vygotsky
describes children's learning processes. Vera John-Steiner
distinguishes between four different patterns in collaborative work:
`Distributed'-, `Complementary'-, `Family'-, and `Integrative'
Collaboration.  Out of these patterns, we will be discussing our
interaction in this project,---involving both research and artistic
practice---in terms of complementary and integrative
collaboration. `Complementary Collaboration' is, according to
John-Steiner, especially common in scientific partnership, in which
diversity in skills and modes of thinking lead to the creation of new
knowledge. However, in addition to scientific contexts one may say
that this pattern is rather typical also of artistic collaboration
within Western art music, where for instance the differences in
training and skills between a composer and a performer support the
project through division of labour. The mode most typical of artistic
collaboration, argues John-Steiner, is the fourth, `Integrative
Collaboration', in which collaboration partners suspend their
differences in style aiming at a common artistic goal. This involves
the merging of visions and technical skills in projects that have
proven to be typical of transitory periods, giving way for
paradigmatic changes of conception. (John-Steiner, 2000, p. 70) In her
book, John-Steiner provides several examples of how the shared risks
and practices in this kind of artistic collaboration has been
essential in many crucial moments in experimental artistic practice,
starting out with the collaboration of Picasso and Braque that gave
birth to cubism. We find this account rather challenging to the
present day practices of composers and performers.
 
\subsection{Musical notation and the Split of The Musician }
\label{sec:music-notat-split}

The history of Western Art Music displays a gradual process of a split
of ``the musician'' into composer and performer respectively
\citep{wis96}. This process is launched with the invention of notation
and further enhanced by the development of copyright and the formation
of the regulative work-concept (Frisk \& \"{O}stersj\"{o},
2007). Although earlier examples may be found, by the early 20th
Century one may say that this split reached its first most definite
forms, for instance in the modes of collaboration between composers
and performers within the circle of the 2nd Viennese school.  At this
point, 'the work' was regarded as synonymous to the score: the ideal
performance should reflect what Adorno (with reference to the work of
the violinist Rudolph Kolisch) termed ``integral
interpretation''---producing an x-ray of the work, and thus reflect
the most exact analysis of the score (Smith, 1986,
pp. 107-108). Especially in a musical discourse in which the work
concept has a regulative function, the score is indeed a very strong
agent. Notation is one of the primary tools by which composer and
performer interact, another of course being the instrument. The
interaction between these two agents in the field of the work is
reflected in the following quote from an interview with Luciano Berio,
in which he considers how virtuosity becomes a parameter in a musical
work:
 
\begin{quote}
  Virtuosity often arises out of a conflict, a tension between the
  musical idea and the instrument, between concept and musical
  substance [\ldots] As is well-know, virtuosity can come to the fore
  when a concern for technique and stereotyped instrumental gestures
  gets the better of the idea, as in Paganini's work---which I'm very
  fond of, but which didn't really shake up the history of music,
  although it did contribute to the development of violin
  technique. Another instance where tension arises is when the novelty
  and the complexity of musical thought---with its equally complex and
  diverse expressive dimensions---imposes changes in the relationship
  with the instrument, often necessitating a novel technical solution
  [\ldots] where the interpreter is required to perform at an extremely
  high level of technical and intellectual virtuosity. (Berio,
  Osmond-Smith, Varga, \& Dalmonte, 1985, pp. 90-91)
\end{quote}

Paganini, who operated in a time prior to the strict division of
labour between composer and performer, could conceive his work in
direct dialogue between himself as a performer/composer and hist
instrument. In the case of Berio and his famous cycle of virtuoso solo
works, the sequenze, this direct relation is replaced by a mediated
interaction between a composer on the one hand and a performer and his
instrument on the other. The tension between a certain notation of a
musical idea and a performer can indeed result in virtuoso performance
but also in outspoken conflict. The reason to this is most often to be
found in a lack of common understanding of the notational and musical
conventions of the composer's music. This can for instance be seen in
the conflict between composers such as Stravinsky and Ravel and many
performers in the early 20th Century. This tension can also result in
the invention of new forms of notation. Luciano Berio's Sequenza for
flute solo displays some of these processes. It was written for the
flautist Severino Gazelloni, a performer specialised in new music and
part of the circle of performers in the Darmstadt school. The score,
such as it was published in 1958, is one of the earliest examples of
the use of proportional notation and the piece has become a true
modernist classic, performed and recorded by most important flautist
of the past decades. The score was revised by the composer and
published in metric notation in 1992. But the story of the flute
sequenza is much more complicated than this, as has been thoroughly
discussed by Cynthia Folio and Alexander Brinkman in a recent book on
Berio's sequenzas edited by Janet Halfyard (Brinkman \& Folio,
2007). In fact, Berio's original notation was a more complex and
detailed metric notation, consistently in 2/8. The proportional
notation came about as a solution to Gazzeloni's problems to cope with
this complex notation. However, already in the 1970's Berio began to
complain about performer's inability to grasp the intentions of this
proportional notation: ``even good performers were taking liberties
that didn't make any sense, taking the spatial notation almost as a
pretext for improvisation.'' (Muller \& Berio, 1997, p. 19) The 1992
version was derived by making a simplification of the original metric
notation, and not as some have assumed as a renotation of the
proportional version from 1958.  One could expect all performers to
immediately switch to the new version as soon as it was
published. However, a striking finding in the research by Folio and
Brinkman is that a great majority of professional flautists that have
recorded the piece dislike the new version, all performers in their
study except for one still play the piece from the 1958-version!
 
In the context of the Darmstadt courses in the 1950's and 60's we find
on the one hand the most radical artistic statements along the line of
this division of labour ---within the development of electronic music
and integral serialism ---and on the other hand attempts to rethink
the interaction between the performer and the score. We will now turn
to a further discussion of the development of ''open form'' in the
1950's.
 
\section{The Open Work }
\label{sec:open-work-}

The emergence of the concept of open form received an early
introduction by the young Umberto Eco in his book The Open Work (Eco,
1989), originally published in Italian in 1962.  His discussion of the
open work is triggered by the experiments with mobile and open form in
musical modernism of the 1950's and 1960's. He starts out by giving
for examples of musical works by Berio, Boulez, Pousseur and
Stockhausen. Already at the outset he also provides a definition of
what the significance of this particular kind of openness might be. Of
course, philosophically speaking no art works are closed entities but
demand from the receiver an active participation in the construction
of the work: ``every reception of a work of art is both an
interpretation and a performance of it, because in every reception the
work takes on a fresh perspective for itself.'' (Eco, 1989) There is
also a difference in degree of openness in different arts. A musical
score always has the function of a construction kit, demanding from
the performer to make a version of the work. However, Eco finds that
this new category of ``open works'' takes this openness to a different
level, not only giving shape to the fine grained details of a work at
his own discretion but taking decisions also on the greater form of
the work.
 
The picture is broadened by a historical perspective on the
simultaneous development of new world views and modalities of
interpretation of the arts. Eco gives an historical account in which
the scope for a multiplicity of interpretations is seen to be
gradually increasing from the Middle Ages up to our time. This
progression is found to start out from the theory of allegory that
allows the reader in the middle ages a total of four different
interpretations in reading the Scriptures: apart from the literal
sense there was also the moral, the allegoric and the anagogical
sense. Obviously this is a limited form of openness where
interpretation is rigidly predetermined and the choice for the reader
is which kind of interpretation to apply in a certain reading. ``The
laws governing textual interpretation are the laws of an authoritarian
regime which guide the individual in his every action, prescribing the
ends for him and offering him the means to attain them.'' (Eco, 1989,
pp. 6-7) Eco continues with examples from the openness of baroque
architecture, allowing for an infinitude of different viewpoints, to
the notion of ``pure poetry'' in pre-romantic and romantic poetry, but
he locates the first conscious poetics of the open work to the
symbolism of the late 19th Century. He continues by referring to the
works of Mallarmé, Kafka and Joyce, and finds that for works like
Mallarmé's ``Livre'' or the musical works mentioned in the outset, a
category within the open work-kind can be identified. For these
radically open works he suggests the term `work in movement'. In these
works\ldots ``the auditor is required to do some of this organizing and
structuring of the musical discourse. He collaborates with the
composer in making the composition.'' However true this may be in the
case of a reader working his way through the mobile structure of
Mallarmé's text, this is not quite true for a listener in the case of
any of these musical works mentioned. In musical `works-in-movement'
it is always the performer(s) who need to organise the final structure
of the piece such as it reaches the audience in a certain performance.
The reasons to this progression are, according to Eco, to be found in
the changing conceptions of how science or contemporary culture views
reality. The hierarchic structure of the medieval world-view is
represented in the closed conception of fixed allegoric forms. The
dynamic openness in the baroque is directly related to the changes in
scientific awareness. By giving up the notion of a single viewing
point, baroque architecture mirrors the Copernican world-view. What
also comes to the fore is a new subjectivity, shifting the attention
from the essence to the appearance, to the psychology of impression
and sensation. The radical openness in some of the music of the
Darmstadt school is then understood as a parallel to the landmarks in
modern science, one aspect being the perceptive ambiguities discussed
in modern psychology and phenomenology. We find it to be problematic
that, in many contemporary discussions of this development, no
distinction was made between aleatoric elements and experiments with
open form.  For instance, Boulez' 3d Piano Sonata was often discussed
as an example of an aleatoric work (Riley, 1966) . There is however a
crucial difference between chance and choice. When the performer is
left with a series of choices regarding the large scale form, for
instance in the Boulez' sonata, this is to leave the greater form of
the work in part to the choice to a collaborating performer, not to
chance operations. (Performers are not chance generators!) In the
context of the split between composer and performer, the increased
dependency between the two agents is an important aspect of these
works. The contribution of the performer to the work is one focal
point of interest in the present study. Another is the concept of the
open work and specifically, the sub category for which Eco coined the
term work-in-motion.
 
There is a close affinity between the concept of the work-in-motion
and the torso. The unfinished nature of an intentionally ``open''
structure leaves a space for the performer which is not unlike the
reconstructions by scholars or artists of historic works left
unfinished by their authors. Interestingly enough, one of the most
important early examples of radically ``open works''---the already
mentioned Boulez' third piano sonata---is still unfinished. Many have
found this to be part of the fascination of the piece, but not so the
composer, who still in 1989 stated that he feels obliged to complete
the intended form in five ``Tropes'':
 
\begin{quote}
  \emph{P.B.} [\ldots] I will finish it, certainly one of these days, because I
  conceived it as a total, global architecture, and now there are only
  two doors instead of five. I feel obliged to make a complete five
  movements.\\  
  \emph{P.M.} Yet there is something appealingly mysterious about
  those other unknown doors.\\
  \emph{P.B.} Ah yes, but they are not mysterious
  for me. (McCallum \& Boulez, 1989, p. 8)
\end{quote}

There is no doubt from the above quote that the composer has not given
up any of his feeling of singular authorship to this work, in spite of
the decades that have passed with a continuously increasing amount of
performances of the piece with many different pianists. It is quite
likely that however much listeners, or for that sake performers, would
argue that the piece would lose some of its mysterious qualities if
the unwritten movements were added to a final score, this would really
be of no significance to Boulez: in the frame of the regulative
work-concept within which he is working, there is no space for
exterior impulses to affect the identity of the work. With the risk of
creating stereotypes from unclear distinctions, one may say that there
is a fundamental difference between the modernist and the
post-modernist approach to the open work.
 
The concept of the open work, and in forms that certainly can be
referred to Eco's category of works-in-movement, has become central to
the Digital Arts. An early example of a work that instead of a
finished structure consists of a frame work that allows participants
to contribute to a continuous process is The File Room (1994) by
Antonio Muntadas. It consists of a database to which anyone can
provide data on censorship. (Dietz, 2002) Although there is a
coordinator who organizes new information, still anyone with internet
access can contribute to the work, which still exists as a permanent
work-in-progress. As such, The File Room, is an excellent example of a
post modern new reading of the open work.


\section{The Instances }
\label{sec:instances-}


The first version of the score consisted of the three materials in the
form that they were constructed, as individually independent and
internally coherent musical processes. It was never the intention to
have the piece performed in that way. In preparation for the first
performance, Henrik produced a performance score that was intended to
give a certain limited amount of freedom of choice in
performance. However, Stefan found there to be several problems with
that version and suggested to make a different (and fixed) version for
that very performance. One problem was practical, the score didn't
seem to allow for actual use in a concert performance with all the
pages and turns back and forth that was a result of the open choices
left to the performer's choice in the course of the actual
performance. The other problem was that he found the greater form not
to be working. So, before leaving for Vietnam, Stefan had made notes
in the score of how to edit the material into a different fixed
version. Well in Vietnam, the score was cut into small chunks and put
together according to a formal outline that Stefan had drawn out. The
electronic part was mapped onto these fragments following the
structure of the three materials.  The second version is a new reading
of the score, in which the greater form is guided by the form of the
modernist film classic, Viking Eggeling's 'Symphonie Diagonale'. This
version was structured by Stefan and Henrik in joint sessions, working
with an audio recording of the first version as one of the main
materials.
 

The third version (in progress) will incorporate the experience
gathered during these two years of collaborative work and performances
of the two first versions. In the organisation of the interactive
electronics, also some of the theoretical work, based also on the
empirical studies, will be taken into account. The artistic solutions
have in many ways been the result of discussions and reflections made
underway during these couple of years. The third version will be the
first to be technically interactive, fulfilling of using Henrik's
timbreMap software.
 
\subsection{The Score }
\label{sec:score-}


In order to unwrap the processes that eventually led to the first
version of Repetition Repeats all other Repetitions there are
circumstances, exterior to the actual process of composing, that needs
to be considered. The first and all other encompassing is the fact
that we were both aware of and had mutually agreed on the piece being
part of a research project. Within the research project---the focus of
which was, and still is, the negotiations taking place between
different agents in composition and interpretation---we performed
theoretical studies on a collaboration between Stefan and composer
Love Mangs (as described in Frisk \& \"{O}stersj\"{o} (2006a)) in
which we observed how the constructive actions would flow back and
forth between the composer and the musician. Our analysis of this
non-linear interaction was that it played an important role in the way
the music was constructed and that the fact that the collaborators
allowed it to happen contributed to unorthodox and creative solutions
to problems. Therefore, already at the outset we agreed that the
composition, in its structure and format, should allow for external
(e.g. non-composer) input and influence---at the time of writing but
also during its phases of interpretation and performance.

As part of the agreement, and perhaps a first prerequisite for a
successful composer-musician collaboration, was also the idea of
partly giving up the traditional view on our respective
roles. Specifically this means that we consciously allowed and
encouraged the `swapping of the roles' that we had observed in the
\"{O}stersj\"{o}/Mangs collaboration. As a composer in the context of
Western culture this is not a trivial thing to partake in. The
understanding of and the values traditionally assigned to the
respective labours `composer' and `interpreter' are fairly rigid and
socially well defined. In particular in Western art music the composer
is expected to be the authority on his own music, its modes of
expression and its meaning. To choose not to shoulder this
responsibility fully and solitarily, but rather share it and let it be
influenced by a number of factors, may easily be taken as a lack of
assurance or, worse, a sign of an unfinished, unserious or hasty
work. Similarly, an interpreter taking part in such a collaboration
may feel anxiety about trespassing into the realms of the composer.

It is notably difficult to define the end of research and the
beginning of artistic work in a project such as this and it is equally
difficult to define the type of collaboration we engage in here. From
a personal note I felt it possible to use my background as an
improvisor, esthetically and stylistically, much more fully than I
have in other composition processes. As has already been noted
'Complementary Collaboration' may be seen as typical to
performer/composer interaction but the present collaboration goes
beyond a ``division of labor'' and among other things, the agreed
collaborative form gave us the confidence and assurance to engage in a
process that in the end changed the object to an extent that we had
not foreseen.

The other important factor to consider while tapping into this process
is the close coupling between Repetition and a similar but slightly
larger collaboration performed within the project The Six Tones. The
Six Tones was initiated in the spring 2006 at a time when only
sketches for Repetition existed and it came to play an important role
as the two pieces were worked out in parallel during the late summer
of that year.

\subsubsection{The Six Tones }
\label{sec:six-tones-}

In May 2006 we had the opportunity to get together with two Vietnamese
master musicians temporarily visiting Malm\"{o}: Ngo Tra My playing
the Dan Bau and Ngyen Thanh Thuy the Dan Tranh. One of the driving
forces behind engaging in this project was the attempt to go beyond a
collage like superposition of two culturally distinct musics and reach
for, not assimilation or integration but rather collocation. As freely
and unprejudiced as possible and with only loose sketches as starting
point we got together all four of us in the composition studio at the
Malm\"{o} Academy of Music to improvise. The outspoken intention was
that these sessions would provide material for a piece by Henrik for
the four of us, with Henrik on laptop.

Stefan had met with Thuy and My prior to our first playing session and
had chosen his instruments based on the properties of the Dan Bau and
the Dan Tranh: The 10-stringed guitar would allow for a bass register
not possible on the 6-stringed guitar and the Banjo had interesting
sonic similiarities with both the Dan Bau and the Dan Tranh and
blended well in with the Vietnamese instruments. As a side note we
found that the five string Banjo, seen as a result of a mix of African
and Western influences had an interesting parallel in the Vietnamese
monochord Dan Bau, an instrument indigeneous to Vietnamese culture but
adapted and amplified with modern guitar pick up microphones.

The sketches we used as starting point for our improvising made use of
the same material I had worked out for Repetition, at the time merely
a tone series in different permutations. For our session I had written
out one of these series as a monophonic melody line laid out between
the three string intruments as if it was one instrument playing (see
Figure X). A number of musical ideas from these sessions (altogether
we met four times in May 2006) had a tremendous influence on how the
ideas for Repetition evolved and how the material was composed.

\begin{itemize}
\item Prior to the Six Tones sessions the idea was to write for
  6-stringed guitar but the register and the possibilities for
  alternate tunings made us choose the 10-string guitar instead.
\item One of the sections of Repetition (the A-material in the
  notation) is a transcription of one of the improvisations on the
  sketch brought to the May 17 session.
\item The scordatura used in Repetition was developed according to the
  needs presented by the Six Tones project. (see Section 4.2.1 for
  more on the scordatura.)
\item Some of the soundfiles used in the first version of Repetition
  used samples recorded during the Six Tones sessions.
\end{itemize}

In the end the first version of Repetition also influenced the first
performance of The Six Tones in Hanoi in October 2006. The B-material
from Repetition was recomposed for the quartet and constitute the
final section in the version we performed.

\subsubsection{Composing Repetition }
\label{sec:comp-repet-}

With regard to the background for Repetition yet another aspect should
be mentioned: Its intertextual relation to Toccata Orpheus by German
composer Rolf Riehm. I became acquainted with Toccata Orpheus when
Stefan and I participated in the workshop Knowledge Lab in Berlin in
June, 2005. Stefan performed the piece (composed in 1990 for six
string guitar solo) and discussed its artistic and interpretative
implications in the workshop. Following this event we repeatedly
discussed Stefan's idea of making a multimedia performance of Toccata
Orpheus that would involve a soundscape by me. In that process we
recorded the piece in its entirety as well as details of it for me to
use as samples in the soundscape. We also made a video recording of it
that I edited and I started elaborating on the idea of, in addition to
the soundscape, also make an electro acoustic piece with video. Though
introverted in its expression Toccata Orpheus is a highly visual piece
as the many different and unusual playing techniques become part of
its construction.

When I started working on the score for Repetition in August 2006 the
Rolf Riehm piece was most certainly a source of
inspiration. Subconciously, because I had so closely studied the piece
and discussed its possible performances with Stefan during a long
period of time. But also consciously. I was inspired by the dramaturgy
involved in the performance of the piece. In Toccata Orpheus the more
radical and strained the movement required of the performer the lesser
the sound---in itself an interesting reversal of force. In preparation
for the soundscape mentioned above I had made an analysis of the piece
based on the movement required to perform each sound and classified
the phrases according to this analysis. I choose to approach
Repetition in a similar manner. I borrowed Stefan's 10-stringed guitar
and constructed a set of sounds based on different playing techniques,
some of which were created together with Stefan and some, such as the
two hand tapping in different tempi, which is used used extensively in
the piece and makes up the C-material, came about on his initiative.

As was mentioned above, the raw material for the piece was a simple
tone series, a version of an infinitely self-repeating series common
to much of Danish composer Per Nørgård's music (an early example being
Libra from 1973, recorded by Stefan in his own transcription for
10-stringed guitar). Conceptually the piece consists of three
variations permutations of the same raw material, called A, B, and
C. These three motives, distinct both sonically and expressively, are,
in a manner of speaking, telling the same story in three different
ways. Though it is only possible for the guitarist to play one of
these motives at a time, irregularly moving back and forth between
them, and with the help of the computer part---also made up of three
corresponding materials, would create the illusion that all three
`stories' were told simultaneously. The guitar and the computer would
merely `give light' or resonate with one version of the story at a
time. The intention was to let the order of the fragments belonging to
the three motives be the result of decisions made in
performance. However, early on in the process we realized that, due to
practical reasons, with a paper score it would be impossible to employ
that level of freedom. Hence the score displayed the three materials
somewhat in paralell to each other and with written instructions about
possible choices.

In this context it should also be mentioned that my analysis software
timbreMap already from the start of the project was intended to
constitute the link between Stefan and the computer part. timbreMap is
a self organizing feature map that, when fully trained, may
automatically detect timbral differences in the input audio
signal. The idea was to have the software `listen' to Stefan and be
able to detect when he moved from playing, say a fragment of the A
material, to some B material. For this reason too it was necessary for
me to compose the three materials in a way that they would be
sufficiently timbrally distinct from each other for the software to
detect a change. At this time, in August 2006, I had a
proof-of-concept version of timbreMap working which gave me a chance
to evaluate and test the different playing styles and their effect on
the `listening' computer.

\paragraph{Motive A }
\label{sec:motive-}

As noted above, this motive is a transcription of the first section of
the first sketch for The Six Tones. The transcription was made from a
recording made on May 24 which was the last of our four sessions with
My and Thuy in Malm\"{o} 2006. They both went back to Hanoi shortly
after. I made the transcription on Stefan's guitar, trying to find
playing techniques that would mimic the sounds of the two Vietnamese
instruments and the Banjo, in part played with an e-bow. Some of the
playing techniques I made up and some, such as the Koto pizzicato,
were suggested by Stefan and are part of contemporary guitar playing
technique. The A motive electronics make use of samples from the same
Six Tones sessions.

Figure 1a: An excerpt of the original sketch for The Six Tones, the
recording of which was used in the transcription. The numbered boxes
correspond to the bar numbers in the score for Repetition.


Figure 1b: The corresponding bars in the score.
 
\paragraph{Motive B}
\label{sec:motive-b-1}


The original idea for this motive was to present a melody derived from
the tone series all by itself. The accompanying harmony was thought to
be used in a separate section. Since the melody did not work on its
own, at least not the way I had conceived of it, and the time being to
short to realize the middle section, I decided to combine the two
elements. The harmony is likewise constructed from the tone series and
all of the chords I played myself and recorded. For the B motive
electronics I made a number of laptop improvisations with the samples
of the chords that I recorded and edited.

\paragraph{Motive C }
\label{sec:motive-c-}

The C section is perhaps the most obscure of all three materials. It
is almost entirely `tapped' on the fingerboard of the guitar using
both hands in fairly complex polyrythmic patterns. Apart from creating
a timbral texture distinct from the other two it allowed for a kind of
two line polyphony very difficult to perform playing the guitar with
standard technique. The polyrhythms are derived from the harmonic
ratios between the fundamental of the tone series (the note where it
starts and in relation to which all its transpositions relate) and
each successive note (C-D has the harmonic ratio of 8:9). The computer
part plays a recording of a physical model of a guitar with glass
strings whose exciter is a recording of the C section performed by
Stefan.

\subsubsection{Summary }
\label{sec:summary-}

Looked at in isolation the production of the first version of the
score was a relatively solitary and non-collaborative process in which
I worked primarily in the poietic domain with occasional input by
Stefan, primarily operating in the esthesic domain. However, this
first version of the score was never performed. One reason is a still
unresolved practical design related difficulty in creating an open
score which allows for the choice of the performer to be made in real
time. Furthermore technical problems and lack of time---the first
performance was only weeks away---made it difficult to complete the
interactive version of the computer part. At the time neither I nor
Stefan was probably fully aware that this version of the score only
constituted the beginning of a long process that has yet (ever? ) to
end. In hindsight it is however clear that it is a mere raw material,
exactly notated on the detailed level but whose few performance rules
regarding the greater form we were to break in the concert versions
that came out of it. The composition process described above was in
some ways the pivotal moment, the turning point, that allowed for a
transformation from a classically conceived open work to a
work-in-motion.

\subsection{The First Version }
\label{sec:first-version-}

When Henrik had established the idea of a close interrelation between
the projects later called The Six Tones and the guitar piece, the next
step was the first session we did together with Ngo Tra My and Thanh
Thuy on May 17. During these two sessions, and also in discussions
before and in between them, the tuning of the 10-string guitar was
established. Henrik chose to use the same tuning for the guitar piece
as well. Looking at the video one may say that the outline of the
scordatura is not guided by Henrik, but rather based on Stefan's
response to some ideas of Henrik's but also some features of the Dan
Tranh. Already Stefan's choice of instruments was made with the
ambition to amalgamate the sound worlds of the Vietnamese instrument
with some Western stringed instruments. The making of the scordatura
for the 10-string guitar takes this ambition a little further. The
first idea, which comes to Stefan already before the May 17 session is
to tune the 1st string to a unison with string three. The first time
he came across this kind of low tuning was in Kent Olofsson's Il Liuto
d'Orfeo (in which the 1st string is tuned to a B flat and the 2nd
string to an F). One main feature of radically lowered top strings is
the possibility to make extensive bends. With two strings tuned to the
same pitch and one being loose enough for bending, the guitar can
produce similar incantations on a single note as one can on two
strings on the Dan Tranh.

\subsubsection{The scordatura}
\label{sec:scordatura}


Figure 2: A chart of the scordaturas used in the process of defining the tuning in the guitar piece

The video consists of one clip from May 17 and basically one from May
18. However, a little bit of irrelevant material has been taken out in
the May 18 clip. In the very beginning of the first clip I bring up
the fact that I have no high string. It will become quite clear that
Henrik is not especially focussed on my adaption of the scordatura at
this stage. The two times I try to get a discussion going about the
tuning he rather roughly changes the subject. Here's the first time:
 
\begin{quote}
  \emph{Stefan}: I have no high strings, I haven't used this yet but I have
  this\ldots [Henrik turns to Thuy, who has simultaneously started
  playing]\\
  \emph{Henrik}: Oh yeah, something like that, and then together?
\end{quote}

It would be a mistake though to interpret this as disinterest in the
topic. However, Henrik was very much more focussed on anything that
Thuy and My did in the session, presumably with the intent to make
them feel comfortable and welcome. The G on the 1st string was
modulated by Henrik to a G quarter tone sharp when he started working
on the guitar piece.
 
Another special feature of the scordatura in these two pieces is the
quarter tone cluster on strings 7-9. One may say that this is
intertextually related to Olofsson's Il Liuto d'Orfeo in which strings
7 and 8 on the 10-string guitar are tuned to quarter sharp C and C
natural respectively. However, one can see in the clip from May 18 how
this tuning is made in response to the characteristics of playing the
Dan Tranh to the left of the bridges and the quarter tones that arise
from that.
 
\begin{quote}
  \emph{Henrik}: [\ldots]the point is that eh, you start playing sort of a
  tremolo maybe on this one string, the one that is closest to a D and
  then you start adding more strings around, so that the sound get's
  fatter and fatter, you know, you get more pitches in\ldots\\
  \emph{Stefan}: What, D and C?\\
  \emph{Henrik}: Well, D and C\#, actually both upwards and
  downwards around D.  [Stefan starts retuning strings 7-9 into a
  quarter tone scordatura]\\
  \emph{Henrik}: Yes, that's it! Yes, exactly! She
  plays on the other side [of the bridges] so she has this strange
  sound
\end{quote}

The third feature of the scordatura that appears in the video is the
low F on string 10. This is already the tuning I am using when we get
to the next element in Henrik's draft for the piece, the C at which we
arrive in the Dan Bau after the arpeggios and dense chords. I point
out to Henrik that this is a natural harmonic on my 10th string and he
immediately responds by suggesting we should do it on Dan Bau and
guitar in unison. At the same time that became a decision to keep the
F tuning on string 10. The rest of the strings kept their standard
tuning and in this way, the tuning of the guitar in the guitar piece
was derived from these sessions with The Six Tones.

\subsubsection{Cut up in Hanoi. Breaking the Rules I.}
\label{sec:cut-up-hanoi}

One should bear in mind that time was short when Henrik presented the
first performance score to the piece. It was arranged in a way that
left the performer a limited amount of choices as to what order the
fragments of the three materials should be played. According to
certain rules (and above all, the distribution of the material in the
score) the performer's set of choices was quite similar to those
facing a pianist preparing a performance of Boulez 3d sonata. When I
started reading the score and trying it out on the instrument I was
concerned with two problems: 1) the intended openness created problems
in terms of page turns that seemed impossible to solve 2) I found the
greater form less convincing than the individual material. In a move
from esthesic to poietic action I posed a rather radical solution to
Henrik that involved breaking the rules of the performance score in
two ways: 1) reorganising the material into a fixed succession of
individual fragments in a manner not based on the set of choices in
the score. 2) I also wanted to have the liberty to further cut and
edit some of the fragments that would at times create new
inter-relations and phrases.

Time was short and the premiere was running close. Henrik agreed to my
suggestions, probably still finding it an accordance with the ``higher
level intentions'' (always nice to give a nod to Randall Dipert) for
the piece. Of course, my suggestion would have made Boulez furious,
but with Henrik it was a perfectly viable choice, not least because of
the collaborative pretext of the project. But of course, one intention
with the first performance score was that the choice of the performer
should be done in the course of performance. My solution was to
instead make a fixed version was not intended to nullify the open form
of the work, only to leave the choices to the pre-performance
phases. In other words I was only partially breaking the rules of the
score.

Having agreed on this with Henrik I sat down to reorganise the
material, giving new succession numbers to the fragments. I decided to
stick with the fist rule of the piece, to always start with the
A1-section.

 When we arrived in Hanoi there was not yet a performance score to the
 piece nor a performance version of the electronics. However, the
 outline for both a score and how to work out a Max patch for the
 electronic part already existed. Stefan had made notes in the first
 attempt at a performance score that Henrik had made, suggesting a
 fixed form for the piece. On the first night we sat out to edit the
 score according to these written instructions. The intention was to
 continue by organising the electronic part after having finalised
 this version of the score. It is important to bear in mind that when
 Stefan wrote the instructions for how to reorganise the material into
 a fixed score, this was not intended as a departure from the original
 ideas of open form. The outspoken intention was to produce a first
 version that should be followed by several other. The main motivation
 for making a fixed version for the premiere was that the practical
 problems of an ``open'' score seemed to complex to handle for this
 once. And also, there was not yet a reliable technical solution for a
 dynamic and flexible electronic part, which forced us to use only
 pre-prepared material for the premiere.
 
In the beginning of the video clip, it may seem like the two parties
stay with their traditional roles: With a pair of scissors Henrik is
cutting bits of the score and editing the performance version. Stefan
is getting seated with the guitar, and starts to try out a
tempo. However, this is in fact a little misleading. The initial
discussion of the tempi emanates from Stefan's organisation of the
present version of the piece and specifically his choice to do a
certain section twice but in different tempi: first in half tempo and
then in its original tempo. Stefan tries two different slow tempi,
first MM=50 and then the actual half tempo, MM=40. The dialogue is
ambiguous, taken literally we seem to agree that MM=40 is a good
tempo, but it is clear from the discussion that the choice is actually
for the faster tempo. In the middle of this discussion, Stefan rises
abruptly and interrupts Henrik while he is about to cut the next bit
for the score:

\begin{quote}
  \emph{Stefan}: Yes, that's right here I had\ldots Here I did such a fresh
  thing that that eh, so that I only play the first chord.\\
  \emph{Henrik}: Oops! [\ldots]\\
  \emph{Henrik}: I cut it here.\\
  \emph{Stefan}: You can cut it slantingly and you get rid of this \ldots\\
  \emph{Henrik}: This was advanced!\\
  \emph{Stefan}: That's right, just like Pro Tools\\
  \emph{Henrik}: You need an authorised editor for this\\
  \emph{Stefan}: Look, isn't that nice!\\
\end{quote}

Apparently, at this point Henrik is not really in charge in the
editing of the score, but rather carries out the instructions in
Stefan's notes. This becomes even clearer later on in the clip. Stefan
finds that there seems to be something wrong in the editing. He picks
up the guitar and plays the section to clarify how it should
go. Henrik still doubts that the instructions are misunderstood:

\begin{quote}
  \emph{Henrik}: Yes, but  it says after C1\ldots new\\
  \emph{Stefan}: Yes, but the new C1, that's the little snippet that I made I
  called it ''C1 new''\\
  \emph{Henrik}: Yes, yes ($\infty$)\\
  \emph{Stefan}: That's why I was thinking the wrong way, we simply have to
  take this one off again, or wait, what does the A page look like,
  can we have it here too, no, it's too long, or, yes, in and for
  itself, but it is a good spot for a page turn, if you want to put
  that page away\\
\end{quote}

At some points, like in the present example, the cutting to pieces of
the score reorganised the material on a level more detailed than was
originally intended. In figure 3a and 3b we can see the snippet cut
out from C4, bars 16-17, and then as it was when instead inserted
after bar 13 of A1.
 

Figure 3a
 
Figure 3b


In the field of `Repetition' at this stage we identify the following
agents to have a major impact on composer and performer:
Authenticity-as-practice, which has been active ever since the
decision to work out a piece in open form. The practice from the early
phases of musical modernism has clearly influenced Henrik in the
phases leading up to the first performance score. Personal
Authenticity, which is a driving force in Stefan's decision to violate
the rules set up in this score and make a fixed version of the piece,
changing both the shape of the individual fragments as well as the
greater form. But Stefan is also working under the influence of
Authenticity-as-practice, being well aware of the modalities of the
historic practice when he chooses to go against the frame set up by
this authenticity. The score is central to the interaction in this
clip, but the agent of `the score' we do not only intend in this
instance the first performance score (which is literally being cut to
pieces in the video) but also the written instructions that Stefan
made of the intended order of the fragments. The action in the video
is mainly guided by these quite authoritative instructions as can be
seen in this excerpt:
 
\begin{quote}
  \emph{Stefan}: No wait, no but\ldots  No wait, now it's wrong!\\
  \emph{Henrik}: Is something wrong?\\
  \emph{Stefan}: This one should be before that\ldots according to my notes
  (Stefan picks up the guitar) So that, this is how it was If you do
  this\ldots (starts playing the intended sequence)

  \emph{Henrik}: That's right\\
  \emph{Henrik}: Yes, but it says after C1\ldots new\\
  \emph{Stefan}: Yes, but the new C1, that's the little snippet that I made I
  called it ``C1 new''\\
  \emph{Henrik}: Yes, yes ($\infty$)\\
\end{quote}

The blurred boundaries between composer and performer make this little
excerpt both entertaining and thought-provoking. Stefan claims that
something is wrong in the editing so far and goes back to again read
the instructions he has made. Henrik looks in the instructions and
questions that they should be misunderstood. Stefan picks up the
guitar and plays the passage, in order to verify that his reading of
the instructions is correct and that there is a mistake in the new
edited version. Convinced by the the sounding music Henrik starts
rearranging the score.
 
The instrument was an important agent in the process of making this
version, which can be clearly seen in the passage we are
discussing. The bar that has been cut up (and which I suggest to
Henrik he should cut slantingly) is chosen because it follows nicely
also in the movement of the hands: A1 ends on an F\# fingered
ordinario by the left hand. The new continuation to it is taken from
the second last bar of C4 (cutting out the resonance from a chord in
the preceding bar in C4), which is a chord including this F\# fingered
on the same string and a similar chord, plucked to the left of the
fingered pitches.

\subsection{The Second Version: Breaking the rules II}
\label{sec:second-vers-break-1}

The second version of Repetition was the result of an idea to combine
the guitar piece with a modernist film classic by the Swedish artist
Viking Eggeling: the silent movie Symphonie Diagonale (1924). Based on
an intuition, a vague memory of the structure of the film, Stefan
presented the idea arguing that the juxtaposition of the two entities
would create an independent work. Though initially very sceptical,
Henrik agreed on giving it a try. By itself the idea of combining two
totally unrelated works, separated in time by 83 years, created in
different media, is remarkably odd. However, the underlying motivation
was to also use the film as a springboard for the making of a
radically different version of the guitar piece. The first version of
Repetition had already been performed four times and it was at this
moment scheduled for another four performances. Since the radical
openness of the greater form of the piece is one of its
work-identifying features we wanted to make a second version that was
as distinct from the previous one as possible.
 
Figure 4a
 

The task that we set up was to create a new version of the guitar
piece that should derive its form from the film, but without making
changes to the music on a detail level. When we met on January 19,
2007 to try out the idea by actually starting to make a version to the
film, we had made no preparations other than obtaining a compressed
version of the film. We started out by watching the film and making a
spontaneous analysis of its elements. It turned out that the first
session would provide several positive surprises. The first one was to
find that it was possible to analyse the film by means of three
categories of images. These categories were assorted on sheets of
paper (see images 4 a-c). In the very beginning of the first clip we
see Henrik naming the materials. (The joke here is that the guitar
materials, named A, B and C respectively, also has corresponding
electronic materials that were unfortunately ALSO named A, B and C,
which has created a lot of confusion in the making of the first two
versions of the piece.) Since the guitar piece also consists of three
distinct materials it appeared as a given to map the three materials
of the film to one material each from the guitar piece.
 
Figure 4b: Note the change from B to A in the mapping. This swapping
of the A and B-materials was suggested by Henrik later in the session
on January 22.
 

The work we did on those days in January 2007 involved a great amount
of artistic choices that were all made in a continuous negotiation
between ``performer'' and ``composer''. Perhaps the most striking
aspect of these hours of video is the almost complete merging of the
agents into one composer-performer. Neither of the agents seems to
have the interpretative precedence, for instance in the first section,
in which we are searching the score for suitable spots with diminuendi
and crescendi.
 

Figure 4c
 

The very first decision we had to take was to break one of the most
outspoken rules of the first performance score, and one of the rules
that the version worked out in Hanoi had still been faithful to: that
of starting the piece with A1. It was quite obvious that it would not
be possible to stick with this rule if the form of the film should
direct the succession of the musical elements.
 
The clip we are discussing is a condensed version of the beginning of
the first working session, which took place on January 19. Some bits
of video with no or irrelevant material have been taken out. The
initial phase was in many ways most interesting, since it involved
many important decisions that were simply carried out in the sessions
to follow.
 
In the making of the second version, two new agents played an
important role: an audio recording of the first version and the
modernist film classic Symphonie Diagonale by the Swedish artist
Viking Eggeling. When we sat down to try out the idea of making a
version of Repetition to the film, we worked from the audio recording,
editing the recording into the film.

One reason for performer-composer collaboration to be bent towards
`complementary' patterns (John-Steiner, 2000) is the different major
artistic tools that the two agents interact with: the instrument and
the score. In the first stages of the present project, the interaction
followed this pattern quite clearly. In the making of the second
version a rather radical change in the collaborative mode can be
observed. We find that an important reason to this was the different
tools that came of use: the performer did not have an instrument, both
agents were reading from a finished score, the main tools were the
computer (which here functions as an instrument played by both) the
audio recording of the piece and the Eggeling film. Since both Stefan
and Henrik are well acquainted with digital editing, the scenario
changed into one where the tools were shared. Following from this, the
interaction moved into an integrative pattern. Both Henrik and Stefan
were now mainly working within the agency of the performer, in keeping
with the intended openness of the greater form of the piece, they are
active in the esthesic domain, mapping material from the score and the
recording to the film. Having said this, we can also look back at the
session in Hanoi and identify a move in the integrative direction
already with Stefan's overtaking of part of the composer's agency in
making the final organisation of the score. However, it is only in the
sessions in January that this change of pattern really comes out.
 
Even if much of the work was done straight from the recording, the
score was still quite important, especially in the stage when all the
strategies were worked out. For instance, the three structural levels
from the score---the A, B and C-materials---functioned as a
fundamental building block for this version. In other words, the
making of the second version was mainly guided by three `texts' the
score, the recording and the Eggeling film.
 
One need only look at the first section to see the importance of the
score in this mapping of the material. After having decided to work
from this fundamental strategy, the first decision we make is to
connect the opening images to a material, which is basically played on
the fretboard. (The reason to this is, rather simplistically, that the
images resemble a fretboard.) These three materials were mapped to
three materials identified in the film. The form of this version was
consistently worked out according to this simple formal
scheme. However, this strategy was combined with further related
formulas, for instance, the idea to allow for two distinct electronic
materials, for instance an A and a B material, to appear together with
a C-material in the guitar. This was called for by the fact that in
the film, all three visual materials at times appeared
simultaneously. As it is not technically possible to perform more than
one of the materials in the guitar part simultaneously (by one single
player) the only possible solution, in order to stick to the
structural idea of this version, was to modify the original rules for
the distribution of the electronic and acoustic materials in the
guitar piece.
 
When this comes up in the video, there is a clear difference in the
approach between Stefan and Henrik to how the rules governing the
first version should affect the version to the film. Stefan is
thinking that the original connection between the guitar material
discussed and the electronics that went with it in the first version
should be kept. But Henrik's reply indicates that this is not his
view. The mapping of electronic materials to guitar materials can be
independent of how it was done in the first version:
 
\begin{quote}
  \emph{Stefan}: That one was nice, but the last thing is a bit
  annoying\ldots that they continue to come\ldots\\
  \emph{Henrik}: Yes, but then we
  can have a soundfile go there.\\
  \emph{Stefan}: Yes\ldots well yes, of course,
  since that segment is by itself isn't it?\\
  \emph{Henrik}: In the performance version we've done it's by itself, yes!
\end{quote}

 
Even if the agency of composer and performer may be shared in the
collaboration between Henrik and Stefan, still they both stick to
their traditional roles as a kind of fundamental or drone during the
collaborative process. Even if Stefan several times decides to break
the rules of the composition, the above quote is an example of how his
intention is not to violate the structure that was intended in the
work. But is this a matter of a performer sticking to
Authenticity-as-Intention? Or is this only a deception caused by the
maze of negotiations that bit-by-bit builds the structure of a
work-in-motion? We find that the later is the case. In fact, Stefan's
understanding of these rules is actually not in accordance with the
original intention of the composer: It was Henrik's intention from the
outset, that the performer should be able to choose between either
combining, for instance, a C-material in the electronics with a
C-material in the guitar part, or to go against this rule and choose
another material. No combination of sound files and acoustic material
should be fixed in the piece. It was in fact only with Stefan's first
version of the score that the electronic materials also received fixed
positions. So in fact, when Stefan is hesitant to change the
connections between the two materials, he is rather influenced by
The-Other-Authenticity than by Authenticity-as-Intention!
 


Figure 5: Excerpt from version 2.


There are some examples of artistic troubleshooting in the video. To
no surprise, what frequently comes up is that formal features in the
film appear to be without apparent equivalent in the guitar
piece. These are problems that we respond to in different ways. In the
beginning of the session we first scan the score for glissandi that
could be used with the sliding visual objects that are an important
parameter in the film. Some suitable passages are quickly
identified. We immediately move to searching for possible
``diminuendos'' that could correspond to some processes in the film
with diminishing visual patterns. Although some of these problems are
also solved already in this first session, some solutions of course
came later. One of the ``diminuendos'' in the film, was in the end
solved on January 23 (Tape 1, approx 40-50 min) by cutting up some
guitar and tape phrases in the A and B materials (se fig 5). These
negotiations can be understood as a series of interacting esthesic and
poietic processes (Frisk \& \"{O}stersj\"{o}, 2007). But also the more
global artistic choices were guided by the same kind of oscillation,
and resulted in these two global strategies:
 
\begin{enumerate}
\item The initial formal strategy was to map the A, B and C materials
  to the three materials identified in the film. The electronic
  material was related to one of these materials each.
\item The second strategy is to allow for the electronic material to
  come from another category and hence, when the film has all
  materials at the same time, the electronic part can represent two of
  them and the live guitar the third.
\end{enumerate}
The modified strategies for how the electronic and acoustic sounds can
be combined, was a result of the demands from the film to have three
materials present at the same time. Interestingly enough, Henrik has
decided to allow for this kind of combinations also in the third
version of the piece. This is another example of how also the
fundamental rules for the piece have been modified while we have been
continuously working on it.

\subsection{The Third Version }
\label{sec:third-version-}

One of the influential findings in the studies preceding the work
Repetition was concerned with the general difference between
composer-performer interaction and performer-computer interaction. The
former is obviously bound to be more dynamic but the tolerance for the
high level of noise in the human communication was thought provoking:
Complete misunderstandings did not halt the process nor did it appear
to lead to false conclusions. If anything it seemed to be taken for
granted by both participants. It was only when looking back at the
video that the misinterpretations were identified. Our conclusion was
that what may be defined as `noise' from the point of view of one
analytical methodology must be considered an integral part of the
artistic communication in the context of the practice. Put
differently, it shows how the classical notion of the `creative
misunderstanding' really can play an important role in artistic
processes.

The approach to performer-computer interaction commonly taken is
however to minimize noise and to take precautions in order to avoid
it. One of the reasons for this is related to the general
understanding of a musical work in the Western Art Music tradition
(somewhat simplified): That any performer, digital or otherwise,
should follow the score exactly as written. Another reason is the
relative closeness of scientific areas such as signal processing and
information theory to electro acoustic music. In Claude E. Shannon's
influential article A Mathematical Theory of Communication (1948):
``The fundamental problem of communication is that of reproducing at
one point either exactly or approximately a message selected at
another point.'' [p. 379] But as we could see from the interactions
between Stefan and Love in Frisk \& \"{O}stersj\"{o} (2006a), in some
cases the message was not even approximately reproduced, it was
completely distorted, and yet it produced information.

There are a number of factors at play here and to simply say that
noise in the interaction may be have positive effects on the
communicative system is a too hasty conclusion. To begin with, as was
also concluded in both Frisk \& \"{O}stersj\"{o} (2006a) and Frisk \&
\"{O}stersj\"{o} (2006b), the subculture created by the participants
in the communication system provides a factor that reduces the entropy
of the system. The stylistic conventions of the music, the meta
knowledge about the collaborator, the other agencies involved and many
other sources help the collaborators to `understand' each other. To
attempt to model and take into account all of these constituents in a
performer-computer interactive system is not yet possible. But when
designing the interactive system it is possible to attempt to look at
the message passing as something more complex than a unidirectional
stream with the primary purpose of letting the computer know where the
performer is currently situated in time. On a technical level the
interaction in the first version (the second version is not
interactive at all in that sense---it has a prerecorded track with the
computer part) followed a traditional pedal-pressing paradigm: The
temporal alignment between performer and computer was controlled by a
pedal pressed by Stefan at times indicated in the score. The tight
synchronization of the acoustic and electronic materials has given the
interaction a certain character. These qualities have become integral
to the piece and in discussions we've had regarding the third version
it has been suggested by Stefan to keep the pedal as a complement to
other means of interaction. The musical timing may be grabbed by the
computer from the pressed pedal but the choice of event is governed by
other sources of information.

The third version is in many ways to be understood as the realisation
of the original ideas with the piece. However, since it has a long way
coming, it has also been strongly affected by the preceding versions
and by the research that have been performed by both Henrik and
Stefan, together and by themselves. As was mentioned above Repetition
was written to be used with the timbre tracking software
timbreMap. Based on a SOM (Self Organized Map) that can be trained to
respond similarily to two different sounds if they share the same
features provided these features were also part of the training
set. Described differently, and to connect to the discussion above,
effectively we can say that the use of timbreMap (or any other piece
of software that implements some kind of neural network) increases the
entropy of the system. timbreMap ``wants'' and expects the input to
correspond to the the training set. In practice it will inform the
system of which kind of material Stefan is playing at the moment (A,
B, or C) while the pedal will be used for metric synchronization. A
pitch tracker that analyzes the current input and extracts its
fundamental will be used for additional information. The final level
of interaction that will be added to this version is the real-time
processing of the guitar. For instance, the physical modelling
synthesis used in the electronics for the C material will be generated
in real time rather than prerecorded.

\section{From Open Work to 'work-in-movement' }
\label{sec:from-open-work-1}

The way Repetition was conceived of prior to its preparatory studies
and prior to the experiences gained from the two hitherto realized
version of it, it was an open work similar to other works by Henrik
(Det Goda/Det Onda for flute and computer, and Drive for Electric
Viola Grande and computer). It was manifested as building blocks
notated and left to the performer to organize. Looking back at Eco's
previous quote from ``The Open Work'' he would probably have called
already this early conception of it a work-in-movement: ``the auditor
is required to do some of this organizing and structuring of the
musical discourse. He collaborates with the composer in making the
composition.'' However, the way the process has evolved, informed by
the knowledge produced in the study performed on Stefan and Love, the
emphasis in the term work-in-movement has travelled from `work' to
`movement'. And the emphasis on the `composer' in the quote from Eco
(the auditor is interchangable but not the composer) should in our
case instead be put on the collaboration: The action is more important
than the object.

The work-in-movement that we imply is a literal construction kit, an
IKEA music to be assembled and re-assembled in a recursive process
that should be allowed to continue outside of the collaboration that
gave rise to it. Stefan and Henrik set it in motion but its
authenticity can only be derived from the movement. An open source
music that may be dismantled and reconstructed, added to and altered,
according to its conglomerate of participants.

\subsection{Documenting the 'work-in-movement': The annotated score. }
\label{sec:docum-work-movem-1}

In order to fully realize the notion of a work-in-movement the process
of documentation---or in musical notation terminology---the work
identifying instructions needs to be reconsidered. The way Repetition
as a work has evolved, a traditional printed score in musical
notation, no matter how detailed the written instructions are, would
be misguiding. The operative word in work-in-movement is precisely the
movement: the change. The difference that makes a
difference. Therefore the score will need to communicate the process
rather than the result. The musical notation will no doubt constitute
an important part of the documentation but it will not be sufficient
in itself.

As we have already discussed, apart from this more musicological
aspect on the notation of Repetition, there are also practical issues
regarding the notation and the score that need to be resolved in order
to fully allow for the work to continue to evolve. If the performer is
to be able to alter the order of the sections according to choices
made in performance the score needs to be adopted. But also if the
performer chooses to do what Stefan did in the first version, settle
on a form prior to performance, the `final' score should easily allow
for this. Finally, adding the third version discussed above---with a
real time interactive computer part---to the list of possible modes of
performance, the score needs to fully document and guide the performer
in how to set up the computer part.

Design considerations for the annotated score: 

\begin{itemize}
\item It should encourage the interpreter to engage in a collaborative
  process similar to the one we have gone through with the primary
  goal not to repeat or recreate our process but to find his or her
  own unique version.  
\item It should allow the performer to look up any part of the
  notation at any instant.
\item It should allow for the interpreter to easily create a static
  form that he/she can use in performance.
\item It should contain all the software and all the soundfiles needed
  to create an interactive as well as a non-interactive version of the
  computer part.
\item It should document important parts of our collaboration and
  allow for other performers to add important parts of their own
  collaborations with the work.
\end{itemize}
Realizing an annotated score fulfilling all of these design goals will
be possible making use of the libIntegra framework and the associated
IXD file format described in \citet{frisk-bull07}. In addition the
Integra `bundle' format (see
http://www.integralive.org/dokuwiki/doku.php/lib:bundles) will need to
be further elaborated and a IXD browser---a web application used to
access the information in a bundle---needs to be developed.

\section{Discussion}
\label{sec:discussion}

What kinds of questions can be answered by a research projects such as
Negotiating The Musical Work? We will in the closing discussion focus
on artistic matters of question and an evaluation of the possible
assets of collaboration between artistic researchers.

\subsection{Where did the work go?}
\label{sec:where-did-work-1}

The way that Repetition has evolved it has become more and more clear
that it is not a piece in the Boulezian tradition of open works. It is
more related to the kind of openness, characteristic of post-modern
works, for instance participation-based works on the internet. Not
only has the piece never been performed in the version in which it was
first notated. The rules set up for its realisation have been changed
and/or put aside with both of the two versions made of the piece so
far. And these changes have been suggested by the performer of the
work, not by its author. It is crucial to bear in mind that this
radical openness in the structure of the work was not intended from
the outset. It is rather the outcome of the collaboration, and a
result not only of the artistic collaboration in itself but, at least
as we analyse it, also a result of the research.


Figure 6: Timeline mapping the most important events in the research as well as the artistic work.

\subsection{Where did the research go?}
\label{sec:where-did-research}

When we first outlined the research project headlined ``Negotiating
the Musical Work'' we were unaware of the fairly radical conclusions
we would draw from it. Our intention was to study the low level
processes in the interaction between a composer and a performer. Our
critique of musical semiology and the mode in which we suggest that
the categories of esthesic and poietic action can be used in the
analysis of artistic processes (Frisk \& \"{O}stersj\"{o}, 2007) led
to the same observations as the empiric study on the collaboration
between Mangs and \"{O}stersj\"{o}: 1) Composition may be regarded as
a complex interaction between esthesic and poietic processes. 2)
Performers may similarly be said to oscillate between these two modes
of artistic activity.

It was an outspoken intention with the present project, to challenge
the traditional boundaries of the agencies of performer and composer
in the Western Art Music tradition. The research has involved video
documentation of the artistic work, providing material that lays the
artistic processes more in the open. As a result, not only do these
processes materialise for an exterior viewer/researcher, they also
become available for the artists involved. The artistic work has been
informed by the research and in some phases guided by the findings in
the reflexive parts of the research.

When we together reflect on our research, we find that an important
aspect is the possibility to oscillate between reflection on ones own
practice and observing the collaborator at work. This was striking
already in the Mangs-study: at one moment Stefan could take the lead,
and the research would then be based on reflections by a participating
artist, in the next it could be Henrik, and then the point of view
would be that of the exterior researcher. Within the collaboration on
Repetition, this oscillation was much quicker and both parties had
both of these roles. Our interaction as researchers has been in the
complementary mode ever since the beginning of the project. We do not
find that there has been any dramatic changes in the way we interact,
but for obvious reasons the rather intense work and travelling
together in the past two years has little by little turned this
collaboration into an increasingly important work for both of us. Both
in the analytical stages and in the writing phases our interaction has
been fairly intense, with a rich exchange of ideas and material. One
of the more interesting findings--a direct result of the method--is
that many patterns in the creative work as well as other aspects of
the interaction, has emerged only when we have returned to the video
recordings. For instance, it was only with the aid of the video
documentation that the process from which the scordatura was derived
could be reconstructed.
 
The project has involved experiments with our respective artistic
practices and reflections on the documentation of this work. One may
also say that the research has been guided by the artistic work in
some respects. For instance, it was not part of the research
methodology to put the work produced in counterpoint with a film and
investigate the outcome of this experiment. In contrast to the
preceding studies, in the present study one may say that the main
result was in the artistic domain. That is, in the first two studies,
we arrived at several conclusions that came to affect the artistic
work. We also developed a terminology and a theoretical approach to
studies of composer-performer interaction. The results of this study
are best represented by the different versions of Repetition, perhaps
taken together with a reflexive description of the processes leading
up to the resulting versions.

How did Repetition end up a radical kind of work-in-motion? Was this a
result of the research also? The research provided a framework in
which we could reflect on our respective practices. This was one of
the major features of the Mangs study. Already at the outset of that
project, the idea was to lay the ground for a new work for guitar and
electronics by Henrik. And when the actual compositional work started
we had drawn many conclusions about the possibilities that could lie
in extended collaborative work between a composer and a performer. It
is only in rear view that the significance of the initial decision, to
follow the implications from the Mangs-study and deliberately welcome
the 'swapping-of-the-roles', came out. In our understanding it is due
to this decision that Repetition evolved into the radical
work-in-motion that it came to be. It was in the collaborative process
that the identity of the work evolved from a classical post-boulezian
conception into a more post-modern work-kind.

%%% Local Variables: 
%%% mode: latex
%%% TeX-master: "../Repetition"
%%% End: 
