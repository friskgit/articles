\newif\ifpdf
\ifx\pdfoutput\undefined
\pdffalse % we are not running PDFLaTeX
\else
\pdfoutput=1 % we are running PDFLaTeX
\pdftrue
\fi

\documentclass[a4paper]{article}
\usepackage[swedish, english]{babel}
\usepackage[T1]{fontenc}

\ifpdf
\usepackage[pdftex]{graphicx}
\renewcommand{\encodingdefault}{T1}
\renewcommand{\rmdefault}{pad}
\pdfcompresslevel=9
\else
\usepackage{graphicx}
\fi

\usepackage{fancyhdr}
\pagestyle{fancy}

\lhead{\small{\textit{Frisk, �stersj�}}}
\chead{}
\rhead{\small{\textit{Negotiating the Musical Work}}}

\title{Negotiating the Musical Work. \\An empirical study on the
inter-relation between composition, interpretation and performance}
\author{Henrik Frisk\\{\small PhD candidate}\\{\small Malm� Academy of
Music - Lund University}\\{\small henrik.frisk@mhm.lu.se} \\ 
Stefan �stersj�\\{\small PhD candidate}\\{\small Malm� Academy of
Music - Lund University}\\{\small stefan.ostersjo@hotmail.com}}

\date{29 January 2006}

\begin{document}
\selectlanguage{english}
\maketitle

\thispagestyle{empty}

\begin{abstract}

In this paper we intend to explore the inter-relations between performer 
 and composer, in the form of a theoretical study, which is intended to
 lay the ground for a new work for guitar and computer by Henrik Frisk
 for Stefan �stersj�. This project is part of our respective artistic
 research projects at the Malm� Academy of Music. 

We find that the ontology of the mixed media work (in lack of a general
 terminology we use the term 'mixed media' in this article to refer to a
 work for instrument(s) and electronic sounds) is closely related to the
 general discussion of score-based works, but it is important to bear in
 mind that the programming of the electronics should also be regarded as
 notation in this discussion and the electronic part is itself another
 object of interpretation for the performer.

From a hermeneutic point of view, performance interpretation of music is
 a special case \cite{levinson,davies,stecker}\nocite{krausz}.
\begin{quote}
If performances and critical interpretations are both representations of
works, they are so in quite different senses. If we ignore these
differences, we can easily be misled to make invalid
inferences. Performances are necessarily constructive; that is, they
necessarily add features that the work leaves vague or
undetermined.\cite{stecker} 
\end{quote}

But not only in cases in which the notation is in some respect unclear
or vague is there a call for constructive elements in
interpretation. Construction is really at the heart of the matter in
performance interpretation. When elements of electronic real-time
 processing, sound synthesis, sound file playback or be it any other
 means of producing electronic
 sounds, are part of the work, yet another level of complexity is added to the issue
 of interpretation. This is closely related to the notion of
authenticity, which is already a powerful factor in the performance of
score-based works. How is this issue to be aproached in mixed media
 works? If the programming of the electronic part is to be regarded as
 a special case of notation, how is this notation 'transcribed' and
 communicated to the performer?

It has been suggested by several writers that performances should be
regarded as works in their own right. From this point of departure,
Peter Kivy \cite{kivy} distinguishes between four kinds of authenticity
in performance, most importantly; 'Authenticity as Intention' (that is,
authenticity as faithfulness to the composer's performance intentions)
and a fundamentally opposed notion of authenticity, which he simply
names 'The Other Authenticity'. Kivy's intention is to
bring out an opposition between these two authenticities: On the one
hand, the obligation to conform with the composer's intention and, on
the other hand, the creative originality that should result in a work in
its own right. But, unlike Kivy, we suggest (as has been done by Stefan
�stersj� and Cecilia Hultberg \cite{ost05}) that the opposite may also
be claimed: The tension between the two imperatives on the performer
makes up a creative field in which truly original instances of works
come out. The different forms of authenticity turns out to be the
definition of a creative field of tension in which the composer and the
performer negotiate and interact towards a version of a work.

A close collaboration between a living composer and a performer, allows
for discussions on the rendering of a mixed media work. We find that this
process could be described as a negotiation towards a version of the
work \cite{kivy} and that this can serve as a model for understanding
similar processes even prior to the existence of any notated
material. This article aims at a closer understanding of the
significance of these negotiations, specifically their meaning in
 relation to the two modes of musical representation; the notation and
 the sonic trace left by the computer part. Further, we aim at creating a broader
platform for reflection on and analysis of our respective artistic
practices.

We approach this complex area empirically by analyzing selections of the
 video documentation of �stersj�'s collaboration with composer Love
 Mangs, as well as versions of a certain passage in a piece for harp and
 computer by Henrik Frisk. Love Mangs' ``Viken'' is a work for guitar,
 banjo, e-bow and real time processing in Max/MSP. In the video material
 we find striking examples of how the traditional roles of "composer"
 and "performer" are exchanged, indicative of the difficulty to
 establish fixity in the definition of these seemingly discrete artistic
 fields. Following Stecker \cite{stecker} we would like to suggest that
 there is always an element of construction in a performance of a piece
 of music, but we also find that there are elements of interpretation in
 the act of composition. By using Molino's and Nattiez' terminology for
 a semiological analysis of music we attempt at describing the interplay
 in terms of poietic and esthesic processes, \cite{nattiez}
 \cite{molino} and with semiotic terminology in general.

\end{abstract}


\bibliography{bibliography}
\bibliographystyle{apalike}
\end{document}