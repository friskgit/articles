% This file was converted to LaTeX by Writer2LaTeX ver. 1.0.2
% see http://writer2latex.sourceforge.net for more info
\documentclass[letterpaper]{article}
\usepackage[latin1]{inputenc}
\usepackage[T1]{fontenc}
\usepackage[swedish]{babel}
\usepackage{amsmath}
\usepackage{amssymb,amsfonts,textcomp}
\usepackage{color}
\usepackage[top=2.54cm,bottom=2.54cm,left=2.54cm,right=2.54cm,nohead,nofoot]{geometry}
\usepackage{array}
\usepackage{hhline}
\usepackage{hyperref}
\hypersetup{pdftex, colorlinks=true, linkcolor=blue, citecolor=blue, filecolor=blue, urlcolor=blue, pdftitle=Rethinking Essay.docx, pdfauthor=Henrik Frisk, pdfsubject=, pdfkeywords=}
% Footnote rule
\setlength{\skip\footins}{0.119cm}
\renewcommand\footnoterule{\vspace*{-0.018cm}\setlength\leftskip{0pt}\setlength\rightskip{0pt plus 1fil}\noindent\textcolor{black}{\rule{0.25\columnwidth}{0.018cm}}\vspace*{0.101cm}}
% Pages styles
\makeatletter
\newcommand\ps@Standard{
  \renewcommand\@oddhead{}
  \renewcommand\@evenhead{}
  \renewcommand\@oddfoot{}
  \renewcommand\@evenfoot{}
  \renewcommand\thepage{\arabic{page}}
}
\makeatother
\pagestyle{Standard}
\title{Rethinking Essay.docx}
\author{Henrik Frisk}
\date{2013-02-22}
\begin{document}
\clearpage\setcounter{page}{1}\pagestyle{Standard}
\section[(Re)thinking Improvisation: from individual to political listening]{(Re)thinking Improvisation: from individual to political listening}

\bigskip


\bigskip

When we designed the International Sessions on Artistic Research and the accompanying festival under the same title as the research project,\textit{ (re)thinking improvisation}, we imagined the event to be a venue for summing up and bringing together various conceptual approaches to improvisation. The aim of the three year research project had been to examine three primary aspects that we found to be inextricably bound up with improvisation, yet often not or only implicitly included in its study and description:\footnote{ This is not to say that these aspects have never been seriously considered in previous research. As for interaction, a notable contribution on the interaction between jazz musicians is Ingrid Monson{\textquoteright}s book {\textquotedblleft}Saying Something: Jazz Improvisation and Interaction.{\textquotedblright} (Monson 1996)\par }


\bigskip

{}--\textit{ Interaction}

{}-- Attentive\textit{ listening}

{}-- Musical\textit{ freedom}


\bigskip

By interaction we intended the social nature of performing as well as the ways in which cultural tools such as musical instruments and other artifacts function as agents in the musical discourse. Attentive listening could be understood as a somewhat Adornian conception of active and informed musical hearing. In what ways does the performing musicians{\textquoteright} active listening inform the ways in which an improvisation unfolds?


\bigskip

However, the outcome of the various subprojects presented a rather complex picture. Studies of the multimodal interaction between improvising performers have indeed been an important strand in the research. But the findings also indicated how the seemingly distinct aspects bleed together. For instance, the most prominent feature of interaction that emerged from the studies was analyzed as different modalities of listening. Hence, the category of attentive listening came to be understood as integral to the interaction between performers. In similar ways, our understanding of the concept of musical freedom pointed more to the function of constraints and conventions than towards any magical entity of artistic freedom or divine inspiration distinct from non-improvised musical forms and traditions.

\subparagraph[Already at the end of the sessions, and even more so when reviewing the submitted papers, as well as the performances and the video from the lab{}-sessions, the outcomes of the event underlined the complexity of the issues at hand. Within the frame of the festival, forms of knowledge production that combine artistic and conceptual ways of thinking were brought together and rather than a conclusion, the materials afford new questions, emphasizing the fluid nature of improvisation as artistic practice. The point of this short introductory essay is not to provide answers but rather to approach this vast material by establishing a meta{}-level perspective through questions such as:]{\textmd{Already at the end of the sessions, and even more so when reviewing the submitted papers, as well as the performances and the video from the lab-sessions, the outcomes of the event underlined the complexity of the issues at hand. Within the frame of the festival, forms of knowledge production that combine artistic and conceptual ways of thinking were brought together and rather than a conclusion, the materials afford new questions, emphasizing the fluid nature of improvisation as artistic practice. The point of this short introductory essay is not to provide answers but rather to approach this vast material by establishing a meta-level perspective through questions such as:}}

\bigskip

\subparagraph[What is the signification of {\textquoteleft}rethinking{\textquoteright} improvisation?]{\textmd{What is the signification of {\textquoteleft}rethinking{\textquoteright} improvisation?}}
\begin{itemize}
\item Doesn{\textquoteright}t rethinking imply definition and how can we define the art of improvisation when the practices are so divergent in different traditions?
\item Are there universals in improvised music and if so, how can they be defined and discussed?
\item What is the impact of the theory/practice divide in improvisation?
\item What is the relation between improvisation in everyday life and in artistic practices?
\item What is the social and political significance of improvisation in artistic practice?
\item In what ways can improvisation guide and inform philosophical enquiries?
\end{itemize}

\bigskip

The structure of the publication is similar to that of the lab sessions during the event. Although the grid makes for an efficient organization of the texts, and although we use the same headings in this essay, our attempt has been to redraw the map as it were. Rather than reading the texts as expressions of the themes of the event we are looking for intertextual connections between the ideas presented in the articles and organize these according to the headings. By re-reading the individual contributions in this way, establishing conceptual connections between the practices and modes of thinking present during the event, we believe that a possibility to begin the process of rethinking improvisation is at hand.

\subsection[(Re)thinking audience and group interaction\ \  \ \  \ \  ]{(Re)thinking audience and group interaction\ \  \ \  \ \  }
The study of social interaction from within a musical artistic practice is one of the broadest and most interesting fields within artistic research. The complex and multimodal means of communication between members of a musical group in performance may have much to say about human communication in general. It is also one of the fields most commonly referred to by both social and computer sciences as an area that to a great extent could inform their own scientific research. The perhaps somewhat na�ve and troublesome, but nonetheless envious, view on the symphony orchestra as an ideal social organisation where a large group of individuals are all following the authority of one leader without losing their own individuality and space for personal expression is the wet dream for any organisational strategist\footnote{ See Bennis \& Nanus (1985), Druker (1988) Traub (1996) for some examples of this romantic view on communication within the orchestra.}. The intricate and lucid interaction between the members of a string quartet is another and less authoritarian example that is often brought up as an ideal form for communication in a small format. However, in reality, musical interaction harbors questions and competences that go far beyond the narrow logic of corporate efficiency and efficiency may not even be the main objective for musical interaction, certainly not for improvisational interaction. Due to its multimodal nature, depending on what angle we are looking at it from different shapes and patterns may emerge.


\bigskip

Marcel Cobussen touches on one of the key aspects of interaction in his description of a performance with the percussionist Han Bennink, that of the function of listening as a multimodal way of connecting individuals and material in the resonance of the sounding music, in the bodies of the performers and listeners and in the space:


\bigskip

The contact between wood and metal strings or wooden sticks led to new acoustic experiences; the contact made the floor sound, a giant surface with an enormous potential of different pitches, volumes, and timbres. In Deleuzian terms, one could say that Bennink formed a rhizome with the floor.


\bigskip

The importance of the in-performance listening as distinct from common listening is also stressed by Johnson and Schwarz: {\textquotedblleft}You should also learn to put yourself in zero position, to be here and now and listen to what \textit{is{\textquotedblright} }(Section 5:4).


\bigskip

Also the furthered abstraction of graphic notation holds a related possibility of modifying the musical-social communication and interaction as is pointed out by Helen Papaioannou (Section 1:5), specifically referring to the repetitive flows of motion that are an integral part of the scores that she is presenting. It is one of the important aspects of her composition \textit{Cogs}, and in the lab-session and the following concert performance (released in the CD-compilation in the present box), the development of specific modes of interaction, related to the design of the scored materials, took shape.


\bigskip

Several events and presentations during \textit{(re)thinking improvisation} explored the interactions between an audience and virtual or spatial contexts. Ingrid Cogne{\textquoteright}s opening installation Boule (Section 1:1) became a parallel to Magali-Ljungar{\textquoteright}s presentation of her concept of {\textquoteleft}virtual reality arts play{\textquoteright} (Section 5:2) .


\bigskip

Gerhard Eckel (Section 1:3) puts the light on human-machine interaction mediated by sound in his installation \textit{Random Access Lattice. }By exploring the natural tendency for human curiosity the players that engage with his hand held instrument and virtual grid are naturally guided towards {\textquotedblleft}readiness and dexterity{\textquotedblright}, possibly leading the way into an outright improvisation. This is not unlike the way computer games are constructed taking the will or intention of the user as an agent in the interactive interface. The relatively complex but speech like sounds that the installation is generating triggers a wish to understand and motivates the user to explore.


\bigskip

The relations between freedom and constraint, as well as between improvisation and composition was also highlighted in the conducted performances during the festival: most notably in the rendering of parts of Cardew{\textquoteright}s \textit{Great Learning}, directed by John Tilbury, as well as in the performance of Gino Robair{\textquoteright}s opera \textit{I, Norton }(Section 4:2) with the Swedish group \textit{Operaimprovisat�rerna} and in the concert and the lab session with different versions of Ture Larsen{\textquoteright}s \textit{Beslutningens anatomi} with Ensemble Ars Nova and guests from around the world. The difference between a sign inscribed as visual gesture in space and as written text is questioned, while a different question also


\bigskip

In the on-going discussion on methods for artistic research Karin Johansson has experimented with auto-ethnographic processes: {\textquotedblleft}auto-ethnographic case study with the aim of obtaining a close-up picture of how I in my musical practice related to the expansive approach{\textquotedblright} (Section 1:4) Although most artistic research projects has probably dealt with some form of auto-ethnography, albeit without calling it thus, Karin{\textquoteright}s recollection of her failed attempts to interview herself puts the focus on the difficulty of the issue of method.


\bigskip

However, the {\textquoteright}failure{\textquoteright} with the self-interview led to a further discovery of what she came to call {\textquoteright}The Pit{\textquoteright}, a state of artistic practice when the subjective sense of the ongoing work is highly negative. She claims that some of the most creative moments in the course of this collaborative research project were characterised by such feelings of vulnerability and insecurity that can be understood as signals approaching the outer limits of ones comfort zone. This of course has a strong bearing on the topic of group interaction, pointing beyond a strife for mutual agreement and a search for flow in the interaction.


\bigskip

Gaelyn and Gustavo Aguilar describes a related but somewhat reversed approach in one of the early projects carried out by the Tug Collective. Defined as a performative ethnography, \textit{Ah, Raza! The Making of an American Artist }uses the theory and methodology of improvisation {\textquotedblleft}to encourage dialogue and action that will seed and extend deeper readings of these themes of recuperation and expansion [...] that people along the U.S./M�xico border are immersed in{\textquotedblright} (Section 2:1). Improvisation is the {\textquotedblleft}crucial unifying component{\textquotedblright} as much as a {\textquotedblleft}critical compositional element{\textquotedblright} and it is rooted in the listening self. The collaborative practice of Tug shows us that the impact of improvisation is not limited to well defined formations as audiences and groups, but may well be free-range, open and non-hierarchical (ibid).

\subsection[(Re)thinking composition and interpretation]{(Re)thinking composition and interpretation}
The recurring question of the relation between the practices of composition and improvisation tends to evoke discussions of definition and demarcations. A starting point when designing the sessions in this event was the conviction that the socially constructed understandings of these concepts are highly diverse in different cultures. Local definitions of these entities must then be understood as provisional and certainly not as absolutes. Sandeep Bagwaati argues along the same lines for a fluid conception of improvisation and composition:


\bigskip

In practice, scores define what is considered context-independent by the composer or by a certain cultural tradition. No score will thus ever totally determine all aspects of a musical performance: some elements of music making will always be contingent - and thus improvised. (Section 3:1))


\bigskip

Further, and this was the logic behind the notion of rethinking composition and musical interpretation, it was assumed that the function of interpretation may be understood differently than as expressed in the common usage of the term as a performance of a score-based work, or for that matter, the making of an analytical interpretation of such a work (Levinson 1993, �stersj� 2008). 


\bigskip

Karin Johanson (Section 1:4) discusses how organists leaning on the liturgical tradition are part of a pre-modern conception of the musician{\textquoteright}s labour. Rather than emphasizing originality they {\textquoteright}adopt the given system, adapt to it and perhaps develop it. Notions of copyright or musical ownership do not exist. Musicians in pre-capitalistic times did not own their compositions and musical works were not seen as having legal identities.{\textquoteright} Hence, musical production within this discourse relates differently to the notion of the musical work.


\bigskip

Musical notation has been used for many different purposes over time and throughout different cultures, creating manifold interrelations between the act of performance and the score (Butt 2002). In Western culture, we have had a strong bent towards thinking of performances as interpretations of a work. But can the opposite also be the case? Can a score be an interpretation of a work? This is indeed how Luciano Berio conceptualizes the final movement of his seminal work {\textquoteleft}Sinfonia{\textquoteright}: {\textquotedblleft}this fifth part may be considered to be the veritable analysis of Sinfonia, but carried out through the language and medium of the composition itself (Berio 1986:6). 


\bigskip

Similar approaches emerged in Anne Douglas and Kathleen Coessen{\textquoteright}s Calendar variations (Section 1:2), at times the original score faded away, leaving place for novel and unique creations. In composition and interpretation alike, various texts emerge as fundamental in the staging of experimental situations. These texts may be scores the traditional sense but can just as well be any other kind of artifact or construct. For instance {\textquotedblleft}in the case of a piece for instrument and electronics, much of the identity of the work is also specified in the computer programming and in the electronic sounds{\textquotedblright} (Frisk \& �stersj� 2006, p. 247). Gerhard Eckel discusses the function of the software in the process of composing the sound sculpture Random Access Lattice, a piece that was displayed during the event:


\bigskip

Evolving the software becomes part of the experimentation and is thus subject to the serendipity and contingency typical for improvisation. During the compositional process, the sculpture functions as the main epistemic object in the experimental system. The epistemic object transforms into a technical object, a black box, once this process has terminated and the piece is finished. Therefore, composition through improvisation may be qualified not only as a po�etic and aesthetic but also an epistemic practice. Improvisation engenders knowledge in the compositional process. (Section 1:3)


\bigskip

Through these multiple perspectives on improvisation and composition it may be concluded that interpretation, improvisation, composition and the musical {\textquoteleft}work{\textquoteright} are fluid but closely interrelated concepts. While definitions may then become more of local, and often narrowly political, statements, a study of the how these concepts interact and bleed into one another appears to us as a way to begin reconsider some of the fundamentals of Western art music which can be thought of as a beginning towards what could become a rethinking of improvisation. \ 


\bigskip


\bigskip

\subsection[(Re)thinking instrumental and computer interaction]{(Re)thinking instrumental and computer interaction}
There is an obvious asymmetry between improvising on traditional instruments and improvising with and on computers. Compared to the conceptual stability of traditional instruments, such as the violin or the piano, the computer is in a constant flux. Updates of both computer hardware and software, along with the relatively short life span of technology, points to the difficulty in developing expert skills and solid performance practices in the domains of interactive electronic music. Furthermore, there is an important conceptual difference between traditional and computer based instruments. Whereas minute control of low level parameters such as vibrato, attack and dynamics is second nature for most instrumental performers the same can be very daunting on computer instruments, partly due to the lack of solid, general and meaningful interfaces. On the other hand the computer{\textquotesingle}s aptitude for control of higher level processes such as form or algorithmic development is unparallelled. In the (Re)thinking sessions many different kinds of computer interaction were presented, from laptop performers such as Jakob Riis and some of the members of the \textit{Lemuriformes} group through the mediated manipulation of Diemo Schwartz and Victoria Johnson and direct manipulation of Cl�o Palacio-Quintin{\textquotesingle}s {\textquotedblleft}hyperflute{\textquotedblright} over to more visually oriented interaction schemes.


\bigskip

In his paper, James Gordon Williams (Section 6:1) points to the feedback between the system consisting of his own improvisational practice and the technology that he explores. One may even go further and argue that a general property of artistic research is that it uses the practice of the researcher to inform the research questions and outcomes. In the case of musical interaction involving technology the feedback between the practice, the research and the technology is an important property that was approached by several of the presenters. While it is true that the technology will by necessity influence the practice, one may also argue that the ways in which musical practice can change our understanding and alter our view on technology is a key issue for this kind of research.


\bigskip

Live coding\footnote{ Live coding is a particular form of electronic music where the computer musician creates the sound generating processes on the fly, in real time. The computer screen is projected on stage for the audience to see the commands, or objects, that constitute the {\textquotedblleft}patch{\textquotedblright}.}, as was presented and demonstrated by the \textit{Lemuriformes}, is a practice in which the feedback between the different modes of expression is particularly important. The Lemuriformes uses the graphical programming environment Max/MSP and through their performance the meaning of the objects placed on the screen is created and recreated in the course of the performance. This kind of signification, however, is actually not particular to computer interaction or live coding but reminds us of Bennink{\textquoteright}s reterritorialization of the floor into a \ becoming-instrument. In the Lemuriformes performance the small rectangles on the screen, further manipulated by the graphic artist redrawing them on the screen, are not meaningful as musical actors until they are experienced as such in the performance. Or, as put by Cobussen, the different actors involved (floor, percussion player, screen, objects) {\textquotedblright}(in)form and create one another{\textquotedblright} (Section 2:4)


\bigskip

In the networked performance with Alan Courtis, Bennett Hogg, Victoria Johnson and Christopher Williams, that (paradoxically) took place in the \textit{(re)thinking composition and improvisation} session the latency in the connection was obvious and the video in the present publication displays how, in the conversation with Alan after the performance, the audience is highly amused by this technical failure. However, none of the participating musicians found the latency to be a problem in the performance. This is analogous to the way Robinson discusses networked performance in his paper: {\textquotedblleft}While we might assume that such latencies restrict or prevent fundamental potentials in improvisation, I contend something of the opposite. Instead, improvisative methodologies are especially poised to make creative sense of latencies and it should come as no small surprise that improvisation weaves prominently in much telematic music{\textquotedblright} (Section 1:6). 


\bigskip

A further discussion, which was not addressed in the sessions, is how telematic performances relate to the multicultural and post-colonial perspectives of society. Indeed, this may be the most significant contribution of this growing performance culture in the way it promises to erase earlier conceptions of centre and periphery. However, we must also bear in mind that in the present day, networked performance is still exclusive for regions of the world in which computers and internet access can be obtained. In this sense, telematic performance may be seen as a reminder of the classical dividing line between those who have economic power and those who do not. 


\bigskip

\subsection[(Re)thinking the actors, vectors and factors of improvisation\ \  \ \  \ \  \ \ ]{(Re)thinking the actors, vectors and factors of improvisation\ \  \ \  \ \  \ \ }
One of the lab-sessions was designed differently to the rest, since it contained a round table discussion of philosophically oriented papers on improvisation. Headed by Marcel Cobussen, the outcome of this discussion was less theory-laden than could be expected. One may say that what brought the contributions together was a focus on the human sensibilities, towards listening, and how, following Cobussen{\textquoteright}s outline of the factors that are at play in any improvisation, how we listen to the other musicians, instruments, audience, technicians, musical or cultural background, space, acoustics and the technology. Contrary to the industrialized and modernist view on society, that line of reasoning can perhaps be further expanded by looking at improvisation as an important factor of life itself. Douglas and Coessens, for example, points to three particular characteristics of human life from which improvisation emerges: unpredictability, unrepeatability, and social complexity (Section 1:2). These requires every individual to continuously employ improvisation in an ever changing manner. Anders Ljungar-Chapelon similarly brings up the unpredictable nature of daily life activities in an attempt to position the concept of improvisation, and asks rhetorically if driving a car is not an enterprise that requires the driver to improvise (Section 3:3).


\bigskip

David Linnros (Section 2:5), however, differs from this point of view and claims that the starting point for any improvisation must be to intentionally depart from everyday modalities of perception in order to achieve the specific focus on the now which is the (impossible) goal for every instance of artistic improvisation. The claim that it is the intention rather than the action that lies at the center of improvisation shuts out the act of improvisation from the daily life. Linnros sees musical improvisation as an activity trapped in the {\textquotedblleft}continuity of coming{\textquotedblright}, distinct to the temporal multiplicity of everyday life. However, an important distinction between Douglas \& Coessens and Linnros is that while Linnros looks at improvisation from the point of view of the improviser, from the inside looking out, Douglas and Coessens{\textquotesingle} perspective is rather from the outside looking in. Perhaps this fact contributes to the difference between their conclusions? But what is then the interrelation between improvisation in everyday life and the arts?


\bigskip

Erik Rynell claims that {\textquotedblleft}the way meaning is produced within theatre is to great extent analogous to how meaning is produced in real life{\textquotedblright} (Section 4:1). Similarly, Douglas and Coessens identify a series of principles of improvisation in everyday life that are common to improvisation in artistic practice. Furthermore, sensitivity as a factor in musical improvisation is a recurring thread in the discussions held during the conference, but may also be seen as a condition, or an actor, that initializes and fuels improvisation. Erik Rynell, however, points out that sensitivity is also an important aspect of theatre and thereby establishes a link between the research of the relations between {\textquotedblleft}words and actions in a given context{\textquotedblright} and the aesthetics of sensitivity.


\bigskip

Sensitivity is also an agent of attentive listening, something which Linnros refers to as ascultation, or {\textquotedblleft}the listening at the stethoscope{\textquotedblright} (Section 2:5). It is to put ones attentive focus to the {\textquotedblleft}inner rhythm of a thing{\textquotedblright} while keeping and holding on to the past in a Husserlian retention; to hear the rhythm of a flow. Listening is also always about action. Magnus Andersson highlights this aspect, asserting that it is only by way of action and the change that it achieves that {\textquotedblleft}we can hear what is going on{\textquotedblright} (Section 2:2). Such a process-based understanding bypasses the Cartesian split: {\textquoteleft}For an analysis to grasp what is at stake in an improvisation, it must acknowledge this corporeal thinking.{\textquoteright} That the body is a thinking and obviously sensitive actor that has an operative impact on the practice of improvisation is put forth by Bjerstedt (Section 2.3), Ostrowski and Cremaschi (Section 5.3), as well as Williams (Section 3.4).


\bigskip

Where we often speak of listening as a way of approaching the other Linnros brings in listening as an {\textquotedblleft}openness towards an outer duration{\textquotedblright} (Section 2:5) in an attempt to tune ones mind to a movement in turn related to Bergson{\textquotesingle}s discussion and definition of intuition. The limited attention span in the highly demanding structural listening is, according to Linnros, closely related to the way Bergsonian intuition requires the intellect to not get in the way, to not let it {\textquotedblleft}interrupt the listening with the incision of abstraction{\textquotedblright} (ibid). Linnros continues by offering a definition of improvisation as this kind of listening coupled with the act of creation and adds that {\textquotedblleft}a Bergsonian concept of improvisation would be nothing but the Bergsonian concepts of perception, listening and creation put together{\textquotedblright} (ibid). Thanks to Bergson, argues Linnros, we have access to tools and concepts with which we can approach improvisation, despite its unique and elusive character.


\bigskip

Marcel Cobussen instead identifies the singularity of the practice by stating that each improvisation is a work of its own, with its own particular prerequisites for study, and its own particular relations. For these reasons, rather than determining a set of general concepts for the study of improvisation, each time {\textquotedblleft}the interactions between minds, bodies, and environment need to be investigated anew{\textquotedblright} \ (Section 2:4). \ It is not the similarity in method (to improvise) that connects the results, it is the \textit{divergence} in the outcome, despite the conceptual similarities, the unifies the differing improvisational strategies.


\bigskip

\subsection[(Re)thinking idioms, conventions and tradition]{(Re)thinking idioms, conventions and tradition}
Although it is true that any improvisation is in a unique space of its own, there are more or less similar contexts for improvisation and these can still be analyzed from similar points of view. A greater part of the concert performances during \textit{(re)thinking improvisation }consisted of encounters between musicians from different cultures and from different stylistic paradigms. The festival audience and the participating artists and scholars could follow new collaborative performances between musicians from traditional music in three continents as well as from different stylistic directions in jazz and free improvisation. The formally structuring components could be a traditional tune, a composed score or a conductor working with specific signals for structuring free improvisation. At the heart of the matter in all such collaborative experiments are the local negotiations of musical meaning. These negotiations take place, not only in discussions during rehearsals but even more in the course of performance. Also, these negotiations were at the heart of the lab session in which, for instance, Sandeep Bagwaati engaged in a conversation with P�r Moberg after P�r{\textquoteright}s performance of what he and his co-musicians conceived of as a {\textquoteleft}Nordic raga{\textquoteright}. On the one hand, the challenge in cross-cultural exchanges is to create situations of mutual learning in which the individual traditions are treated with sufficient knowledge and respect, and on the other hand, these encounters demand a constant acknowledgement of the fluidity of cultural and personal identities and a willingness to engage in negotiations that redefine the material and psychological tools that contribute to our musical and cultural identities.


\bigskip

There is a striking connection between this experience of musicians from different cultures and that of the artists involved in the Calendar Variations project mentioned above and how they chose to


\bigskip

yield meaning through the coming together of different viewpoints within a shared territory, encountering the other in all its social, ecological and artistic facets. As we questioned our experience, new thoughts, ideas, emotions, possibilities enter in, enriching that experience through seemingly inexhaustible paths and trajectories. (Section 1:2, p 4) \ 


\bigskip

One may say that the common artistic strategy was to explore difference as a parameter in creative collaboration: to allow the friction between different modes of expression to become a vehicle for artistic innovation. 


\bigskip

Renewal can also come from within a tradition, when performers adopt novel perspectives on their own practice, such as expressed in the performances and the writings of Susanne Rosenberg and Olof Misgeld: {\textquotedblleft}To actually play in the tradition means to create one{\textquoteright}s own variants - in dialogue with the tradition{\textquotedblright} (Section 6:2) This is in line with the studies by Rosenberg on the inevitable change that happens as a consequence of songs passed along orally. A shared element in their accounts is the conviction that the transmission of a tradition is manifested, not in literal preservation, but in the creative renewal through dialogical interaction.


\bigskip

James Gordon Williams addresses the historical, socio-cultural, and political aspects of musical experimentation, referring to his own project with the feedback piano as {\textquotedblleft}a radical, sonic assault on hegemony manifested in the musical, cultural, political, and spiritual realms{\textquotedblright} (Section 6:1). He further discusses how musical innovation, and provides examples of how the history of musical feedback {\textquoteleft}represents sonic manifestations of agency{\textquoteright} that have specific meanings as political resistance. 

\subsection[(Re)thinking text and action]{(Re)thinking text and action}
Text and action, taken together, may draw our mind towards drama, but text and action have both wider and more narrow connotations, for instance as defining aspects of singing, also without dramatic action. Most importantly, the concept of text has been widened considerably through the linguistic turn in continental philosophy and it is in this wider perspective it is understood here. Rynell brings in situatedness, the element in acting which has the closest relation to improvisation, and asserts that {\textquotedblleft}situated acting is acting that is based on understandings of contexts that are real (as is fundamentally the case in performance art) and/or are conceived as real{\textquotedblright} (Section 4:1). Perhaps one may employ an alternative reading and look at how Sten Sandell (Section 4:4) uses the spatial properties of the church in Kalv to {\textquoteleft}situate{\textquoteright} his performance there.


\bigskip

Sara Wil�n (Section 4:3) points to the many layers of texts that are present in any musical performance, be it drawn from notated or written sources or from the personal \ inner musical library (Folkestad 2012) of oral traditions. This intertextuality is taken even further in improvisatory practice that draws also on literary texts and drama, such as in her own practice as opera-improviser. The situatedness of the action and the play between idioms and conventions from different genres creates a complex scenario in which the singers take part in an interperformative (Haring 1988; Parks 1988) {\textquoteleft}dialogue with genres that also become a vehicle for immediate expressions of different layers of subjectivity{\textquoteright}.


\bigskip

Due to the limited time available rehearsals of Gino Robair{\textquoteright}s opera \textit{I, Norton }(Section 4:2)\textit{,} the solid experience of the group \textit{Operaimprovisat�rerna }came as a blessing and Robair comments that this unusual situation - trained opera singers \textit{and} seasoned improvisers - resulted in an unusual performance. The musicians and actors were allowed to conduct their own actions and initiatives and, comments Robair, {\textquotedblleft}although I influenced the performance with various cues, the ensemble was responsible in large part for the final realization and success of the performance{\textquotedblright} (Ibid) We believe that it was specifically in the interperformativity that emerged in the encounter between the idioms and traditions embodied by the singers that created this dynamic relation to the text of Robair{\textquoteright}s \textit{I, Norton}.

\subsection[Looking ahead: What is the signification of {\textquoteleft}rethinking{\textquoteright} improvisation?]{Looking ahead: What is the signification of {\textquoteleft}rethinking{\textquoteright} improvisation?}
Why think about music? The famous saying {\textquoteleft}talking about music is like dancing about architecture{\textquoteright} (perhaps best represented in its variable form {\textquoteleft}talking about\_\_\_ is like\_\_\_ about\_\_\_{\textquoteright}) captures the common intuition that musical knowing is essentially distinct from analytical thinking. What can be gained by talking about rather than merely making music? How do our minds work when we make music? Along the same lines, Henk Borgdorff poses the question whether it is {\textquotedblleft}possible to achieve a linguistic-conceptual articulation of the embedded, enacted and embodied content of artistic research (Borgdorff 2012, p 170)?


\bigskip

But what do we mean by thinking? We believe that it is necessary to acknowledge how much of the most serious {\textquoteleft}thinking{\textquoteright} in the field of music is indeed situated outside the verbal domain and best described, following Merleau-Ponty, as a \textit{thinking-in-music} (Merleau-Ponty 1964). Or, speaking with Jacques Attali, we wish to point towards ways of theorizing \textit{through }music rather than about it (Attali 1977, p. 4). By acknowledging our ability for\textit{ thinking-through-listening} (�stersj� 2008), we may not only arrive at a rethinking of musical improvisation but also, the beginning of an epistemology of artistic research can be identified.


\bigskip

\subsubsection[Resonance: from individual to political listening]{Resonance: from individual to political listening}
Following Nancy (2007, p. 67), we find the shared understandings created through listening, the {\textquoteleft}resonance{\textquoteright} of the sound within the bodies of perceiving subjects, to be one of the \ fundamentals of a \textit{thinking-through-listening}. This reverberation stretches from the personal, to the intra-personal into the collective and political domains of human existence. Hence, in the resonance of an aesthetic phenomenon like feedback in music, many political concepts are also articulated and a shared knowledge is constructed. 


\bigskip

In an article on the embodied political entities in Turkish call to prayer, Eve McPherson emphasizes the interaction between listeners and practitioners in a way that further specifies our understanding of the function of resonance in musical listening:


\bigskip

Moreover, as the call to prayer is publicly expressed sound, its agents of meaningful processing and interpretation are both those who produce the sound, muezzins, and those who hear the sound, the local residents. These combined agents produce and take in the sound, and for a collective moment are affected by its generation and seem to have come to an agreement about what this embodied practice contains in terms of historical and social information. (McPherson 2011, p. 16)


\bigskip

Edward Said reminds us of the political nature of all cultural activity (2003:27). The way in which this resonance connects different levels of human life opens up for a cross-reference of seemingly distinct levels of cultural activities. By adopting a political perspective on the concept of musical style, the common battle-field made up of different stylistic approaches to improvisation can be re-thought as different manifestations of political action (though not often intended as such). Like Said, Attali points to the political nature of all music and the ways in which it is closely associated with commercialism and mass consumption. But music, however, also heralds a subversion and a possibility for {\textquotedblleft}a radically, new organization{\textquotedblright} that is yet unimagined (Attali 1977, p.5). We may look at artistic research as a space in which such expressions of radically new organizations could be developed, although, having said this, the relation between artistic freedom and academic structures needs to be carefully considered. \textit{(Re)thinking Improvisation} was intended and also organized with the ambition to acknowledge and address the complexity of the relations between artistic communities and the institutions of musical education, by bringing together several independent concert organizers with the researchers in the project into its program committee. It is essential for the credibility and assessment of artistic research that it is clearly situated also outside of the academy. For the political dimension of artistic research (and of any musical practice) to emerge, it must be situated in a particular social, theoretical, cultural, and philosophical framework which we believe has been developed within post-colonial and feminist epistemologies. In the sociological analysis of cultural institutions, both with Adorno and in Bourdieu, we see approaches towards new understandings of the inter-relations between the individual and the social. Attali strikes a similar note as in Adorno{\textquoteright}s critique of mass-culture and music as a commodity deprived of all its meaning, but at the same time situates this discussion in a contemporary and political understanding of these forces:


\bigskip

[...] music is not innocent: unquantifiable and unproductive, a pure sign that is now for sale, it provides a rough sketch of the society under production, a society in which the informal is mass produced and consumed, in which difference is artificially recreated in the multiplication of semi-identical objects. (Attali, 1977:5)


\bigskip

A radical critique of institutional structures is indeed necessary but not sufficient. The movement towards a {\textquotedblleft}radically new organisation{\textquotedblright} must also involve a new understanding of our subject positions (Hall 2000) and the hybridity of migratory identities. The film-maker and feminist researcher Trinh Minh-ha discusses the nature of this shift of perspective and emphasizes the need to not only focus on self-expression of the world but to challenge the individual by the production of {\textquotedblleft}texts{\textquotedblright} that question the systems of domination. This perspective, {\textquotedblleft}while it must insist on the self as the site for politicization, would equally insist that simply describing one{\textquoteright}s experience of exploitation or oppression is not to become politicized{\textquotedblright} (Trinh 1991, p. 163-164). This shift involves more than just a personal development or new understandings but must take shape in artistic expression and {\textquotedblleft}to know-to speak in a different way{\textquotedblright} (ibid p 164).


\bigskip

\textbf{The politics of musical traditions, notation and of improvisation}

What is the signification of conforming with the traditions of an idiomatic style of improvisation? From a political perspective, the answer is highly dependent on the context. When the Chinese sheng player Wu Wei joins the Vietnamese/Swedish group The Six Tones in a performance of \textit{He Moi} - a Vietnamese tune from traditional Cheo theatre - an exploration of the boundaries between distinct idiomatic traditions (musical idioms that have always had a political significance) is launched through a dialogue that would have been unthinkable without the freedom of mind that may emerge from the openness of true listening. Speaking with Jean-Luc Nancy one may ask, {\textquoteleft}what secret is at stake when one truly listens, that is, when one tries to capture or surprise the sonority rather than the message (2007:5)?{\textquoteright} In Wu Wei{\textquoteright}s explorations of the sonorities that emerge in this no man{\textquoteright}s land between different musical traditions, a music is revealed that is, again, anything but innocent. The complexity of the relations between the musical traditions of Vietnam and China has much resonance in the political violence in the past. But also the individual traditions are informed and influenced by, as well as expressions of, the two countries{\textquoteright} politics. Returning to Nancy, sound has never been apolitical and there is no way in which a clear demarcation can be drawn between message and sonority. But what is the secret that is at stake in a performance such as that of Wu Wei? Is it not found in the modes of listening? The politics of a {\textquoteleft}true listening{\textquoteright} is a challenge to the Western concept of the oriental {\textquoteleft}other{\textquoteright}. In this space, in which meaning has to be negotiated in between traditions and musical idioms, true listening is the primary source for human interaction%
%Stefan �stersj�:
%man kan trixa lite med slutet p� detta stycke!
%
.\footnote{ A further discussion of the transformation of traditions and the function of openness in listening to the other is found in the text by Nguyen \& �stersj� in the present publication.}


\bigskip

In the present day, traditional Swedish music and culture has become part of the current political debate surrounding the Swedish national democratic party (Sverigedemokraterna). To little surprise many folk musicians in Sweden felt uneasy about the way the national democrats hijacked traditional culture, including music, and turned it into a vehicle for their nationalist propaganda. But is the influence from other cultures a threat to national tradition and identity? Obviously, the interest in the preservation of national culture grew out of national romanticism in the 19th Century. The roots of the current movement for traditional Swedish music is in other words based on conservative political values. However, much water has flown under the bridges between the defenders of national culture in those days and the folk movement of the 1960s. The pendulum of political bias can swing quite radically over the centuries. In the present day, there is a growing awareness of the necessity of change as a fundamental of the transmission of a tradition. The investigations of Susanne Rosenberg and Olof Misgeld into the basic functions of variation and improvisation in traditional Swedish music can be understood as a de-mystification of the way traditions are communicated and how they live on. Though there was no outspoken political intention behind, the current political debate has become a context that gives these musical practices further political relevance. 


\bigskip

Helen Papaioannou shifts the focus from the composer{\textquoteright}s intention towards the realtime interaction between the performers by way of a score that does not represent sounding results as much as it is intended to evoke action. The authority of the score and its imagined representation of the composer{\textquoteright}s intention in the regime of \textit{werktrue} and the hierarchies that it assumes is thus by-passed. In similar ways, Sandeep Bagwaati{\textquoteright}s interactive score in the series of {\textquoteleft}Comprovisations{\textquoteright} creats a space where meaning can be negotiated in relation not only to the composer{\textquoteright}s inscriptions in the score but also in relation to the traditions embodied by the performers. Kim Ngoc Tran Thi, in her composition titled \textit{Move}, makes the identity of the composition highly dependent on the identity of the performers and specifically on the hybridity that emerged through the long-term blending of traditional Vietnamese music and experimental western modes of expression in the work of \textit{The Six Tones}. All these examples of works also released in the present publication point to the further political implications of a shift in the view of the composer and of the score in relation to a globalized and post-colonial society. 


\bigskip

By situating the diverse kinds of music making discussed above in a wider discourse, a politicalization may be said to occur. However, our claim is that in essence, it is the resonance within the listening subject that reveals the political nature of all artistic practice. This claim has very little to do with what is commonly referred to as political art and goes beyond any specific modelling of the artistic output along political lines. Also, it is not linked only to verbal discourse \textit{on }music but it is rather a discourse \textit{in }music (Folkestad 1996) that does have the potential to create the radicalization that Attali discusses. 


\bigskip

Like much artistic practice and research neither the project (re)thinking, nor the event (re)thinking improvisation have been activities that have come to complete and final end, but should be regarded as a contribution to the widening field of research into musical improvisation. Although limited in scope and as always in human activities also replete with shortcomings, it was an attempt to address and bring together some of the many expressions of this shift in contemporary musical practices. The present publication contains essays, both in writing and in musical performance, that share some of these qualities. It is our hope that (re)thinking improvisation may provide a spark into the emerging field of artistic research and contribute to a development in which experimental work can constitute a bridge between the academy and the many musical subcultures, the laboratories in which we can experiment, and provide a site where we can learn, with reference to Trinh Minh-ha, - {\textquotedblleft}to know-to speak in a different way{\textquotedblright}.


\bigskip


\bigskip

\subsection[References]{References}
Bennis, W. G., \& Nanus, B. (1985). Leaders: The strategies for taking charge. New York: Harper and Row

Berio, Luciano (1986): \textit{Sinfonia}, Program Note, (trans Underwood, J) Erato Disques, 1986.

Borgdorff, H. (2012). The conflict of the faculties : perspectives on artistic research and academia. Leiden University Press, 2012.

\textcolor[rgb]{0.14117648,0.11764706,0.1254902}{Denzin, Lincoln \& Giardina (2006): Disciplining qualitative research, }\textit{\textcolor[rgb]{0.14117648,0.11764706,0.1254902}{International Journal of Qualitative Studies}}\textcolor[rgb]{0.14117648,0.11764706,0.1254902}{ in Education Vol. 19, No. 6, November-December 2006, pp. 769--}\textcolor[rgb]{0.14117648,0.11764706,0.1254902}{782, Routledge}

\textcolor[rgb]{0.14117648,0.11764706,0.1254902}{Druker, P. F. (1988). The society of new organizations. Harvard Business Review, 70(5), 95 -- 104}

\textcolor[rgb]{0.14117648,0.11764706,0.1254902}{Eco, U. (1989). }\textit{\textcolor[rgb]{0.14117648,0.11764706,0.1254902}{The Open Work}}\textcolor[rgb]{0.14117648,0.11764706,0.1254902}{ (A. Cancogni, Trans.). London: Hutchinson Radius.}

\textcolor[rgb]{0.14117648,0.11764706,0.1254902}{Folkestad, G. (1996). }\textit{\textcolor[rgb]{0.14117648,0.11764706,0.1254902}{Computer based creative music making : young people{\textquotesingle}s music in the digital age}}\textcolor[rgb]{0.14117648,0.11764706,0.1254902}{. G�teborg: Acta Universitatis Gothoburgensis}

Folkestad, G. (2012). {\textquotedblleft}Digital tools and discourse in music: The ecology of composition{\textquotedblright} in Musical Imaginations:Multidisciplinary perspectives on creativity performance and perception, ed Hargreaves, MacDonald \&Miell, Oxford University Press, Oxford

\textcolor[rgb]{0.14117648,0.11764706,0.1254902}{Frisk, H., \& �stersj�, S. (2006). Negotiating the Musical Work: An empirical study. Proceedings of the International Computer Music Conference 2006 (pp. 242--249). San Francisco, Calif.: Computer Music Assoc.}

\textcolor[rgb]{0.14117648,0.11764706,0.1254902}{Frisk, H. (2008). }\textit{\textcolor[rgb]{0.14117648,0.11764706,0.1254902}{Improvisation, Computers, and Interaction: Rethinking Human-Computer Interaction ThroughMusic}}\textcolor[rgb]{0.14117648,0.11764706,0.1254902}{. PhD thesis, Malm� Faculty of Fine and Performing Arts, Lund University.}

\textcolor[rgb]{0.14117648,0.11764706,0.1254902}{Gibson, J. J. (1986). }\textit{\textcolor[rgb]{0.14117648,0.11764706,0.1254902}{The ecological approach to visual perception}}\textcolor[rgb]{0.14117648,0.11764706,0.1254902}{. Hillsdale, N.J.: LEA.}

Hall, Stuart: Cultural Identity and Diaspora (1993) in Williams, Patrick \& Laura Chrisman eds. \textit{Colonial Discourse \& Postcolonial Theory: A Reader}. \ New York: Harvester Wheatsheaf, 1993

\textcolor[rgb]{0.2,0.2,0.2}{Hall, Stuart (2000): Who Needs Identity?}\textit{\textcolor[rgb]{0.2,0.2,0.2}{ }}\textcolor[rgb]{0.2,0.2,0.2}{in }\textit{\textcolor[rgb]{0.2,0.2,0.2}{Identity: A Reader}}\textcolor[rgb]{0.2,0.2,0.2}{ edited by Paul du Gay, Jessica Evans, Peter Redman}

\textcolor[rgb]{0.2,0.2,0.2}{Haring, L. (1988). }\textit{\textcolor[rgb]{0.2,0.2,0.2}{Interperformance}}\textcolor[rgb]{0.2,0.2,0.2}{. In de Gruyter (ed), }\textit{\textcolor[rgb]{0.2,0.2,0.2}{Fabula}}\textcolor[rgb]{0.2,0.2,0.2}{, Volume 29 (1), 365-372.}

Levinson, G. (1993). Performative vs. Critical Interpretations of Music. The Interpretation of Music: philosophical essays. Clarendon Press, Oxford.\ \ \ \ \ \ \ \ 

Merleau-Ponty, M. (2002). \textit{Phenomenology of Perception}. London: Routledge.

\textcolor[rgb]{0.14117648,0.11764706,0.1254902}{Mozart, Leopold: (1951/1776) }\textcolor[rgb]{0.2,0.2,0.2}{A Treatise on the Fundamental Principles of Violin Playing, Oxford University Press}

\textcolor[rgb]{0.2,0.2,0.2}{Nancy, J.-L. (2007). }\textit{\textcolor[rgb]{0.2,0.2,0.2}{Listening}}\textcolor[rgb]{0.2,0.2,0.2}{ (C. Madell, Trans.). New York: Fordham University Press.}

\textcolor[rgb]{0.2,0.2,0.2}{Novotny, H. (2011). }\textit{\textcolor[rgb]{0.2,0.2,0.2}{Foreword}}\textcolor[rgb]{0.2,0.2,0.2}{. In (}Biggs and Karlsson (2010))

Parks, W. (1988). Interperformativity and Beowulf. \ In \textit{Narodna Umjetnost, }\textit{Croatian Journal of Ethnology and Folklore Research}, 1989: 26:1 pp. 25-34. Retrieved from\href{http://hrcak.srce.hr/index.php?show=clanak&id_clanak_jezik=98757&lang=en}{ }\url{http://hrcak.srce.hr/index.php?show=clanak&id_clanak_jezik=98757&lang=en}


\bigskip

Traub, J. (1996). What CEO{\textquoteright}s could learn from the Orpheus Chamber Orchestra. New Yorker Magazine, 100 -- 105.

\textcolor[rgb]{0.2,0.2,0.2}{W�rner, K. (1963). Stockhausen: Life and work, University of California press.}

\textcolor[rgb]{0.14117648,0.11764706,0.1254902}{�stersj�, S. (2008). }\textit{\textcolor[rgb]{0.14117648,0.11764706,0.1254902}{SHUT UP {\textquotesingle}N{\textquotesingle} PLAY! Negotiating the Musical Work.}}\textcolor[rgb]{0.14117648,0.11764706,0.1254902}{ Lund University, Malm�.}

\textcolor[rgb]{0.14117648,0.11764706,0.1254902}{�stersj�, S (2013) The resistance of the Turkish Makam and the Habitus of the performer. Contemporary Music Review}

Trinh Minh-ha: \textit{When The Moon Waxes Red} (1991) New York: Routledge
\end{document}
