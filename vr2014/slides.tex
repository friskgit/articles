\documentclass{article}
\usepackage[english]{babel}
\usepackage[T1]{fontenc}
\usepackage{url}
\usepackage[utf8]{inputenc}
\usepackage{enumitem}
\usepackage{csquotes}
\usepackage{fixltx2e}
\renewcommand{\encodingdefault}{T1}
%\usepackage{hyperref}
\usepackage[pdftex]{graphicx}
%\usepackage[authoryear,round]{natbib}
\usepackage[style=authoryear,natbib=true,backend=biber]{biblatex}
\bibliography{~/shared-home/Documents/svn/admin/conf/biblio/bibliography}
\renewcommand{\rmdefault}{pad}
\renewcommand{\sfdefault}{pfr}

\begin{document}

\title{Att bygga ett rörligt konstnärligt forskningsfält}
\author{Henrik Frisk\\Kungl. Musikhögskolan, Stockholm\\Musikhögsk. i Malmö, LU\\{\small henrik.frisk@kmh.se}}
\date{}

\maketitle

\noindent
Förslag på presentation till Vetenskapsrådets årliga symposium om konstnärlig forskning 2014. Presentationen kan begränsas till 15 minuter (gärna tid för frågor/diskussion därefter). 

\section*{Abstract}
Efter några decennier av konstnärlig forskning i Sverige har vi kommit långt. Många avhandlingar och seniora projekt, ett flöde inom flera fält som är fördelat på många olika ämnen och teman. Floran av metoder som åberopas blir större och teoriutvecklingen går långsamt framåt. Men det är min uppfattning att det finns ett område där vi fortfarande har långt kvar: forskningsprojekten relaterar inte till varandra i tillräcklig stor utsträckning. Resultatet blir att vi får en samling isolerade kunskapsöar utan broar vilket får till följd att vi inte bygger en dynamisk konstnärlig forskningsidentitet. Konstnärlig forskning är ofta definierad utifrån en individuellt orienterad frågeställning men jag vill hävda att detta inte är del av problemet. Det finns ingen motsättning mellan starkt individualistiskt orienterade forskningprojekt och en bred förankring i tidigare forskning. Jag vill också hävda att vi kan inte bygga ett fungerande forskningsområde utan att de enskilda projekten bygger på en bred kännedom om fältet och att denna tydligt redovisas i arbetet. 

\section{Den modernistisk/romantiska konstnärsmyten}
Den konstnärliga forskningen har en gyllene chans att göra upp med bilden av den solitära, manliga konstnären som kommunicerar med sin publik genom verket. Det handlar inte nödvändigtvis om att rasera denna bild utan om att öppna upp för andra modeller och för att visa på vad som händer i kommunikationskedjorna mellan tanke, skapande, värdering, presentation, tolkning, omvärdering, nyskapande, etc. 

Kan konstnärlig forskning lyckas med öppna upp skapandeprocesserna så rymmer forskningen en möjlighet till en ny konst.

\section{Det subjektiva}

Den subjektiva utgångspunkten, eller det fenomenologiska perspektivet är centralt i konstnärlig forskning som får sin validitet utifrån det subjektiva angreppet. I debatten kan man lätt se en motsättning mellan detta perspektiv och det vetenskapliga vilket är ett antagande som inte är korrekt. Som Merlau-Ponty skriver, även en betraktelse över empiriska data är en individuell bedömning. Men det finns ingen motsättning mellan det subjektiva anslaget och det kollektiva kunskapsbyggandet. Tvärtom, det konstnärliga forskningsfältet får sin dynamik genom att de enskilda projekten vävs samman i en dynamisk väv, precis som allt konstnärligt skapande har gjort tidigare. 


\section{Den modernistisk/romantiska skapelsen}
I konstnärsmyten ryms också den konstnärliga skapelseberättelsen. Den mystiska inspirationen. Denna har fått träda tillbaka en smula på senare år då neurologerna har börjat intressera sig för den och kan visa att kreativiteten inte alls nödvändigtvis är mystisk eller har sin härkomst från något högre väsen. Vilket vi alla redan visste. Men även debatten om kulturmannen och romanen ``Egenmäktigt förfarande'' har bidragit till diskussionen och utvecklingen.  Den konstnärliga identiteten upprätthålls rent socialt och genom tanken på originalitet som estetiskt kriterium. 

Det kanske viktigaste skälet till att skapandet delvis är höljt i dunkel är förstås att det som utgör inspirationen, eller kreativiteten, inte utan vidare låter sig beskrivas. Den är, för att tala med Freud, kodad som en primär process, av definition icke-verbal.

% Konstnärens material och själva källa till skapandet är en del av dennes identitet. Utan att egentligen ha studerat detta vågar jag ändå säga att det finns fördelar med att hålla på sina idéer, inte minst i den mer konceptuellt orienterade konsten.

%Men det är starka och djupt nedärvda strukturer som gör att tonsättaren sällan talar om skapelseprocessen annat än 

När vi så vill bygga upp en konstnärlig forskning med fast förankring i det konstnärliga skapandet så måste vi samtidigt förhålla oss till den upphovsrättsligt orienterade kreativiteten och den forskningsmässigt orienterade genomskinligheten och vidareutvecklingen. Det är min uppfattning att vi hittills har haft för låga ambitioner kring denna relation och dess utvecklingspotential.

\section{Citeringar i konstnärlig forskning}

Många har kommenterat den konstnärliga forskarens förkärlek för filosofi och idéhistoria. Ett typiskt forskningsprojekt använder sig av en hel flora av filosofiskt orienterade texter som citeras och återanvänds. Detta är naturligtvis ypperligt och kan gagna både filosofiämnet och den konstnärliga praktiken men det är inte det som är ämnet idag.

Vad som bara ytterst sällan förekommer i min erfarenhet är referenser till andra konstnärliga verk som idé och kunskapsbärande element. Detta får en rad konsekvenser varav den kanske viktigaste är att trovärdigheten i att visa på ens eget skapande som kunskapsbärande faller om man inte hänvisar till andra, liknande projekt. Eller, uttryckt annorlunda, om jag som konstnärlig forskare inte tror på, eller utnyttjar, den kunskap som finns i konsten omkring mig kan jag få problem att hävda kunskapsaspekten i mitt eget arbete.

\includegraphics[]{./img/duchamp.jpg}

Själva idén om vidareutvecklingen av konstnärliga verk alltid funnits som en del av konsten. Som exempel kan nämnas:

\begin{itemize}
\item L.H.O.O.Q. av Marcel Duchamp
\item Nattvarden av Elisabeth Ohlson Wallin
\item Brahms, Schnittke och Lutoslawskis versioner av Paganinis a-moll caprice
\item Anders Hultqvists version av Beethovens 5e symfoni.
\end{itemize}

\includegraphics[]{./img/nattvard.jpg}

\section{Vilka krav finns det?}
Forskningsmiljöns ställer krav på det man kallar redlighet i forskningen. Detta inkluderar att man redovisar sina källor.

I Högskoleförordningen kan man läsa om högskolan och universitetets ansvar:

\begin{quote}
  16§ En högskola som genom en anmälan eller på något annat sätt får
  kännedom om en misstanke om oredlighet i forskning, konstnärlig
  forskning eller utvecklingsarbete vid högskolan ska utreda
  misstankarna.
\end{quote}

VR har samlat sina riktlinjer på webplatsen Codex (\url{http://codex.vr.se/etik6.shtml}) där man bl.a. skriver:

\begin{quote}
  Med fusk och ohederlighet inom forskningen avses avsiktlig
  förvrängning av forskningsprocessen, genom bedrägliga handlingar som
  faller inom en eller flera av följande kategorier:
  \begin{itemize}
  \item fabricering av data

  \item stöld eller plagiering av data, t ex hypoteser eller metoder
    från annan forskares manuskript, ansökningshandling eller
    publikation (utan angivande av källa)

  \item förvrängning av forskningsprocessen på annat sätt, t.ex. genom
    felaktig användning av metodik, genom ohederlig inklusion eller
    exklusion av data, genom bedräglig analys av data som avsiktligt
    förvränger tolkningen, eller genom ohederlighet mot
    anslagsgivare. (Riktlinjer för etisk värdering av medicinsk
    humanforskning)
  \end{itemize}
\end{quote}

Vetenskapsrådet och SUHF har i en senare skrivelse föreslagit följande definition för Sverige:

\begin{quote}
  Vetenskaplig oredlighet innefattar handlingar eller underlåtelser i
  samband med forskning, vilka leder till falska eller förvrängda
  forskningsresultat eller ger vilseledande uppgifter om en persons
  insats i forskningen. För ansvar krävs att den vetenskapliga
  oredligheten begåtts uppsåtligen eller av grov oaktsamhet.
\end{quote}

\section{Hur efterföljs detta?}

Man ska komma ihåg att det främsta problemet med en forskningsmiljö som inte lär av sin egen produktion, där kunskap som emanerar från forskaren och projektet till stor del stannar hos densamme, är att fältet inte kommer utvecklas i den takt det skulle kunna göra. Huruvida det i något fall blir ett formellt problem och vad det isf betyder är egentligen en annan fråga. Därför ska formuleringarna ur regelverket ovan ses som just riktmärken, men det är inte desto mindre med utgångspunkt i dessa som vi ska skapa vår egen praxis.

Mitt intryck från de avhandlingar jag har läst och från de projektrapporter jag har läst är att vi är ganska dåliga på att:

\begin{itemize}
\item knyta an till existernade konstnärlig forskning
\item knyta an till pågående forskning internationellt
\item knyta an till konstnärlig praktik på ett organsierat sätt, även sådan som inte är del av ett forskningsprojekt
\end{itemize}

Däremot är vi ganska bra på att:

\begin{itemize}
\item hänvisa (löst) till genomförd forskning
\item hänvisa (löst) till existerande konstnärliga verk
\end{itemize}

\section{Vad behöver vi göra?}

Jag menar att vi måste ta ett grepp om dessa frågor och i mycket större utsträckning uppmuntra doktorander och forskare att undersöka fältet och ta in, inte bara andra konstnärliga forskningsprojekt, utan även andra konstnärliga projekt och utnyttja dessa både som källa till kunskap och som case studies. Problemet blir större då fältet som helhet är ganska litet och graden av specialisering är väldigt hög. 

Frågor som man då kan ställa och kanske behöver diskutera är bland annat:

\begin{itemize}
\item Hur ska citeringar till konstnärliga verk ske?
\item Hur ska vi som handledare vägleda doktoranderna till att mer aktivt söka information i andra konstnärliga arbeten?
\item Hur ska vi som institutioner värdera forskningsarbetenas kvalitet på detta område, och hur ska vi agera om något arbete inte håller måttet?
\item Hur ska bidragsgivare bedöma och värdera ansökningar och projektrapporter utifrån deras omvärldsanalys?
\end{itemize}

\section{Att bygga hållbar kunskap}

För att kunna bygga en hållbar miljö för konstnärlig forskning så är det utan tvekan helt avgörande att vi också kan bygga vidare på varandras resultat. För att det ska lyckas måste vi naturligtvis redovisa våra resultat på ett sätt att andra kan ta del av dem, men vi måste också redovisa våra egna källor. Dessa källor inkluderar då alltså även utgångspunkterna för vår konstnärliga metod och praktik. Nästa led i detta är att andra forskare tar till sig befintlig forskning och delvis släpper på konstnärens fiktivt autonoma position och bejakar öppenheten i skapandet.

% \printbibliography
\end{document}
