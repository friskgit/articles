\documentclass[a4paper]{article}
\usepackage[swedish, english]{babel}
\usepackage[T1]{fontenc}
\usepackage[authoryear,round]{natbib}
\usepackage{url}
\usepackage[utf8]{inputenc}

\usepackage[pdftex]{graphicx}
\renewcommand{\encodingdefault}{T1}
%\renewcommand{\rmdefault}{pad}
\usepackage{graphicx}

\usepackage{fancyhdr}
\pagestyle{fancy}

% \lhead{\small{\textit{Henrik Frisk}}}
% \chead{}
% \rhead{\small{\textit{Improvisation and the self}}}

\title{Improvisation and the self: to listen to the other}
%\author{Henrik Frisk\\{\small Assistant Professor}\\{\small Royal College of Music in Stockholm}\\{\small henrik.frisk@kmh.se}}
%\date{\today}

\begin{document}
\selectlanguage{english}
\maketitle

\thispagestyle{empty}

\section*{ }

%Se sid 148 in Proust and The Signs, slutet av första stycket.
%Generally, what is meant by the invitation to listen to the other is to listen to the other musicians in the group. Should the improviser not listen to the audience then? When we listen, what is it we allow ourselves to be influenced by, and what part of our own expression should remain untouched by our listening? Obviously, \footnote{Resonance is a key word in Jean-Luc Nancy's book on listening.}  It might as well be stated, there are no general wrongs or rights in listening, only intuition and sensibility. At one point, there may be a demand for absolute and unconditioned control, and at another it may be necessary to completely give in to the other. 

% An idealized, albeit untrue in most cases, idea of free improvisation is that it is free from habits.\footnote{That it is untrue is easy to discern from listening to almost }
% In music, listening is the interface to the world outside. Without it the self gets trapped.

%Going through a preliminary limitation of the concept of the self in artistic production, pointing to the importance of dismantling the mysterious nature of the self. In artistic research, the self and the subject has a central role, and for that reason it is important to develop ones sense of its agency. 

To listen to the other. The phrase raises an incalculable number of issues. One of the central topics in teaching improvisation is to learn how to listen to those with whom one plays, but in my experience the most difficult task to listen to the self. The point of listening to the other in performance is not completely to give up the self, nor is it to become the other, but to attune to, or find resonance with the other. It is in the interaction between two or more musicians the open and unbound improvisation is unfolding, in the space between adjusting to the other, and listening to the self. In this paper, I will use my artistic practice in the Swedish-Vietnamese group The Six Tones as a context for approaching some of these questions concerning self and other by means of three concepts, each of which have a profound influence on the self: freedom, habit and individuality. Though these concepts are very broad and deep, they will be tackled in a relatively limited and practical context. 

The impact of freedom, being such an essential concept in the understanding of improvisation, is closely related to some of the more social and political topics that I will touch upon, and can be understood in a number of ways, such as freedom \emph{of the self} and freedom \emph{from the self}. Habit is a factor that can both limit the space for freedom and participate in allowing for it, and there are a number of examples of habit destruction as a creative force. Habit is part of the constitution of the self and detaching habit from the self may be a method to introduce change. Individuality, an equally important aspect of jazz and improvised music, is interesting and complex and can participate in turning freedom-of-the-self into a power of domination. Freedom is a necessary condition for the individual expression which in turn may reduce the space of freedom to others. In The Six Tones, situated as it is in an intercultural context, the meanings of self, other, listening, habit and individuality are investigated also outside a purely musical realm. 
Starting from the conceptually, stylistically as well as geographically local perspective the paper is ending with a discussion on center and periphery in a post-colonial context, and, as a preliminary remark, I believe that it is both possible and important to engage in a social and political discourse, and I am confident that the art field, and specifically the field of artistic research, is an appropriate arena for this.

I am not attempting to use a theoretical framework and applying it to the artistic work flow. Rather, I am concerned with understanding what was going on in the early stages of our work in the group, and to understand my own initial and consequent reactions. By filtering these experiences through the thoughts presented below, and understanding my own reactions as expressions of a system of domination I can approach my own artistic practice as a vehicle for social and political thinking through music, and I can approach the difficult questions concerning self, habit and freedom. The method employed is a recursive cycle through stages of artistic practice, reflection, evaluation, and theory as practice. 

\section*{ }
\label{sec:self-artist-prod}

The notion of the self is a philosophically complex issue, and I am not claiming that this text is an exhaustive account of all possible angles. My primary interest is the role of the self as seen from a very practical perspective: In a musical interaction how can the self at the same time be responsive to the other, be free and be individual and what is the epistemology for this aspiration?
By virtue of the subjective nature of improvisation, and many other creative practicies, the role of the self is essential to the various processes in artistic work such as creation, evaluation, construction and presentation. Many theories have been developed to explain the operations of the ``mysterious nature of the self'' \citep[][p. 167]{griffiths10}, subject to constant change, but the mere attempt to define its nature appears to change it. That ones reflection upon the self in fact alters it is a founding property of most kinds of therapy, though there is no way of knowing whether it is the understanding of the self or the self itself that is altered. The connection and interdependence of self with time and place, with the other, with the body and the socio-political spheres and power relations makes it clear that it is in a consistently uncompleted state, in a capacity of constant becoming. Anthropologist Gregory Bateson, to whom we will return later in this text, identifies the self as an aggregation of ``habits of perception and adaptive action'' \citep[p. 242]{bateson72:steps}. Although his definition may appear to be too formal in this context, bringing in habit as a defining property of the self is significant, and part of my argument here is that altering habitual responses is a way to alter the self. 

%If these are the ways in which the self is constructed, the self is also constructing the other. However, 
%If it is claimed that the self is also claimed 

As an illustration of one of the ways in which the meeting with the other changes, not only the self but also the other, Deleuze and Guattari claims it to be possible to move across, or transgress, the border between self and other: ``Whenever there is transcoding, we can be sure that there is not a simple addition, but the constitution of a new plane, as of a surplus value. A melodic or rhythmic plane, surplus value of passage or bridging'' \citep[p. 346]{deleuze80}.\footnote{See also \citep[][p. 140]{semetsky2011}} I would like to argue that improvisation in particular is a powerful site for exploring questions concerning subjectivity, identity and the spheres of self and other, and many of the principal aspects of the meeting between self and other can be anticipated and expanded in the domain of musical practice. Counterpoint, and the simultaneous interaction between several voices at once, would render any verbal discussion incomprehensible and impossible to follow, whereas in music, it may even do the opposite: create greater clarity. Listening is fundamental to all musical practices and listening to the other is an essential aspect of any relation: ``since music making is something we inevitably do with others (whether they are present or not), musical dialogue is \emph{fundamentally} ethical in nature'' \citep[p. 164]{benson03}. 

Although very difficult to define, freedom in general is a recurring concept in the discussion of musical improvisation. What is the function of freedom in the constitution of the self? In a more general context Hanna \citet{arendt77} claims that ``to raise the question, What is freedom? seems to be a hopeless enterprise'', and freedom is no less complicated when discussed in relation to musical practices. Free jazz has been a genre since Ornette Coleman's epic release \emph{Free Jazz: A Collective Improvisation} \citep{coleman60} and ever since has there been a discussion as to what has actually been freed in the process. Is it the musician that is liberated or the music? Or something else? The free jazz movement in the USA in the 1960's was intimately linked to the civil rights movement adding a political dimension to the discussion. On the surface improvisation may seem as a means to create music that is free from the chains of the formal structures that, e.g. notation imposes on the musician. Subjectivity and individuality is, and has been, a powerful agent in much of jazz and improvised music, as it has rejected the generality that other kinds of music has advocated.\footnote{After all, composing for symphony orchestra will only make sense if the sounds it offers are relatively general. Hence, the symphony orchestra is a machine that proposes a finite set of sounds and as such it would cease to work if its musicians would begin to promote their own individual and singular sound similar to how Ben Webster and Johnny Hodges had done in the Duke Ellington Orchestra.} Even though we know that such descriptions are not correct, the prevailing notion is that jazz is a music that may be created on the spot and whose substance is defined by the will of the improviser rather than by external factors or structures. 

Many have rightly criticized the understanding of jazz and improvised music as devoid of planning, and ignorant to history and memory, but the field is complex and the concept of improvisation will not let itself be defined on only one axis. The idea of the improvising musician as a ``mystic who is unable to describe his or her own creative process is a staple of conventional cultural wisdom about jazz'' \citep[p. 170]{lewis-1}, proposed by established, privileged and canonical composers as well as by improvising musicians themselves. It should however be noted that many of the attempts to target improvisation as unforeseen, unplanned, spontaneous and based primarily on ``feelings'' is to a significant degree a political strategy with which jazz musicians has been kept outside the central structures of cultural funding. An expression that according to a popular view to a significant degree is made up in the moment cannot be taken seriously in a culture otherwise driven by the auteur. 


The claim on jazz musicians to be both strongly individual and free at the same time quickly becomes problematic as the two concepts are connected. With reference to one's right to be individual one may may end up using one's personal freedom to claim the right to control the situation at the expense of the freedom of the other. This is a surprisingly common mode in jazz improvisation where both freedom and power of expression and musical readability are highly valued. In his book \emph{The Philosophy of Improvisation} Gary \citet{peters09} calls it the ``aporia of freedom''. Though freedom is generally thought of as a positive concept he claims it to be a mistake to neglect its ``questionable duality'' (p. 21): ``my freedom comes at the expense of the other's freedom, my own autonumous world at the expense of the other's heteronomy'' \citep[p. 165]{benson03}. This duality is perhaps even more obvious in the light of the mythical view of the creator, a Kantian genius, who enjoys the undeniable freedom of the artist. To this artist subjectivity and individuality are not choices, they belong to his vocation and constitute the very nature and value of artistic work. As an emblematic representation of the notion of the true creative virtuouso Pierre Boulez expressed a sheer lack of understanding for Cage's idea of setting ones own intention to the side. To avoid or even neglect the qualified projection of the self in determination would be simply irresponsible \citep{boulez64}.\footnote{See also my discussion in \citet[p. 144-5]{frisk2013}} In the context of improvisation the autonomous creator may not be as intrusive, but the combination of creativity as an immanent and individually constituted property, and freedom will indeed risk to circumbscribe the freedom of the other. Furthermore, the romantic nineteenth century artist has created a mythology so powerful that, still today, it has an impact on authors, composers and musicians alike. The creative act is so strongly soldered to this image that even the understanding of an improvising musician, whose creativity depends not on work creation but on the real-time impulses in performance is informed by it and as such, the individual self is influenced. 

%Although the autonomous creator is 

%It may be argued that the autonomous creator is 

%The most elementary form of the transmutation of chance would lie in the adoption of a philosophy tinged with Orientalism that masks a basic weakness in compositional technique; Boulez p.42

% Furthermore, freedom, as in being a free improviser has a 

% As freedom is as a central as problematic issue in improvisation this topic is vary interesting in this context. Benson goes on to ask how the true dialogue can be maintained and the autonomous dialogue suppressed? First of all we should remember that once we leave the domain of philosophy and enter the world of practice the dividing lines may not be so clear cut. I am not sure if those who have promoted the right to freedom in music would agree that they have subsequently repudiated the possibility for an ethical relation to the other in general, but it is clear that this topic has some relevance in some kinds of practices, as we shall see. What more is, even if I reduce my subjective autonomy, limit my freedom and approach the other as infinitely other, it is not a guarantee that I will remain ethical.


Although the search for an individual sound in most cases is a very conscious act there is a corresponding search for the pure, or unconscious, expression, exemplified by Ornette Coleman's attempts to short-circuit the habitual aspects of his saxophone playing. In order to be able to ``create as spontaneously as possible--'without memory,' as he has often been quoted as saying'' \citet[p. 117]{litzweiler92}, without any 'real' training he started playing the violin and the trumpet. These instruments gave him the freedom to play and improvise in a manner that his memory made it difficult for him to do on saxophone. When playing the saxophone he would be partly ruled by his meta-knowledge, his knowledge \emph{about} playing the saxophone. Habits encoded mentally as well as bodily would also influence him and, to Ornette Coleman, this was a hindrance to his spontaneity. On the violin he adopted a highly original technique that allowed him to bypass ``not only the jazz tradition, but Western musical traditions altogether. He had no teachers or guides to show him how to play trumpet and violin and purposely avoided learning standard techniques'' \citet[p. 117]{litzweiler92}. Freedom of memory and freedom of influence from extra-musical parameters. The 'unknown' instruments gave Coleman a sense of \emph{internal} freedom, liberated from the physical memory associated with his saxophone playing. He approached a self expression where the transformation from intention to result was not ruled by a preconceived notion of what it should sound like. Coleman identify the embodied memory as perhaps the most important dimension in the struggle to be free, and by using a new tool he neutralizes the impact of the habits related to his saxophone playing.
%And the question comes to mind, revisiting Peters' notion of the aporia of freedom: What was the impact of Coleman's abandoning of tradition on the musicians he played with?

%\section*{The practice}
\section*{ }
\label{sec:tu-dai-oan}

In early 2006 Stefan Östersjö and I initiated a project together with Ngyen Thanh Thuy and Ngo Tra My, two Vietnamese musicians temporarily visiting Sweden at the time, as guest teachers at Malmö Academy of Music. Thuy plays dan tranh, a traditional Vietnamese zither played by plucking the strings with the right hand, adding vibrato and glissandi with the left hand. The dan tranh is related to the Korean kayagum and the Japanese koto. The dan bau, played by My, is a single chord instrument played with a bamboo plectrum with the right hand while altering the pitch with the left hand by pushing, or pulling, a rod, thus stretching or relaxing the string. Different overtones can be produced depending on where the string is plucked and the sound of the string is picked up with a magnetic pickup and amplified through a small speaker. Since 2006 we have done a number of tours and projects in many different constellations and settings under the name The Six Tones.

The Six Tones is an encounter between traditional Vietnamese music and experimental Western European music, and since the very beginning the main objective has been to find forms for interaction between these two musical cultures on more or less equal terms. However, apart from the musical intercultural intentions the group has become a site for experimentation and examination of the political and social meaning and impact of that ambition. Experimentation is a central concept to us and ``by definition, experimental data must be able to behave in a way not predicted by the hypothesis. Thus, the experiment is conceived as an excellent setting for exploration and discovery, a perfect opportunity for an encounter with the new, the unforeseen, and the unfamiliar'' \citep[p. 165]{corbett2000}. In order to truly encounter the new and the unforeseen, challenging different aspects of the notion of center and periphery was necessary: is Western art music the norm and traditional Vietnamese music an exotic other? Are Stefan and I `visiting' a music outside of our own sphere, or is it rather Thuy and My that are forced to approach us. Is it at all possible to communicate on equal terms in a context that holds so many economic and social inequalities? Are we as Westerners able to rid ourselves of the colonial heritage that in many respects still govern our interactions with the east when we meet Thuy and My in this group? These are questions belong to the larger project of The Six Tones, and will not be thoroughly probed in this text, but it is nevertheless possible to distil the questions into the more individually oriented: What is the role of the self in the encounter with the other? Even though my own interest in the self in artistic practice started more than a decade before we initiated The Six Tones the project strengthened my belief that self, individuality, freedom and habit were important agents whose interrelations are substantial in, and outside of, my musical practice. 
%In Gregory Bateson's theory  

%the way I dealt with my own understanding of my self in music  

To begin with we had to reevaluate our own musical identities, and for myself I had to question my roles as composer and improviser and reconsider what my level of influence should be, and what it could become. In order to create the necessary preconditions for the two different musical traditions to intermingle rather than coexist void of deeper interaction, the individual influence on the musical structures had to be carefully negotiated. As a result of our lack of experience in such collaborations our first meetings were very tentative. I found it extremely difficult to balance my own initiatives and leaving enough space for the input of Thuy and My. One reason was the lack of a shared language and another was the social asymmetry between the two subgroups. The fact that the geographic origin of the four members was paralell to their gender made the collaboration saturated with disparity and inequality. Although music can be seen as a neutral form for communication with the prospect of compensating social differences, it may equally well disguise them. While we were in the known environment of the music academy--at home musically, culturally and socially--they were visitors and foreigners without language or context. 

Reflecting upon the situation brings forth questions concerning identity, culture, power and habit, all of which are to some degree interrelated, and in a culturally and socially homogeneous context many of these questions are not even raised. They are unnecessary as much of the negotiation has been going on in a larger context, outside of the rehearsal space. Common signifiers, references and aesthetic negotiations, inherited and nurtured since the early development of musician and composer are easily accessible. A collaboration, however, does not have to be between two different musical and artistic cultures in order to raise issues such as those we encountered in The Six Tones. Simply bringing together musicians from different genres may create obstacles difficult, or plain impossible, to maneuver. 


At our first meeting in the composition studio at the Malmö Academy of Music I became incredibly self aware of the asymmetry between Stefan and I and Thuy and My. Given the history of Vietnam in particular, and the history of the white man in general, the fear that my identity, individuality and cultural background would get in the way of Thuy's and My's freedom to participate on their own terms. But instead of letting them 
%% FINNS SPIVAK MED FORTFARANDE???
\emph{speak for themselves}, I leaned on my preconception of what it means to be a female Vietnamese visitor in Sweden. While trying to compensate for what I perceived as vulnerability I accomplished the opposite: I subjugated them to my own understanding of the world, the context, the music and our interaction. Because they could not speak within our regulatory framework they remained voiceless. This is an archetypal way in which the Western subject has dealt with the other, it is habitual response.\footnote{I will return to this question towards the end of the paper.} Much later I learned that I was wrong in my assumptions of Thuy and My. Yes, they felt they were in a foreign environment with limited latitude, but initially they had no problem with the interaction with us, other than my behaviour. After all, they were not only foreigners, they were also professional musicians ready to get involved in a new project. It is possible to argue that the predicament was now settled, that there was no need to dig deeper into the imbalance between the two subgroups. Part of my argument here, however, is that some of the behavioural patterns involved in an intercultural meeting such as this are centuries old and will influence the self even after the insight. As is so well described by Edward Said, for a change to be carried out it is not enough to speak of the asymmetry, it is necessary to also restore that which was once converted: 

\begin{quote}
  Formally the Orientalist sees him-self as accomplishing the union of Orient
  and Occident, but mainly by reasserting the technological, political, and
  cultural supremacy of the West. History, in such a union, is radically
  attentuated if not banished. Viewed as a current of development, as a
  narrative strand, or as a dynamic force unfolding systematically and
  materially in time and space, human history—of the East or the West—is
  subordinated to an essentialist, idealist conception of Occident and
  Orient. Because he feels himself to be standing at the very rim of the
  East-West divide, the Orientalist not only speaks in vast generalities; he
  also seeks to convert each aspect of Oriental or Occidental life into an
  unmediated sign of one or the other geo-graphical
  half. \citep[246-7]{said1978}
\end{quote}

In the following I will attempt to describe the continuous growth of the group that followed this initial, and largely unsuccessful encounter, and describe the evolution of our project through our interpretation of the song \emph{Tu Dai Oan}.


% These are the general questions that I will aproach in this chapter. The more specific inquiries % concern the role and identity of electronic instruments in these contexts, both in terms of how % I interact with them and how the output of my efforts interact with the sound of the acoustical % intruments.

\subsection*{ }
\label{sec:tu-ddai-oan}

\emph{Tu Dai Oan} is a popular traditional Vietnamese tune in the Oan mode. Although the idea of playing Vietnamese traditional music in The Six Tones occurred soon after our first meetings in 2006 \emph{Tu Dai Oan} was first picked up in 2007 when we started working on the version that we have since been playing. In Vietnam the song is very popular and it is often heard played on a dan tranh, an instrument on which the tune is naturally idiomatic. Stefan did a transcription for ten-stringed guitar and to have a greater control over vibrati and glissandi, he played it with a slide. Such ornamentations are important in the Vietnamese tradition and the musical mode depicts how, and where, to perform them.

The decision to do a trio version of \emph{Tu Dai Oan} for dan tranh, ten-stringed guitar and live electronics was an attempt to create a structure with a wide range of expressive opportunities. Being a plucked string instrument with a wooden resonance box, the ten-stringed guitar bridges the gap between the dan tranh, and the electronics. Distinct from the Vietnamese lute, the ty ba,\footnote{The Ty Ba is closely related to the Chinese Pipa.} the ten-stringed guitar, has many properties in common with the dan tranh. The challenge to create a coherent version of the tune was obviously not resolved merely by instrumentation, and the predicament we found ourselves in 2006 had to be avoided. Thuy is a master musician in the tradition, Stefan had at the time practiced playing Vietnamese music on his instrument for about six months and I had explored it for about the same time period. With no more than a rudimentary sense for Vietnames music Stefan and I had little understanding for the nuances of the tradition, and, at the time, Thuy had only just begun to explore contemporary Western music. Furthermore, we had barely no commen spoken language. With the ambition to create a shared space to explore the music without beeing too closely tied to neither Thuy's tradition nor our own while at the same time retain enough signifying traits from both styles of music to make them identifiable, performance was the only useful means of communication. Hence, we had to improvise. 

In his presentation at EMS 2006 \emph{Appropriation, exchange, understanding} British electronic music authority Simon \citet{emmerson06} points to how musicians have always exchanged concepts and ideas through the act of performance itself, often without language. But Emmerson also brings up how any mode of exchange involves some kind of distortion, reduction, impoverishment or loss, and continues: ``While some of these losses will be an inevitable result of global social change, the ethical question of knowledge and awareness cannot be avoided'' (ibid.).\footnote{This quote appears only in the abstract of the paper (see: \url{http://www.ems-network.org/spip.php?article292}, visited July 13, 2013), not in the paper itself.} Though our awareness of the complexity of the project grew over time, the social and political dimensions were part of The Six Tones since the beginning, but the important issue, raised by Emmerson, is how to identify the values that could be jeopardized in a collaboration. He also points out that the idea of intercultural music is commonly channeled through Western technologies such as notation, and performed using European performance practicies. The overarching goal with The Six Tones, however, was to dismantle the binary between distinction between East and West and not disregard any of the  performance traditions involved. We were looking for a dynamic meeting between the traditions, and exchange rather than apropriation of knowledge to be at the centre. Looking back at the process Emmerson is indeed correct when concluding that in intercultural projects there is a need to develop ``a sensitivity to different significant sound qualities and behaviours, as well as different aesthetic and cultural values, in a very practical sense, so that we are aware of what is lost in an intercultural transaction.'' (p. 8)

% I believe is how to identify the signifying trait of a voice. What is it that identifies the Vietnamese music? 

A working session and a concert at the The Vietnam National Academy of Music in the fall of 2006 turned out to be a pivotal moment in the development of The Six Tones. If the first meeting in Malmö was terribly hesitant and governed by failed attempts to counteract the perceived inequity within the group, the visit to Hanoi had a notable impact on the development of the project. To encounter Thuy and My in their own country made a tangible difference, reinforced by a temporary reversal of the roles as Stefan and I were now the visitors in a foreign country with little understanding of the codes and of the culture.

This was emphasized by learning about the gender roles in Vietnam, different to those in the West. A striking number of the positions held by women in Vietnam are in Europe traditionally held by men. The dean, as well as a majority of the other significant positions at the Academy of Music, are held by women and many of the tasks at the other end of the hierarchy, such as cleaners and secretaries, are carried out by men. According to \citet{VanKy2002} women in Vietnam has had a strong position historically, but the situation which we encountered in Hanoi is most likely influenced rather by the responsibilities Vietnamese women had to carry in the Vietnam War than by the histoical evidences of matriarchy. However, to experience this subtle but important difference clearly affected the relations in the group, and whether or not there had ever been good reasons to treat Thuy and My cautiously as fragile, sensible and subordinate women, seeing them in their home country made it clear there was no longer any need to do so.\footnote{For a more in depth and comprehensive account of gender roles in Vietnam, see \citet{drummond2004}} As a consequence, thanks to a merely rudimentary insight into the Vietnamese society, the field for interaction in the group had a radically altered premise, the main difference being the way in which I could approach Thuy and My. I was to some extent able to break free from my preconception of them as foreigners, victimized by definition, break free from the standard response, the habit. I was now able to listen.
%To put it in the words of \citet{spivak1988}, I had learned to let the subaltern speak, and by now I was listening. 

It may seem self evident that a closer contact with a foreign culture and social system, the music of which one is interacting with, and attempting to get better acquainted with, results in a more natural and less strained communication, and that the opposite, lacking first hand information about the specific music and culture, results in the kind of confusion that was seen at the unfortunate start of the Six Tones. However, the issue at stake here is not solely epistemological. Gregory Bateson claims that in: 
\begin{quote}
  the natural history of the living human being, ontology and epistemology
  cannot be separated. His (commonly unconscious) beliefs about what sort of
  world it is will determine how he sees it and acts within it, and his ways
  of perceiving and acting will determine his beliefs about its nature. The
  living man is thus bound within a net of epistemological and ontological
  premises which--regardless of ultimate truth or falsity--become partially
  self-validating for him. \citep[p. 314]{bateson72:steps}
\end{quote}

Looking at human behaviour as a holistic, cybernetic system in the way Bateson is suggesting, one may return once again to the first rehearsal and look at it as an unstable system with no means for self-correction. With good intentions I tried to compensate for a postulated inequality assuming that I, as in my `self', could correct the imbalance. According to Bateson that would have been impossible. The stability of a complex system, such as a group of musicians playing together, is a function of the product of all the parts of the system (of all the ``transformations of difference'' (p. 316) as Bateson calls it) and there is no way in which one part of the system can control all other unilaterally. On the contrary, at any point in time, every part of the system has to adapt their actions according to information from within the system. In other words, the problem was not so much that we did not have a language, but that we were unequipped to pick up the existing information within the group, and adjust accordingly. To instead fall back on habit, as I did, thinking that the self, by itself, can counterbalance lack of information is a characteristic of an Occidental attitude which, according to Bateson, has a cultural and social predisposition towards thinking about the self as a delimited agent performing purposive action upon objects rather than seeing the holistic aspects of the system. Although arriving at the issue from very different angles there is a parallelism between Bateson's Occidental self unable to see himself as part of a larger, mutually dependent system, Said's description of the Orientalist reasserting the cultural supremacy of the West and Emmerson's appeal, in intercultural projects, to develop sensitivity towards different aesthetic and cultural values. All of them indentify a problematic aspect of the Western self in the encounter with the non-European other. 

It was not until the year after our first visit in Hanoi that we started working with \emph{Tu Dai Oan}. In the years to follow we would play it numerous times and we continued to develop the form and the expression. Although referred to as an improvisation \emph{Tu Dai Oan} is part of a tradition of playing that is in fact very fixed and which only allows for a limited set of possible permutations. Even if our intention was not primarily to propagate a traditional style of playing we hoped to maintain enough significant traits of the original piece for the music to be recognizable as coming out of the Vietnamese musical heritage. Learning from our earlier experiences our method was to move forward cautiously and in constant dialogue with Thuy, the only one with a solid experience from playing traditional Vietnamese music. In many of the rehearsals\footnote{The rehearsals discussed here were carried out at the Electronic Music Studios in Stockholm (EMS) in the late winter of 2009 and we have near to complete video recordings of them. The actual events in the rehearsals are discussed in greater detail in \citet{Ostersjo2013}} we let Thuy have the initiative while Stefan and I would stay in the background, only occasionally fronting ideas or commenting performances. At this point we already had some common experiences as well as a greater understanding for our respective musical, social and cultural backgrounds, but we were still in absence of a common language, which is, returning to Emmerson, not a seldom situation in intercultural projects. 
In preparation for the upcoming Scandinavian tour, as part of my repository I had made a virtual instrument with the intention to make it sound like a combination of dan tranh and dan bau. This was initself a break with how I usually work. Generally more interested in sounds that are the result of an interaction, to prepare a pitched instrument to play notes was not at all natural to my electronic music practice. There was however a need for me to be able to take initatives to which Thuy could react. A property of live interactive electronic music in real time, with a focus on processing or live sampling is that no sound can be produced until a sound has been received and I had seen that this was a problem in our group. In order to become more interactive I had to become less so. Once I arrived at a relatively satisfying compromise with the instrument, I entered all the pitches of \emph{Tu Dai Oan} in an array and let the instrument step through the pitches for each trigger, allowing me to focus on the timing of the onset, the intonation and the glissandi. Though there was nothing wrong with this instrument the desired effect, to aurally bridge the conceptual and cultural gap between the acoustic Vietnamese dan tranh and the electronics, was lacking. In fact, the more I obscured this instrument from its original sound, the more effects I added to it, the better it seemed to be working. The concept of it, and its conceptual afiliation to the glissandi and vibrati of the other instruments appeared to be more important than the sound quality. Even after the modifications of it, however, this simple synthesizer was so far from the techniques I would ususally use, that to actually play it in performance was incredibly awkward in the beginning. I had to forcefully break with my aesthetics and habits of playing electronics, the reward being that the performances worked really well.


Not being able to efficiently discuss and negotiate the performances in the rehearsals made it necessary to use a trial and error method. By going through short iterations of cycles of play--evaluate--alter, we slowly raised our awareness of what could work. Through this method we were able to not only practice our own communication, or refine our group as a cybernetic system, it turned out to also be an effective way to teach each other about some of the specifics of our respective playing traditions and, perhaps even more importantly, to learn how to negotiate parts of our performance traditions. One of the early choices we did concerning the form was to expand the traditionally rather free introduction--the part of the tune where the performer had real freedom to improvise in a Western sense. We inserted an improvised section in the middle, and at the end we added an extended improvisation. The dan tranh and the guitar played the tune and I joined in primarily in the improvisatory sections. This was a form kept intact over the years of playing \emph{Tu Dai Oan}, in itself a way to keep an alliance to traditional Vietnamese origin of the tune, but it was also an efficient way to allow ourselves to re-negotiate the details of the structure of the performance. By staying with the form we could develop our musical interaction and attune ourselves to the ``transforms of differences'' \citep[p. 318]{bateson72:steps}, something that allowed for quite radical changes to the shape of the form.

Simon \citet{emmerson06} is warning us of the risk of masking, that some aspect or porperty of one sound will obscure some property of another. This line of thinking can be expanded to the level where one culture may mask another. Colonialism among other things, has resulted cultural appropriation or cultural imperialism, and in The Six Tones appropriation was something we thought we could identify, but the subtle concept of masking is difficult to spot.
Masking will probably occur to a certain extent in any kind of music, but the question asked by Emmerson: ``have we masked something ‘significant’ as seen from within the culture?'' is material: It is not so much \emph{if} something is lost as \emph{what}, what the importance of this property is, and from what perspective the loss may be experienced. Perhaps my physical modeling instrument was problematic as a virtual copy of the dan tranh than it was as a deviation from the real instrument? Perhaps it masked the Vietnamese instrument due to its similarity with it, but reinforced it as it grew different? 
The way I developed my part in \emph{Tu Dai Oan} was in effect a movement away from my initial respect for the Vietnamese tradition instead approaching a genuinely experimental mode of playing. However, at the same time, this development had nothing to do with a lack of respect, quite the contrary. I feel inclined to argue that the attitude I had in the beginning of the project, before our first trip to Hanoi, was lacking in respect. My assumptions about Vietnamese music, though constructed in good faith, were not informed by the tradition itself but rather by my prejudices about it.
When \citet{emmerson06} warns us about masking and writes that if the exchange continues, ``in time the masked element may disappear as it no longer functions within the music'' we must not take it too literal. In our case, to think that we could erase or destroy parts of the living tradition of Vietnamese music would be to overestimate the influence and power of our group. Regardless of the validity of this otherwise legitimate concern, our experience in The Six Tones is that we could go quite far mixing the two modes of expression without any masking of significant traits in the original music. The more important attunement performed was that in the social dimension. As this grew stronger, our musical artifacts did also.

Finally, it is interesting to note that while Stefan saw it necessary to go much deeper into the theory and practice of playing Vietnamese traditional music, I was more concerned with \emph{not} making authenticity a parameter in my playing. One reason is that Stefan's instrument has a certain affinity with the instruments we worked with, whereas electronics finds no evident response in the Vietnamese musical tradition. Over time I became more and more audacious in my experiments with the music.\footnote{By audacious I mean that, although I am less concerned with right or wrong, and less focused on the history and idiomatics of the tradition as it takes shape in my own playing, I obviously have a deep respect for the Vietnamese musical tradition as it is carried on by master musicians such as Thuy and My.} The consequence of this attitude was that in concerts I took bold chances, sometimes resulting in `errors' and some of these experiments eventually resulted in changes to the dynamics of the form. The effect of my `error' assumed the unintended function of Coleman's violin, the ``abrupt disappointment of expectations of meaning'' that makes us reconsider what we heard and how we experienced it \citep{barthes68:death_of}. Returning to the idea of the group as a cybernetic system we can use Bateson's language and arrive at the experimental conclusion that once the system has come to a point where transforms of differences are communicated efficiently between the different parts even great discontinuities, such as my errors, are handled well. I will argue, however, the basis for that claim is that experimentation is an agreed method and, most importantly, that the self is prepared to break with habits and listen to the other.

% Following this, maybe we can arrive at an experimental redefinition and subcategory of Emmerson's concept of masking: Masking occurs when the understanding of the conditions for interaction is weak or missing altogether. Even though the skills of the performers may be outstanding, if the members are unable to contextualize events and enterprise of the other members, in relation to the (sub)culture they are a part of, the result will lack in contour and some significant aspect of the expression may get lost. 

% \section*{Center and periphery}
% \label{sec:center-periphery}
\section*{ }
\label{sec:discussion}


\begin{quote}
  A clarified political and methodological commitment to the dismantling of
  systems of domination which since they are collectively maintained must [\ldots] be collectively
  fought. \citep[p. 215]{said2000}
\end{quote}

There is an obvious tendency to always look at western art music as the center and whatever is external to it as the periphery. The Eurocentric view is rooted in the concept of the West as the social, economical and political focal point in the world in which the music of, say, a Vietnamese musician will always grounded in the periphery. As such, it can serve as a peculiar and colorful complement but never engage in an encounter with the West on equal terms. Attributing values to it such as `beautiful' or `masterful' does not change its locus and does not move its status closer to the center. Quite the opposite: aestheticising the other, or the expressions of the other, is an effective way to keep it locked out. Many writers and scholars have brought these issues to discussion. Apart from the works already cited in this paper, to only mention a few, we find George Lewis using Somers' ideas of the epistemological other to discuss the situation of African American jazz musicians \citep{lewis-1}, post-colonial theorist and philosopher Gayatri Chakravorty Spivak who rhetorically asks \emph{Can the subaltern speak}?, Edward Said discussing the enormous inequality in the war in Palestine in \emph{Permission to narrate}, and Gloria Jean Watkins, also known as bell hooks, who approaches her own background in racist America in the significant text \emph{Marginality as site of resistance}\cite{HooksBell1990}. 

Deleuze's and Guattari's supposition that in transcoding, becoming-other is a way to resolve the opposition between self and other, east and west, center and periphery, is quite forcefully rejected by  \citet{spivak1988}. Taking a broad view on the world she is posing important questions concerning the continuous marginalization of those without, or with only limited, access to the sources of cultural imperialism. In her surevey, as has already been mentioned, because the Eurocentric subjectivity, according to Spivak epitomized by Deleuze and Foucault, threatens to further obscure the subaltern:\footnote{There have been attempts to seek redress for Deleuze, Guattari and Foucault and prove that their thinking was not rooted in Eurocentricity and that it will not necessarily lead to oppression of the other \citep[See e.g.][]{robinson2010}. }
\begin{quote}
  It is not only that everything they read, critical or uncritical, is caught within the debate   of the production of that Other, supporting or critiquing the constitution of the Subject as   Europe.  It is also that, in the constitution of that Other of Europe, great care was taken to   obliterate the textual ingredients with which such a subject could cathect, could occupy (invest?) its   itinerary. \citep[p. 75]{spivak1988}
\end{quote}

What is the significance of these complex issues concerning economy, hyper-capitalism, world domination and post-colonialism in the context of contemporary music? How can the deconstruction of the concepts of center and periphery be applied to the artistic practice of a group consisting of two Vietnamese and two Swedish musicians? Why is it necessary to consider inherited power structures when approaching the seemingly simple task of creating a workable platform for musical and cultural interaction? What impact does it have on the notion of the self? My hypothesis here is that the self is constituted of behavioural habits, conscious as well as unconscious, as suggested by \citet{bateson72:cyber-self}, and that these are culturally encoded with ideas concerning freedom and individuality, and in the arts these are built on the idea of artistic projective self \citep{frisk2013}. Although it is easy to understand that the habits and the cultural codes are different in other cultures, according to postcolonial thinking, knowing is not enough \citep[See e.g.][]{said2000,frisk-ost13}: to let the other speak, and to allow oneself to listen, it is necessary to break some of these habits. After several years of working together my experience is that, within the context of The Six Tones, I do not have the experience that I have to limit my artistic latitude, nor that any of the other members do it. The reason we have arrived here, however, is that we initially worked consciously with breaking our habits and limiting our freedom some of which I have described above. I argue that artistic practice is a useful arena for exploring these and similar questions. The topic that I have brought up in this paper, to foster the social and political dimension of musical interaction through improvisation, by exploring the self and the consequences of freedom and habit formation, may be successfully investigated through the practice itself. 

At a time when art in general, and music in particular, is commodified to a degree that not even Adorno could have anticipated, artistic research is one of the few remaining fields that has the potential to withstand entrepreneurial tendencies in the music academies and within the field of music itself, and ceaselessly engage in the important artistic and social questions that lay ahead of us. 

% At the first meeting with Thuy and My, based on conjecture instead of knowledge, I started making choices and taking actions to protect them from the institutionalized impact of western influence. In the process I accomplished exactly that which I was trying to avoid: marginalizing the other. I was so painfully aware of the structural imbalance between us that I reinforced, rather than rectified, our social relation. 


% How, then, can I freely explore my individual, subjective and original artistic expression while still remaining open to the other? How can I avoid that my freedom limit the freedom of the other? 


% In my experience it is dificult to make general assumptions based on artistic practice. The solutions we came up with in The Six Tones may or may not be applicable in another context. The approach, however, may be generalized and applied to other and similar contexts. To go beyond the purely musical and consider the interactive potential in the sociopolitical realm, a process through which the necessary data may be acquired to then go back and develop the musical possibilities. 


 

%Failure to understand the wider consequences, or a malfunctioning relation between center and periphery, in which transcoding, to use the vocabulary of Deleuze and Guattari, is unattainable, will inescapably lead to asymmetrical relations and inequality. If we do not manage to put this, essentially colonial view behind us we will remain trapped in our self-centeredness. 
%It should by now be clear that Thuy and My were subaltern at the time of our first rehearsals described above. Both in terms of their social and political situation in the context within which they operated and in terms of how I approached them. 


%\nocite{biggs10}
\bibliography{/home/henrikfr/Documents/articles/biblio/bibliography.bib} \bibliographystyle{chicago}
\end{document}

%Morwenna Griffiths, in her text \emph{Research and the self} argues that ``arts-based, practice-based research needs to address the issue of the self of the researcher'' \citep[][p. 167]{griffiths10}. The subjective nature of all artistic practices and artistic research gives the self a central position in all aspects of the different processes of creation, evaluation, construction and presentation. . In line with her argument, I think it is safe to put forward that the self is not easily defined. 

% In order to build a preliminary model of the self that may serve as a point of departure I will draw on Griffiths' model, which in itself is ``influenced by Arendt's concept of `the human condition'''. Not as building blocks or as parts that are all interrelated to one another, but rather as leaking and fluid concepts that taken together may form an exploratory field of reference:

%A significant type of transcoding in the general sense according to Deleuze and Guattari is that when a code receives fragments or parts of a different code such as the fly, part of whose code is found in the spider's web. This becoming-other is carried out in a creative act of confirmation of the other: ``Whenever there is transcoding, we can be sure that there is not a simple addition, but the constitution of a new plane, as of a surplus value. A melodic or rhythmic plane, surplus value of passage or bridging.'' \citep[p. 346]{deleuze80}.

%Becoming-other is Deleuze's and Guattari's concept of transgressing the border between self and other, a transparent subject that is allowed to move across the borders: ``Whenever there is transcoding, we can be sure that there is not a simple addition, but the constitution of a new plane, as of a surplus value. A melodic or rhythmic plane, surplus value of passage or bridging.'' \citep[p. 346]{deleuze80}.
%In their discussion on becoming-other Deleuze and Guattari identified what they saw as a possibility to move across these borders: ``Transcoding is one of the Deleuzian neologisms employed to underline an element of creativity, of invention and of crossing – traversing – borders between `self' and `other' '' \citep[][p. 140]{semetsky2011}. In this formative, or rather transformative, activity they discussed as transcoding, not only self and other could approach each other
%That the idea of transcoding may be problematic and an expression of a generally Eurocentric view is a topic raised by Gayatri Chakravorty Spivak in an essay that we will return to later. For now, we should consider the significance of the references to music in the quote from \emph{Mille Plateaux} above. 


%The first mistake, as is shown by Hooks, is to think of marginality as a space one wishes to surrender and give up to instead gravitate towards the center. Describing the other side of the railway tracks (the dividing line between the two communities) as the site of the center, she is also identifying marginality as ``the site of radical possibility, a space of resistance'' (ibid., p. 341). Hooks and her friends and relatives would at times trespass into the other domain, to work ``as maids, as janitors, as prostitutes'' and when they did ``there were laws to ensure our return''. But to refuse to give in to the expectation of wanting to relocate from the margin to the center is to invalidate these laws and dismantle their meaning. In a globalized world we may think that the regions constituting the inside and the outside are large continents and political systems such as east versus west, or democracy versus dictatorship, but the local demarcations still exist. Vietnam, being both politically and economically in the periphery, is in every respect marginalized as the other, the foreign, the different and the obscure. And, just as there were laws for Bell Hooks and her friends when they trespassed, there are laws to ensure the return of Vietnamese visitors in Sweden. We may think that the global perspective has broadened our view on the world and blurred the boundaries but in the eyes of the legislators in the west there is no doubt as to what is the center and what is the periphery. 

% Deleuze's and Guattari's supposition that in transcoding, becoming-other is a way to resolve the opposition between self and other, east and west, center and periphery, is quite forcefully rejected by  \citet{spivak1988}. In the already classic essay \emph{Can the subaltern speak?} she is taking a broad view on the world, posing important questions concerning the continuous marginalization of those without, or with only limited, access to the sources of cultural imperialism. In her surevey, as has already been mentioned, because the Eurocentric subjectivity, according to Spivak epitomized by Deleuze and Foucault, threatens to further obscure the subaltern.
%\begin{quote}
%   It is not only that everything they read, critical or uncritical, is caught within the debate   of the production of that Other, supporting or critiquing the constitution of the Subject as   Europe.  It is also that, in the constitution of that Other of Europe, great care was taken to   obliterate the textual ingredients with which such a subject could cathect, could occupy (invest?) its   itinerary. \citep[p. 75]{spivak1988}
%\end{quote}
%There has been attempts to seek redress for Deleuze, Guattari and Foucault and prove that their thinking was not rooted in Eurocentricity and that it will not necessarily lead to oppression of the other \citep[e.g.][]{robinson2010}. 

%However, the main thread of Spivak's argument, according to me, is not concerned with what is done or not done to the ontological other, but with the particulars of the perspective of the other:

% \begin{quote}
% Can the subaltern speak? What must the elite do to watch out for the continuing construction   of the subaltern? The question of `woman' seems most problematic in this context. Clearly, if   you are poor, black and female you get it in three ways. If, however, this formulation is   moved from the first-world context into the postcolonial (which is not identical to the   third-world) context, the description `black' or `of color' loses persuasive   significance. The necessary stratification of colonial subject-constitution in the first phase   of capitalist imperialism makes `color' useless as an emancipatory signifier. Confronted by   the ferocious standardizing benevolence of most US and Western European human-scientific   radicalism (recognition by assimilation), the progressive though heterogeneous withdrawal of   consumerism in the comprador periphery, and the exclusion of the margins of even the center   periphery articulation (the `true and differential subaltern'), the analogue of   class-consciousness rather than race-consciousness in this area seems historically,   disciplinary and practically forbidden by Right and Left. \citep[p. 90]{spivak1988}
% \end{quote}


%Slavoj Zizek draws a line from the Lacanian notion that justice as equality is founded on ``our envy of the Other who has what we do not have, and who enjoys it''. The response to the unevenness in the innocent context of this first rehearsal with the Six Tones was based on the thesis that, though I could not compensate for what I perceived of as a lack of possibilities, I instead reduced my own latitude consistent with what Zizek refers to as equally shared prohibition:

%\begin{quote}
%  \ldots coffee without caffeine, cream without fat, beer without alcohol, [\ldots] warefare without casualties, [\ldots] politics without politics, up to today's tolerant liberal multiculturalism as an experience of Other deprived of its Otherness (the idealized Other who dances fascinating dances and has an ecologically sound holistic approach to reality, while features like wife-beating remain out of sight).
%\end{quote}



%%% LAGT TILL UNNECESSARY
%What Bell Hooks is referring to in the text cited above is an institutionalized oppression and marginalization that has been going on for centuries and clearly operates on a completely different scale compared to the Six Tones. Merely reading about it will not let me understand the experiences described. What it does allow me to do, however, is to understand that the effects and the processes in the development going on locally in The Six Tones was similar to those operating on a larger scale. In a two way process the practice contributed to making me aware of the issues, allowing us to continue to highlight the instinctive tendency to treat the other based on the assumption that we hold the best solution. It is in this sense that almost all artistic activity also has a potential to engage in a political consideration, reflection and introspection. Not in the meaning that the artistic expression itself needs to be politically imbued, but rather that the site for artistic practice and artistic research is taken advantage of as a site also for politically oriented questions. 

%This is at the core of the issue: 


% And this is our problem. We do not know how you learn without domination, without power. We don not know % what it means to learn without succombing to the stuctures of those who teach. Our schools have alays been % structured in terms of domination, in terms of the subject giving in to the greater means. We, at the % center, are now the the ones who need to learn what it means to be in the margin and those in the margin are % naturally accepting a position in the structural center, without giving up there privileged position outside % the center. 

% \section*{Subjectivity}
% \label{sec:habit-self}

% The beginning of contemporary jazz that exploded in the 1950s with the Be Bop movement preceded the related reaction against serialism most prolifically promoted by John Cage has not been acknowledged for its influence. Chance operations and indeterminacy, so effectively denounced by Pierre \citet{boulez64} in his article \emph{Alea} claims the composer ``has chosen henceforth to be meticulous in imprecision'' (p. 44) and rather condescendingly talks about what they perceive as the failure of indeterminacy in composition. There is no doubt that this debate was heated nor that Boulez was absolutely convinced that the best means to compose music was to be meticulous in \emph{precision}. According to Boulez, the composer is making the choices and is the speaking subject, to think or attempt to do otherwise is plain wrong (ibid.).

%\begin{quote} 
%In spite of best intentions and most earnest attempts, I am unable to make out the precise reason for this fear to approach the true problem of composition. Perhaps this phenomenon also is due to a kind of fetishism of numeral selection--a position that is not only ambiguous but completely unsound when the work under investigation structurally refuses these procedures, which are, after all, excessively coarse and elementary. \citep[p. 44]{boulez64}
%\end{quote}

% Cage, the main proponent for indeterminacy at the time, had a corresponding lack of understanding and interest for jazz, in many regards seemingly similar to some of Cage's ideas. George \citet{lewis-1} contextualizes this relation, or lack of relation, and in his widely influential paper \emph{Improvised Music after 1950: Afrological and Eurological Perspectives} brings up Cage's discussion on jazz with the journalist Michael Zwerin\footnote{In the interview Cage is invited to share his thoughts on jazz and agrees to do it while at the same time stating that jazz is not something he thinks much about at all. \citep[In]{lewis-1}} within a sociological context: ``The work of black artists is defined by whiteness as the primitive (yet improving)   work of children'' \citep[p. 104]{lewis-1}.

%\begin{quote} 
%The colloquy between Cage and Zwerin [\ldots] displays whiteness in its defining   role. Zwerin, though supposedly taking the side of jazz, ends up agreeing with Cage that jazz could   use some work. The work of black artists is defined by whiteness as the primitive (yet improving)   work of children: `But jazz is still young, and still evolving'; jazz could benefit from serious   study of `our' models; already, it has started to explore areas `suggested by Ives'; `jazz is   getting freer' though the use of tone-rows, and `getting away from the time dependence--inferring   it rather than clobbering you with it all the time'; and so on. \citep[p. 104]{lewis-1} 
%\end{quote}

% These two events, Boulez article and Zwerin's interview, took place only a few years apart in the mid 60's and they clearly have something to say about the changes music was going through at the time and the antagonism that existed: The great European composer is hitting on the great American composer in turn hitting on jazz in general and black musicians in particular. The New York School of composers threatened the authority and the dominant power of the composer. Cage, while proposing a very open and decentralized attitude towards music and art was until his death in fact a very strong and influential proponent of his own view of the world of music. And, as much as he liked to remove intention from his performances, and regardless of his enigmatic writings on the subject, he was the one setting the boundaries for his compositions (most of them) and he published them as scores just as Boulez would publish his. 

% However, it is not so much these debates themselves that are my interest here, but the position of subjectivity portrayed. To Boulez it is unthinkable that anyone else than the composer can be the subject; not the performer nor the listener, it is the composer who is speaking. The aesthetic turmoil at the time is likely to be one reason for these debates, the social and racist order another one, but I find it difficult to not see Boulez' attack on indeterminacy and Cage's patronizing attitude towards jazz as a defensiveness towards a means to organize musical material that could threaten their own respective power positions. A circle as an image of center and periphery turns into a spiral with multiple centers and multiple margins, but the structure is retained. The role of the composer, in itself a relatively new invention introduced at the time when notation divided the musician in two parts, the originator and the executor \citep{wis96,frisk-ost06-2}, is in itself under attack when the improviser rejoins these into one and the same agent. The binary division between composer and musician is deconstructed by the improviser.

% When writing the history of AACM, which was formed also in the mid 1960's, George Lewis gives us some insight in the discussion concerning composers and musicians. A suggestion to have a requirement that the music of AACM should be original was turned down which opened up for a different view on composition as a collaborative process. Rather than adjusting to the common view, the perspective in the center, AACM allowed themselves to formulate their own take on their activities. According to Lewis, they avoided to:

% \begin{quote}
%   reproduce the division of labor between `composer' and `performer' that characterized Western classical   music. Rather, to these musicians, being `m musician' meant working out of a hybridized model of creative   practice that negotiated between individuality and collective membership, and which assumed primary   creative agency for each artist. \citep[p. 103]{lewis2008}
% \end{quote}

% As the binary opposition between free and non-free is questioned it is no longer a problem that jazz, holding improvisation in high esteem, also contains compositional elements as an integral part of its creation, nor  that the performance of improvised musics rest on a disciplined practice with a firm set of tools for musicians to use to accomplish their expressive goals. For these reasons, perhaps self evidently, it is important to acknowledge the different dynamics of the terminologies, or, as is so eloquently put by Sara Ramshaw: ``Improvisation can be neither purely spontaneous nor completely determined by the   musical structures with which it engages. It must be both responsive to otherness and have   some stable or determined dimension in order to endure as jazz improvisation'' \citep{ramshaw2006}.

% and to the protesting against the oppression of African Americans all over the country, 
%and in the political domain Arendt finds a possible discourse concerning freedom:

% \begin{quote}
% The field where freedom has always been known, not as a problem, to be sure, but as a fact of everyday life, is the political realm. [\ldots]
% %And even today, whether we know it or not, the question of politics and the fact that man is a being endowed with the gift of action must always be present to our mind when we speak of the problem of freedom; for action and politics, among all the capabilities and potentialities of human life, are the only things of which we could not even conceive without at least assuming that freedom exists, and we can hardly touch a single political issue without, implicitly or explicitly, touching upon an issue of man's liberty.  
% Freedom, moreover, is not only one among the many problems and phenomena of the political realm properly speaking, such as justice, or power, or equality; freedom, which only seldom---in times of crisis or revolution---becomes the direct aim of political action, is actually the reason that men live together in political organization at all. Without it, political life as such would be meaningless. \citep{arendt77}
% \end{quote}

%Sara \citet{ramshaw2006} is similarly focused on the political domain and its impact and interrelation with musical freedom and improvisation. She employs a deconstruction on the concepts of invention and improvisation, ``both constituted in their singularity'' by which they can be ``starkly contrasted to the dominant conception of Western law''. After all, law is more often than not expected to privilege ``generality and universality over unpredictability and arbitrariness.'' 
%Ramshaw turns to Derrida's discussion of the difficult interdependence of the singular and the general. By applying this notion to the critical study of jazz improvisation she contest that improvisation is an expression of ``sheer spontaneity'': ``A deconstructive reading reveals that improvisation so defined can be neither total in jazz nor totally absent in law. Instead, the singular event exists solely as aporia in both fields.''

%\begin{quote}
% tension consequently exists between the ``spontaneous'' conception of jazz improvisation and   the more context-driven model. This tension is intrinsic to jazz improvisation   itself. Improvisation can be neither purely spontaneous nor completely determined by the   musical structures with which it engages. It must be both responsive to otherness and have   some stable or determined dimension in order to endure as jazz improvisation. \citep{ramshaw2006}
%\end{quote}

%In Ramshaw's deconstruction of the idea of improvisation as sheer spontaneity, however, the ``unpacking of the aporetic nature of singularity reveals the [\ldots] necessity of jazz form''. In other words, a

%But, to further complicate things, although it is true that improvisation is not solely a spontaneous process that creates itself in the moment, does not rule out that precisely this aspect may be seen as its driving force. While spontaneity is not the only signifying aspect of musical improvisation it may still be part of its momentum, part of what makes it a unique communicative process. The spontaneous act works from the inside, but this fact is obviously not unique for jazz and improvised music but an important property of all artistic activities. 

%The personal expression is of great importance in many art forms. In jazz, to develop a sound of your own is critical and many of the great jazz musicians such as Coleman Hawkins, Betty Carter, Charlie Parker, Billie Holiday, Lennie Tristano, Carla Bley, Albert Ayler have in some ways redefined and stretched the limits of their instruments through their highly skilled, individual and original output. Their \emph{sound}, i.e. not only the sound of their instruments, their articulation and phrasing but also their harmonic and melodic language and their general aesthetic, has become their particular musical identity. This is often referred to as `being personal' and to lack a well molded identity of this kind is often regarded a failure. In my own schooling as a jazz musician, the one thing I remember most clearly is being reminded of the importance of having a personal sound. Drawing on Heidegger's notion of \emph{Eignetlichkeit} \citet{benson03} sees the expectation of making `the piece my own' as an aspect not limited to jazz improvisation, but common also to classical Western music and, above all, very important: ``So `making the piece one's own' and `being oneself' are in some sense \emph{necessary} to a good performance'' (p. 166). 

%Gregory Bateson's antropological studies on how the habit systems of one sex in a community not only is different, but related and interdependent to those of another sex in complementary patterns such as dominance--subdominance and succoring--dependence. This strikes a chord with postcolonial theory where the identity of a decolonized nation or group of people continue to be defined by the colonizer in a binary us-and-them relation.

% My argument in this chapter is that all of these matters have a critical impact on the individual and collective artistic processes. To limit the thinking to intramusical parameters will not reveal the true nature of the complexity of an artistic interaction such as the one described here. Musical autonomy has been questioned for many years but my suggestion here is that if experimentation in intercultural interactions is to take place it is necessary to consider the field of social interaction.

% Furthermore, my musical thinking is likely to be influenced by these questions anyhow. How can I engage in a true musical encounter and experimentation with the Six Tones that avoids simply layering material?
% How can I avoid that my own playing obscures the identity of the Vietnamese music as Vietnamese and still contribute to the sound of the group? How can I avoid that my own musical input takes over, masks the other elements, while at the same time remain close to my personal ambitions and aesthetic preferences with the music? 

%According to \citet{VanKy2002} women in Vietnam have developed a special fortuity ever since the dissolution of the Confucian predominance of men over women and the Vietnamese folklore has numerous indications to the domination of female over male. In the 20th century, ``during the period 1929–36, there were three feminist newspapers or at least newspapers that supported women’s rights'' and it is clear that ``Vietnamese women were claiming their equal rights in a male society'' (\emph{ibid} p. 101). In other words, 
%Gender was now a . and it turned an aspect of the group to which I had reacted unreflected into one which was now part of the dynamics of the group. 

% but the attempt to transgress these borders is not simple and proved to be impossible in the project described below. Not even in the abstract realm of music is it without risks. As intriguing the ideas put forward by Deleueze and Guattari are, they have also been criticized, and Gayatri Chakravorty Spivak finds the idea of the transcendent Eurocentric subject that makes itself transparent problematic in the ways it ignores the long standing structures of power and oppression: ``Spivak's principal claim in \emph{Can the Subaltern Speak?} concerns the `transparency' of the subject. In her view, Deleuze `restores[s] the category of the sovereign subject' (p. 278), by which she means that in the manner of classical Western philosophy, he deploys an essentialised subject of oppression, which acts with respect to the object of a singular emancipatory project'' \citep[][(citations from Spivak, 1988)]{robinson2010}.

% Deleuze's and Guattari's supposition that in transcoding, becoming-other is a way to resolve the opposition between self and other, east and west, center and periphery, is quite forcefully rejected by  \citet{spivak1988}. In the already classic essay \emph{Can the subaltern speak?} she is taking a broad view on the world, posing important questions concerning the continuous marginalization of those without, or with only limited, access to the sources of cultural imperialism. In her surevey the Eurocentric subjectivity, according to Spivak epitomized by Deleuze and Foucault, threatens to further obscure the subaltern.
% \begin{quote}
%   It is not only that everything they read, critical or uncritical, is caught within the debate   of the production of that Other, supporting or critiquing the constitution of the Subject as   Europe.  It is also that, in the constitution of that Other of Europe, great care was taken to   obliterate the textual ingredients with which such a subject could cathect, could occupy (invest?) its   itinerary. \citep[p. 75]{spivak1988}
% \end{quote}
% There has been attempts to seek redress for Deleuze, Guattari and Foucault and prove that their thinking was not rooted in Eurocentricity and that it will not necessarily lead to oppression of the other \citep[e.g.][]{robinson2010}. 
