\documentclass[a4paper]{article}
\usepackage[swedish, english, icelandic]{babel}
\usepackage[T1]{fontenc}
\usepackage[authoryear,round]{natbib}
\usepackage{url}
\usepackage[utf8]{inputenc}

\usepackage[pdftex]{graphicx}
\renewcommand{\encodingdefault}{T1}
\renewcommand{\rmdefault}{pad}
\usepackage{graphicx}

\usepackage{fancyhdr}
\pagestyle{fancy}

\lhead{\small{\textit{Henrik Frisk}}}
\chead{}
\rhead{\small{\textit{Improvisation and the self}}}

\title{Improvisation and the self: to listen to the other}
\author{Henrik Frisk\\{\small Assistant Professor}\\{\small Royal College of Music in Stockholm}\\{\small henrik.frisk@kmh.se}}
\date{\today}

\begin{document}
\selectlanguage{english}
\maketitle

\thispagestyle{empty}

\section*{Introduction}

%Se sid 148 in Proust and The Signs, slutet av första stycket.

To listen to the other. The topics implied by that phrase are almost too convoluted even to begin to try to unwrap. To listen to one self is not easier. If we demarcate the field and talk about the improviser listening to the other we arrive at one of the central topics in learning and teaching improvisation. To listen to the other. (In this context, to listen to one self is even more difficult in my opinion.) Generally, what is meant by this invitation is to listen to the other musicians in the group. Should the improviser not listen to the audience?\footnote{In 2012 there was an interesting debate on the topic initiated by saxophonist Greg Osby. The discussion attracted a lot of different musicians and made it clear that it is not necessarily the right thing to listen to the audience and the question may not be so easy to attack. See \url{http://www.openskyjazz.com/2012/03/greg-osby-on-the-audience-in-general/}} When we listen, what is it we allow ourselves to be influenced by and what part of our own expression should remain untouched by our listening? Obviously, the point of listening to the other in performance is not completely to give up the self, to become the other but to attune to, or find resonance with the other. It is in the interaction that the open and unbound improvisation is unfolding, between adjusting to the other while hanging on to the possibility for taking the imitative. There are no general wrongs or rights because only in the instantaneous moment can one decide what path to take. At one point, there may be a demand for absolute and unconditioned control, and at another it may be necessary completely to give in to the other. Yet another situation will require one to go with the sound, the audience or the space. 

In this paper, I will use my own artistic practice within a Swedish-Vietnamese group as a context for approaching some of these questions. Going through a preliminary limitation of the concept of the self in artistic production, pointing to the importance of dismantling the mysteries nature of the self. In artistic research, the self and the subject has a central role, and for that reason it is important to develop ones sense of its agency. The impact of freedom, being such an essential concept in the thinking on improvisation, is discussed. Freedom is also closely related to some of the more social and political topics brought up. Different means to detach habit from the self to arrive at a more spontaneous and unconscious expression of freedom is brought up after which I will present the group whose work has given rise to this paper. Situated as it is in an intercultural context the meanings of self, other and listening are modified and expanded which allows us also to revisit and rethink their original context. Ending in a discussion on the center and periphery in a post-colonial context, we have moved from the conceptually, stylistically as well as geographically, local to the global perspective. As a preliminary remark, I believe that it is both possible and important to engage in a social and political discourse, and I am confident that the art field and specifically the field of artistic research is an appropriate arena for this.


\section*{The self}
\label{sec:self-artist-prod}

The notion of the self is a philosophically complex issue, and I am not claiming this text is exhaustive in that sense. My primary interest is the role of the self from a very practical perspective: How do I experience myself and my intentions and how does this experience relate to those with whom I am interacting? How does my awareness of the self influence my work, creativity and ethical understanding of the current situation? Is freedom an important aspect of the self and, if so, how does this freedom relate to, and influence the other? 

Morwenna Griffiths, in her text \emph{Research and the self} argues that ``arts-based, practice-based research needs to address the issue of the self of the researcher'' \citep[][p. 167]{griffiths10}. The subjective nature of all artistic practices and artistic research gives the self a central position in all aspects of the different processes of creation, evaluation, construction and presentation. That this is even possible and that so many theories have been developed to explain its operations, and its locus, points to what Griffiths calls the ``mysterious nature of the self''. In line with her argument, I think it is safe to put forward that the self is not easily defined. It is not static  but subject to constant change and the mere attempt to define its nature appears to change it. The connection of self with time and place, with the other, with the body and the socio-political spheres and power relations makes it obvious that it is in a consistently uncompleted state, in a capacity of becoming.

That ones reflection upon the self in fact alters it is a founding property of most kinds of therapy, though there is obviously no way of knowing whether it is the understanding of the self or the self itself that is altered. In order to build a preliminary model of the self that may serve as a point of departure I will draw on Griffiths' model, which in itself is ``influenced by Arendt's concept of `the human condition'''. Not as building blocks or as parts that are all interrelated to one another, but rather as leaking and fluid concepts that taken together may form an exploratory field of reference:

\begin{quote}
  (1) Each self is \emph{unique}, and its response to circumstances \emph{is not determined}. Further, (2) the process is continuing: we are always in the state of \emph{becoming}, always unfinished. (3) We make ourselves in \emph{relation to others}. [\ldots] (4) The circumstances that influence--and are influenced by--a self include \emph{specificities of time and place}. [\ldots] (5) \emph{embodiment} is crucial. The world is understood through the body and also perceptions of our bodies constrain our relationships with others and ourselves. So, (6) a self constructs itself in all aspects of the self: who and what it can be and which relationships are ones of belonging. (ibid.)
\end{quote}

It is not, however, enough to look at the different ways in which the self is constructed, through interaction and otherwise. We must also include and consider the fact that the self is continuously constructing the other, and similarly, how the dynamics of the self is influenced by social and political powers. Finally, it is important to remember that these domains themselves may be influenced and even altered by how the relation between self and other is constituted. In their discussion on becoming-other Deleuze and Guattari identified what they saw as a possibility to move across the borders between self and other: ``Transcoding is one of the Deleuzian neologisms employed to underline an element of creativity, of invention and of crossing – traversing – borders between `self' and `other' '' \citep[][p. 140]{semetsky2011}. In this formative, or rather transformative, activity they discussed as transcoding, not only self and other could approach each other, also rhythm, for example: ``rhythm is the Unequal or the Incommensurable that is always undergoing transcoding'' \citep[p. 346]{deleuze80}. The most significant type of transcoding in the general sense according to Deleuze and Guattari is that when a code receives fragments or parts of a different code such as the fly, part of whose code is found in the spider's web. This becoming-other is carried out in a creative act of confirmation of the other: ``Whenever there is transcoding, we can be sure that there is not a simple addition, but the constitution of a new plane, as of a surplus value. A melodic or rhythmic plane, surplus value of passage or bridging.'' \citep[p. 346]{deleuze80}. That the idea of the transcendent Eurocentric subject that makes itself transparent, is problematic in the ways it ignores the long standing structures of power and oppression is brought up by postcolonial thinker Gayatri Chakravorty Spivak to whom we will return: ``Spivak's principal claim in \emph{Can the Subaltern Speak?} concerns the `transparency' of the subject. In her view, Deleuze `restores[s] the category of the sovereign subject' (p. 278), by which she means that in the manner of classical Western philosophy, he deploys an essentialised subject of oppression, which acts with respect to the object of a singular emancipatory project'' \citep[][(citations from Spivak, 1988)]{robinson2010}.

That the issue of the transparent subject may be problematic, is a topic raised by Spivak in an essay that we will return to later. For now, we should consider the significance of the references to music in the quote from Mille Plateaux above. I would like to argue that music in general and improvisation in particular are powerful sites for exploring ethics and questions concerning subjectivity, identity and the sphere of self and other.

Deleuze and Guattari's reference to music in the quote above is significant. Many of the principal aspects of the meeting between self and other can be anticipated and expanded in the domain of musical practice. Counterpoint, and the simultaneous interaction between several voices at once, would render any verbal discussion incomprehensible and impossible to follow whereas in music it may even do the opposite, create a greater clarity. To listen is a fundamental of all musical practices and listening to the other is an essential aspect of the ethical relation. Or, as Bruce Ellis Benson puts it, ``since music making is something we inevitably do with others (whether they are present or not), musical dialogue is \emph{fundamentally} ethical in nature'' \citep[p. 164]{benson03}. At this point the romantic view of the solitary genius whose output is so distinct it renders all other voices numb has to be questioned once again.

The world is not legible, it is audible, announces Jacques \citet[p. 3]{attali85}. Despite the extreme focus on visual communication in the hyper-capitalism of the early 21st century the chasm, as it is referred to by Marcel \citet{cobussen08} in his excursion in musical spirituality, between what is the seeing subject and the seen object has no equivalent in sound: ``The ocular subjectivity implies a not-involved witnessing, a necessary distance, an external relationship: the seeing subject can be located at the edge of the world. Conversely, the ear has no opposite''.\footnote{For an artistic research project with a focus on seeing and being seen, see the PhD dissertation \citet{leiderstam06}.} The ear is immersed in sound and has no way to `shut its eyes'. One may speculate if it is this possibility of detachment from what we see that is being exploited by media; because we can consume images without getting fully involved. When listening I am within the sound, captivated in it. Listening ``is musical when it is music that listens to itself. It returns to itself, it reminds itself of itself, and it feels itself as resonance itself: a relationship to self deprived, stripped of all egoism and all ipseity'' \citep[p. 67]{nancy2007}. What would inevitably restrain the potential for openness towards the other in listening is the desire for autonomy, although in the musical practice the dividing line between openness towards the other and the autonomy of the subject is considerably more complicated. We will return to this question later in the context of performance.


\section*{Freedom and spontaneity}
\label{sec:freed-artist-prod}

Although very difficult to define, freedom in general is a recurring concept in the discussion of musical improvisation. Or, as put by Hanna \citet{arendt77}: ``To raise the question, what is freedom? seems to be a hopeless enterprise.'' Freedom is no less complicated when discussed in relation to musical practices. Free jazz is a genre since Ornette Coleman's epic release \emph{Free Jazz: A Collective Improvisation} \citep{coleman60}. Ever since has there been a discussion as to what has actually become free in the process. Is it the musician that is liberated or the music? Or something else? The free jazz movement in the USA in the 1960's was intimately linked to the civil rights movement and to the protesting against the oppression of African Americans all over the country. And it is in the political domain that Arendt finds a possible discourse concerning freedom:

\begin{quote}
  The field where freedom has always been known, not as a problem, to be sure, but as a fact of   everyday life, is the political realm. And even today, whether we know it or not, the question   of politics and the fact that man is a being endowed with the gift of action must always be   present to our mind when we speak of the problem of freedom; for action and politics, among   all the capabilities and potentialities of human life, are the only things of which we could   not even conceive without at least assuming that freedom exists, and we can hardly touch a   single political issue without, implicitly or explicitly, touching upon an issue of man's   liberty.  Freedom, moreover, is not only one among the many problems and phenomena of the   political realm properly speaking, such as justice, or power, or equality; freedom, which only   seldom---in times of crisis or revolution---becomes the direct aim of political action, is   actually the reason that men live together in political organization at all. Without it,   political life as such would be meaningless. \citep{arendt77}
\end{quote}

Sara \citet{ramshaw2006} is similarly focused on the political domain and its impact and interrelation with musical freedom and improvisation. She employs a deconstruction on the concepts of invention and improvisation, ``both constituted in their singularity'' by which they can be ``starkly contrasted to the dominant conception of Western law''. After all, law is more often than not expected to privilege ``generality and universality over unpredictability and arbitrariness.'' On the surface improvisation may seem as a means to create music that is free from the chains of the formal structures that, e.g. notation imposes on the expressive possibilities of the musician. As we will see, singularity is and has been a powerful agent in much of jazz and improvised music rejecting the generality that other kinds of music has advocated, open it to the immediate and unmediated influence of the individual performer.\footnote{After all, composing for symphony orchestra will only make sense if the sounds it offers are relatively general. Hence, the symphony orchestra is a machine that proposes a finite set of sounds and as such it would cease to work if its musicians would begin to promote their own individual and singular sound similar to how Ben Webster and Johnny Hodges had done in the Duke Ellington Orchestra.} Even though we know that such descriptions are not correct, the prevailing notion of jazz is that it is a music that may be created on the spot and whose substance is defined not so much by external factors or structures as by the will of the improviser. Ramshaw turns to Derrida's discussion of the difficult interdependence of the singular and the general which eventually led him to challenging ``the pure presence of singularity in invention, along with the universality, which is said to propel occidental law.'' By applying this notion to the critical study of jazz improvisation she contest that improvisation is an expression of ``sheer spontaneity'': ``A deconstructive reading reveals that improvisation so defined can be neither total in jazz nor totally absent in law. Instead, the singular event exists solely as aporia in both fields.''

Many have rightly criticized the understanding of jazz and improvised music as devoid of planning and ignorant to history and memory, but the field is somewhat more complex and the concept of improvisation will not let itself be defined on only one axis. Furthermore, the opposition against freedom as in Free Jazz is as questioned by jazz musicians as it is by scholars and composers \citep{lewis-1}. The idea of the improvising musician as a ``mystic who is unable to describe his or her own creative process is a staple of conventional cultural wisdom about jazz'' (ibid.), for which established, privileged and canonical composers are as guilty for as are the improvising musicians themselves. It should however be noted that many of the attempts to target improvisation as unforeseen, unplanned, spontaneous and based primarily on ``feelings'' is to a significant degree a political strategy with which jazz musicians has been kept outside the central structures of cultural funding. An expression that according to a popular view to a large degree is made up in the moment cannot be taken seriously in a culture otherwise driven by the auteur. In Ramshaw's deconstruction of the idea of improvisation as sheer spontaneity, however, the ``unpacking of the aporetic nature of singularity reveals the [\ldots] necessity of jazz form''. In other words, as the binary opposition between free and non-free is questioned it is no longer a problem that a music such as jazz, which holds improvisation in high esteem, also contains compositional elements as an integral part of its creation. Nor it is a problem that the performance of improvised musics rest on a disciplined practice with a firm set of tools for musicians to use to accomplish their expressive goals. For these reasons, perhaps self evidently, it is important to acknowledge the different dynamics of the terminologies, or, as is so eloquently put by Ramshaw, a:

\begin{quote}
 tension consequently exists between the ``spontaneous'' conception of jazz improvisation and   the more context-driven model. This tension is intrinsic to jazz improvisation   itself. Improvisation can be neither purely spontaneous nor completely determined by the   musical structures with which it engages. It must be both responsive to otherness and have   some stable or determined dimension in order to endure as jazz improvisation. \citep{ramshaw2006}
\end{quote}

But, to further complicate things, although it is true that improvisation is not solely a spontaneous process that creates itself in the moment, does not rule out that precisely this aspect may be seen as its driving force. While spontaneity is not the only signifying aspect of musical improvisation it may still be part of its momentum, part of what makes it a unique communicative process. The spontaneous act works from the inside, but this fact is obviously not unique for jazz and improvised music but an important property of all artistic activities. 

The personal expression is of great importance in many art forms. In jazz, to develop a sound of your own is critical and many of the great jazz musicians such as Coleman Hawkins, Betty Carter, Charlie Parker, Billie Holiday, Lennie Tristano, Carla Bley, Albert Ayler have in some ways redefined and stretched the limits of their instruments through their highly skilled, individual and original output. Their \emph{sound}, i.e. not only the sound of their instruments, their articulation and phrasing but also their harmonic and melodic language and their general aesthetic, has become their particular musical identity. This is often referred to as `being personal' and to lack a well molded identity of this kind is often regarded a failure. In my own schooling as a jazz musician, the one thing I remember most clearly is being reminded of the importance of having a personal sound. Drawing on Heidegger's notion of \emph{Eignetlichkeit} \citet{benson03} sees the expectation of making `the piece my own' as an aspect not limited to jazz improvisation, but common also to classical Western music and, above all, very important: ``So `making the piece one's own' and `being oneself' are in some sense \emph{necessary} to a good performance'' (p. 166). 

The claim on jazz musicians to be both strongly individual and free improvisers at the same time quickly becomes problematic as the first requirement may influence or limit the second. To attempt to simultaneously do both, one may may end up using ones freedom to claim the right to control the situation at the expense of the freedom of the other. This is a surprisingly common mode in jazz improvisation where both freedom and power of expression and musical readability are highly valued. Gary \citet{peters09} brings it up in his book \emph{The Philosophy of Improvisation} and calls it the ``aporia of freedom''. Freedom is generally thought of as something positive, deliberating and emancipatory but it is a mistake, according to Peters, to neglect ``freedom's questionable duality'' (p. 21). And the reasoning by Peters finds some resonance in Emmanuel Levinas' writings, on of the authorities in the philosophy of ethics. The other as an object, even as a known object, and counterpart has been strongly questioned by Levinas who has instead argued for the irreducible otherness, the other as infinitely other.\footnote{Although the details of his ethics and his method is beyond the scope of this short text I find it important to introduce his thoughts in this context.} The creative freedom that we often take for granted and sometimes regard as a necessary state is questioned by Levinas because of the implications it has on the other: ``For my freedom comes at the expense of the other's freedom, my own autonumous world at the expense of the other's heteremony'' \citep[p. 165]{benson03}. As freedom is as a central as problematic issue in improvisation this topic is vary interesting in this context. Benson goes on to ask how the true dialogue can be maintained and the autonomous dialogue suppressed? First of all we should remember that once we leave the domain of philosophy and enter the world of practice the dividing lines may not be so clear cut. I am not sure if those who have promoted the right to freedom in music would agree that they have subsequently repudiated the possibility for an ethical relation to the other in general, but it is clear that this topic has some relevance in some kinds of practices, as we shall see. What more is, even if I reduce my subjective autonomy, limit my freedom and approach the other as infinitely other, it is not a guarantee that I will remain ethical.


Although the search for an individual sound in most cases is a very conscious act there is a corresponding search for the pure, or unconscious, expression, exemplified by Ornette Coleman's attempts to short-circuit the habitual aspects of his saxophone playing. In order to be able to ``create as spontaneously as possible---'without memory,' as he has often been quoted as saying''\citet[p. 117]{litzweiler92}, without any 'real' training he started playing the violin and the trumpet. These instruments gave him the freedom to play and improvise in a manner that his ``memory'' made it difficult for him to do on saxophone. When playing the saxophone he would be partly ruled by his meta-knowledge---his knowledge \emph{about} playing the saxophone. Expectations encoded mentally as well as bodily would also influence him and, to Ornette Coleman, this was a hindrance to his spontaneity. The trumpet, and even more so, the violin---the violin was not really an instrument used in contemporary jazz so there were no model to follow---he had learned himself. On the violin he adopted a highly original technique that allowed him to bypass ``not only the jazz tradition, but Western musical traditions altogether. He had no teachers or guides to show him how to play trumpet and violin and purposely avoided learning standard techniques''.\citet[p. 117]{litzweiler92} Freedom of memory and freedom of influence from extra-musical parameters. The 'unknown' instruments gave Coleman a sense of \emph{internal} freedom, liberated from the physical memory associated with his saxophone playing. He approached a self expression where the transformation from intention to result was not ruled by a preconceived notion of what it should sound like. Marcel Duchamp is known for a similar act to rid himself of acquired knowledge: ``I unlearned to draw. The  point was to forget \emph{with my hand}'' \citep[Duchamp, as quoted in][s.29]{tomkins65}. Like Coleman, Duchamp was using a (new) tool (the ruler) to revolt against the tradition and the expression ``forget with my hand'' is significant here as it puts the focus on the physicality of the action. Both Coleman and Duchamp identify the memory of the body, the embodied self, as perhaps the most important dimension in the struggle for the free. And the question comes to mind, revisiting Peters' notion of the aporia of freedom: Duchamp acted alone but what was the impact of Coleman's abandoning of tradition on the musicians he played with?


Musical notation and the division of labor into composer and performer is a relatively recent invention in the history of music and improvisation as an expression of musical freedom is often seen as the exception. Perhaps it is not so much improvisation that is free but music based on preconceived and composed structures that is constrained and lack freedom? In that case, would it not be more appropriate to talk about reinstating freedom in all aspects of musical creation and abandon what are seen by many as a problematic dichotomy between improvised and composed music?\footnote{Which, I should add, I strongly believe is an erroneous model. There is no opposition between composition and improvisation, the are two very different processes and one can effortlessly exist within the other at any time. I suspect the reason there is a persistent desire to keep them in opposition has to do with social and political issues.} In fact, improvisation as such is no guarantee for achieving expressive freedom. In some improvising genres and musical cultures the freedom of improvisation may be defined by completely different standards and sometimes improvisers are so strictly tied to a particular aesthetics or style that on the surface freedom may not appear to be a strong agent.\footnote{Some of these expressions could be referred to as idiomatic improvisation, as labeled by Derek \citet{bailey92}.} But even in improvised music that is strongly identified with freedom, its stylistic qualities may be so prominent that the meaning and impact of freedom may be debated. Looking at it from the other side, however, even in music with a strong idiomatic identity, such as bebop, in which performers are musically and socially tied to a defined, and in a sense, limited set of phrases, the organization of the material is still freely decided by the musician. And if we approach the idiom from a slightly wider angle, and at a greater time scale, we can clearly see that there is a huge difference between the stylistic interpretation made by Charlie Parker and that made by Thelonius Monk. Both are exponents for bebop but have approached the idiom freely, with exceptional individuality, and with a greatly varied aesthetics as a consequence.

\section*{Subjectivity}
\label{sec:habit-self}

The beginning of contemporary jazz that exploded in the 1950s with the Be Bop movement preceded the related reaction against serialism most prolifically promoted by John Cage has not been acknowledged for its influence. Chance operations and indeterminacy, so effectively denounced by Pierre \citet{boulez64} in his article \emph{Alea} claims the composer ``has chosen henceforth to be meticulous in imprecision'' (p. 44) and rather condescendingly talks about what they perceive as the failure of indeterminacy in composition. There is no doubt that this debate was heated nor that Boulez was absolutely convinced that the best means to compose music was to be meticulous in \emph{precision}. The composer is making the choices and is the speaking subject, to think or attempt to do otherwise is plain wrong:

\begin{quote} 
In spite of best intentions and most earnest attempts, I am unable to make out the precise reason for this fear to approach the true problem of composition. Perhaps this phenomenon also is due to a kind of fetishism of numeral selection--a position that is not only ambiguous but completely unsound when the work under investigation structurally refuses these procedures, which are, after all, excessively coarse and elementary.
\citep[p. 44]{boulez64}
\end{quote}

Cage, the main proponent at the time for indeterminacy, or meticulous imprecision as Boulez would call it, had a corresponding lack of understanding and interest for jazz, in many regards seemingly similar to some of Cage's ideas. George \citet{lewis-1} contextualizes this relation, or lack of relation, in his widely influential paper Improvised Music after 1950: Afrological and Eurological Perspectives brings up Cage's discussion on jazz with the journalist Michael Zwerin\footnote{In the interview Cage is invited to share his thoughts on jazz and agrees to do it while at the same time stating that jazz is not something he thinks much about at all. \citep[In]{lewis-1}} within a sociological context:

\begin{quote} 
The colloquy between Cage and Zwerin [\ldots] displays whiteness in its defining   role. Zwerin, though supposedly taking the side of jazz, ends up agreeing with Cage that jazz could   use some work. The work of black artists is defined by whiteness as the primitive (yet improving)   work of children: `But jazz is still young, and still evolving'; jazz could benefit from serious   study of `our' models; already, it has started to explore areas `suggested by Ives'; `jazz is   getting freer' though the use of tone-rows, and `getting away from the time dependence--inferring   it rather than clobbering you with it all the time'; and so on. \citep[p. 104]{lewis-1} 
\end{quote}

These two events, Boulez article and Zwerin's interview took place only a few years apart in the mid 60's and clearly has something to say about the great changes that music was going through at the time. The great European composer is hitting on the great American composer in turn hitting on jazz in general and black musicians in particular. The New York School of composers threatened the authority and the dominant power of the composer. Cage, while proposing a very open and decentralized attitude towards music and art was until his death in fact a very strong and influential proponent of his own view of the world of music. And, as much as he liked to remove intention from his performances, and regardless of his enigmatic writings on the subject, he was the one setting the boundaries for his compositions (most of them) and he published them as scores just as Boulez would publish his. 

However, it is not so much these debates themselves that are my interest here, but the position of subjectivity portrayed. To Boulez it is unthinkable that anyone else than the composer can be the subject; not the performer nor the listener, it is the composer who is speaking. The aesthetic turmoil at the time is likely to be one reason for these debates, the social and racist order another one, but I find it difficult to not see Boulez' attack on indeterminacy and Cage's patronizing attitude towards jazz as a defensiveness towards a means to organize musical material that could threaten their own respective power positions. A circle as an image of center and periphery turns into a spiral with multiple centers and multiple margins, but the structure is retained. The role of the composer, in itself a relatively new invention introduced at the time when notation divided the musician in two parts, the originator and the executor \citep{wis96,frisk-ost06-2}, is in itself under attack when the improviser rejoins these into one and the same agent. The binary division between composer and musician is deconstructed by the improviser.

When writing the history of AACM, which was formed also in the mid 1960's, George Lewis gives us some insight in the discussion concerning composers and musicians. A suggestion to have a requirement that the music of AACM should be original was turned down which opened up for a different view on composition as a collaborative process. Rather than adjusting to the common view, the perspective in the center, AACM allowed themselves to formulate their own take on their activities. According to Lewis, they avoided to:

\begin{quote}
  reproduce the division of labor between `composer' and `performer' that characterized Western classical   music. Rather, to these musicians, being `m musician' meant working out of a hybridized model of creative   practice that negotiated between individuality and collective membership, and which assumed primary   creative agency for each artist. \citep[p. 103]{lewis2008}
\end{quote}

\section*{The Six Tones}
\label{sec:tu-dai-oan}

In early 2006 Stefan Östersjö and I initiated a project together with Ngyen Thanh Thuy and Ngo Tra My, two Vietnamese musicians playing Dan Tranh and Dan Bau respectively. Both of them were temporarily visiting Sweden at the time, as guest teachers at Malmö Academy of Music, and the first time we met was in the composition studio at the school. Dan Tranh is a traditional Vietnamese zither played with a finger plectrums attached to your right hand fingers while the left hand is used to create glissandi and vibrato. Dan Bau is a monochord played with a bamboo plectrum with one hand and controlling the pitch with the other by stretching or relaxing the string. Different overtones can be produced by plucking the string at various places. The sound of the string is picked up with a transducer microphone amplified through a small speaker.

Later the group was named The Six Tones and it constitutes an encounter between traditional Vietnamese music and experimental Western European music. Since the start the main objective has been to find forms for interaction between different cultures on more or less equal terms, and it has become a site for experimentation and examination of the meaning and impact of that ambition. Experimentation is a central concept and as John \citet{corbett2000} argues, ``by definition, experimental data must be able to behave in a way not predicted by the hypothesis. Thus, the experiment is conceived as an excellent setting for exploration and discovery, a perfect opportunity for an encounter with the new, the unforeseen, and the unfamiliar'' (p. 165). In order to truly be able to encounter the new and the unforeseen, challenging different aspects of the notion of ``center'' and ``periphery'' was necessary: is Western art music the norm and traditional Vietnamese music an exotic other? Are Stefan and I `visiting' a music outside of our own sphere, or is it rather Thuy and My that are forced to approach us. Is it at all possible to communicate on equal terms? The social impact of the Eurocentric view of the world, however, should not be underestimated. Stefan and I belong to what Mark \citet{slobin1987} labels ``the superculture'' (p. 31), and the complex political and economic asymmetry between east and west plays an important role in our understanding of the other in our multi layered work with traditional Vietnamese music in general, and with The Six Tones in particular.

From the outset the idea with the project was to aim for a music whose identity is neither Vietnamese, nor Swedish or European, but both at the same time, or, preferably, a music with its own distinct character. We wanted to avoid the simple superimposition of one tradition on top of the other, instead trying to consistently have the two elements coexist on equal grounds. Returning to \citet{slobin1987} we can use his three broad categories of intercultures:
\begin{enumerate}
\item \emph{Industrial interculture} which evokes the notion of a commodified system whose main function is to project the first world order, spiced with an unobtrusive element of difference (p. 61)
\item \emph{Diasporic interculture} emerges from the subcultural interactions across the borders of nations (p. 64)
\item \emph{Affinity interculture} describes a ``global, political, highly musical network'' in which musicians are interacting communicating through a negotiated musical space (p. 68)
\end{enumerate}
At the time we were not fully aware of the implications of our ambitions at the time, nor had we thought much about the political dimension of our endeavor, but our goal was most closely related to the affinity interculture category above. To accomplish such a network would by no means be simple, and we still have much work to do in the future to approach it. In the process, however, we have had to reevaluate our own musical identities, and for myself I had to question my roles as composer and improviser, reconsider what my level of influence should be, and what it could become. In order to create the necessary preconditions for the two different musical traditions to intermingle rather than coexist without any sense of deeper interaction, the individual influence on the musical structures had to be carefully molded to fit the context. Our first meetings were very tentative and I found it extremely difficult to balance my own initiatives with leaving enough space for the input of Thuy and My. One reason was the lack of a shared language and another was the social asymmetry between the two subgroups. On the one hand Stefan and myself as two white men who, at home in the common environment of the music academy invites two Vietnamese women, visitors and foreigners in an alien environment, both musically and culturally. There is no easy solution to that situation.

Reflecting upon it brings forth questions concerning identity, culture, power, (post)colonialism and many other issues. My argument in this chapter is that all of these matters have a critical impact on the individual and collective artistic processes. To limit the thinking to intramusical parameters will not reveal the true nature of the complexity of an artistic interaction such as the one described here. Musical autonomy has been questioned for many years but my suggestion here is that if experimentation in intercultural interactions is to take place it is necessary to consider the field of social interaction.
%Är det här en utveckling av syftet ska in? Eller i ett slutord? Kanske detta ändå ska utvecklas?
Furthermore, my musical thinking is likely to be influenced by these questions anyhow. How can I engage in a true musical encounter and experimentation with the Six Tones that avoids simply layering material?
How can I avoid that my own playing obscures the identity of the Vietnamese music as Vietnamese and still contribute to the sound of the group? How can I avoid that my own musical input takes over, masks the other elements, while at the same time remain close to my personal ambitions and aesthetic preferences with the music? 

In a culturally and socially homogeneous context many of these questions are not even raised. They are unnecessary as much of the negotiation has been going on in a larger context, outside of the rehearsal space. Common signifiers and references, aesthetic negotiations, inherited and nurtured since the early development of the musicians and composers. The collaboration does not have to be between two different musical and artistic cultures, simply bringing together musicians from different genres may create obstacles difficult, or plain impossible, to maneuver. With a reference to Umberto Eco, Stefan Östersjö and I have previously discussed the formation of a subculture within which these obstacles can be dealt with in terms of a semiological system:

\begin{quote}
  ``Any attempt to establish the referrent of a sign will force us to define this referrent with   the terminology of an abstract entity.'' This is what Eco calls the ``cultural convention''.   \cite[p. 61-6]{eco71} Defining a cultural context as the referrent resolves some issues in   the analysis of performed music as the listener or concert-goer can be defined as belonging to   a cultural entity - a cultural entity that may be used as the code to decipher the message,   which in this case is the acoustical trace left by the performed music. However, in our study   we are looking at a not yet existing work - a work in progress. Following the same model we   might try to understand the symbolic system in relation to a subculture created by the agents   involved.  Both composer and performer are working within the frame of their own cultural   context which defines their respective understandings of the evolving work. The subculture is   the result of the interaction between these two contexts. The musical work becomes the sign or   the message, the agents the signifiers and the subculture the signified or the code. Apart   from the fact that this model inevitably becomes a very rough generalization     of reality, it is also highly dependent on the definition of the cultural context, which in   the case of composer-performer interaction prior to the existence of a work becomes a very   complex task.  \citep{frisk-ost06-2}
\end{quote}

However, at the time of our first meetings with the Six Tones we had not yet understood the need of creating a subculture to attack the creative boundaries set up by the lack of a common platform. And eventually we arrived at a slightly different path to resolve the difficulties in a collaboration on  equal terms than the one we describe in the quote above. At this meeting, in the composition studio of the Malmö Academy of Music, I became incredibly self aware of the asymmetry between Stefan and I, the two western men, and our Vietnamese female colleagues. Given the history of Vietnam in particular, and the history of the white man in general, I was afraid that simply because of my identity and cultural background, my activities and my ideas would mask, or get in the way of, Thuy's and My's freedom to participate on their own terms. This would effectively also mask parts of the culture they carried with them. But instead of letting them \emph{speak for themselves} I subjugated them to my own understanding of the world, the context, the music and our interaction. Because they could not speak within our framework (they were the visitors, we were the norm), they remained voiceless. As we shall see this is an archetypal way in which the Western subject has dealt with the other in the first world. In the following I will attempt to describe the process that followed the first initial and largely unsuccessful encounter, and describe the evolution of our project, in particular our interpretation of the song T\'{u} \DH \d{a}i O\'{a}n which we have played since some of our first concerts in 2006. 

% These are the general questions that I will aproach in this chapter. The more specific inquiries % concern the role and identity of electronic instruments in these contexts, both in terms of how % I interact with them and how the output of my efforts interact with the sound of the acoustical % intruments.

\subsection*{T\'{u} \DH \d{a}i O\'{a}n}
\label{sec:tu-ddai-oan}

T\'{u} \DH \d{a}i O\'{a}n is a popular traditional Vietnamese tune in the O\'{a}n mode. Although the idea of playing Vietnamese traditional music in the Six Tones occurred soon after our first meetings it was not until we did a tour in Sweden in 2007 that we picked up T\'{u} D\d{a}i O\'{a}n and started working on the hybrid version that we have been playing since then. In Vietnam the song is often heard played on a \DH \`{a}n tranh and it is idiomatically very natural on this instrument. Stefan did a transcription of it for 10-stringed guitar that he played with a slide in order to have a greater control over vibrati and glissandi. These two expressive devices are particularly important to the Vietnamese tradition and the musical mode depicts how, and where, to perform e.g. vibrato.

We had decided to do a trio version of T\'{u} \DH \d{a}i O\'{a}n for \DH \`{a}n tranh, 10-stringed guitar and live electronics. The challenge still was to create a coherent version of the tune though Thanh Thuy was a master musician in the tradition, Stefan had practiced playing Vietnamese music on a non-idiomatic instrument for about six months and I had explored it for about the same time period but without an instrument that made any kind of sense as far as the tradition of  T\'{u} \DH \d{a}i O\'{a}n goes. The 10-stringed guitar at least has some properties shared with the \DH \`{a}n tranh, it is a string instrument with a wooden resonance box. The electronics is in every sense remote from the performance tradition of Vietnamese traditional music. Our ambition was to create a shared space in which we can explore the tradition and our alteration of the tradition coming from western aesthetics, but from a theoretical point of view it was not obvious how that could have been accomplish given the circumstances.

In his presentation \emph{Appropriation, exchange, understanding.} British electronic music authority Simon \citep{emmerson06} points out how musicians have always exchanged concepts and ideas through the act of performance itself, often without language. As explained above, in the early meetings of the Six Tones we hardly had the choice of using language. Performance was the only relatively useful means of communication. But Emmerson brings up how any mode of exchange involves some kind of distortion, reduction, impoverishment or loss and continues: ``While some of these losses will be an inevitable result of global social change, the ethical question of knowledge and awareness cannot be avoided'' (ibid.). What was the value that we could jeopardize in this collaboration and to what extent were we aware that there was a risk involved? 

In 2006 Stefan and I went to Vietnam for a working session and a concert at the Hanoi conservatory. This visit was a pivotal moment in the development of the group. If our first meeting in Malmö was terribly hesitant and governed by, mainly my, attempt to counteract the inequity as it was perceived by my prejudiced notion of what our relations were the visit to Vietnam turned everything upside down. To encounter Thuy and My in their own surroundings made a big difference, reinforced by my own experience of what it means to not understand the codes and the culture temporarily reverted the position of our roles. This was emphasized by what I saw as a parallel reversal of the gender systems in Vietnam. Though I later realized I was wrong in assuming that Vietnam was a matriarchy it was striking how many of the the positions that in Europe are traditionally held by men, in Vietnam was held by women. The dean, as well as a majority of the other significant positions at the conservatory, were held by women and many of the tasks at the other end of the job hierarchy such as cleaners and secretaries where held by men.

This was clearly an important insight for me that affected my relation to Thuy and My. Whether there had ever been any good reasons to treat them cautiously as fragile and sensible, seeing them in their home country made it obvious there was no longer any need to do so. Women in Vietnam have developed a special fortuity ever since the dissolution from the ``Confucian predominance of men over women''\citep{VanKy2002} and the Vietnamese folklore has numerous indications to the domination of female over male. In more modern times, ``in Vietnam during the period 1929–36, there were three feminist newspapers or at least newspapers that supported women’s rights'' and it is clear that ``Vietnamese women were claiming their equal rights in a male society'' (\emph{ibid}). In other words, even though the situation in Hanoi today to a large degree is influenced by the responsibilities that Vietnamese women had to carry in the war on USA there is a long tradition for female influence in the country. These insights, however, should be put into context of the organize criminal networks dealing with trafficking and kidnapping to which Vietnamese women fall prey to as much from as do other exposed groups in the region.

It may seem self evident that a closer contact with a foreign culture and social system, the music of which one is interacting and working with, results in a more natural and less strained communication. And opposite, that the lack of this first hand information of, and understanding for, Vietnamese music and culture was a reason for the unfortunate first start of this project. As a consequence, despite a merely rudimentary insight into the Vietnamese society, the field for interaction in the group had a radically altered premise, and the main difference was in the way I could approach Thuy and My. I was to some extent able to break free from my preconception of them as foreigners, victimized by definition. To put it in the words of \citep{spivak1988}, I had learned to let the subaltern speak, and by now I was listening. 

It was not until the year after our first visit in Hanoi that we first started working with T\'{u} \DH \d{a}i O\'{a}n. But in the years to follow we would play it numerous times and continued to work with the form and the expression. Although referred to as improvisation T\'{u} \DH \d{a}i O\'{a}n is part of a tradition of playing that is in fact very fixed and which only allows for a limited set of possible permutations. Even if our intention was not primarily to propagate a traditional style of playing we would endeavor to maintain enough significant traits for the music to be recognizable as coming out of the Vietnamese heritage, we had to move forward cautiously and in constant dialogue with Thuy, the only one with a solid experience from playing that music. In the rehearsals that followed\footnote{The rehearsals discussed here were carried out at the Elctronic Music Studios in Stockholm in the late winter of 2009 and we have near to complete video recordings of them. The actual events in the rehearsals are discussed in greater detail in \citet{Ostersjo2013}} we let Thuy retain the initiative while Stefan and I would stay in the background, occasionally fronting or commenting ideas. At this point we already had common experiences and a much greater understanding for our respective musical, social and cultural preunderstandings but we were still in absence of a common language, which is, returning to \citet{emmerson06}, not an uncommon situation in the tradition of mediating performance parameters in rehearsals. For us, not being able to efficiently discuss and negotiate the performances, made it necessary to use a trial and error method in very short iterations: Play--evaluate--alter--play--evaluate\ldots in small incremental steps. One of the early choices we did concerning the form was to keep the tune and expand the traditionally rather free introduction. In the tune we inserted an improvised section and at the end we added an extended improvisation. The Dan Tranh and the guitar played the improvisations and the tune and I played primarily the improvisatory sections.

In preparation for these rehearsals, and the upcoming Scandinavian tour, I had built a virtual instrument using a physical model of a plucked string. I had played around with it attempting to make it sound like a combination of Dan Tranh and Dan Bau. Once I arrived at a relatively satisfying compromise I altered it further and added affects to it. I entered all the pitches of T\'{u} \DH \d{a}i O\'{a}n in an array and let the instrument step through the array for each trigger allowing me to focus on the timing of the onset and the intonation and the glissandi. Though there was nothing wrong with this instrument the desired effect, to bridge the conceptual and cultural gap between the acoustic Vietnamese Dan Tranh and the electronics, was lacking. In fact, the more I obscured this instrument from its original sound, the more effects I added to its path, the better it appeared to be working. There seemed to be very little I could do that would obscure the the tune, which was also thanks to the form we had chosen.

Emmerson, furthermore, goes on discussing the different modes of exchange that we may have access to when different musical cultures collide, and the ``particular mix of these may result in a range of outcomes: on the one extreme, appropriation with no exchange or understanding – for example, a composer ‘plundering local colour for sampling’ - through to true exchange with the possibility of real mutual understanding'' \citep{emmerson06}. Even if we were not interested in a merge of the musical traditions we were certainly clear about wanting to avoid appropriation. So was I appropriating Vietnamese music when I build the instrument as to blend in with the acoustic performance? Or was Stefan when he played the 10-stringed guitar with a slide in order to make it sound like an idiomatic Vietnamese instrument? According to Emmerson the unsuccessful exchange between two or more idioms is one where properties of one hide properties of another which he explores using the term masking:

\begin{quote}
  [A] situation where two sounds are played together and one masks the other (or a perceptual aspect of the other) such that it can no longer be perceived. We can generalize this from sound, to performance and even to aesthetic aspects of music. Throw two traditions of music making together and aspects of one may mask aspects of the other (sound subtlety, performance practice tradition and aesthetic intent). This may be inevitable in any intercultural work as there are bound to be incompatibilities. But we must ask - have we masked something ‘significant’ as seen from within the culture? \citep{emmerson06}
\end{quote}

Masking will probably occur to a certain extent in any kind of music, but the question asked by Emmerson at the end of the quote is material: It is not so much \emph{if} something is lost as \emph{what}, and what the importance of this property is. Perhaps my physical modeling instrument was more problematic as a virtual copy of the Dan Tranh than it was as a more radical deviation from the real instrument? Perhaps it masked the Vietnamese instrument due to its similarity with it, but reinforced it as it grew different? 
%Though we kept altering the energy and the direction of the tune, making new connections between the different sections of it, we kept the form more or less unchanged. 
The way I developed my part was in effect a movement away from my initial respect for the Vietnamese tradition approaching a more genuinely experimental mode of playing. However, at the same time, this development had nothing to do with a lack of respect, quite the contrary. I feel inclined to argue that the attitude I had in the beginning of the project, before our first trip to Hanoi, was lacking in respect. My assumptions about Vietnamese music, though constructed in good faith, were not informed by the tradition itself but rather by my prejudices about it.
When Emmerson warns us about masking and writes that if the exchange continues, ``in time the masked element may disappear as it no longer functions within the music'' we must not take it too literal. In our case, to think that we could erase or destroy parts of the living tradition of Vietnamese music would be to overestimate the influence and power of our group. Regardless of the validity of this otherwise legitimate threat, our experience  in The Six Tones was that we could go quite far mixing the two modes of expression without any masking of significant traits in the original music. The more important attunement performed was that in the social dimension. As this grew stronger, our musical artifacts did also.

Finally, it is interesting to note that while Stefan saw it necessary to go much deeper into the theory and practice of playing Vietnamese traditional music, I was more concerned with \emph{not} making authenticity a parameter in my playing. One obvious reason is that Stefan's instrument, the guitar, has a certain affinity with some of the instruments we worked with, whereas electronics finds no evident response in the Vietnamese musical tradition. Fact is, over time I became more and more audacious\footnote{By audacious I mean that, although I was less concerned with right or wrong, and less focused on the history and idiomatics of the tradition as it would take shape in my own playing, I obviously had, and still have, a deep respect for the Vietnamese musical tradition as it is carried on by master musicians such as Thuy and My.} in my experiments with the music. The consequence of my attitude was that in concerts I sometimes took chances with the material that led to very unexpected results. Opposite to what one may think, the effect was that it eventually resulted in an alteration of the structure of the form. The effect of my `error' assumed the unintended function of Coleman's violin or Duchamp's forgetting hand. It was the ``abrupt disappointment of expectations of meaning'' like a surrealist jolt that made us reconsider what we heard and how we experienced it \citep{barthes68:death_of}. Maybe we can arrive at an experimental redefinition and subcategory of Emmerson's concept of masking: Masking occurs when the understanding of the conditions for interaction is weak or missing altogether. Even though the skills of the performers may be outstanding, if the members are unable to contextualize events and enterprise of the other members, in relation to the (sub)culture they are a part of, the result will lack in contour and some significant aspect of the expression may get lost. 


\section*{Center and periphery}
\label{sec:center-periphery}

There is an obvious tendency to always look at western art music as the center and whatever is external to this as the periphery. To allow the differing musical traditions to coexist on equal terms, this tendency need to be cautiously dealt with. The Eurocentric view is rooted in the concept of the West as the social, economical and political focal point in the world. Failure to understand the wider consequences or a malfunctioning relation between center and periphery, in which transcoding, to use the vocabulary of Deleuze and Guattari is unattainable, will inescapably lead to asymmetrical relations and inequality. If we do not manage to put this, essentially colonial, view behind us we will remain trapped in our self-centeredness. Consequently we will always, to take only one example, look at the Vietnamese musician as one whose music is grounded in the periphery. As such it can serve as a peculiar and colorful complement to our own traditions but never engage in an encounter on equal terms. Attributing values to it such as `beautiful' or `masterful' does not change its locus and does not move its status closer to the center. Quite the opposite. Aestheticising the other, or the expressions of the other, is an effective way to keep it looked out.

Many writers and scholars have brought these issues to the fore front. George Lewis in his works on jazz musicians and turns to Margeret R. Somers' ideas of the epistemological other, post-colonial theorist and philosopher Gayatri Chakravorty Spivak rhetorically asks \emph{Can the subaltern speak}?, Edward Said in \emph{Permission to narrate} about the enormous inequality in the war in Palestine, and Gloria Jean Watkins, also known as Bell Hooks, approaches her own background in racist America in the short but significant \emph{Marginality as site of resistance}\cite{HooksBell1990}. The first mistake, as is shown by Hooks, is to think of marginality as a space one wishes to surrender and give up to instead gravitate towards the center. Describing the other side of the railway tracks (the dividing line between the two communities) as the site of the center, she is also identifying marginality as ``the site of radical possibility, a space of resistance'' (ibid., p. 341). Hooks and her friends and relatives would at times trespass into the other domain, to work ``as maids, as janitors, as prostitutes'' and when they did ``there were laws to ensure our return''. But to refuse to give in to the expectation of wanting to relocate from the margin to the center is to invalidate these laws and dismantle their meaning. In a globalized world we may think that the regions constituting the inside and the outside are large continents and political systems such as east versus west, or democracy versus dictatorship, but the local demarcations still exist. Vietnam, being both politically and economically in the periphery, is in every respect marginalized as the other, the foreign, the different and the obscure. And, just as there were laws for Bell Hooks and her friends when they trespassed, there are laws to ensure the return of Vietnamese visitors in Sweden. We may think that the global perspective has broadened our view on the world and blurred the boundaries but in the eyes of the legislators in the west there is no doubt as to what is the center and what is the periphery. 

Deleuze's and Guattari's supposition that in transcoding, becoming-other is a way to resolve the opposition between self and other, east and west, center and periphery, is quite forcefully rejected by  \citet{spivak1988}. In the already classic essay \emph{Can the subaltern speak?} she poses important questions concerning the continuing marginalization of those without or with only limited access of the cultural imperialism, taking a very broad view on the world. No one is safe, not even the icons of central European thinking, as we have already mentioned. The Eurocentric subjectivity, according to Spivak epitomized by Deleuze and Foucault, threatens to further obscure the subaltern:

\begin{quote}
   It is not only that everything they read, critical or uncritical, is caught within the debate   of the production of that Other, supporting or critiquing the constitution of the Subject as   Europe.  It is also that, in the constitution of that Other of Europe, great care was taken to   obliterate the textual ingredients with which such a subject could cathect, could occupy (invest?) its   itinerary. \citep[p. 75]{spivak1988}
\end{quote}

There has been attempts to seek redress for Deleuze, Guattari and Foucault and prove that their thinking was not rooted in Eurocentricity and that it will not necessarily lead to oppression of the other \citep[e.g.][]{robinson2010}. However, the main thread of Spivak's argument, according to me, is not concerned with what is done or not done to the ontological other, but with the particulars of the perspective of the other:

\begin{quote}
Can the subaltern speak? What must the elite do to watch out for the continuing construction   of the subaltern? The question of `woman' seems most problematic in this context. Clearly, if   you are poor, black and female you get it in three ways. If, however, this formulation is   moved from the first-world context into the postcolonial (which is not identical to the   third-world) context, the description `black' or `of color' loses persuasive   significance. The necessary stratification of colonial subject-constitution in the first phase   of capitalist imperialism makes `color' useless as an emancipatory signifier. Confronted by   the ferocious standardizing benevolence of most US and Western European human-scientific   radicalism (recognition by assimilation), the progressive though heterogeneous withdrawal of   consumerism in the comprador periphery, and the exclusion of the margins of even the center   periphery articulation (the `true and differential subaltern'), the analogue of   class-consciousness rather than race-consciousness in this area seems historically,   disciplinary and practically forbidden by Right and Left. \citep[p. 90]{spivak1988}
\end{quote}

It should by now be clear that Thuy and My were subaltern at the time of our first rehearsals described above. Both in terms of their social and political situation in the context within which they operated and in terms of how I approached them. Based on conjecture instead of knowledge, I started making choices and taking actions to protect them from the institutionalized impact of western influence. In the process I accomplished exactly that which I was trying to avoid: marginalizing the other. I was so painfully aware of the structural imbalance between us that I reinforced, rather than rectified, our social relation. Instead of responding to the (musical) situation and seeking to approach them as `the infinitely other', I enforced my transparent subjectivity. Slavoj Zizek draws a line from the Lacanian notion that justice as equality is founded on ``our envy of the Other who has what we do not have, and who enjoys it''. The response to the unevenness in the innocent context of this first rehearsal with the Six Tones was based on the thesis that, though I could not compensate for what I perceived of as a lack of possibilities I instead reduced my own latitude consistent with what Zizek refers to as equally shared prohibition:

\begin{quote}
  \ldots coffee without caffeine, cream without fat, beer without alcohol, [\ldots] warefare without casualties, [\ldots] politics without politics, up to today's tolerant liberal multiculturalism as an experience of Other deprived of its Otherness (the idealized Other who dances fascinating dances and has an ecologically sound holistic approach to reality, while features like wife-beating remain out of sight).
\end{quote}

What is the significance of these complex issues concerning economy, hyper-capitalism, world domination and post-colonialism in the context of contemporary music? How can we apply this deconstruction of the concepts of center and periphery in the artistic practice of a group consisting of two Vietnamese and two Swedish musicians? Why is it be necessary to consider inherited power structures when approaching the seemingly simple task of creating a workable platform for musical and cultural interaction? The task here is not to use a theoretic framework in order to apply it to an artistic work flow. Rather, I am concerned with understanding what was going on in the early stages of our work, and to understand my own initial and consequent reactions. By filtering these experiences through the thoughts presented here, and understanding my own reactions as expressions of a system of domination I can approach my own artistic practice as a vehicle for social and political thinking-through-music. The method employed is to recursively cycle through stages of artistic practice, reflection, evaluation, and theory as practice. As a consequence it will influence both my practice and my understanding for it and for its extra musical potential. 

What Bell Hooks is referring to in the text cited above is an institutionalized oppression and marginalization that has been going on for centuries and clearly operates on a completely different scale compared to the Six Tones. Merely reading about it will not let me understand the experiences described. What it does allow me to do, however, is to understand that the effects and the processes in the development going on locally in The Six Tones was similar to those operating on a larger scale. In a two way process the practice contributed to making me aware of the issues, allowing us to continue to highlight the instinctive tendency to treat the other based on the assumption that we hold the best solution. It is in this sense that almost all artistic activity also has a potential to engage in a political consideration, reflection and introspection. Not in the meaning that the artistic expression itself needs to be politically imbued, but rather that the site for artistic practice and artistic research is taken advantage of as a site also for politically oriented questions. 

This is at the core of the issue: How can I freely explore my individual, subjective and original artistic expression while still remaining open to the other? How can I avoid that my freedom limit the freedom of the other? Though these are no easy questions I argue that artistic practice is a useful arena for exploring these and similar related questions. The topic that I have brought up in this paper, to foster the social, ethical and political dimension of musical interaction through improvisation, by exploring the self and the consequences of freedom and habit formation, may be successfully investigated through the practice itself, and through the method of thinking-through-practice. At a time when art in general, and music in particular, is commodified to a degree that not even Adorno can have anticipated, artistic research is one of the few remaining fields that can withstand entrepreneurial tendencies in the music academies and within the field of music itself, and engage in the important artistic and social questions that lay ahead of us. 

% And this is our problem. We do not know how you learn without domination, without power. We don not know % what it means to learn without succombing to the stuctures of those who teach. Our schools have alays been % structured in terms of domination, in terms of the subject giving in to the greater means. We, at the % center, are now the the ones who need to learn what it means to be in the margin and those in the margin are % naturally accepting a position in the structural center, without giving up there privileged position outside % the center. 


\nocite{biggs10}
\bibliography{bibliography} \bibliographystyle{plainnat}
\end{document}