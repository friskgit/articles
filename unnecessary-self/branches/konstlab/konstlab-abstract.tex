\documentclass[a4paper]{article}
\usepackage[swedish, english]{babel}
\usepackage[T1]{fontenc}
\usepackage[authoryear,round]{natbib}

\usepackage{graphicx}

\renewcommand{\rmdefault}{pad}

\usepackage{fancyhdr}
\pagestyle{fancy}

\lhead{\small{\textit{Henrik Frisk}}}
\chead{}
\rhead{\small{\textit{Det (o)n�dv�ndiga jaget}}}

\title{Det (o)n�dv�ndiga jaget: Improvisation, motst�nd och interaktion\\\vspace{.6cm}
\large{Konstlab, G�teborg, 31/10, 2011}}
\author{Henrik Frisk, PhD\\{\small Musikh�gskolan i Malm�/Kungliga Musikh�gskolan i Stockholm}\\{\small mail@henrikfrisk.com}}
% \date{}

\begin{document}
\selectlanguage{english}
\maketitle

\thispagestyle{empty}

%\section{Introduction}

\noindent
Fokus i detta work-in-progress kretsar kring f�rst�elser av jaget i improvisation och interaktion i min egna konstn�rliga praktik, huvudsakligen som saxofonist, improvisat�r, med eller utan liveelektronik. Tanken �r att problematisera och delvis omv�rdera f�rst�elsen av jaget i de sammanhang som jag �r verksam i, utifr�n fr�gest�llningar som, till exempel, hur min v�sterl�ndska identitet p�verkar och p�verkas av m�tet med andra kulturer, och hur min relation till `den andra' p�verkas av de situationer jag befinner mig i. M�nniska-maskin interaktion �r ett specialfall i detta sammanhang: Vad �r betydelsen av min identitet och sj�lvuppfattning n�r den part jag interagerar med saknar ett 'jag'? Helt central i denna unders�kning �r begreppet motst�nd; idiomatiskt motst�nd, instrumentets motst�nd, psykologiskt motst�nd, kulturellt motst�nd, teknikens motst�nd, etc., och i teknikinteraktionen uppst�r nya typer av motst�nd i �verg�ngarna mellan det reella och det virtuella.

\end{document}