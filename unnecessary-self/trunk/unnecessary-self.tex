\documentclass[a4paper]{article}
\usepackage[swedish, english]{babel}
\usepackage[T1]{fontenc}
\usepackage[authoryear,round]{natbib}

\makeatletter
\renewcommand\bibsection%
{
  \section*{Referenser
    \@mkboth{\MakeUppercase{\refname}}{\MakeUppercase{\refname}}}
}
\makeatother

\usepackage{graphicx}

\renewcommand{\rmdefault}{pad}

\usepackage{fancyhdr}
\pagestyle{fancy}

\lhead{\small{\textit{Henrik Frisk}}}
\chead{}
\rhead{\small{\textit{Det (o)nödvändiga jaget}}}

\title{Det (o)nödvändiga jaget: Improvisation, motstånd och interaktion\\\vspace{.6cm}
\large{Konstlab, Göteborg, 31/10, 2011}}
\author{Henrik Frisk, PhD\\{\small Musikhögskolan i Malmö/Kungliga Musikhögskolan i Stockholm}\\{\small mail@henrikfrisk.com}}
\date{}

\begin{document}
\selectlanguage{english}
\maketitle

\thispagestyle{empty}

%\section{Introduction}

\noindent

\section{Method}
\label{sec:method}

Integrative Learning comes in many varieties: connecting skills and knowledge from multiple sources and experiences; applying skills and practices in various settings; utilizing diverse and even contradictory points of view; and, understanding issues and positions contextually."


``According to Somers and Gibson (1994, 38), social groups often perform such constructions "to consolidate a cohesive self-identity and collective project.'' \citep[p. 103]{lewis-1}

\section{Abstract}

The focus of this project revolves around the understanding of the self in improvisation and interaction in the context of an artistic musical practice with and without electronics. The ambition is to further understand the meaning and impact of the self in those contexts. Key agents here are freedom, resistance and identity. 
%idiomatic resistance, instrumental resistance, psychological resistance, %cultural resistance, and so forth.

Although very difficult to define, freedom in general is a recurring concept in the discussion of improvisation. On the surface improvisation may seem as a means to create music that is free from the chains of the formal structures that notation or idiomatics imposes on the expressive possibilities of the musician, and open it up to the immediate and unmediated influence of the individual. A music that may be created on the spot and whose substance is defined, not so much by external factors, as by the will of the improviser(s). Ornette Coleman's important 1959 recording ``Free Jazz'' has contributed significantly to the idea that improvisation and freedom are coupled.\footnote{According to XXX however, Coleman's intention with the title was that it be read as an uppmaning, that jazz needed to be set free, rather than a defining name for the music they played. The fact that most of the music on the record was played to a very loose groove as played by Billy Higgins and lacked formal and harmonic structures owes a lot to the confusion.} Eventually it gave name to the Free jazz movement that evolved in Europe as well as in the US where the demand for social freedom in the civil rights movement found a parallel in the free jazz.

This social dimension is significant. freedom in improvised music is not


%Furthermore, the occurence of non-improvised music, which is a nececessary counterpart for identifying improvisation as an expression of freedom, is inferior to the tradition of improvised music, in time as well as geographically. It is almost only in the occidental world, in the last three to four hundred years that music has been notated rather than communicated orally. The music of the rest of the world has always had improvisation as a defining element.


To think that improvisation per se is freedom is, however, an error. 

Improvisation, however, is no guarantee for that expressive freedom. Some improvisers are so intimitely bound to a particular aesthetics or style that the notion of freedom may appear to be completely lost. Each instrument, musician and musical as well as social constellation constructs their own context against which any expression will be valued and compared. Even in very small groups it is possible to identify value systems with relative freedom. In the paper ``Negotiating the Musical Work'' we discussed the notion of subculture:

\begin{quotation}
  we might try to approach this symbolic system in relation to a common   context, or subculture created by the agents involved in it. Both composer   and performer are working within the frame of their own cultural contexts   which defines their respective understandings of the evolving work. The   subculture is a result of interaction, and negotiation ('\emph{What is it we     are developing?}', '\emph{How are we talking about it?}', etc.), between   the two agents and their inherent cultural contexts. Their mutual   expectations and their understanding or imagination of the work in progress   is of importance when they attempt at co-ordinating their actions, for   instance towards a definition of the performance instructions.
\end{quotation}

Furthermore, there is nothing to indicate that freedom 
%% Detta citat skall sättas i relation till Bateson och The algorithms of the heart.
\begin{quotation}
  The real theme of a work is therefore not the subject the words designate,   but the unconscious themes, the involuntary archetypes in which the words,   but also the colors and the sounds, assume their meaning and their life. Art   is a veritable transmutation of substance. By it, substance is spiritualized   and physical surroundings dematerialized in order to refract essence, that   is, the quality of an original world. This treatment of substance is   indissociable from style.\citep[p.47]{deleuze72}
\end{quotation}

\begin{quoatation}
  Essence is not only individual, it individualizes. \citep[p.43][]{deleuze72}
\end{quoatation}

Kafka, Borroughs, Virilio, Focault, Deleuze and many others have all explored the institutions of control and confinement so essential to the 20th and the beginning of the 21st centuries.

\begin{quotation}
Here too, from the standpoint of a certain Freudianism, we can discover the   principle of an inverse relation between repetition and consciousness,   repetition and remembering, repetition and recognition (the paradox of the   Repetition and Difference 15 ‘burials’ or buried objects): the less one   remembers, the less one is conscious of remembering one’s past, the more one   repeats it - remember and work through the memory in order not to repeat it.   Self-consciousness in recognition appears as the faculty of the future or   the function of the future, the function of the new.  Is it not true that   the only dead who return are those whom one has buried too quickly and too   deeply, without paying them the necessary respects, and that remorse   testifies less to an excess of memory than to a powerlessness or to a   failure in the working through of a memory? \citep[p. 14-5][]{deleuze94}
\end{quotation}

\begin{quotation}
  When the continuity of affective escape is put into words, it tends to take   on positive connotations. For it is nothing less than the perception of   one's own vitality, one's sense of aliveness, of changeability (often   signified as "freedom"). One's "sense of aliveness" is a continuous,   nonconscious self-perception (unconscious self-reflection). It is the   perception of this self-perception, its naming and making conscious, that   allows affect to be effectively analyzed-as long as a vocabulary can be   found for that which is imperceptible but whose escape from perception   cannot but be perceived, as long as one is alive. \citep{massumi95}
\end{quotation}

The beginning of contemporary jazz that exploaded in the 1950s with the Be Bop movement preceeded the related reaction against serialism most prolifically promoted by John Cage has not been acknowledged for its influence. Chance operations and indeterminancy, so effectively denounced by Pierre Boulez et al. in their 1964 article Alea \citep{boulez64} ``one has chosen henceforth to be meticulous in imprecision''



The personal expression is of great importance in many art forms. In jazz, to develop a sound of your own is critical and many of the great jazz musicians such as Coleman Hawkins, Betty Carter, Charlie Parker, Billie Holiday, Lennie Tristano, Carla Bley, Albert Ayler have in some ways redefined and stretched the limits of their instruments through their highly skilled, individual and original output. Their sound has become their particular identity. Although the search for an individual sound in most cases is a very conscious act there is a corresponding search for the pure, or unconscious, expression, exemplified by Ornette Coleman's attempts to short-circuit the habitual aspects of his saxophone playing, instead picking up the trumpet and the violin. Marcel Duchamp similarly spoke about ridding himself of acquired knowledge: ``I unlearned to draw. The  point was to forget \emph{with my hand}.'' 
%\citep[Duchamp, as quoted in][s.29]{tomkins65} 
Like Coleman, Duchamp was using a (new) tool (the ruler) to revolt against the tradition and the expression ``forget with my hand'' is significant here as it puts the focus on the physicality of the action.

But the particularity of the self can also, in certain cases, get in the way. Simon Emmerson reconsiders Trevor Wishart's ideas on sonic masking applying them on the meeting between two traditions. Aspects of one sound from one tradition may mask those of another or, slightly rephrased, one `self', e.g. the conscious, may mask another, e.g. the unconscious. Dividing up the self in a conscious and an unconscious part may, along with other dichotomies surrounding the self, be questioned. In his non-dualistic thinking the American writer David Henry Thoreau, who anticipated John Cage's ideas of non-intentionality by almost a century, makes the point that it is not until you cease to try to understand that you can truly see.
 
To understand and unwrap och vad måste jag lämna utanför min kontroll? Vad är skillnaden mellan estetik och immanent behov och hur kan den skillnaden, om den alls finns, förstås?

Jag ska börja med några berättelser som både visar på bakgrunden till varför jag intresserar mig för dessa frågor och samtidigt kontextualiserar och härleder frågeställningarna till min egen verksamhet.

Ten years ago I was preparing for a concert on a tour arranged by the Swedish concert institute. On this tour I was playing with a free jazz quartet (saxophone, guitar, bass and homebuilt instruments, and drums) and for different reasons I found it very difficult to find my way into the music of this group. Among the members of the group there were different opinions about aesthetics and about the way in which we should apporach our task. The problems grew throughout the tour and by the time we arrived to the small town where we were playing the night in question I felt like the situation was getting out of hand. I had no code to follow except for my own, which I felt was quite out of hand in the present context. This particular night we were at a venue that had a long tradition of presenting national and international jazz, for the most part somewhat more traditional acts than ours were represented. This only added to my frustration and my feeling of being outside of my own comfort zone. About an hour before the show a senior amateur saxophonist came into the backroom to chat with me. I had met him before and he was very keen on talking to me about mouthpieces, saxophones and reeds. We chatted for a while but I started to feel the panic; how could I possibly satisfiy the expectations of this man, the other members of the band and the rest of the audience at the same time? Their expectations, I felt, represented three very different attitudes towards musik making. What artistic method could I employ to solve this dilemma whithout loosing myself in the process? By the time I was going up on the bandstand I was so confused and hindered that I started to doubt whether I could at all play. In a moment of clairvoyance, however, it occured to me that the only option I had was to try to play as if I had never seen a saxophone before. I played as if I had no real conception of what a saxophone should sound like, let alone how it should be played. To play as if I did not know how to play.

%Den första utspelade sig för ganska exakt tio år sedan under en Rikskonserterturné. Av olika anledningar hade jag svårt att på ett naturligt sätt hitta rätt i musiken vi spelade. Problemet blev allt större och när vi kom till ett traditionstyngd jazzklubb i Norrköping och en äldre amatörsaxofonist som jag träffat vid en tidigare spelning på samma ställe kom in i logen innan konserten och började prata med mig om saxofoner, munstycken och rör blev det så småningom akut. När det var dags att gå upp på scenen kännde jag mig så förvirrad och hämmad att jag inte hade en aning om hur jag skulle ta mig igenom kvällen. I ett ögonblick av klarsynthet - eventuellt var det så att det var det enda möjliga valet - bestämde jag mig för att spela som om jag inte kunde spela saxofon, att spela utan att bry mig om om det jag spelade lät ``bra''. Spela som om jag inte kunde spela saxofon.

Another concert scene with a long tradition of presenting improvised music in Stockholm was promoting a concert with a temporary group doing a small tour in Scandinavia. This group featured a few imporovisers and one classically trained musician and the tour was working out quite well. In the group we found ways to deal with the obstacles and issues that came up in performances and otherwise. This particular concert was no exception. Halfway through the econd set, however, I thought I recognized a person coming into the room and as a consequence my playing changed. It was a fairly radical change and even when listening back to the recording I could hear the difference. I even managed to pull my fellow musicians along. I was suriprised in the moment and while listening back to the concert at how abrupt the change was and I felt embarassed. I was taught that my expression and my artistic choices should come from the inside. To be a strong musician ment to not let yourself be influenced by external factors but to be in control. I should be the master of my own playing and I  solely should make the choices.

%Den andra historien utspelar sig på en likaledes traditionstyngd jazzscen i Stockholm. Jag spelade med en liten hybridgrupp med både improvisationsmusiker och klassiska musiker. Konserten fungerar bra men halvägs in i andra avdelningen kom en person in i lokalen som jag kännde och som en direkt följd av detta ändras min spelstil. Förändringen var ganska radikal, även i efterhand när jag lyssnar på inspelningen kan jag höra förändringen. Jag blev själv oerhört förvånad och kännde mig lite skamsen; i min uppfattning borde mitt spel och mitt uttryck komma innifrån, inte utifrån. Jag borde själv vara herre över vad jag spelade eller inte spelade.

In 2005 Stefan Östersjö and myself initiated a long and still standing collaboration with two vietnamese musicians. The dean of the artistic faculty of Lund University Håkan Lundström had helped us to get in touch with Thanh Thuy and Tranh My and the first time we met all four of us at the Malmö Academy of Music we set out to try some sketches of mine. These were basically structured improvisations and the ambition was that these would provide input for a new piece for the quartet. When the composition was completed it was named The Six Tones which also gave name to the group. In this context I became incredibly self aware of the asymmetry between Stefan and I, the two western men, and our Vietnamese female colleagues. I was afraid that my activities and my ideas would mask their origin and the culture they carried which, given the history of Vietnam in particular and the history of the white man in general, may actually have been a relevant concern. Furthermore, the thing that I was most interested in was the Vietnamese traditional music and the way they deployed improvisation. If my ambitions or activities would conceal these aspects of the music we created the original idea of this project would have been lost. The consequence of my worries, however, was that I became so hesitant to take any initative that the session almost collapsed. In hindsight Thuy and My told me that more than anything they were confused and wondered whether I at all knew what I wanted or was after. But that was not the problem, I had a clear idea on what it was I was after. The problem was that I could not talk about it nor make it clear out of fear for looking like some kind of oppressor. My self consciousness concerning the general idea of the power relations between myself and others came in the way for my abilities to communicate about my self.

% Den tredje historien är relativt ny och härstammar från början av mitt samarbete med två kvinnliga vietnamesiska musiker och min kollega Stefan Östersjö i gruppen som sedan blev The Six Tones. Det var första gången vi träffades alla fyra och vi skulle jobba med några improvisationer som skulle utgöra utgångspunkten i ett stycke för kvartetten som jag jobbade med. Jag var otroligt självmedveten om den kulturella assymetrin mellan den västerländska mannen och de asiatiska kvinnorna och var så försiktig med att ta initiativ att sessionen närmast havererade. Thuy och My blev mest förvirrade och undrade om jag egentligen visste vad jag ville. Vilket jag i och för sig visste men jag saknade förmågan att i den situationen omsätta det i aktion.

A lot can be said about these three stories. They may not very original, similar experiences may have been had by many musicians and artists. The reason that I bring them up in this context is that together they provide the backdrop to my questions concerning the different manifestations of the self and how these can and will get in the way for expressing ones identity as a reflection of the self in music in general and improvisation in particular. They form the foundation for the current artistic resaerch project and for my interest in the dynamics of the self in improvisation. The common nominator in these stories is the way in which the self, and consciousness about the self and others, alters the planned artistic and expressive trajectories and the focus of this essay is how the impact and meaning of these intersections between the internal and external worlds of artistic expression can be discussed. The questions I am interested in approaching are: What happens when the self gets in the way? When it is not allowed to work freely? When the conscious self and the subconscious appears to be in conflict?

%Det finns mycket att säga om dessa tre berättelser men sammantaget utgör de bakgrunden till varför jag vill bättre förstå jagets dynamik i framförallt improvisation. För det är just de olika manifestationerna av jaget som är den gemensamma nämnaren. Men, jag kommer återkomma till berättelserna så småningom.

To listen to the other is central concept in ethics. To meet the other with respect and understanding regardless of how the other is approaching oneself is intrinsic to the christian message. In the teaching of jazz and improvised music the concept of listening to the other is absolutely essential, although there are many important accounts of the opposite attitude, to concsiously \emph{not} listen. 
% deny the other space for equal opportunities of expression. 
We will return to the many exceptions to active listening and look at some examples where non-listening may actually be equally efficient. But let us first focus on the question of \emph{what} and \emph{who} it is we are listening to when we listen. Should our listening be limited to our fellow musicians or should we also listen to our audiences? When we listen, what is it we allow ourself to be influenced by and what part of our own expression should remain untouched by our listening?

Obviously, the point of listening to the other in performance is not to completely give up the self and become the other but to attune to, or find a resonance with the other. It is in the interaction that the open and unbound improvisation is unfolding, between adjusting to the other while hanging on to the possibility for taking the initative. There are no general wrongs or rights because only in the instantaneous moment can one decide what path to take. In one instance there may be a demand for absolute and unconditioned control and in the next it may be necessary to completely give in to the other. Yet another situation will require one to go with the sound, the audience or the space.

The personal expression is an important agent in many artforms and in jazz it is essential. The personal sound including the actual sound, the phrasing and articulation, the patterns, the melodic style, and the harmonic universe is of paramount importance to the success of the musician. Sonny Stitt always claimed that he came up with his sound and style independently of Charlie Parker but was never fully acknowledged as he appeared as a copy of Bird, less original than the ``true''inventor of BeBop. Parker was more original because he came forth prior to Stitt. Most succesful jazz artist have redefined the possibilities of their instruments in one way or another and achieved an undsiputed level of originality. Coleman Hawkins, Betty Carter, Charlie Parker, Billie Holiday, Carla Bley, Albert Ayler are all musicians whose sound has become their identity.

In improvised music, according to my own experience it is difficult 

Att lyssna på den andre är centralt i den västerländska och kristna etiken. Att bemöta den andre med förståelse och respekt oavsett hur denna bemöter en själv. Detta är också en viktig princip i jazzen och improvisationen. Det är i samspel med den andre som den dynamiska improvisationen vecklar ut sig. Men, vad är det man lyssnar på när man lyssnar på den andre? Och på vilket sätt skall man påverkas av den andre och vad ska man låta förbli opåverkat? Är det bara dem man spelar med som man ska lysssna på eller ska lyssnandet även inkludera publiken?

I jazzen, liksom i många konstformer, är det personliga uttrycket viktigt. Att bygga upp en klangvärld som är typisk för en själv är helt centralt och alla stora jazzmusiker har i någon mån omdefinierat sitt instrument genom sin särpräglade spelstil; Coleman Hawkins, Betty Carter, Charlie Parker, Billie Holiday, Carla Bley, Albert Ayler. Deras identitet är deras `sound'. Vi behöver förvisso inte begränsa oss till jazzen, sedan länge (åtminstone sedan modernismen) är originalitet (som inte minst Dalhaus har påpekat) ett estetiskt värdeord. Alla stora uppovsmän och kvinnor uppvisar originalitet och särskiljer sig från mångden i sin egen samtid.

Samtidigt, som en parallell rörelse, finns sökandet efter det ``rena'' uttrycket, det som passerar förbi medvetandet, förbi det självmedvetna jaget. Ibland är strävan efter orignaliteten själva källan till sökandet efter det av medvetnadet obefläckade uttrycket; om varje individ är unik borde också det genuint personliga uttrycket vara originellt. Ornette Coleman talar om ett så spontant skapande som möjligt, om en kreativitet utan minne.\footnote{Se \citet[s.117]{litzweiler92}.} Han talar om hur hans spel innan han nådde framgångar var mera ärligt än det sedan blev och valde att börja spela trumpet och violin för att slippa onödig kunskap.\footnote{Intervju med Ornette Coleman i \citet[s.33]{taylor77}} För övrigt påfallande likt Marcel Duchamps tal om traditionens fängelse och att glömma med handen: ``I unlearned to draw. The  point was to forget \emph{with my hand}.'' \citep[Duchamp, citerad i][s.29]{tomkins65}. Ett i mina öron djupt fascinerande citat som inte minst drar undan mattan för den cartesiska dualismen och separationen av kropp och själ.


På skivan ``The empty foxhole'' från 1966\footnote{\citet{coleman66}} spelar han tillsammans med sin tioåriga son Denardo Coleman på trummor och beskriver sin tillfredställelse över att spela med någon som inte behövde bry sig om kritiker eller konsertarrangörer, utan som kunde spela och vara fri.\footnote{\citet[Ornette Coleman citerad i][s.121]{litzweiler92}} Detta hör Ornette för att han lyssnar på Denardo men också för att han har förmågan att lyssna på sig själv; han kan ju konstatera att sonen besitter en egenskap han själv har förlorat. Det yttre lyssnandet, att lyssna på den andre, kompletteras av ett inre lyssnande. Och, jämfört med att lyssna på den andre så är det betydligt svårare att lyssna på sig själv.

Är det då möjligt att samtidigt manifestera sin identitet, att låta jaget kontrollera uttrycket, och samtidigt vara fri från (över)jagets inflytande? Finns det en motsättning? John Cage såg det som sin uppgift att befria musiken från jagets intentioner, vare sig det rörde sig om hans egen intentionalitet eller musikerns. Och även om han hade lite till övers för jazzen, och egentligen inte var intresserad av improvisation, ligger det nära till hands att se kopplingar mellan Colemans metoder och Cages ambitioner. Den fenomenologiska intentionaliteten kan sägas vara just kopplingen mellan den inre och den yttre världen men Cage ville inte att vare sig han själv eller de musiker som framförde hans musik skulle avse eller styra flödet: ``I have nothing to say, and I am saying it''. Alltså, till skillnad från jazzens fokus på identitet och individuellt uttryck vill Cage uppleva världen och omgivningen så som den är, inte skapa en kopia av den.

<första historien>

Turning back to Duchamp, who introduced the notion of the ``personal
'art coefficient' '' , in a discussion concerning the creative
act. Specifically Duchamp is referring to the immanent processes,
those in which the artist alone is involved and which ultimately lead
to ``art in the raw state---\`{a} l'\'{e}tat brut'' and, in short, it
constitutes the difference between the artistic intention and its
realization. The gap goes unnoticed by the artist and it represents
his (or her) inability ``to express fully his
intention''\footnote{\citet{duchamps57}} Duchamp concludes that ``the personal
'art coefficient' is like an arithmetical relation between the
unexpressed but intended and the unintentionally
expressed.''\footnote{\citet{duchamps57}} It is primarily the description of
the art coefficient as a \emph{relation} that is interesting with
regard to the current discussion.

Kanske kan man förklara de två polerna - det inre lyssnandet och det yttre lyssnandet som ett uttryck för en cartesisk dualism. En åtskillnad av det fysiska och det metafysiska, mellan själ och materia. Den amerikanske författaren David Henry Thoreau, som föregrep John Cages idéer om ickeintentionalitet med närmare hundra år, förespråkade tvärtom en samexistens mellan naturen och jaget, mellan de yttre och inre intrycken.\footnote{\citet[s.37]{shultis98}} När han gör en distinktion mellan att se (``seeing'') och att titta (``looking''), där det senare implicerar att man också försöker förstå, förespråkar han i grund och botten samma filosofi, eller estetik, som Cage. Det är först när man upphör att försöka förstå som man verkligen kan betrakta eller höra.\footnote{\citet{thoreau41}} Även om Thoreau räknas till transcendentalisterna är han också en objektivist (till skillnad från Emmerson som är en projektivist) som de-centraliserar jaget och ser verkligheten som något som inte uteslutande medieras av detta. Bort från det personliga uttrycket och mot det föränderliga jaget som är mottagligt för yttre influenser. Det genomskinliga, närmast osynliga jaget ställs mot det jag som projicerar sig själv.

<läs avsnitt>

Är det då huvudsakligen uppfattningen av jaget och jagets roll som skiljer Cage från Coleman, eller är det snarare ett estetiskt ställningstagande? Coleman beklagar att hans instrument, altsaxofonen, är så starkt förknippad med Charlie Parker och Cage tar avstånd från Beethoven i synnerhet och harmonik i allmänhet. Eller har de, som Cage nog hade hävdat,  i grunden olika infallsvinklar?

Låt oss för tillfället lämna jaget som något som antingen begränsar och skapar eller jaget som något som betraktar och ``är'' och resonerar med naturen och istället betrakta en mer Freudianskt uppdelning av jaget. I Freudiansk teori delar man upp mental aktivitet i primära och sekundära processer. De primära är icke-verbala och drömlika och företar de sekundära som är det reflekterande och medvetna jagets uttryck. Eller som antropologen Gregory Bateson uttrycker det: ``Drömmar är metaforer kodade som primära processer''. Och konst generellt är: ``an exercise in communicating about the species of
unconsciousness [\ldots] a play behaviour whose function is [\ldots]
to practice and make more perfect communication of this
kind.''\footnote{\citet[p. 137]{bateson72}}

Men det kommer alltid att finnas en motsättning mellan primär och sekundärt kodad information och inget annat än förvirring kommer bli resultatet när man försöker avkoda det omedvetna och dokumentera det i det medvetnas språk som alla som har försökt att förklara en dröm i ord har erfarit:

\begin{quote}
  [The] algorithms of the heart, or, as they say, of the unconscious,
  are, however, coded and organized in a manner totally different from
  the algorithms of language. And since a great deal of conscious
  thought is structured in terms of the logics of language, the
  algorithms of the unconscious are double inaccessible. It is not
  only that the conscious mind has poor access to this material, but
  also that when such access is achieved. \emph{e.g.}, in dreams,
  art, poetry, religion, intoxication, and the like, there is still a
  formidable problem of translation.\footnote{\citet[p. 139]{bateson72}}
\end{quote}

I primära processer är enligt Bateson fokus i diskursen på relationen; relationen mellan jaget och andra eller relationen mellan jaget och omgivningen. Med andra ord, det omedvetnas processer kommuniceras i mötet med andra, inte i en översättning utan genom mötet i sin egen logik. Och, kanske mer än någonting annat är improvisation en synnerligen effektiv metod som har potential att tillgängligöra de primära processerna. 

Och, för att återkoppla till mina inledande berättelser så kan man förstå mitt förändrade spel när personen jag känner kommer in i rummet som en helt naturlig konsekvens av att ogivningen, och dess förutsättningar förändrades, och att det därför är helt naturligt att mitt spel förändrades. Det var helt enkelt en kommunikation i primära processer. I den tredje historien så är det avsaknaden av relation som i kombination med en analytisk (teoretisk) och reflekterande utgångspunkt som gör att kommunikationsproblemen blir åtskilligt större än våra språkliga problem.

Att inte lyssna på den andre behöver inte vara oetiskt i improvisation. Det är tvärtom en viktig möjlighet och handlar mer om att lysnnandet riktas inåt än att man inte lyssnar alls. Jaget är inte antingen det passiva lyssnande jaget eller det extroverta, projicerande men för att bibehålla kontakten med det inre är det nådvändigt att motstå vanan och nödvändigt att utveckla metoder för att försäkra sig om att det inte är vanan som styr utan lyssnandet. och därför måste jaget ibland ges upp för att det ska bli möjligt att återfinna det och den information och kunskap som det bär på.


\vspace{0.5cm}
...
\vspace{0.5cm}

Maskin och människa är två, till synes, i grunden inkompatibla enheter. Den ena är oerhört dynamisk och självanpassande, den andra behöver ständig hjälp och vägledning för att åstadkomma något överhuvudtaget. Och även om båda dessa beskrivningar kan passa in på båda enheterna så gör de det sällan inför samma utmaning eller i samma kontext. Man kan tycka att de i så fall borde utgöra varandras helt naturliga komplement, men så är det inte heller. Problemet, menar jag, kan sammanfattas i bristande förståelse för hur de två kan interagera med varandra, hur rummet \emph{dem emellan} skall utformas. Av den anledningen är denna interaktion speciellt intressant i det sammanhang där vi försöker förstå jagets dynamik. Min interaktion med datorn kan genuint sakna intention samtidigt som den i någon mening kräver att jag genom den kontrollerar maskinen

% (Här har utvecklingen gått långsamt: liksom på sextiotalet skriver vi alltsom oftast på ett tangentbord för att kommunicera med datorn för att bara nämna ett exempel.)


% Turning back to Duchamp, who introduced the notion of the ``personal
% 'art coefficient' '' , in a discussion concerning the creative
% act. Specifically Duchamp is referring to the immanent processes,
% those in which the artist alone is involved and which ultimately lead
% to ``art in the raw state---\`{a} l'\'{e}tat brut'' and, in short, it
% constitutes the difference between the artistic intention and its
% realization. The gap goes unnoticed by the artist and it represents
% his (or her) inability ``to express fully his
% intention''\footnote{\citet{duchamps57}} Duchamp concludes that ``the personal
% 'art coefficient' is like an arithmetical relation between the
% unexpressed but intended and the unintentionally
% expressed.''\footnote{\citet{duchamps57}} It is primarily the description of
% the art coefficient as a \emph{relation} that is interesting with
% regard to the current discussion. The roles of the two factors
% (unintended and intended) are likely to be different in free
% improvised music and the 'intention' that Duchamp is speaking of here
% is probably not meaningful in relation to Coleman and other proponents
% of the free jazz movement.\footnote{From within the field of jazz and
%   free form the ``I just play what I hear'' paradigm is common and it
%   is difficult to see how 'intention' in the sense here discussed can
%   be part of the equation. However, this is from within. From an
%   analytical point of view ``I just play what I hear'' is most
%   certainly an intention.}  But the notion of the ``unintentionally
% expressed'' as having a relation to the ``unexpressed but intended''
% is useful when looking at Coleman's unorthodox play of the trumpet and
% the violin and something we may use when we go back to the subject of
% computers and improvisation. Furthermore, the personal art coefficient
% possibly holds an abstract relation to the notion of the structure of
% the unconscious, whose discourse is also focused on the relationships.

\bibliography{/home/henrikfr/Documents/svn/admin/conf/biblio/bibliography} \bibliographystyle{plainnat}
\end{document}