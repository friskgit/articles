
\documentclass[a4paper]{article}
\usepackage[swedish, english]{babel}
\usepackage[T1]{fontenc}

\usepackage{graphicx}

\usepackage{fancyhdr}
\pagestyle{fancy}

\lhead{\small{\textit{Henrik Frisk}}}
\chead{}
\rhead{\small{\textit{Improvisation och maskiner: Motstånd och interaktion}}}

\title{Improvising the Self: Machines, resistances and interaction in musical improvisation}
\author{Henrik Frisk, PhD\\{\small Malmö Academy of Music - Lund University}\\{\small henrik.frisk@mhm.lu.se}}
\date{\today}

\begin{document}
\selectlanguage{english}
\maketitle

\thispagestyle{empty}

\section{Abstract}

The focus of this project revolves around the understanding of the self in improvisation and interaction in the context of my artistic work with and without electronics. My ambition is to further understand the meaning and impact of the self in those contexts. A key agent here is resistance; idiomatic resistance, instrumental resistance, psychological resistance, cultural resistance, to only mention a few.

The personal expression is of great importance in many art forms. Although the search for an individual sound in most cases is a very conscious act there is also a corresponding search for the pure, somewhat unconscious, expression.  Ornette Coleman's attempts to short-circuit the habits of his saxophone playing, instead picking up the trumpet and violin is one example and Marcel Duchamp who similarly spoke about ridding himself of acquired knowledge (``I unlearned to draw. The point was to forget with my hand'') is another.

The particularity of the self can get in the way in other ways too. The British composer Simon Emmerson reconsiders Trevor Wishart's ideas on sonic masking applying them on the meeting between two traditions in which the character of one tradition can mask the identity of the other. According to Gregory Bateson the operations of the unconscious follows a different kind of logic from conscious, reflective thoughts and points out that art is a way to gain access to the information streams of the unconscious. However, in the non-dualistic thinking of the American writer David Henry Thoreau, that anticipated John Cage's ideas of non-intentionality by almost a century, the difference between the conscious and unconscious self is less important. 

In what ways can these different ideas on the self and identity be useful to my own work and my own understanding of the self in improvisation? The presentation will make use of my own artistic practice as a means to discuss in what ways I can get access to the different aspects of the self and their friction with the other agents involved in improvisation.

\end{document}
%%% Local Variables: 
%%% mode: latex
%%% TeX-master: t
%%% End: 
