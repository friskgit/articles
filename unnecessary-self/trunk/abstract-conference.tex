
\documentclass[a4paper]{article}
\usepackage[swedish, english]{babel}
\usepackage[T1]{fontenc}

\usepackage{graphicx}

\usepackage{fancyhdr}
\pagestyle{fancy}

\lhead{\small{\textit{Henrik Frisk}}}
\chead{}
\rhead{\small{\textit{Improvisation och maskiner: Motstånd och interaktion}}}

\title{Improvising the Self: Machines, resistances and interaction in musical improvisation}
\author{Henrik Frisk, PhD\\{\small Malmö Academy of Music - Lund University}\\{\small henrik.frisk@mhm.lu.se}}
\date{\today}

\begin{document}
\selectlanguage{english}
\maketitle

\thispagestyle{empty}

\section{Abstract}


The focus of this project revolves around the understanding of the self in improvisation and interaction in the context of an artistic practice with and without electronics, with the ambition to further understand the meaning and impact of the self in those contexts. A key agent here is \emph{resistance}; idiomatic resistance, instrumental resistance, psychological resistance, cultural resistance, and so forth.

The personal expression is of great importance in many other art forms. In jazz, to develop a sound of your own is critical and many of the great jazz musicians such as Coleman Hawkins, Betty Carter, Charlie Parker, Billie Holiday, Lennie Tristano, Carla Bley, Albert Ayler have in some ways redefined and stretched the limits of their instruments through their highly skilled, individual and original output. Their sound has become their particular identity. Although the search for an individual sound in most cases is a very conscious act there is a corresponding search for the pure, or unconscious, expression, exemplified by Ornette Coleman's attempts to short-circuit the habits of his   saxophone playing, instead picking up the trumpet and violin. Marcel Duchamp similarly spoke about ridding himself of acquired knowledge: ``I unlearned to draw. The  point was to forget \emph{with my hand}.'' 
%\citep[Duchamp, as quoted in][s.29]{tomkins65} 
Like Coleman, Duchamp was using a (new) tool (the ruler) to revolt against the tradition and the expression ``forget with my hand'' is significant here as it puts the focus on the physicality of the action.

But the particularity of the self can also, in certain cases, get in the way. Simon Emmerson reconsiders Trevor Wishart's ideas on sonic masking applying them on the meeting between two traditions. Aspects of one sound from one tradition may mask those of another or, slightly rephrased, one `self', e.g. the conscious, may mask another, e.g. the unconscious. Dividing up the self in a conscious and an unconscious part may along with other dichotomies surrounding the self be questioned. In his non-dualistic thinking the American writer David Henry Thoreau, who anticipated John Cage's ideas of non-intentionality by almost a century, makes the point that it is not until you cease to try to understand that you can truly see.
 
To understand and unwrap the significance of the different aspects of the meaning of the self in artistic practice I will make use of the notion of \emph{primary process} as the operations of the unconscious, leaning on Gregory Bateson, who points out that art is a way to gain access to the information streams of the unconscious which are, in Freudian language, structured in terms of \emph{primary process}


---


The focus of this project revolves around the understanding of the self in improvisation and interaction in the context of my artistic work with and without electronics. My ambition is to further understand the meaning and impact of the self in those contexts. A key agent here is resistance; idiomatic resistance, instrumental resistance, psychological resistance, cultural resistance, to only mention a few.

The personal expression is of great importance in many art forms.  In jazz, to develop a sound of your own is critical and many of the great jazz musicians such as Lester Young, Charlie Parker, Geri Allen, Billie Holiday, Lennie Tristano, Carla Bley and Albert Ayler have in some ways redefined and stretched the limits of their instruments through their highly skilled, individual and original output. Their sound has become their identity. Although the search for an individual sound in most cases is a very conscious act there is a corresponding search for the pure, somewhat unconscious, expression, exemplied by Ornette Coleman's attempts to short-circuit the habits of his saxophone playing, instead picking up the trumpet and violin. Marcel Duchamp similarly spoke about ridding himself of acquired knowledge: “I unlearned to draw. The point was to forget with my hand.”

But the particularity of the self can also, in certain cases, get in the way.  Simon Emmerson reconsiders Trevor Wishart's ideas on sonic masking applying them on the meeting between two traditions. Aspects of one sound from one tradition may mask those of another or, slightly rephrased, one notion of the self, e.g. the conscious, may mask another, e.g. the unconscious. According to Gregory Bateson the operations of the unconscious follows a different kind of logic and points out that art is a way to gain access to the information streams of the unconscious. Dividing up the self in a conscious and an unconscious part may be questioned along with other dichotomies surrounding the self. In his non-dualistic thinking the American writer David Henry  Thoreau, who anticipated John Cage's ideas of non-intentionality by almost a century, is discussed.

To listen and remain open to the other is a way to understand the self and the structures of fear and power within the self. Perhaps equally important and difficult, however, is to be able to listen to the self. In this presentation I will use my own artistic practice as a means to further understand in what ways I can get access to the different aspects of the self and their friction with the other agents involved in improvisation.

%\footcite[139]{bateson72}



 %  Habitual muscular responses, learned ``patterns'', which may even get % triggered unconsciously are common and known to all who have played and % practiced a musical instrument. What we see here and in Coleman's use of % alternate instruments, is an attempt to subdue the influence these % ``patterns'' may have on the artistic output with the primary goal to get % closer to the pure subjectivity, or the pure personal expression.

% Art is in itself a way to gain access to the information streams of
% the unconscious which are, in Freudian language, ``structured in terms
% of \emph{primary process}, while the thoughts of consciousness
% (especially verbalized thoughts) are expressed in \emph{secondary
%   process}''.\footcite[139]{bateson72} 


% Anthropologist Gregory Bateson points out that 

% Is it possible, or even desirable, to unlearn what 

% This project continues some of the ideas introduced in my 2008 artistic PhD % dissertation \emph{Improvisation, Computers, and Interaction: Rethinking % Human-Computer Interaction Through Music}


% I jazzen, liksom i många konstformer, är det personliga uttrycket viktigt. Att % bygga upp en klangvärld som är typisk för en själv är helt central och alla % stora jazzmusiker har i någon mån omdefinierat sitt instrument genom sin % särpräglade spelstil; Coleman Hawkins, Betty Carter, Charlie Parker, Billie % Holiday, Carla Bley, Albert Ayler. Deras identitet är deras `sound'. % Samtidigt, som en parallell rörelse, finns sökandet efter det ``rena'' % uttrycket, det som passerar förbi medvetandet, förbi det självmedvetna jaget. % Ornette Coleman talar om ett så spontant skapande som möjligt, om en % kreativitet utan minne.\footnote{Se \citet[s.117]{litzweiler92}. För övrigt % påfallande likt Marcel Duchamps tal om traditionens fängelse och att glömma % med handen: ``I unlearned to draw. The  point was to forget \emph{with my % hand}.'' \citep[Duchamp, citerad i][s.29]{tomkins65}.} Han talar om hur hans % spel innan han nådde framgångar var mera ärligt än det sedan blev och valde % att börja spela trumpet och violin för att slippa onödig % kunskap.\footnote{Intervju med Ornette Coleman i \citet[s.33]{taylor77}} På % skivan ``The empty foxhole'' från 1966\footnote{\citet{coleman66}} spelar han % tillsammans med sin tioåriga son Denardo Coleman på trummor och beskriver sin % tillfredställelse över att spela med någon som inte behövde bry sig om % kritiker eller konsertarrangörer, utan som kunde spela och vara % fri.\footnote{\citet[Ornette Coleman citerad i][s.121]{litzweiler92}}

% Är det då möjligt att samtidigt manifestera sin identitet, att låta jaget % kontrollera uttrycket, och samtidigt vara fri från (över)jagets inflytande? % Finns det en motsättning? John Cage såg det som sin uppgift att befria musiken % från jagets intentioner, vare sig det rörde sig om hans egen intentionalitet % eller musikerns. Och även om han hade lite till övers för jazzen, och % egentligen inte var intresserad av improvisation, ligger det nära till hands % att se kopplingar mellan Colemans metoder och Cages ambitioner. Den % amerikanske författaren David Henry Thoreau, som föregrep John Cages idéer om % ickeintentionalitet med närmare hundra år, bröt med den tidsenliga dualismen % och förespråkade en samexistens mellan naturen och jaget, mellan de yttre % intrycken och jaget.\footnote{\citet[s.37]{shultis98}} Han kommer också nära % Coleman när han gör en distinktion mellan att se (``seeing'') och att titta % (``looking''), där det senare implicerar att man också försöker förstå. Det är % först när man upphör att försöka förstå som man verkligen kan % betrakta.\footnote{\citet{thoreau41}}
\end{document}