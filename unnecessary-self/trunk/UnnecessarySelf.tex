\documentclass[a4paper]{article}
\usepackage[swedish, english]{babel}
\usepackage[T1]{fontenc}
\usepackage[authoryear,round]{natbib}
\usepackage[utf8]{inputenc}

\makeatletter
\renewcommand\bibsection%
{
  \section*{References
    \@mkboth{\MakeUppercase{\refname}}{\MakeUppercase{\refname}}}
}
\makeatother

\usepackage{graphicx}

\usepackage{fancyhdr}
\pagestyle{fancy}

\lhead{\small{\textit{Henrik Frisk}}}
\chead{}
\rhead{\small{\textit{}}}

\title{The (un)necessary self\\\vspace{.6cm}
\large{improvisation, freedom and the subliminal}}
\author{Henrik Frisk, PhD\\{\small Malmö Academy of Music}\\{\small henrik.frisk@mhm.lu.se}}
\date{}

\begin{document}
\selectlanguage{english}
\maketitle

\thispagestyle{empty}

%\section*{Introduction}

\noindent

\section*{Introduction}

The focus of this essay revolves around the understanding of the self in improvisation and interaction in the context of an artistic musical practice. The ambition is to further understand the meaning and impact of the self in this practice. The key agents are freedom, identity and originality.

%%% BAKGRUND
The background, and in a sense also the foreground, of these thoughts is to be found in my artistic practice, exemplified by three recollections of experiences from my work. My background in jazz and free improvised music and my otherwise fairly diverse artistic practice, including composition, electronic and interactive music and sound art, has obviously shaped my understanding of the work I do. My principle artistic and aesthetic interests are questions pertaining to open contribution and the fields discussed in this paper, such as freedom, interaction and self. This work is furthermore a continuation of some of the central topics developed in my thesis \citep{frisk08}, such as:

\begin{itemize}
\item \emph{Work-in-movement}. This is a concept established by Umberto \citet{eco68} that I introduced as a work type encompassing radically open works. It requires different modes of representation, as the traditional musical score is too restrictive and is not able to communicate its most central aspects: the collaboration, negotiation and interaction in the conception, realization and documentation of the work. Work-in-movement does not necessarily distinguish between composition and improvisation, although for the latter, some kind of frame is needed for the concept to be meaningful. As a specification it is geared towards modes of interaction and openness involved in all phases of the work.

\item \emph{Interaction-as-difference}. I proposed that in human-computer interaction (HCI) the methodology of control (interaction-as-control) in certain cases should be abandoned in favour of a more dynamic and reciprocal mode of interaction, interaction-as-difference. This kind of interaction is an activity concerned with inducing differences \emph{that make a difference} \citep{bateson72:steps} and suggests parallelism rather than the typical click-and-response mode of HCI. In essence, the movement from \emph{control} to \emph{difference} is a result of rediscovering the power of improvisation as a method for organizing and constructing musical content. Interaction-as-difference is to be understood as an alteration of the more common paradigm of direct manipulation in HCI.
 
\item \emph{Giving up of the self}. I suggested the notion of \emph{giving up of the self} as the common denominator between the two previous concepts and as one of the important conditions for an improvisatory and self-organizing attitude towards musical practice that allows for interaction-as-difference. Only if one is able and willing to accept the loss of priority of interpretation, if one is willing to give up or disregard faithfulness to ideology or idiomatics, is the idea of interaction-as-difference conceivable. Hence, the giving up of the self is not to succumb to someone else but is rather contingent on the degree to which one is willing to engage in a dialogue on the creative process and to allow others to influence it. I.e., by giving up compositional control and replacing it with an interactive negotiation in the form of collaboration, the process realizes all of these three topics.
\end{itemize}

These ideas were eye-openers for me during my work on my thesis, but also, as is commonly the case, they posed new questions and demanded further study. In the romantic tradition of creativity, neither the freedom of the artist nor the autonomy of the work, is negotiatiable. The giving up of the self was an idea that departed from that concept and from the understanding of the artistic self as a strong and defined subjectivity with a clear and organized apprehension of what the artistic output should consist of, what its form and shape should be, and how its general means of operation should be conceived. To give up the self is not to abandon these qualities but to always be willing and prepared to negotiate them. 

%%% FRIHET
Although very difficult to define, freedom in general is a recurring concept in the discussion of improvisation. On the surface improvisation may seem like a means to create music that is free from the chains of the formal structures that notation or idiomatics imposes on the expressive possibilities of the musician, one that opens it up to the immediate and unmediated influence of the individual, music that may be created on the spot and whose substance is defined not so much by external factors as by the will of the improviser(s). Ornette Coleman's important 1959 recording ``Free Jazz'' has contributed significantly to the idea that improvisation and freedom are coupled. It gave name to the free jazz movement that evolved in the US, closely followed by Europe, where the demand for social freedom in the civil rights movement found a parallel in the music.

Whether or not improvisation can be said to be free as such, or can be considered an expression of freedom, is not my main interest here. Rather, my focus is on the significance of the concept of freedom in the practice and discourse on improvisation. How can this notion of freedom, whether negative as in `free-from' or positive as in `free-to' \citep{peters09}, operate alongside other important agents in the improvisatory practice, such as interaction, tradition, context and technique? Neither is it my objective to enter into a discussion of the value of a particular means for organisation of artistic work, or specific aesthetic attitudes in artistic practice. I am, however, interested in how subconscious and embodied knowledge, or intuitive action, is given significance in the creative act, how this significance influences the self and, conversely, how the self is influenced by these aspects. The trajectories of the self may be drawn further and will influence, and be influenced by, much larger systems in the sociopolitical domain.

The iconified Western auteur mentioned above has been under attack at least since the sixties. The mythical creator behind nonnegotiable works of art enjoys a natural freedom of expression and does not have to answer to criticism. My argument \citep{frisk08} is that the view of this Kantian genius is still very influential to how we teach and present art. For myself, merely realizing this was not enough. I needed to more profoundly understand what problems I had with this role in relation to my own artistic practice, and in what ways I could neutralize the expectation of being in control. The decisive moment occurred when I was working on the interactive sound installation \emph{etherSound}, where I was forced to accept that a large part of the compositional decisions had to be made by the users of the system rather than myself. In order to fulfil the idea behind the piece I had to come to terms with the fact that I was not able to restrain the input of the flow of users. I had to give up that which \citet{boulez64} refers to as ``the 'finished' aspect of the Occidental work, its closed cycle'' (p. 51) and approach the ``open work'' that \citet{eco68} discusses, to further attempt ``to reach that point where only language acts, `performs', and not `me'' \citep{barthes77}. Obviously, as an improviser the notion of the open work was familiar to me, but the dynamically open work that I was now approaching was not something I had experience with. Furthermore, the view of the authoritative creator is also something improvisers are often confronted with, and the view of musicians and composers as being absolutely clear about all details of their work is as prominent in improvised music as it is anywhere else. 

My project  was a pursuit to move further away from the kinds of composers, works and authors that \citet{boulez64} discusses in his 1964 article \emph{Alea}, and to instead move towards the convergence point of creation and interaction. This radicalization of the role of the creator asked for a new work concept and an altered view on interaction\footnote{This was however not a linear process. The different ideas grew simultaneously and influenced each other. The control methodology of human-technology interaction gave rise to thoughts on an ethics of technology and questioning my own ambitions in my use of technology inspired a reevaluation and appreciation of the complexity of human and artistic interaction.}, but as a consequence it also called for a review of the self, even after the notion of giving up of the self was staged. The questions that were raised as a result of these ideas in my thesis and that I approach in this text relate to the position of self and other: How can the self enable or discourage interaction? What aspects of the self and of expression should be open for influence from the other? What are the factors that set boundaries on my individual expression, and how may this individuality be manifested without inhibiting the other? What is the meaning of freedom and how can it affect the other? These are inquiries that engage social and political discussion as well as aesthetic positioning.

\section*{I or Eye}
\label{sec:i-or-eye}

\textit{In Stockholm about ten years ago, a venue with a long tradition of presenting improvised music was promoting a concert with a group that I was part of. At the time the group was doing a small tour in Scandinavia and it featured a few improvisers and one classically trained musician. This was a group where we communicated well and always found ways to deal with the obstacles and issues that came up in the performances Despite the fairly low number of visitors in the first set this particular concert was no exception. After intermission there was a bigger crowd and halfway through the second set I thought I recognized a person coming into the room, walking past the stage. As a consequence my playing changed. It was a fairly radical change and even when listening back to the recording I can hear the difference. I also managed to pull my fellow musicians along in this new direction. At the moment I was surprised by the effect the episode had on my playing and listening back to the concert I find it odd that the change occurred so abruptly. I was embarrassed. I had been taught that my expression and my artistic choices should come from the inside. To be a strong musician meant not letting external factors have an influence, but instead being in control of oneself. Hence, according to my view at the time, I should be the master of my own playing and I myself should make the choices, regardless what happens around me. Only those with whom I play may influence me.}\footnote{As an example of an extreme application of this posture is Sonny Rollins' performance at the Opus 40 quarry garden in upstate New York where he falls off the stage in the middle of his solo and breaks his heel, all while he continues to play. After the fall, lying down \citep{rollins86}.}

\subsection*{---}

To listen to the other is a central concept in ethics. To meet the other with respect and understanding regardless of how the other is approaching oneself is intrinsic to the Christian message. In the teaching of jazz and improvised music, the concept of listening to the other is absolutely essential, although there are many important accounts of the opposite attitude, to consciously \emph{not} listen. \emph{What} and \emph{who} is it that we are listening to when we listen? Should our listening be limited to our fellow musicians or should we also listen to our audiences? When we listen, what is it we allow ourselves to be influenced by, and what part of our own expression should remain untouched by our listening? Obviously, the point of listening to the other in performance is not to completely give up the self and become the other but to attune to, or find a resonance with the other. It is in the interaction that the open and unbound improvisation is unfolding, between adjusting to the other while hanging on to the possibility for taking the initiative. Only in the instantaneous moment can one decide what path to take. In one instance there may be a demand for absolute and unconditioned control and in the next it may be necessary to completely give in to the other. What is the position of the self when listening is at the centre?

In the following I will attempt to elaborate on these questions and the idea of the projecting, autonomous performer for whom the individual will is the sole guideline. There are countless stories of intransigent musicians whose artistic and expressive choices are exclusively their own, even when this means they suffer economic loss. Pierre \citet{boulez64} describes how structure is what dictates and constrains the compositional process. The composer alone is in control of this structure. The antagonist in this argument is the American experimental tradition in general and John Cage in particular. If Pierre Boulez is the emblematic representation of the Kantian notion of the true creative genius whose genuinely individual and enigmatic inspiration constitutes the creative force, the American experimental tradition can truly be said to stand in its opposition. The European composer, according to Boulez, takes responsibility for the work and does not neglect the ``the choice inherent in every kind of creation'' (p. 55). He does not give way to become ``meticulous in imprecision'' (ibid. p. 44), as Boulez claims the indeterminist does. With its focus on chance and repetition, this American reaction to several trends in the 20th-century European music is clearly a rewarding scapegoat for Boulez the serialist. 

In the book \emph{Silencing the Sounding Self} Christopher \citet{shultis98} is primarily concerned with the difference between making process-oriented rather than goal-oriented art \emph{within} American music and literature, but his argument can also be seen as an illustration of the more radical difference between serialism and chance operations. The question asked as a point of departure is whether the artist is controlling the process or whether he or she coexists as part of it (p. XVI). With Ralph Waldo Emerson and Charles Ives on one side\footnote{Emerson, who was familiar with Hegel and inspired by the Swedish mystic Emanuel Swedenborg, did promote dualism in \emph{Nature} \citep[e.g. ch. 6, `Idealism'][]{emerson2001}: ``In these cases, by mechanical means, is suggested the difference between the observer and the spectacle,--between man and nature'' (ibid.) and his influence on American metaphysics was substantial. But  he is commonly known to have communicated a nondualistic view as well, as in the essay \emph{The Over-Soul}: ``[The soul] is not wise, but it sees through all things. Then is it glad, young and nimble. It is not called religious, but it is innocent. It calls the light its own, and feels that the grass grows and the stone falls by a law inferior to, and dependent on, its nature. Behold, it saith, I am born into the great, the universal mind''  \citep[p. 250]{emerson2000}.} and Henry David Thoreau and John Cage on the other, he argues that in the work of the former, the self controls the process, whereas in that of the latter, the self coexists with it. The dualist--nondualist dichotomy is what Shultis uses to mark the important change that first Thoreau, and later John Cage, brought to American art. They both embraced a view in which the self and nature were united. 
%They do not impose themselves through their experiments they merely observe. 

In \emph{Walden} \citet{thoreau2004} carries out a two-year-long stay in the New England woods pursuing the exploration of the role of the self in solitude--a project with some interesting kinship to artistic research in that it probes a hypothesis using an experimental method which renders not answers but new questions. The Cartesian notion of dualism, on the other hand, is the idea that mind and matter belong to ontologically different classes. Though they may unite through a transcendental correspondence, nature and humanism are separate from one another. ``For Emerson the human self is in control'' writes \citet[][p. 14]{shultis98}. The abstract symbolism and extensive use of references and quotations are identifying marks of Ives' music and one of the reasons they were dismissed by Cage. I am not convinced that an artist with such great variation and massive output may be successfully exploited as an example of a single aesthetic theory, but it is undeniably so that \emph{Central park in the dark}, for example, is a form of program music that paints a representational sound picture only possible if nature is seen as distinct from the self. Shultis, however, goes further and claims that Ives through his music is ``symbolizing not what was observed but a translation of the observation into something else: a symbol of that something that corresponds to the memory of what happened rather than its actual occurrence'' \citep[p. 27]{shultis98}.

Was my own idea of giving up the self a move towards a nondual view of the act of creation? I am attracted to Thoreau's writings and his philosophy, which I read as a move towards the reinvention of ethical listening and understanding, as well as an attempt to turn to sources of knowledge other than text based. Its somewhat romantic appearance is nonetheless very radical, proved by the fact that he conceptually anticipated one of the more controversial composers of the 20th century by some eighty years. As intriguing as the two poles of the continuum from dual to nondual are, my interest is what happens in the zone in between. In his investigation of literary criticism and music, Christopher Shultis further categorizes Emerson and Thoreau as belonging to the projectionists and objectivists respectively: ``The active `eye' is the projective `I' [\ldots] On the other hand there is the objective `I' which observes rather than projects'' \citep[][p. 61-2]{shultis98}. According to Shultis, for Thoreau the artist, writer or speaker is merely a medium, not a subject. This idea holds a strong intertextual relation to Cage's aesthetics of indeterminacy and non-intention. Artistic expression is not a means to convey a message; if anything, it is a method to create resonances in certain contexts. This seemingly objective and transparent `I' is precisely the critical point that upsets Boulez's systematic and structural agency of production: To avoid or even neglect the qualified projection of the self in determination is simply irresponsible.\footnote{However, I am in doubt that he would embrace the idea of the projective self as promoted in works by Emerson and Ezra Pound, either.} As powerful, engaging and pedagogic as the homophones Eye and I may seem, I will contest that the dividing line may not be so clear. As an expression of the transcendental movement, Emerson's exposition of the transparent eye-ball advocates nature and accentuates nature mysticism much in the same ways that Thoreau did in Walden:

\begin{quote}
I become a transparent eye-ball; I am nothing; I see all; the currents of the Universal   Being circulate through me; I am part or particle of God. The name of the nearest friend   sounds then foreign and accidental: to be brothers, to be acquaintances,--master or   servant, is then a trifle and a disturbance. I am the lover of uncontained and immortal   beauty. In the wilderness, I find something more dear and connate than in streets or   villages. In the tranquil landscape, and especially in the distant line of the horizon,   man beholds somewhat as beautiful as his own nature. \citep[p. 8] {emerson2001}
\end{quote}

Even if there is a distinction between the ideas in \emph{Nature} and Thoreau's famous absent speaker\footnote{``The peculiarity of a work of genius is the absence of the speaker from his speech'' \citep[][p. 264]{thoreau1891}} what is most interesting to me is not the differences in kind but the altered perspective and the movement `in between'. For Emerson, ``transparency remains within the concrete `I' of the self'' \citep[][p. 61]{shultis98}, whereas Thoreau sought to make the \emph{self} transparent. These two methods of approaching the world, as the transparent eye or the transparent self, are not mutually exclusive. It is possible that both the projective eye and the objective I will at times become unstable, and that one may coalesce into the other. Exploring this continuum allows me to reposition within any imaginable listening position and listen to the audience, the critic or myself without the self getting lost: giving up the self in order to retain it. 

\section*{Creation or discovery. Freedom}

\textit{Ten years ago I was preparing for a concert on a tour arranged by the Swedish National Concert Institute. On this tour I was playing with a free jazz quartet (saxophone, guitar, bass and home-built instruments, and drums), and for varoius reasons I found it very difficult to find my way into the music of this group. Among the members of the group there were different opinions about aesthetics and about the way in which we should approach our task. The problems grew throughout the tour and by the time we arrived to the small town where we were playing the night in question I felt like the situation was getting out of hand. I had no code to follow except for my own, which I felt was quite inappropriate in the present context. We were at a venue that had a long tradition of presenting national and international jazz; for the most part, somewhat more traditional acts than ours were represented. This only added to my frustration and my feeling of being outside of my own comfort zone. About an hour before the show a senior amateur saxophonist came into the backroom to chat with me. I had met him before, and he was very keen on talking to me about mouthpieces, saxophones and reeds. We chatted for a while but I started to feel the panic; how could I possibly satisfy the expectations of this man, the other members of the band and the rest of the audience at the same time? Their expectations, I felt, represented three very different attitudes towards music making. What artistic method could I employ to solve this dilemma without losing myself in the process? By the time I was going up on the bandstand I was so confused and hindered that I started to doubt whether I could at all play. In a moment of clairvoyance, however, it occurred to me that the only option I had was to try to play as if I had never seen a saxophone before. I played as if I had no real conception of what a saxophone should sound like, let alone how it should be played: to play as if I did not know how to play.}

\subsection*{---}

This recollection is in one sense the opposite of the previous one. Rather than making an attempt to fulfil the expectations of my acquaintance, I did what he least expected. His presence contributed to the stressfulness of the situation, but in my response I was not primarily addressing him. I believe that the clue to understand the behaviour that led to the choices I made is to look at the negative impact that habit can have on expressive freedom. By forcefully breaking my habitual musical and instrumental responses, I was able to communicate. 

Drawing the distinction between the projective and the objective self has some correspondence in the idea of artistic work, such as improvisation, as either an act of creation or an act of discovery. The dividing line between these two poles is blurred, to say the least. More than perhaps anyone else, Kant has provided us with the image of the genius artist whose powers of creation remain mystical and hidden. For him, the difference between discovery and creation was exemplified by the difference between the sciences and the arts:
\begin{quote}
Thus we can readily learn all that \emph{Newton} has set forth in his immortal work on the   Principles of Natural Philosophy, however great a head was required to discover it; but we   cannot learn to write spirited poetry, however express may be the precepts of the art and   however excellent its models. The reason is that \emph{Newton} could make all his steps, from   the first elements of geometry to his own great and profound discoveries, intuitively plain   and definite as regards consequence, not only to himself but to everyone else. But a   \emph{Homer} or a \emph{Wieland} cannot show how his Ideas, so rich in fance and yet so full of   thought, come together in his head, simply because he does not know and therefore cannot   teach others. \citep[][p. 113]{kant2007}
\end{quote}
According to Kant not even the artists themselves can understand what the nature of creation consists of.\footnote{Were this true, all and any artistic education, not to mention artistic research, would obviously be utterly pointless. Although now on the decline, I believe that some of the resistance against artistic research that we have seen over the years has its roots in Kant's romantic view on the arcane acts of artistic creation. Widely regarded as futile, should it somehow succeed in uncovering some parts of its hidden layers, artistic practice would forever be transfigured.} At best we can reject Kant's description of artistic and scientific creativity as dated, but it is difficult not to see it as a crude pastiche of the processes involved.

Though it is possible to analyze my frustration in the situation described above, in which everything I had learned suddenly lost meaning, I find it difficult to describe it as an act of pure creation. I lacked conscious access to the method that could provide me with a possible solution, and it is plausible to assume that the key came to me in a moment of (unconscious) inspiration. This, however, does not make it into a pure act of creativity; it was much closer to a discovery, albeit a negative one: the discovery of how not to play the saxophone. In reality it is not very difficult to deconstruct the dichotomy of discovery and creation in creative work, be it artistic or scientific. This is what \citet{benson03} successfully does when he introduces context as one of the primary agents, explaining that the important factor that often seems to be neglected is that any artistic activity takes place within, and is related to, a practice, tradition or community. In music, as well as in many other art forms, the practices, or discourses, and traditions are layered and create complex structures. To a certain extent the practices are self-regulative in that they constantly develop, sometimes by leaps and bounds. Furthermore, ``discourses (or practices) have certain texts [\ldots] that are taken to be authoritative'' (p. 42), which acts as a kind of dynamic resistance that allows for commentaries to be made--commentaries that may later form the basis for new texts. The versatility of these structures makes it difficult, perhaps not even meaningful, to attempt to understand an artistic practice such as improvisation as either discovery or creation. As it relates to context and practice, one's own as well as that of one's co-musicians and that of the venue, the tradition and the idiom, there is room for both discovery and creation. Benson turns to Shakespeare and, in the light of the discussion on discovery versus creation, asks about the famous line `To be or not to be...' from Hamlet: ``So what exactly was involved when writing that memorable line?'' (p. 44). Shakespeare certainly did not invent the English language, and it is very likely that someone had, at some time, formulated a similar or even identical phrase before him. What he did, however--and this is what is significant in this discussion--is he ``took that line (whatever its origin) and imbued it with a certain significance by placing it within a particular context'' (\emph{ibid.}). Following this line of thought and in this limited example we can conclude that one of the most famous lines from one of the greatest geniuses of western literature does not fulfil the first requirement of Kant's definition of a genius: To be original in the sense that what is created has not existed before. The comparison may seem unfair but I strongly believe that the requirement for originality \emph{in general} needs to be contested and rethought. 

%%% FÖRSTA HALVAN FINNS MED I SARC.
Closely connected to this discussion on context, creation, discovery and practice is the concept of freedom. That freedom is a difficult topic is no news. Hanna Arendt points to the political domain, which we will return to in the next section. Without freedom, she claims, political life would be meaningless:

% Concerning democracy and freedom of speech the individual responsibility is an important factor.  
\begin{quote}
And even today, whether we know it or not, the question of politics and the fact that man is a being endowed with the gift of action must always be present to our mind when we speak of the problem of freedom; for action and politics, among all the capabilities and potentialities of human life, are the only things of which we could not even conceive without at least assuming that freedom exists, and we can hardly touch a single political issue without, implicitly or explicitly, touching upon an issue of man's liberty.  
%Freedom, moreover, is not only one among the many problems and phenomena of the political realm properly speaking, such as justice, or power, or equality; freedom, which only seldom---in times of crisis orrevolution---becomes the direct aim of political action, is actually the reason that men live together in political organization at all. Without it, political life as such would be meaningless. 
\citep{arendt77}
\end{quote}

%FRÅN SARC
Musical notation and the division of labour into composer and performer are relatively recent inventions in the history of music, and improvisation as an expression of musical freedom is often seen as the exception. Could we perhaps say that it is not so much that improvisation is free, but that music based on preconceived and composed structures is constrained and absent of freedom? In that case, would it not be more appropriate to talk about reinstating freedom in all aspects of musical creation and abandon what are seen by many as a problematic dichotomy between improvised and composed music?\footnote{Which, I should add, I strongly believe is an erroneous model. There is no opposition between composition and improvisation, these are two very different processes and one can effortlessly exist within the other at any time. I suspect the reason there is a persistent desire to keep them in opposition has to do with social and political issues.} In fact, improvisation as such is no guarantee for achieving expressive freedom. In some improvising genres and musical cultures, the freedom of improvisation may be defined by completely different standards, and sometimes improvisers are so strictly tied to a particular aesthetics or style that on the surface freedom may not appear to be a strong agent.\footnote{Some of these expressions could be referred to as idiomatic improvisation, as labeled by Derek \citet{bailey92}.} But even in improvised music that is strongly identified with freedom, its stylistic qualities may be so prominent that the meaning and impact of freedom may be debated. Looking at it from the other side, however, even in music with a strong idiomatic identity, such as bebop, in which performers are musically and socially tied to a defined and, in a sense, limited set of phrases, the organization of the material is still freely decided by the musician. And if we approach the idiom from a slightly wider angle, and on a greater time scale, we can clearly see that there is a huge difference between the stylistic interpretation made by Charlie Parker and that made by Thelonius Monk. Both are exponents for bebop but have approached the idiom freely, with exceptional individuality, and with a greatly varied aesthetics as a consequence.

The claim on jazz musicians to be both strongly individual and free improvisers at the same time quickly becomes problematic, as the first requirement influences or limits the second. To attempt to do both at the same time, one may end up using one's freedom to claim the right to control the situation at the expense of the freedom of the other. In essence this is interaction-as-control, and it is a surprisingly common mode in jazz improvisation. In his book, \emph{The Philosophy of Improvisation}, Gary \citet{peters09} discusses the phenemenon as the ``aporia of freedom''. Freedom is generally thought of as something positive, deliberating and emancipatory, but it is a mistake, according to Peters, to neglect ``freedom's questionable duality'' (p. 21). In an interview by Roger Dean in the book \emph{New Structures in Jazz and Improvised Music}, saxophonist and composer Anthony Braxton gives a remarkable account of his doubts concerning the interpretation that the idea of musical ``freedom that was being perpetrated in the sixties might not have been the healthiest notion'' \citep[][p. 22]{peters09}. Consequently, in his book on the emergence of  AACM\footnote{The Association for the Advancement of Creative Musicians. A still active, nonprofit organization for creative music, collectively run by musicians, that was initiated in 1965 in Chicago, USA but which can now be found in several cities throughout the US.}, of which Braxton became a member in 1966, George Lewis points out that they rarely spoke of their music as `free jazz', `avant-garde' or even `black music', though the association had a lot in common with the other black grassroots organizations that were being formed at the time \citep[p. 98][]{lewis2008}. The balance between the individual and the collective \footnote{Muhal Richard Abrams claimed that AACM was a collection of individuals \citep[][p. 498]{lewis2008}}, between freedom and adaptation to the collective, and between composer and performer have likely contributed to the internal freedom from freedom itself \citep{peters09}. It is again the context that needs to be considered. For the Chicago musicians that started AACM in the 60s, they created a frame within which individuality, as well as freedom and collectiveness, could manifest themselves as agents of creativity. %This h relative freedom. 

In the paper ``Negotiating the Musical Work'' we discussed the notion of subculture as a means to develop and better understand the ideas that arise in our collaboration:

\begin{quotation}
We might try to approach this symbolic system in relation to a common context, or subculture created by the agents involved in it. Both composer and performer are working within the frame of their own cultural contexts which defines their respective understandings of the evolving work. The subculture is a result of interaction, and negotiation ('\emph{What is it we are developing?}', '\emph{How are we talking about it?}', etc.), between the two agents and their inherent cultural contexts. Their mutual expectations and their understanding or imagination of the work in progress is of importance when they attempt at co-coordinating their actions, for instance towards a definition of the performance instructions. 
\end{quotation}

Is it possible to suggest that this subculture may develop a sense of freedom both in relation to the surrounding context and between its members? In this field a symbolic system may emerge that can freely redefine itself in the course of the artistic work, both in its preparation and its execution. In the text cited above we discuss the context of an emerging collaborative composition, and we were observing how our understanding of it was shaped both by the work itself but also by our individual interpretations of what was going on. I.e., in the work process the practice as well as our understanding of the practice was evolving at the same time. Would this be a working definition of freedom? That the concepts are being formed and reshaped in the work process? In this case the effect would be that freedom in improvisation is not the freedom of one musician at the expense of that of another, but rather is something that takes place in the context of the artistic practice and that has to be constantly renegotiated.

%% Ställ i perspektiv till Barthes citatet om origin och destination.
If now we return to the situation in the concert described above: What was the significance of my non-playing of the saxophone that evening? Was I giving myself too much freedom, i.e., did my co-musicians suffer? I think, in a way, I was. What is more, rather than acting out my own frustration, I could have acted more sensibly to the other members of the group. Instead of looking at my dilemma as a personal problem I should have realized that it was a collective one. In that dialogue, provided all members shared a kind of sensibility, we as a group could have become much freer, and in that earned freedom we could have communicated better among ourselves and with our listeners. Meanwhile, when I forgot how to play the saxophone I created a commentary to the musical discourse that we were engaged with in the group. This commentary could have provided us with a recreated `text' which may also have boosted our collective freedom and development.\footnote{The reason was not realized was due to the fact that the group was discontinued soon after this tour, for reasons not specifically related to the events described here.} Although I was not aware of it at the time, the act of forgetting is a common technique to short-circuit the habits of playing. Ornette Coleman wanted to ``create as spontaneously as possible---'without memory,' as he has often been quoted as saying'' \citep[][p. 117]{litzweiler92} and without any 'real' training he started playing the violin and the trumpet. In the process his memory and meta-knowledge about saxophone playing was neutralized, and he felt he could approach a truer expression.\footnote{There are many other examples. Marcel Duchamp talked about forgetting with his hand, DJ Spooky has talked about forgetting with his turntables and it all leads us back to Nietsche's concept of ``active forgetting''.} 

The answer to the question of the significance of the rupture created in the performance above may lie in the way the self and the body interoperates. Creation and discovery alike are activities that rely on the way also our tactile senses function, and in an embodied process the `I' and the `Eye' are inseparable. 
%The pure ideas of the ideal individual in transcendentalism that gave rise to the formation of these concepts, is more readily a subjective and social individual conscious of the inherent complications of the purely visual.
% DEATH OF THE AUTHOR SKA KOMMA HÄR OCH BINDA IHOP COLEMAN, DUCHAMP, MASSUMI OCH LEDA ÖVER TILL I/EYE
In Roland Barthes seminal essay \emph{The Death of the Author} we find many interesting ideas that parallel the concepts of embodiment, de-individualization and habit destruction as ``abrupt disappointment of expectations of meaning''\citep{barthes77}. The importance of individuality in many expressions and the ego-centered view on artistic production mentioned in the beginning of this text are in many respects related to the role and significance of the author brilliantly interrogated by Barthes. Although my discussion has been focused on the role of the improvising musician, Barthes's discussion revolves around the literary author, a role significantly different. In a live improvised performance of music it becomes difficult, if not impossible, to separate the creator from the music as Barthes suggests we do with the creator and the writing.\footnote{The disengagement of the work from the author is a hermeneutic thought brought forward also by Paul \citet{ric91} among others. The author is not the one-way sender of a message the way we often want to see it, because in the act of writing he is removed from the work \citep[See also][]{frisk-ost06-2}. What Barthes instead suggests is to sacrifice the author in order to reinstate the position of the reader.} However, the way in which the artistic 19th century genius has been shaped has created a mythology so powerful that it has had an impact on much of our understanding of \emph{any} artistic figure, authors, composers and musicians alike. The creative act is so strongly soldered to this romantic image that even the understanding of an improvising musician, whose creativity depends not on work creation, but on the real-time impulses in performance, which are very volatile by nature, is informed by this notion. In opposition this romantic view Barthes claims that:

\begin{quote}
Writing is the destruction of every voice, of every point of origin. Writing is that neutral, composite, oblique space where our subject slips away, the negative where all identity is lost, starting with the very identity of the body writing. \citep[p. 142]{barthes77}
\end{quote}

Furthermore, the preoccupation with the author is a consequence of the view of the work as emanating from its creator. Hence, the only relevant way to understand the work is through understanding the author's background, life, and context: ``The \emph{author} still reigns in histories of literature, biographies of writers, interviews, magazines, as in the very consciousness of men of letters anxious to unite their person and their work through diaries and memoires \citep[p. 143]{barthes77}.'' The reading in the larger sense of the word is the process of decoding the message, not in an act of critique but an act based on a reconstruction of the author, recombining the parts that he constitutes and, through this structure, being able to understand the true meaning of the work. As we know, this is in essence the focus of traditional musicology, to reveal the composer bit by bit and understand his work through the history of his life: Where did he live? Who was his maid? What did he eat? Where did he study? Though these questions may well be relevant for the study of our cultural and social history, the extreme focus on the individuality of the composer has had a strong influence on the interpretation and reading of his work at the expense of the position of the listener and that of the performer.

If we transfer Barthes's statement that ``a text's unity lies not in its origin but in its destination'' \citep[][p. 148]{barthes77} to the domain of music, the subjectivity of the performer would not be operative in the act of listening to an improvisation. Its unity is instead in the destination, the listener. Even today, almost a half century after the text was first published, this notion is still provocative, but Jean-Luc \citet{nancy2007} go even further:

\begin{quote}
It is not a hearer, then, who listens, and it matters little whether or not he is musical. Listening is musical when it is music that listens to itself. It returns to itself, it reminds itself of itself, and it feels itself as resonance itself: a relationship to self, stripped of all egoism and all ipseity. Not `itself', or the other, or identity, or difference, but alteration and variation[\ldots] (p. 67)
\end{quote}

%%%FLYTTA?

\section*{Self and other}
\label{sec:improvisation-other}

\textit{In 2005 Stefan Östersjö and I initiated a long and still standing collaboration with two Vietnamese musicians, eventually named \emph{The Six Tones}. The dean of the artistic faculty of Lund University had helped us to get in touch with Nguyen Thanh Thuy and Ngo Tra My, two master musicians whose primary musical interests up until then had been traditional Vietnamese music. They played Dan Tranh, a Vietnamese zither, and Dan Bau, an electrically amplified mono-chord, respectively. Neither Stefan nor I had any previous experience of playing Vietnamese music and Thanh Thuy and Tra My had very little experience playing contemporary Western music. The first time all four of us met in the composition studio at the Malmö Academy of Music to play, Stefan on guitars and I on laptop, we set out to try some sketches of mine. It was a set of loosely structured improvisations and the ambition was that these would provide input for a piece for the quartet that I would compose. One of Stefan's and my primary interests in initiating this project was to understand more about the way improvisation is used in traditional Vietnamese music and explore ways in which we could create a common platform between our respective traditions. In the session, and with these goals in mind, I became incredibly self aware of the dissymmetry between Stefan and I, the two western men, and our Vietnamese female colleagues. Given the history of Vietnam in particular, and the history of the white man in general, I was afraid that simply because of my identity and cultural background, my activities and my ideas would get in the way of Thuy's and My's origin and mask the culture they carried with them. I believe this was a relevant concern, but the problem was that I only had awareness of the inequality and, no knowledge of what to do about it. The purpose of the project would easily have been defeated if we had not found a way to deal with the imbalance. However, the consequence of my misguided concern  was that I became so hesitant to take any kind of initiative that the session almost collapsed. Thuy and My told me in hindsight that, more than anything, they were confused and wondered whether I knew what I wanted at all or what I was after. In fact, that was not the problem--I had a very clear idea about what it was I was after musically--but out of fear for appearing as an authoritarian leader, I lost the ability to express my intentions clearly. My self-consciousness concerning the general idea of the power relations in the group got in the way of my abilities to communicate.}

\subsection*{---}

As we may observe in this incidence, the particularity of the self can get in the way. Simon Emmerson reconsiders Trevor Wishart's ideas on sonic masking \citep{wis96}, applying them on the meeting between two musical traditions. Aspects of one sound from one tradition may mask those of another; or, slightly rephrased, one `self', e.g., the conscious, may mask another, e.g. the unconscious. Emmerson, furthermore, goes on to discuss the different modes of exchange that we may have access to when different musical cultures collide, and the ``particular mix of these may result in a range of outcomes: on the one extreme, appropriation with no exchange or understanding--for example, a composer `plundering local colour for sampling'--through to true exchange with the possibility of real mutual understanding'' \citep{emmerson06}. Even if we were not interested in a merging of the musical traditions we were certainly clear about wanting to avoid appropriation. So was I appropriating Vietnamese music when I applied electronics to the acoustic performance? Or was Stefan doing so when he played the 10-stringed guitar with a slide in order to make it sound like an idiomatic Vietnamese instrument? According to Emmerson the unsuccessful exchange between two or more idioms is one where properties of one hide properties of another:

\begin{quote}
[A] situation where two sounds are played together and one masks the other (or a perceptual aspect of the other) such that it can no longer be perceived. We can generalize this from sound, to performance and even to aesthetic aspects of music. Throw two traditions of music making together and aspects of one may mask aspects of the other (sound subtlety, performance practice tradition and aesthetic intent). This may be inevitable in any intercultural work as there are bound to be incompatibilities. But we must ask--have we masked something ‘significant’ as seen from within the culture? \citep{emmerson06}
\end{quote}

Masking will probably occur to a certain extent in any kind of music, but the question asked by Emmerson at the end of the quote is material: It is not so much \emph{if} something is lost as \emph{what} is lost, and what the importance of the property is. The ambition to avoid masking my new Vietnamese friends resulted in a collapse in which nearly everything was masked.  

Acknowledging or questioning a system of domination or an unequal relationship is not in itself a means to transform it. To move beyond merely describing it and in order to politicize and defy despotism, it is necessary to question the self and assume an altered perspective. In the words of Trinh T Minh-ha: ``It is not sufficient to know the personal but to know--to speak in a different way'' \citep[][p. 164]{trinh91}. 
The social impact of the Eurocentric view of the world should not be underestimated. Stefan and I belong to what Mark \citet{slobin1987} labels ``the superculture'' (p. 31), and the complex political and economic imbalance between East and West plays an important role in our understanding of the other in our multi layered work with traditional Vietnamese music in general, and with The Six Tones in particular. From the outset the idea with the project was to aim to create a music whose identity was neither Vietnamese, nor Swedish or European, but both at the same time, or, preferably, music with its own distinct character. We wanted to avoid the simple superimposition of one tradition on top of the other, instead aiming for the coexistence of the two elements on equal grounds. Slobodin defines three categories of intercultural work:

\begin{enumerate}
\item \emph{Industrial interculture}, which evokes the notion of a commodified system whose main function is to project the first world order, spiced with an unobtrusive element of difference (p. 61)
\item \emph{Diasporic interculture}, which emerges from the subcultural interactions across the borders of nations (p. 64)
\item \emph{Affinity interculture}, which describes a ``global, political, highly musical network'' in which musicians are interacting and communicating through a negotiated musical space (p. 68)
\end{enumerate}

At the time we were not fully aware of the implications of our ambitions, nor had we thought much about the political dimension of our endeavour, but our process was most closely related to the above category of affinity interculture category above. Our network has grown significantly since the start, both in Vietnam and in Europe, and the context for the group is now multidimensional both geographically and stylistically. %On the political and social level, however, I think it is fair to say that we still mave much work to do.

Social activist Gloria Jean Watkins, also known as bell hooks, approaches her own background in racist America in the significant \emph{Marginality as site of resistance} \citep{HooksBell1990}. Describing the railway tracks as the demarcation between her home ground and the centre she identifies marginality as ``the site of radical possibility, a space of resistance'' (p. 341). The first mistake, as is shown by hooks, is to think of marginality as a space one wishes to surrender and give up to instead gravitate towards the centre. Hooks and her friends and relatives would at times trespass into the other domain, to work ``as maids, as janitors, as prostitutes'', and when they did, ``there were laws to ensure our return''. To refuse to give in to the expectation of wanting to relocate from the margin to the centre is to invalidate these laws and dismantle their meaning. Vietnam, being both politically and economically in the periphery, is in every respect marginalized as the other, the foreign, the different and the obscure. And, just as there were laws for bell hooks and her friends to ensure their return when they trespassed, our Vietnamese co-musicians have learned that there are laws to ensure their return to Vietnam from Sweden as well. We may think that the global perspective has broadened our view on the world and blurred the boundaries, but in the eyes of the legislators in the West there is no doubt as to what the centre is, and what the periphery is. Furthermore, though we may think that the regions demarcating the inside and the outside are large continents and political systems such as East versus West, or democracy versus dictatorship, the many uprisings in the suburbs of cities in countries such as Sweden, Great Britain and France show us that the local territories are also disunited and parted.

What bell hooks is referring to in the text cited above is an institutionalized oppression and marginalization that has been going on for centuries and clearly operates on a completely different scale compared to The Six Tones. As Trinh T Minh-ha reminds us, merely reading about it will not let us understand the experiences described. What it does allow us to do, however, is to understand that the effects and the processes in the development within the The Six Tones are similar to those operating on a larger scale. 
%In a two way process the practice contributed to making me aware of the issues, allowing us to continue to highlight the instinctive tendency to treat the other based on the assumption that we hold the best solution. 
It is in this sense that artistic activity also has the potential to engage in a political consideration, reflection and introspection, not in the meaning that the artistic expression itself needs to be politically imbued, but rather that the site for artistic practice and artistic research can be taken advantage of as a site also for politically oriented questions. 

The response to the unevenness in the relatively innocent context of this first rehearsal with The Six Tones was based on the thesis that, rather than stepping back and allowing social interaction to take place, I evaded the uneasiness by action, thereby disallowing change to take place. In his reading of Lacan, Slavoj \citet{zizek2011} points to interpassivity as: ``\emph{I am passive through the Other}'' (p. 26). And further: ``In a group situation in which some tension threatens to explode, the obsessional talks all the time in order to prevent the awkward moment of silence that would compel the participants to openly confront the underlying tension'' (p. 26). Eventually, according to \v{Z}i\v{z}ek, we move from the ``the contemporary redefinition of politics as the art of expert administration as politics without politics, up to today's tolerant liberal multiculturalism as an experience of Other deprived of its Otherness'' (p. 38). Though it is important to remember that once we started playing in the session described above, we relatively quickly approached a working situation in which some of the issues discussed here were resolved, it is equally as important to recognize that political topics also may infringe on artistic practice unless they are properly identified. Furthermore, my main point is that the practice may effectively provide a contingent unfolding of the same topics.


\section*{Discussion}
 listening is the desire for autonomy, although musical practice the dividing line between openness towards the other and the autonomy of the subject is considerably more complicated, as we have seen in the three examples from my own practice presented here.

A lot can be said about these stories, used as point of departure. They are perhaps not very original; many musicians and artists may have had similar experiences. My main concern, however, has been to attempt to understand the ways in which the self can, and will, interact with my creativity and my interaction with the other: the self as \emph{I} or \emph{Eye}, as a vehicle or inhibitor of freedom, and as obstructing social and political power structures. The common denominator in these stories is the way in which the self, and consciousness about the self and other, alters the planned artistic and expressive trajectories, and the focus of this essay is how the impact and meaning of these intersections between the internal and external worlds of artistic expression can be discussed. Looking at the different positions, or perspectives, of the self, the subconscious activities and choices made in performance become accessible and possible to penetrate. Approaching a critical view on freedom reveals its dual nature, and the dynamics and destruction of habit formation allows for an improvisation that may also be non-free. I agree with \citet{griffiths10} that ``understanding the self and its place in research is crucial in the carrying out and presentation of arts-based, practice-based research'' (p. 185). Griffiths also points to how artistic research may become an important counterpart to the way that neo-liberalism valorises the impersonal (though it is always referred to as the \emph{personal} choice). I would like to go even further and argue that through artistic practice, with the help of artistic research, we are able to approach difficult socio-political issues and become aware of the necessary transformations needed within the domains of self and of society.

We are reminded by \citet{deleuze94} that ``self-consciousness in recognition appears as the faculty of the future or the function of the future, the function of the new'' (p. 14-5) and part of my argument here has been that self-consciousness in recognition is also the first instance that will allow the new. In all three recollections my failure to recognize my instant responses as valid made me analyze my behaviour as irrational and flawed. Instead of seeing the self as capable of responding soundly to a given situation, its reactions were seen in relation to the socially or culturally moulded reference image of the performer or composer. This image is ruled  by the awareness of culturally defined roles and may be seen as the faculty of the past, the opposite of self-consciousness in recognition. The improvising musician is in many cases as crammed with artistic codes as is the composer or the writer, and the concept of freedom often associated with improvisation makes it even more complex. To be free and authoritative are central properties of the jazz musician, problematic in themselves and together they may become plain confusing. Both of the first two stories are related to these concepts and possible solutions lie in regarding the group as a dynamic subculture and changing the focus from the origin to the destination. Barthes' claim that writing is not the creation of a voice but its destruction, a space where all identity is lost may seem counterproductive to my case, but the loss of identity is where the self may be found. This is a claim initself exemplified by the third story where identity was the obstruction in the first phase of our inter-cultural project.

The self in artistic practice is constituted by a complex weave of interrelated aspects including, apart from the psychological, the social, cultural, political, aesthetic and philosophical. In a globalized world, not limited to the intercultural context, it is useful to attempt to deconstruct common binaries central to most Western artistic production, and decisive to the way our cultural understanding of the role of creation has been shaped. These include producer--consumer, performer--listener, improvised--composed,\footnote{Though this is not a binary opposition I include it here for the reason that it is so commonly discussed as a dichotomy.} and dual--non-dual. When the impact of these, and similar, concepts are penetrated, the self can be informed by what is now going on in the process rather than by what has traditionally shaped the it, e.g. the self as a composer is defined by the particular needs imposed by the context and not by what is culturally or aesthetically expected of that role. The subsequent reciprocal effect is that a new opening for examining the potential consequence in the political dimension is revealed. The first instance of change, of the future, is to determine what needs to be changed, and artistic practice is a location where this may be aptly established.

% The personal expression is an important agent in many artforms and in jazz it is essential. The personal sound including the actual sound, the phrasing and articulation, the patterns, the melodic style, and the harmonic universe is of paramount importance to the success of the musician. Sonny Stitt always claimed that he came up with his sound and style independently of Charlie Parker but was never fully acknowledged as he appeared as a copy of Bird, less original than the ``true''inventor of BeBop. Parker was more original because he came forth prior to Stitt. Most succesful jazz artist have redefined the possibilities of their instruments in one way or another and achieved an undsiputed level of originality. Coleman Hawkins, Betty Carter, Charlie Parker, Billie Holiday, Carla Bley, Albert Ayler are all musicians whose sound has become their identity. This is often referred to as having a personality and to lack a well molded identity of this kind is often regarded a failure. In my own schooling as a jazz musician, the one thing I remember most clearly is being reminded of the importance of having a personal sound.

% \section*{Method}
% \label{sec:method}

% Integrative Learning comes in many varieties: connecting skills and knowledge from multiple sources and experiences; applying skills and practices in various settings; utilizing diverse and even contradictory points of view; and, understanding issues and positions contextually."


% ``According to Somers and Gibson (1994, 38), social groups often perform such constructions "to consolidate a cohesive self-identity and collective project.'' \citep[p. 103]{lewis-1}

% with these different manifestations of the self and how these can, and will, interact with as well as get in the way for expressing ones identity as a reflection of the self in music in general and improvisation in particular. They form the foundation for the current artistic resaerch project and for my interest in the dynamics of the self in improvisation.
%  The questions I am interested in approaching are: What happens when the self gets in the way? When it is not allowed to work freely? When the conscious self and the subconscious appears to be in conflict?

% The particularity of the self can, in certain cases, get in the way. Simon Emmerson reconsiders Trevor Wishart's ideas on sonic masking applying them on the meeting between two traditions. Aspects of one sound from one tradition may mask those of another or, slightly rephrased, one `self', e.g. the conscious, may mask another, e.g. the unconscious. Dividing up the self in a conscious and an unconscious part may, along with other dichotomies surrounding the self, be questioned. In his non-dualistic thinking the American writer David Henry Thoreau, who anticipated John Cage's ideas of non-intentionality by almost a century, makes the point that it is not until you cease to try to understand that you can truly see.


% and the reason the concept of freedom becomes so problematic Even the expectation to be original limits the % potential to be free. As a free improvisor I should consequently also have the freedom to not be original % and not be individual. And it gets even more problematic and paradoxical if we start thinking about the % freedom to choose to be nonfree. Part of the problem is the concept of freedom itself. Another part of it is % that there is a tendency to see
  
%Furthermore, the occurence of non-improvised music, which is a nececessary counterpart for identifying improvisation as an expression of freedom, is inferior to the tradition of improvised music, in time as well as geographically. It is almost only in the occidental world, in the last three to four hundred years that music has been notated rather than communicated orally. The music of the rest of the world has always had improvisation as a defining element.

% Against this idea speaks some pivotal moments in the development of jazz into what would later be called free jazz. One is an infamous recording of Sonny Rollins from 1964 

% This freedom may for example  

% Furthermore, there is nothing to indicate that freedom 
% %% Detta citat skall sättas i relation till Bateson och The algorithms of the heart.
% \begin{quotation}
%   The real theme of a work is therefore not the subject the words designate,   but the unconscious themes, the involuntary archetypes in which the words,   but also the colors and the sounds, assume their meaning and their life. Art   is a veritable transmutation of substance. By it, substance is spiritualized   and physical surroundings dematerialized in order to refract essence, that   is, the quality of an original world. This treatment of substance is   indissociable from style.\citep[p.47]{deleuze72}
% \end{quotation}

% \begin{quotation}
%   Essence is not only individual, it individualizes. \citep[p.43][]{deleuze72}
% \end{quotation}

% Kafka, Borroughs, Virilio, Focault, Deleuze and many others have all explored the institutions of control and confinement so essential to the 20th and the beginning of the 21st centuries.

% \begin{quotation}
% Here too, from the standpoint of a certain Freudianism, we can discover the   principle of an inverse relation between repetition and consciousness,   repetition and remembering, repetition and recognition (the paradox of the   Repetition and Difference 15 ‘burials’ or buried objects): the less one   remembers, the less one is conscious of remembering one’s past, the more one   repeats it - remember and work through the memory in order not to repeat it.   Self-consciousness in recognition appears as the faculty of the future or   the function of the future, the function of the new.  Is it not true that   the only dead who return are those whom one has buried too quickly and too   deeply, without paying them the necessary respects, and that remorse   testifies less to an excess of memory than to a powerlessness or to a   failure in the working through of a memory? \citep[p. 14-5][]{deleuze94}
% \end{quotation}

% \begin{quotation}
%   When the continuity of affective escape is put into words, it tends to take   on positive connotations. For it is nothing less than the perception of   one's own vitality, one's sense of aliveness, of changeability (often   signified as "freedom"). One's "sense of aliveness" is a continuous,   nonconscious self-perception (unconscious self-reflection). It is the   perception of this self-perception, its naming and making conscious, that   allows affect to be effectively analyzed-as long as a vocabulary can be   found for that which is imperceptible but whose escape from perception   cannot but be perceived, as long as one is alive. \citep{massumi95}
% \end{quotation} 

% Massumi's definition of intensity is another interpretation of the layered nature of perception, somewhat related to Bateson's algorithm's of the heart. He the notion of a difference between different layers of communication and meaning. Massumi investigates the difference between content and effect in an image, where the content is the communicable, the indexing of its content to intersubjective meanings. The effect it has, its impact and duration is what he calls the intensity.

% \begin{quote}
%   Intensity is beside [the conscious-autonomic mix], a nonconscious, never-to-conscious autonomic remainder.   It is outside expectation and adaptation, as disconnected from meaningful sequencing,   from narration, as it is from vital function. It is narratively de-localized, spreading   over the generalized body surface, like a lateral backwash from the functionmeaning   interloops traveling the vertical path between head and heart.
% \end{quote}

% When Ornette fights convention in his playing he does it by short-circuiting the habit involved with his saxophone playing. His sensation of intensity is dampened by the regularity of the expression. His experience that what he does on the saxophone is in line with what Charlie Parker did before him turns his music in to an idiomatic expression that carries a meaning with it. Perhaps we may compare Coleman's situation to the statement that  language ``is not simply in opposition to intensity. It would seem to function differentially in relation to it'' \citep{massumi95} and conclude that what Coleman was after was to increase intensity by reducing the matter-of-factness of his saxophone playing because ``matter--of--factness dampens intensity'' (ibid.). In this case I have compared the factor of language to recognizability in music which may indeed seem dubious and the comparison is not actually intended. However, let us consider Massumi's definition of intensity:

% \begin{quote}
%   Intensity is qualifiable as an emotional state, and that state is
%   static--temporal and narrative noise. It is a state of suspense, potentially
%   of disruption. It's like a temporal sink, a hole in time, as we conceive of
%   it and narrativize it. It is not exactly passivity, because it is filled
%   with motion, vibratory motion, resonation. And it is not yet activity,
%   because the motion is not of the kind that can be directed (if only
%   symbolically) toward practical ends in a world of constituted objects and
%   aims (if only on screen). \citep{massumi95}
% \end{quote}

% If these are properties of intensity it is perhaps fair to assume that a style of playing, and idiom, that has a history and is known to many listeners as well as to the player in question, qualifies as an opposition to intensity and, hence, it will also risk at dampening or reducing intensity.

%Slavoj \citet{zizek2011} draws a line from the Lacanian notion that justice as equality is founded on ``our envy of the Other who has what we do not have, and who enjoys it''. The response to the unevenness in the innocent context of this first rehearsal with the Six Tones was based on the thesis that, though I could not compensate for what I perceived of as a lack of possibilities, I instead reduced my own latitude consistent with what Zizek refers to as equally shared prohibition:

% \begin{quote}
%  \ldots coffee without caffeine, cream without fat, beer without alcohol, [\ldots] warefare without casualties, [\ldots] politics without politics, up to today's tolerant liberal multiculturalism as an experience of Other deprived of its Otherness (the idealized Other who dances fascinating dances and has an ecologically sound holistic approach to reality, while features like wife-beating remain out of sight).
% \end{quote}


%% I SARC
% The beginning of contemporary jazz that exploaded in the 1950s with the Be Bop movement preceeded the related reaction against serialism most prolifically promoted by John Cage has not been acknowledged for its influence. Chance operations and indeterminancy, so effectively denounced by Pierre Boulez et al. in their 1964 article Alea claims the composer ``has chosen henceforth to be meticulous in imprecision'' \citep[p. 44]{boulez64} and rather condescendingly talks about what they perceive as the failure of indeterminacy in composition. There is no doubt that this debate was heated nor that Boulez et al. was absolutely convinced that the best means to compose music was to be meticulous in \emph{precision}. The composer is making the choices and is the speaking subject, to think or attempt to do otherwise is plain wrong:

% \begin{quote} 
% In spite of best intentions and most earnest attempts, I am unable to make out the precise reason for this fear to approach the true problem of composition. Perhaps this phenomenon also is due to a kind of fetishism of numeral selection--a position that is not only ambiguous but completely unsound when the work under investigation structurally refuses these procedures, which are, after all, excessively coarse and elementary.
% \citep[p. 44][]{boulez64}
% \end{quote}

% Cage, the main proponent at the time for indeterminacy, or meticulous imprecision as Boulez would call it, had a corresponding lack of understanding and interest for jazz, in many regards seemingly similar to some of Cage's ideas. George \citet{lewis-1} contexualizes this relation, or lack of relation, in his widely influential paper Improvised Music after 1950: Afrological and Eurological Perspectives brings up Cage's discussion on jazz with the journalist Michael Zwerin\footnote{In the interview Cage is invited to share his thoughts on jazz and agrees to do it while at the same time stating that jazz is not something he thinks much about at all. \citep[In]{lewis-1}} within a sociological context:

% \begin{quote} 
% The colloquy between Cage and Zwerin [\ldots] displays whiteness in its defining   role. Zwerin, though supposedly taking the side of jazz, ends up agreeing with Cage that jazz could   use some work. The work of black artists is defined by whiteness as the primitive (yet improving)   work of children: `But jazz is still young, and still evolving'; jazz could benefit from serious   study of `our' models; already, it has started to explore areas `suggested by Ives'; `jazz is   getting freer' though the use of tone-rows, and `getting away from the time dependence--inferring   it rather than clobbering you with it all the time'; and so on. \citep[p. 104][]{lewis-1} 
% \end{quote}

% These two events, Boulez article and Zwerin's interview took place only a few years apart in the mid 60's and clearly has something to say about the great changes that music was going through at the time. The great European compser is hitting on the great American composer in turn hitting on jazz in general and black musicians in particular. What the New York School of composers did was that they threatened the power of the dominant power of the composer. Cage, while proposing a very open and decentralized attitude towards music and art was until his death in fact a very strong and influential proponent of his own view of the world of music. And, as much as he liked to remove intention from his performances, and regardless of his enigmatic writings on the subject, he was the one setting the boundaries for his compositions (most of them) and he published them as scores just as Boulez would publish his. 

% However, it is not so much these debates themselves that are my interest here, but the position of subjectivity portrayed. To Boulez et al. it is unthinkable that anyone else than the composer can be the subject; not the performer nor the listener, it is the composer who is speaking. The aesthetic turmoil at the time is likely to be one reason for these debates, the social and racist order another, but I find it difficult to not see Boulez' attack on indeterminacy and Cage's patronizing attitude towards jazz as a defensiveness towards a means to organize musical material that could threaten their own respective power positions. Furthermore, the role of the composer, initself a relatively new invention introduced at the time when notation divided the musician in two parts, the originator and the executor \citep{wis96,frisk-ost06-2}, is initself under attack when the improviser rejoins these into one and the same agent. The binary division between compser and musician is deconstructed by the improviser.

%%%%%%%%%%%%%%%%%%%%%%%%%%%%%%%%%%%%%%%%%%%%%%%%%%%%%%%%%%%%%%%%%%%%%%%%%%%%%%%%%%%%%%%%%%%%%%%%%%%%%%%%

% Att lyssna på den andre är centralt i den västerländska och kristna etiken. Att bemöta den % andre med förståelse och respekt oavsett hur denna bemöter en själv. Detta är också en viktig % princip i jazzen och improvisationen. Det är i samspel med den andre som den dynamiska % improvisationen vecklar ut sig. Men, vad är det man lyssnar på när man lyssnar på den andre? % Och på vilket sätt skall man påverkas av den andre och vad ska man låta förbli opåverkat? Är % det bara dem man spelar med som man ska lysssna på eller ska lyssnandet även inkludera % publiken?

% I jazzen, liksom i många konstformer, är det personliga uttrycket viktigt. Att bygga upp en % klangvärld som är typisk för en själv är helt centralt och alla stora jazzmusiker har i någon % mån omdefinierat sitt instrument genom sin särpräglade spelstil; Coleman Hawkins, Betty Carter, % Charlie Parker, Billie Holiday, Carla Bley, Albert Ayler. Deras identitet är deras `sound'. Vi % behöver förvisso inte begränsa oss till jazzen, sedan länge (åtminstone sedan modernismen) är % originalitet (som inte minst Dalhaus har påpekat) ett estetiskt värdeord. Alla stora uppovsmän % och kvinnor uppvisar originalitet och särskiljer sig från mångden i sin egen samtid.

% Samtidigt, som en parallell rörelse, finns sökandet efter det ``rena'' uttrycket, det som % passerar förbi medvetandet, förbi det självmedvetna jaget. Ibland är strävan efter % orignaliteten själva källan till sökandet efter det av medvetnadet obefläckade uttrycket; om % varje individ är unik borde också det genuint personliga uttrycket vara originellt. Ornette % Coleman talar om ett så spontant skapande som möjligt, om en kreativitet utan % minne.\footnote{Se \citet[s.117]{litzweiler92}.} Han talar om hur hans spel innan han nådde % framgångar var mera ärligt än det sedan blev och valde att börja spela trumpet och violin för % att slippa onödig kunskap.\footnote{Intervju med Ornette Coleman i \citet[s.33]{taylor77}} För % övrigt påfallande likt Marcel Duchamps tal om traditionens fängelse och att glömma med handen: % ``I unlearned to draw. The  point was to forget \emph{with my hand}.'' \citep[Duchamp, citerad % i][s.29]{tomkins65}. Ett i mina öron djupt fascinerande citat som inte minst drar undan mattan % för den cartesiska dualismen och separationen av kropp och själ.


% På skivan ``The empty foxhole'' från 1966\footnote{\citet{coleman66}} spelar han tillsammans % med sin tioåriga son Denardo Coleman på trummor och beskriver sin tillfredställelse över att % spela med någon som inte behövde bry sig om kritiker eller konsertarrangörer, utan som kunde % spela och vara fri.\footnote{\citet[Ornette Coleman citerad i][s.121]{litzweiler92}} Detta hör % Ornette för att han lyssnar på Denardo men också för att han har förmågan att lyssna på sig % själv; han kan ju konstatera att sonen besitter en egenskap han själv har förlorat. Det yttre % lyssnandet, att lyssna på den andre, kompletteras av ett inre lyssnande. Och, jämfört med att % lyssna på den andre så är det betydligt svårare att lyssna på sig själv.

% Är det då möjligt att samtidigt manifestera sin identitet, att låta jaget kontrollera % uttrycket, och samtidigt vara fri från (över)jagets inflytande? Finns det en motsättning? John % Cage såg det som sin uppgift att befria musiken från jagets intentioner, vare sig det rörde sig % om hans egen intentionalitet eller musikerns. Och även om han hade lite till övers för jazzen, % och egentligen inte var intresserad av improvisation, ligger det nära till hands att se % kopplingar mellan Colemans metoder och Cages ambitioner. Den fenomenologiska intentionaliteten % kan sägas vara just kopplingen mellan den inre och den yttre världen men Cage ville inte att % vare sig han själv eller de musiker som framförde hans musik skulle avse eller styra flödet: % ``I have nothing to say, and I am saying it''. Alltså, till skillnad från jazzens fokus på % identitet och individuellt uttryck vill Cage uppleva världen och omgivningen så som den är, % inte skapa en kopia av den.

% <första historien>

% Turning back to Duchamp, who introduced the notion of the ``personal
% 'art coefficient' '' , in a discussion concerning the creative
% act. Specifically Duchamp is referring to the immanent processes,
% those in which the artist alone is involved and which ultimately lead
% to ``art in the raw state---\`{a} l'\'{e}tat brut'' and, in short, it
% constitutes the difference between the artistic intention and its
% realization. The gap goes unnoticed by the artist and it represents
% his (or her) inability ``to express fully his
% intention''\footnote{\citet{duchamps57}} Duchamp concludes that ``the personal
% 'art coefficient' is like an arithmetical relation between the
% unexpressed but intended and the unintentionally
% expressed.''\footnote{\citet{duchamps57}} It is primarily the description of
% the art coefficient as a \emph{relation} that is interesting with
% regard to the current discussion.

% Kanske kan man förklara de två polerna - det inre lyssnandet och det yttre lyssnandet som ett % uttryck för en cartesisk dualism. En åtskillnad av det fysiska och det metafysiska, mellan själ % och materia. Den amerikanske författaren David Henry Thoreau, som föregrep John Cages idéer om % ickeintentionalitet med närmare hundra år, förespråkade tvärtom en samexistens mellan naturen % och jaget, mellan de yttre och inre intrycken.\footnote{\citet[s.37]{shultis98}} När han gör en % distinktion mellan att se (``seeing'') och att titta (``looking''), där det senare implicerar % att man också försöker förstå, förespråkar han i grund och botten samma filosofi, eller % estetik, som Cage. Det är först när man upphör att försöka förstå som man verkligen kan % betrakta eller höra.\footnote{\citet{thoreau41}} Även om Thoreau räknas till % transcendentalisterna är han också en objektivist (till skillnad från Emmerson som är en % projektivist) som de-centraliserar jaget och ser verkligheten som något som inte uteslutande % medieras av detta. Bort från det personliga uttrycket och mot det föränderliga jaget som är % mottagligt för yttre influenser. Det genomskinliga, närmast osynliga jaget ställs mot det jag % som projicerar sig själv.

% <läs avsnitt>

% Är det då huvudsakligen uppfattningen av jaget och jagets roll som skiljer Cage från Coleman, % eller är det snarare ett estetiskt ställningstagande? Coleman beklagar att hans instrument, % altsaxofonen, är så starkt förknippad med Charlie Parker och Cage tar avstånd från Beethoven i % synnerhet och harmonik i allmänhet. Eller har de, som Cage nog hade hävdat,  i grunden olika % infallsvinklar?

% Låt oss för tillfället lämna jaget som något som antingen begränsar och skapar eller jaget som % något som betraktar och ``är'' och resonerar med naturen och istället betrakta en mer % Freudianskt uppdelning av jaget. I Freudiansk teori delar man upp mental aktivitet i primära % och sekundära processer. De primära är icke-verbala och drömlika och företar de sekundära som % är det reflekterande och medvetna jagets uttryck. Eller som antropologen Gregory Bateson % uttrycker det: ``Drömmar är metaforer kodade som primära processer''. Och konst generellt är: % ``an exercise in communicating about the species of
% unconsciousness [\ldots] a play behaviour whose function is [\ldots]
% to practice and make more perfect communication of this
% kind.''\footnote{\citet[p. 137]{bateson72}}

% Men det kommer alltid att finnas en motsättning mellan primär och sekundärt kodad information % och inget annat än förvirring kommer bli resultatet när man försöker avkoda det omedvetna och % dokumentera det i det medvetnas språk som alla som har försökt att förklara en dröm i ord har % erfarit:

% \begin{quote}
%   [The] algorithms of the heart, or, as they say, of the unconscious,
%   are, however, coded and organized in a manner totally different from
%   the algorithms of language. And since a great deal of conscious
%   thought is structured in terms of the logics of language, the
%   algorithms of the unconscious are double inaccessible. It is not
%   only that the conscious mind has poor access to this material, but
%   also that when such access is achieved. \emph{e.g.}, in dreams,
%   art, poetry, religion, intoxication, and the like, there is still a
%   formidable problem of translation.\footnote{\citet[p. 139]{bateson72}}
% \end{quote}

% I primära processer är enligt Bateson fokus i diskursen på relationen; relationen mellan jaget % och andra eller relationen mellan jaget och omgivningen. Med andra ord, det omedvetnas % processer kommuniceras i mötet med andra, inte i en översättning utan genom mötet i sin egen % logik. Och, kanske mer än någonting annat är improvisation en synnerligen effektiv metod som % har potential att tillgängligöra de primära processerna. 

% Och, för att återkoppla till mina inledande berättelser så kan man förstå mitt förändrade spel % när personen jag känner kommer in i rummet som en helt naturlig konsekvens av att ogivningen, % och dess förutsättningar förändrades, och att det därför är helt naturligt att mitt spel % förändrades. Det var helt enkelt en kommunikation i primära processer. I den tredje historien % så är det avsaknaden av relation som i kombination med en analytisk (teoretisk) och % reflekterande utgångspunkt som gör att kommunikationsproblemen blir åtskilligt större än våra % språkliga problem.

% Att inte lyssna på den andre behöver inte vara oetiskt i improvisation. Det är tvärtom en % viktig möjlighet och handlar mer om att lysnnandet riktas inåt än att man inte lyssnar alls. % Jaget är inte antingen det passiva lyssnande jaget eller det extroverta, projicerande men för % att bibehålla kontakten med det inre är det nådvändigt att motstå vanan och nödvändigt att % utveckla metoder för att försäkra sig om att det inte är vanan som styr utan lyssnandet. och % därför måste jaget ibland ges upp för att det ska bli möjligt att återfinna det och den % information och kunskap som det bär på.


% \vspace{0.5cm}
% ...
% \vspace{0.5cm}

% Maskin och människa är två, till synes, i grunden inkompatibla enheter. Den ena är oerhört % dynamisk och självanpassande, den andra behöver ständig hjälp och vägledning för att åstadkomma % något överhuvudtaget. Och även om båda dessa beskrivningar kan passa in på båda enheterna så % gör de det sällan inför samma utmaning eller i samma kontext. Man kan tycka att de i så fall % borde utgöra varandras helt naturliga komplement, men så är det inte heller. Problemet, menar % jag, kan sammanfattas i bristande förståelse för hur de två kan interagera med varandra, hur % rummet \emph{dem emellan} skall utformas. Av den anledningen är denna interaktion speciellt % intressant i det sammanhang där vi försöker förstå jagets dynamik. Min interaktion med datorn % kan genuint sakna intention samtidigt som den i någon mening kräver att jag genom den % kontrollerar maskinen

% (Här har utvecklingen gått långsamt: liksom på sextiotalet skriver vi alltsom oftast på ett tangentbord för att kommunicera med datorn för att bara nämna ett exempel.)


% Turning back to Duchamp, who introduced the notion of the ``personal
% 'art coefficient' '' , in a discussion concerning the creative
% act. Specifically Duchamp is referring to the immanent processes,
% those in which the artist alone is involved and which ultimately lead
% to ``art in the raw state---\`{a} l'\'{e}tat brut'' and, in short, it
% constitutes the difference between the artistic intention and its
% realization. The gap goes unnoticed by the artist and it represents
% his (or her) inability ``to express fully his
% intention''\footnote{\citet{duchamps57}} Duchamp concludes that ``the personal
% 'art coefficient' is like an arithmetical relation between the
% unexpressed but intended and the unintentionally
% expressed.''\footnote{\citet{duchamps57}} It is primarily the description of
% the art coefficient as a \emph{relation} that is interesting with
% regard to the current discussion. The roles of the two factors
% (unintended and intended) are likely to be different in free
% improvised music and the 'intention' that Duchamp is speaking of here
% is probably not meaningful in relation to Coleman and other proponents
% of the free jazz movement.\footnote{From within the field of jazz and
%   free form the ``I just play what I hear'' paradigm is common and it
%   is difficult to see how 'intention' in the sense here discussed can
%   be part of the equation. However, this is from within. From an
%   analytical point of view ``I just play what I hear'' is most
%   certainly an intention.}  But the notion of the ``unintentionally
% expressed'' as having a relation to the ``unexpressed but intended''
% is useful when looking at Coleman's unorthodox play of the trumpet and
% the violin and something we may use when we go back to the subject of
% computers and improvisation. Furthermore, the personal art coefficient
% possibly holds an abstract relation to the notion of the structure of
% the unconscious, whose discourse is also focused on the relationships.

\nocite{biggs10}
\bibliography{/run/media/henrikfr/Homer/Home/Documents/svn/admin/conf/biblio/bibliography} \bibliographystyle{abbrvnat}
\end{document}