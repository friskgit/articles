\documentclass[a4paper]{article}
\usepackage[swedish, english]{babel}
\usepackage[T1]{fontenc}
\usepackage[authoryear,round]{natbib}

\makeatletter
\renewcommand\bibsection%
{
  \section*{Referenser
    \@mkboth{\MakeUppercase{\refname}}{\MakeUppercase{\refname}}}
}
\makeatother

\usepackage{graphicx}

\renewcommand{\rmdefault}{pad}

\usepackage{fancyhdr}
\pagestyle{fancy}

\lhead{\small{\textit{Henrik Frisk}}}
\chead{}
\rhead{\small{\textit{Det (o)nödvändiga jaget}}}

\title{The (un)necessary self\\\vspace{.6cm}}
\author{Henrik Frisk, PhD\\{\small Royal College of Music, Stockholm}\\{\small mail@henrikfrisk.com}}
\date{}

\begin{document}
\selectlanguage{english}
\maketitle

\thispagestyle{empty}

%\section{Introduction}

\noindent

%\section{Abstract}


The focus of this presentation revolves around the understanding of the self in improvisation and interaction in the context of an artistic musical practice with and without electronics. The ambition is to explore the meaning and impact of the self in those contexts. Key agents here are improvisation, freedom, resistance and identity. 
%idiomatic resistance, instrumental resistance, psychological resistance, %cultural resistance, and so forth.

Although very difficult to define, freedom in general is a recurring concept in the discussion of improvisation. On the surface improvisation may seem as a means to create music that is free from the chains of the formal structures that notation imposes on the expressive possibilities of the musician, and open it up to the immediate and unmediated influence of the individual. A music that may be created on the spot whose substance is defined, not so much by external factors, as by the will of the improviser(s). Is free improvisation an expression that allows improvisers to be free, or one that allows them to freely express their already defined egos? Improvisation in general is no guarantee for that expressive freedom. Some improvisers are so tied to a particular aesthetics or style that the notion of freedom may appear to be completely lost.

The personal expression is of great importance in many other art forms. In jazz, to develop a sound of your own is critical and many of the great jazz musicians such as Coleman Hawkins, Betty Carter, Charlie Parker, Billie Holiday, Lennie Tristano, Carla Bley, Albert Ayler have in some ways redefined and stretched the limits of their instruments through their highly skilled, individual and original output. Their sound has become their particular identity. Although the search for an individual sound in most cases is a very conscious act there is a corresponding search for the pure, or unconscious, expression, exemplified by Ornette Coleman's attempts to short-circuit the habitual aspects of his saxophone playing, instead picking up the trumpet and the violin. Marcel Duchamp similarly spoke about ridding himself of acquired knowledge: ``I unlearned to draw. The  point was to forget \emph{with my hand}.'' Like Coleman, Duchamp was using a (new) tool (the ruler) to revolt against the tradition and the expression ``forget with my hand'' is significant as it puts the focus on the physicality of the action.

By using my own practice I will attempt to show that the self is a very complex agent in artistic production and one that is not easily defined in terms of free or non-free, strong or weak, or precise or imprecise. Furthermore, as Duchamp and Coleman has pointed out, the connection between the self and the body is an important factor. What is the relation between self and identity? In-performance freedom is in strong contrast to the ideologies of modernism and has caused people like Pierre Boulez to dismiss those who advocate chance operations in music as being ``meticulous in imprecision'' but what is the meaning of precision in the self?

% Furthermore, the occurence of non-improvised music, which is a nececessary counterpart for identifying improvisation as an expression of freedom, is inferior to the tradition of improvised music, in time as well as geographically. It is almost only in the occidental world, in the last three to four hundred years that music has been notated rather than communicated orally. The music of the rest of the world has always had improvisation as a defining element.

% But the particularity of the self can also, in certain cases, get in the way. Simon Emmerson reconsiders Trevor Wishart's ideas on sonic masking applying them on the meeting between two traditions. Aspects of one sound from one tradition may mask those of another or, slightly rephrased, one `self', e.g. the conscious, may mask another, e.g. the unconscious. Dividing up the self in a conscious and an unconscious part may, along with other dichotomies surrounding the self, be questioned. In his non-dualistic thinking the American writer David Henry Thoreau, who anticipated John Cage's ideas of non-intentionality by almost a century, makes the point that it is not until you cease to try to understand that you can truly see.

\end{document}