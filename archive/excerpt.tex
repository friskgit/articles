I will now return to \emph{Archive Fever} where the dualism between experience and cognition on the one hand, and the hypomnesic nature of the archive and its exteriority on the other, is a central topic. Derrida repeatedly comes back to how the external writing of the archive results in an, in every respect, different record than the recording of the perceptual apparatus: ``Because the archive [\ldots] will never be either memory or anamnesis as spontaneous, alive and internal experience. On the contrary: the archive takes place at the place of originary and structural breakdown of said memory.'' \citep[p. 11]{derrida1998} This could be understood from the rather direct and simple point of view that relying on an external source of memory restructures one's own, internal memory. Given the information surplus in the digital era, and the extent to which we commonly rely on auxiliary mnemonic devices this somewhat susceptible relationship, I believe, may be recognized by many. 

Though our technological devices, services and functions at times appear to be working akin to our experience of how our internal memory processing functions, at this level there is a irrevocable the dividing line between how the inside and outside is structured: ``There is no archive without a place of consignation, without a technique of repetition, and without a certain exteriority. No archive without outside.'' \citep[p. 11]{derrida1998} This is something he returns to many times in the text to follow. Perhaps this ``spontaneous, alive and internal experience'' is the non replaceable, non negotiable site of the materiality of artistic practice in music? The aspect that separates it from any representation of the said practice?

As clear and irrevocable the dividing line between inside and outside may be it does not prevent that forces emanating from either side of this influences the other: also this relationship is deconstructed.

The dualism of the two systems, however, the hypomnesic, technical archive and the internal memory and experience does not separate them from influencing each other, at least in abstract ways. 

Through its structure of writing the archive can never be a neutral site for recording, if such an idea ever existed. It acts as a political as well as an economic force upon that which it records.

\begin{quote}
  It is thus the first figure of an archive, because \emph{every} archive, we will draw some inferences from this, is at once \emph{institutive} and \emph{conservative}. Revolutionary and traditional. An \emph{eco-nomic} archive in this double sense: it keeps, it puts in reserve, it saves, but in an unnatural fashion, that is to say in making the law (\emph{nomos}) or in making people respect the law. (p. 7)
\end{quote}




In this statement, however, lies also the clear and irreversible dividing line between inside and outside: ``There is no archive without a place of consignation, without a technique of repetition, and without a certain exteriority. No archive without outside.'' \citep[p. 11]{derrida1998} This is something he returns to many times in the text to follow. Perhaps this ``spontaneous, alive and internal experience'' is the non replaceable, non negotiable site of the materiality of artistic practice in music? The aspect that separates it from any representation of the said practice?
