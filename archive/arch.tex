\hypertarget{the-archive-that-writes-itself}{%
\subsection{The archive that writes
itself}\label{the-archive-that-writes-itself}}

Henrik Frisk

\hypertarget{abstract}{%
\subsubsection{Abstract}\label{abstract}}

In artistic research the question of how to document the artistic
practice that is the focus of the research is important, but largely
unresolved. This paper discusses the various ways in which the
materiality of the artistic practice may be represented in different
forms for documentation. Identified as a twofold processes in which it
is first necessary to identify which kind of data might represent the
materiality of the artistic process and then, to work out methods to
document it, a number of different systems for documenting music is
discussed. The particular case of open form compositions is introduced
as an example of how aesthetic perspectives must influence the recording
of the artistic process. The notion of the personal archive is further
considered as a tool that allows for reflection and introspection, and
leads to a discussion concerning different analogies of writing, and of
perception as a form for writing. Artistic practice is a complex
activity for which there will never be one universal method for
documentation, and a preliminary conclusion is drawn that it is not in
the structurality of the data that the potential lies, but in the way
the different layers generated by the process are interconnected.
Deconstructing the roles of the writer, and the reader and the different
notions of writing, may widen the perspective and prevent the archive
from restructuring and continuously narrowing down what may be seen as
valid data in the materiality of artistic practice.

\hypertarget{key-words}{%
\subsubsection{Key words}\label{key-words}}

Documentation database, artistic research, materiality of artistic
practice, music archive

\hypertarget{introduction}{%
\subsubsection{Introduction}\label{introduction}}

One of the great challenges in artistic research is that the material
basis of the artistic practice may be both data and result. The
challenge amounts to the fact that, in the absence of a direct
representation of the practice in question, the relevance of the
research may be lessened. This is not necessarily a defining property of
\emph{all} artistic research -- there are in fact a number of ways in
which artistic research can be thought of and explored -- but
practice-based artistic research in music, where the practice itself
plays a significant role, brings the issues in the present discussion to
the surface in interesting ways. If the material basis of the artistic
research may be both data and result, there are methodological concerns
that are intimately linked to how the practice is documented. A
documentation that consists merely of a recording of the result of a
long process may not do justice to the material aspect of the practice,
though, obviously, such recordings may still be of great interest to
other analytical articulations. The challenge may be examined critically
both from the point of view of the design of such documentation systems,
and from the point of view of the practice itself. From a preliminary
standpoint, there are two aspects to this. The first is concerned with
deciding what part of a performance or practice is relevant given a
particular research question. The second involves the technology that
makes it possible to extract, store and re-enact the data. As will be
discussed in this paper, these two aspects are not only clearly
dependent, but may also influence each other to a significant degree.

With respect to the opening statement, it may be necessary to unpack the
meaning of the terminology, specifically the meaning and context of
\emph{materiality} and \emph{data}. With `data', I refer to research
data in an abstract sense: information obtained through observation,
experimentation, reading or other means, and which constitute the raw
material for the research. This may or may not be represented as digital
data, and its nature may be very varied. Due to the novelty of the field
of artistic research there is no solid agreement as to what artistic
research data actually consists of, let alone how it should be
documented, and it remains a very important but largely unresolved
question.\footnote{This is quite literally the question currently
  addressed in Sweden. In relation to new national guidelines for open
  access to research data, on behalf of the Swedish government and by
  way of the Swedish Research Council, there is a need for artistic
  university colleges and universities to define what artistic research
  data actually constitutes in the various fields of research carried
  out (Vetenskapsrådet, 2017).} The issue may be divided into several
subtopics but first and foremost it is important to distinguish between
gathering and documenting the various aspects of the material practice
on the one hand, and the documentation of the artistic output on the
other. The latter may carry with it traces of the practice, but may also
efficiently disguise it. Part of the virtuoso tradition in music, for
example, is to hide away the material practice and make the performance
appear effortless.\footnote{The challenge of differentiating data from
  results is by no means unique to artistic research.} Even if the
result may be easily recorded, the processes that led to the result are
commonly equally important to the research process, but not always as
easily documented. How can the integrity of the research data be
preserved in an artistic research project that may contain a number of
different kinds of data, as well as raw material from the artistic
process? How can this research data contribute to recording the
materiality of the practice?

If the meaning of \emph{data} is relatively easy to define, the meaning
of \emph{material} is slightly harder to determine given the term's
philosophical connotations.\footnote{Later in this text, however, we
  will come across Derrida's somewhat different account of the meaning
  of materiality, but the specifics of his usage are not of paramount
  importance for this context. Nevertheless, it is important I believe
  to distinguish between them.} When speaking of the material basis of
the artistic practice, the materiality I refer to is related to a
Marxist notion of historical materialism. Sometimes referred to as
\emph{old materialism}, it claims that it is through material conditions
that history and collective consciousness should be understood, `not theX
other way around, thereby famously reversing Hegelian idealism' (Beetz,
2016, p. 25). According to Marx our knowledge about the conditions of
life should emanate from our experience of it, from the bottom up, so to
speak:

\begin{quote}
The production of ideas, of conceptions, of consciousness, is at first
directly interwoven with the material activity and the material
intercourse of men, the language of real life. Conceiving, thinking, the
mental intercourse of men, appear at this stage as the direct efflux of
their material behaviour. (Marx and Engels, 1970, p. 47)
\end{quote}

A founding principle for artistic research has from the beginning been
the notion that knowledge about artistic practice, and the thinking
around it, emanates primarily from the `material activity' of the
practice. In the performing arts, specifically music, it is the very
condition of being engaged in music artistically which forms the basis
for the understanding of the same. In this sense, the material reality
of the practice of artistic research is closely related to Marx's and
Engel's historical materialism. However, the great interest in \emph{new
materialism} in the last few decades, not least in the field of artistic
research,\footnote{The anthology \emph{Carnal Knowledge: Towards a `New
  Materialism' Through the Arts} (Barrett and Bolt, eds 2013) gives an
  overview of the impact of the `material turn' on the field of artistic
  research.} and the ways in which it is related to the current
discussion, and the theme of this issue, makes it necessary to further
the discussion on materialism here.

To unpack and contextualise the topic I will make a brief detour and
return to my PhD dissertation from 2008 (Frisk, 2008). In an attempt to
deconstruct the relations between user, data and interface --
specifically the nature of a possible musical relation between a user
and a digital system mediated by an interface -- I argued for the need
to rethink the human--computer interaction in general, and musical
interaction with computers in particular. The method I used was based on
my artistic practice and it was through my practice that an expanded
view on technology interaction was revealed. For me, the role of the
technology was not that of a passive agent which I could (learn to)
control. Instead, through my artistic practice modes of interaction were
made available in a process of blurring the boundary between myself and
the technology I was using. A necessary condition for this blurring to
take place, however, is that the user accepts the computer as something
beyond the structure of the digital, and beyond a paradigm of control. A
musician interacting with a technology that signifies only itself, will
by definition have to adapt to it. Such an interactive paradigm, built
as it is upon control, is rather distinct from the model I proposed,
based on difference, interaction-as-difference:

{[}In{]} an interactive system in music that deploys some notion of
cybernetics, the computer must not be seen as an isolated `player' but
as part of a whole that includes the performer(s) with which the system
is interacting. Furthermore, it will not be possible for the performer
to gain control over the computer, or the computer over the performer,
without the system failing or regressing. (Frisk, 2008, p. 98)

Theorizing technology as an entity with agency, rather than an inanimate
and passive player that acts only as an ouput from an input, may
contribute to a dismantling of the hierarchy between man and machine. As
such it has some striking similarities with new materialism. In my
thesis these ideas, however, were not developed in an analysis of the
structure of the actants involved in the network, to allude to Bruno
Latour's actor--network theory (Latour, 2007), but through the attempt
to understand it from the point of the practice itself. Hence, whereas
new materialism in general, and vibrant materialism (Bennett, 2009) in
particular, can be seen as a method (Apter et al., 2016, p. 23), for me
the artistic practice was the method. That difference is of particular
interest in this context and my main motivation for leaning towards an
understanding of materialism aligned with `old materialism'.

If, then, the material aspect of music making is important in artistic
research, how can this materiality be represented in the documentation
of the practice? Even with a narrowed down definition of materiality it
is not self-evident what part of an artistic practice has significance
as material. Furthermore, most kinds of musical representations, be it
recordings or scores, carry with them some impression of materiality. To
complicate matters even more, the experience of `playing' is not limited
to professional music making but expands into a broad field of musical
practices that have in common a material relationship to its making.
This is beautifully summarised in Roland Barthes's essay `Musica
practica' (Barthes, 1971) when he draws a line between the music one
plays and the music one listens to.\footnote{For a more developed
  discussion on Barthes and the musical amateur, see Frisk (2016)} { }
The experience of playing music, however, extends into the sphere of
listening, and when listening to a performance of a piece one has once
played, the materiality of the practice may be re-evoked. As a musician,
I know that this sensation is not limited to pieces once played; many
different kinds of music I listen to may conjure up a tangible feeling
of having a material relation to the music heard, even evoking muscular
reactions as if I was in fact playing. In other words, the result of an
artistic process such as a concert or a recording, may in some cases
carry with it a sense of materiality, but, commonly in such cases it is
the listeners material involvement that is represented rather than that
of the artistic practice experienced. The challenge, and the overarching
question for this paper, remains one of identifying which kind of data
might represent the materiality of the artistic process on the one hand,
and work out methods to document it, on the other.

\hypertarget{documenting-music}{%
\subsubsection{Documenting music}\label{documenting-music}}

At the risk of generalising, the artistic research process may be said
to operate along two equally important and complementary axes. One is
the actual artistic process and the result, and the other is the
collection and preservation of data and the documentation of the result.
The knowledge and the means for archiving them, however, are still
insufficiently explored. There have been attempts to come up with
general solutions. One of the more notable ones is the Research
Catalogue (RC),\footnote{https://www.researchcatalogue.net/} an online
archive for artistic research and the home for a number of online
journals for artistic research such as Journal of Artistic Research
(JAR)\footnote{http://www.jar-online.net/} and the Finnish
Ruukko.\footnote{http://ruukku-journal.fi/en} RC is an online based
computer implementation of the notion of a blank sheet: an attempt to
create a transparent platform for sharing artistic research. It is
described as follows on its website:

The Research Catalogue (RC) is a searchable database for archiving
artistic research. RC content is not peer reviewed, nor is it highly
controlled for quality, being checked only for appropriateness. As a
result, the RC is highly inclusive. The open source status of the RC is
essential to its nature and serves its function as a connective and
transitional layer between academic discourse and artistic practice,
thereby constituting a discursive field for artistic research. (Research
Catalogue, 2017)

RC introduces the notion of the `exposition' as a concept denoting the
way artists choose to display, or layout, their work in the database as
research. It is a two-way process in which the artistic practice is
first documented and then exposed. The documentation may be everything
from a recording of a performance to a documentation of some isolated
part of the artistic process. RC gives no recommendations or instruction
as to \emph{how} or \emph{what} should be documented, neither on how to
represent the documentation. The exposition is at the centre and it
should guide the documentation, not the other way around. RC is not a
database in the traditional sense, but rather a format for presenting
and sharing research. Through studying the expositions an understanding
of the materiality of the underlying practice may surface, but only if
the documentation reveals it.

Large scale attempts to create systems for archiving digital data
include the EU and UNESCO sponsored CASPAR project -- Cultural, Artistic
and Scientific knowledge for Preservation, Access and Retrieval
(Douglas, 2007; Roeder, 2006; Bachimont et al., 2003; Cuervo, 2011). The
scope of CASPAR exceeds by far that of the Research Catalogue, so a
comparison would be both unfair and uninformative. But apart from UNESCO
and world heritage sites, driving partners in CASPAR were both GRM
(Groupe de Recherches Musicales) and IRCAM (Institut de Recherche et
Coordination Acoustique/Musique), and one of the sub-goals of the
project was to establish means for preserving electro-acoustic music.
Among the challenges with documenting and archiving technology-dependent
art is that technology, in many cases, grows out of fashion long before
the art work itself. It is not uncommon, for example, for a piece of
music to employ commercial hardware such as synthesizers or computers,
or software, for which support from the manufacturers is discontinued as
models become outdated. Given the speed of the development of new
technology, in some cases this may occur in less than a decade. One
example is the French composer Tristan Murail who has composed works for
the Yamaha DX7 synthesizer. Although it still exists, the instrument is
not maintained, nor are the formats for programming it general. In this
case the materiality of the practice is tied to a technology without
which the practice is undermined. Preserving the technology makes
possible future developments of the practice and contributes to the
understanding of its uses.

It was the attempt of the Integra project,\footnote{http://integra.io/portfolio/integra-project/}
another pan-European project financed by the EU, to approach these
challenges by documenting the abstract technology behind the works
rather than the technological solutions themselves. Abstract
definitions, such as an algorithm, can be implemented again and again as
technology develops. However, this is an extremely time-consuming
process if employed on existing musical works. Nevertheless, a basis for
a structure for documenting artistic works was created in Integra,
originally with the intention of documenting the different elements in
the Integra system (such as computer programs, contact information for
musicians and composers and institutions with access to information).
Preserving abstract definitions of technologies may allow for greater
insight into the material practice of such technology, and may approach
a method for documenting practice.

It is possible to distinguish between a number of types of musical
archives, all with slightly different impact, belonging to different
contexts and with different purposes. First, the musical score is in
itself an archive. It is a systematised way to organise a material that
makes it possible to recreate what was originally conceived at a later
point in time. It specifies an ideal version as well as the initial
authority for all future interpretations. By way of its architecture the
score makes available the contents of the piece and leans on a
structural hierarchy through which the individual elements are to be
understood. The economy of this archive defines what needs to be
registered and what is best left to its interpreters. The dividing line
between what is stipulated by the score and what is left open is
obviously in constant flux. Given the nature of a written musical score
in print, however, the potential for openness is still quite limited in
the sense that the content, the symbols themselves and their order, is
rarely under negotiation, only the interpretation of them. Hence, the
traditional printed score is a preservation archive that structures what
kind of music is made possible by it, even though it may require
considerable input from the interpreter.\footnote{Highly polyrhythmic or
  polymetric music is only with great difficulty notated in traditional
  notation. It is only really pitch-based music that is well suited to
  traditional notation and traditional symbols.} It creates a system of
values that defines the usability of the music as a commodity similar
to, say, the CASPAR project above.

The benefit of looking at the script, the musical score, as itself being
an archive -- normally conceived of as an entry in an archive -- is the
way in which it makes the inscription, the grammar and the practice of
deciphering the information, stand out as important features of the
archive. The system of organisation and the writing of new entries are
the fundamental building blocks of any archive. In a digital era that
registers almost everything we do in our daily life it is easy to forget
about these organizational systems (inherent to all archives), and the
way they trace our actions. For the same reason, it is important to
think about how to best organise archival systems that contribute to
research in music.

An even stricter and much more detailed archive for music is the digital
recording. For each second of recorded music it makes roughly 100,000
entries or more; digital samples encoded into a file of great temporal
resolution. The digital, as pointed out by Aden Evens (2005, p. 79), is
sterile and by itself unproductive until it goes through a
transformation `that draws a line of contact between the digital and the
human' (p. 79). In the end, it is technology that mediates between the
digital and the actual as the digital is not useful by and of itself. To
some extent inspired by Baudrillard (to whom we will return) among
others, Evens sees the digital as pure representation, in contrast to
the actual in a way which sheds some light on the later discussion of
this paper:

Trapped in the abstract, the pure digital operates at a remove from the
vicissitudes of concrete, material existence, and this distance lends it
its qualities of perfection: repeatability, measurability,
transportability, etc. But the digital's divorce from the actual is also
a constraint, denying it and direct power.'' (Evens, p. 79)

The great advantage of regarding the digital, in an abstract sense, as
an archive is also, however, its biggest inexpediency. Without an
efficient interface whatever is represented digitally is of limited use.
For this reason, the power of the interface is great. The digital
recording is a preservation archive relatively agnostic as to what kind
of music it can represent, but with a high degree of specification on a
detailed level. It successfully records an entire performance with great
resolution, but at the expense of making all events structurally equal.
Digital audio is merely bits of data of equal length, incomprehensible
to the human eye. Hence, the process of recording music digitally
archives according to the power of digital structures, removing the
dynamic properties of the performance and fixing it within its own
economy of organization. Should the elements be restructured ever so
little, the recorded sounds would be rendered unrecognizable. The
question concerning the digital is clearly of relevance to this
discussion: can that which is digitally encoded be anything other than
digital?\footnote{This was a central question in my thesis, as discussed
  above.}

In a recent project exploring means for the preservation of electro
acoustic music, the importance of including the artistic process of
creation, along with the artistic output, is supported:

Our approach focuses on the knowledge involved during the creative
process, which involves but it not limited to technology. Preservation
of electro acoustic and mixed music requires a suitable framework for
archiving composer's idiosyncratic musical software and documenting the
work throughout its creative process. It involves archival policies for
digital assets that the creative process produces, together with
relevant knowledge to ensure meaningful usability {[}\ldots{}{]}
therefore enabling re-production of the work. (Boutard and Guastavino,
2012, p. 750)

In this article Boutard and Guastavino also give a comprehensive
overview on some of the most important projects in the field in the last
few decades, including some of the projects mentioned below. For some of
these, as well as for Boutard and Guastavino, the re-production of the
works archived is part of the main purpose. This goal most likely guides
the documentation strategies, and may have some impact on decisions made
concerning the methods for both documentation and preservation.

Media databases for storing and preserving musical works generally need
to be highly structured in order to be useful and in the context of
music they can be seen as augmentations to the musical score. In such
archives the score is stored along with additional information about the
composition and bibliographic data about the creator and the performers
etcetera. In other words, they allow for an extension of the score,
commonly as a set of meta-data. However, in the discussion concerning
the materiality of the artistic practice, it is important to remember
that documenting \emph{the artist} is not necessarily the solution.

The kind of huge music listening services provided online such as
Spotify, Naxos Music Library and Google Play Music, are simultaneous
developments, reductions and augmentations, not of the score but of the
standard commercial recording formats. They are developments because
they introduce a social network layer that connects different kinds of
musics together in ways that are only meaningful for large collections
of music. The networking layer is similar to a notion of unstructured
meta-data as it makes the data more easily accessible. On the other
hand, these services commonly reduce the amount of information about the
recordings contained in them (compared to what is often available on CD
covers). Hence, they are reductions of the traditional recording.
Finally, to an even higher degree than musical scores, they explore the
value of music as commodity, and due to their scope and popularity they
can be seen as economic and political augmentations of the CD-recording.

Common to all of these examples of musical archives, perhaps with the
exception of the Research Catalogue, is that they preserve the present
state or, rather, the state at the time the data was registered. They
are not generally built for recording change over time, such as, for
example, the process of the development from the definition of a musical
work to its first performance. Since time is an essential part of any
attempt to truthfully describe a process, including all descriptions of
musical practice and the processes behind it, this appears a rather big
limitation. They are built on a traditional notion of the archive as a
system for preservation. At the risk of falling back on a semiotic view
on the musical work, they assume the esthesic perspective. Hence, the
current question concerning the representation of the materiality is not
yet addressed. How, then, can an archive that includes a notion of time
be developed, which connects and successfully records how artistic
processes and musical practice unfold and develop in time?

\hypertarget{documenting-openness}{%
\subsubsection{Documenting openness}\label{documenting-openness}}

One of the aesthetic tendencies in the 1960s which has an impact on the
current discussion was the introduction of chance and openness in
musical works. In his famous book \emph{Opera aperta,} Umberto Eco (Eco,
1968) found new tendencies in works by modernist masters Luciano Berio
and Pierre Boulez among others. These works introduced the idea that
part of the construction of the work was left for the interpreter to do,
even in some rare cases in active collaboration with the audience. The
more radical version of the open work, seen in the work by Henri
Pousseur, is labelled by Eco as a \emph{work-in-movement}: `It invites
us to identify inside the category of ``open'' works a further, more
restricted classification of works which can be defined as ``works in
movement'', because they characteristically consist of unplanned or
physically incomplete structural units.' (Eco, 1968, p. 22) A
work-in-movement is a latent, or prospective, possibility rather than a
fact, yet to be realised, whose authenticity lies not in the intentions
of the composer but rather in the collaboration between the different
agents involved in its creation. (Eco, 1968; Frisk, 2008)

Departing from Eco's reasoning in \emph{Opera aperta}, Swedish guitarist
Stefan Östersjö and I developed an artistic method that rested firmly on
the idea of the work as a continuously developing field of
possibilities, in our collaboration on my composition \emph{Repetition
repeats all other repetitions}. Originally conceived of as a fairly
traditional contemporary piece for instrument and electronics, it
developed into what may be called a work-in-movement. The identity of
this work is located in change rather than fixity. In short, the
composition invites interpreters to create their own version of the work
out of an assembly of segments that could be combined in a number of
different manners (Frisk, Coessens and Östersjö, 2014). In a few
articles published early in the process we discuss how our view on the
work developed in the process (Frisk and Östersjö, 2006a; Frisk and
Östersjö, 2006b).

The development of \emph{Repetition repeats all other repetitions}
coincided with the development of the documentation database for the
Integra project mentioned above. It allowed for rethinking both what an
appropriate score could look like for a work-in-movement, but also how
the process of creating and developing such a piece could be documented.
Though the piece originally had a fairly detailed musical score there is
a large number of additional data which are of great importance for the
reading. Had these segments of data only consisted of written
instructions in musical notation the challenge of creating this
particular work's documentation may have been slightly easier. However,
for a work-in-movement to work as such the documentation needs to
contain not only all previous versions and their modes of construction,
but also all the different parts in terms of electronic sounds (sound
files, software for interaction, DSP processes for altering the acoustic
sounds, etc.). Once the concept of documenting all past performances and
all related data first surfaced, the way forward, it appeared, was in
finding methods to defy the propagating level of noise as the archive
grows bigger.\footnote{Noise, of course, is also the result when `all'
  music becomes available in enormous online databases.}

Collecting and organizing the material of \emph{Repetition repeats all
other repetitions} over time hinted at a possible solution to the
question of documenting the materiality of my artistic practice. Many
questions still remain however. What is useful to document and what is
not? When does the collection produce fixity rather than change, if
change is what is desired? How is time represented in the flat structure
of the archive? Despite many concerns with regard to the archival
process, the idea of a documentation database for the piece appeared as
a sensible solution and may in the end provide a valid example of how a
documentation of an artistic work sensible to the materiality of the
process may be organised.\footnote{The database was developed from the
  work done in the Integra project and a technical description of its
  structure is given in ArtDoc -- an experimental archive and a tool for
  artistic research (Frisk, 2017).}

\hypertarget{the-personal-archive}{%
\subsubsection{The personal archive}\label{the-personal-archive}}

The question of \emph{what} to document is obviously entangled with the
question of \emph{how} to do it. The attempt to document musical
practice almost exclusively also involves a transformation to the
digital realm at some point -- both in the ways that the actual process
is encoded, and in the way it is archived and meta-data is applied. In
that sense, today, the digital is almost ubiquitous. This was not the
case for Walter Benjamin, however, who, in the short essay `Excavation
and memory' from 1932 with metaphorical references to archaeology and
psychology writes:

Language has unmistakably made plain that memory is not an instrument
for exploring the past, but rather a medium. It is the medium of that
which is experienced, just as the earth is the medium in which ancient
cities lie buried. He who seeks to approach his own buried past must
conduct himself like a man digging. Above all, he must not be afraid to
return again and again to the same matter; to scatter it as one scatters
earth, to turn it over as one turns over soil. For the `smatter itself'
is no more than the strata which yield their long-sought secrets only to
the most meticulous investigation. (Benjamin, 2005, p. 576)

It is through repetition and conscientious analysis of findings that the
layers of memory and the subconscious (the psychoanalytical dimension of
Benjamin's text is impossible to ignore) are possible to excavate
properly, but the very essence of the findings lies not in the thing
found, but in the `soil' where it was first found. When one's memories
are probed again and again one is eventually able to understand the
findings anew; the secrets are decoded and become clear, as images cut
off from all earlier association. To Benjamin, these are the `treasures
in the sober rooms of our later insights' (p. 576). The way time is
brought into the discussion is interesting. At first, we may or may not
understand what we experience or remember, or even the meaning of what
we find in our excursions, but through a complex process of dissociation
and re-association time will distil the findings and separate the
meaningless from the usefully consequential; the real treasures.

But the findings themselves are not enough, the enhanced awareness that
only experience and knowledge can give us are also necessary, and the
continuous and cautious `probing of the spade' is equally important in
the process of uncovering. `And the man who merely makes an inventory of
his findings, while failing to establish the exact location of where in
today's ground the ancient treasures have been stored up, cheats himself
of his richest prize.' (Benjamin, 2005, p. 576) In other words, the
dynamic between what has been found and the location of this finding is
of significance. The ground I open, which unravels my findings, is part
of the information that can make me understand what I have found. In
that sense, as a consequence, the person doing the finding also plays an
important role in the process: `genuine memory must {[}\ldots{}{]} yield
an image of the person who remembers.' (p. 576)

This short text by Benjamin gives an insight into his own obsessive
archiving of his practice.\footnote{Much of the following is based on
  the documentation in Marx et al., (2007).} Benjamin's personal archive
is particularly interesting in the way that it appears to document not
only what he worked on, and materials related to books or essays that he
was writing, but also what he did in general. Not only lists of books
that he read for a particular project, but all books he read or had ever
read. Not only correspondence with individuals he worked with, but all
kinds of correspondence, including lists of expressions that his son
invented and similar personal details. The lists, written on small
archive cards or sheets of paper, were continuously edited, with entries
scored out and other notes added. He meticulously and apparently
tirelessly archived virtually everything that he came across,
constructing an archive that is (at best) difficult to decipher, but is
also ahead of its time conceptually. It is through this project, his own
archive, that his ideas on memory and excavation must be understood. But
here lies also the danger: `As far as the collector concerned, his
collection is never complete; for let him discover just a single piece
missing, and everything he's collected remains a patchwork' (Benjamin,
1999, p. 211). No collection can be complete and if at any point in
time, any item is missing from the archive it will appear inadequate.

Commonly, an archive of musical material archives representations of
musical content, and to be meaningful it has to go beyond a mere
collection of resources. An archived score is relatively easy to
represent accurately but is a poor representation of the actual music. A
recording of a performance is an accurate representation of the sound
but a poor representation of the material performance. Furthermore,
within the category of musical recordings, there is a continuum between
the concert documentation and the edited recording. Though both the
score and the recording are reductions of the materiality of the
performance to an archivable documentation/representation format, the
concert recording may be seen as the more authentic representation of a
performance than the edited recording. Connected in both time and space,
the score along with a recording can be a phenomenal catalyst for
musical materiality, while at the same time it structures one's
perception of the music. There is a similarity between Benjamin's
collecting and my experiments with documenting \emph{Repetition repeats
all other repetitions}. The relentless storing of unrelated data is one
common aspect, but more important is the way the collection allows for
the possibility of returning again and again. After all, the desire to
allow for continuous reflection, along with the belief that the act of
reflection is informative, is one of the motivations behind the ambition
to document the material practice. How, then, may such a large archive
of musical content be structured so that we are not cheated of the
richest prize, as Benjamin puts it? How may we create systems for
collecting musical knowledge and experience, that can be explored and
excavated for new meanings? To further investigate this, it is necessary
to consider in what ways musical experience may be `written'.

\hypertarget{the-scene-of-writing}{%
\subsubsection{The scene of writing}\label{the-scene-of-writing}}

In \emph{Archive fever: a freudian impression} (\emph{Le Mal
d'Archive}), a late text on the nature of the archive written in the
beginning of the internet era, the French philosopher Jacques Derrida
points (hardly surprisingly) to the dangers of the structures of the
archival process. The book is based on a lecture given in London during
the international colloquium `Memory: the question of archives' in 1994.
This was in the early days of electronic communication, but the text is
strikingly clairvoyant of what we have come to learn about the impact of
digital technology today. Much of the content departs from an earlier
but related text: `Freud and the scene of writing' (Derrida, 1978) and
before pursuing the discussion concerning the nature of the archive I
will go over some of the central arguments that Derrida makes in this
earlier text with regard to Freud's concept of perception and memory as
these may prove to be useful here. Derrida addresses Freud's rethinking
of writing as a means for accessing the possible relations between
perception, memory and experience; for understanding how the outside
world is registered within the psyche. The following is not intended as
an analysis of this rather complex text, but as an attempt to unpack
some of the central concepts used in \emph{Archive fever} for the
purpose of shedding light on the discussion concerning the documentation
of musical practices.

From the short text `A note upon the ``Mystic Writing Pad''' (Included
in Freud, 1997, p. 207-12) Derrida, construes three different analogies
concerning the relation between writing and perception, each with
increased levels of introspection and contrast. The metaphor of writing
is here to be understood as simultaneously appropriating `the problems
of the psychic apparatus in its structure and that of the psychic text
in its fabric' (Derrida, 1978, p. 259). Considering recent developments
in the fields of cognition and ecological psychology the following
discussion should be regarded as a philosophical elaboration on the
metaphor of writing and, as such, how it impacts on the notion of the
archive.

The first analogy concerns the conception of writing as a material
expression of something stored in one's memory. It is a means to jot
something down on a piece of paper, forget about it and later revitalise
the memory by means of reading the writing. Ideally this kind of writing
should satisfy both the need for unlimited capacity and indefinite
preservation: rather similar to the obviously impossible demands we
place on current archives for research data for example. It is writing
as externalization of memory, a representation of memory outside of the
psychic apparatus and separated from it by perception.

The Mystic Pad is a small device that Freud describes as `a slab of dark
brown resin or wax with a paper edging; over the slab is laid a thin
transparent sheet, the top end of which is firmly secured to the slab
while its bottom end rests upon it without being fixed to it' (Freud,
1997, p. 209). The method of writing on the pad is similar to scratching
in wax or clay with the addition that the surface is easily wiped clean
again and made ready for new impressions. According to Freud the Mystic
Pad can be seen as a model of the perceptual apparatus: it records but
leaves no visible traces and can be said to be infinitely ready to
receive anew. The writing on the pad is an illustration of the second
analogy of writing and corresponds to the writing on the body, the first
imprint on reception, but before it is registered by consciousness. This
writing `supplements perception before perception even appears to
itself' (Derrida, 1978, p. 282).

If the two first analogies have been related to the space of writing,
the third is to a significant degree concerned with the time of writing.
Time is here understood as the consequence of the distribution of
symbols and impressions, and as the result of the various strata in the
psyche. Time as in memories being forgotten, re-remembered and
re-inscribed, or, in the words of Freud, as `cathectic innervations':
`as though the unconscious stretches out feelers, through the medium of
the system Pcpt.-Cs., towards the external world and hastily withdraws
them as soon as they have sampled the excitations coming from it'
(Freud, 1997, p. 211-2). This discontinuous functioning of the
perceptual system, according to Freud, `lies at the bottom of the origin
of the concept of time' (p. 212).

Derrida's reading of Freud is multifaceted, and the brief discussion
here only touches upon a fragment of its prospect. As a deconstruction,
however, `Freud and the scene of writing' would be incomplete if it did
not also critically examine Freud's conclusions, and one of the issues
brought forward does have some impact on the topic of materiality.
Questioning the relation between spontaneous memory, and the pure
absence of spontaneity observed in the machine, Derrida claims the
following:

{[}Freud does not{]} examine the possibility of this machine, which, in
the world, has at least begun to \emph{resemble} memory and increasingly
resembles it more closely. Its resemblance to memory is closer than that
of the innocent Mystic Pad: the latter is no doubt infinitely more
complex than slate or paper, less archaic than palimpsest; but, compared
to other machines for storing archives, it is a child's toy. (Derrida,
1978, pp. 286-7)

The archive is a representation of the machine referred to by Derrida
and the question of what the relation between an external writing and
the internal experience may be is important for the current discussion.
So is the question of the relation between the different analogies of
writing, and the notion of a memory machine that may resemble memory: in
the development of this technology for storing the question is not so
much what is saved, but what may be lost in the process.

This rather long overview is included here to provide a frame for the
central discussion of this paper: how can the material aspect of music
making, of the artistic practice in music, be understood, and what is
the relation between this practice and any effort to provide an archive
for it? Considering the three analogies of writing outlined above, could
these be translated to the realm of music? Are they meaningful in
relation to the nature of playing music (`playing' should here be
understood in the widest sense of the word)? Before engaging in the
attempt to re-contextualise them, it is important to identify that the
metaphor of writing, though similar to a specific practice like musical
composition, in many respects has fundamentally different properties
compared to playing music in general, and these should be given due
consideration. These differences partly amount to the organization of
the underlying structure of the two respective practices, but also to
how \emph{writing} easily misleads us to think about the production of a
text that stands by itself, whereas the physical traces of music can
mainly be heard, and thus experienced.

If we consider the analogy of writing as an externalisation of a memory,
the first thing that comes to mind is composition, which can be said to
be a system for remembering music by means of inscribing notes on paper.
For Freud such notation would be a `materialized portion of my mnemic
apparatus' (Freud, 1997, p. 207), but it would be a reduction of what we
think of as music to limit its exteriorisation to musical notation.
Instead, the first analogy of writing in the context of music should
include all kinds of musical activities that emanate from the mind or
memory, and results in a physical trace of some sorts: singing in the
shower, improvising, performing an interpreted composition, composing a
score, composing electronic sounds, etcetera. This would require us to
rethink what writing means: what is referred to here is a sort of
writing onto the world. Reconnecting to Benjamin's modes of reflection
over collected material, memories probed over and over again, perhaps
one may imagine a system that evokes a feedback loop over the three
analogies of writing that at least conceptually could be meaningful in
the current discussion.

Secondly, the analogy of a writing that leaves no permanent traces may
be seen to correspond to perception as a system separated from memory.
Thinking of memory as a physical storage unit (which of course it is
not) it is easy to imagine that it can run out of space, that we reach a
point where we cannot remember any more. In contrast, our perceptual
apparatus, unless struck by illness, has almost unlimited capacity. We
can listen and listen. In fact we cannot \emph{not} listen. But not only
the ears perceive music, the entire body is written when it is struck by
the vibrations from sound waves. The writing on the body, on perception,
however, is elusive and disappears immediately which is why it is also
always ready to be written anew.

Finally, in relation to the notion of time, the third and most profound
analogy of writing is the one most easily translated to music and
playing. Derrida's claim that `time is the economy of a system of
writing' (Derrida, 1978, p. 284) is effortlessly transformed to a
musical context. Beyond its immediate understanding the analogy is based
on the way Freud imagines that there is a movement from within the
psyche that reaches out towards the external world and the discontinuity
involved in this process is what introduce time. There are a number of
interpretations of this that would make sense in material musical
practice. The continuous act of intonation performed by a musician, the
subtle adjustments made in performance against a memorised piece of
music or a tonal system, the temporal adjustment to a pulse, or the
improvisers sensibility to the co-musician's activities are all examples
of `cathectic innervations' in the musical practice.

Even if some of these analogies of musical writing also apply to musical
listening, or even to listening in general, I would claim that they are
all important aspects of the materiality of artistic practice in music.
One's writing of music -- the writing of music onto one's body and the
(subconscious) reflective, discontinuous impulses reaching out into to
external world -- are all integral parts of a broad view of musical
practices. The question now is not if any one of these modes or
analogies may be archived without any central aspect of them being lost,
because the immediate answer to that question is of course `no'. It is
yet unimaginable to store experiences independently of the one
experiencing, though being able to do this is the theme for many science
fiction stories. The more accurate question to pose would be along the
following lines: what is the relationship between the live experience of
a musical practice described in terms of analogies of writing, and
possible representations and documentations of these activities in
various forms for archives? Another, related question, that will be
brought up in the next section is: in what ways may our different
systems of representation (archives) affect our internal systems of
writing?

\hypertarget{writing-the-past}{%
\subsubsection{Writing the past}\label{writing-the-past}}

I will now return to \emph{Archive fever} where the dualism between
experience and cognition on the one hand, and the hypomnesic nature of
the archive and its exteriority on the other, is a central topic.
Derrida repeatedly comes back to how the external writing of the archive
results in a different (in every respect) record than that of the
recording of the perceptual apparatus:

Because the archive {[}\ldots{}{]} will never be either memory or
anamnesis as spontaneous, alive and internal experience. On the
contrary: the archive takes place at the place of originary and
structural breakdown of said memory. (Derrida, 1998, p. 11)

This could be understood from the rather direct and simple point of view
that relying on an external source of memory restructures one's own,
internal memory. Given the information surplus in the digital era, and
the extent to which we commonly rely on auxiliary memory devices this
somewhat susceptible relationship, I believe, may be recognised by many.
Though many of the available technological devices, services and
functions that surround us at times appear to be working akin to how
internal memory processing functions, conceptually there is an
irrevocable dividing line between how the inside and outside is
structured: `There is no archive without a place of consignation,
without a technique of repetition, and without a certain exteriority. No
archive without outside.' (Derrida, 1998, p. 11) Derrida returns to this
dialectic many times in the text and, as was pointed out above, it was
alluded to already in `Freud and the scene of writing'. Perhaps the
uniqueness of the materiality of artistic practice in music may allude
to the antithesis of the archive given by Derrida: `anamnesis as
spontaneous, alive and internal'? The quality that separates it from any
representation of the said practice? Is then the very attempt at
documenting a material practice futile?

The dualism of the two systems -- the technically oriented, external
archive, and the internal memory and human experience -- does not
restrain them from influencing each other, at least not in abstract ways
which was briefly discussed in the first section of this paper. As a
consequence, also this seemingly definite relationship is deconstructed.
Because as clear and permanent the separation between inside and outside
may be, the extrinsic archive nevertheless projects its characteristics
onto its users. The structurality of the writing of the archive
disqualifies it from being a neutral site for recording, if such an idea
ever existed. It acts as a political as well as an economic force upon
that which it records:

It is thus the first figure of an archive, because \emph{every} archive,
we will draw some inferences from this, is at once \emph{institutive}
and \emph{conservative}. Revolutionary and traditional. An
\emph{eco-nomic} archive in this double sense: it keeps, it puts in
reserve, it saves, but in an unnatural fashion, that is to say in making
the law (\emph{nomos}) or in making people respect the law. (Derrida,
1998, p. 7)

More specifically, the archive does not only record but also conditions
what may be written through its control over the structurality of the
writing:

{[}T{]}he technical structure of the archiving archive also determines
the structure of the achievable content even in its very coming into
existence and in its relationship to the future. The archivization
produces as much as it records the event. (Derrida, 1998, p. 17)

Hence, even the \emph{wish} to archive and to make content accessible in
a structured format creates delimitation and determined articulations
that exclude as much as they make available. Derrida then goes on to
make the observation that if technology is an archivisation process that
produces as much as it records, this

means that in the past, psychoanalysis would not have been what it was
(no more than so many other things) if E-mail, for example, had existed.
And in the future it will no longer be what Freud and so many
psychoanalysts have anticipated, from the moment E-mail, for example,
became possible. (Derrida, 1998, p. 17)

One may want to argue against this considering what we now know about
electronic communication, and the example of Freud obviously has a
particular meaning considering the psychological dimension with a
certain focus on the suppressed and unconscious. But on the other hand,
it is safe to assert that the impact of communication technologies on
all aspects of society and culture has been anything but insignificant.
Hence, not only does the archive to some extent determine what may be
archived, it also influences what is written and in what ways it may be
written. For the current discussion, an obviously important
consideration is how the impact of all different kinds of archiving
systems employed may affect both artistic and research practices.

Finally, one other important aspect of the present-day archive fever
needs to be considered. In the relentless wish to reposit potential
experiences in easy to handle, downloadable packages, we miss out not
only on the physical artefacts themselves -- the origin of the
experience, so to speak -- but also may the focus on the past rather
than the future, risk at becoming the driving force. The archive
persistently points to history, institutive, yes, but also conservative:
`And the word and the notion of the archive seem at first, admittedly,
to point toward the past, to refer to the signs of consigned memory, to
recall faithfulness to tradition.' (Derrida, 1998, p. 33) While this may
be self-evident, and to some extent the very purpose of many of the
archiving initiatives of today, such as some of the preservation
initiatives mentioned earlier, in the section `Documentation databases',
is to designate the past and preserve a tradition that may otherwise be
lost. The principle argument for doing this is not that the contents of
the archive itself should allow for development, but that the
availability of it triggers activities that points ahead and bridges the
gap between past, present and future.\footnote{Later in the text Derrida
  points out how the archive also satisfies the future, as a `question
  of the future, {[}\ldots{}{]}, the question of a response, of a
  promise and of a responsibility for tomorrow' (p. 36) but does so only
  in an enigmatic sense.} Part of Derrida's argument, however, is that
the relations between the inner and the outer modes of writing may not
be as yielding as they may first appear. Unreflectively relying on the
archive may alter the view on what should actually be archived and how
it should be structured. As was discussed above, this archival activity
may under certain circumstances also modify how the process to be
archived develops over time. Hence, the ambition to document the
materiality of the practice may change the practice itself: instead of
gaining insights about the practice, the practice is adopted to fit the
models of documentation.

The danger of the way the metaphor of writing has been used here is its
tendency towards an individual act of writing, as in \emph{one writer}
writing \emph{a} \emph{narrative}. The different articulations of
writing discussed, however, hopefully counteracts this modernist
inclination and instead points to the multiplicity of possible modes of
writing. This is given some support by Derrida who, concerning the
Mystic Pad, states that `the subject of writing is a \emph{system} of
relations of strata' (Derrida, 1978, p. 285).\footnote{\emph{ArtDoc},
  the documentation database developed out of the process with
  \emph{Repetition repeats all other repetitions} mentioned earlier in
  the text, is built on the idea of a system of relations of strata.}

\hypertarget{discussion}{%
\subsubsection{Discussion}\label{discussion}}

Following Derrida, it is impossible to be in control of the archive. Its
logic is so substantially different from human experience, and it is
only through a thorough understanding of the difference, of the lack of
concept of the archive, that it is possible to bridge the consequences
of the differences. Is it even desirable to be in control of the archive
or of the archival process? Understanding that the structure of the
archive participates in writing the content makes it possible to design
systems in ways that reinforces the intended results rather than
disguises them. Knowing that the archive will always have the tendency
to point to the past rather than the future can be used both to avoid
this to influence the results and the data, but also in order to design
systems that are prone to change over time. The risks, however, are
still that we mistake the representation for the real. A few years after
Derrida's \emph{Archive fever}, Jean Baudrillard, in the essay `The
automatic writing of the world', takes a typically dystopic point of
view on the relationship between what he calls the real and the double,
or the real and the representation. He gives some support to the notion
of the incompatibility between the inner and external, the real and the
virtual, anamnesis and hypomnesis:

The perfect crime is that of an unconditional realization of the world
by the actualization of all data, the transformation of all our acts and
all events into pure information: in short, the final solution, the
resolution of the world ahead of time by the cloning of reality and the
extermination of the real by its double. (Baudrillard, 1996, p. 25)

The idea that everything is achievable, everything may be documented,
and everything may be verified has today become second nature to such a
degree that it is easy to see the representation as the real. Even if an
experienced listener can re-engage many of the material properties lost
in a recording of a concert, the representation only encompasses a
subset of the information available to the listener of the performance.
The transformation of music from something experienced in a live
performance to something which may be contained, packaged and
transmitted has clearly been fuelled by the commercial powers of the
music industry. It has also participated, I would claim, in the
development of the concept of the musical work. If the work is defined
by its achievable traces -- its score and its recording -- it becomes
self-contained and is much more easily distributed and legally protected
(Attali, 1985).

The view of the (digital) technology of the archive as an engulfing
force which determines not only what is written but also what is read
may however be critiqued. As remarked at the beginning of this paper, it
is possible to give up the wish to control the technology and thereby
allowing for a different approach to the question. Metaphorically
speaking one should allow the archive to structure itself and in an act
of reading, allow it to continuously restructure itself. It is in the
wish to make the archive resemble the real that the archive that writes
itself really becomes a problem. To further develop these thoughts, the
flat ontology of new materialism may actually become useful. Moving from
recording the minute interactions in a process of musical composition,
by way of documenting several instances of musical works, and on to
connecting different art forms may yield new, and important, insights.
Or, as put by Giuliana Bruno:

After all, we should consider that art, architecture, fashion, design,
film, and the body all share a deep engagement with the world of objects
and their superficial matters, including such things as the materials of
the canvas, the wall, and the screen. If materiality defines an art
practice it can also act as a connective thread between separate art
forms, creating a productive exchange. We cannot disregard the ways in
which contemporary artists are engaged in this connective mode of
investigating material practice, incorporating different material
formations in a productive dialogue, on the surface tension of media.
(Apter et al., 2016, p. 15)

The thinking about archiving musical materiality within the framework of
\emph{Repetition repeats all other repetitions} gave rise to the
possibility of rethinking the concept of the musical work. It is perhaps
obvious, but nevertheless necessary to point out here, that the concept
of the musical work also will shape the idea of what materiality in
music consists of. If the work is defined by its continuously changing
artistic, social and political interactions through the work's
interrelations between other works, and its own contents, the chances
are that its dynamic properties should be understood as essential. An
open work definition and a multidimensional system for storing the data
of the work was one of the aims that grew out of \emph{Repetition
repeats all other repetitions}.

Although the discussions held by Benjamin on the one hand, and Derrida
on the other, are very different in scope and focus, they also converge
at interesting points. Derrida's various analogies of writing, of
perception as a form for writing, brings to mind Benjamin's poetic claim
that true memory must `yield an image of the person who remembers'
(Benjamin, 2005, p. 576). On the surface, however, the archive as an
externalised memory machine obviously does not allow for the person who
remembers unless the machine allows its memory device to be continuously
reconfigured in which case its role as archive becomes futile. This
reconfiguration is however possible to mimic without the integrity of
the data being lost, if the focus is moved from the structurality of the
data to the relations between data entries, between the strata of data.
A preliminary conclusion in the ongoing investigation of the main
question about the possible representations of materiality of the
artistic practice in music, the idea of strata of interconnected pieces
of data appears promising. It would allow for representations of time
and for asymmetrical objects of data to be interconnected. It would also
allow for continuous reconfigurations of such objects without the
integrity of the data itself being lost. As such it does not privilege
the writer over the reader of the archive. And here, the question
concerning what constitutes the actual data may find an experimental and
likewise preliminary answer. Benjamin reminds us of the importance to
keep in mind the person \emph{reading} and not only the person writing.
The deconstruction of the roles of the writer, and the reader and the
different notions of writing, though impossible to record and archive
properly, may at least widen the perspective and prevent the (digital)
archive from restructuring and continuously narrowing down what we see
as valid data in the materiality of artistic practice.

The idea put forth by Benjamin, that memory is the medium rather than
the instrument for exploring the past, perhaps allows us to look at the
archive, not as a storage container that can be read from and written
into, but instead as a medium that allows for a subject to read and
write, creating a multi-layered system of interconnected relations.
Surely, this will not be without a significant effort, and with a
resulting depth that may be difficult to disentangle. But as Benjamin
concludes, only to the most meticulous digger will the matter itself,
the `richest prize' (Benjamin, 2005, p. 576), reveal itself.

\hypertarget{references}{%
\subsubsection{References}\label{references}}

\begin{quote}
Research Catalogue 2017. About the Research Catalogue {[}online{]}.
Available at:
\textless{}\href{https://www.researchcatalogue.net/portal/about}{{https://www.researchcatalogue.net/portal/about}}\textgreater{}
{[}Accessed May 15, 2017{]}

Apter, E., et al. 2016. A questionnaire on materialisms. In:
\emph{October} -, pp. 3-110.

Attali, J., 1985. \emph{Noise: the political economy of music}.
Trans.~by B. Massumi. Vol.~16. Theory and History of Literature.
Minneapolis: University of Minnesota Press.

Bachimont, B. et al., 2003. Preserving interactive digital music: a
report on the MUSTICA research initiative. \emph{Proceedings. Third
international conference on web delivering of music.web delivering of
music}. WEDELMUSIC.

Barrett, E. and Bolt, B. eds., 2013. \emph{Carnal knowledge: towards a
'New Materialism' through the arts}. London: I. B. Tauris.

Barthes, R., 1971. Musica practica. In: \emph{Music, image, text.}
Trans.~~by S. Heath. London: Fontana Press pp. 149-54.

Baudrillard, J., 1996. \emph{The perfect crime}. Trans.~by Chris Turner.
London: Verso.

Beetz, J., 2016. \emph{Materiality and subject in marxism,
(post-)structuralism, and material semiotics}. London: Palgrave
Macmillan.

Benjamin, W., 2005. \emph{Selected writings, Volume 2, 1927--1934}. M.W.
Jennings et al., eds. Cambridge, Mass.: Belknap Press of Harvard
University Press.

Benjamin, W., 1999. \emph{The arcades project.} R. Tiedemann, ed.
Cambridge, Mass.: Belknap Press of Harvard University Press.

Bennett, J., 2009. \emph{Vibrant matter: a political ecology of things}.
A John Hope Franklin Center Book. Duke University Press.

Boutard, G. and Guastavino, C., 2012. Archiving electroacoustic and
mixed music: significant knowledge involved in the creative process of
works with spatialisation. \emph{Journal of Documentation} 68.6, pp.
749-771.

Cuervo, A., P., 2011. \emph{Preserving the electroacoustic music legacy:
a case study of the Sal-Mar construction at the University of Illinois}.
\emph{Notes}. 68.1, pp. 33-47.

Derrida, J., 1978. Freud and the scene of writing. In: \emph{Writing and
difference.} Chicago: Routledge \& Kegan Paul Ltd.

Derrida, J. 1998. \emph{Archive fever: a freudian impression}. Religion
and Postmodernism. Chicago: University of Chicago Press.

Douglas, J., 2007. General study 03 final report : preserving
interactive digital music --- the MUSTICA initiative. \emph{InterPARES 2
project}. Available
at:\href{http://www.interpares.org/ip2/ip2_general_studies.cfm?study=26}{{http://www.interpares.org/ip2/ip2\_general\_studies.cfm?study=26}}
{[}Accessed 30 June 2017{]}.

Eco, U., 1968. \emph{Opera aperta}. (English translation, published in
1989). Hutchinson Radius.

Evens, A., 2005. \emph{Sound ideas: music, machines, and experience.}
Vol.~27. Theory out of bounds. Minneapolis: University of Minnesota
Press.

Freud, S., 1997. \emph{General psychological theory: papers on
metapsychology}. Vol.~6. Collected papers. New York: Simon and Schuster,
Touchstone.

Frisk, H., 2008. \emph{Improvisation, computers, and interaction:
rethinking human-computer interaction through music}. PhD thesis. Malmö
Faculty of Fine and Performing Arts, Lund University.

Frisk, H., 2016. Hell is full of musical amateurs, but so is heaven.
\emph{Seismograf}. Available at:
http://seismograf.org/fokus/composer-performer. {[}Accessed May 30,
2017{]}

Frisk, H 2017. ArtDoc - an experimental archive and a tool for artistic
research. In: CMMR, \emph{13th International symposium on computer music
multidisciplinary research}. Porto, Portugal, Sep. 2017.

Frisk, H., Coessens, C., and Östersjö, S., 2014. Repetition, resonance
and discernment. In: \emph{Artistic experimentation in eusic}. Ed. by D.
Crispin and B Gilmore. Gent: Orpheus Institute Series.

Frisk, H. and Östersjö, S., 2006a. Negotiating the musical work. an
empirical study. In: ICMA, \emph{Proceedings of the International
Computer Music Conference 2006}. San Francisco, Calif.: Computer Music
Assoc., pp. 242-9.

Frisk, H. and Östersjö, S., 2006b. Negotiating the musical work. an
empirical study on the inter-relation between composition,
interpretation and performance. In: EMS -06 \emph{Proceedings:
Terminology and Translation. Electroacoustic Music Studies.} Available
at:
\textless{}{\url{http://www.ems-network.org/spip.php?article245}\textgreater{}}
{[}Accessed May 30, 2017{]}.

Latour, B., 2007. \emph{Reassembling the social: an introduction to
actor-network-theory}. Oxford: OUP.

Marx, K. and Engels, F., 1970. \emph{The german ideology}. New York:
International Publishers.

Marx, U. et al., 2007. \emph{Walter Benjamin's archive: images, texts,
signs}. Minneapolis: Verso.

Vetenskapsrådet, 2017. \emph{Nationella riktlinjer för öppen tillgång
till vetenskaplig information}. Available
at:\textless{}{\url{https://www.vr.se/omvetenskapsradet/regeringsuppdrag/avrapporterade2015/avrapporterade2015/nationellariktlinjerforoppentillgangtillvetenskapliginformation.4.7e727b6e141e9ed702b1307e.html}\textgreater{}}.
{[}Accessed August 1, 2017{]}.

Roeder, J. 2006. Authenticity of digital music : key insights from
interviews in the MUSTICA project. . \emph{InterPARES 2 project}.
Available
at:\textless{}\href{http://www.interpares.org/display_file.cfm?doc=ip2_gs03_authenticity_roeder_v2.pdf}{{http://www.interpares.org/display\_file.cfm?doc=ip2\_gs03\_authenticity\_roeder\_v2.pdf}}\textgreater{}
{[}Accessed 30 June 2017{]}.
\end{quote}
