% Created 2016-10-24 Mon 16:35
\documentclass[11pt,a4paper]{article}
\usepackage[english]{babel}

%% Disable when not using html output
%\usepackage{tex4ht}
%\usepackage{pxfonts}
%%%%%%%%%%%%%%%%%%%%%%%%%%%%%%%%%%%%

\usepackage[T1]{fontenc}
\usepackage[utf8]{inputenc}
\usepackage{enumitem}
\usepackage{csquotes}
\usepackage{ebgaramond}
\usepackage{helvet}
\usepackage{titlesec}
\usepackage{titling}
\usepackage{setspace}

\usepackage[hyphens]{url}
\usepackage[hidelinks]{hyperref}
\hypersetup{breaklinks=true}
\urlstyle{same}

\onehalfspacing
\usepackage[
  margin=3.5cm,
  includefoot,
  footskip=30pt,
]{geometry}
% \usepackage[]{fullpage}
\renewcommand{\encodingdefault}{T1}

%% Enable for graphics 
%\usepackage[pdftex]{graphicx}
%%%%%%%%%%%%%%%%%%%%%%%%%%%%%%%%%%%%

\titleformat{\section}
  {\normalfont\sffamily\Large\bfseries}
  {\thesection}{1em}{}

\usepackage{setspace}
\usepackage[style=authoryear-ibid,natbib=true,backend=biber,hyperref=false]{biblatex}
\bibliography{/Users/henrik_frisk/Dropbox/Documents/articles/biblio/bibliography}

\pretitle{\begin{center}\sffamily\huge\bfseries}
\preauthor{\begin{center}
            \large\sffamily \lineskip 0.5em%
            \begin{tabular}[t]{c}}
              \predate{\begin{center}\sffamily\large}
                
\author{Henrik Frisk}
\title{The archive that writes itself}

\begin{document}

\maketitle

\section*{Introduction}
\label{sec:introduction}
\noindent
One of the great challenges in artistic research is that the material basis of the artistic practice may be both data and result, but without an active presence of some kind of representation of the practice the relevance of the research may be lessened. This is not necessarily a defining property of \emph{all} artistic research--there are in fact a number of ways in which artistic research can be thought of and explored--but practice based artistic research in music where the practice itself plays a significant role makes the issues in the present discussion surface in interesting ways. If the material basis of the artistic research may be both data and result, there are methodological concerns that are intimately linked to how the practice is documented.
A documentation that consists merely of a recording of the result of a long process, may not make justice to the material aspect of the practice, though, obviously, such recordings may still be of great interest to other analytical articulations. 
The challenge may be critically examined from both the point of view of the design of such documentation systems, and from the point of view of the practice itself. From a preliminary standpoint there are two aspects to this: The first is concerned with deciding what part of a performance or practice is relevant to document given a particular research question. The second involves the technology that makes it possible to extract, store and reenact the data. As will be discussed in this paper, these two aspects are not only clearly dependent, but may also influence each other to a significant degree.

Going back to the opening statement, it may be necessary to unpack the meaning of the terminology, specifically the meaning and context of \emph{materiality} and \emph{data}. With data I refer to research data in an abstract sense: that information obtained through observation, experimentation, reading or other means, and that constitute the raw material for the research. This may, or may not, constitute digital data, and it may, or may not, be representable as such, and its nature may be very volatile.
Due to the novelty of the field of artistic research there is no solid agreement as to what artistic research data actually consists of, let alone how it should be documented, and it remains a very important but largely unresolved question.\footnote{This is quite literally the question currently addressed in Sweden. In relation to new national guidelines for open access to research data, on behalf of the Swedish government and by way of the Swedish Research Council there is a need for artistic university colleges and universities to define what research data actually constitutes in the various fields of research carried out \citep{vr2017}.} The issue may be divided into several subtopics but first and foremost it is important to distinguish between how to gather and document the various aspects of the material practice on the one hand, and the documentation of the artistic output on the other.
The latter may carry with it traces of the practice, but may also efficiently disguise it. Part of the virtuoso tradition in music, for example, is to hide away the material practice and make the performance appear effortless.\footnote{The challenge to discriminate between data and result is by no means unique to artistic research.}
Even if the result may be easily recorded, the processes that led to the result is commonly equally important to the research process, but not always as easily documented. How can the integrity of the research data be preserved in an artistic research project that may contain a number of different kinds of data, as well as raw material from the artistic process?
How can this research data contribute to recording the materiality of the practice?

If the meaning of \emph{data} is relatively easy to define, the meaning of \emph{material} is slightly more complicated to sort out given how this term is loaded with philosophical connotations.\footnote{Later in this text, however, we will come across Derrida's somewhat different account of the meaning of \emph{materiality}, but the specifics of his usage of it is not of paramount importance for this context. Nevertheless, it is important I believe to distinguish between them.}
% In this short text at least two slightly different understandings will be encountered.
When speaking of the material basis of the artistic practice, the materiality I refer to is related to a Marxist notion of historical materialism. Sometimes referred to as \emph{old materialism}, it claims that it is through material conditions that history and collective consciousness should be understood, ``not the other way around, thereby famously reversing Hegelian idealism.'' \citep[p. 25]{beetz2016} According to Marx our knowledge about the conditions of life should emanate from our experiences thereof, from the bottom up, so to speak:
\begin{quote}
  The production of ideas, of conceptions, of consciousness, is at first directly interwoven with the material activity and the material intercourse of men, the language of real life. Conceiving, thinking, the mental intercourse of men, appear at this stage as the direct efflux of their material behaviour. \citep[p. 47]{marx1970}
\end{quote}
%A similar idea, that the material activity of the artist pertains has been
A founding principle for artistic research has ever since the beginning been the notion that knowledge about the artistic practice, and the thinking about it, emanates primarily from the ``material activity'' of the practice. In the performing arts, specifically music, it is the very condition of being engaged in music artistically that forms the basis for the understanding of the same. In this sense the material reality of the practice in artistic research is closely related to Marx and Engel's historical materialism. However, the great interest in \emph{new materialism} in the last few decades, not least in the field of artistic research,\footnote{The anthology \emph{Carnal Knowledge: Towards a 'New Materialism' Through the Arts} \citep{barrett2013} gives an overview of the impact of the 'material turn' on the field of artistic research.} and the ways in which it is related to the current discussion, and the theme of this issue, makes it necessary to further the discussion on materialism here.

To unpack and contextualize the topic I will make a detour and return to my PhD dissertation from 2008 \citep{frisk08phd}. In an attempt to deconstruct the relations between user, data and interface - specifically the nature of a possible musical relation between a user and a digital system mediated by an interface - I argued for the need to rethink human-computer interaction in general, and musical interaction with computers in particular. The method I used was based on my artistic practice and it was through my practice that an expanded view on technology interaction was revealed. For me, the role of the technology was not that of a passive agent that I could learn to control. Instead, through my artistic practice modes of interaction were made available in a process of blending between myself and the technology I was using.
A necessary condition for this blending to take place, however, is that the user accepts the computer as something beyond the structure of the digital, and beyond a paradigm of control. A musician interacting with a technology that signifies only itself, will by definition have to adopt to it. Such an interactive paradigm, built as it is upon control, is rather distinct from the model I suggested, based on difference, interaction-as-difference:
\begin{quote}
  [In] an interactive system in music that deploys some notion of cybernetics, the computer must not be seen as an isolated ‘player’ but as part of a whole that includes the performer(s) with which the system is interacting. Furthermore it will not be possible for the performer to gain control over the computer, or the computer over the performer, without the system failing or regressing. \citep[p. 98]{frisk08phd}
\end{quote}

%% In other words, in the attempt to control the machine and making it resemble its maker the user is 

Theorizing technology as an entity with agency, rather than an inanimate and passive player that acts only as a result to an input, may contribute to a dismantled hierarchy between man and machine. As such it has some striking similarities with new materialism. In my thesis these ideas, however, were not developed in an analysis of the structure of the actants involved in the network, to allude to Bruno Latour's Actor-Network theory \citep{latour2007}, but through the attempt to understand it from the point of the practice itself. Hence, whereas new materialism in general, and vibrant materialism \citep[][]{bennett2009} in particular, can be seen as a method \citep[p. 23]{joselit2016}, for me the artistic practice was the method. That difference is of particular interest in this context and my main motivation for leaning towards an understanding of materialism aligned with ``old materialsm''.

If then, the material aspect of music making is important in artistic research, how can this materiality be represented in the documentation of the practice? Even with a narrowed down definition of materiality it is not self evident what part of an artistic practice has significance as material. Furthermore, most kinds of musical representations, be it recordings or scores, bear with them some impression of materiality.
%Hence, musical materiality may be seen as a continuum, ranging from, say, being physically engaged in a performance of music on one side, to silently reading a score and imagining the music on the other. To make a distinction between one kind of material practice and another may not be possible, and in this context, nor is it fruitful. 
To complicate matters even more, the experience of ``playing'' is not limited to professional music making but expands into a broad field of musical practices that have in common a material relationship to its making. This is beautifully summarized in Roland \textcite{barthesMus}'s essay \emph{Musica Practica} when he draws a line between the music one plays and the music one listens to.\footnote{For a more developed discussion on Barthes and the musical amateur, see \citep{frisk2016b}} The experience of playing music, however, extends into the sphere of listening and, when listening to a performance of a piece one have once played, the materiality of the practice may be re-evoked. As a musician I know that this sensation is not limited to pieces once played; many different kinds of music I listen to may conjure up a tangible feeling of having a material relation to the music heard, even evoking muscular reactions as if I was in fact playing. In other words, the result of an artistic process such as a concert or a recording, may in some cases carry with it a sense of materiality, but, commonly in such cases it is the listeners material involvement that is represented rather than that of the artistic practice experienced.
The challenge, and the overarching question for this paper, remains to identify what kind of data that may represent the materiality of the artistic process on the one hand, and work out methods to document it, on the other.

\section*{Documenting music}
\label{sec:docum-datab}

Highly generalized, the artistic research process may be said to operate along two equally important and complementary axes. One is the actual artistic process and the result, and the other is the collection and preservation of data and the documentation of the result. The knowledge and the means for archiving them, however, are still insufficiently explored. There have been attempts to come up with general solutions. One of the more notable ones is the Research Catalogue,\footnote{https://www.researchcatalogue.net/} an online archive for artistic research and the home for a number of online journals for artistic research such as Journal of Artistic Research (JAR) \footnote{http://www.jar-online.net/} and the Finnish Ruukko.\footnote{http://ruukku-journal.fi/en} Research Catalogue is an online based computer implementation of the notion of a blank sheet: an attempt to create a transparent platform for sharing artistic research. Research Catalogue is described as follows on its website:

\begin{quote}
  The Research Catalogue (RC) is a searchable database for archiving artistic research. RC content is not peer reviewed, nor is it highly controlled for quality, being checked only for appropriateness. As a result, the RC is highly inclusive. The open source status of the RC is essential to its nature and serves its function as a connective and transitional layer between academic discourse and artistic practice, thereby constituting a discursive field for artistic research. \citep{rc2017}
\end{quote}

RC introduces the notion of the ``exposition'' as a concept denoting the way artists chose to display, or layout, their work in the database as research. A two way process in which the artistic practice is first documented and then exposed. The documentation may be everything from a recording of a performance to a documentation of some isolated part of the artistic process. RC gives no recommendations or instruction as to \emph{how} or \emph{what} should be documented, neither on how to represent the documentation. The exposition is at the center and it should guide the documentation, not the other way around. RC is not a database in the traditional sense, but rather a format for presenting and sharing research. By studying the expositions an understanding of the materiality of the underlying practice may surface, but only if the documentation reveals it.


Large scale attempts to create systems for archiving digital data include the EU and UNESCO sponsored CASPAR project: \emph{Cultural, Artistic and Scientific knowledge for Preservation, Access and Retrieval} \citep{Douglas2007,Roeder2006,Bachimont:03,cuervo2011} The scope of CASPAR by far exceeds that of the Research Catalogue so a comparison would be both unfair and uninformative. But apart from UNESCO and world heritage sites, driving partners in CASPAR were both GRM (Groupe de Recherches Musicales) and IRCAM (Institut de Recherche et Coordination Acoustique/Musique) and one of the sub-goals of the project was to establish means for preserving electro acoustic music. Among the challenges with documenting and archiving technology dependent art is that technology, in many cases, grows out of fashion long before the art. It is not uncommon, for example, for a piece of music to employ commercial hardware such as synthesizers or computers, or software, for which support from the manufacturers is discontinued as models become outdated. Given the speed of the development of new technology, in some cases this may occur in less than a decade. One example is the French composer Tristan Murail who has composed works for the Yamaha DX7 synthsizer. Although it still exists the instrument is not maintained, nor are the formats for programming it general, but specific for this particular piece of equipment.
In this case the materiality of the practice is tied to a technology without which the practice is undermined. Preserving the technology makes possible future developments of the practice, and contribute to the understanding of its uses.

It was the attempt of the Integra project,\footnote{http://integra.io/portfolio/integra-project/} another pan European project financed by the EU, to approach these challenges by documenting the abstract technology behind the works rather than the technological solutions themselves. Abstract definitions, such as an algorithm, can be implemented again and again as technology develops. However, this is an extremely time consuming process if employed on existing musical works. Nevertheless, a basis for a structure for documenting artistic works was created in Integra, originally with the intention to document the different elements in the Integra system (such as computer programs, contact information to musicians and composers and to institutions with access to information). Preserving abstract definitions of technologies may allow for greater insight into the material practice of said technology, and may approach a method for documenting practice.

It is possible to distinguish between a number of possible types of musical archives, all with a slightly different political impact. First, the musical score is in itself an archive. It is a systematized way to organize a material that makes it possible to, at a later point in time, recreate what was originally conceived. It specifies an ideal version as well as the initial authority for all future interpretations. By way of its architecture the score makes available the contents of the piece and leans on a structural hierarchy through which the individual elements are to be understood. The economy of this archive defines what needs to be registered and what is best left to its interpreters. The dividing line between what is stipulated by the score and what is left open is obviously in constant flux. Given the nature of a written musical score in print, however, the potential for openness is still quite limited in the sense that the content, the symbols themselves and their order, is rarely under negotiation, only the interpretation of them. Hence, the traditional printed score is a preservation archive that structures what kind of music is made possible by it, even though it may require much input from the interpreter.\footnote{Highly polyrhythmic or polymetric music is only with great difficulty notated in traditional notation, and only pitch based music makes sense to notate with traditional notation and traditional symbols.} It creates a system of values that defines the usability of the music as a commodity similar to, say, the CASPAR project above.

The benefit of looking at the script itself, the musical score, as an archive--normally thought of rather as an entry in an archive--is the way in which it makes the inscription, the grammar and the practice of deciphering the information, stand out as important features of the archive. The system of organization and the writing of new entries are the fundamental building blocks of any archive. In a digital era that registers almost everything we do in our daily life it is easy to forget about these organizational systems (inherent to all archives), and the way they trace our actions. For the same reason it is important to think about how to best organize archival systems that contribute to research in music.

An even stricter and much more detailed archive for music is the digital recording. For each second of recorded music it makes roughly 100.000 entries or more; digital samples encoded into a file of great temporal resolution. The digital, as pointed out by Aden Evens \citep[p. 79]{evens05} is sterile and by itself unproductive until it goes through a transformation ``that draws a line of contact between the digital and the human'' \citep[p. 79]{evens05}. In the end it is technology that mediates between the digital and the actual as the digital is not useful by and of itself. To some extent inspired by Baudrillard (whom we will return to) among others, Evens sees the digital as pure representation, in contrast to the actual in a way that sheds some light on the later discussion of this paper:

\begin{quote}
Trapped in the abstract, the pure digital operates at a remove from the vicissitudes of concrete, material existence, and this distance leads it its qualities of perfection: repeatability, measurability, transportability, etc. But the digital's divorce from the actual is also a constraint, denying it and direct power.'' \citep[p. 79]{evens05}
\end{quote}

%% lägg in referens till avhandlingen igen.
The great advantage of regarding the digital as an archive is also its biggest inexpediency, and without an efficient interface it is of limited use. Hence, the power of the interface is great. The digital recording is a preservation archive relatively agnostic as to what kind of music it can represent, but with a high degree of specification on a detailed level. It successfully records an entire performance with great resolution, but at the expense of making all events structurally equal. Digital audio is merely bits of data of equal length and incomprehensible to the human eye. Hence, the process of recording music digitally archives according to the power of digital structures, removing the dynamic properties of the performance and fixating it within its own economy of organization. Should the elements be restructured ever so little the recorded sounds would be rendered unrecognizable. The question concerning the digital is clearly of relevance to this discussion: Can that which is digitally encoded be anything other than digital?\footnote{This was a central question in my thesis, as discussed above.}

In a recent project exploring means for the preservation of electro acoustic music, the importance to include the artistic process of creation, along with the artistic output is given some support:
\begin{quote}
  Our approach focuses on the knowledge involved during the creative process, which involves but it not limited to technology. Preservation of electro acoustic and mixed music requires a suitable framework for archiving composer’s idiosyncratic musical software and documenting the work throughout its creative process. It involves archival policies for digital assets that the creative process produces, together with relevant knowledge to ensure meaningful usability [\ldots] therefore enabling re-production of the work. \citep{boutard2012}
\end{quote}
In this article Boutard and Guastavino also give a comprehensive overview on some of the most important projects in the field in the last few decades, including some of the projects mentioned below. For some of these, as well as for Boutard and Guastavino the re-production of the works archived is part of the main purpose. This goal most likely guides the documentation strategies, and may have some impact on decisions made concerning the methods for both documentation and preservation.

Media databases for storing and preserving musical works generally need to be highly structured in order to be useful and in the context of music they can be seen as augmentations to the musical score. In such archives the score is stored along with additional information about the composition and bibliographic data about the creator and the performers etc. In other words, they allow for an extension of the score, commonly as a set of meta-data. However, in the discussion concerning the materiality of the artistic practice, it is important to remember that documenting \emph{the artist} is not necessarily the solution.

The kind of huge music listening services provided online, are simultaneous developments, reductions and augmentations, not of the score but of the standard commercial recording formats. They are developments because they introduce a social network layer that connects different kinds of musics together in ways that are only meaningful for large collections of music. The networking layer is similar to a notion of unstructured meta-data as it makes the data more easily accessible. On the other hand these services commonly reduce the amount of information about the recordings contained in them (compared to what is often available on CD covers) and employ a predominant focus of collection at the expense of context and and structured meta-data. Hence, they are reductions of the traditional recording. Finally, to an ever higher degree than musical scores they explore the value of music as commodity, and due to their scope and popularity they can be seen as economical and political augmentations of the CD-recording.

Common to all of these examples of musical archives, perhaps with the exception of the Research Catalogue, is that they preserve the present state or, rather, the state at the time of registration. They are not generally built for recording change over time, such as, for example, the process of the development from the definition of a musical work to its first performance. Since time is an essential part of any attempt to truthfully describe a process, including all descriptions of musical practice and the processes behind it, this appears a rather big limitation. They are built on a traditional notion of the archive as a system for preservation. At the risk of falling back on a semiotic view on the musical work, they assume the esthesic perspective. Hence, the current question concerning the representation of the materiality is not yet addressed. How, then, can an archive that includes a notion of time be developed, that connects and successfully records how artistic processes and musical practice unfolds and develops in time?

% The discussion so far, ranging from archives for the preservation of musical works, via the restructuring of the ontology of the musical work, the personal archive and memory as medium leads us into a discussion concerning the writing, impression, inscription and printing of the different capacities of the human psyche introduced by Freud and continued by Derrida. Derrida's deconstruction of the archive in \emph{Archive Fever} is situated in this psychoanalytical framework.
%\section*{Archival strategies, musical scores and open works}
%\label{sec:archival-strategies}


\section*{Documenting openness}
\label{sec:open-work-repetition}
One of the aesthetic tendencies in the 1960s which has an impact on the current discussion was the introduction of chance and openness in musical works. In the famous text \emph{The Open Work} Umberto Eco \citep{eco68} found new tendencies in works by modernist masters Luciano Berio and Pierre Boulez among others. These works introduced the idea that part of the construction of the work was left for the interpreter to do, even in some rare cases in active collaboration with the audience. The more radical version of the open work, seen in the work by Henri Pousseur, is labeled by Eco as a \emph{work-in-movement}: ``It invites us to identify inside the category of `open' works a further, more restricted classification of works which can be defined as `works in movement', because they characteristically consist of unplanned or physically incomplete structural units.'' \citep[p. 22]{eco68} A work-in-movement is a latent, or prospective, possibility rather than a fact, yet to be realized, whose authenticity lies not in the intentions of the composer but rather in the collaboration between the different agents involved in its creation.\citep{eco68,frisk08phd} %Composing a \emph{work-in-movement} consists of supplying raw material, delivering a potential work rather than a finished one. 

Departing from Eco's reasoning in The Open Work, Swedish guitarist Stefan Östersjö and I developed an artistic method that leaned strongly on the idea of the work as a continuously developing field of possibilities in our collaboration on my composition \emph{Repetition Repeats all other Repetitions}. Originally conceived of as a fairly traditional contemporary piece for instrument and electronics it advanced into what may be called a work-in-movement. The identity of this work is located in change rather than fixity. In short, the composition invites interpreters to create their own version of the work out of an assembly of segments that could be combined in a number of different manners \citep{friskcoessens2013}. In a few articles published early in the process we discuss how our view on the work developed in the process \citep{frisk-ost06,frisk-ost06-2}.

The development of \emph{Repetition Repeats all other Repetitions} coincided with the development of the documentation database for the Integra project mentioned above. It allowed for rethinking both what an appropriate score could look like for a work-in-movement, but also how the process of creating and developing such a piece could be documented. Though the piece originally had a fairly detailed musical score there is a large number of additional data that is of great importance for the reading. Had these segments of data only consisted of written instructions in musical notation the challenge of creating this particular work's documentation may have been slightly easier. However, for a work-in-movement to work as such the documentation needs to contain not only all previous versions and their modes of construction, but also all the different parts in terms of electronic sounds (sound files, software for interaction, DSP processes for altering the acoustic sounds, etc.). Once the concept of documenting all past performances and all related data first surfaced, the way forward, it appeared, was in finding methods to defy the propagating level of noise as the archive grows bigger.\footnote{Noise, of course, is also the result when ``all'' music becomes available in enormous online databases.}

Collecting and organizing the material of \emph{Repetition Repeats all other Repetitions} over time hinted at a possible solution to the question of documenting the materiality of my artistic practice. Many questions still remains however. What is useful to document and what is not? When does the collection produce fixity rather than change, if change is what is desired? How is time represented in the flat structure of the archive?
Despite many concerns with regard to the archival process, the idea of a documentation database for the piece appeared as a sensible solution and may in the end provide a valid example of how a documentation of an artistic work sensible to the materiality of the process may be organized.\footnote{The database was developed from the work done in the Integra project and a technical description of its structure is given in \emph{ArtDoc - an Experimental Archive and a Tool for Artistic Research} \citep{frisk2017}.}

%Although the database developed in the Integra project was mainly designed for the preservation of works, as a starting point it turned out to be apt also for the current context. After a few initial proofs of concept the database ArtDoc has slowly been developed to provide the context for \emph{Repetition Repeats all other Repetitions} and it is still in a state of experimentation.

%Looking back at the discussion concerning \emph{Repetition Repeats all other Repetitions}, where the idea of 
 %As a result we see a huge market for those that attempt to create structure out of the digital pandemonium.

% To create structure where there is none has been very lucrative for the big internet search companies and similar ventures that gather loosely structured information and present it in user friendly ways. We can look at these companies as the archaeologists of contemporary media. They do not need to establish the precise origin of their findings as long as the information is valid. MIR, or Music Information Retrieval, is a related field of research for which one of the goals is to automatically create meaningful semantic structure in large collections of musical data.

% In the following section, however, I will discuss approaches that may allow the documentation of rehearsal data, performer interaction, gestural data, listener interaction and many other kinds of data well aware that this storage may preserve rather than develop knowledge.
% \section*{Archiving in practice}

%% Här behövs en överledning

\section*{The personal archive}
\label{sec:archiving-practice}


The question of \emph{what} to document is obviously entangled with the question of \emph{how} to do it. The attempt to document musical practice almost exclusively also involves a transformation to the digital realm at some point - both in the ways that the actual process is encoded, and in the way it is archived and meta-data is applied. In that sense, today, the digital is almost ubiquitous.
This was not the case for Walter Benjamin, however, who, in the short essay \emph{Excavation and Memory} from 1932 with metaphorical references to archaeology and psychology writes:

\begin{quote}
Language has unmistakably made plain that memory is not an instrument for exploring the past, but rather a medium. It is the medium of that which is experienced, just as the earth is the medium in which ancient cities lie buried. He who seeks to approach his own buried past must conduct himself like a man digging. Above all, he must not be afraid to return again and again to the same matter; to scatter it as one scatters earth, to turn it over as one turns over soil. For the 'matter itself' is no more than the strata which yield their long-sought secrets only to the most meticulous investigation. \citep[p. 576]{benjamin2005}
\end{quote}

\noindent
It is through repetition and conscientious analysis of findings that the layers of memory and the subconscious (the psychoanalytical dimension of Benjamin's text is impossible to ignore) are possible to excavate properly, but the very essence of the findings lies not in the thing found, but in the 'soil' where it was first found. When one's memories are probed again and again one is eventually able to understand the findings anew; the secrets are decoded and become clear, as images that are sliced off from all earlier association. To Benjamin, these are the ``treasures in the sober rooms of our later insights.'' \citep[p. 576]{benjamin2005} The way time is brought into the discussion is interesting. At first, we may or may not understand what we experience or remember, or even the meaning of what we find in our excursions, but through a complex process of dissociation and re-association time will distill the findings and separate the meaningless from the usefully consequential; the real treasures.

But the findings themselves are not enough, the enhanced awareness that only experience and knowledge can give us are also necessary, and the continuous and cautious ``probing of the spade'' is equally important in the process of uncovering. ``And the man who merely makes an inventory of his findings, while failing to establish the exact location of where in today’s ground the ancient treasures have been stored up, cheats himself of his richest prize.'' \citep[p. 576]{benjamin2005} In other words, the dynamic between what has been found and the location of this finding is of significance. The ground that I open, that unravels my findings, is part of the information that can make me understand what I have found. In that sense, as a consequence, also the person who finds plays an important role: ``genuine memory must [\ldots] yield an image of the person who remembers.'' \citep[p. 576]{benjamin2005}

This short text by Benjamin gives an insight in his own obsessive archiving of his own practice.\footnote{Much of the following is based on the documentation in \citet{benjamin2007}} Benjamin's personal archive is particularly interesting in the way that it appears to document not only what he worked on, and materials related to books or essays that he was writing, but also what he did in general. Not only lists of books that he read for a particular project, but all books that he read or had ever read. Not only correspondence with individuals that he worked with, but all kinds of correspondence, including lists of expressions that his son invented and similar personal details. The lists, written on small archive cards or sheets of paper, were continuously edited by entries scored out and notes added. He meticulously and apparently tirelessly archived virtually everything that he came across constructing an archive that is difficult to decipher, at best, but also one that is conceptually ahead of its time. It is through this project, his own archive, that his ideas on memory and excavation must be understood. But here lies also the danger: ``As far as the collector concerned, his collection is never complete; for let him discover just a single piece missing, and everything he's collected remains a patchwork'' \citep[p. 211]{benjamin1999}. No collection can be complete and if at any point in time, any item is missing from the archive it will appear inadequate.

Commonly, an archive of musical material archives representations of musical content, and to be meaningful it has to go beyond a mere collection of resources. An archived score is relatively easy to represent accurately but is a poor representation of the actual music. A recording of a performance is an accurate representation of the sound but a poor representation of the material performance. Furthermore within the category of musical recordings, there is a continuum between the concert documentation and the edited recording. Though both the score and the recording are reductions of the materiality of the performance to an archivable documentation/representation format, the concert recording may be seen as the more authentic representation of a performance than the edited recording. Connected in both time and space the score along with a recording can be a phenomenal catalyst for musical materiality, while at the same time it structures one's perception of the music.
There is a similarity between Benjamin's collecting and my experiments with documenting \emph{Repetition Repeats all other Repetitions}. The relentless storing of unrelated data is one common aspect, but more important is the way the collection allows for the possibility of returning again and again. After all, allowing for continuous reflection, along with the belief that the act of reflection is informative, is one of the motivations behind the ambition to document the material practice.
How, then, may such a large archive of musical content be structured so that we are not cheated of the richest prize, as Benjamin puts it? How may we create systems for collecting musical knowledge and experience, that can be explored and excavated for new meanings? To further investigate this it is necessary to consider in what ways musical experience may be written.


%How can we create archives 
%, or more, which will provide a loose theoretical framework to the discussion in this paper.
%Though this will constitute part of the theoretical frame for the discussion, the method employed will depart from my own artistic practice: the musical materiality discussed here will be that experienced by me in performance. 

% The hypomnesic archive is a place for collecting and storing thoughts, knowledge: making inscriptions that may later be revisited. As such it is a tool of reflection, of turning back and reconsidering concepts formed at an earlier stage.


\section*{The scene of writing}
\label{sec:scene-writing}

In \emph{Archive Fever: A Freudian Impression} (Le Mal d'Archive), a late and read worthy text on the nature of the archive written in the beginning in the beginning of the internet era, the French philosopher Jacques Derrida, to little surprise given his critical method, points to the dangers of the structures of the archival process. The book is based on a lecture given in London during an international colloquium on ``Memory: The Question of Archives'' in 1994. This was in the early days of electronic communication but the text is strikingly clairvoyant of what we now know has become the impact of digital technology. Much of the content departs from a much earlier, but related text: \emph{Freud and the scene of writing} \citep{der78}. Before pursuing the discussion concerning the nature of the archive I will go over some of the central arguments that Derrida makes in this earlier text with regard to Freud's concept of perception and memory as these may prove to be useful here. The following is not intended as a complete analysis of this rather complex text, but should mainly be regarded as an attempt to unpack some of the central concepts used in \emph{Archive Fever}.

Derrida addresses Freud's rethinking of writing in \emph{Freud and the scene of writing} as a metaphor for accessing the possible relationships between perception, memory and experience; for understanding how the outside world is registered within the psyche. By looking at psychical content as something that is the result of different kinds of writing, a writing that is, according to Derrida, in its very essence, graphic, he draws a picture that clearly separates internal experience from external storage.
%Furthermore, and as a consequence, the psychical apparatus itself will here be represented by a device that prints onto one's consciousness:
%As a writing machine.

% So far the discussion has primarily followed the trajectory layed out in the two texts by Derrida concerning writing in general. Verbal representation is as central to Freud as textual is to Derrida. Hence, how can this discussion be translated to the field of music?

From the short text \emph{A Note upon the “Mystic Writing Pad”} \citep[Included in:][p. 207-12]{freud1997} Derrida,  construes three different analogies concerning the relation between writing and perception, each with increased levels of introspection and contrast. The metaphor of writing is here to be understood as simultaneously appropriating ``the problems of the psychic apparatus in its structure and that of the psychic text in its fabric.'' \citep[p. 259]{der78} Considering recent developments in the fields of cognition and ecological psychology the following discussion should be regarded as a philosophical elaboration on the metaphor of writing and, as such, how it impacts on the notion of the archive.

The first analogy is similar to the most obvious understanding of the word \emph{writing} and concerns writing as a material expression of something stored in one's memory. A means to jot something down on a piece of paper, forget about it and later revitalize the memory by way of the writing. Ideally this kind of writing should satisfy both the need for unlimited capacity and indefinite preservation: rather similar to the obviously impossible demands we place on current archives for research data for example. It is writing as externalization of memory, a representation of memory outside of the psychic apparatus and separated from it by perception.

\emph{The Mystic Pad} is a small device that Freud describes as ``a slab of dark brown resin or wax with a paper edging; over the slab is laid a thin transparent sheet, the top end of which is firmly secured to the slab while its bottom end rests upon it without being fixed to it.'' \citep[p. 209]{freud1997} The method of writing on the pad is similar to scratching in wax or clay with the addition that the surface is easily wiped clean again and made ready for new impressions. According to Freud the Mystic pad can be seen as a model of the perceptual apparatus: it records but leaves no visible traces and can be said to be infinitely ready to receive anew. The writing on the pad is an illustration of the second analogy of writing and corresponds to the writing on the body, the first imprint on reception, but before it is registered by consciousness. This writing ``supplements perception before perception even appears to itself.'' \citep[p. 282]{der78}

If the two first analogies have been related to the space of writing, the third is to a significant degree concerned with the time of writing. Time as the consequence of the distribution of symbols and impressions, and as the result of the various strata in the psyche. Time as in memories being forgotten, re-remembered and re-inscribed, or, in the words of Freud, as ``cathectic innervations'': ``as though the unconscious stretches out feelers, through the medium of the system Pcpt.-Cs., towards the external world and hastily withdraws them as soon as they have sampled the excitations coming from it.'' \citep[][p. 211-2]{freud1997} This discontinuous functioning of the perceptual system, according to Freud, ``lies at the bottom of the origin of the concept of time.'' \citep[p. 212]{freud1997}

Derrida's reading of Freud is multifaceted, and the brief discussion here only touches upon a fragment of its prospect. As a deconstruction, however, \emph{Freud and the scene of writing} would be incomplete if it did not also critically examine Freud's conclusions.
One of the comments brought forward does have some impact on the topic of materiality.
Questioning the relation between spontaneous memory, and the pure absence of spontaneity observed in the machine, Derrida claims that Freud does not:
\begin{quote}
examine the possibility of this machine, which, in the world, has at least begun to \emph{resemble} memory and increasingly resembles it more. Its resemblance to memory is closer than that of the innocent Mystic Pad: the latter is no doubt infinitely more complex than slate or paper, less archaic than palimpsest; but compared to other machines for storing archives, it is a child's toy. \citep[286-7]{der78}
\end{quote}
 The archive is a representation of this machine referred to by Derrida and the question of what the relation between an external writing and the internal experience may be is at the center of the current discussion.

This rather long overview of what may, or even should, be considered a logocentric view of the human psyche\footnote{Derrida is often accused for logocentrism--sometimes erroneously or for the wrong reasons--but in the beginning of \emph{Freud and the scene of writing} he actually makes the claim that logo-phonocentrism is ``rather a necessary, and necessarily finite, movement and structure: the history of the possibility of symbolism \emph{in general}.'' \citep[p. 247]{der78}} is included here to provide a frame for the central discussion of this paper: how can the material aspect of music making, of the artistic practice in music, be understood, and what is the relation between this practice and any effort of providing an archive for it? Considering the three analogies of writing above, could these be translated to the realm of music? Are they meaningful in relation to the nature of playing--playing should here be understood in the widest sense of the word--music? Before engaging in the attempt to re-contextualize them, it is important to identify that the metaphor of writing, though similar to a specific practice like musical composition, in many respects has fundamentally different properties compared to playing music in general, and these need to be considered. These differences partly amount to the organization of the underlying structure of respective practice, but also to how \emph{writing} easily misleads us to think about production of a text that stands by itself, whereas the physical traces of music can mainly be heard, and thus experienced.

If first we consider the analogy of writing as an externalization of a memory, the first thing that comes to mind is composition which can be said to be a system for remembering music by means of inscribing notes on paper. For Freud this such notation would be a ``materialized portion of my mnemic apparatus'' \citep[p. 207]{freud1997}, but it would be a reduction of what we think of as music to limit its exteriorization to musical notation. Instead, the first analogy of writing in the context of music should include all kinds of musical activities that emanates from the mind or memory, and results in a physical trace of some sorts: singing in the shower, improvising, performing an interpreted composition, composing a score, composing electronic sounds, etc. This would require us to rethink what writing means: what is referred to here is a sort of writing onto the world.
Reconnecting to Benjamin's modes of reflection over collected material, memories probed over and over again, perhaps one may imagine a system that evokes a feedback loop over the three analogies of writing that at least conceptually could be meaningful in the current discussion.

Secondly, the analogy of a writing that leaves no permanent traces may be seen to correspond to perception as a system separated from memory. Thinking of memory as a physical storage unit (which of course it is not) it is easy to imagine that it can run out of space, that we reach a point where we cannot remember anymore. In contrast, our perceptual apparatus, unless struck by illness, has almost unlimited capacity. We can listen and listen, in fact we cannot \emph{not} listen. But not only the ears perceive music, the entire body is written when it is struck by the vibrations from sound waves. The writing on the body, on perception, however, is elusive and disappears immediately which is why it is also always ready to be written anew.

Finally, related to time, the third and most profound analogy of writing is the one most easily translated to music and playing. Derrida's claim that ``time is the economy of a system of writing'' \citep[p. 284]{der78} is effortlessly transformed to a musical context. Beyond its immediate understanding the analogy is based on the way Freud imagines that there is a movement from within the psyche that reaches out towards the external world and the discontinuity involved in this process is what introduce time. There are a number of interpretations of this that would make sense in material musical practice. The continuous act of intonation performed by a musician, the subtle adjustments made in performance against a memorized piece of music or a tonal system, the temporal adjustment to a pulse, or the improvisers sensibility to the co-musician's activities are all examples of ``cathectic innervations'' in the musical practice.

Even if some of these analogies of musical writing also applies to musical listening, or even listening to sounds in general, I would claim that they are all important aspects of the materiality of artistic practice in music. One's writing of music, the writing of music onto one's body and the (subconscious) reflective, discontinuous impulses reaching out into to external world, are all integral parts of a broad view of musical practices. The question now is not if any one of these modes or analogies may be archived without any central aspect of them being lost, because the immediate answer to that question is of course no. It is yet unimaginable to store experiences independently of the one experiencing, though being able to do this is the theme for many science fiction stories. The more accurate question to pose would be along the lines of: what is the relationship between the live experience of a musical practice such as it is described in terms of analogies of writing, and possible representations and documentations of these activities in various forms for archives? Another, related question, that will be brought up in the next section is: In what ways may our different systems of representation (archives) affect our internal systems of writing?

\section*{Writing the past}
\label{sec:archive-fever}

I will now return to \emph{Archive Fever} where the dualism between exerience and cognition on the one hand, and the hypomnesic nature of the archive and its exteriority on the other, is a central topic. Derrida repeatedly comes back to how the external writing of the archive results in an, in every respect, different record than that of the recording of the perceptual apparatus:
\begin{quote}
Because the archive [\ldots] will never be either memory or anamnesis as spontaneous, alive and internal experience. On the contrary: the archive takes place at the place of originary and structural breakdown of said memory. \citep[p. 11]{derrida1998}
\end{quote}
This could be understood from the rather direct and simple point of view that relying on an external source of memory restructures one's own, internal memory. Given the information surplus in the digital era, and the extent to which we commonly rely on auxiliary memory devices this somewhat susceptible relationship, I believe, may be recognized by many. Though many of the available technological devices, services and functions that surround us at times appear to be working akin to how internal memory processing functions, conceptually there is an irrevocable dividing line between how the inside and outside is structured: ``There is no archive without a place of consignation, without a technique of repetition, and without a certain exteriority. No archive without outside.'' \citep[p. 11]{derrida1998} Derrida returns to this separation many times in the text and as was pointed out above, already in \emph{Freud and the Scene of Writing} it was a guiding principle: ``The machine--and, consequently, representation--is death and finitude within the psyche.'' \citep[p. 286]{der78} Perhaps the uniqueness of the materiality of artistic practice in music may allude to the antithesis of the archive given by Derrida: ``anamnesis as spontaneous, alive and internal''? The quality that separates it from any representation of the said practice?
Is then the very attempt at documenting a material practice futile?

The dualism of the two systems, the technically oriented, external archive, and the internal memory and human experience, does not restrain them from evoking influence on each other, at least not in abstract ways which was briefly discussed in the first section of this paper. As a consequence also this seemingly definite relationship is deconstructed. Because as clear and permanent the separation between inside and outside may be, the extrinsic archive nevertheless projects its characteristics onto its users. The structurality of the writing of the archive disallows it from being a neutral site for recording, if such an idea ever existed. It acts as a political as well as an economic force upon that which it records:

\begin{quote}
  It is thus the first figure of an archive, because \emph{every} archive, we will draw some inferences from this, is at once \emph{institutive} and \emph{conservative}. Revolutionary and traditional. An \emph{eco-nomic} archive in this double sense: it keeps, it puts in reserve, it saves, but in an unnatural fashion, that is to say in making the law (\emph{nomos}) or in making people respect the law. \citep[p. 7]{derrida1998}
\end{quote}
More specifically, the archive does not only record but also conditions what may be written through its control over the structurality of the writing:
\begin{quote}
[\ldots] the technical structure of the archiving archive also determines the structure of the archivable content even in its very coming into existence and in its relationship to the future. The archivization produces as much as it records the event. \citep[p. 17]{derrida1998}
\end{quote}
Hence, even the \emph{wish} to archive and to make content accessible in a structured format creates delimitation and determined articulations that exclude as much as it makes available.

Derrida goes on to make the observation that if technology is an archivization process that produces as much as it records, this
\begin{quote}
means that in the past, psychoanalysis would not have been what it was (no more than so many other things) if E-mail, for example, had existed. And in the future it will no longer be what Freud and so many psychoanalysts have anticipated, from the moment E-mail, for example, became possible. \citep[][p. 17]{derrida1998}
\end{quote}
One may want to argue against this considering what we now know about electronic communication, and the example of Freud obviously has a particular meaning considering the psychological dimension with a certain focus on the suppressed and unconscious. But on the other hand it is safe to assert that the impact of communication technologies on all aspects of society and culture has been anything but insignificant. Hence, not only does the archive to some extent determine what may be archived, it also influences what is written and in what ways it is written. For the current discussion it is worth considering how the impact of all different kinds of archiving systems employed may affect both artistic and research practices.

Finally, one other important aspect of the present day archive fever needs to be considered. In the relentless wish to reposit potential experiences in easy to handle, downloadable packages, we miss out not only on the physical artifacts, the origin of the experience, that are not as easily encoded, the focus on the past rather than the future, risks at becoming the driving force. The archive persistently points to history, institutive, yes, but also conservative: ``And the word and the notion of the archive seem at first, admittedly, to point toward the past, to refer to the signs of consigned memory, to recall faithfulness to tradition.'' \citep[p. 33]{derrida1998} While this may be self evident, and to some extent the very purpose of many of the archiving initiatives of today, such as some of the preservation initiatives mentioned earlier, in the section \emph{Documentation databases} \ref{sec:docum-datab}, is to designate the past and preserve a tradition that may otherwise be lost. The principle argument for doing this is not that the contents of the archive itself should  allow for development, but that the availability of it triggers activities that points ahead and bridges the gap between past, present and future.\footnote{Later in the text Derrida points out how the archive also satisfies the future, as a ``question of the future, [\ldots], the question of a response, of a promise and of a responsibility for tomorrow'' \citep[p. 36]{der78} but does so only in an enigmatic sense.} Part of Derrida's argument, however, is that the relations between the inner and the outer modes of writing may not be as yielding as they may first appear. Unreflectedly relying on the archive may alter the view on what should actually be archived and how it should be structured. As was discussed above, this archival activity may under certain circumstances also modify how the process to be archived develops over time. Hence, the ambition to document the materiality of the practice may change the practice itself: instead of gaining insights about the practice, the practice is adopted to fit the models of documentation.

The danger of the metaphor of writing the way it has been used here is its tendency towards an individual act of writing, as in \emph{one writer} writing \emph{a narrative}. The different articulations of writing discussed, however, hopefully counteracts this modernist inclination and instead points to the multiplicity of possible modes of writing. This is given some support by Derrida who, concerning the Mystic Pad, states that ``the subject of writing is a \emph{system} of relations of strata.'' \citep[p. 285]{der78}\footnote{ArtDoc, the documentation database developed out of the process with \emph{Repetition Repeats all other Repetitions}, is built on the idea of a system of relations of strata.} 

\section*{Discussion}
\label{sec:discussion-1}

Following Derrida it is impossible to be in control of the archive. Its logic is so substantially different from human experience, and it is only through a thorough understanding of the difference, of the lack of concept of the archive, that it is possible to bridge the consequences of the differences. Is it even desirable to be in control of the archive or of the archival process? Understanding that the structure of the archive participates in writing the content makes it possible to design systems in ways that reinforces the intended results rather than disguises them. Knowing that the archive will always have the tendency to point to the past rather than the future can be used both to avoid this to influence the results and the data, but also in order to design systems that are prone to change over time. The risks, however, are still that we mistake the representation for the real. A few years after Derrida's \emph{Archive Fever}, Jean Baudrillard, in the essay \emph{The Automatic Writing of the World}, takes a typically dystopic point of view on the relationship between what he calls the real and the double, or the real and the representation. He gives some support to the notion of the incompatibility between the inner and external, the real and the virtual, anamnesis and hypomnesis:

\begin{quote}
  The perfect crime is that of an unconditional realization of the world by the actualization of all data, the transformation of all our acts and all events into pure information: in short, the final solution, the resolution of the world ahead of time by the cloning of reality and the extermination of the real by its double. \citep[p. 25]{baudrillard96}
\end{quote}

The idea that everything is archivable, everything may be documented and everything may be verified is so second nature today that it is easy to see the representation as the real. Even if an experienced listener can re-engage many of the material properties lost in a recording of a concert, the representation only encompasses a subset of the information available to the listener of the performance. The transformation of music from something experienced in a live performance to something which may be contained, packaged and transmitted has clearly been fueled by the commercial powers of the music industry. It has also participated, I would claim, in the development of the concept of the musical work. If the work is defined by its archivable traces--its score and its recording--it becomes self contained and is much more easily distributed and legally protected \citep{attali85}.

The view upon the (digital) technology of the archive as an engulfing force that determines not only what is written but also what is read may however be critiqued. As pointed to at the beginning of this paper it is possible to give up the wish to control the technology and thereby allowing for a different approach to the question. Metaphorically speaking one should allow the archive to structure itself and in an act of reading, allow it to continuously restructure itself. It is in the wish to make the archive resemble the real that the archive that writes itself really becomes a problem. To further develop these thoughts the flat ontology of new materialism may actually become useful. Moving from recording the minute interactions in a process of musical composition, by way of documenting several instances of musical works, and on to connecting different art forms may yield new, and important, insights. Or, as put by Giuliana Bruno:

\begin{quote}
  After all, we should consider that art, architecture, fashion, design, film, and the body all share a deep engagement with the world of objects and their superficial matters, including such things as the materials of the canvas, the wall, and the screen. If materiality defines an art practice it can also act as a connective thread between separate art forms, creating a productive exchange. We cannot disregard the ways in which contemporary artists are engaged in this connective mode of investigating material practice, incorporating different material formations in a productive dialogue, on the surface tension of media. \citep[p. 15]{joselit2016}
\end{quote}

The thinking about archiving musical materiality within the framework of \emph{Repetition Repeats all other Repetitions} gave rise to the possibility of rethinking the concept of the musical work. It is perhaps obvious, but nevertheless necessary to point out here, that the concept of the musical work also will shape the idea of what materiality in music consist of. If the work is defined by its continuously changing artistic, social and political interactions through the work's interrelations between other works, and its own contents, chances are that its dynamic properties are seen as essential. An open work definition and a multidimensional system for storing the data of the work was one of the aims that grew out of \emph{Repetition Repeats all other Repetitions}.

%Specifically  of writing comes to mind. 
Although the discussions held by Benjamin on the one hand, and Derrida on the other, are very different in scope and focus, they also converge at interesting points. Derrida's various analogies of writing, of perception as a form for writing, described by Derrida brings to mind Benjamin's poetic claim that true memory must ``yield an image of the person who remembers.'' \citep[p. 576]{benjamin2005} On the surface, however, the archive as an externalized memory machine obviously does not allow for the person who remembers unless the machine allows its memory device to be continuously reconfigured in which case its role as archive becomes futile. This reconfiguration is however possible to mimic without the integrity of the data being lost, if the focus is moved from the structurality of the data to the relations between data entries, between the strata of data. A preliminary conclusion in the ongoing investigation of the main question about the possible representations of materiality of the artistic practice in music, the idea of strata of interconnected pieces of data appears promising. It would allow for representations of time and for asymmetrical objects of data to be interconnected. It would also allow for continuous reconfigurations of said objects without the integrity of the data itself to be lost. As such it does not privilege the writer over the reader of the archive. And here, the question concerning what constitutes the actual data may find an experimental and likewise preliminary answer. Benjamin reminds us of the importance to keep in mind the person \emph{reading} and not only the person writing. The deconstruction of the roles of the writer, and the reader and the different notions of writing, though impossible to record and archive properly, may at least widen the perspective and disallow the (digital) archive to restructure and continuously narrow down what we see as valid data in the materiality of artistic practice.

The idea put forth by Benjamin, that memory is the medium rather than the instrument for exploring the past, perhaps allows us to look at the archive, not as a storage container that can be read from and written to, but instead as a medium that allows for a subject to read and write, creating a multilayered system of interconnected relations. Surely, this will not be without a significant effort, and with a resulting depth that may be difficult to disentangle. But as Benjamin concludes, only to the most meticulous digger will the matter itself, the ``richest prize'' \citep[p. 576]{benjamin2005} reveal itself.


\printbibliography
\end{document}