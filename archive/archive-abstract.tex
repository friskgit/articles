% Created 2016-10-24 Mon 16:35
\documentclass[12pt]{article}
\usepackage[english]{babel}

%% Disable when not using html output
%\usepackage{tex4ht}
%\usepackage{pxfonts}
%%%%%%%%%%%%%%%%%%%%%%%%%%%%%%%%%%%%

\usepackage[T1]{fontenc}
\usepackage{url}
\usepackage[utf8]{inputenc}
\usepackage{enumitem}
\usepackage{csquotes}
\usepackage{fixltx2e}
\renewcommand{\encodingdefault}{T1}

%% Enable for graphics 
%\usepackage[pdftex]{graphicx}
%%%%%%%%%%%%%%%%%%%%%%%%%%%%%%%%%%%%

\usepackage{setspace}
\usepackage[style=authoryear,natbib=true,backend=biber,firstinits=true,hyperref=false]{biblatex}
\bibliography{./../biblio/bibliography}

%% Enable when using PDF output
 \renewcommand{\rmdefault}{pad}
 \renewcommand{\sfdefault}{pfr}
%%%%%%%%%%%%%%%%%%%%%%%%%%%%%%%%%%%


\author{Henrik Frisk}
\title{The archive that writes itself}

\begin{document}

\maketitle

% I, however, had something

\section{Introduction}
\label{sec:introduction}


\noindent
In artistic research the material basis of the artistic practice may be both data and result. The way in which either should be represented in the research is however largely an unsolved issue. Particularly in the context of performative arts and music. This paper will attempt to critically examine the relations between artistic practice in music and its possible representations in various forms for archives.

Today the digital is almost ubiquitous. The attempt to document musical practice almost exclusively also involves a transformation to the digital realm at some point - both in the ways that the actual process is encoded and in the way it is archived and meta-data is applied to it. The importance of both of these two phases of documentation is described by Walter Benjamin in the following passage (here with metaphorical reference to archaeology, excavation and memory): ``And the man who merely makes an inventory of his findings, while failing to establish the exact location of where in today's ground the ancient treasures have been stored up, cheats himself of his richest prize.''\citep[p. 576]{benjamin2005} A contemporary example of the predominant focus of collection at the expense of context and meta-data are the popular commercial music listening databases. This is of course consistent with the rest of the internet, probably the largest archive ever created, and, as such, in complete absence of structure. %As a result we see a huge market for those that attempt to create structure out of the digital pandemonium.

To create structure where there is none has been very lucrative for the big internet search companys and similar ventures that gather loosely structured information and present it in user friendly ways. These companys are the archeologists of contemporary media. They do not need to establish the precise origin of their findings as long as the information is valid. 

\section{Background}
\label{sec:background}

An archive of musical material most often archives representations of musical content, and to be meaningful it has to go beyond a mere collection of resources. An archived score is relatively easy to represent accurately but is a poor representation of the actual music. A recording of a performance is an accurate representation of the sound but a poor representation of the material performance. There is a continuum between the two categories of what we might call the concert documentation and the edited recording. Both of these are reductions of the materiality of the performance to the archivable documentation/representation format. How, then, may an archive of musical content be structured so that we are not cheated of the richest prize, as Benjamin puts it? 

Jacques Derrida, to little surprise, points to the dangers of the structures of the archival process: ``[\ldots] the technical structure of the archiving archive also determines the structure of the archivable content even in its very coming into existence and in its relationship to the future. The archivization produces as much as it records the event.'' \citep[p. 17]{derrida1998} Even the \emph{wish} to archive and to make content accessible in a structured format creates delimitation and determined articulations that exclude as much as it makes available.

%, or more, which will provide a loose theoretical framework to the discussion in this paper.
The discussion so far is well connected to some of the central philosophical discussions in the last century. Derrida's deconstruction of the archive is situated in a psychoanalytical framework that rests on the notion of the impression/inscription/recording of the unconscious. Though this will constitute part of the theoretical frame for the discussion, the method employed will depart from my own artistic practice: the musical materiality discussed here will be that experienced by me in performance. Furthermore, in artistic research rehearsal data, performer interaction, gestural data, listener interaction and many other kinds of data may need to be collected, stored, shared and researched. In my own practice and research I have designed an experimental archive for addressing some of these needs and this experience will also be discussed here.

\section{Archiving in practice}
\label{sec:archiving-practice}

\section{Discussion}
\label{sec:discussion}


%Hence, in this paper this will be the theoretical framework. 
%However, it will also be situated in my own experience as an artistic researcher engaged in this topic. Specifically I will present some of my thoughts in relation to the experimental database for artistic work that has been in development during several years.

% Benjamin points to the impact of context. The location for the excavators finding is more important than the what is found.




\printbibliography
\end{document}