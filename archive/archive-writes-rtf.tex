% Created 2016-10-24 Mon 16:35
\documentclass[11pt,a4paper]{article}
\usepackage[english]{babel}

%% Disable when not using html output
\usepackage{tex4ht}
%\usepackage{pxfonts}
%%%%%%%%%%%%%%%%%%%%%%%%%%%%%%%%%%%%

\usepackage[T1]{fontenc}
\usepackage{url}
\usepackage[utf8]{inputenc}
\usepackage{enumitem}
\usepackage{csquotes}
\usepackage{ebgaramond}
\usepackage{helvet}
\usepackage{titlesec}
\usepackage{titling}
\usepackage{setspace}
\onehalfspacing
\usepackage[
  margin=3.5cm,
  includefoot,
  footskip=30pt,
]{geometry}
% \usepackage[]{fullpage}
\renewcommand{\encodingdefault}{T1}

%% Enable for graphics 
%\usepackage[pdftex]{graphicx}
%%%%%%%%%%%%%%%%%%%%%%%%%%%%%%%%%%%%

\titleformat{\section}
  {\normalfont\sffamily\Large\bfseries}
  {\thesection}{1em}{}

\usepackage{setspace}
\usepackage[style=authoryear-ibid,natbib=true,backend=biber,hyperref=false]{biblatex}
\bibliography{/Users/henrik_frisk/Dropbox/Documents/articles/biblio/bibliography}

\pretitle{\begin{center}\sffamily\huge\bfseries}
\preauthor{\begin{center}
            \large\sffamily \lineskip 0.5em%
            \begin{tabular}[t]{c}}
              \predate{\begin{center}\sffamily\large}
                
\author{Henrik Frisk}
\title{The archive that writes itself}

\begin{document}

\maketitle

\section*{Introduction}
\label{sec:introduction}
\noindent
One of the great challenges in artistic research is that the material basis of the artistic practice may be both data and result, but without an active presence of some kind of representation of this practice the relevance of the research may be lessened. This is not necessarily a defining property of \emph{all} artistic research--there are in fact a number of ways in which artistic research can be thought of and explored--but practice based artistic research in music where the practice itself plays a significant role makes the issues in the present discussion surface in interesting ways. If the material basis of the artistic research may be both data and result, there are methodological concerns that are intimately linked to how the practice is documented and how this result is stored. These challenges may be critically examined from both the point of view of the design of such documentation systems, and from the point of view of the practice itself. From a preliminary standpoint there are two aspects to this: The first is concerned with deciding what part of a performance or practice is relevant to document given a particular research question? The second involves the technology that makes it possible to extract, store and reenact the data. These two aspects are obviously not independent, as will be discussed later in this paper, influence each other to a significant degree.

Going back to the opening statement it may be necessary to unpack the meaning of some of the terminology, specifically \emph{materiality} and \emph{data}. With data I refer to research data in an abstract sense: that information obtained through observation, experimentation, reading or other means and that constitute the raw material for the research. This may or may not constitute digital data and its nature may be very volatile. In artistic research the topic of documentation of research data is still a very important but largely unresolved question. Due to the novelty of the field artistic research data is no solid agreement as to what it actually consists of.\footnote{This is quite literally the question currently addressed in Sweden. On behalf of the Swedish government artistic colleges and universities are now beginning to review what artistic research data may constitute.} The issue may be divided into several subtopics but first and foremost it is important to distinguish between how to gather and document the research data on one hand, and the documentation of the artistic output on the other. Even if the result may be easily documented, the processes that led to the result is commonly equally important to the research process, and not always as easily documented. How can the integrity of the research data be preserved in an artistic research project that may contain a number of different kinds of data, as well as raw material from the artistic process? 

The meaning of \emph{material} is perhaps slightly more complicated to sort out given how this term is loaded with philosophical connotations.\footnote{Later in this text, however, we will come across Derrida's somewhat different account of the meaning of \emph{materiality}, but the specifics of his usage of it is not of paramount importance for this context. Nevertheless, it is important I believe to distinguish between them.} In this short text at least two slightly different meanings will be encountered. In the opening sentence the meaning is closely related to a Marxist notion of \emph{materialism}. It is through material conditions that history and collective consciousness should be understood, ``not the other way around, thereby famously reversing Hegelian idealism.'' \citep[p. 25]{beetz2016} According to Marx our knowledge about the conditions of life should emanate from our experiences with life, from the bottom up, so to speak:
\begin{quote}
  The production of ideas, of conceptions, of consciousness, is at first directly interwoven with the material activity and the material intercourse of men, the language of real life. Conceiving, thinking, the mental intercourse of men, appear at this stage as the direct efflux of their material behaviour. \citep[p. 47]{marx1970}
\end{quote}
What this would mean in the context of artistic research in music is that knowledge and thinking about music emanates primarily from the practice and not the other way around. It is the very condition of being engaged in music artistically that forms the basis of our understanding of the same. With these two rudimentary definitions of \emph{data} and \emph{materiality} it is clear that the representation of the data that stems from the material practice needs to pertain a certain notion of the original materiality itself. The overarching question for this paper is if and how this can be achieved.

\section*{Documentation music}
\label{sec:docum-datab}

Both of the axes of the artistic research process--the documentation of the process and the result and the collection and preservation of data--are equally important, but the knowledge and the means for archiving them are still insufficiently explored. There have been attempts to come up with general solutions. One of the more notable ones is the Research Catalogue,\footnote{https://www.researchcatalogue.net/} an online archive for artistic research and the home for a number of online journals for artistic research such as Journal of Artistic Research (JAR) \footnote{http://www.jar-online.net/} and the Finnish Ruukko.\footnote{http://ruukku-journal.fi/en} Research Catalogue is an online based computer implementation of the notion of a blank sheet: an attempt to create a transparent platform for sharing artistic research. Research Catalogue is described as follows on its website:

\begin{quote}
  The Research Catalogue (RC) is a searchable database for archiving artistic research. RC content is not peer reviewed, nor is it highly controlled for quality, being checked only for appropriateness. As a result, the RC is highly inclusive. The open source status of the RC is essential to its nature and serves its function as a connective and transitional layer between academic discourse and artistic practice, thereby constituting a discursive field for artistic research. \cite{rc2017}
\end{quote}

RC introduces the notion of the ``exposition'' as a concept denoting the way artists chose to display, or layout, their work in the database as research. A two way process in which the artistic practice is first documented and then exposed, in other words, RC is not a tool for documenting artistic research. Documentation may be everything from a recording of a performance to a documentation of some isolated part of the artistic process. JAR gives no recommendations or instruction as to \emph{how} or \emph{what} should be documented, neither on how to represent the documentation. The exposition is at the center and it should guide the documentation, not the other way around. RC is not a database in the traditional sense, but rather a format for presenting and sharing research with some basic methods for interacting on research projects.

Large scale attempts to create systems for archiving digital data include the EU and UNESCO sponsored CASPAR project: \emph{Cultural, Artistic and Scientific knowledge for Preservation, Access and Retrieval} \citep{Douglas2007,Roeder2006,Bachimont:03,cuervo2011} The scope of CASPAR by far exceeds that of the Research Catalogue so a comparison would be both unfair and uninformative. But apart from UNESCO and world heritage sites, driving partners in CASPAR were both GRM (Groupe de Recherches Musicales) and IRCAM (Institut de Recherche et Coordination Acoustique/Musique) and one of the isolated goals of the project was to establish means for preserving electro acoustic music. Among the challenges with documenting and archiving technology dependent art is that technology, in many cases, grows out of fashion long before the art. It is not uncommon, for example, for a piece of music to employ commercial hardware such as synthesizers or computers, or software, for which support from the manufacturers is discontinued as models become outdated. Given the speed of the development of new technology, in some cases this may occur in less than a decade. One example is the french composer Tristan Murail whose many works relied on the Yamaha DX7. Although it still exists the synthesizer is not maintained and the formats for programming it are not general, but specific for this specific piece of equipment.

It was the attempt of the Integra project,\footnote{http://integra.io/portfolio/integra-project/} another pan European project financed by the EU, to approach these challenges by documenting the abstract technology behind the works rather than the technological solutions themselves. Abstract definitions, such as an algorithm, can be implemented again and again as technology develops. However, this is an extremely time consuming process if employed on existing musical works. Nevertheless, a basis for a structure for documenting artistic works was created in Integra, originally with the intention to document the different elements in the Integra system (such as computer programs, contact information to musicians and composers and to institutions with access to information).

It is possible to distinguish between a number of possible, different types of musical archives, all with a slightly different political impact. First, the musical score is in itself an archive. It is a systematized way to organize a material that makes it possible to, at a later point in time, recreate what was originally conceived. It specifies an ideal version as well as the initial authority for all future interpretations. By way of its architecture the score makes available the contents of the piece and leans on a structural hierarchy through which the individual elements are to be understood. The economy of this archive defines what needs to be registered and what is best left to the users of it, primarily the interpreters. The dividing line between what is stipulated by the score and what is left open is obviously in constant flux. Given the nature of a written musical score in print, however, the potential for openness is still quite limited in the sense that the content, the symbols and their order, is rarely under negotiation, only the interpretation of them. Hence, the traditional printed score is a preservation archive that structures what kind of music is made possible by it even though the music may require much input from the interpreter.\footnote{Highly poly rhythmic or poly metric music is only with great difficulty notated in traditional notation, and only pitch based music makes sense to notate with traditional notation and traditional symbols.} It creates a system of values that defines the usability of the music as a commodity.

The benefit of looking at the script itself, the musical score, as an archive--normally thought of rather as an entry in an archive--is the way in which it makes the inscription, the grammar and the practice of deciphering the information, stand out as important features of the archive. The system of organization and the writing of new entries are the fundamental building blocks of any archive. In a digital era that registers almost everything we do in our daily life it may be easy to forget about these organizational systems inherent to all archives, and the way they trace our actions. For the same reason it is important to think about how to best organize archival systems that contribute to research in music.

An even stricter and much more detailed archive for music is the digital recording. For each second of recorded music it makes roughly 100.000 entries or more; digital samples encoded into a file of great temporal resolution. The digital, as pointed out by Aden Evens \citep[p. 79]{evens05} is sterile and by itself unproductive until it goes through a transformation ``that draws a line of contact between the digital and the human'' \citep[p. 79]{evens05}. In the end it is technology that mediates between the digital and the actual as the digital is not useful by and of itself. To some extent inspired by Baudrillard (whom we will return to) among others, Evens sees the digital as pure representation, in contrast to the actual in a way that sheds some light on the later discussion of this paper:
\begin{quote}
``Trapped in the abstract, the pure digital operates at a remove from the vicissitudes of concrete, material existence, and this distance leads it its qualities of perfection: repeatability, measurability, transportability, etc. But ths digital's divorce from the actual is also a constraint, denying it and direct power.'' \citep[p. 79]{evens05}
\end{quote}

The great advantage of regarding the digital as an archive is also its biggest inexpediency, and without an efficient interface it is of limited use. Hence, the power of the interface is great. The digital recording is a preservation archive relatively agnostic as to what kind of music it can represent, but with a high degree of specification on a detailed level. It successfully records an entire performance with great resolution, but at the expense of making all events structurally equal. Digital audio is merely bits of data of equal length and incomprehensible to the human eye. Hence, the process of recording music digitally archives according to the power of digital structures, removing the dynamic properties of the performance and fixating it within its own economy of organization. Should the elements be restructured ever so little it would render the recorded sounds unrecognizable.

In a recent project exploring means for the preservation of electro acoustic music, the importance to include the artistic process of creation, along with the artistic output is given some support:
\begin{quote}
  Our approach focuses on the knowledge involved during the creative process, which involves but it not limited to technology. Preservation of electro acoustic and mixed music requires a suitable framework for archiving composer’s idiosyncratic musical software and documenting the work throughout its creative process. It involves archival policies for digital assets that the creative process produces, together with relevant knowledge to ensure meaningful usability [\ldots] therefore enabling re-production of the work. \cite{boutard2012}
\end{quote}
In this article Boutard and Guastavino also give a comprehensive overview on some of the most important projects in the field in the last few decades, including some of the projects mentioned below. For some of these, as well as for Boutard and Guastavino the re-production of the works archived is part of the main purpose. This goal most likely guides the documentation strategies, and may have some impact on decisions made concerning the methods for both documentation and preservation.

Media databases for storing and preserving musical works generally need to be highly structured in order to be useful and in the context of music they can be seen as augmentations to the musical score. In such archives the score is stored along with additional information about the composition and bibliographic data about the creator and the performers etc. In other words, they allow for an extension of the score, commonly as a set of meta-data.

The kind of huge online music listening services provided online, on the other hand, are simultaneous developments, reductions and augmentations, not of the score but of the standard commercial recording formats. They are developments because they introduce a social network layer that connects different kinds of musics together in ways that are only meaningful for large collections of music. The networking layer is similar to a notion of unstructured meta-data as it makes the data more easily accessible. On the other hand these services commonly reduce the amount of information about the recordings contained in them (compared to what is often available on CD covers) and employ a predominant focus of collection at the expense of context and and structured meta-data. Hence, they are reductions of the traditional recording. Finally, to an ever higher degree than scores they explore the value of music as commodity, and due to their scope and popularity they can be seen as economical and political augmentations of the CD-recording.

Common to all of these examples of musical archives, perhaps with the exception of the Research Catalogue is that they preserve the present state or, rather, the state at the time of registration. They are not generally built for recording change over time, such as the development from the definition of the work to its first performance. Since time is an essential part of any attempt to truthfully describe a process, including all descriptions of musical practice and the processes behind, this appears a rather big limitation. How can an archive that includes a notion of time be developed, that connects and successfully records how artistic processes and musical practice unfolds and develops in time?

% The discussion so far, ranging from archives for the preservation of musical works, via the restructuring of the ontology of the musical work, the personal archive and memory as medium leads us into a discussion concerning the writing, impression, inscription and printing of the different capacities of the human psyche introduced by Freud and continued by Derrida. Derrida's deconstruction of the archive in \emph{Archive Fever} is situated in this psychoanalytical framework.
%\section*{Archival strategies, musical scores and open works}
%\label{sec:archival-strategies}


\section*{Documenting openness}
\label{sec:open-work-repetition}
One of the aesthetic tendencies in the 1960s which has an impact on the current discussion was the introduction of chance and openness in musical works. In the famous text \emph{The Open Work} Umberto Eco \citep{eco68} found new tendencies in works by modernist masters Luciano Berio and Pierre Boulez among others. These works introduced the idea that part of the construction of the work was left for the interpreter to do, even in some rare cases in active collaboration with the audience. The more radical version of the open work, seen in the work by Henri Pousseur, is labeled by Eco as a \emph{work-in-movement}: ``It invites us to identify inside the category of `open' works a further, more restricted classification of works which can be defined as `works in movement', because they characteristically consist of unplanned or physically incomplete structural units.'' \citep[p. 22]{eco68} A work-in-movement is a latent, or prospective, possibility rather than a fact, yet to be realized, whose authenticity lies not in the intentions of the composer but rather in the collaboration between the different agents involved in its creation.\citep{eco68,frisk08} %Composing a \emph{work-in-movement} consists of supplying raw material, delivering a potential work rather than a finished one. 

Departing from Eco's reasoning in The Open Work, Swedish guitarist Stefan Östersjö and I developed an artistic method that leaned strongly on the idea of the work as a continuously developing field of possibilities in our collaboration on my composition \emph{Repetition Repeats all other Repetitions}. Originally conceived of as a fairly traditional contemporary piece for instrument and electronics it advanced into what may be called a work-in-movement. The identity of this work is located in change rather than fixity. In short, the composition invites interpreters to create their own version of the work out of an assembly of segments that could be combined in a number of different manners \citep{friskcoessens2013}. In a few articles published early in the process we discuss how our view on the work developed in the process \citep{frisk-ost06,frisk-ost06-2}.

The development of \emph{Repetition Repeats all other Repetitions} coincided with the development of the documentation database for the Integra project mentioned above. It allowed for rethinking both what an appropriate score could look like for a work-in-movement, but also how the process of creating and developing such a piece could be documented. Though the piece originally had a fairly detailed musical score there are a large number of additional data that is of great importance for the reading. Had these segments of data only consisted of written instructions in musical notation the challenge of creating this particular work's documentation may have been slightly easier. However, for a work-in-movement to work as such the documentation needs to contain not only all previous versions and their modes of construction, but also all the different parts in terms of electronic sounds (sound files, software for interaction, DSP processes for altering the acoustic sounds, etc.). As the concept of documenting all past performances and all related data first surfaced, the key, it appears, lies in finding methods to defy the propagating level of noise as the archive grows bigger.\footnote{Noise, of course, is also the result when ``all'' music becomes available in enormous online databases.} Despite these concerns the idea of a documentation database for the piece appeared as a sensible solution. Although the database developed in the Integra project was mainly designed for the preservation of works, as a starting point it turned out to be apt also for the current context. After a few initial proofs of concept the database ArtDoc has slowly been developed to provide the context for \emph{Repetition Repeats all other Repetitions} and it is still in a state of experimentation.\footnote{I have developed the database from the work done in the Integra project and in a forthcoming paper for the CMMR 2017 conference in Porto a technical description of its structure is given.}

%Looking back at the discussion concerning \emph{Repetition Repeats all other Repetitions}, where the idea of 
 %As a result we see a huge market for those that attempt to create structure out of the digital pandemonium.

% To create structure where there is none has been very lucrative for the big internet search companies and similar ventures that gather loosely structured information and present it in user friendly ways. We can look at these companies as the archaeologists of contemporary media. They do not need to establish the precise origin of their findings as long as the information is valid. MIR, or Music Information Retrieval, is a related field of research for which one of the goals is to automatically create meaningful semantic structure in large collections of musical data.

% In the following section, however, I will discuss approaches that may allow the documentation of rehearsal data, performer interaction, gestural data, listener interaction and many other kinds of data well aware that this storage may preserve rather than develop knowledge.
% \section*{Archiving in practice}

\section*{The personal archive}
\label{sec:archiving-practice}
Today the digital is almost ubiquitous. The attempt to document musical practice almost exclusively also involves a transformation to the digital realm at some point - both in the ways that the actual process is encoded, and in the way it is archived and meta-data is applied. If previously the question of what to archive was discussed we now move towards the question of \emph{how} to document and preserve. The importance of both of these two phases is described by Walter Benjamin in the short essay \emph{Excavation and Memory} from 1932 with metaphorical references to archaeology and psychology:

\begin{quote}
Language has unmistakably made plain that memory is not an instrument for exploring the past, but rather a medium. It is the medium of that which is experienced, just as the earth is the medium in which ancient cities lie buried. He who seeks to approach his own buried past must conduct himself like a man digging. Above all, he must not be afraid to return again and again to the same matter; to scatter it as one scatters earth, to turn it over as one turns over soil. For the 'matter itself' is no more than the strata which yield their long-sought secrets only to the most meticulous investigation. \citep[p. 576]{benjamin2005}
\end{quote}

\noindent
It is through repetition and conscientious analysis of findings that the layers of memory and the subconscious (the psychoanalytical dimension of Benjamin's text is impossible to ignore) are possible to excavate properly, but the very essence of the findings lies not in the thing found, but in the 'soil' where they ware first found. When one's memories are probed again and again one is eventually able to understand the findings anew; the secrets are decoded and become clear, as images that are sliced off from all earlier association. To Benjamin, these are the ``treasures in the sober rooms of our later insights.'' \citep[p. 576]{benjamin2005} The way time is brought into the discussion is interesting. At first, we may or may not understand what we experience or remember, or even the meaning of what we find in our excursions, but through a complex process of dissociation and re-association time will distill the findings and separate the meaningless from the usefully consequential; the real treasures.

But the findings themselves are not enough, the enhanced awareness that only experience and knowledge can give us are also necessary, and the continuous and cautious ``probing of the spade'' is equally important in the process of uncovering. ``And the man who merely makes an inventory of his findings, while failing to establish the exact location of where in today’s ground the ancient treasures have been stored up, cheats himself of his richest prize.'' \citep[p. 576]{benjamin2005} In other words, the dynamic between what has been found and the location of this finding is of significance. The ground that I open, that unravels my findings, is part of the information that can make me understand what I have found. In that sense, as a consequence, also the person who finds plays an important role: ``genuine memory must [\ldots] yield an image of the person who remembers.'' \citep[p. 576]{benjamin2005}

This short text by Benjamin gives an insight in his own obsessive archiving of his own practice.\footnote{Much of this following is based on the documentation in \citet{benjamin2007}} Benjamin's personal archive is particularly interesting in the way that it appears to document not only what he worked on, and materials related to books or essays that he was writing, but also what he did in general. Not only lists of books that he read for a particular project, but all books that he read or had ever read. Not only correspondence with individuals that he worked with, but all kinds of correspondence, including lists of expressions that his son invented and similar personal details. The lists, written on small archive cards or sheets of paper, were continuously edited by entries scored out and notes added. He meticulously and apparently tirelessly archived virtually everything that he came across constructing an archive that is difficult to decipher, at best, but also one that is conceptually ahead of its time. It is through this project, his own archive, that his ideas on memory and excavation must be understood.

An archive of musical material most often archives representations of musical content, and to be meaningful it has to go beyond a mere collection of resources. An archived score is relatively easy to represent accurately but is a poor representation of the actual music. A recording of a performance is an accurate representation of the sound but a poor representation of the material performance. Furthermore within the category of musical recordings, there is a continuum between the concert documentation and the edited recording. Though both are reductions of the materiality of the performance to an archivable documentation/representation format the concert recording may be seen as the more authentic representation of a performance than the edited recording. How, then, may an archive of musical content be structured so that we are not cheated of the richest prize, as Benjamin puts it? How may we create systems for collecting musical knowledge and experience, that can be explored and excavated for new meanings? 


%How can we create archives 
%, or more, which will provide a loose theoretical framework to the discussion in this paper.
%Though this will constitute part of the theoretical frame for the discussion, the method employed will depart from my own artistic practice: the musical materiality discussed here will be that experienced by me in performance. 

% The hypomnesic archive is a place for collecting and storing thoughts, knowledge: making inscriptions that may later be revisited. As such it is a tool of reflection, of turning back and reconsidering concepts formed at an earlier stage.


\section*{The scene of writing}
\label{sec:scene-writing}

In \emph{Archive Fever: A Freudian Impression} (Le Mal d'Archive), a late and read worthy text on the nature of the archive written in the beginning in the beginning of the internet era, the French philosopher Jacques Derrida, to little surprise given his critical method, points to the dangers of the structures of the archival process. The book is based on a lecture given in London during an international colloquium on ``Memory: The Question of Archives'' in 1994. This was in the early days of electronic communication but the text is strikingly clairvoyant of what we now know has become the impact of digital technology. Much of the content departs from a much earlier, but related text: \emph{Freud and the scene of writing} \citep{der78}. Before pursuing the discussion concerning the nature of the archive I will go over some of the central arguments that Derrida makes in this earlier text with regard to Freud's concept of perception and memory as these may prove to be useful here. The following is not intended as a complete analysis of this rather complex text, but should mainly be regarded as an attempt to unpack some of the central concepts used in \emph{Archive Fever}.

Derrida addresses Freud's rethinking of \emph{writing} in \emph{Freud and the scene of writing} as a metaphor for accessing the possible relationships between perception, memory and experience; for understanding how the outside world is registered within the psyche. By looking at psychical content as something that is the result of different kinds of writing, a writing that is, according to Derrida, in its very essence, graphic, he draws a picture that clearly separates internal experience from external storage.
%Furthermore, and as a consequence, the psychical apparatus itself will here be represented by a device that prints onto one's consciousness:
%As a writing machine.

% So far the discussion has primarily followed the trajectory layed out in the two texts by Derrida concerning writing in general. Verbal representation is as central to Freud as textual is to Derrida. Hence, how can this discussion be translated to the field of music?

From the short text \emph{A Note upon the “Mystic Writing Pad”} \citep[Included in:][p. 207-12]{freud1997} Derrida,  construes three different analogies concerning the relation between writing and perception, each with increased levels of introspection and contrast. The metaphor of writing is here to be understood as simultaneously appropriating ``the problems of the psychic apparatus in its structure and that of the psychic text in its fabric.'' \citep[p. 259]{der78} Considering recent developments in the fields of cognition and ecological psychology the following discussion should be regarded as a philosophical elaboration on the metaphor of writing and, as such, how it impacts on the notion of the archive.

The first analogy is similar to the most obvious understanding of the word \emph{writing} and concerns writing as a material expression of something stored in one's memory. A means to jot something down on a piece of paper, forget about it and later revitalize the memory by way of the writing. Ideally this kind of writing should satisfy both the need for unlimited capacity and indefinite preservation: rather similar to the obviously impossible demands we place on current archives for research data for example. It is writing as externalization of memory, a representation of memory outside of the psychic apparatus and separated from it by perception.

\emph{The Mystic Pad} is a small device that Freud describes as ``a slab of dark brown resin or wax with a paper edging; over the slab is laid a thin transparent sheet, the top end of which is firmly secured to the slab while its bottom end rests upon it without being fixed to it.'' \parencite[p. 209]{freud1997} The method of writing on the pad is similar to scratching in wax or clay with the addition that the surface is easily wiped clean again and made ready for new impressions. According to Freud the Mystic pad can be seen as a model of the perceptual apparatus: it records but leaves no visible traces and can be said to be infinitely ready to receive anew. The writing on the pad is an illustration of the second analogy of writing and corresponds to the writing on the body, the first imprint on reception, but before it is registered by consciousness. This writing ``supplements perception before perception even appears to itself.'' \citep[p. 282]{der78}

If the two first analogies have been related to the space of writing, the third is to a significant degree concerned with the time of writing. Time as the consequence of the distribution of symbols and impressions, and as the result of the various strata in the psyche. Time as in memories being forgotten, re-remembered and re-inscribed, or, in the words of Freud, as ``cathectic innervations'': ``as though the unconscious stretches out feelers, through the medium of the system Pcpt.-Cs., towards the external world and hastily withdraws them as soon as they have sampled the excitations coming from it.'' \citep[][p. 211-2]{freud1997} This discontinuous functioning of the perceptual system, according to Freud, ``lies at the bottom of the origin of the concept of time.'' \citep[p. 212]{freud1997}

Derrida's reading of Freud is multifaceted, and the brief discussion here only touches upon a fragment of its prospect. As a deconstruction, however, \emph{Freud and the scene of writing} would be incomplete if it did not also critically examine Freud's conclusions. One of the critiques brought forward does have some loosely connected impact on the topic of materiality and effectively points to a questioning of the relation between spontaneous memory, and the pure absence of spontaneity observed in the machine: Freud does not ``examine the possibility of this machine, which, in the world, has at least begun to \emph{resemble} memory and increasingly resembles it more. Its resemblance to memory is closer than that of the innocent Mystic Pad: the latter is no doubt infinitely more complex than slate or paper, less archaic than palimpsest; but compared to other machines for storing archives, it is a child's toy.'' \citep[286-7]{der78} The archive is a representation of this machine referred to by Derrida and the question of what the relation between an external writing and the internal experience may be is at the center of the current discussion.

This rather long overview of what may, or even should, be considered a logocentric view of the human psyche\footnote{Derrida is sometimes accused for logocentrism--sometimes erroneously or for the wrong reasons--but in the beginning of \emph{Freud and the scene of writing} he actually makes the claim that logo-phonocentrism is ``rather a necessary, and necessarily finite, movement and structure: the history of the possibility of symbolism \emph{in general}.'' \citep[p. 247]{der78}} is included here to provide a frame for the central discussion of this paper: how can the material aspect of music making, of the artistic practice in music, be understood, and what is the relation between this practice and any effort of providing an archive for it? Considering the three analogies of writing above, could these be translated to the realm of music? Are they meaningful in relation to the nature of playing--playing should here be understood in the widest sense of the word--music? Before engaging in the attempt to re-contextualize them it is important to identify that the metaphor of writing, though similar to a specific practice like musical composition, in many respects has fundamentally different properties compared to playing music in general, and these need to be considered. These differences partly amount to the organization of the underlying structure of respective practice, but also to how \emph{writing} easily misleads us to think about production of a text that stands by itself,\footnote{However, Derrida's deconstruction of Freud's short text on the \emph{A Note upon the “Mystic Writing Pad”} has speech much more than text as its point of arrival.} whereas the physical traces of music can mainly be heard, and thus experienced.

If first we consider the analogy of writing as an externalization of a memory, the first thing that comes to mind is composition which can be said to be a system for remembering music by means of making notes, inscriptions on paper. For Freud this note is a ``materialized portion of my mnemic apparatus'' \citep[p. 207]{freud1997}, but it would be a reduction of what we think of as music to limit its exteriorization to musical notation. Instead, the first analogy of writing in the context of music should include all kinds of musical activities that emanates from the mind or memory, and results in a physical trace of some sorts: singing in the shower, improvising, performing an interpreted composition, composing a score, composing electronic sounds, etc. This would require us to rethink what writing means: what is referred to here is a sort of writing onto the world.

Secondly, the analogy of a writing that leaves no permanent traces may be seen to correspond to perception as a system separated from the memory. Thinking of memory as a physical storage unit (which of course it is not) it is easy to imagine that it can run out of space, that we reach a point where we cannot remember anymore. In contrast, our perceptual apparatus, unless struck by illness, has almost unlimited capacity. We can listen and listen, in fact we cannot \emph{not} listen. But not only the ears perceive music, the entire body is written when it is struck by the vibrations from sound waves. The writing on the body, on perception, however, is elusive and disappears immediately which is why it is also always ready to be written anew.

Finally, related to time the third and most profound analogy of writing is the one most easily translated to music and playing. Derrida's claim that ``time is the economy of a system of writing'' \citep[p. 284]{der78} is effortlessly transformed to a musical context. Beyond its immediate understanding the analogy is based on the way Freud imagines that there is a movement from within the psyche that reaches out towards the external world and the discontinuity involved in this process is what introduce time. There are a number of interpretations of this that would make sense in material musical practice. The continuous act of intonation performed by a musician, the subtle adjustments made in performance against a memorized piece of music or a tonal system, the temporal adjustment to a pulse, or the improvisers sensibility to the co-musician's activities are all examples of ``cathectic innervations'' in the musical practice.

Even if some of these analogies of musical writing also applies to musical listening, or even listening to sounds in general, I would claim that they are all important aspects of the materiality of artistic practice in music. One's writing of music, the writing of music onto ones body and the (subconscious) reflective, discontinuous impulses reaching out into to external world, are all integral parts of a broad view of musical practices. The question now is not if any one of these modes or analogies may be archived without any central aspect of them being lost because the immediate answer to that question is of course no. It is yet unimaginable to store experiences independently of the one experiencing, though being able to do this is the theme for many science fiction stories. The more accurate question to pose would be along the lines of: what is the relationship between the live experience of a musical practice such as it is described in terms of analogies of writing and possible representations and documentations of these activities in various forms for archives? Another, related question, that will be brought up in the next section is: in what ways may our different systems of representation (archives) affect our internal systems of writing?

\section*{Writing the past}
\label{sec:archive-fever}

I will now return to \emph{Archive Fever} where the dualism between exerience and cognition on the one hand, and the hypomnesic nature of the archive and its exteriority on the other, is a central topic. Derrida repeatedly comes back to how the external writing of the archive results in an, in every respect, different record than that of the recording of the perceptual apparatus: ``Because the archive [\ldots] will never be either memory or anamnesis as spontaneous, alive and internal experience. On the contrary: the archive takes place at the place of originary and structural breakdown of said memory.'' \citep[p. 11]{derrida1998} This could be understood from the rather direct and simple point of view that relying on an external source of memory restructures one's own, internal memory. Given the information surplus in the digital era, and the extent to which we commonly rely on auxiliary memory devices this somewhat susceptible relationship, I believe, may be recognized by many. Though many of the available technological devices, services and functions that surround us at times appear to be working akin to how internal memory processing functions, conceptually there is an irrevocable dividing line between how the inside and outside is structured: ``There is no archive without a place of consignation, without a technique of repetition, and without a certain exteriority. No archive without outside.'' \citep[p. 11]{derrida1998} Derrida returns to this separation many times in the text and as was pointed out above, already in \emph{Freud and the Scene of Writing} it was a guiding principle: ``The machine--and, consequently, representation--is death and finitude within the psyche.'' \citep[p. 286]{der78} Perhaps the uniqueness of the materiality of artistic practice in music may allude to the antithesis of the archive given by Derrida: ``anamnesis as spontaneous, alive and internal''? The quality that separates it from any representation of the said practice?

The dualism of the two systems, the technically oriented, external archive, and the internal memory and human experience, does not restrain them from evoking influence on each other, at least not in abstract ways. As a consequence also this seemingly definite relationship is deconstructed. Because as clear and permanent the separation between inside and outside may be, the extrinsic archive nevertheless projects its characteristics onto its users. The structurality of the writing of the archive disallows it from being a neutral site for recording, if such an idea ever existed. It acts as a political as well as an economic force upon that which it records:

\begin{quote}
  It is thus the first figure of an archive, because \emph{every} archive, we will draw some inferences from this, is at once \emph{institutive} and \emph{conservative}. Revolutionary and traditional. An \emph{eco-nomic} archive in this double sense: it keeps, it puts in reserve, it saves, but in an unnatural fashion, that is to say in making the law (\emph{nomos}) or in making people respect the law. (p. 7)
\end{quote}

More specifically, the archive does not only record but also conditions what may be written through its control over the structurality of the writing: ``[\ldots] the technical structure of the archiving archive also determines the structure of the archivable content even in its very coming into existence and in its relationship to the future. The archivization produces as much as it records the event.'' \citep[p. 17]{derrida1998} Hence, even the \emph{wish} to archive and to make content accessible in a structured format creates delimitation and determined articulations that exclude as much as it makes available.

Derrida goes on to make the observation that if technology is an archivization process that produces as much as it records, this  ``means that in the past, psychoanalysis would not have been what it was (no more than so many other things) if E-mail, for example, had existed. And in the future it will no longer be what Freud and so many psychoanalysts have anticipated, from the moment E-mail, for example, became possible.'' \citep[][p. 17]{derrida1998} One may want to argue against this considering what we now know about electronic communication, and the example of Freud obviously has a particular meaning considering the psychological dimension with a certain focus on the suppressed and unconscious. But on the other hand it is safe to assert that the impact of communication technologies on all aspects of society and culture has been anything but insignificant. Hence, not only does the archive to some extent determine what may be archived, it also influences what is written and in what ways it is written. For the current discussion it is worth considering how the impact of all different kinds of archiving systems employed may affect both artistic and research practices.

Finally, one other important aspect of the present day archive fever needs to be considered. In the relentless wish to reposit potential experiences in easy to handle, downloadable packages, we miss out not only on the physical artifacts, the origin of the experience, that are not as easily encoded, the focus on the past rather than the future, risks at becoming the driving force. The archive persistently points to history, institutive, yes, but also conservative: ``And the word and the notion of the archive seem at first, admittedly, to point toward the past, to refer to the signs of consigned memory, to recall faithfulness to tradition.'' \citep[p. 33]{derrida1998} While this may be self evident, and to some extent the very purpose of many of the archiving initiatives of today, such as some of the preservation initiatives mentioned earlier, in the section \emph{Documentation databases} \ref{sec:docum-datab}, is to designate the past and preserve a tradition that may otherwise be lost. The principle argument for doing this is not that the contents of the archive itself should  allow for development, but that the availability of it triggers activities that points ahead and bridges the gap between past, present and future.\footnote{Later in the text Derrida points out how the archive also satisfies the future, as a ``question of the future, [\ldots], the question of a response, of a promise and of a responsibility for tomorrow'' \citep[p. 36]{der78} but does so only in an enigmatic sense.} Part of Derrida's argument, however, is that the relations between the inner and the outer modes of writing may not be as yielding as they may first appear. Unreflectedly relying on the archive may alter the view on what should actually be archived and how it should be structured. As was discussed above, this archival activity may under certain circumstances also modify how the process to be archived develops over time. 

The danger of the metaphor of writing the way it has been used here is its tendency to focus on an individual act of writing, as in \emph{one writer} writing \emph{a narrative}. The different kinds of writing discussed her, however, hopefully counteracts this leaning and instead points to the multiplicity of possible modes of writing as is a key element. Derrida writes, concerning the Mystic Pad, that ``the subject of writing is a \emph{system} of relations of strata.'' \citep[p. 285]{der78} This is significant in relation to ArtDoc as its most important data is not the documented entries themselves, the sound file or the score, but the system of relations between these. Furthermore, the same content could be given a different relational structure rendering a different version of the data. In other words, in ArtDoc it is truly the system of relations of strata that is the subject of writing, rather than the author of one of the entries in one of the stratum in the archive.

\section*{Discussion}
\label{sec:discussion-1}

It is impossible to be in control of the archive. Its logic is so substantially different from human experience and it is only through a thorough understanding of the difference, of the lack of concept of the archive, that it is possible to bridge counteract the consequences of the differences. Understanding that the structure of the archive participates in writing the content makes it possible to design systems in ways that reinforces the intended results rather than disguises them. If we know that the archive will always have the tendency to point to the past rather than the future, also this can be used both to avoid this to influence the results and the data but also in order to design systems that are prone to change over time. The risks, however, are still that we mistake the representation for the real. A few years later than Derrida, Jean Baudrillard, in the essay \emph{The Automatic Writing of the World}, takes a typically dystopic point of view on the relationship between what he calls the real and the double, or the real and the representation. He gives some support to the notion of the incompatibility between the inner and external, the real and the virtual, anamnesis and hypomnesis:

\begin{quote}
  The perfect crime is that of an unconditional realization of the world by the actualization of all data, the transformation of all our acts and all events into pure information: in short, the final solution, the resolution of the world ahead of time by the cloning of reality and the extermination of the real by its double. \citep[p. 25]{baudrillard96}
\end{quote}

The idea that everything is archivable, everything may be documented and everything may be verified is so second nature today that it is easy to mistake the representation for the real thing. Even if an experienced listener can reengage many of the properties lost in a recording of a concert the representation only encompasses a fraction of the information available to the listener of the performance. The transformation of music from something experienced in a live performance to something which may be contained, packaged and transmitted has clearly been fueled by the commodification of music. It has also participated, I would claim, in the development of the musical work concept. If the work is defined by its archivable traces--its score and its recording--it becomes self contained and is much more easily distributed and legally protected. On the other hand, if the work is defined by its continuously changing artistic, social and political interactions through the work's interrelations between other works and its own contents there will be a greater chance that its dynamic properties are seen as essential. An open work definition and a multidimensional system for storing the data of the work was one of the aims that grew out of \emph{Repetition Repeats all other Repetitions}, and beginning to develop ArtDoc was an attempt to provide the structure, a method and a complement to other forms of documentation.

%Specifically  of writing comes to mind. 
Although the discussions held by Benjamin on the one hand, and Derrida on the other, are very different in scope and focus, they also converge at interesting points. The the various analogies of writing, of perception as a form for writing, brings to mind Benjamin's poetic claim that true memory must ``yield an image of the person who remembers.'' \citep[p. 576]{benjamin2005} On the surface the archive as an externalized memory machine obviously does not allow for the person who remembers unless the machine allows its memory device to be continuously reconfigured. This reconfiguration is however possible to mimic if the focus is moved from the structurality of the data to  the relations between data entries, between the strata of data. Benjamin also reminds us of the importance to keep in mind the person \emph{reading} and not only the process of writing, although this may coincide with the subject of writing. The idea that he put forth, that memory is the medium rather than the instrument for exploring the past, perhaps allows us to look at the archive, not as a storage container that can be read from and written to, but instead as a medium that allows for a subject to read and write, creating a multilayered system of relations of strata. Not without a significant effort though and with a resulting depth that may be difficult to disentangle. But as Benjamin concludes, only to the most meticulous digger will the matter itself, the ``richest prize'' \citep[p. 576]{benjamin2005} reveal itself.

\printbibliography
\end{document}