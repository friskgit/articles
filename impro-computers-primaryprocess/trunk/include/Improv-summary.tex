\section*{Summary}
\label{sec:introduction}
Why would someone want to use a computer in which physical contact
with the sound is virtually impossible, to perform free improvised
music, that thrives on expressive sensibility? The computer has no
'natural' way of controlling its sound synthesis processes; its
interface is non-intuitive. In contrast to most other physical musical
instrument the performer has no physical contact with the sounds
produced.  Furthermore, the computer holds no unified notion of what
the computer-as-instrument, consist of. But despite its shortcomings
the computer is an increasingly common 'participant' in contemporary
free improvised music. Precisely the improvisatory \emph{sensibility}
is of import: master improvisers such as Charlie Parker or John
Coltrane are aural representations of the full blown complexity of
human sensitivity in general and improvisatory sensitivity in
particular. While the unconscious may be said to be structured in
terms of primary process (and the conscious in terms of the logics of
language)\footnote{\cite{bateson72}} I argue that the improviser's
\emph{personal narrative}\footnote{\cite{lewis-1}} as a reflection of
his or her \emph{unconscious}, constitutes an interface between these
two different aspects of consciousness. Furthermore, according to
Bateson, though the \emph{fact} of 'skill' is conscious, the
sensations and qualities of it will always be unconscious. Is 'skill'
then a way to get in touch with thew unconscious? Perfection of skill
is the formation of habit and habit, according to Bateson, is a major
economy of conscious thought. But, turning to the American saxophonist
and jazz musician Ornette Coleman as an example we can note that,
rather than perfecting his skill, to him it was of interest to
\emph{increase} resistance---i.e. reduce skill---in his practice by
embracing instruments he wasn't trained to play (the violin and the
trumpet). Visual artist Marcel Duchamp similarly tried to consciously
``forget with [his] hand'' as he expressed it. In their own ways, both
Duchamp and Parker have consciously attempted to oppose habit in order
to more fluently be able to speak the language of the
unconscious. This, I argue, may be one of the reasons musicians choose
to play the computer in the context of freely improvised music, despite
its shortcomings: Because it increases resistance and resists habit
formation.

%%% Local Variables: 
%%% mode: latex
%%% TeX-master: "../Improvisation"
%%% End: 
