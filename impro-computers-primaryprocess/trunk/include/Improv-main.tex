% Resistence, control and freedom.

\section*{}
\label{sec:introduction}

Improvisation is a difficult matter. As a concept it is attributed to
a wide range of activities and it has positive as well as negative
connotations. One general understanding is that improvisation is a
substitute for organization and pre-meditated actions: \emph{I haven't
  had the time to prepare for this talk, I'll have to improvise}. The
ethnomusicologist Bruno Nettl concludes that ``[i]n the conception of
the art music world, improvisation embodies the absence of precise
planning and discipline'' and points to the racist implications of the
musical equivalent to the above statement and asks whether
``[i]mprovisation as the music of people who don't plan ahead'' is the
``white musical world's way of expressing a racist
ideology?''\footnote{\citet[p.7]{nettl98:2}} Whether racist or not the
understanding of musical improvisation as a method used as a
'substitute for preparation' has influenced areas in which the
consequences are wide ranging and go cross ethnic
boundaries,\footnote{In Sweden jazz- and improvising musicians are
  generally payed only 1/3 of what their classical chamber music
  colleagues are payed. One of the motivations behind this asymmetry
  sometimes quoted is that classically trained musicians need to
  practice (obviously implying that improvising musicians need not).}
and there has been a long academic and social tradition to discredit
improvisatory musical expressions in favor for text based,
pre-organized and reflected artistic output: ``So within the realm of
art music, improvisation is on a low rung, just as musics outside the
realm of art music are often associated with the inferior practice of
improvisation.''\footnote{\citet[p. 9]{nettl98:2}. See also
  \citet{lewis-1,bailey92}} I believe the first part of Nettl's
statement is slowly beginning to change; improvisation is discussed
seriously in several fields of musical practice that belong to the
`art music world'.\footnote{For a recent PhD thesis on organ
  improvisation practices, see \citet{johansson08}.}But the point
Nettl makes that has impact with regard to the present discussion is
how the distinctions made between planning, organization and structure
(composition) on the one hand and freedom and ambiguity
(improvisation) on the other are not so easily made: ``discipline,
intricacy, and control of complexities all play major roles in various
kinds of improvisation, as in Indian music with its detailed rules for
proceeding, or in organists' practice of improvising fugues on given
themes, to mention only two. And yet Western classical musicians are
more inclined to see the quality of improvisation as emotional rather
than intellectual, as free rather than
controlled.''\footnote{\citet[p. 10]{nettl98:2}} So, according to
Nettl, some qualities that are usually associated with
non-improvisatory musical practices---preparation and control---are
indeed inherent to improvisatory practices. But despite this, despite
the high level of discipline in some improvised music, it is generally \emph{perceived} as non-controlled and
free (because it is improvised?). However, the point I will put
forward here is quite the contrary, that non-discipline,
non-preparation and abandoning control of complexities are integral
aspects of some kinds of improvised music and something which is
strived for by some musicians and artists and the issue is much more
complex than the 'preparation' and 'no preparation' dichotomy. 

In this essay I will attempt to show that lack of detailed control and
preconception may in fact be an important aspect of some jazz and
freely improvised music,\footnote{This is the term used by Derek
  Bailey \citep[p. 83]{bailey92} but, as he points out, the identity
  of the style suffers from some confusion. Bailey refers also to
  terms such as 'total improvisation', 'open improvisation', 'free
  music', 'improvised music': ``It is a logical situation: freely
  improvised music is an activity which encompasses too many different
  kinds of players, too many different attitudes to music, too many
  different concepts of what improvisation is, even, for it all to be
  subsumed under one name'' (Ibid)} and I suggest that the various
ways in which computer technology is used in these genres in some
cases may be looked at as part of a methodology whose primary goal is
to 'un-learn' idiomatically and stylistically bound musical
knowledge. To understand and unwrap these issues I will make use of
the conscious/unconscious continuum based on the theories of
anthropologist Gregory Bateson and, just as Bateson, I will look at
the notion of \emph{primary process}.\footnote{Somewhat simplified, in
  Freudian language the primary process is the operation of the
  unconscious: Dreams are ``metaphors coded according to \emph{primary
    process} \citep[p. 135]{bateson72}} In addition I will be
discussing the practice of a few artists, both from the point of view
of how they describe their work, but also in relation to the idea that
freely improvised music is an activity that mainly takes place in the
unconscious.

\subsubsection*{Computers and improvisation}
\label{sec:comp-as-impr}

By some (all?) standards computers are poor improvisers. We can
program computers to calculate planet trajectories and to help us find
patterns in extremely large data spaces, but one of the more difficult
things to model or imitate in a machine is the human ability to
spontaneously act upon stimuli. Although current computer science can
build machines that are able to decode complicated real world patterns
and---within a limited frame---construct viable reactions to these
inputs, in many cases it is more efficient for the machine, and
certainly for the programmer, to evaluate \emph{all} possible
responses in a simulation and make a choice according to a set
rule. The quantity ``all possible'' is in this case a function of the
accuracy of the sensory interface of the computer. Rarely would a
human musician use such a method. Infinitely more economic for a human
improviser is to pick one possible alternative based on
\emph{intuition}. I.e. what is in many cases the easiest alternative
for a human---to act based on experience and empirically receive the
impact of the action---is the most difficult for the
machine/programmer. The human capacity for making ``good'' choices
based on intuition is enormous and as a method it allows us to
approach the ``conditions of experience''.\footnote{\citet[p. 27]{deleuze88}}

If the notion of intuition or spontaneous response to stimuli is
difficult to model in a computer, even more difficult is it to attempt
to simulate truly spontaneous action, such as
improvisation. Improvisation, and the building blocks that constitutes
it, may not occur as a response to some (external) stimuli. Think of a
solo improvisation such as \emph{Line 1} of Evan Parker's
\emph{Lines Burnt in Light} CD.\footnote{\citet{parker01}} This, the first
track of the CD, is recorded before the audience arrived so there are
no co-musicians and no immediate listeners. Yet, the music is
flowing. How would a computer program have to be constructed in order
to produce beauty in the way Parker does, let alone \emph{any} music,
in that same situation?

So much for the computer as a \emph{player}; as a machine that
attempts to mimic the human capacity to improvise. The more common use
of the computer in improvised music however is as an
\emph{instrument}, as an extended synthesizer. In this capacity the
issues discussed above are partly resolved as the function of the
computer is merely to generate sound at the discretion of a performer,
much like any other musical instrument. The human 'computer-player'
becomes the interface between the machine and the world. But the
instrument paradigm comes with its own set of challenges that needs to
be addressed. Firstly we need to consider the intricacies of computer
sound synthesis. A computer does not have \emph{a} sound but rather
comprises the possibility of (nearly) \emph{any} sound. At first this
may seem like a marvelous thing---\emph{any sound at the tip of my
  fingers}---but, in fact, it is equally likely that this becomes a
creative hindrance. In the musical example above, Evan Parker can
\emph{play} with the contextual pre-understanding of what a soprano
saxophone 'should' sound like, and, at will, deviate from this. If
\emph{any} sound is possible there is no tension between expectation
and production. Secondly, to say that \emph{any} sound is possible is
not quite true: A \emph{representation} of any sound is possible,
but---and this is true for many synthesis techniques---the kind of
minute variation and dynamic change that constitute the very notion of
a musical sound is still difficult to achieve on the computer. This is
a programming challenge, a need to further develop synthesis
techniques, but it is also a question of the interface between
musician and computer, which brings us to the third point: The
interface, or controller. The experimental music keyboard player and
audio artist Bob Ostertag gives a striking account of the issues
involved in his article \emph{Human Bodies, Computer Music}: ``The
problem [\ldots] still is how to get one's body into the [\ldots]
performance.''\footnote{\citet{ostertag02}} When playing the computer (and
this is true for most electronic and electro-acoustic instruments) the
player has no physical contact with the sound. In the case of the
laptop, there is not even a meaningful mapping between the keyboard
and any kind of sound production. The computer musician needs
``controllers'' with which the performer can interact, not with the
computer, but with the sound. But, as Ostertag points out ``[d]espite
years of research and experimentation [\ldots] there is still no new
instrument sufficiently sophisticated to allow anyone to develop even
a rudimentary virtuosity with it. I believe that this failure is
rooted in the premise that the problem lies in inadequate
controllers. The bigger problem is this: What exactly are we going to
control with these controllers we would like to
invent?''\footnote{\citet[p. 13]{ostertag02}} By posing the last question
Ostertag moves to the next level of this discussion: Once we have a
sound synthesis that has a reasonable range of parameters and a
controller with which we can interact with and control our computer
instrument, we still have not solved the problem of mapping a physical
movement to a parameter in the software. Furthermore, for a
traditional instrument some of these mappings are not static but
subject to change over time which implies that, for our computer
instrument, we need a separate piece of software that knows about the
controller and knows about the synthesis and, preferably, has access
to some information about the performer. Ostertag summarizes the
problem: ``If I had some really wild controller that doesn't exist now
but that I could dream up---such as a big ball of a mudlike substance
that I could stick my hands into, squeeze and stretch, jump up and
down on, throw against the wall and wrap around my head, resulting in
a variety of parameter streams that would be seamlessly digitized and
fed to the computer---even if I had such a thing I don't know how I
would use it.''\footnote{\citet[p. 14]{ostertag02}}

The dreary picture painted here makes the computer look like a pretty
useless instrument for improvised music. And yet, the computer is most
definitely an agent in the development of experimental improvised
music. Despite its obvious shortcomings as a musical
instrument---non-intuitive (musical) user interface, no physical
contact with the sounds produced, no unified notion of what the
computer-as-instrument consist of, etc---there is a number of
musicians willing to challenge its limitations. Why is that? Why would
someone want to use a machine in which physical contact with the sound
is virtually impossible, to perform a music that thrives on expressive
sensibility? Before examining these questions concerning the validity of
the computer as a instrument for improvisation, I will give a personal
reflection on the significance of sensibility in freely improvised
music. As contradictory as it may seem, I believe that it is important
to more fully 'understand' the incomprehensible nature of the
unconscious processes that are central to improvisation, in order to
understand why the computer may appear as an interesting instrument in
this context. We will see how strategies employed by some artists may
have influenced some groups of musicians to embrace the computer, as
well as other electronic instruments.

\subsubsection*{Improvisation and sensitivity: The
  conscious/unconscious divide}
\label{sec:impr-sens}

To listen to master improvisers such as Charlie Parker or John
Coltrane is to experience an aural representation of the full blown
complexity of human sensitivity. In these and other cases of great improvisers,
uncompromising individuality and perfectly synchronized group
collaborations are not contradictions but co-exists in a dynamic
relationship. The mere presence, force and density of some of their
best performances, engraved into the vinyl of the LPs, makes them truly
``gorgeous artifacts not even deaf people should miss...'' as the poet
and writer LeRoi Jones wrote on the cover of Coltrane's
Ascension.\footnote{\citet{leroi65}} The refinement in interaction,
instrumental technique, sound and musicality is so intimately coupled
in the expression that other possible representations, such as notated
transcriptions, would only result in reductions so crippled as
rendered meaningless. It is significant that LeRoi
Jones\footnote{a.k.a. Amiri Baraka} uses the \emph{deaf} and the
notion of an \emph{artifact} in his poetic description of Coltrane's
music. The music, so tangible and irrefutable it renders artifacts, so
real that even those who cannot hear can sense them. The spontaneous
sensibility at play here, conveyed by these and many other great
musicians in concert and on recordings is what George Lewis has
labelled \emph{afrological} sensibility,\footnote{\citet{lewis-1}} and George
Russel has referred to as \emph{intuitive
  intelligence}.\footnote{George Russel, in a conversation with
  Ornette Coleman \citep[see][]{ornette85}, concludes that the reason
  Coleman and the members of his band are able to start playing, in
  time, without counting the tunes off is thanks to ``intuitive
  intelligence'', according to Russel a property of African-American
  culture. In an interview with Ingrid Monson Russel returns to
  intuitive intelligence giving the following description: ``It's
  intelligence that comes from putting the question to your intuitive
  center and having faith, you know, that your intuitive center will
  answer. And it does.'' \cite[George Russel quoted
  in][p. 154]{monson98}} Lewis discusses how the ``personal narrative''
is a part of the individual signature of the Afrological improviser
summarized by the conception of his \emph{sound}.
\begin{quote}
  Moreover, for an improviser working in Afrological forms, 'sound,'
  sensibility, personality, and intelligence cannot be separated from
  an improviser's phenomenal (as distinct from formal) definition of
  music. Notions of personhood are transmitted via sound, and sound
  become signs for deeper levels of meaning beyond pitches and
  intervals.\footnote{\citet[p. 117]{lewis-1}}
\end{quote}
By listening to the diversity of expression to be found in tenor
saxophone contemporaries such as Lester Young and Coleman Hawkins or
John Coltrane and Stan Getz it is not difficult to apprehend that
personal narrative is an important agent in jazz . Their individual
sounds (both in the meaning of the timbre of the sound of their
saxophones, and by virtue of their individual expressions) are so
different from one another that, for a listener unfamiliar to
jazz and to the tenor saxophone, it would not be easy to be sure that
all four of them were playing the same instrument. And Lewis's
suggestion that the personhood of the improviser is disseminated
through sound---even when recorded---in a manner that transcends
musical codes gives the musical 'voice' of the improvising musician
the status of an alternative communication system. But for this
language there is no universal grammar. To say the least, such a
system of communication is difficult to imagine being portrayed by a
machine.

In the very first section of Derek Bailey's influential book on
improvisation he writes about the impossibility to describe
improvisation ``for there is something central to the spirit of
voluntary improvisation which is opposed to the aims and contradicts
the idea of documentation.''\footnote{\citet[ix]{bailey92} Also cited
  in \citet[12-3]{nettl98:2}} Though the elusive character of
improvised music may turn it into a somewhat special case, it would be
reasonable to assert the same thing about the experience of listening
to any kind of music, no matter how or with which method it was
produced. The anthropologist Gregory Bateson writes about art in
general as ``an exercise in communicating about the species of
unconsciousness [\ldots] a play behaviour whose function is [\ldots]
to practice and make more perfect communication of this
kind.''\footnote{\citet[p. 137]{bateson72}} (The ``play behaviour'' in the
citation may be ambiguous in this context. Bateson is
referring to a model he is using to explain interaction in systems
that rely on iconic communication, i.e. the way play is used here it
is not related to music.\footnote{\citep[See][]{bateson72:play}}) The
improviser's personal narrative as a reflection of his or her
unconscious, as an interface between the conscious and the
unconscious. And this is why it has to remain in a contradictory
relation to the traditional notion of documentation; because, only
confusion can come out of the attempt to decode unconscious
expressions in the language of consciousness:
\begin{quote}
  [The] algorithms of the heart, or, as they say, of the unconscious,
  are, however, coded and organized in a manner totally different from
  the algorithms of language. And since a great deal of conscious
  thought is structured in terms of the logics of language, the
  algorithms of the unconscious are double inaccessible. It is not
  only that the conscious mind has poor access to this material, but
  also that the when such access is achieved. \emph{e.g.}, in dreams,
  art, poetry, religion, intoxication, and the like, there is still a
  formidable problem of translation.\footnote{\citet[p. 139]{bateson72}}
\end{quote}
Art, and free improvisation in particular, I would argue, is, in a
manner of speaking, a means for translating the unconscious of one
individual into an 'object' possible for a potential listener to
recreate. This is however not a translation into the structure of the
conscious and the ``logic of language''. It is in the meeting between
the improviser and the listener that the 'translation' occurs, in a
creative operation in which the listeners re-creates that which
resonates with something in their own personhood (consciously or
unconsciously). However, the meeting may, as in the case with my
listening to Parker and Coltrane, be displaced in time and space. It
may just as well take place through a recording ``immanently open to
re-creation''.\footnote{\citet[p. 26]{Born2005}. Georgina Born is
  discussing this issue from a slightly different perspective and she
  makes use of Alfred Gell's concepts of the distributed art work,
  which she applies to improvisation and digital technology. She terms
  this distributed capacity of music ``relayed creativity'' (p. 26)}
This faceless meeting between improviser and listener resonates well
with Bateson's statement that in primary process ``the focus of the
discourse is upon the \emph{relationships}.''\footnote{\citet[p. 139]{bateson72}}
Specified as a ``relationship in the more narrow sense of relationship
between self and other persons or between self and the environment''
the unconscious becomes the location of the communication between
musician and listener, a communication different in type from verbal
messages. Bateson further specifies these relations as ``the
subject matter of what are called feelings''\footnote{\citet[p. 140]{bateson72}}
which brings us back to Coltrane and Leroi Jones: ``[\ldots] Trane is now a
scope of feeling.''\footnote{\citet{leroi65}} The formulation is striking:
With a focus on the subliminal, as if Coltrane provided us with a
territory of 'inner' sensations to explore, he is simultaneously
projector and receiver of feelings.

So far the discussion has been centered around listening to, and
appreciating a performance of improvised music, i.e. looking at
improvisation from the outside. Now we will instead turn to the act
itself. If we assume that Bateson is right and that the greater part
of the communication that takes place in art (music) is structured
according to the principles of primary process, how do the
improviser get access to the unconscious? Does it happen 'by itself'
or are there conscious operations that can be performed that allows
the artist access to his or her 'inner' life? Bateson speaks about
``[t]he \emph{code}'' whereby the artist accomplishes the
transformation from conscious, verbally coded, actions (\emph{I want
  to improvise}) to the unconsciously signed output of the
action.\footnote{\citet[p. 130]{bateson72}} 
% To begin with, it is important to
% remember that the unconscious and the conscious only represents the
% two extremes on a continuum of possibilities. 
The better I know something the less conscious I become of my wisdom:
``There is a process whereby knowledge sinks to deeper and deeper
levels of the mind.''\footnote{\citet[p. 135]{bateson72}} This is obviously
related to 'habit' (walking, breathing, playing) and 'skill' but, as
pointed out by Bateson, there is a peculiarity with relation to the
arts and the practicing artist: ``to practice has always a double
effect. It makes [the artist], on the one hand, more able to do
whatever it is he is attempting; and, on the other hand, by the
phenomenon of habit formation, it makes him less aware of how he does
it.''\footnote{\citet[p. 138]{bateson72}} As we will see, and perhaps contrary to
what one may think, the fact that the 'how' becomes unconscious
through habit does not necessarily mean that the 'what' exploits a
greater sensibility or moves to a deeper level. Furthermore, the skill
of the artist, the point of which is, among other things, to convey
information about his or her legitimacy (``Since I know how to play
'the changes', I may play 'wrong' notes''), may have no, or even the
opposite meaning in some contexts. Bateson brings up American visual
artist Jackson Pollock---for whom chance was an important agent---as
an example: ``In these cases, a larger patterning seems to propose the
illusion that the details must have been
controlled.''\footnote{\citet[p. 148]{bateson72}} In other words, where a
'skilled' performer needs to communicate the 'I'm-in-control' message
to the listener (conscious) in order to move beyond it, the performer
for whom skill is not important (unconscious) may leave it to the
listener, or to the larger structure of the performance, to construct
the detailed order.

\section*{Without memory}
\label{sec:without-memory}

The American saxophonist and jazz musician Ornette Coleman, whose 1960
record release \emph{Free Jazz: A Collective
  Improvisation}\footnote{\citet{coleman60}} (The cover for the
original release of the LP record was a reproduction of a painting by
Jackson Pollock.) lend its name to the entire \emph{Free Jazz}
movement, had already in the late 1950's established himself as a
leading figure on the avant-garde jazz scene. As for all alto
saxophone players he had to deal with the intimidating heritage of
Charlie Parker: ``No other instrument is so loaded with jazz
convention as the alto saxophone.''\footnote{\citet[In my translation
  from Swedish][p. 12]{glanzelius67}} And, at least to the outside world,
he was successful at it since the \emph{lack} of Parker influences was
one of the criticisms against
Coleman\footnote{\citet[See][61-2]{litzweiler92}}. But in order to be
able to ``create as spontaneously as possible---'without memory,' as
he has often been quoted as
saying''\footnote{\citet[p. 117]{litzweiler92}}, without any 'real'
training he started playing the violin and the trumpet. These
instruments gave him the freedom to play and improvise in a manner
that his ``memory'' made it difficult for him to do on saxophone. When
playing the saxophone he would be partly ruled by his
meta-knowledge---his knowledge \emph{about} playing the
saxophone. Expectations encoded mentally as well as bodily would also
influence him and, to Ornette Coleman, this was a hindrance to his
spontaneity. The trumpet, and even more so, the violin---the violin
was not really an instrument used in contemporary jazz so there were
no model to follow---he had learned himself. On the violin he adopted
a highly original technique that allowed him to bypass ``not only the
jazz tradition, but Western musical traditions altogether. He had no
teachers or guides to show him how to play trumpet and violin and
purposely avoided learning standard
techniques''.\footnote{\citet[p. 117]{litzweiler92}} A perhaps even more
radical step was taken by Ornette Coleman on the record \emph{The
  Empty Foxhole}\footnote{\citet{coleman66}} where he engaged his own
10-year old son to play the drums. Again was the criticism that the
music didn't sound the way it ``should''. One, perhaps axiomatic,
comment said it sounded ``like a little kid fooling
around''\footnote{\citet[Freddie Hubbard cited in][p. 121]{litzweiler92}}
but for Coleman this was a deliberating experience that, for a moment,
relieved him of some of the pressures from the 'outside world': ``I
felt the joy playing with someone who hasn't had to care if the music
business or musicians or critics would help or destroy to express
himself honestly.''\footnote{\citet[Ornette Coleman quoted
  in][p. 121]{litzweiler92}} Freedom of memory and freedom of influence
from extra-musical parameters. The to Ornette 'unknown' instruments
gave him a sense of internal freedom, liberated from the physical
memory associated with his saxophone playing. Playing with his son
gave him a sense of external freedom where the undestructed
na\"{i}vity of the child gave him access to the sound of the honest
and pure self-expression with less noise between the (unknown)
intention and the result. Or, rather, associated him with the
possibility for a self-expression where the transformation from
intention to result is not ruled by a pre-conceived notion of what the
result should sound like and where the transference is not influenced
by external forces such as the economic powers that surround the
creative activity.

A comment by Marcel Duchamp---one of the great innovators of visual
arts of the 20th century---shows us that the wish to outsmart learned
and inherited habits is not limited to the field of free
improvisation. In a study for is great work from the early years
\emph{The Bride Stripped Bare by Her Bachelors, Even (The Large
  Glass)} (1915-1923) he used ``a mingling of paint and 'non-art'
materials that had not as yet received the name of
collage''\footnote{\citet[p. 29]{tomkins65}} in an attempt to ``avoid the
old-fashioned form of drawing.''\footnote{Duchamp, as quoted
in \citet[p. 29]{tomkins65}} The ``old-fashioned'' drawing was for Duchamp what
the sound of Charlie Parker was to Ornette Coleman. A source that had
to be reversed and unlearned, however not forgotten. In effect similar
to Deleuze's description of Bergson's conception of memory as ``a
function of the future'' and ``only a being capable of memory could
turn away from its past.''\footnote{\citet[p. 45]{deleuze02}} Duchamp continues
by asking:

\begin{quote}
  Could one do it without falling into that groove? Mechanical drawing
  was the answer---a straight line drawn with a ruler instead of the
  hand, a line directed by the impersonality of the ruler. The young
  man was revolting against the old-fashioned tools, trying to add
  something that was never thought of by the fathers. Probably very
  na\"{i}ve on my part. I didn't get completely free of that prison of
  tradition, but I tried to, consciously. I unlearned to draw. The
  point was to forget \emph{with my hand}.\footnote{Duchamp, as quoted
    in \citet[p. 29]{tomkins65}.}
\end{quote}

The concept to \emph{consciously unlearn} seems inconceivable: Is
it really possible to forget in a conscious manner? However, just as
Coleman, Duchamp was using a (new) tool (the ruler) to revolt against
the tradition. And the expression ``forget with my hand'' is
significant here as it puts the focus on the physicality of the
action. Habitual muscular responses, learned ``patterns'', which may
even get triggered unconsciously are common and known to all who have
played and practiced a musical instrument. What we see here and in
Coleman's use of alternate instruments, is an attempt to subdue the
influence these ``patterns'' may have on the artistic output with the
primary goal to get closer to the pure subjectivity, or the pure
personal expression. 

A slightly different method has been employed by the already mentioned
British free improvised music saxophonist Evan Parker\footnote{With
  regard to the significance of the unconscious in free improvised
  music it may be noted that Evan Parker's record label is called
  \emph{Psi}, referring to various variants of
  parapsychological phenomena. The mission statement for the record
  label Psi ``is to present unique statements from individuals making
  music their own way regardless of genre'' (See
  \url{http://www.emanemdisc.com/psi.html}).} whose saxophone playing
is characterized by extensive use of circular breathing, alternate
playing techniques such as split tones, over blowing and
multiphonics. The sheer complexity of the sonic result is enough to
relieve him and his instrument from the burdens of tradition. But in
this context it is interesting to note that ``Parker has worked to
liberate different bodily aspects of his playing---the fingers, the
tongue, the larynx, and the breath---so that each physical system may
achieve a substantial degree of independence.''\footnote{\citet[p. 49]{borgo05}}
Parker has consciously put himself in control and, through practicing,
he has made it possible to 're-wire' the links between the various
parts of his playing apparatus (the instrument \emph{and} the
body). This would imply that \emph{consciously unlearning} may be
achieved through practicing; unlearning by doing (which is no less
inconceivable). The result of Parker's method, in particular in his
solo improvisations, is to a certain extent a clear sonic
representation of the re-wired physicality of his technique. Alone on
soprano saxophone he can create a two to three part polyphony which
``allows him to circumvent obvious ascending and descending phrases in
a way that challenges the dominant conceptual mapping [\ldots] of
pitch relationships in vertical space.''\footnote{\citet[p. 50]{borgo05}} That
the method has been successful is obvious from merely listening to his
music, but he also explicitly states that he ``believes the best part
of his playing to be beyond his conscious control and his rational
ability to understand''.\footnote{\citet[p. 52]{borgo05}}


% Turning back to Duchamp, who introduced the notion of the ``personal
% 'art coefficient' '' , in a discussion concerning the creative
% act. Specifically Duchamp is referring to the immanent processes,
% those in which the artist alone is involved and which ultimately lead
% to ``art in the raw state---\`{a} l'\'{e}tat brut'' and, in short, it
% constitutes the difference between the artistic intention and its
% realization. The gap goes unnoticed by the artist and it represents
% his (or her) inability ``to express fully his
% intention''\footnote{\citet{duchamps57}} Duchamp concludes that ``the personal
% 'art coefficient' is like an arithmetical relation between the
% unexpressed but intended and the unintentionally
% expressed.''\footnote{\citet{duchamps57}} It is primarily the description of
% the art coefficient as a \emph{relation} that is interesting with
% regard to the current discussion. The roles of the two factors
% (unintended and intended) are likely to be different in free
% improvised music and the 'intention' that Duchamp is speaking of here
% is probably not meaningful in relation to Coleman and other proponents
% of the free jazz movement.\footnote{From within the field of jazz and
%   free form the ``I just play what I hear'' paradigm is common and it
%   is difficult to see how 'intention' in the sense here discussed can
%   be part of the equation. However, this is from within. From an
%   analytical point of view ``I just play what I hear'' is most
%   certainly an intention.}  But the notion of the ``unintentionally
% expressed'' as having a relation to the ``unexpressed but intended''
% is useful when looking at Coleman's unorthodox play of the trumpet and
% the violin and something we may use when we go back to the subject of
% computers and improvisation. Furthermore, the personal art coefficient
% possibly holds an abstract relation to the notion of the structure of
% the unconscious, whose discourse is also focused on the relationships.

\subsubsection*{Why computers?}
\label{sec:why-computers}

Ornette Coleman, Marcel Duchamp and Evan Parker have all three, in
their own ways, consciously attempted to oppose habit in the way they
carry out their respective practices. They have done so, I believe, in
order to more fluently be able to speak the language of the
unconscious and, hence to be more adept in conveying their message. In
these processes there are no logical, linear operations, no
rationality: ``A jazz musician may choose his own resistance. [\ldots]
Resistance is not a goal for which the musician will strive, it is a
constantly stretchable challenge.''\footnote{\citet[My
translation][p. 8]{glanzelius67}} Rather than trying to minimize friction
and make 'playing' easier, they have made it more difficult and
increased the resistance on purpose. This is consistent with Aden
Evens suggestion that ``[t]he instrument's resistance holds within it
its creative potential, which explains why improvisers focus so
explicitly on aggravating the instrument's
resistance.''\footnote{\citet[p. 162]{evens05} For more on the notion of
  \emph{resistance} see Marcel Cobussen's essay in this issue.}

Here is where we can go back to the questions regarding the role of
the computer as an instrument in free improvised music: Why would
someone want to use a machine in which physical contact with the sound
is virtually impossible, to perform a music that thrives on expressive
sensibility? Precisely because of its 'shortcomings'. Because it
\emph{increases} resistance. Because it resists habit
formation. ``Habit is a major economy of conscious
thought''\footnote{\citet[p. 141]{bateson72}} which is why the lack of standards
and consistency with regard to the implementation and control of
computer based instruments may appear frustrating on the analytical
level but work perfectly well on the artistic level. Just as the
``impersonality of the ruler'' allowed Duchamp to revolt the
``old-fashioned tools'' the computer may allow the musician to forget
with his hand. In this case in a double sense due to the lack of
body-sound interaction particular to the computer based
instrument. But again, the detachment of the body from the instrument,
though it may prove a hindrance on some occasions, it may also be a
guarantee to disallow habit to take control: ``The musician who just
allows his body to do what it will cannot be immerse in the music, for
he is not even engaged by the music; the one who lets his body play is
not putting himself on the line, not tying his own fate to the
imminent death in the music, and so fails to draw out the ownmost
possibility of the music in the moment.''\footnote{\citet[p. 143]{evens05}}

The frustration Bob Ostertag is experiencing\footnote{\citet{ostertag02}} with
computer based, and electronic, instruments is actually consistent
with my theory here. Even though it is easy to agree with him in his
critique of the nature of many of these instruments and their inherent
lack of bodily possibilities, we must also acknowledge that Ostertag
has successfully played them for over 30 years. The defeat is
experienced on the consciously analytical level and doesn't
necessarily manifest itself on lower levels of the mind where the
\emph{improviser} Ostertag continues to perform, in principle
unaffected.  If Ornette Coleman was in his 30's today, I'm sure he
would have chosen to play laptop---instead of, or in addition to, the
trumpet and the violin---as a complement to his saxophone.\footnote{It
  should be noted that the Coleman/Pat Metheny CD \emph{Song
    X} \citep{metheny86} was one of the earliest in which electronic
  instruments and samplers where used, even though Coleman himself
  does not indulge in the use of electronics.}

If I have managed to give an incentive for the use of the
computer-as-instrument in the context of free improvised music, the
situation for the computer-as-improviser is by the same reasons even
more disconsolate. The main issue is the computer's abilities to
interface with its surroundings, or rather, its lack thereof. If it
cannot absorb relevant information, at least with regard to its fellow
musicians and the sounds they are producing, it will scarcely be able
to move beyond the point of an advanced (albeit incredibly advanced)
CD player. Or, somewhat less derogatory, in the words of composer
Pauline Oliveros: ``Unless the styles of the musicians improvising
were already absorbed by the machine then what information would there
be to calculate a response? If the outcome is known in advance it is
not free improvisation, it is historical
improvisation.''\footnote{\citet[p. 122]{oliveros08}} Personally, I still
think a computer-as-improviser may be deployed with success. But the
details of that process will be the subject of another essay.


% ``By choosing he is also accounts for the reasons behind his
% musicianship.''\footcite{glanzelius67}

% In his book \citefield{title}{gell98} Alfred Gell builds an
% anthropological theory of art, the details of which goes well beyond
% the scope of this essay,\footnote{See Georgina Born's essay on
%   mediation in jazz and improvised electronic musics for an adaption
%   of Gell's theory to the current subject of improvisation. In
%   addition Born provides a criticism of Gell's theories: ``for all
%   Gell's concern to defend the properly anthropological scale of his
%   ideas, we cannot divorce his analysis from the insights provided by
%   sociology and history.''\citep[22]{born05}} but his view on how the
% artefact (in his book Gell deals exclusively with object-based art)
% becomes instances of the subjectivity of the artist is interesting :
% ``Here we have, in public, accessible, form, the `continuum of
% continua of protentional and retentional modifications' described by
% Husserl for the purposes of elucidating the purely \emph{subjective}
% process of cognition, or consciousness. In other words, as a
% distributed object, Duchamp's consciousness, the very flux of his
% being as an agent, is not just 'accessible to us' but has assumed this
% form. Duchamp has simply \emph{turned into} this object, and now
% rattles around the world, in innumerable forms, as these detached
% person-parts, or idols, or skins, or cherished valuables.''

% \section{Improvisation with computers}
% \label{sec:impr-with-comp}

% Electronic music and the body: Iyah, Ostertag, evens (150.)

% Though it will be difficult to
% separate personal traits that are a result of practice and learning,
% from features acquired from an honest and unadultarated link to ones
% unconscious (if that is at all possible), the high esteem given
% 'personality' in jazz and freely improvised music is of
% importance. (It should be noted that my primary concern here is not
% 'prove' that the conscious/unconscious divide is a existent power, but
% merely that, based on the examples I give, it is a experienced agent
% of importance to some artists.) 

% The examples I will bring up below are evidently not
% representative of the entire field of improvised music (they are not
% intended to be) but it is interesting to note that the criticism against
% the practice of improvisation based on its lack of conscious
% strategies on the surface 


% ``The notion that there are musicians who, as it were, can do anything
% they want on the spur of the moment is strange to the classical
% musician, who is scandalized by such lack of discipline but also
% attracted by its prsumed liberty.''\citep[7]{nettl98:2}

% In an attempt to get closer to the unconscious, or perhaps rather to
% the artistic expresssion which is not molded by conscious decisions,
% practice, and external sensations, but which is 'pure' and 'inner' and
% 'personal', knowledge is in some cases dismissed, devalued or
% deliberately bypassed. 

% The basic problem according to anthropologist Gregory Bateson is that
% the

% To not prepare oneself
% does not guarantee access to this different kind of organsiation. But
% it is a method to attempt to disallow the socially, culturally and
% morally regulated conscious to tamper with the immediate, pure
% expression of intuition. Other methods by which acces to the
% unconscious may be gained include intoxication, religion and
% dreams. 


% Art is in itself a way to gain access to the information streams of
% the unconscious which are, in Freudian language, ``structured in terms
% of \emph{primary process}, while the thoughts of consciousness
% (especially verbalized thoughts) are expressed in \emph{secondary
%   process}''.\footcite[139]{bateson72} 


% Differences like these contribute to making jazz (among
% other improvisatory musics) difficult to analyze in a quantitative
% manner.

% The concept of
% 'resistence' becomes a key player. 



% ``If improvisation has often been described with respect to the
% expectations and responses of listeners in a familiar milieu, it can
% also be treated as an art that enables performers to control their
% dependence on habitual responses.''

% When Baraka writes that ``[\ldots] Trane is now a scope of
% feeling'',\footcite{leroi65} emphasis is again put on the
% subliminal. As if Coltrane provided us with a territory of 'inner'
% sensations to explore. There would be no way to now what part of the
% expression was added by us, as listeners and what aspects are immanent
% to the music, but in our interaction with the music it is only our
% intuition that can provide us with the map. A simultaneous projector
% and receiver of feelings.

% ``Improvisation (like 'freedom' or 'music')
% is a topic that forces us to confront formidable problems of
% translation.''\footcite{blum98} Though Stephen Blum is primarily
% concerned with the translation between different cultures and
% languages in the article quoted above, within the Western contemporary
% music tradition many different and contrasting understandings of
% improvisation may be traced. 

% However, in this paper I will attempt at showing that improvisation is
% not necessarily lack of organisation but rather a manifestation of
% organization of a different kind.

% Musical improvisation, as a special case of creativity, is to me
% one of the most fantastic and elaborate things imaginable.

% I will not here attempt to
% unravel all possible readings of 'improvisation' but I will focus on
% improvisation as a method to gain access to---though possibly still
% unconsciously---subconsciousness and how this aproach links in with
% the use of technology in improvised musics.

% In its negative connotation it could mean that I am
% ill prepared. That I have not planned what to do or how to do it. In
% positive terms it could instead mean that I am spontaneous and that I
% trust my intuition to guide me. That I allow my unconscious to rule my
% unreflected actions without interference of the mind or my
% consciousness. Whether the statement is perceived positively or negatively
% depends largely on the context in which it is uttered.
% % And to further narrow
% % the broad concept of improvisation down, 


%%% Local Variables: 
%%% mode: latex
%%% TeX-master: "../Improvisation"
%%% End: 
