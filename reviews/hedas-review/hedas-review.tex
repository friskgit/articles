\documentclass[a4paper]{article}
\usepackage[swedish, english]{babel}
\usepackage[T1]{fontenc}
\usepackage[authoryear,round]{natbib}
\usepackage{url}
\usepackage[utf8]{inputenc}
\usepackage{fullpage}
\renewcommands{\encodingdefault}{T1}

\title{Recension av \emph{Linjer: musikens rörelser - komposition i förändring}}
\author{Henrik Frisk, Art. Dr.}
\date{\today}

\begin{document}
\selectlanguage{english}
\maketitle

\noindent
En av de första saker man slås av när man börjar titta i Kim Hedås avhandling \emph{Linjer: musikens rörelser - komposition i förändring} är omfattningen. Sexton kompositioner och en bok på 310 sidor ger en mastig upplevelse och även om den är helt i linje med övriga avhandlingar som producerats de senaste åren bör man kanske ändå fundera över hur stor en konstnärlig doktorsavhanling egentligen bör vara. En DVD med dokumentationen av de konstnärliga verken är föredömligt producerad och gjord för läsning i en webläsare snarare än en DVD-spelare och även om inga digitala format har särskilt lång livslängd så lär det öppna HTML-formatet överleva det slutna DVD-formatet. 

De sexton kompositionerna, liksom de 30 kapitlen är indelade i 5 linjer: 

\begin{itemize}
\item Rörelse
\item Identitet
\item Tid
\item Minne
\item Rum.
\end{itemize}

Den klara strukturen gör materialet överskådligt trots sin omfattning även om texten erbjuder en del motstånd. Det finns ingen direkt information om hur linjerna har tillkommit men som jag uppfattar det så har de vuxit fram ur forskningsprocessen och, så att säga, använts som ett raster för att gruppera musiken, delvis i efterhand - även om jag är säker på att något av de senare verken har tillkommit \emph{genom} rastret snarare än tvärtom.
% Jag ska försöka leda er genom min läsning av avhandlingen. Jag utgår från musiken - redan spelad - och presenterar relevant text varefter jag går tillbaks till musiken osv. Lite som ett Rondo och för enkelhetens skull så tar vi avdelningarna i samma ordning som de kommer i boken.
I slutet av första kapitlet två för kanske all forskning helt centrala ställningstaganden: ``Det är omöjligt att förbli anonym'' och ``Jag behöver gå in som mig själv''\footnote{Nu håller säkert inte alla med om att det subjektiva förhållningssättet är en förutsättning för all forskning men i artikeln \emph{Beyond Validity} diskuterar jag och Stefan Östersjö just det. \citep{frisk-ost13}}
%, men jag vill bestämt hävda att det är ett problem om en naturvetenskaplig forskare inte har en tydlig uppfattning av sitt eget inflytande på processerna. Möjligtvis kan man göra en distinktion mellan forskaren som är medveten om sin subjektivitet och forskaren för vilken det subjektiva är själva grundförutsättningen}. 
I inledningen finns forskningsfrågan beskriven som: ``Hur kan förändringarna, som relationer mellan det som är musik och det som inte är musik ger upphov till, skapa möjligheter för komposition?'' Min omedelbara fundering kring forskningsfrågan är hur det ens är möjligt att avgöra vad som otvetydig är musik och vad som inte är det, men den diskussionen återkommer vi till längre fram. Metoden, som vi också ska återkomma till, beskrivs helt kort som ``En reflexiv rörelse mellan undersökningens olika delar.''

Julia Kristevas begrepp intertextualitet är en uppenbar referens när temat är relationer mellan olika iterationer av konstnärligt arbete och det lyfts fram som en immanent del av arbetet tillsammans med dialog, hypertextualitet, intermedialitet och ekfras. Ekfras kanske framstår som det mest användbara begreppet och det som närmast harmonierar med avhandlings tematik, men det förekommer tyvärr bara i början, bl.a. i ett citat ur boken Intermedialitet \citep{lund2002} och i samma citat igen på sidan s. 260. Det är synd för Kim går här miste om en chans att ställa sin konstnärliga praktik mot en teoribildning med stor relevans, individualitet och potential.

\subsection*{Identitet: Raivadiado}

I fjärde kapitlet berör Kim förutom relationen musik/icke musik också motsättningar som liv och död och konst och natur natur. En biografisk del om isländske tonsättaren Jon Leifs används som struktur till vad som kanske kan ses som ett teorikapitel men som också är en inledning till \emph{Vattnet}, ett körstycke som inte ingår i avhandlingen, men framförallt till \emph{Raivadiado}. Här hittar vi ännu en pusselbit på vägen mot att förstå detta arbete: i ett utsnitt ur en essä från ett tidigt skede av forskarutbildningen skriver hon: ``Lika främmande känner jag inför att rekonstruera mina tankar genom åren. Jag har alltid varit i ett nu.'' En viktig förklaring som också visar på en förändringsprocess då avhandlingen jag har framför mig är en dokumentation av ett arbete som är sammankopplat via noder i flera dimensioner som inte kan ha producerats bara i ett nu och som definitivt också innehåller rekonstruktion i olika faser.

Sen följer kapitlet om just \emph{Raivadiado} och dess olika andra inkarnationer, ett av de mest hänförande styckena musik i avhandlingen. Det härstammar egentligen ur ett annat verk, Kims musik till teateruppsättning av Dödsdansen 2007 och genom att vara samtidigt idiomatiskt och naturligt -- naturligt i betydelsen att dess utveckling är organisk och okonstlad -- pressar det gränserna det hörbara och öppnar för ett lyssnande på flera olika plan. Men här kan man också skönja svagheten i avhandlingens tydliga struktur, en svaghet som i och för sig många konstnärliga forskningsprojekt delar: strukturen i presentationen ger inte alltid rättvisa åt komplexiteten i innehållet. I strävan efter en enkel och välstrukturerad ingång riskerar man att för formens skull förenkla det komplexa utan att det därför blir mer lättförståeligt. \emph{Raivadiado} skulle lätt kunna passa in under vilken som helst av de fem linjerna, så varför begränsa det till bara en? Vad man vinner i struktur riskerar man förlora i innehåll. Första delen avslutas med en reflektion om processer och förändringar. Det är första gången vi möter Bergson som känns som en viktig referens för Linjer. Flera av avhandlingens teman är centrala i hans arbete och det är i detta kapitel som jag först förstår hur annorlunda denna text är, hur jag blir tvungen att ge upp att reduktionistiskt försöka analysera utsagorna, utan istället placera mig i textens flöde. 

%Att börja. Att börja och sedan att forsätta, och om 

Att börja att tala är att gå in i diskursens riskabla ordning skriver Cecilia \citet{rosengren2009} som Kim citerar i kapitel 7. Att föra in kontinentalfilosofin i ett arbete som strävar efter att studera relationerna mellan det som är musik och det som inte är det gör mig fundersam och en smula förvirrad. Vad är det i forskningsfrågan jag inte förstår? Utifrån vad jag läst och hört fram till nu, hade det inte varit mer framkomligt att söka dekonstruera den binära relation mellan musiken och det som inte är musiken? Om jag läser in lite mellan raderna: är Kims konstnärliga praktik en diskurs (i Foucaults mening) vars utveckling beror på mötet med det som inte är del av den diskursen från början? Men var passar då Deleuze blivande som citeras på nästa sida i avhandlingen in i bilden? Deleuze metod för att komma runt dikotomier som människa och natur (och kanske musik och ickemusik), är att hänvisa till blivande som ett tillstånd som gör att distinktionen mellan det enas slut och det andras början blir omöjlig att göra. Jag funderar över om inte forskningsfrågan snarare ligger närmare: \emph{Hur och under vilka förutsättningar blandar sig livet/världen/naturen i det som vi definierar som musik?}

I kapitel 8 möter vi igen identifikationen med nuet fast med en ansträngning att se tiden som något sammanhållet. Kan vi se ett stycke komponerad musik som en bit sammanhållen tid som för komponisten fungerar som en markör i tiden som man kan återvända till? Ändå har jag själv ofta slagits av hur fel mitt minne har haft i dessa sammanhang. Även om jag känslomässigt brukar kunna återvända genom musiken har ofta perceptionen förvrängts och blivit del av den historia som man skriver och just detta är väl en del av den konstnärliga forskningens dilemma: Med vilken metod kan processen återbesökas och dekonstrueras utan att en av processens mest centrala bitar, det subjektiva anslaget, blir lidande? Jag kan vara fundersam kring den reflexiva rörelsen som metod. Just reflexion är ett ord som ständigt återkommer i diskussionen om den konstnärliga forskningens metoder. Då ett av syftena med forskningsmetoden är att skapa nya gränsytor till det som ska beforskas vill jag se hur reflexion som forskningsmetod i så fall skiljer sig från reflexion i konstnärligt skapande, eller, i det tillfälle den konstnärliga metoden också är forskningsmetoden, på vilket sätt bidrar metoden till synliggöra det som tidigare varit dolt?

% I funderingarna kring  minne, perception och taktilitet samt kvantmekanik närmar sig reflektionen även kognitiv neurovetenskap
% Här närmar sig reflexionen även kognitiv neurovetenskap i funderingarna kring minne, perception och taktilitet samt kvantmekanik och når tanken om lyxen det innebär att ens musik landar hos någon annan, när Kim berättar om mötet med regissören som lyssnar - och hör - och spelar. Kanske kan vi kalla det intermusikalitet: mötet mellan två musiker/musiker som finner resonans.


\subsection*{Identitet: Historien lyder}

Historien lyder är ett samarbete mellan Kim Hedås, Christina Ouzinidis, Teater Weimr och Ensemble Ars Nova. Christina Ouzinidis, själv doktorand på teaterhögskolan i Malmö, är en av dramtikerna som varit med att bygga upp Teatr Weimars rykte som en av de mer progressiva postdramatiska teatergrupperna i Sverige. Jag såg uppsättningen i Stockholm på teater Galeasen vilket gör det uppenbart att dokumentationen av ett konstnärligt arbete aldrig kan vara verket.

Utgånspunkten för Historien Lyder var ett hörspel och partituret är tio improvisationsmodeller med instruktioner, som på ett konceptuellt plan till en början påminner om Stockhausens Kurzwellen eller Plus-Minus. Initialt improvisation som genom olika processer styrs mot fast form för tre instrument och tre skådespelare. Till de tre instrumenten kommer ett elektroakustiskt spår, inspelningar av rösterna, och, kanske viktigast, ett fjärde gemensamt instrument som utgörs av trion i sin helhet. Processen beskrivs som en följd av händelser och ställningstagande men tyvärr får vi inte veta mer om relationen mellan text och musik eller mellan regi, scenografi, musik och skådespel. Det är en diskussion som jag är säker på hade kunnat tillföra mycket till den här avhandlingen. Det är sannolikt svårt att utröna vad som är musik och vad som inte är det borde mötena med Ouzinidis och Teatr Weimar vara högst relevanta utifrån forskningsfrågan. Ett mer generellt resonemang om musikdramatikens identitet får vi dock i efterföljande kapitel genom Hans \cite{gefors2011} avhandling och Kaija Saariaho opera L'Amour de loin. Här kommer kanske också en förklaring till varför de egna processerna inte beforskas: ``För mig känns det problematiskt att skriva om mig själv, mina val, mina tankar, mina drömmar.''

\subsection*{Tid: Illusion och Intermezzo}

\emph{Illusion} är ett samarbete med Petra Gipps och hennes Refugium som presenterades på Kivik Art Centre första gången. En ljud och arkitekturinstallation i naturen där lyssnaren omges av ljud från sex högtalare, en musik utan början eller slut. \emph{Illusion} inleder bokens tredje del ``Tid'' och musiken använder sig av rumsliga förflyttningar av ljudet som, skriver Kim, är ``händelser i ett musikaliskt flöde'' och skapar positioner som ger både ``rum och tid'', en mening som leder mig tillbaka till musikdramatiken, till Schopenhaurs kausalitet och Richard Wagners. ``I stycket Illusion har det förflutna funktionen att osäkra fortsättningen'' (s. 150) skriver Kim och här blir jag osäker om vi talar om installationen eller bara musiken (och var går gränsen?), men \emph{Illusion} framstår av den beskrivningen som en evig cliffhanger. Något förvånande skriver Kim sedan att tankar om tid och minne kan vara viktiga för kompositionsprocessen men är inte del av det hörbara och ska heller inte vara det. Det är en struktur som endast tjänar kompositionsprocessen. Men var slutar i så fall den? Och, återigen, vilken funktion har samarbetet med Petra Gipps här och vilken roll spelar strukturen i det gemensamma arbetet?

% ``Att börja är roligt. Att göra färdigt är tråkigare.'' (s. 158) Kapitel 14 är en reflektion över tid och komponerande och kapitlet slutar i ett viktigt klargörande: ``Jag tycker musiken är viktigast'' (s. 162). Är den viktigare än dess relation till det som inte är musik undrar jag? Beskrivningen av musiken, som minnesanteckning för sig själv och för lyssnaren; hur kan detta tidsliga flöde antecknas? (Är det inte det som gör det så svårt? Att en beskrivning eller metafor just *inte* är i tiden utan utanför tiden?)

Frågan om dokumentation kommer fram i kapitel 15 (s. 167), om än helt kort. En kritisk hållning till behovet av att dokumentera allt, viktigt som oviktigt, högt som lågt. Samarbetet med Petra Gipps leder fram till en annan typ av samarbete, det med designern Thomas Laurien. Musik och film. Två rum som ger varandra nya rumsliga möjligheter. \emph{Intermezzo} är musik som har återanvänts och relaterar bakåt till andra verk av Kim Hedås (Möbelmusik, Still liv, Bröllopsmusik) och projektet presenterades hösten 2010. \emph{Intermezzo} består av ``ljud som är hämtade från olika källor, ett visst sätt att behandla dessa disparata ljud genom kompositionsarbetet och vissa givna förutsättningar.'' Tyvärr får vi inte veta hur filmen och musiken påverkade varandra. Musiken till \emph{Intermezzo} var hämtad från ett annat projekt och detta faktum gör det tydligt för Kim hur förändringen sker genom kontexten och hon ställer frågan ``Vad gör det som inte hörs för lyssningen?'' (s. 181). Men vad är det som egentligen \emph{hörs} och vad lyssnandet innebär. Kan man inte säga att det som påverkar \emph{lyssnandet} också hörs? Svaret Kim ger är att själva musiken består av kombinationen av det hörbara och det ohörbara.

Jag lyssnar och ser jag på \emph{Intermezzo}. Tiden både rör sig framåt och står stilla, är fryst. Ungefär som en operaaria där handlingen, eller dramat, stannar upp men musiken fortsätter, och stannar upp. Och jag blir så otroligt nyfiken på hur kombinationen gjordes. Eftersom musiken redan var färdig, var det bilderna som synkroniserades? Eller är det en slump att detta förhållande uppstår? Vem gjorde vad? Hur? Varför får jag inte veta mer?

\subsection*{Minne}

Det fjärde avsnittet, Minne, återvänder till temat fortsättning och förvandling. En bruten process, ett stycke som ska mixas om, allt är så självklart tills man sitter där då inget längre finns kvar annat än som en lätt dimma. Situationen där den konstnärliga processen är inte längre åtkomlig beskrivs poetiskt. Det får mig att tänka på drömmen som när man vaknar är så självklar men i den stund man ska återberätta den är den bortflugen. Den finns på något sätt i minnet men inte i orden. Inte som metafor, utan på riktigt, men ändå inte tillgänglig. Freud kallar det primära och sekundära processer och det är i översättningen mellan dessa medvetandelager som det kan gå så snett. Konsten är en primär process och språket och medvetandet en sekundär och den konstnärliga forskningens uppgift är delvis att skapa ett gränssnitt mellan dessa två.

Som hastigast flyger några korta stycken om jaget och den andre, Kristevas främlingskap och Rimbauds lek med första och tredje person i ``Je est un autre''. Kim skiver om behovet av att förvandla, först sig själv, sedan kompositionen, för att kunna komma vidare. Men det är alltså framförallt en förvandling i jaget och medierat genom den konstnärliga praktiken Kim talar om, och inte mötet med den andre? Igen undrar jag vilken förändring detta möte innebar och hur det utvecklade sig. Var det motvilligt? Lätt? Svårt? Uppbyggande? Nedbrytande? Var det välkommet? Detta är stora ämnen och Kristevacitaten hänger lite i luften alltmedan jag som läsare får gissa mig fram till avsikten. Jag vänder mig till musiken och anar, kanske känner att jag förstår, men det flyter ganska snabbt undan igen. 

På s. 197 kommer en översikt över några av de som tidigare har ställt sig frågan kring musikens och konstens relation till andra element, som text. Kurt Schwitters t.ex., Stockhausen, Duchamp, Cage och Öyvind Fahlström. En mening med ett ensamt ``Ja'' ger mig känslan att Kim skriver under på tankarna kring uttryckens flyktighet. Och här ansluter avhandlingen \emph{implicit} till dessa förvandlingar som nittonhundratalets konstliv kännetecknades av -- men någon mer ingående diskussion får vi inte ta del av. Ord blev musik, konservburkar blev konst, påhittade språk blev litteratur, tystnad blev musik, pissoarer blev konst, maskiner blev musik, ordlekar blev poesi och oljud blev ljud blev musik. Är tjugohundratalets förvandling att musik blir forskning? 

I kapitel 22 går det vidare, ifrån en förvandling från det ena till det andra mot ett adderande av det ena \emph{till} det andra. Här hittar vi behovet av repetition, kontinuitet och tradition. Och en avslutande mening fatsnar jag för: ``Redan första gången kan händerna i knät få impulser till att dirigera jaget in i lyssnandet, redan första gången kan musikens mening uppfattas och allt kan skickas blixtsnabbt, bredvid den reella tiden, in i minnet.'' När jaget dirigeras in i lyssnandet så är det inte, som jag läser det, frågan om en transformering från ett tillstånd till ett annat, utan något som läggs till något annat. Det är kroppsligheten, händerna rör sig, som skapar förutsättningen för att lyssnandet och jaget sammanfogas. Utan att fundera över hur ett lyssnande \emph{utan} ett jag kan fungera så närmar sig detta utvidgade lyssnande ett etiskt lyssnande i sin öppenhet. Fjärde avsnittet avslutas med en annan redogörelse för ett kroppsligt och taktilt förhållningssätt i några ensamma meningar i listform. ``Minnet i en rörelse'', kroppsminnet och intoneringen, känslan av det kalla, och därigenom ostämda instrumentet. Plågan när violinisterna inte stämmer som de borde. Men här slutar vi istället vid lyssnandet som en frigörelse från det kroppsliga: ``En lättnad, jag kan lyssna.'' Igen känner jag mig osäker på vad avsikten är. Läsandet tvingar mig att sluta tänka reduktionistiskt och istället närma mig texten liksom musiken med stor öppenhet, ständigt beredd på att omvärdera mina uppfattningar. 

\subsection*{Rum: Part}

Sista och femte avsnittet, Rum, börjar med något som närmast kan liknas vid ett teorikapitel. Det framgår inte i boken men texten är producerad som del av en musikestetisk kurs på Stockholms Universitet och temat är lyssnande och perception. Här möter vi bl.a. Jean-Luc \citet{nancy2007} genom hans bok \emph{Listening} och Pauline \cite{oliveros05} genom hennes \emph{Deep Listening}. Kapitlet är, som anges på sidan 219 ``ett försök att öppna upp och vidga de ramar innanför vilka förståelsen av lyssnandet till musik kan diskuteras.'' Kapitel 26 skulle jag behöva en smula mer information till för att förstå i detta sammanhang men det består av en serie poetiska texter med fotnoter till ganska långa textavsnitt. Några av dessa fotnoter har förkommit tidigare. 

Part är en fascinerande installation och karta över ett abstrakt fält, konkret tecknat. Jag förstår när jag läser att det finns mer material från det skede då dessa verk växte fram och funderar över vad de skulle ha kunnat lära mig. Egentligen är det synd att det inte finns mer av reflexiv rörelse här mellan part och knot och musik och arkitektur och mellan Kim och Petra och mellan de olika platser där dessa verk stälts ut.

Det är i samarbetet med Gipps som jag först förstår innebörden av forskningsfrågan. Kanske är det så det ska vara, när man tagit sig igenom avhandlingen så har man grepp om frågan som ställts initialt? Sista kapitlets inledande ``Musiken är'' (s. 276) känns spontant kontraintuitiv till detta arbete. Längre ner, dock, läser jag att ``avsikten inte är att ge slutliga lösningar på de frågor som är aktuella.'' ``Meningen är \emph{inte} att med ord besvara frågorna.'' (s. 279) En bra sammanfattning av de fem delarna följs av en kort genomgång av framtida utvecklingsmöjligheter och några appendix om konstnärlig forskning, om studiemiljön och om de teorier som Kim har vänt sig till eller genom. Efter en engelsk sammanfattning och referenser har vi nått till slutet av musiken och texten.

\subsection*{Coda}

Åter till forskningsfrågan: ``Hur kan förändringarna, som relationer mellan det som är musik och det som inte är musik ger upphov till, skapa möjligheter för komposition?'' Utifrån denna fråga, vad kan jag berätta om möjligheterna som skapats för Kim Hedås? Tveklöst har samarbetena med Gipps, Ouzinidis och Laurien skapat just möjligheter för komponerandet men från avhandlingen får vi inte veta mycket om det annat än resultatet. Vi vet att det påverkar men inte hur. Det får mig att tänka att forskningsfrågan i detta fall faktiskt egentligen inte är frågan utan \emph{metoden}. Kim använder mötet med det andra, med komponerandets ontologiska motsats, som en metod att förstå hur processerna kan utvecklas och hur hon kan driva sitt eget arbete vidare, i ett försök att optimera sina egna konstnärliga processer. Att se frågan som metod löser delvis också ett annat problem, nämligen att avhandlingen inte positionerar sig gentemot andra konstnärliga arbeten som rör just relationen mellan det som är ``musik och det som inte är musik''. Att diskutera vad som är musik och vad som inte är det i en post-Cage era utan att beröra sextiotalets experiment på området gör det onödigt svårt att förstå vad som är Kims avsikt men om detta istället är metoden snarare än frågan blir inte just den bakgrundsanalysen lika påträngande viktig. 

Kim Hedås avhandling \emph{Linjer: musikens rörelser - komposition i förändring} är ett gediget arbete. Det är en avhandling som placerar sig i den tradition som Göteborgs Universitet har etablerat för konstnärlig forskning i den nya konstnärliga examensordningen. Språket är vackert, nästan poetiskt, och reflekterande. Svaren är formulerade som möjligheter, där ibland många sådana ges. Frågeställningarna, till en början en smula vaga och inte omedelbart närvarande genom hela hela arbetet flyter ibland ut till utkanten för att sedan återvända och uppträda mitt i centrum igen. Här finns således många öppningar för fortsatt forskning. Arbetet kräver sin lyssnare och ju närmare man kommer det, desto tydligare blir det, men tydligheten uppenbarar sig närmast på ett musikaliskt plan snarare än ett strukturellt. Kim Hedås Linjer ger oss på så sätt ett ganska radikalt bud på individuell, intermedial konstnärlig kunskapsutveckling och jag vågar påstå att konstnärlig forskning i musik i Sverige ligger bra till.

\bibliography{/Volumes/500GB/Home/Documents/svn/admin/conf/biblio/bibliography} \bibliographystyle{plainnat}
\end{document}