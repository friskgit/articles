\documentclass{article}

\usepackage[english]{babel}
\usepackage{ucs}
\usepackage[utf8x]{inputenc}
\usepackage[T1]{fontenc}
\usepackage{hyperref}
\usepackage[pdftex]{graphicx}

\def\museincludegraphics{%
  \begingroup
  \catcode`\|=0
  \catcode`\\=12
  \catcode`\#=12
  \includegraphics[width=0.75\textwidth]
}

\begin{document}

\title{Linjer}
\author{Henrik Frisk}
\date{September  6, 2013}

\maketitle



.\label{1}Kims Disputation

\section{Sammanfattning av Linjer}

Kim Hedås avhandling heter Linjer: musikens rörelser - komposition i förändring. Det är en passande titel på en också omfattande avhandling som jag ska försöka sammanfatta. Sexton kompositioner och en bok på 310 sidor utan att räkna översättningen och referenserna ger en mastig upplevelse. DVDn är föredömligt tydligt producerad och jag är väldigt glad att se att den är gjord för ``läsning'' i en webläsare. DVDn innehåller även texten elektroniskt. Texten är föredömligt fri från språkliga fel och misstag vilket för en såpass omfattande text måste ha inneburit ett stort arbete.

De sexton kompositionerna är indelade i 5 linjer:

\begin{itemize}
\item Rörelse
\item Identitet
\item Tid
\item Minne
\item Rum.
\end{itemize}

Dessa fem linjer utgör också grupperingen av bokens 30 kapitel, följt av en avslutning, ett appendix och naturligtvis referenser. Den klara strukturen gör materialet tillgängligt trots sin omfattning. Som jag uppfattar det så har de fem linjerna vuxit fram ur forskningsprocessen och, så att säga, använts som ett raster för att gruppera musiken. Trots det är jag säker på att något av de senare verken har tillkommit /genom/ rastret snarare än tvärtom.

Jag ska försöka leda er genom min läsning av avhandlingen här genom att utgå från musiken och därifrån presentera relevant text varefter vi går tillbaks till musiken osv. Lite som ett Rondo och för enkelhetens skull så tar vi avdelningarna i samma ordning som de kommer i boken.

Raivadado som är sprunget ur ett annat verk, Kims musik till teateruppsättning av Dödsdansen 2007. Ett hänförande verk som lätt skulle kunna höra till alla 5 spåren eller linjerna. Genom att vara samtidigt idiomatiskt och naturligt pressar det gränserna och öppnar för ett lyssnande på flera olika plan.

\textless{}Raivadado\textgreater{}

Innan kapitlet om Raivadado finner vi i slutet av första kapitlet, efter Inledningen, finns två för kanske all forskning helt centrala ställningstaganden: ``Det är omöjligt att förbli anonym'' och ``Jag behöver gå in som mig själv''. I inledningen finner vi metoden och forskningsfrågan, definierad som:

``Hur kan förändringarna, som relationer mellan det som är musik och det som inte är musik ger upphov till, skapa möjligheter för komposition?''

Metoden, som vi ska återkomma till, beskrivs helt kort som ``En reflexiv rörelse mellan undersökningens olika delar''.

Kristevas intertextualitet, en uppenbar referens när temat är relationer mellan olika iterationer av konstnärligt arbete, kanske mindre så när det rör relationerna mellan icke-synkrona företeelser, lyfts fram som en immanent del av arbetet tillsammans med dialog, hypertextualitet, intermedialitet och ekfras. Ekfras kanske framstår som det mest användbara begreppet och det som närmast harmonierar med avhandlings tematik, men det förekommer bara i början, bl.a. i ett citat ur boken Intermedialitet från 2002 och i samma citat igen på sidan 260.

I fjärde kapitlet berör Kim förutom relationen musik/icke musik, liv/död och konst/natur. En biografisk del om isländske tonsättaren Jon Leifs används som struktur till vad som kanske kan ses som ett teorikapitel men som också är en inledning till Vattnet, ett körstycke som inte ingår i avhandlingen, men framförallt till Raivadado. Här hittar vi ännu en pusselbit i förståelsen av detta arbete i ett utsnitt ur en essä från ett tidigt skede av forskarutbildningen där hon skriver: ``Lika främmande känner jag inför att rekonstruera mina tanker genom åren. Jag har alltid varit i ett nu.'' En viktig förklaring som ändå visar på en förändringsprocess då avhandlingen jag har framför mig är en dokumentation av ett arbete som är sammankopplat via noder i flera dimensioner som inte kan ha producerats i ett nu.

Sen följer kapitlet om Raivadado och dess olika andra inkarnationer, förutom den vi finner på DVDn. Första delen avslutas med en reflektion om processer och förändringar. Det är första gången vi möter Bergson som känns som en viktig referens för Linjer. Flera av avhandlingens teman är centrala i hans arbete. Det är i detta kapitel som jag först förstår hur annorlunda denna text är, hur jag blir tvungen att ge upp att reduktionistiskt försöka analysera utsagorna, utan istället placera mig i textens flöde.

Andra delen börjar med\dots{} att börja. Att börja och sedan att forsätta, och om början som diskursens riskabla ordning som Cecilia Rosengren skriver. Deleuze terminologi för att komma runt de binära relationerna, blivandet, eller som på engelska flitigt förekommande becoming: becoming-other eller becoming-molecular. Vilket leder fram till en variant på forskningsfrågan: Hur blandar sig livet/världen/naturen sig i det som vi definierar som musik? Denna fråga, i relation till teorin som inledde kapitlet tänkte jag att vi skulle komma tilbaka till senare.

I kapitel 8 möter vi igen identifikationen med nuet fast nu med en ansträngning att se tiden som något sammanhållet. Men är inte ett stycke komponerad musik en bit sammanhållen tid som för komponisten fungerar som en markör som man kan återvända till? Jag har själv ofta slagits av hur fel mitt minne har haft, även om jag känslomässigt brukar kunna återvända genom musiken. Här närmar sig reflexionen även kognitiv neurovetenskap i funderingarna kring minne, perception och taktilitet samt kvantmekanik och når tanken om lyxen det innebär att ens musik landar hos någon annan, när Kim berättar om mötet med regissören som lyssnar - och hör - och spelar. Kanske kan vi kalla det intermusikalitet: mötet mellan två musiker/musiker som finner resonans.

\textless{}Historien lyder\textgreater{}

Historien lyder är en föreställning jag själv såg i Stockholm på teater Galeasen. Ett samarbete mellan Kim Hedås, Christina Ouzinidis, Teater Weimr och Ars Nova. Christina Ouzinidis, själv doktorand på teaterhögskolan i Malmö är en av dramtikerna som varit med att bygga upp Teatr Weimars rykte som en av de mer progressiva postdramatiska teatergrupperna i Sverige.

Utgånspunkten för Historien Lyder var ett hörspel och inget annat. Ett tomt blad som väntade på att fyllas. En början som den vi stötte på i kapitel två. Partituret är tio improvisationsmodeller med instruktioner, som på ett konceptuellt plan till en början påminner om Stockhausens Kurzwellen eller Plus-Minus men som senare läggs fast. Intialt improvisation som genom olika processer styrs mot fast form för tre instrument och tre skådespelare. Till de tre instrumenten kommer ett elektroakustiskt spår, inspelningar av rösterna, och, kanske viktigast, ett fjärde gemensamt instrument som definieras som trion i sin helhet. Processen beskrivs som en följd av händelser och ställningstagande men tyvärr får vi inte veta mer om relationen mellan text och musik.

I efterföljande kapitel 12 får vi däremot en diskussion om musikdramatikens identitet genom Hans Gefors avhandling och Kaija Saariaho opera L'Amour de loin, och i efterföljande avsnitt en fundering kring identitet där Kim skriver: ``För mig känns det problematiskt att skriva om mig själv, mina val, mina tankar, mina drömmar.'' Här finner vi också ett användbart ställningstagande när hon skriver att det inte är kompositionsprocessen utan tankarna bakom komponerandet och tankarna kring rollen som tonsättare och att kommunicera med andra som står i centrum.

\textless{}Illusion\textgreater{}

Illusion är ett samarbete med Petra Gipps och hennes Refugium som presenterades på Kivik Art Centre första gången. En ljud och arkitekturinstallation i naturen där lyssnaren omges av ljud från sex högtalare, musik utan början eller slut. Illusion inleder tredje delen ``Tid'' och utnyttjar bl.a. surroundljud med rumsliga förflyttningar av ljudet som, skriver Kim, är ``händelser i ett musikaliskt flöde'' och skapar positioner som ger både ``rum och tid'', en mening som leder mig tillbaka till musikdramatiken, via Schopenhaurs kausalitet och Wagners Parsifal och tid till rum transformationen (``Zum Raum wird hier die Zeit''). Här stöter vi på den övergivna idéns dramaturgi (s. 150: ``visade sig omöjligt och tur var väl det för idén visade sig vara usel''). ``I stycket Illusion har det förflutna funktionen att osäkra fortsättningen'' och här blir jag osäker om vi talar om installationen eller bara musiken (var går gränsen?) men Illusion framstår av den beskrivningen som en evig cliffhanger. Men detta är inte det hörbara, det ska inte vara det hörbara. Det är en struktur som tjänar kompositionsprocessen.

``Att börja är roligt. Att göra färdigt är tråkigare.'' (s. 158) Kapitel 14 är en reflektion över tid och komponerande. Återigen det tomma bladet. Kapitlet slutar i ett viktigt klargörande: ``Jag tycker musiken är viktigast'' (s. 162). Är den viktigare än dess relation till det som inte är musik undrar jag? Beskrivningen av musiken, som minnesanteckning för sig själv och för lyssnaren; hur kan detta tidsliga flöde antecknas? (Är det inte det som gör det så svårt? Att en beskrivning eller metafor just \emph{inte} är i tiden utan utanför tiden?)

Frågan om dokumentation kommer fram i kapitel 15 (s. 167), om än helt kort. En kritisk hållning till behovet av att dokumentera allt, viktigt som oviktigt, högt som lågt. Segreto är en ljudlåda och del i ytterligare ett projekt med arkitekten Petra Gipps och det leder fram till en annan typ av samarbete, det med designern Thomas Laurien. Musik och film. Två rum som ger varandra nya rumsliga möjligheter. Intermezzo är musik som har återanvänts och relaterar bakåt till andra verk av Kim Hedås (Möbelmusik, Still liv, Bröllopsmusik) och samarbetet presenterades hösten 2010. Intermezzo består av ``ljud som är hämtade från olika källor, ett visst sätt att behandla dessa disparata ljud genom kompositionsarbetet och vissa givna förutsättningar.'' Tyvärr får vi inte veta hur filmen och musiken påverkade varandra. Genom att Bröllopsmusik och Intermezzo är samma musik ser Kim hur förändringen sker genom kontexten och ställer frågan ``Vad gör det som inte hörs för lyssningen?'' (s. 181), en sats som får mig att fråga mig vad som egentligen /hörs/ och vad lyssnandet innebär. Kan man inte säga att det som påverkar \emph{lyssnandet} också \emph{hörs}? Svaret Kim ger är att själva musiken består av kombinationen av det hörbara och det ohörbara.

\textless{}Intermezzo\textgreater{}

Vad angår kombinationen av musik och film/grafik lyssnar/ser jag på avsnittet vid 3:09 och framåt. Tiden både rör sig framåt och står stilla, är fryst. Ungefär som en operaaria där handlingen, eller dramat, stannar upp men musiken fortsätter, och stannar upp. Och jag blir så otroligt nyfiken på hur kombinationen gjordes. Eftersom musiken redan var färdig, var det bilderna som synkroniserades? Eller är det en slump att detta förhållande uppstår? Vem gjorde vad? Hur? Vi får återkomma till det\dots{}

Det fjärde avsnittet, Minne, återvänder till temat fortsättning och förvandling. En bruten process, ett stycke som ska mixas om, allt är så självklart tills man sitter där då inget längre finns kvar annat än som en lätt dimma. Processen är inte längre åtkomlig. Det får mig att tänka på drömmen som när man vaknar är så självklar men i den stund man ska återberätta den är den bortflugen. Den finns på något sätt i minnet men inte i orden. Inte som metafor, utan på riktigt, men ändå inte tillgänglig. Freud kallar det primära och sekundära processer och det är i översättningen mellan dessa medvetandelager som det kan gå så snett. Konsten är en primär process och språket och medvetandet en sekundär.

Som hastigast flyger några korta stycken om jaget och den andre, Kristevas främlingskap och Rimbauds lek med första och tredje person i ``Je est un autre''. Kim skiver om behovet av att frövandla, först sig själv, sedan kompositionen för att kunna fortsätta. Men det är alltså framförallt en förvandling i jaget och medierat genom den konstnärliga praktiken Kim talar om, och inte mötet med den andre. Igen undrar jag vilken förändring detta möte innebar och hur det utvecklade sig. Var det motvilligt? Lätt? Svårt? Uppbyggande? Nedbrytande? Var det välkommet?

På s. 197 kommer en översikt över några av de som tidigare har ställt sig frågan kring musikens och konstens relation till andra element, som text. Kurt Schwitters t.ex., Stockhausen, Duchamp, Cage och Öyvind Fahlström. En mening med ett ensamt ``Ja'' ger mig känslan att Kim skriver under på tankarna kring uttryckens flyktighet. Och här ansluter avhandlingen implicit till dessa förvandlingar som nittonhundratalets konstliv kännetecknades av. Ord blev musik, konservburkar blev konst, påhittade språk blev litteratur, tystnad blev musik, pissoarer blev konst, maskiner blev musik, ordlekar blev poesi och oljud blev ljud blev musik. Är tjugohundratalets förvandling att musik blir forskning?

Men i kapitel 22 går det vidare, ifrån en förvandling från det ena till det andra. Här hittar vi behovet av repetition, kontinuitet och tradition. Och en avslutande mening fatsnar jag för: ``Redan första gången kan händerna i knät få impulser till att dirigera jaget in i lyssnandet, redan första gången kan musikens mening uppfattas och allt kan skickas blixtsnabbt, bredvid den reella tiden, in i minnet.'' När jaget dirigeras in i lyssnandet så är det inte, som jag läser det, frågan om en transformering från ett tillstånd till ett annat, utan något som läggs till något annat. Det är kroppsligheten, händerna rör sig, som skapar förutsättningen för att lyssnandet och jaget sammanfogas. Utan att fundera över hur ett lyssnande \emph{utan} jag kan fungera så närmar sig detta utvidgade lyssnande ett etiskt lyssnande i sin öppenhet.

Kapitlet avslutas med ytterligare en redogörelse för ett kroppsligt och taktilt förhållningssätt. ``Minnet i en röresle'', kroppsminnet och intoneringen, känslan av det kalla instrumentet, och därigenom ostämda instrumentet? Plågan när violinisterna inte stämmer som de borde. Men här slutar vi i lyssnandet som frigörelse från det kroppsliga: ``En lättnad, jag kan lyssna''

Sista och femte avsnittet, Rum, börjar med något som närmast kan liknas vid ett teorikapitel. Jag gissar att det delvis eller helt rör sig om en text producerad inom en musikestetisk kurs på SU eftersom ord som ``kurslitteratur'' förekommer. Temat är lyssnande och perception. Här möter vi bl.a. Jean-Luc Nancy genom hans bok Listening och Pauline Oliveiros genom hennes Deep listening. Texten är, som anges på sidan 219 ``ett försök att öppna upp och vidga de ramar innanför vilka förståelsen av lyssnandet till musik kan diskuteras. Vad som avses med ''denna text``, om det är kapitlet, avsnittet eller hela boken, är inte helt klart för mig men lyssnandet borde här kunna ersättas med lyssnandet och utövandet. och längre fram, på sidan 227, skriver Hedås att ''för att överhuvudtaget kunna diskutera musik, på ett konstruktivt sätt, måste frågorna öppnas upp: historiskt, politiskt, socialt och kulturellt.``

Kapitel 26 skulle jag behöva en smula mer information till för att förstå i detta sammanhang men det består av en serie poetiska texter med fotnoter till ganska långa textavsnitt. Några av dessa fotnoter har förkommit tidigare. Kapitel 27 är till stor del avskrifter av anteckningar inför de små ljudlådorna i Segreto, som jag förstår det är det första samarbetet med arktitekten Petra Gipps. Kapitel 29 är ockå det till stor del anteckningar inför själva samarbetet.

\textless{}Part\textgreater{}

I serien Skift som Hedås gjort tillsammans med Petra Gipps återfinns alltså

\begin{itemize}
\item Refugium/Illusion
\item Segreto
\item Part och Knot
\end{itemize}

Part är en fascinerande installation och karta över ett abstrakt fält, konkret tecknat. Jag förstår när jag läser att det finns mer material från det skedet då dessa verk växte fram och funderar över vad de skulle ha kunnat lära mig och egentligen är det synd att det inte finns mer av reflexiv rörelse här mellan part och knot och musik och arkitektur och mellan Kim och Petra och mellan de olika platser där dessa verk stälts ut.

Det är i samarbetet med Gipps som jag först förstår innebörden av forskningsfrågan. Kanske är det så det ska vara, när man tagit sig igenom avhandlingen så har man grepp om frågan som ställts initialt? Sista kapitlets inledande ''Musiken är`` (s. 276) känns spontant kontraintuitiv till detta arbete som berör frågor som förändring, möte, intermedialitet. Längre ner, dock, läser jag att ''avsikten inte är att ge slutliga lösningar på de frågor som är aktuella.`` ''Meningen är \emph{inte} att med ord besvara frågorna.`` (s. 279) En bra sammanfattning av de fem delarna följs av en kort genomgång av framtida utvecklingsmöjligheter och några appendix om konstnärlig forskning, om studiemiljön och om de teorier som Kim har vänt sig till eller genom. Efter en engelsk sammanfattning och referenser har vi nått till vägs ände.

Hur ser du på min genomgång Kim? Har jag träffat


\section{Inledning frågor}

En konstnärlig avhandling är ett arbete som är samtidigt en början, en fortsättning och ett avslut. Det är en början på ett nytt sätt att angripa sitt konstnärliga arbete. Den är en fortsättning på det konstnärliga arbete man har befunnit sig i under ett antal år (för det är i princip nödvändigt att ha en praktik innan man beforskar den). Och den är ett avslut på den första delen av en forskarutbildning som liksom det konstnärliga arbetet kan (ska?) befinna sig i ständig rörelse. Som avslut kan den ha ringat in ett område som man eller någon annan fortsätter att arbeta på, men den kan likaledes ha ringat in ett eller flera områden man inte kommer arbeta vidare med.

\subsection{Det som är musik och det som inte är det}

Förekommer fram till sidan 39, sedan 93, sedan 131, 273:

Vad som inte är musik, eller vad som är musik, får ingen direkt definition så jag tänkte vi kunde prata lite om det. Efter Cage, ja egentligen kan vi gå tillbaka till författaren, anarkisten och transcendentalisten Henry David Thoreau för att konstatera att tanken om att naturen och musiken står nära varandra, sedan mitten på 1800-talet. Cage, som följde upp Thoreaus tankar gick ännu längre och över huvud taget såg vi från första halvan av nittonhundratalet en upplösning mellan musik och text, musik och teater, musik och arkitektur och musik och teknik för att nämna några. Var finns skiljelinjen mellan musik och det som inte är musik för dig?

Hur blandar sig livet/världen/naturen sig i det som vi definierar som musik? Som jag berättade i min läsning hittar jag flera motsatspar samtidigt som det i kapitel 8, t.ex. finns en kort redogörelse för din positiva inställning till den nya fysiken där inte ens atomernas beståndsdelar nödvändigtvis rör sig i fasta banor. Du skriver om glidning mellan tillstånd och längst ner på sidan 86 skriver du om musikens öppenhet som gör det omöjligt att bestämma de tillstånd som musiken rör sig emellan. Hur kan vi då vara säkra på vad som faktiskt är musik och vad som inte är det?

En definition: ariktekturen för stå för det som inte är musik.

Du skriver på s. 273 ''Vilket ansvar kan musik och arkitektur ha?``

Först, om projektet med arkitektur har har visat sig vara en konstruktiv plats, vilka andra har du funnit? Var alla samarbeten värdefulla på sitt sätt eller finns det de som för dig och för din verksamhet har större bäring på ditt arbete? Idéerna i sammanfattningen på s. 273 är inspirerande och hela arbetet med ljudinstallationer verkar vara en kreativ smältdegel, men jag är fortfarande nyfiken på svaret på frågan vilket ansvar muik och arkitektur kan ha och på vilket sätt och vilken riktning relationen mellan det som är och det som inte är musik utvecklas i.

I sista kapitlet 30. finns en kort text som antyder att förklaringar kan vara konstruktiva trots att utgångpunkten är att vi ''inte tänker på förklaringen``. Denna dikotomi, mellan förklaring och icke-förklaring ansluter till andra som dyker upp; musik/natur, musik/inte musik, etc. står i kontrast till den teori du delvis använder, som Deleuze och Bergson, och Kristeva. På s. 36 tar du avstånd från Gadamer som hävdar att konst och natur inte är motsatspar (vilket iofs inte är det samma som att de är ens). Deleuze, genom Colebrook, vill också genom blivandet upplösa människa-natur distinktionen. Det här har jag svårt att få ihop, men det kan ha att göra med en fråga vi kommer till om en stund, men vilken betydelse har dessa binära relationer för dig?


\subsubsection{Part}

''Part kompletteras av ett schema där snitt/planer/vyer, musikanvisningar och texter från den konstnärliga processen visas.`` (s. 269) Vilka är dessa texter? Om vi nu ser avhandlingen som ett fält, kunde man inte ha lagt in de här på DVDn också? Tanken på en navigerbar ickelinjär massa tycker jag själv är ett attraktivt format för konstnärliga avhandlingar.

Musiken kombinerar reella ljud med syntetiska. En del barnröster igen. Klockor och föremål.

På sidan 289 skriver du om konsertversionen av Part att ''arkitekturen här är initialt inkomponerad i musiken``. I den engelska översättningen på sidan 322 är det annorlunda formulerat. Var det i reduceringen av musiken som du arbetade med att fånga in, sas, arkitekturen, eller hur arbetade du med det? Tanken på en version som utforskar det rumsliga i samband med arkitektur och som sedan redigeras för att utnyttja det akustiska rummets egenskaper för sitt uttryck är oerhört spännande. Hur fungerade dessa två typerna av rum med/mot varandra? Uppstod en konflikt? Den tanken ligger ju nära till hands eftersom konsertsalen ofta rymmer ett estetiserande, den gör musiken till mera musik, och jag kan föreställa mig att det lätt kan uppstå en motsättning?




\subsection{Jaget och musiken - metod}

(s. 189) Rimbaud: ''Je est un autre``: Rimbauds motsättning mellan första och tredje person ringar effektivt in problemet som han såg i att betrakta verkligheten i dualistiska termer, inifrån och ut eller centrum-periferi.

Kristeva: främlingsskapet inför sig själv härstammar från konflikten mellan urinvånaren och invandraren. När man inte är sitt ursprung och inte sin närvaro uppstår oförståelsen inför sin egen person. Det är från mötet med den andra som denna problematik härstammar och Kristeva skriver utifrån sina egna erfarenheter som invandrare i Frankrike dit hon flyttade från Bulgarien.

Du skriver om främlingsskapet gentemot processen. En process som blivit bruten av tiden och inte löngre är tillgänglig trots att minnet av den är klart. En annan möjlig tolkning av vad som hänt är att problemet du upplever har med skillnaden mellan minnet, eller det immanenta, och praktiken, själva skapandet. Du kan sätta dig in i processen i minnet, men inte i praktiken.

s. 90: Så fort det blir allmänt tappar jag intresset. Det behövs ett subjekt för att kunna komponera.

s. 132: Jag vill vara mig själv genom komponerandet.

s. 189: ''Möjligheten att tänka sig själv annorlunda, att göra sig annorlunda (för sig själv)``

s. 220: ''jaget som Nancy diskuterar är alltså inte ett personligt [\dots{}] utan mer som en nödvändig punkt som musiken förhåller sig till.``

s. 220--1 ''dyker istället musiken upp och blir det 'själv' utifrån vilket lyssnandet kan förstås.``

Var står du i relation till texten? Till lyssnandet? Till det som inte är musik?

\paragraph{Metod}

''En reflexiv rörelse mellan undersökningens olika delar``.

En viktig del av metoden, menar jag, är att undersökningen rör din musik och att den inte gör några anspråk på allmängiltighet, eller generalitet. Det är förstås en implicit del av den konstnärliga forskningen men eftersom du, som du skriver på s. 132, trivs bättre utan att behöva ''visa dig`` annat än genom musiken, att konstnärlig forskning inte behöver vara centrerad kring ett jag, men samtidigt tappar intresset när det blir för allmänt, så ser jag, möjligtvis helt felaktigt, en liten spricka eller ambivalens. Du talar också om musiken som ett subjekt. Var känner du att du står i relation till din avhandling?

Jag menar själv att det är helt avgörande för konstnärlig forskning att subjektet är en tydlig och öppen del av arbetet. Och det verkar vara din hållning också. Men är då inte titeln lite väl allmänn?



\subsection{Teori}

s. 208--9 Här har du citat som jag som läsare upplever hänger i luften. Utan att du behöver svara i detalj, just dessa citat, vad är syftet med att fälla in utsnitt ur andra texter utan att jag får veta hur du ser på dem, hur de enligt dig relaterar till tematiken som diskuteras?


\subsection{Dokumentation}

På s. 167 går du till frontalangrepp mot dokumentation och behovet av att dokumentera allt. KAn du utveckla det i relation till ditt arbete här?

JAg tycker själv att det är synd att jag inte får veta mer om arbetet ''bakom`` och innan det färdiga resultatet. Historien lyder t.ex.. Eller Segreto. Det känns som ett medvetet val att \emph{inte} dokumentera, men kan du berätta lite om det?


\subsection{Att diskutera musik}

På s. 227 skriver du: ''För att överhuvudtaget kunna diskutera musik, på ett konstruktivt sätt, måste frågorna öppnas upp: historiskt, politiskt, socialt och kulturellt.`` Jag är helt enig med dig om detta. Det är genom expansionen av fältet, genom att se hur frågorna förhåller sig till världen utanför, även till världen utanför det som inte är musik, som vi kan betrakta dynamiken i konsten. Ändå har dessa aspekter ganska litet utrymme i din avhandling. Det är främst genom Skikt det kommer fram (2. 273). Kan du utveckla de här dimensionerna, alltså det historiska, politiska, sociala och kulturella? (Även s. 227)


\subsection{Lyssnandet}

s. 231 Lyssnandet som återskapandet eller helt enkelt skapandet. Denna dimension, om lyssnaren som medskapare är inte så närvarande genom texten. Utifrån frågan ''Måste man förstå musiken?`` förs ett resonemang om musikens behov eller ickebehov av metaanalys.

s. 245 ''Som tonsättare är jag mest intresserad av musikens 'insida', men det betyder inte att jag är ointresserad av musikens 'utsida', som lyssnaren hör.`` Det framstår som om du ser det som så att musikens insida, den kreativa poietiska fasen, den projicerande, endast är något för tonsättaren medan lyssnare nås av ett objekt vars utsida de endast kan se. Nattiez och Molino's utskällda musikaliska semiologi pratar om en estesisk fas där lyssnaren ges möjlighet att återskapa verket. Roland Barthes tog också ett krafttag för att återupprätta läsarens status på bekostnad av författarens. Hur tänker du på lyssnarens roll?


\subsection{I valet mellan mycket och lite}

I min läsning och lyssning så kan jag lätt föreställa mig att det finns mycket mer att säga om i princip alla verken. Vad var anledningen till att du tyckte att så här pass mycket material skulle finnas med, snarare än att välja ut några få?


\subsection{Lyssnaren som medskapare}

s. 236


\subsection{Ljudets och bruets egenskaper}

s. 228


\subsection{Slutord}

Kim Hedås avhandling \emph{Linjer: musikens rörelser - komposition i förändring} är ett gediget arbete. Det är en avhandling som placerar sig i den tradition som GU har etablerat för konstnärlig forskning i den nya konstnärliga examensordningen. Språket är vackert, nästan poetiskt, och reflekterande. Svaren är formulerade som möjligheter, där ibland många sådana ges. Den huvudsakliga frågan ''Hur kan förändringarna, som relationer mellan det som är musik och det som inte är musik ger upphov till, skapa möjligheter för komposition?``, till en början en smula vag och inte omedelbart närvarande genom hela hela arbetet flyter ibland ut till utkanten för att sedan återvända och uppträda mitt i centrum igen. Här finns således många öppningar för fortsatt forskning vilket är ett viktigt kriterium för en god avhandling. Arbetet kräver sin lyssnare och ju närmare man kommer det, desto tydligare blir det, men tydligheten uppenbarar sig närmast på ett musikaliskt plan snarare än ett strukturellt. Kim Hedås Linjer ger oss på så sätt ett ganska radikalt bud på individuell, intermedial konstnärlig kunskapsutveckling. Det ser ljust ut för konstnärlig forskning i musik i Sverige.





\subsubsection{Sand}

Varför bara ett utdrag? Otroligt vackert.


\subsubsection{Dödsdansen}

Svårt att ta ställning till utan dramat? Kim spelar piano!


\subsubsection{Bara, B til GDH}

Stillastående och slående lika. Är det olika musik?


\subsubsection{Illusion och Illusion/Refugium}

Poetiska.


\subsubsection{Segreto}

Stillastående, men effektiv som tidsrörelse


\subsubsection{Intermezzo}

Hänförande: 3:09 Enastående välgjort som arian i en opera. Tillstånd som hålls fast i.

sid 180: Genom att gå in på detaljer\dots{}


\subsubsection{Under luften}

Märklig musik. Nästan ingen kommentar.




\subsubsection{Knot}

Kör. Varken Part eller Knot är vidare väl beskrivna. Vilken är t.ex. kören? Är det din musik?



\subsection{Metod}

En reflexiv rörelse mellan undersökningens olika delar






\section{Errata}

s. 270: Länk till musikverket längst ner: \url{http://statensmusikverk.se/artikel/musik-moter-arkitektur/}

s. 129: ''I inledningen till detta textmaterial\dots{}`` men det som åsyftas finns på sidan 301.

s. 294: ''Guettari`` ska vara Guattari.

s. 214: ''En tredje``, sjätte raden uppifrån. Vad är det som har räknats upp till tre?

s. 220: ''utan istället`` stryk istället.

s. 99



\end{document}
