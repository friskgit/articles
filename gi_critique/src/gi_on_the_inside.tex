% Created 2020-10-29 tor 12:56
% Intended LaTeX compiler: pdflatex
\documentclass[11pt]{article}
\usepackage[utf8]{inputenc}
\usepackage[T1]{fontenc}
\usepackage{graphicx}
\usepackage{grffile}
\usepackage{longtable}
\usepackage{wrapfig}
\usepackage{rotating}
\usepackage[normalem]{ulem}
\usepackage{amsmath}
\usepackage{textcomp}
\usepackage{amssymb}
\usepackage{capt-of}
\usepackage{hyperref}
\usepackage[english]{babel}
\usepackage[lf]{ebgaramond}
\usepackage{sectsty}
\allsectionsfont{\sf}
\hypersetup{colorlinks=true,linkcolor=black,urlcolor=black}
\usepackage[style=authoryear-ibid,natbib=true,backend=biber,hyperref=false]{biblatex}
\renewcommand*{\nameyeardelim}{\space}%
\renewcommand{\postnotedelim}{: }%
\bibliography{./gi_biblio.bib}
\setcounter{secnumdepth}{0}
\author{Henrik Frisk, henrik.frisk@kmh.se}
\date{\today}
\title{On the intuition of a machine}
\hypersetup{
 pdfauthor={Henrik Frisk, henrik.frisk@kmh.se},
 pdftitle={On the intuition of a machine},
 pdfkeywords={},
 pdfsubject={},
 pdfcreator={Emacs 26.3 (Org mode 9.4)}, 
 pdflang={English}}
\begin{document}

\maketitle

\section*{Introduction}
\label{sec:orga947df7}
One of the great challenges of any research project that spans over several years is to keep the project contained and avoid it from going off in directions less useful for answering the questions posed. Yet, it has to be open enough to allow for unexpected findings in the fringes of the practice. This is especially true of artistic research projects, as they have a tendency to be interdisciplinary and sometimes broad-brushed, which in fact may be seen as one of the qualities of the field. Furthermore, the artistic practice and its contexts, which may sometimes itself be difficult to delimit, are at least at the outset the original bounds of an artistic research project. The difficulties, however, remains to know when a trajectory should be given up or stayed with: When is this particular issue exhausted in the context of the project? It is through method development that a field of research practice can evolve, and we, as artists and researchers, can learn to become better at making those decisions.

The main purpose of this essay is to explore the methods used in the artistic research project \emph{Goodbye Intuition} from the point of view of the often repeated claim that artistic research is about acquiring knowledge from within the process, from the inside.\footnote{Since I have been an active part in many of the workshops and laboratories performed by the group \emph{Goodbye Intuition} from the initiation of the project in 2017, I am obviously involved in, and influenced by, any findings and discussions and in that regard I have been on the inside of it. In some of the workshops I also played with the group, but in general my role has been that of an external partner, or critical friend.} This claim is often put forward as opposed to scientific research in which, in this extremely simplified model, the researching subject is not as often entangled with the object of research. This is not only a simplified model, it is also not accurate as has been discussed in numerous contexts.\footnote{See \citet{frisk-ost13} for an overview.} Exactly what constitutes this privileged \emph{inside}, or how it is made accessible to an artist is perhaps less commonly addressed. On the surface it is perhaps principally a way to explain the central ambition of much artistic research, namely to research the artistic process. This is achieved either by doing it while being engaged in it, or by working with documentation of the practice that allows the researcher to reenter the process. The research performed in \emph{Goodbye Intuition} has been performed in both of these modes, but it is primarily the work that has taken place in the actual artistic process that is the main concern here. I am not concerned with the results of \emph{Goodbye Intuition} here, nor with the actual process other than in methodological terms. For this I would like to direct the reader to the paper \citetitle{Frisk2020} \citep{Frisk2020} as well as to other texts on this website.\footnote{See \url{https://www.researchcatalogue.net/view/411228/424771} (visited on October 12, 2020)}

At a larger scale the definition of the outside/inside divide points to there being an epistemological difference between these positions that rests on there being a knowledge in artistic practice that is only accessible from the inside, and another kind, accessible also from the outside that is primarily oriented towards analysis. Research on art as this latter form is sometimes referred to has been more concerned with traditions--their developments and disruptions--than with finding the what the nature of the art is from the inside, although there are examples of that too \citep[see e.g. ][ ]{monson96}. This has probably something to do with the particular status that this \emph{inside} has, and the ways in which it is inaccessible for everyone except the artist--why else would it be important that artists do the research? This is somewhat at odds with a common view on scientific analysis where the researcher is \emph{outside} the thing analyzed which thereby allows for an objective stance. But, as is well known by now, neither position is a good description of how research is performed. However, for the following discussion there are some interesting aspects of the way that deduction and analysis is traditionally understood.

The \emph{inside} is often mystified and given a shimmer of romantic light, seen as the sight for true creativity. This convention should be challenged by artistic research, but, unfortunately, is sometimes instead reinforced by it. The main challenge lies in the lack of credible research methods with which the \emph{inside} can be properly investigated, and this should be a focus area for artistic researchers for the years to come. This essay is an attempt to open up the discussion and explore the notion of the \emph{inside} as a site for metaphysics that has much in common with other human activities. The French philosopher Henri Bergson and his theory of duration, metaphysics and intuition constitutes a point of departure when looking at the methods employed and developed in the project \emph{Goodbye Intuition}. One of the advantages of his theory is that the gap between metaphysical experience and analytical knowledge is reduced, and by placing oneself on the \emph{inside} of the object one may experience it in the process of its unfolding. Bergson is particularly interesting through the unified theory he proposes in which the possibility for an experience from the \emph{inside} is not a privilege only to a practitioner, such as an artist, but a method for any type of experience. 

\section*{On the experience of machine improvisation}
\label{sec:org5be1141}
The meaning of the expression \emph{Goodbye Intuition} is on the one hand, at least for a musician that has engaged in improvisation, easy to understand. There is something stifling about the need to be intuitive when music making so often is about knowledge, skill and experience. On the other hand, trying to make music together with other musicians without intuition would be extremely difficult, if not impossible. So, is the project \emph{Goodbye Intuition} then reducible to the attempt to raise awareness on the various roles that intuition may have in musical practice? No, it is more complex than that, and the discussion here touches on a wide range of topics that are all related to concepts like interpretation, communication, narrative, form, content, and style. At the center of the interrogation is an improvising machine developed during the course of the project, henceforth named \emph{Kim-Auto}, or KA, whose purpose it is--among other things--to provoke the discussion between the project members and move it out of the various safe zones that all musicians build up for themselves. That being perhaps the practical aspect, the machine also provided intellectual resistance. The quality of the possible music afforded by improvising with KA was hard to judge: By which standards can the output of an improvising machine be judged without also judging ourselves? Who has the power of authority on what is "good music", and how transparent are these roles? What is the actual autonomy of KA and how does it affect ours?

These questions can then be expanded and used to further our understanding of much larger questions pertaining to the giant revolution in which data is the currency used to gain economic, social and political control. It is an enormous transformation where human labor is replaced by automated processes, and where transactions are even more abstract and opaque than ever. This change has been described and almost fetishized since over a century in literature and media in a way that makes it difficult to discern whether it is the fiction that has produced the future or if the future simply was predestined in a way that made it easy for the science fiction narrative to project it. Unless we subscribe to determinism, the most rational explanation appears to be that the present is so influenced by culturally iconic descriptions of the future produced portrayed in the past, that these are realized. That many of these commonly has had a negative connotation, such as \emph{Skynet} created by \emph{Cyberdyne Systems} in the \emph{Terminator} movies, or the surveillance society of Orwell's \emph{Nineteen Eighty-Four} may not have mattered. The benefits of this future may still have appeared to be greater than the risks. This discussion is obviously closely related to the huge topic of artificial intelligence (AI) and AI in music production has had a long history. Specific technologies have been created for different reasons and with different purposes for many years already. AI is already used in many areas of music production: audio processing plug-ins, composition tools, analysis tool and many other types of software explore different kinds of machine intelligence in order to assist musicians to better perform their tasks. To distinguish what part of the result is actually the outcome of human, rather than machine intelligence, is not a distinction that is easily drawn, nor is it necessarily an interesting one. As long as the intelligent machine has been created by humans the distinction is mainly one in time: AI can be seen as merely a delayed action determined by a human programmer. This may not be easily detected though, as a feature of connectionist strategies for artificial intelligence is that the structures that guide them are hidden and perhaps one of the greater dangers with AI is to think that it is the machine that is intelligent.

However, the use of technology in artistic practices, intelligent or otherwise, can obviously have a range of different objectives. First, from a general standpoint one may argue that art should engage in available technologies for the simple reason that this contributes to our understanding of its social and cultural impact. Though this general notion is sometimes contested, most famously by Heidegger who instead warned about the ways in which technology frames the human capacity \citep{heidegger93}, it may be considered equally reasonable to claim that AI should be a field for artistic exploration. Secondly, from the point of view of innovation, it has been seen in the past that artists' use of technology push the boundaries for what is possible \citep{harris1999}. Although this has arguably been true, the resources that the multinational technology and media industry now control, along with the increased  technological complexity, make it more difficult for an independent artist, or even an institution or a university, to invent methods of production that can compete with the powers of these companies. The artistic qualities in and of themselves may be uncontested, but as far as creative power is concerned the playing field has changed. Thirdly, the challenge to create a machine that is able to compose music with the same level of integrity, witfullness and inspiration as that of a great musician may be seen as a test of the potentiality of the technology. A machine that can compose music is likely to be able to perform a whole range of tasks with a high level of proficiency. However, unlike, say, a self driving car, it is difficult to fully grasp the (economic) value of such an achievement. Finally, there is an underlying ethical dilemma that should concern the whole field of AI, also the field of music. The emergence of AI depends on designers and programmers that create the systems, and as has been shown through studies such as \citet{snow2018} technology as a valueless blank slate needs to be contested. With an increased use of intelligent technology in artistic practice, there is simultaneously a broadening of the practice where not only the programmer of the software but also the designer and manufacturer of the hardware should be included as significant agents in the artistic development.

A rudimentary machine such as KA, an automated improviser, may have nothing directly to say about the dystopic development of an economy governed by data acquisition, but it nevertheless opens up a window to the question of an ethics of communication technology in a way that is of relevance for the wider discussion of this field. The sensibility of a musician improvising  with another musician is an activity that is still exceptionally difficult to model in a machine, and one that we still know very little about. It can by itself be seen as one of the critical things that still separates us from machines and that cannot be reduced to a statistical or analytical model. But to study this, and better understand interaction between musicians in performance will provide us with data that may prove to be invaluable in understanding how to construct ethically sustainable human-machine, or machine-machine systems. One aspect of this that points to the complexity of the present study, and that the method used in \emph{Goodbye Intuition} allowed us to explore, is the various ways that we as musicians anthropomorphize the machine. The lengths to which a human musician is willing to go in order to attempt to understand the other is fascinating and provocative. In other words, it is sometimes difficult to tell whether a reaction from a musician playing with a machine is primarily coupled with the machine behaving like a human, or whether the human performer produces the idea of a fictional man-machine that responds like a human. In musical performance the subject feels sympathy for the other even if the other is a machine. I will discuss this in more detail further on in this text, but for now it will suffice to say that the particular relation that a musician has to whoever, or whatever, they are listening to and interacting with has a big impact on the turnout on how one may understand the interaction. That the object for interaction is a thing, or a piece of software, makes this relationship more difficult to establish, but by anthropomorphizing the machine the relationship is easier to comprehend for the subject. Hence, the machine in this case is not a means to an end, or merely an object for interpretation. It becomes the \emph{other}, possible to sympathize with, and this point is important in the discussion in the next sections.

At its best musical interaction is one in which the other is understood through a constant negotiation. This understanding does not require knowing what or who the other is, it rests on an ethics of listening, and on the willingness to engage. The main interest may be said to be focused on the question on what the music \emph{does}, rather than what it \emph{is}, which is a central theme in Susan Sontag's \citetitle{Sontag1986} that will be discussed more later. Far from all musical interaction fulfills this objective, but all possible interactions have the potential to do so. For these reasons music specifically, and artistic practice in general, are useful practices to study in order to better understand the impact that the quality of the communications with machines may have. Meanwhile, to return to the introductory remarks on the epistemological status of artistic research, a useful, albeit not conclusive, definition for what artistic research in music is could be summarized as in the following: The study of what artistic practice does based on the assumption that the \emph{doing} in the practice rests on a knowledge that may be communicated. There is a some empirical justification in the assumption that the \emph{doing} of the music has an epistemological impact on our well being. Judging from the many testimonies that music has the potential to move us, engage us, and that it has a deliberating effect on us, or may make us angry, happy, sad or euphoric \citep{viper20,horwitz18,juslin19}, it is not difficult to understand the curiosity in finding out what exactly what it is that arouses our sensibilities and emotions. In other words, one of the primary drives behind artistic research is not \emph{if} there are emotional responses connected to practicing and listening to music, but rather \emph{how} these responses are established.

\section*{On intuition}
\label{sec:orge977bb5}
French philosopher Henri Bergson sought to address the problem of what knowledge one may have of the world exterior to oneself, and one of the central tenets in this effort is the method of intuition. It may appear bordering on defensive to get into an in depth investigation of the term \emph{intuition} in a project that that has attempted to do away with it altogether. In the way that it is used in the title of the project, however, it points to the ordinary use of the word, the meaning of which can be summarized as the "ability to understand or know something immediately based on your feelings rather than facts."\footnote{Intuition. (n.d.). In \emph{Cambridge Dictionary online}. Retrieved from \url{https://dictionary.cambridge.org/dictionary/english/intuition}} Though not completely unrelated, Bergson's notion of \emph{intuition} is rather different, and therefore also of interest to the present discussion. Intuition as a method will by necessity include also other modes of thinking than what is given by the more common view of intuition. The point here, though, is not to give a full account of Bergson's philosophy, nor of the method. In order to better understand its kinship to the method development in \emph{Goodbye Intuition}, and the potentiality of the work that was performed within the project, I will give an overview of what it entails. Anyone with a particular philosophical interest, however, should obviously go to the sources for more information. Furthermore, the method contains certain idiosyncrasies that may not always fully harmonize with the way the project has developed.

To understand intuition in the a Bergsonian way it may also be necessary to contrast it with other uses and definitions of intuition. The more general interpretation of intuition, as has already been alluded to, relates to the things we do without thinking about them; the intuitive knowledge that something is, for example, wrong or dangerous. In this sense intuition may be likened to an automated system that gives us pre-reflected information about what is going on in the world around us. In phenomenology intuition has a slightly different meaning. Intuition gives the subject first-person knowledge whereby an object can be said to be \emph{intuited}. Kant's distinction between concept and intuition may confuse the field even more though it is clearly a point of departure for Bergson's argument here. It is however in contrast to Kant that Bergson defines his own theory of metaphysics.

In his essay \citetitle{Bergson1912} Bergson defines two incommensurable ways to approach an object: either from a point of view through signs and concepts--a \emph{relative} perspective--or through entering into the object, exploring it from the inside--an \emph{absolute} apprehension. This second method is achieved by entering into a sympathy with the possible states of the object which allows him, as he describes it, to "insert myself in them by an effort of imagination." \citep[p. 2]{Bergson1912} This, he continues, enables him to "no longer grasp the movement from without, remaining where I am, but from where it is, from within, as it is in itself." \citep[p. 3]{Bergson1912} The latter is what he refers to as the \emph{absolute} knowledge: "the absolute is the object and not its representation, the original and not its translation, is perfect, by being perfectly what it is" \citep[p. 5-6]{Bergson1912}. This is a quite radical proposition, in contrast to the dominant theories of mediated perception. The concept of actually being able to possess the object, or rather, its motion as Bergson will say, in itself makes possible a range of conceptions.

The example that Bergson gives us to better understand what he means by representational knowledge is a photographic model of a city. One where all angles and all surfaces have been photographed and documented to achieve something similar to the street view that online map programs sometimes offer. Exploring such a model can obviously never be equated with the experience of being in the city and is hence relative knowledge of the city. Another example given is the translation of a poem into different languages.\footnote{As Swedish artist Andreas Gedin has proved, sequential translations of a poem into multiple languages changes the character of its content.} Each such translation can give the reader a good idea of the meaning of the poem, sometimes with even more details, but it will never form a complete image, and would "never succeed in rendering the inner meaning of the original" \citep[p. 5]{Bergson1912}. One of the central propositions is that analytical knowledge as is the result of a relative perspective is always a reduction of the thing being analyzed. \emph{Analysis} is what might be referred to as an empirical understanding of which a defining aspect is its reductive force. By dividing the object into ever smaller elements we may create analytical precision, but that eventually resonate more with what we already know than with the thing we analyze. The thing analyzed then becomes merely a function of the receiver rather than an object in and of itself. Such an analysis creates representations that are discontinuous in the flow of time:

\begin{quote}
In its eternally unsatisfied desire to embrace the object around which it is compelled to turn, analysis multiplies without end the number of its points of view in order to complete its always incomplete representation, and ceaselessly varies its symbols that it may perfect the always imperfect translation. It goes on, therefore, to infinity. But intuition, if intuition is possible, is a simple act. \citep[p. 8]{Bergson1912}
\end{quote}

The \emph{absolute} is given from this \emph{intuition} and an \emph{intellectual sympathy} with an object that allows one to perceive its unique qualities beyond words. Hence, the \emph{perfect absolute} should be understood in relation to the \emph{imperfect analysis}. Metaphysics is what allows one to experience what it means to have an \emph{intuition} of the object at hand. To Bergson, the science of intuition is metaphysics, and metaphysics is "the science which claims to dispense with symbols" \citep[p. 9]{Bergson1912}. Rather than engaging in representations of objects by means of symbols and concepts, we are urged to understand the object from within. This may be compared to self-reflection and Bergson gives a description of the various strata of introspection in this process towards the center of the self. Moving from the outside, through a protecting "crust" made up of all the perceptions from the outside world, via memories that are interpretations of the perceptions, down to the motor habits that are both connected and detached from memories and perceptions. But at well the core, Bergson describes the continuous flux of a concatenation of states, the beginnings and ends of which are merged together and extend back and forward. The metaphor used here is that of a coil constantly unrolled and rolled up again, although, admittedly, this comparison is far from perfect. There are no two identical moments in consciousness and the rolling up of the coil may thus be misleading. Even going back in memory to a past event invades that memory with all the present and past events. This evokes a passage in his earlier work, \citetitle{bergson91}, of the cone whose tip is moving over a moving plane, and where the point of the cone represents the present and the cone itself the accumulated memories and experiences: 

\begin{quote}
The bodily memory, made up of the sum of the sensori-motor systems organized by habit, is then a quasi-instantaneous memory to which the true memory of the past serves as base. Since they are not two separate things, since the first is only, as we have said, the pointed end, ever moving, inserted by the second in the shifting plane of experience, it is natural that the two functions should lend each other a mutual support. So, on the one hand, the memory of the past offers to the sensori-motor mechanisms all the recollections capable of guiding them in their task and of giving to the motor reaction the direction suggested by the lessons of experience. It is in just this that the associations of contiguity and likeness consist. But, on the other hand, the sensori-motor apparatus furnish to ineffective, that is unconscious, memories, the means of taking on a body, of materializing themselves, in short of becoming present. \citep[p. 152-3]{bergson91}
\end{quote}

The sensory motor-habits are informed by memories through which they will be guided to do the work they are set out to do, and because no single memory is ever stable--it is always altered by the present in the interaction between what Bergson refers to as the "pointed end" and the past memory--the present experience is continuously altered by past experience which in turn is influencing the present. Hence, the general understanding of intuition, as an action guided by feelings rather than by facts, may be said to still hold true, but a comparison quickly becomes very difficult without defining what constitutes a \emph{feeling} or a \emph{fact}. More interesting, however, is the connection brought up between sensori-motor mechanisms and past experience, and the fact that this connection is not only going one-way, from memory to habit, but also from habit back to memory. Embodied memory is in a changing flux and in constant interaction with experience and habit. There is an inclination to understand learned behavior, such as playing an instrument or lifting a glass of water, as a pre-conscious act, almost as if independent of reflection. To this there is a whole range of relevant research in embodied cognition and consciousness that is outside the scope of this essay. Here the purpose is mainly to look at intuition as a method for a better understanding of musical interaction with electronic instruments.

It is in thinking about embodiment and motor-habits that Bergson's understanding of what an intuition can be is perhaps best understood. If I move my leg or my hand I can only access the information that guides this movement through intuition. Analyzing the movement fails to understand it completely since the analysis only pins the movement to a sequence of states: first the arm was here, then here, then there. The actual change, the mobility or, as Bergson would put it, the duration, is only possible to understood through intuition. Furthermore, any new experience within a movement, and any past experience will introduce change:

\begin{quote}
When you raise your arm, you accomplish a movement of which you have, from within, a simple perception; but for me, watching it from the outside, your arm passes through one point, then through another, and between these two there will be still other points; so that, if I began to count, the operation would go on forever. \citep[p. 6]{Bergson1912}
\end{quote}

Bergson is again critiquing an analytical understanding as an activity that is not able to fully understand movement, but only apprehend a sequence of states and concepts. Conceptual knowledge is reached by way of concepts that represent the thing of which knowledge can be said to be had. These, says Bergson, have the "disadvantage of being in reality symbols substituted for the object they symbolize" \citep[p. 17]{Bergson1912}, and as such they demand very little of us. Furthermore, each concept merely expresses a comparison between itself and the object that resembles it. By way of this resemblance we imagine that by taking concept after concept and putting them side by side we are actually reconstructing the analyzed object. Through analysis we create concepts and symbols that allow us to structure and organize the world in a systematic manner. If metaphysics, then, is a serious science by which the world can be understood and experienced, rather than just a mind operation, "it must transcend concepts in order to reach intuition" \citep[p.21]{Bergson1912}. I have learned to move my arm, and every new piece of information about what I can do with it will add to my arm-moving-knowledge, and intuition is the modality through which this process is carried out. For a subject able to observe the thing from the inside, intuitively, there are no states, only duration and mobility informed by experience and knowledge.  

The methods of analysis and the method of intuition proposed by Bergson has important relevance to the field of artistic research in general, and to the project \emph{Goodbye Intuition} specifically, though perhaps less because of the project's title and more due to the way in which the method development has evolved over time. As was discussed in the introduction, one of the defining ideas of artistic research is that there is a difference between knowledge that has been acquired from observing an artistic practice, and knowledge that is the result of practicing art. A common metaphor used to described the difference between these two modalities is to see the former as knowledge from the outside, and the latter as knowledge from the inside. Simplified like this the comparison becomes problematic, as the concepts of \emph{inside} and \emph{outside} are rarely stable, and difficult to determine. Also, the word \emph{inside} here may be confusing as it has not the same meaning as Bergson's \emph{inside} but, as we will see, may coincide in a useful way. The inside as a qualifier of a perspective is here unique to the artist, whereas in Bergson's method it is a possible perspective for any kind of experience. In Sören Kjörup's contribution to \citetitle{biggs10} he comments on what may be the difference between the outside and the inside perspective claiming that:

\begin{quote}
if artistic research is supposed to be different from all other kinds of research, it is natural to focus on the artist as the researcher, and what is specific for the artist is her or his privileged access to her or his own creative process. \citep[p. 25]{kjorup10}
\end{quote}

This "privileged access" may be seen as an alternative to knowledge mediated by symbols and concepts, but a particular difficulty lies in the relation between traditional epistemology and this inside perspective. One of the recurring themes in the early discussions on the identity of artistic research was, and still is, what kind of knowledge artistic knowledge and methodology could lead to, and what kind of relation it would have to other kinds of knowledge. How can artistic research as a discipline make sense "in a way that does not subordinate [it] to other activities of thought taken to be superior modes of knowing" \citep[p. 150]{johnson2010}? Johnson further stresses that the potential in this kind of knowing lies in the experience, rather than empirical knowledge:

\begin{quote}
The \emph{research} here would not be geared toward the accumulation of empirical facts or propositional knowledge, although that might be part of the story. Instead, \emph{arts research} would be inquiry into how to experience and transform the unifying quality of a given experience in search of deepened meaning, enhanced freedom, and increase of connections and relations. \citep[p. 150]{johnson2010}
\end{quote}

Henk Borgdorff is another often mentioned reference and has been one of the key figures in defining the development of European artistic research. His claim that "it is not formal knowledge that is the subject matter of artistic research, but thinking in, through, and with art" \citep[p. 143]{borgdorff2012} has gained traction in the field with its focus on creating art. It is possible to contest the proposed opposition between formal knowledge and what may be extracted from thinking in, through or with art, but what Borgdorff is pointing to is that there is a possibility for a "special articulation of the pre-reflective, non-conceptual content of art." \citep[p. 143]{borgdorff2012} This may be understood as leaning on a Kantian notion of aesthetic experiences as pre-conceptual, invoking a free play of imagination, with the consequence that the kind of mediation through conceptual and symbolic representation that characterize most forms of knowledge is sidestepped. This creates a number of conflicts, the most obvious of which is the question of what kind of knowledge we are dealing with if the art (practice) is understood directly, in and of itself. How can something that evades conceptualization at all be represented in a stable manner? In other words, the question is how this unmediated experience may be useful to the artistic researcher and, more importantly: how may it be described in a way that makes the results relevant for other researchers?

This is, without a doubt, one of the more pressing and still controversial questions in artistic research and the point on which this discipline is most often criticized, but it is not my aim to address it directly here.\footnote{For a more complete discussion on this topic, see \citet{frisk-ost13}.} Instead, my focus is to approach it through  Bergson's method of intuition in the context of the method development of the project \emph{Goodbye Intuition}, how I have approached the method in my attempt to understand what KA is and can be in an artistic practice. Bergson's method of intuition may contribute to showing that not only is it possible to gain formal knowledge from artistic research in a methodologically sound manner, but also that the difference compared to other fields of knowledge production is perhaps less significant than what is commonly believed. Artistic research could in this regard be seen as a possibility to widen the perspectives of how the formation of knowledge may take place.

Bergson's description of the two profoundly different ways of knowing a thing, through \emph{relative} and \emph{absolute} knowledge may at first adhere to the idea of the difference between conceptual and non-conceptual knowledge. As was described above, however, it is in fact in contrast. According to Bergson, through the method of intuition, and by way of \emph{sympathy}, it is in every way possible to achieve absolute knowledge about a thing through an understanding from the \emph{inside}. Without this inside access one is left with the option of analysis from the outside, and regardless of how many different perspectives this analysis is performed from, it will never fully capture the true \emph{motion} of the object resulting in relative knowledge. Yet, this is the faculty of knowledge that is often seen as the principle mode of thinking. The contradictions between this and the intuitive, absolute, knowledge that Bergson is arguing for:

\begin{quote}
arise from the fact that we place ourselves in the immobile in order to lie in wait for the moving thing as it passes, instead of replacing ourselves in the moving thing itself, in order to traverse with it the immobile positions. They arise from our professing to reconstruct reality--which is tendency and consequently mobility--with precepts and concepts whose function it is to make it stationary. \citep[p. 67]{Bergson1912}
\end{quote}

One central aspect of the distinction between analytical and intuitive knowledge made here is that the intuitive, being in the motion or the duration, can always develop concepts and form the basis for analytical knowledge, whereas it is impossible to reconstruct motion from fixed concepts. An analysis may result from intuition, but intuition cannot arise from analysis. The analysis is performed on one particular state of the duration, and from multiple analyses or states it is possible to imagine that the mobile may be reconstructed by simply adding the different states together. This is however the critical point that Bergson objects against: It is only through intuition that the variability of reality may be fully experienced as mobility, and a succession of states is radically different in nature. It is a series of frozen frames of time, one slice at a time. The error in thinking that reality may be accessed through analysis, claims Bergson, "consists in believing that we can reconstruct the real with these diagrams. As we have already said and may as well repeat here--from intuition one can pass to analysis, but not from analysis to intuition" \citep[p. 48]{Bergson1912}

One important aspect of this is how the whole relates to the parts. Following the reasoning from above, dividing a whole up in parts is possible, but reconstructing the whole from these parts is utterly impossible. Again taking the example of the moving hand, it is possible to analyze its movement as a succession of states, but impossible to reconstruct the motion from these states. This holds true regardless of the precision and resolution of the collected frames--there will always be a space between current and previous frame. This corresponds to the Aristotelian view that the whole is always prior to the parts and that things can not be reduced to parts without loosing a significant part their identity.\footnote{See Aristotle \emph{Politics}, \RN{1}.2 and \emph{Metaphysics} \RN{7}, 1035b.} The hand is not a hand, except by name, unless it can perform the function of a hand. If the hand is removed from the body it is only a hand by appearance:

\begin{quote}
What a thing is always determined by its function: a thing really is itself when it can perform its function; an eye, for instance, when it can see. When a thing cannot do so it is that thing only in name, like a dead eye or one made of stone, just as a wooden saw is no more a saw than one in a picture. The same, then, is true of flesh, except that its function is less clear than that of the tongue. \footnote{See Aristotle, \emph{Meteorology} IV, 12}
\end{quote}

In essence this means that the hand as an object is less than the hand that performs its function, and that the hand, when still a part of the body, obviously stretches out beyond the merely physical hand; it is connected to the arm with muscles and tendons and to the brain with nerves. This is true of most instruments that has a specified function. A hammer can be said to only be a hammer when it performs as a hammer, and a piano is only a piano when it is performed by a pianist. The hammer as a sign is signified by both the physical hammer in general, and the hammer that performs a function, although the latter, of course, is dependent on a whole range of other things and energies in order to perform. From a more personal perspective, for me as a saxophonist, my saxophone is similarly an object--a saxophone, or a musical instrument. However, once I pick it up and start playing it, it is a rather different thing. In my hands it becomes an extension of my body similar to how the hand is a part of my body. This may be primarily a result of the many hours of practice that I have performed on it, and as such it becomes an instrument of my musical ambition. But as soon as I put it down it is nothing more than a piece of brass again, and in that state it is as useless as the severed hand.\footnote{There is obviously a whole range of social, cultural and political connotations attached to the immobile object which commonly allows us to contextualize it in a way that allows for a much wider impression of the instrument than that. The purpose here is to compare the object in use with the unused object.}

It may at first be counter intuitive to think of a musical instrument as two different kinds of objects depending on if they are being played or not, and this fact is further complicated by the fact that instruments also hold socio-cultural meanings that are independent, but related, to the music they are used perform, multiplying their possible signifiers manifold. With electronic instruments it is slightly different due to the fact that they commonly lack a physical interface that has a clear relation to the core sound producing engine. If the instrument is also intelligent in some way, that is, if its musical output is greater than what it is given as input, the relations can become very obscure. In other words, while the musician playing an acoustic instrument quite easily becomes \emph{one} system which it is possible to gain an intuitive understanding of, that is, understanding it from within, for the electronic instrument this is much more convoluted.

The consequence of this discussion is that it is not enough to merely attempt to perceive the movement from within, as Bergson is urging us to do, it is first necessary to see the full context of the system for which the intuition is desired, and then experience it from the inside. In other words, to understand an object from the inside, to fully understand its mobility, it is important to understand what the boundaries of the system are. If I wish to understand what it means to play the saxophone, it is not the saxophone I need to enter inside, nor is it merely my own ambition with playing an instrument. I need to engage with the larger system that contains both myself and the instrument. This unity is what creates the conditions for expression and musical creativity, and analyzing the parts by themselves will only tell us what the parts are capable of. Even if I manage to explore the saxophone from the inside, I will only be able to understand it as an independent object. Only if I see the integrated system, and only if I manage to get on the inside of it, will I be able to fully understand it, and the way it is conditioned by its motion.

Then, what does it mean to get on the inside of a system such as a computerized instrument, the boundaries of which are seemingly unknown, and may include programmers and designers that are disconnected from the performer in both time and space? An electronic instrument that is connected to the internet and that continuously fetches information that influences its output in live performance is a very different thing from an acoustic musical instrument. Especially so if the mapping of the physical input from one or several performers is hidden from both performers and audience, or even remapped during the course of the performance. What is the embodied relation between such an instrument and its performer that allows for it to be understood as a unified system? For a typical computer based instrument in the shape of a software running on a computer, which is what KA is, in what ways can it be embedded in a similar way?

As alluded to above, an argument can be made that for self-playing systems involving some kind of artificial intelligence, the system that rules the musical outcome is much larger than what may be experienced at first. What I see when I start such a program on my computer, what I experience to be the system in play, is just myself and the computer, where in reality it may involve previous input and output, as well as the programmer's various positions and biases. The programmer is concealed in this system, detached in both time and space, which makes it difficult to fully understand their impact. In this sense the electronic musical instrument is a system which is by some degree larger and more elaborate than an acoustic instrument. Furthermore, many electronic instruments, by their immediate relation to engineering and science, lend themselves naturally to an understanding based on analysis rather than intuition, which enforces their role as mere tools. The development of musical artificial intelligence and advanced electronic instruments has obviously profited from this which in turn has contributed to their quick development. At the same time there is a great interest in interfaces and systems that convey a certain stability that encourages long term development between performer and instrument, such as modular synthesizers. In other words, whereas quick development cycles of software serves its purpose at several stages of the design of the instrument, it may become an obstacle when trying to understand it in a musical context, and as an integrated part of a larger system.

Clearly, part of the challenge is understanding the interface between musician and an instrument, and part of it is understanding what type of electronic instrument is being used. My attempt here is not to make a general theory of the various types of electronic instruments that could be of interest, but instead to mainly focus on the particular case of KA of \emph{Goodbye Intuition}. Hence, it should be noted that other types of electronic instruments that holds a whole range of possible modes of engagements, allow for quite a different set of possibilities than what a piece of software does, completely lacking a physical interface. But I believe that it is possible to approach also other kinds of systems with the same method, and even though the results would be different, this could contribute to furthering the knowledge about the musical opportunities of electronic instruments.

The question now is how it may be possible to achieve the kind of \emph{sympathy} that Bergson is referring to towards KA that will allow for an inside experience of the system as a combination of the performer and the instrument? He describes it in a way that has a great deal of resonance with artistic practice in general and the work in \emph{Goodbye Intuition} specifically when referring to the \emph{absolute} movement and the states of mind that gives him access to the inside:

\begin{quote}
I also imply that I am in sympathy with those states, and that I insert myself in them by an effort of imagination. Then, according as the object is moving or stationary, according as it adopt one movement or another, what I experience will vary. And what I experience will depend neither on the point of view I may take up in regard to the object, since I am inside the object itself, nor on the symbols by which I may translate the motion, since I have rejected all translations in order to possess the original.\citep[p. 2-3]{Bergson1912}
\end{quote}
In the following sections of this essay I will show some examples of how this method may be used in artistic practice.

\section*{Method development in Goodbye Intuition}
\label{sec:org167d210}
As already stated, \emph{Goodbye Intuition} is an artistic research project that is  methodologically relatively closely connected to the stream of artistic research projects in music and neighboring fields of practice over the last decade. The Swedish projects \emph{Music in Disorder}\footnote{See \url{http://musicindisorder.se/} (visited on October 8, 2020)} and \emph{(Re)Thinking Improvisation} \citep{friskostersjo2013} both shared the kind of discursive based methodologies with \emph{Goodbye Intuition}. These, which Östersjö and myself have developed further using methods such as stimulated recall, \citep[See e.g. ][ ]{FriskOstersjo2020} have proven to be rather effective to surface the knowledge embedded in musician's practices. The general idea is to allow for a safe space where the musicians involved practice under scrutiny are allowed to discuss what they did, would have liked to do, and did not like that they did. This involves the possibility to rethink, or redo the performance, starting it anew so to speak, after which a new round of reflection may start. Commonly, and in order to increase the possible strata of reflections, guests are invited to the sessions that can introduce new and alternative perspectives.

In \emph{Goodbye Intuition} the purpose was not merely to discuss the performances and interactions of the group members, but specifically to investigate in what ways the improvising machine could participate in the improvisations and in the re-thinking of the improvisatory strategies of the members of the group. The purpose involved to see how the machine could disrupt the improvisational motor-habits, and give rise to aesthetically different music than that of the group's usual performances. As already mentioned, these sessions were usually carried out with one or more external guests in the room and more often than not, with only one musician at the time playing with KA. In most sessions that I was part of there would have been a new set of features of KA that had to be discussed, and a lot of the introductory discussions commonly circled around the core issue for the project, relating to its core research questions:

\begin{itemize}
\item How do we improvise with "creative" machines, how do we listen, how do we play?
\item How will improvising within an interactive human-machine domain challenge our roles as improvisers?
\item What music emerge from the human-machine improvisatory dialogue?
\end{itemize}

There is clearly no way to quantitatively compare a \emph{with-machine-improvisation} to a \emph{without-machine-performance}, so an investigation directed towards answering these questions has to be carried out qualitatively. The quoted word "creative" in the first research question points to the fact that there is no real creativity present within the machine's performance since its activities are governed mainly by statistical analysis of the material being fed to it. Hence, the only true creative act is that performed by the human improviser.

To work through the different stages of development of KA through conversation is only one of many possible means to approach this task. Conversation on improvisation is a discourse interwoven with aesthetic judgments, social practices and stylistic preferences. Nevertheless, if the goal is to reach beyond language and approach the music from the inside, in Bergsonian terms, conversations and discussions about the music is an effective method. However, it would likewise be fair to argue \emph{against} using discourse, because language, by its symbolic nature, may be seen to disguise the inside rather than relieve it. Roland Barthes describes the photograph as "always invisible: it is not it that we see" \citep[p. 6]{Barthes1980}, and similarly, in a method for artistic research in music the purpose may be said to be to remove the layers of signification that surrounds the music and understand and analyze it for what it is. In this case the discourse on improvisation may be said to disguise the potentiality of the development of the music, and it is only through unwrapping this discourse, through discussion paradoxically, that the concealed qualities may surface. This is by no means simple and it requires time and patience and it is important to appreciate all the different kinds of discussion on music that may take place.

Susan Sontag, in her still highly topical \citetitle{Sontag1986}, points to the dangers of interpretation to excavate and destroy the thing interpreted in search for its true and inner meaning \citep[ch. 3]{Sontag1986}. This kind of interpretation has an ideological undertone that organizes the contents in a way that fits the purpose of the interpreter, and thus results in a situation where events "have no meaning without interpretation" \citep[ch. 3]{Sontag1986}. Sontag continues: "To interpret is to impoverish, to deplete the world--in order to set up a shadow world of 'meanings.' It is to turn the world into \emph{this} world" \citep[Ch. 4]{Sontag1986}. The focus, according to Sontag, should be on what art \emph{is} rather than what it \emph{means} \citep[Ch. 9]{Sontag1986}. This should also be the focus on the discursive method proposed here, and that was carried out in \emph{Goodbye Intuition}.\footnote{It should be pointed out that this is not related to (Critical) Discourse Analysis since we did not attempt to do any systematic analysis of the discourse, but many of the core aspects of discourse analysis holds true also here in that the discourse here is a social practice, and the way we talk about e.g. improvisation is, as has already been mentioned, to a high degree shaped by the larger discourse on music.} The main point is not to explain what is or should be going on, but to make it possible to move inside of it, to observe the ways in which it is ticking from the inside, so to speak. In order to avoid "intuition", that is, the automated responses of the motor-habits that are the result of years and years of training, it is necessary to get on the inside of these habits by means of a different kind of intuition, and experience them in motion: to be in the same duration.

An example of how the disruption of discourse can create access to the material reality of the artwork is Annie Dorsen's \emph{Hello Hi There} \citep{dorsen2010}. The work is derived from a televised debate between Noam Chomsky and Michel Foucault the voices of whom are rendered by two chat bots. In a single stroke the discursive power balance of the two men is neutralized and given a comic nature, although some of the uttering they make is clearly derived from the two philosophers talking. Annie Dorsen took part in one of the \emph{Goodbye Intuition} workshops in which we discussed similar topics. This particular session was an important step for my own understanding of the complex inter dependencies between \emph{Goodbye Intuition} the group, the stylistic properties and the machine improviser.

One of the principal difficulties, however, remains to get on the inside of an electronic instrument such as KA. Even if we in \emph{Goodbye Intuition} had a rudimentary idea of the logic of the system it took us a long time to understand how it worked in a way that would make it possible to predict--predict in the wide sense of the word--the output. Furthermore, in the ambition to develop the \emph{sympathy} that Bergson is referring to, knowledge \emph{about} KA is analytical knowledge acquired from the outside. Helpful, but not the core issue here. When playing with a human improviser who has knowledge about the same general genre as oneself, reaching a common understanding of the field of possibilities, is usually not that difficult. Neither of the players will obviously be able to predict exactly what the other will play, but this is rarely the point with improvising together anyway. How musicians are able to negotiate the field of possibilities as efficiently as they are is a question that may be addressed through artistic research. It points to the multi-modality of communication that human-human interaction allows for, and how this is severely limited in human-machine interaction. Which brings us back to the conversational and practice based method of \emph{Goodbye Intuition}. It succeeded in creating an understanding of the musical possibilities of KA within the group that eventually allowed for artistically consistent improvisations. Through this method the members of the group managed to infer the properties of the larger system of a musician improvising with KA and create an \emph{intuitive} understanding of it. This understanding operates from the inside and proceeds to analysis that promotes the discursive practice of the group, which in turn allows for a wider understanding. This is not to be confused with knowledge about how KA operates technically speaking, but it is an understanding of what the possible musical actions are, and what the delimitation of the system are.\footnote{There are several documented examples where the parts fall into place, so to speak, from the workshop in Stockholm in 2019. A documentation of this laboratory can be seen here: \url{https://www.researchcatalogue.net/view/411228/431482} (accessed 1 September 2019)} 

To summarize, the project's polemic stance against intuition can be compared to Susan Sontag's reasoning in \citetitle{Sontag1986}: there is no true, inner meaning of art that is revealed through interpretation, and likewise, there is no naturally intuitive musical response that is brought to the forefront through introspection.\footnote{This is my own reading of the project's intention and it may not coincide completely with those of the project leaders.} "Against interpretation" may translate to "against motor-habits", the learned responses that may appear to be intuitive, but are in fact automated through repetition and conditioned by cultural and stylistic practice. In order to understand the motor-habits and how they operate from the inside, Bergson's method of \emph{intuition} may be used, from where they may be dismantled and understood. From this perspective it is also possible to sympathize with larger systems including a machine improviser: Where interpretation disguises the object as it is, intuition can deliver it.

\section*{At the inside of the system}
\label{sec:org200aa55}
As an example of how I have attempted to move to the \emph{inside} of KA I have created three small examples using existing sound material that I have fed the software with. The idea behind this was to see to what extent KA would decipher the content, and use it in a way that is meaningful in relation to the material itself, and how I thereby understand the system from the inside. The first example uses a range of samples from a piece I did in 2010 called \emph{The Mystic Writing Pad}.\footnote{The second movement can be heard here: \href{../../../../Dropbox/Music/music/Frisk/Mystic-tape.aif}{Mystic Writing Pad, II}} The composition has a middle movement that is purely electronic and which is the part used here. In the first and third part of the piece there is a midi guitar part that uses a large number of micro tonal samples. Both the samples for the guitar instrument and the raw files for the second movement was added to KA to make an improvisation. In the first instance I used merely the guitar samples which, without much interaction, created a nice repetitive, only slightly varied, track. This is expected as the software analyzes the content for information to be used in the playback of the material. If there is little or no variation from sample to sample, consequently there is very little variation in the output. In the case of the guitar samples, each sound file is the exact same length, and only the pitch differs. In the tuning of the samples the octave is divided in 23 steps for a total of 83 samples. Nevertheless, I liked the sound of these sometimes quite complex harmonies as one possible layer of the music. The material for the second movement of \emph{The Mystic Writing Pad} is more varied with samples of different lengths and character, and adding this material to KA instantly created a more differentiated and musical result. The repetitiveness of the initial test is retained and phrases are created in a rhythmically consistent manner with all phrases having about the same length.\footnote{The result can be heard here: \href{/home/henrikfr/Docs/uppdrag/GI/kimauto/final\_project/final\_stuff/KA\_example\_2.wav}{Etude \#1}. This track, as well as the following examples, is mixed to a binaural format and is best listened to on a pair of closed back headphones.\label{orgc839b28}} The spatialization was controlled by the MIDI output of KA. The music has here a dreamy, repetitive character, and reducing the options and limiting the material has the benefit of achieving understanding: departing from known material I can establish a dialogue with the system. Had I made this track without KA, which is not really a sensible comparison because the very premise was to make a track with KA, I would have chosen a more dynamic rhythmical layout that was less predictable on the level of the musical phrases.

In the second, and quite short example,\footnote{\href{../../../uppdrag/GI/kimauto/final\_project/final\_stuff/KA\_example\_1.wav}{Etude \#2} can be listened to here. See the notes in \textsuperscript{\ref{orgc839b28}} for listening instructions.} I added another archive of sounds to the patch. A similarly repetitive structure but with a wider range of sounds. The first example had mainly synthesized sounds whereas the new material added here also contain field recordings used in the video piece \emph{Machinic Propositions}.\footnote{See \url{https://vimeopro.com/feed2212/mongrel/video/304376625} for a web version of the piece (visited October 15, 2020). For a more in depth discussion of it see \citet{frisk2017e,frisk_elberling_2019}.} Also in this example the MIDI output is used to spatialize the output. By feeding KA another archive of sounds with a rather different set of qualities I was able to explore how the repetitiveness of the original set of sounds also influenced the second set of sounds, although the patterns were dispersed in a slightly different fashion, influenced by the variation. After these two experiences I felt I was actually able to predict the third and last stage of my attempts to compose and improvise with KA. It is important to note that the way I make use of it here is somewhat different from the primary way in which we have worked with it in the workshops: I have not fed KA a live signal that it is recording in real time and uses to structure its output as well as use to collecting sounds from the input.

The last example\footnote{\href{../../../uppdrag/GI/kimauto/final\_project/final\_stuff/KA\_example\_3.wav}{Etude \#3}. See the notes in \textsuperscript{\ref{orgc839b28}} for listening instructions.} takes all of the material from the first two tests but now also adding voice samples, also taken from \emph{Machinc Propositions}. At this point the textures sometimes get much more dense, and because the repetitiveness is reduced by the timbral and structural variation in the sound archives, it is also sometimes more sparse. Hence, the musical result can perhaps be said to be closer to the way I had organized the material in the original pieces. This point is of some relevance, because it means that I was able to get under the skin of KA, get on the inside of it, and create music in a way similar to what I would have done without the machine. In Bergsonian terms I have developed a \emph{sympathy} that makes it possible for me to not just work from the point of an analysis of KA as a tool, but rather explore it as a part of my general attempt to create. Another way of expressing this is to say that I have transgressed the boundary between myself and the machine and allowed an improvisation in kind, not on the software, and not with it, but \emph{within} it. As such it also defines the possibilities and limitations. The prize here is in case I should ever have wanted it, I have to give up the idea of being in control of the machine.

\section*{Summary}
\label{sec:org2c3bcd7}
I find that one of the challenges in artistic practice in music is to find a proper balance between a systematic approach to the treatment of the musical and non musical material, and some kind of intuitive method with which the system can be manipulated and mastered. The system, which can be almost any conceptual structure, including a tonal system, a sound source and its possible variations, a data set to sonify, a synthesis model, or whatever is of relevance to the artistic idea. The manipulation part of it can also be any one technique, in a range of possible methods. The purpose is to allow oneself to get acquainted with the system to the point where it becomes second nature when explored, examined and scrutinized. One should reach the point that allows a truly \emph{play} on the system, to be able to improvise with it. It is possible to get to this point quickly, sidestepping some aspects of this process at the risk of ending up doing a repetition of something previously done. But in order to experiment and move into new grounds I find this balance between system and practice invaluable.

The mastering of the systems, or concepts, is not a question of interpreting them. The main purpose here is not, to paraphrase Sontag again, to find out what the concept means and dwell into it, but rather to understand what it does \citep[Ch. 9]{Sontag1986}. Any one system by itself is rarely what makes a piece work. Instead it is how it is set in motion, how it operates; to understand what it can do under different conditions. Or, to relate back to the discussion of the relation between the whole and the parts: A concept as a point of departure for an artistic expression is like a part without a whole. Before it can be made useful it has to be conceptualized within the frame of the expression, and instantiated together with the practice with which it can unite. This can be achieved through play and improvisation. Considering my exercises with KA I started with a concept, a set of sounds, that by themselves were not really interesting. They have a certain meaning to me, because I have gone through the process of playing with them in a different context. But now, as I feed them to KA, they are empty. It is in the playing process that I interact with KA, and it is here that the concept is set in motion. For some of the sounds I fed it I could not reach a point that felt meaningful, but through the process of making the three etudes I managed to not only feel that \emph{we}, KA and me, were playing the sounds in a way that I could sympathize with. More importantly, however, I also reached a position in which the boundary between myself and KA was dismantled and I could understand the larger system consisting of me and KA from the \emph{inside}.

For a long time I have been interested in the ways in which technology affects how and what I play. One of the defining ideas has been to understand the interaction with technology as an encounter with another, and understand it less as a tool and more as an agent that I influence as much as it influences me. In my PhD thesis \citep{frisk08phd} I labeled this \emph{interaction-as-difference}  and this something I have continued to work with since then. One of the research questions in the thesis dealt with ways that would improve the sense of interaction in improvised music performances as well as in composed music. The project led to questioning what interaction actually means in artistic practice and took a stance against the defining character of what I labeled \emph{interaction-as-control}. The main thread was the idea that all kinds of encounters with tools of some kind in the artistic process is an opportunity to enter into a larger sphere of interaction that creates difference. The prevailing idea of \emph{control}, on the other hand, controlling the technology, is a drive that may have a negative effect on the possibilities of the system. If it is alien to me to go into a improvised music performance and attempt to control whoever I am playing with, which it is, I should not do so with any other agent that I engage with in the process. This reasoning has its roots in an ethical standpoint, also related to an aesthetic choice more concerned with the human capacity for collaboration than with personally and individually rooted attempts at artistic expression.

In a romantic view of artistic practice inspiration is typically seen to emanate from a spark of genius, sometimes subconscious, or linked to divine inspiration \citep{frisk2013}. Partly this may be related to the fact that art, dreams and subconsciousness may be seen to share the same code \citep[for a more in-depth discussion of this, see ][ ]{frisk2015,frisk09:improv}, a code which is not accessible through ordinary language. This brings us back to the Kantian idea of the beautiful, aesthetically pleasing experience that brings about a \emph{free play} of imagination and understanding and a judgment of beauty \citep[for a discussion on Kant's judgment of beauty, see ][]{Coate2018}. Cognition on this level is a category quite different from conceptual appreciation, and one of the things Bergson criticizes in \citetitle{Bergson1912} discussed above. What he is proposing as a means for approaching not only aesthetic judgment, but any attempt at understanding movement, is radically different from a practice focused on introspection and analysis. The \emph{duration} that is central to Bergson's philosophy is the common mode of operation for any musician, and it makes sense to also try to understand it as a movement. It is dynamic in the widest sense of the word, dynamic with regard to context in non real-time, as a multiplicity of possibilities, questioning the notion of a \emph{work}, constantly re-evaluating and transforming according to changing needs. What follows from this is the potential for a deep and holistic understanding of artistic practice in music.

Instead of the self centered artistic work departing from the identity of the originator it is possible to approach a musical collaboration as a potential for interaction that allows for differences to occur. These differences make a difference when they are encountered from the inside, so to speak, in a manner that short circuits the motor-habits of intuition. It does not matter much if the 'other' is an instrument, a computer running a particular software, or a musician performing--there are differences of course, but the general approach is similar--to understand it from the inside as Bergson asks us too makes the experience blend together. With regard to the particular examples presented above I can listen back to them and think that I would probably have done it differently if I was doing it without KA. When others are allowed to enter the constructive and defining phases of a musical process, one's sense of identity and control is likewise likely to be reconfigured. Through the dynamics of the larger system of oneself and one's collaborator the impact of a particularly individual articulation will be less salient, or meaningful. Common identifiers such as "my aesthetics" or "my expression" that are meant to define and limit a field of possible artistic outcomes also introduce a hierarchy that organizes this field. The hierarchy is part of the idiom of control, or \emph{interaction-as-control}, and it may be neutralized by approaching the system of collaborators from the inside, as a whole. Hence, what I think about the result of my examples from the point of view of my individual preferences is only partly relevant in this context. When I understand myself and KA as a system of \emph{interaction-as-difference} it is only within that frame that the result is meaningful, and from the inside this system contains a wide range of agents that influence the field of possible outcomes.

\section*{Acknowledgments}
\label{sec:orgd666ffc}
   I wish to thank all the members of \emph{Goodbye Intuition}: Ivar Grydeland,
Morten Qvenild, Sidsel Endresen and Andrea Neumann as well as David
Toop and Annie Dorsen, all of whom have contributed to the thoughts
presented in this article.

\section*{Bibliography}
\label{sec:org4513767}
\printbibliography
\end{document}