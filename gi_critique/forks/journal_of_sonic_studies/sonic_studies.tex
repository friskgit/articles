% Created 2024-02-01 Thu 11:08
% Intended LaTeX compiler: pdflatex
\documentclass[11pt]{article}
\PassOptionsToPackage{hyphens}{url}
\usepackage[utf8]{inputenc}
\usepackage[T1]{fontenc}
\usepackage{graphicx}
\usepackage{longtable}
\usepackage{wrapfig}
\usepackage{rotating}
\usepackage[normalem]{ulem}
\usepackage{amsmath}
\usepackage{amssymb}
\usepackage{capt-of}
\usepackage{hyperref}
\bibliography{./gi_biblio.bib}
\setcounter{secnumdepth}{0}
\author{Henrik Frisk, henrik.frisk@kmh.se}
\date{\today}
\title{Sound intuition}
\usepackage{calc}
\newlength{\cslhangindent}
\setlength{\cslhangindent}{1.5em}
\newlength{\csllabelsep}
\setlength{\csllabelsep}{0.6em}
\newlength{\csllabelwidth}
\setlength{\csllabelwidth}{0.45em * 0}
\newenvironment{cslbibliography}[2] % 1st arg. is hanging-indent, 2nd entry spacing.
 {% By default, paragraphs are not indented.
  \setlength{\parindent}{0pt}
  % Hanging indent is turned on when first argument is 1.
  \ifodd #1
  \let\oldpar\par
  \def\par{\hangindent=\cslhangindent\oldpar}
  \fi
  % Set entry spacing based on the second argument.
  \setlength{\parskip}{\parskip +  #2\baselineskip}
 }%
 {}
\newcommand{\cslblock}[1]{#1\hfill\break}
\newcommand{\cslleftmargin}[1]{\parbox[t]{\csllabelsep + \csllabelwidth}{#1}}
\newcommand{\cslrightinline}[1]
  {\parbox[t]{\linewidth - \csllabelsep - \csllabelwidth}{#1}\break}
\newcommand{\cslindent}[1]{\hspace{\cslhangindent}#1}
\newcommand{\cslbibitem}[2]
  {\leavevmode\vadjust pre{\hypertarget{citeproc_bib_item_#1}{}}#2}
\makeatletter
\newcommand{\cslcitation}[2]
 {\protect\hyper@linkstart{cite}{citeproc_bib_item_#1}#2\hyper@linkend}
\makeatother\begin{document}

\maketitle
\section*{Introduction}
\label{sec:orgf219904}
A sound is a complex compound of various kinds of information.
In the process of working with sound in artistic practice, be it a sound installation, a musical composition, or something else, the access to the sound itself is at the same time both completely unproblematic and very enigmatic.
It has a source, and discernible sonic qualities, it may have a direction and a space, and it may give rise to several emotions, sometimes several at the same time.
A sound can be completely metaphysical, imagined as part of a dream, or it can be so physical that it is felt by the entire body.
It can be painful or beautiful, harsh or soft.
The balance between the various components will obviously shift depending on what kind of sound it is, and all of the parameters of the sound as a whole usually change to a significant degree over time: the attack and the decay of a sound may be extremely different from one another.
The same continuous sound can start as a perfectly sourced sound and end in an almost completely abstracted and deconstructed noise.
How can this amorphic compound be understood in artistic practice? How can it be analyzed, and what kind of analysis participates in a widened understanding of the meaning of the sound, rather than a deductive reduction of it? How can artistic research provide complementary knowledge about the function of sound in art and in other contexts?

This work departs from ideas developed in the artistic research project Goodbye Intuition (GI).\footnote{The project started in 2017 and was concluded in 2020 and was financed by NMH, Norwegian Artistic Research Programme, Norwegian Center for Technology in Music and the Arts (NOTAM) and Royal College of Music in Stockholm (\url{https://nmh.no/en/research/projects/goodbye-intuition-1}).}
The focus of GI was on improvising with "creative" machines.
Playing with a machine that has little or no sense of intuition, not in its standard meaning, nor the meaning explored in this article, reveals the complexity of the role that intuition plays when working with sound.
One of the results of GI was the realization that we projected our  intuitive sense unto the machine improviser,\footnote{It may be debated whether or not this particular improvising machine did act with some notion of intuition. Although it had some simple version of machine learning built into it one of the explicit goals was nevertheless to counteract intuition. Listen to sound examples x1 and x2 for an illustration.} with the result that we experienced that there was an intuitive relation between us as musicians and the machine.
The work discussed here is my  continuation of the ideas explored in GI, primarily the impact and meaning of intuition in artistic practice.
I have been working with methods similar to those used and developed in GI (see \cslcitation{15}{Frisk 2020}) where the basic process is carried out in an artistic lab session in which my  artistic processes have been documented, and out of which particular events have been extracted and studied.
The specific events I have chosen to study have been selected based on my  inner understanding of the practice, to some extent relying on purposeful sampling (\cslcitation{27}{Patton 2002}).
Intuition, in the sense that Bergson uses the word, has then been explored as a possible method in artistic research in sound practices.
Hence, artistic practice constitutes the main empirical data from which the results are gathered.
Using intuition as a method questions concerning the understanding of sound are approached from my particular practice of working with composition and improvisation in music:\footnote{I see improvisation and composition as two aspects of music making and although there are conceptual differences between them I use them interchangeably as activities that partake in the creation of music.}
How can I understand sounds and sonic gestures in time and space as carriers of meaning?
This question should be seen as an attempt to break with the scientific imperative, here partly represented by the tools I often work with in the studio, and instead explore an intuitive relation to the material these tools may \emph{result} in.
It is not the analytical measurement of a gesture or of a sound but rather the the analysis of an intuition of it that I am interested in.

When playing an instrument the sound is at least to some degree woven together with the gestures one performs as the music evolves.
For the electronic music composer sitting behind a computer screen for hours, the situation is different.
Working with tiny adjustments to microscopic segments of the sonic material relieves the distance that may occur between the composer and the material they work with.
If the sound is seen as an object that is to some degree independent of where and how it was produced, and where, why, and how it is heard, this may not be seen as a problem.
If one is interested in engaging in a more holistic view of sound practices such as I suggest in this paper several questions that will arise of which these three are of interest:

\begin{enumerate}
\item Questions relating to the contents of the sound, its ontology, its shape, its gestural aspects, and its space are necessary to approach in the artistic practice in the studio. One by one they may be analyzed deductively, but the general design of the tools with which this is made possible simultaneously makes it difficult to successfully understand them together, as a whole.
\item The tools of a modern studio do allow the composer to turn from a tiny detail of a sound to the whole in a blink of an eye which is helpful but may further contribute to a fragmentation of the material.
\item The system of reproduction of sound in the studio is in almost all cases completely different from the one that the music will ultimately be played back on.  For that reason, the composer will necessarily have to engage in a constant act of imagination to complement the sound produced in the studio, sometimes without much knowledge of the ultimate performance space.
\end{enumerate}

The hypothesis explored in this paper is that these questions, and others, may be addressed by exploring the impact of intuition involved in the practice which in turn rests on an artistic sensibility to imagine how sound may be experienced.\footnote{See Ingman (\cslcitation{21}{2022}) and Thompson (\cslcitation{33}{2009}) for a discussion on artistic sensibility. Especially the latter who describes it as something that "navigates the inner world of the psyche as well as permeating outside-in-the-wo rid spaces, as a flow that emanates from both spheres. It promotes flexible, affective responses to ideas from having reflected and acted upon them." As we will see later this provides an interesting counterpart to intuition.}
As a composer and improviser, I can make myself intuit sound in a way that makes it possible for me to engage in a fluid relation to the material--this is to some extent what I train to do as a musician.
In this case, the sound is not an object, but rather an ongoing activity, and despite my inner knowledge of it, a fuller analysis is difficult to achieve.
The sound is not merely perceived but created and recreated in the act of inner and outer\footnote{Inner listening in the situation of improvising is the listening to the self (\cslcitation{14}{Frisk 2014}) and the outer listening is to listen to the other \emph{and} to listen to the sound as a listener.} listening in a process where the signification of the various aspects of the sound is consciously and constantly changing.
It should be noted that in this paper I am primarily discussing listening experiences from the point of view of the artist, not listening in general.
My knowledge and experience as a practicing musician allow me to understand the flow of sound in time and act upon it in ways that may not even be useful to a listener.
This understanding is not easily translated to analytical knowledge and it is different in kind to the notion of sound as an object. 
I can transform my listening both in time and space and imagine what the music will sound like in another environment: I can experience it, not as an \emph{external object} that I exploit, but \emph{from within it}, together with it: from the "inside".
The terminology here comes from Bergson, as will be clear later in the text, and it has advantages but also obvious difficulties. The word "inside" carries with it an unfortunate connotation that makes assumptions about a division that may not make sense, and can even be counter-productive.
In reality, there is an ongoing continuum between inner and outer that makes either position difficult, and less meaningful, to pinpoint, but this will be further discussed later on in the text.

There have been many attempts to approach the possible ways in which to understand sound.
French theorist Pierre Schaeffer (\cslcitation{29}{1977}) and his very influential work made sound objects accessible to the listener phenomenologically through the process of reduced listening where the source of the sound should be disregarded.
Dennis Smalley (\cslcitation{31}{1986}) developed the concept of \emph{spectromorphology} that builds on Schaeffer's theory and is an analytical tool to trace the spectral changes of a sound over time.
William Gaver (\cslcitation{18}{1993}) opened up an ecological perspective and pointed to a new ontology of sound, and Eric Clarke (\cslcitation{9}{2005}) further developed the ecological approach.
Other attempts include the concept of sound as an \emph{(Un)repeatable object} (\cslcitation{12}{Dokic 2007}) and numerous theoretical and artistic explorations and developments of the above theories.
\section*{Artistic Research and the knowledge claim}
\label{sec:org1f3ea21}
Ever since the early days of artistic research, there has been a discussion about the difference between this and other kinds of research.
The difference, according to one common argument is that since the artistic researcher is exploring the artistic process in the making, the research is performed from an \emph{inside} perspective.\footnote{It may be helpful to bring up the terminology of \emph{emic} and \emph{etic} commonly used in ethnography and anthropology and other research fields. The \emph{emic} field research would here relate to the \emph{inside} perspective of the artistic researcher.}
This may even be seen as one of the defining ideas of the epistemology of artistic research: there is a difference between knowledge that has been acquired from observing an artistic practice, and knowledge that is the result of practicing art.
If the artistic researcher is researching from an inside perspective, the vantage point for other kinds of research would then be from an analytical perspective, observing from the outside.
These metaphors are crude representations of what goes on in research, and like was noted above, this has the unfortunate conceptual drawback of creating a dichotomy between the inside and an implied outside, which is neither entirely correct nor is it particularly useful for, say, the development of interdisciplinary research.
The often cited categories (and their variations such as those proposed by Borgdorff (\cslcitation{4}{2007})) that Frayling (\cslcitation{13}{1993})  put forth in his article \emph{Research in art and design} ‘research into art’, ‘research for art’, and ‘research through art’, are loosely pointing in the same direction: there are distinct research modalities from which various types of results emerges.
These often referred to categories are rarely stable, nor are they exclusive: any research practice, artistic as well as scientific, in music, is likely to touch on all of these modalities. 

To explore the idea of artistic research from an inside perspective it is not enough to merely consider the perspective of the researcher.
Exploring internalized how-to knowledge, and the belief systems that surround the practice demands stable and transparent methods for revealing the processes in action.
This is discussed by Galdon and Hall (\cslcitation{17}{2022}) concluding that "this type of implicit knowledge creates a problem around how we can be sure that tacit knowledge is communicated and acted upon in a manner consistent with its generation" (p. 919). As this paper is written from the perspective of design education it may be used as a critique of the notion of an \emph{inside} as a qualifier unique to the artistic researcher.
Furthermore, any argument put forth from this \emph{inside} perspective may be assigned a parochial nature difficult to contest.
When the claim is that the artistic researcher by definition produces research from the \emph{inside} the critical discussion concerning the results must be considered.

Much has been written about these topics over the years of the development of the field of artistic research. In the contribution by Sören Kjørup (\cslcitation{23}{2010}) in the \emph{Routledge Companion to Research in the Arts} he argues that:
\begin{quote}
if artistic research is supposed to be different from all other kinds of research, it is natural to focus on the artist as the researcher, and what is specific for the artist is her or his privileged access to her or his own creative process. (\cslcitation{23}{Kjørup 2010, 25})
\end{quote}

This "privileged access" could be seen to harbor a possibility for revealing a kind of knowledge that is sometimes mediated by symbols and concepts, but which is primarily founded on unmediated experience, a somewhat paradoxical situation where the goal is to bring forth that which is by nature hidden.
Naturally, one of the recurring themes in the early discussions on the identity of artistic research was, and still is, how to understand its nature, and what kind of relation it should have to other kinds of knowledge.
How can something that evades conceptualization at all be represented stably?
How may this unmediated experience be useful to the artistic researcher and others?
These questions are still of relevance in artistic research and the point on which this discipline is most often criticized.\footnote{For a broader discussion on this topic, see Frisk and Östersjö (\cslcitation{16}{2013}).}
These questions also rely on the fact that the artist, with their privileged access, knows how to gain access to the experience, or what is often referred to as tacit knowledge, and that this is the source for the methodological mangling, conceptualization. and eventually, the expression of meaningful knowledge.
It only makes sense to attempt to answer these questions if there is meaningful knowledge in an artistic practice.
How to attempt to determine how this may be approached within the field of artistic research in sound and music is the primary focus of the discussion of this paper.
I will approach the topic through Bergson's method of intuition in the context of my  practice as a musician.
\section*{The method of intuition}
\label{sec:orge12a29e}
French philosopher Henri Bergson sought to address the problem of what knowledge one may have of the world exterior to oneself, and one of the central tenets of this effort was the method of intuition that he developed.
It was a recurring theme in his work but in this paper, I mainly draw upon his short text \emph{An Introduction to Metaphysics} (\cslcitation{2}{Bergson 1912}).
Intuition as a method will by necessity include also other modes of thinking, but the point here is not to give a full account of Bergson's philosophy nor of the method's full implications.\footnote{Bergson's notion of intuition as a method has been both criticized (\cslcitation{8}{Clair 1996}) and praised (\cslcitation{10}{Deleuze 1988}) by many thinkers ever since he first published on the subject, it has been explored affirmatively in post-colonial theory (\cslcitation{11}{Diagne 2008}) as well as feminist readings (\cslcitation{35}{Tuin 2011}).}
With it, I am proposing a method with which the question above may be addressed.
Contrary to Bergson's point of view, Bertrand Russel, one of Bergson's fiercest critics, saw intuition and instinct as incapable of creating anything new, something only the intellect can achieve. Intuition, Bergson writes, "is greater, as a rule, in children than in adults, and in the uneducated than the educated" discrediting the epistemological capacity of intuition as a means of learning and understanding (\cslcitation{28}{Russell 1981}).
Furthermore, from a point of view of cognitive science the developments over the last few decades may seem to have rendered Bergson's theories obsolete, but from a philosophical and metaphysical point of view, however, there has been a continuing interest in his work (\cslcitation{24}{Lawlor 2003}; \cslcitation{30}{Shklar 1958}; \cslcitation{22}{Kelly 2010}; \cslcitation{20}{Hirai 2023}). 


The more general interpretation of intuition relates to the things we do without thinking about them; the intuitive knowledge that something is, for example, wrong or dangerous.
Intuition may be likened to an internalized and automated system that pre-reflectively makes us act upon what is going on in the world around us, perhaps more akin to instinct.
In phenomenology, intuition has a slightly different meaning.
Intuition gives the subject first-person knowledge and in this sense an object can be said to be \emph{intuited}.
Bergson's use of intuition is described by Kelly (\cslcitation{22}{2010}) "as a method of reflecting on instinctual or sympathetic engagement with things in all their flux before the framework of practical utility obfuscates our relation to them and to life." (p. 10)
It is this meaning of intuition that the rest of this paper is leaning on.

In the essay mentioned above, \emph{An introduction to metaphysics}, Bergson (\cslcitation{2}{1912}) defines two incommensurable ways to approach an object: either from a point of view through signs and concepts--a \emph{relative} perspective--or through entering into the object, exploring it from the inside--and \emph{absolute} apprehension. This exploration from the inside is achieved by entering into what he calls \emph{sympathy with the possible states of the object} which allows for inserting oneself "in them by an effort of imagination"  (\cslcitation{2}{Bergson 1912} , p. 2).  This enables him to "no longer grasp the movement from without, remaining where I am, but from where it is, from within, as it is in itself" (\cslcitation{2}{Bergson 1912} , p. 3). The latter is what he refers to as \emph{absolute} knowledge: "The absolute is the object and not its representation, the original and not its translation, is perfect, by being perfectly what it is." (\cslcitation{2}{Bergson 1912} , p. 5-6)

An example that Bergson gives to describe representational knowledge is a photographic view of a city.
If all angles and all surfaces of a city area have been photographed and documented to achieve something similar to the street view that online maps now offer, a reasonably highly detailed replica of the space may be achieved.
Exploring such a model, however, can not be equated with the experience of being in the city. It will offer a representation, and as such much can be gathered about the space but it will still be a qualitatively different experience.
Another example Bergson gives is the translation of a poem into different languages.\footnote{As Swedish artist Andreas Gedin has proved, sequential translations of a poem into multiple languages does not only offer different nuances but sometimes a completely different expression.}
Each such translation can give the reader a sensible idea of the meaning of the poem, sometimes even revealing new articulations, but it would, claims  Bergson (\cslcitation{2}{1912} , p. 5). "never succeed in rendering the inner meaning of the original"

One of Bergson's central propositions here is that the kind of knowledge that arises from a \emph{relative} perspective is always a reduction of the thing under consideration.
Scrutinizing the object from an outside perspective allows for analytical precision, but whatever comes out of this process is always a reduction:
\begin{quote}
In its eternally unsatisfied desire to embrace the object around which it is compelled to turn, analysis multiplies without end the number of its points of view in order to complete its always incomplete representation, and ceaselessly varies its symbols that it may perfect the always imperfect translation. It goes on, therefore, to infinity. But intuition, if intuition is possible, is a simple act. (\cslcitation{2}{Bergson 1912} , p. 8)  
\end{quote}

The \emph{absolute}, on the other hand, is given from \emph{intuition} and \emph{intellectual sympathy} with the object.
The \emph{intuition} allows for a perception of the object's unique qualities which Bergsom points out, is the \emph{perfect absolute} in contrast to the \emph{imperfect analysis}.
The science of intuition is metaphysics, and metaphysics is "the science which claims to dispense with symbols" (\cslcitation{2}{Bergson 1912} , p. 9).

The one reality that is almost always seized from within is when we engage in self-reflection.
Bergson describes the various strata the process of introspection provides when slowly moving toward the center of the self.
A protecting "crust" is the first layer and it is made up of all the perceptions from the outside world.
Then memories of interpretations of perceptions are encountered, followed by motor habits that are both connected and detached from the other layers.
At the core Bergson describes the continuous flux of a concatenation of states in an ongoing movement back and forth.
The metaphor used here is that of a coil constantly unrolled and rolled up again through the various layers--motor habits, memories, and the outer crust--out on the outside and back in again.
Admittedly, this comparison is far from perfect and the idea of the rolling up of the coil may be misleading.
It still has some merit in the current context, however, though slightly different from how Bergson intended it.
It brings in a possible deconstruction of the two poles in Bergson's model as the movement between the various strata in this metaphor can be seen as a continuum from the outside to the inside.
An analysis is rarely \emph{exclusively} analytical or intuitive, outside or inside, but more often a motion where both perspectives contribute to knowledge. 
This evokes a passage in an earlier work, (\cslcitation{3}{Bergson 1991}), which gives a to some extent different image of the movements back and forth through presence, memory, and experiences.
Conscious practice is displayed here as a cone whose tip is moving over a similarly moving plane, and the point of the cone represents the present and the cone itself the accumulated memories and experiences: 

\begin{quote}
The bodily memory, made up of the sum of the sensori-motor systems organized by habit, is then a quasi-instantaneous memory to which the true memory of the past serves as base. Since they are not two separate things, since the first is only, as we have said, the pointed end, ever moving, inserted by the second in the shifting plane of experience, it is natural that the two functions should lend each other a mutual support. So, on the one hand, the memory of the past offers to the sensori-motor mechanisms all the recollections capable of guiding them in their task and of giving to the motor reaction the direction suggested by the lessons of experience. It is in just this that the associations of contiguity and likeness consist. But, on the other hand, the sensori-motor apparatus furnish to ineffective, that is unconscious, memories, the means of taking on a body, of materializing themselves, in short of becoming present. (\cslcitation{2}{Bergson 1912} , p.152-3)
\end{quote}

The sensory motor habits are informed by memories through which they will be guided to do the work they are set out to do, and because no single memory is ever stable--it is always altered by the present in the interaction between what Bergson refers to as the "pointed end" and the past memory--the experience is continuously altered by past experience, which in turn is influencing the present.
Interesting for the current discussion is the connection brought up between sensori-motor mechanisms and past experience, and the fact that this connection is not only going one way--from memory to habit--but also from habit back to memory.
Embodied memory is in a changing flux and constant interaction with experience and habit.
There is an inclination to understand learned and deeply integrated behavior, such as playing an instrument or lifting a glass of water, as pre-reflective acts independent from reflection.

It is in thinking about embodiment and motor habits that Bergson's understanding of what an intuition can be is perhaps best understood.
When I move my leg or my hand I have a unique insight into what is going on, one that would be difficult, or impossible, to acquire from the outside in the same way.
Analyzing the movement from an outside perspective will fail to understand it completely since the analysis only pins the movement to a sequence of states.
The actual change, the mobility or, as Bergson would put it, the duration, is only possible to understand through intuition he claims.
Furthermore, any new experience within such a movement, as well as any past experience will introduce change in the system.
\begin{quote}
When you raise your arm, you accomplish a movement of which you have, from within, a simple perception; but for me, watching it from the outside, your arm passes through one point, then through another, and between these two there will be still other points; so that, if I began to count, the operation would go on forever. p.6
\end{quote}

I have learned to move my arm, and every new piece of information about what I can do with it will add to my arm-moving-knowledge, and intuition is the modality through which knowledge about the process is gathered.
For a subject able to observe the thing from the inside, intuitively, there are no states, only duration and mobility informed by experience and knowledge.  
Without this inside access, one is left with the option of a conceptual analysis from the outside, and regardless of how many different perspectives this analysis is performed from, it will never fully capture the true \emph{motion} of the object.
The contradictions between this and the intuitive knowledge that Bergson is arguing for
\begin{quote}
arise from the fact that we place ourselves in the immobile in order to lie in wait for the moving thing as it passes, instead of replacing ourselves in the moving thing itself, in order to traverse with it the immobile positions. They arise from our professing to reconstruct reality--which is tendency and consequently mobility--with precepts and concepts whose function it is to make it stationary. (\cslcitation{2}{Bergson 1912} , p. 67)
\end{quote}

One central aspect of the distinction between analytical and intuitive knowledge made here is that the intuitive, being in motion or duration, can always develop concepts and form the basis for analytical knowledge, whereas it is impossible to reconstruct motion from fixed concepts.
An analysis may result from intuition, but intuition cannot arise from analysis.
An analysis from the outside is performed on one particular state of the duration, and from multiple analyses or states, it is possible to imagine that the mobility may be reconstructed by simply adding the different states together.
This is the critical point that Bergson objects against:
It is only through intuition that the variability of reality may be fully experienced as mobility.
A succession of static states is radically different, it is a series of frozen frames of time added together, one slice after the other.
The error in thinking that reality may be accessed through analysis, claims Bergson, "consists in believing that we can reconstruct the real with these diagrams. As we have already said and may as well repeat here--from intuition, one can pass to analysis, but not from analysis to intuition" (\cslcitation{2}{Bergson 1912} , p. 48) 

To conclude this brief overview of Bergson's theory of intuition there are a few things that I claim connects it to the general discussion of the uncertainty of the epistemology of artistic research: what knowledge can we expect from this research? The inside perspective is often brought up as a significant aspect of artistic research.
If this perspective can be approached through intuition and analyzed as Bergson is suggesting intuition should be a valid method, both in general and in relation to the ambition of this paper.
Furthermore, self-reflection is a recurring concept in the discussion of both artistic research and experience, as noted by Borgdorff (\cslcitation{5}{2010}): "Art’s epistemic character resides in its ability to offer the very reflection on who we
are, on where we stand, that is obscured from sight by the discursive and conceptual
procedures of scientific rationality." (p. 50)
self-reflection, according to Bergson, is a way to understand and develop intuition as it fully relies on the inside perspective.
self-reflection develops intuition which enables access to an inside perspective that may then be analyzed and communicated in a continuum of moving between internal and external vantage points.
\section*{Intuition and sound in practice}
\label{sec:org3be775d}
As was hinted at at the beginning of this paper, one of the obstacles in artistic research are the questions concerning (1) the methods that allow for observing relevant information about the artistic practice in sound, and (2) the means of presenting this information in an accessible manner.
I will mainly discuss the first of these which I argue may be addressed using the proposed method of intuition.
\subsection*{Acoustic instruments and interaction}
\label{sec:org4868876}
Playing an acoustic instrument is a complex activity that involves a lot of interaction between the instrument and the musician. Practicing the instrument over many years allows for the development of a very instinctual relation to the instrument.
As a saxophonist, when I pick up and play the saxophone I do not experience it as an external object that I analyze deductively.
I enter into a sympathy with it which allows for an intuitive understanding of the processes I engage in: I am \emph{listening from the inside} to the sum of the parts that are currently playing.
There is an intimate relationship between learning and intuition, and the more I learn about my instrument the greater the possibility of entering into sympathy with it.
The process of learning to play an instrument is often compared to other embodied activities such as cycling: when learned they eventually become second nature and to some extent pre-reflective.\footnote{With the important difference that one commonly spends a whole life to improve ones skills in music whereas cycling is learned, at least for practical purposes, and mastered very quickly.}

Time is of the essence and, following Bergson, through the method of intuition, the continuous flow may be experienced.
To succeed in entering into sympathy with the play situation, however, it may not be the saxophone as an object I need to understand, and the notion of "an object" may be misleading altogether.
Rather, it is the larger system, containing both myself and the instrument and its context that I need to engage with.
This unity creates the conditions for expression and nuanced musical creativity, and analyzing these parts by themselves will only relieve what the parts are capable of, not the whole.
Only if I manage to get "on the inside" of the integrated system will I be able to fully understand it and the way it is conditioned through motion and duration.

The sensory and auditory feedback I get from the instrument continuously adopts how I play it, and how much pressure I put into it, and this input depends on the structurality of the instrument and the system as a whole: My motor habits are changing as I play, which changes the feedback I get from the instrument.
In other words, to perceive an object from the inside it is first necessary to understand the way the object integrates with me whereby its status as an independent object to some extent is dissolved.
The moments of circular breathing in the short excerpt from my piece \emph{Concinnity} undoubtedly make it necessary for me, in performance, to focus intensely on things that are usually second nature even when playing a wind instrument.
I need to plan my breathing and the spaces I leave for the electronics that are generated from my playing. Additionally, the tuning adds an extra challenge that requires that I keep an inner focus.

\begin{figure}[htbp]
\centering
\includegraphics[width=.9\linewidth]{img/concinnity_1.jpeg}
\caption{\label{fig:org5046a03}The recording is from a concert at CCRMA, Stanford in December 2023. <sound concinnity.wav>}
\end{figure}

However, if some part of the system changes it is often not enough to only make a small adjustment, the whole system may need to be reconfigured, and certain things need to be learned again.
One example of a change in the system is if I have a cut on my lip or if I have wounded a finger or similar.
This alters the inside perspective and may take some time to adjust to.
The instrument is embodied, a process that is the result of sympathizing with it and gradually creates a system that I can approach intuitively.
This integration is part of learning an instrument and may be quick in simpler instances, and take a lifetime in more complex ones.

But also sensory data that are external to the saxophone-musician system has an impact on what and how I play.
The moving coil that Bergson describes is a metaphor for how learning also depends on past experiences and events outside of oneself.
This back-and-forth process is not, however, limited to two dimensions, but is in constant motion in a multi-dimensional space that involves all aspects of the system.
From a concert in Tokyo in November 2023 with  Rikard Lindell on a modular synthesizer I need to, at times, incorporate these external sounds into my  expression which, when I listen back to this live recording, works very well at the end of this segment where the processing of the saxophone is entirely integrated in the musical form and the specific character of the music played, but is less successful at the beginning of it where it takes some time for us to find the direction in which to drive the music.
Through this excerpt, we attune each other's systems and create overlapping areas through which we may approach our respective intentions intuitively.

\begin{figure}[htbp]
\centering
\includegraphics[width=.9\linewidth]{img/tokyo.jpg}
\caption{\label{fig:org8abf1f2}Duo on modular synthesizer and saxophone/computer from November 2024. The recording was only made for documentation purposes. <sound tokyo.wav>}
\end{figure}
\subsection*{The computer as instrument}
\label{sec:org20ce392}
In contrast to the musician-instrument relation described above the musician-computer relation is of a more convoluted nature.
What I see when I start an application on my computer, what I experience to be the system in play, is just myself and the hardware and some software, where in reality it may involve previous input and output, as well as various positions and biases some of which may be disguised.
In this sense, the electronic musical instrument is a system that is by some degree larger and more elaborate than an acoustic instrument.
What does it mean to get on the inside of such systems?
What part of the system has agency, and to what extent is the creative act distributed rather than controlled by the musician?
The extent to which such a system stretches out into the unknown is significant. 
It may include programmers and designers that are disconnected from the performer in both time and space, yet connected to the instrument and its design properties.\footnote{A more in depth discussion on these topics may be found in \emph{Aesthetics, Interaction and Machine Improvisation} which also includes the impact of self organizing systems and AI (\cslcitation{15}{Frisk 2020}).}
There may be a range of hidden layers, obscured from both performers and audience, and that can be remapped during the course of the performance.
An electronic instrument that is connected to the internet and that continuously fetches information that influences its output in live performance is a special case, and such a system is significantly different from an acoustic musical instrument.
Intuition, I believe, is still a valid method here, but it requires a few considerations which I will discuss in the following sections.

In this context, it is also worth noting that a certain general merge between fields of arts practices and science in general has occurred that makes possible a further critique of Bergson's division between analysis and intuition.
Regardless of the extent to which the field of artistic research has reiterated the importance of the difference between the sciences and the arts, the computer is to a significant degree one of the main tools that both fields use.
In other words, the artistic research lab is not technologically different from that of the science lab, and the primary tool for deductive analysis is also the primary tool for much of music production today as pointed out by Tresch and Dolan (\cslcitation{34}{2013}).
Though the methodologies of the two fields differ to a significant degree the merge is profound, and the universality of computers may conceal the fact that the technology, rather than merely supporting the creative work, also controls it in ways that are not obvious.
The agency of the various parts of the system is blurred.
More importantly, in this merge of the computer as a tool and instrument, and other instruments for artistic practice in music, there may be a risk that the scientific nature of the machine constrains the possibility of engaging intuitively with the system of artistic production.
As was noted above, many electronic instruments, by their immediate relation to engineering and science, lend themselves naturally to an understanding based on representation rather than intuition, which enforces their role as mere tools.
The method of entering into sympathy with a recorded sound in a technological system, and understanding it from the \emph{inside} without getting lost in the various ways that the systems for reproduction extend in space and time is accessible yet complex.
In my  experience the deductive methods of analysis previously mentioned that pertain to the underlying structures on which many of the tools in the electronic music studio are built, may disrupt both practice and thinking.
Understanding a recursive filter or a signal processing device, let alone an AI-enhanced digital compressor or a generative audio plugin requires an insight into the analytical aspects of sound that may disperse the intuitive focus of the artist's working methods.
\subsection*{Listening strategies and sonic gestures}
\label{sec:org75ce55e}
When it comes to listening perhaps the question should rather be if \emph{any} listening can be said to \emph{not} be carried out from "the inside", using Bergson's terminology.
There are several widespread listening practices, like Pauline Oliveros' \emph{Deep Listening} method (\cslcitation{26}{Oliveros 2005}) that propose methods towards this goal, and that are independent of Bergson's notion of intuition.
With this in mind technologically mediated listening in a studio while in the process of working artistically with sound still provides a mix of modalities that has an impact on the present discussion.

Bergson's proposed method may prove to be useful about understanding listening and creativity in the process of composition
As was discussed at the beginning of this paper, one of the challenges in artistic research is to get access to the specific kinds of knowledge that the artistic process generates and makes use of. It appears reasonable to assume that a close relation between reflective thinking, through a Bergsonian method of intuition, within the actual practice as it takes place may help to gain insight into this knowledge.

\begin{figure}[htbp]
\centering
\includegraphics[width=.9\linewidth]{img/locomotion.png}
\caption{\label{fig:org6ec3d75}\emph{Locomotion} is a piece for three spaces and 60 speakers. The sound attached here was collected for this project but is presented here unprocessed. This is best listened to in headphones as it is a binaural rendering of the recording.  <sound rain\textsubscript{at}\textsubscript{night}\textsubscript{binaural.wav}>}
\end{figure}

Furthermore, there are aspects of a sound that \emph{require} the listener to be within the \emph{mobility} of the sound to understand them. The spatiality of sound can both be purely imagined and highly concrete and it is an aspect of the sound which is very difficult to extract with deductive methods. The recording attached to \ref{img_3} was made for a large piece for 60 speakers that premiered in 2019 in Stockholm. Listen to it using headphones. It is raining and the dripping water is at the front of the sound field, but there are also other sounds intruding, sometimes quiet.
As a listener, one may move inside of the sound, and all of its discrete aspects, including the particular spatial character of all the component sounds.
Information about it may be gathered through an intuitive analysis, from within the experience of listening and the spatial nature helps to do this.
An experienced sound designer is likely to be able to recreate at least parts of this soundscape with samples and synthesis based on such an analysis.
Signal analysis of the same recording may provide a large amount of additional information about the sound from which many aspects of it can be recreated, whereas others are extremely difficult to synthesize merely from an analysis.
Especially the spatiality of the sound is difficult to emulate merely from the technical analysis.

One of the advantages of working from the experience rather than merely the analysis is that for the listener the memories are entangled with our listening.
The listener's experience with being in similar environments in the past allows them to reconstruct the space and the way it transforms over time.
In an act of intuition the past and the future, as in the wish to recreate the sound, gets connected, which can be a powerful advantage compared to the deductive analysis.\footnote{Which is of course a valuable additional piece of the puzzle.} 

Returning now to the musician working with abstract sound in the studio, their listening situation is many times quite different as the relation between the sound and its source may be blurred to a high degree.
In these cases, the move to past experiences as a method for contextualizing and understanding the sound may be less obvious, in particular when the ambition is to create \emph{new} sounds.
However, it should be clear that the ability to use listening and reflection consciously paves the way for an understanding of sound that allows for knowledge  that is exclusive to this activity and cannot be replaced by other tools.
This discussion touches on several topics that are outside of the scope of this paper, such as a general phenomenology of sound perception, music semiology (\cslcitation{25}{Nattiez 1975}), reduced listening (\cslcitation{29}{Schaeffer 1977}) and many other theories.
Instead, I wish to focus the discussion on how Bergson's method, here as in \emph{listening from the inside}, can be useful in artistic research by putting forth a few examples.

One such example is the attempt to stage data transformations in composition where one type of sonic gesture provides information for another.
An obvious example may be a sound whose pitch in a  sweeping gesture falls from high to low.
The gesture of the pitch envelope may be transformed into a parameter to control the spatial transformation of the same sound, such as a spatial transformation from top to bottom.\footnote{This is sometimes referred to as audioparity (\cslcitation{36}{Valle 2018}), or self-audioparity (\cslcitation{6}{Catena 2021}). The latter refers to a recursive interaction between parameters of the sound. 'Spatial Sonorous Object' as discussed by Catena (\cslcitation{7}{2022}) is an analytical tool for understanding these possible transformation in a music analytical way.}
There are certain mappings between different domains that appear more generic than others, but in general, they are subjective.
The process of accessing them relies on a \emph{listening from the inside} that also engages the memory of past experiences which further influences the way the sound is understood.
As another example, imagine a mono recording of a car driving by.
Although there is no spatial information in the recording, for a listener who has seen and heard a car passing by it is not unlikely that the spatial information is added implicitly in the act of listening. 
Understanding these gestures on a detailed level also relies on listening practices that are embedded with compositional intent.
Sound itself becomes the source for the further development of the material in composition, and access to the various layers of the sound is supported by an intuitive mode of listening.
From this intuition, an analysis can be performed that allows for the discovery of sonic properties that may be used to construct methods for sound synthesis and compositional strategies.
This method can give rise to information about the elements of the  artistic practice that are useful also in an artistic research context.
\subsection*{Compositional practices and intuition}
\label{sec:org3d2a727}
When composing I obviously rely heavily on trying to intuitively understand the sonic materials I work with.
A current project I work with departs from a relatively simple idea with sonic material derived from basic analog oscillators to generate sound waves in a studio setting that mimics the electroacoustic composition studio in the 1960's.
The overarching goal of the project is to attempt to introduce change in my working process by replacing the modern studio of production and limit myself to the technologies that were available prior to the introduction of, in particular, the computer.\footnote{This, then, is related to the discussion earlier that the digital studio has influence on the practice of composition, and partly related to the fact that the computer has become a general instrument with which it becomes increasingly difficult to maintain originality. The attempt is to change the conditions for the composition process in order to focus on the act of listening.}
This, I hope, will allow me to better understand how various kinds of technologies affect my creative process.
This work in progress will only be presented briefly, and the main objective here is to point to another possible way of using Bergson's method of intuition and to understand the impact that it may have.

The general compositional idea departs from the beating that occurs between pitches in certain harmonic relations, typically when the pitch difference between two pitches is small.\footnote{A rough sketch for the basic layout of this composition was made in 1994 but was never completed.} The use of beatings is common in many contexts and is described in detail by Herman von Helmholtz (\cslcitation{19}{Helmholtz 1954}) in his seminal work \emph{On the Sensations of Tone}.
Sonic effects like the interference that gives rise to beatings\footnote{More complex auditory phenomena, like combination tones are discussed by Aron (\cslcitation{1}{2023}) in the thesis \emph{Phainesthai: Discovering Auditory Processes as a Tool for Musical Composition} which goes into depth with the artistic possibilities with playing with acoustic phenomena that only occurs through the act of listening. For a description of the difference between combination tones and beatings, see Helmholtz (\cslcitation{19}{1954} , p. 159).} show an example of a certain transgression of the sound that invites a widened listening experience: an effect arises that sometimes masks the original sounds and which allows the sound of the beating to take over: a "new and peculiar phenomena arise which we term interference"  (\cslcitation{19}{Helmholtz 1954}). 
There is nothing new about using interference in electronic music, and it is widely used in synthesis and processing.
What make it interesting in this context is the way it creates a sonic topology that guides the listening.
When still discernible the original sounds together with the added beating makes it possible to navigate the sound in multiple dimensions in the act of listening.
A sine wave by itself is not a harmonic sound and lacks the attractiveness of complex sounds, but two sine waves sounding together can in some cases be enough to create a dynamic sound that may contribute to drawing the listener into it.
In the simple example in \ref{fig:musicex_1} this process is exemplified with very simple means.

\begin{figure}[htbp]
\centering
\includegraphics[width=.9\linewidth]{img/musicex_1.png}
\caption{\label{fig:org622598b}Using three 7-limit intervals, the fourth and two closeby intervals, this simple example shows how the beating pattern is introduced with the 75/56 pitch creating 1.557 beats per second. Changing the interval to a smaller difference (75/56 and 98/75, too small to show in symbolic notation, has a beating of 8.534 beats per second) increases the speed of the beating and with the last interval, the speed of the beating decreases slightly. See also}
\end{figure}

Although the beating patterns between two intervals can be easily calculated,\footnote{The frequency of the beating between two simple tones is derived from subtracting the frequencies of the two tones \(f_1-f_2 = b\)} the sounding result of the interference is different than the calculation and, again following Bergson's idea that an analysis from the outside will be a reduction compared to one performer from the inside.
In the example in \ref{fig:musicex_1} the sound is a mono file which can be heard either as just one channel or the same sound in both channels, depending on your playback device. Putting two tones that generate a beating in different places in the sound field impacts on the experience.
In the short example in <sound beatings\textsubscript{binaural.wav}> the two sine waves are spaced apart (left and right) in the beginning and gradually panned to the center portraying the impact that space has on this effect.

For the composition, the pitch relations that I use are derived from a set of improvisations from which I deduct the patterns that I wish to continue working with based on Tenney (\cslcitation{32}{2008})'s harmonic space proposed in \emph{On ’Crystal Growth’ in Harmonic Space}.
Once I have found the series of sounds and continuous transformations that I wish to work with, I notate the pitches and the transformations I played.
This last aspect is added to maintain a certain conceptual stability to the process.\footnote{Also the notation is carried out using an add on to the program LilyPond that I developed for the purpose.}
The notation in this case is an abstraction of the analysis derived from the intuitive act of listening and tuning.
An excerpt of one such notated improvisation may be found in \ref{fig:musicex_2} and it should be noted that the main point of this exercice is for me to get acquainted with the material and tune my listening to the various forms for interferences in the intervals in the scale.

\begin{figure}[htbp]
\centering
\includegraphics[width=.9\linewidth]{img/musicex_2.png}
\caption{\label{fig:org098f65e}A short improvisation on a set of 7-limit intervals for which the first few bars have been loosely notated. Each of the 16 notes has been given its  position in a circle surrounding the listener with the root at the back of the listener. Also, this recording is binaural <sound beatings2\textsubscript{bounce.wav}>}
\end{figure}

The next stage in the process (which I have not started yet) involves a realization of the notation of the material back into sound, which will be performed in a studio environment designed in collaboration with \emph{Elektronmusikstudion} in Stockholm.
This studio has been equipped with signal generators, filters, tape recorders, mainly from the nineteen fifties and sixties that we have acquired from the large collection kept by the Swedish National Collections of Music, Theatre, and Dance.
In comparison to the digital studio used almost exclusively today, much of this equipment is noisy and inexact and the work process involves tedious repetition and is error-prone.
In the studio, I will interpret the notation with the tools available to me and record it using reel-to-reel tape recorders.
Although I have used a computer to generate some of the material as well as the notation, in the act of realization of the material I will limit myself to the equipment in the studio. 
Because of slight errors in the oscillator, the inexactness of the tape recorder, and the human factor, the end result will be an approximation of the seemingly exact notation.
It is only through listening that the acceptable margin of error can be assessed, and, in other words, the "correctness" may only be evaluated from the inside of the sound, not from the system alone.


It is incontestable that there is an active mode of listening in most compositional practices and I am not proposing that the listening performed in this project is different in nature.
As was the case with the saxophone-musician system described above, however, it is only partly correct to claim that it is from within the \emph{sound} that the intuitive relation to the material occurs in the different steps of the process.
The role of the listening here will to a much higher degree be connected to, and affected by, the larger system including all aspects of the activity.
The notation affects the listening, as does the equipment made to render the sound, as well as the system in which pitches were chosen.
This is where Bergson's method of intuition makes sense as a means to understand the artistic process:
Intuition allows me to engage with the system of production from within, but it requires that I acknowledge all of its parts from the moment of the birth of the concept, through the choice of pitches, timbres, and rhythms, to the notation and to the reinterpretation of the notation for the analog tone generators and the tape recorder.
Hence, this process that stretches out over both time and space, allows for a different modality of listening, different from what one may gather from listening to the sound alone without knowledge of, or access to, the information of the larger system.
This is comparable to how listening to the recording of rainfall at night discussed above, is an experience that depends on past experiences as well as present, and even future when the sound is decoded in an act of creative imagination.
Under the right circumstances, intuition can operate freely in this system and make me better understand where in the chain of elements adjustment needs to be made.
It may also reveal biases of the various parts of the system and the effect they have. 

I can engage analytically within this intuition, which is basically what may be referred to as reflection, and this analysis may also contribute to changes in the process.
With analysis from the outside, in Bergson's terminology, important information may be gathered, but the integrated understanding of the entire system will be difficult to achieve, as the parts of the analysis will be derived from different modalities: the sounding result and the memory of prior processes, such as the notation and conceptual development, will not be part of the same structure.
If I work in the studio I can use a spectral analysis tool to gather information about the sound, and I can learn a great deal about it this way, but if I wish to have a deeper understanding of the sound and its origin and meaning I need be able to also navigate in the larger network of activities that led to the sound.
I need to move in a continuum from outside to inside, from analysis to intuition.
\section*{Final reflections}
\label{sec:org132b467}
The discussion concerning how scientific technology such as advanced studio technology may require a mode of reflection different from the intuition of artistic sensibility  and artistic methods may have some added relevance in the contemporary landscape of advanced automated systems.
Part of the efficiency of already simple AI systems lies in the fact that the layers of operation between input and output are usually disguised.
There is no way of engaging intuitively with the AI in the way that is proposed in this article since only part of the entire system can be known: if the output is erroneous some parameters may be tweaked, but the system as a whole is extremely enigmatic.
The compositional system described earlier relies on access to all of the operations between input and output, and on the notion that these are integral parts of the whole without which much of the process will be, at best, difficult to navigate.

Returning to the main research question concerned with how sonic gestures can be understood as carriers of meaning the answer is that they do.
I do believe, however, that the studies presented here give some relevance to the fact that there is an inside perspective from which knowledge and information may be gathered and that it may be navigated with the method of intuition.
Just as Bergson makes clear, and which I have pointed out several times already, this knowledge is different in nature from what may be gathered from the outside perspective.
Hence, I believe that Bergson's method of intuition can lead to an understanding of sound within the process of playing or composing, and through the various elements on which the sound is dependent.
The epistemological nature of artistic practice in music, however, is complex and the proposed method is not enough in and of itself.
Nevertheless, intuition, as described here, may provide us with a method with which the artistic researcher may observe their  practice and extract relevant data.
This is a process that is productively informed by the method of intuition, and a process where important information may be gathered through intuitive analysis.
It is in the recursive interaction between this analysis, and the decision-making \emph{in the reflection upon these results}, that I argue are specific to artistic knowledge in music.
The question of how to present this information in ways that contribute to the general development of knowledge in the field is a larger question beyond the scope of this paper.

Sound, the way I have discussed it here, is not a thing, not an object, that we listen to.
It is a system of interrelated threads the meaning of which is much larger than the actual sound by itself. To engage artistically with sound is to attempt to understand the trajectories of this system, and each sound heard in this process may be intuited through the internal structure of this system.
Since artistic processes that are to a high degree already governed by a mode of intuition (in the traditional sense of the word) and sensibility the method proposed by Bergson is both interesting and useful as it allows for a different theoretical input.
It helps me to understand the material I am working with as well as the in-time process that I engage with when making decisions about the different steps in the process. The specific knowledge in this practice partly lies in the ways these decisions are being made, not merely in what material is being discovered.
To point out that listening to music is immersive may be unnecessary, but due to the fields of technology and artistic practice in the studio merging, putting the focus back on the attempt to understand the object from an inner immersiveness has relevance, so long as the definition of 'the object' that is being listened to is well considered.
\section*{Bibliography}
\label{sec:orge1a237f}
\begin{cslbibliography}{1}{0}
\cslbibitem{1}{Aron, Luka. 2023. “Phainesthai: Discovering Auditory Processes as a Tool for Musical Composition.” Royal College of Music, Stockholm.}

\cslbibitem{2}{Bergson, Henri. 1912. \textit{An Introduction to Metaphysics}. G. P. Putnam’s Son: New York.}

\cslbibitem{3}{———. 1991. \textit{Matter and Memory}. Zone Books, NY.}

\cslbibitem{4}{Borgdorff, H. 2007. “The Debate on Research in the Arts.” \textit{Dutch Journal of Music Theory} 12 (1): 1–17.}

\cslbibitem{5}{———. 2010. “The Production of Knowledge in Artistic Research.” In \textit{The Routledge Companion to Research in the Arts}, edited by M. Biggs and H. Karlsson, 44–63. Routledge.}

\cslbibitem{6}{Catena, Stefano. 2021. “Real-Time Algorithmic Timbral Spatialisation: Compositional Approaches and Techniques.” \textit{Proceedings of the 18th Sound and Music Computing (Smc) Conference, Torino, 2021}, 338–44.}

\cslbibitem{7}{———. 2022. “Concepts and Approaches in Analysing Spatial Gestures: A Link between Mozart and Acousmatic Music.” In \textit{2022 Xxiii Colloqui D’informatica Musicale, Ancona}.}

\cslbibitem{8}{Clair, André. 1996. “Merleau-Ponty Lecteur et Critique de Bergson. Le Statut Bergsonien de L’intuition.” \textit{Archives de Philosophie}, 203–18.}

\cslbibitem{9}{Clarke, Eric F. 2005. \textit{Ways of Listening: An Ecological Approach to the Perception of Musical Meaning}. Oxford University Press, USA.}

\cslbibitem{10}{Deleuze, Gilles. 1988. \textit{Bergsonism}. Translated by H. Tomlinson and B. Habberjam. Zone Books, Urzone Inc, New York.}

\cslbibitem{11}{Diagne, Soulemane Bachir. 2008. “Bergson in the Colony: Intuition and Duration in the Thought of Senghor and Iqbal.” \textit{Qui Parle} 17 (1): 125–45.}

\cslbibitem{12}{Dokic, Jérôme. 2007. “Two Ontologies of Sound.” \textit{The Monist} 90 (3): 391–402.}

\cslbibitem{13}{Frayling, Christopher. 1993. “Research in Art and Design.” \textit{Royal College of Art Research Papers Series.} 1 (1).}

\cslbibitem{14}{Frisk, Henrik. 2014. “Improvisation and the Self: To Listen to the Other.” In \textit{Soundweaving: Writings on Improvisation}, edited by F. Schroeder and M. Ó hAodha. Cambridge Scholars Publishing.}

\cslbibitem{15}{———. 2020. “Aesthetics, Interaction and Machine Improvisation.” \textit{Organised Sound} 25 (1): 33–40. \url{https://doi.org/10.1017/S135577181900044X}.}

\cslbibitem{16}{Frisk, Henrik, and Stefan Östersjö. 2013. “Beyond Validity: Claiming the Legacy of the Artist-Researcher.” \textit{Swedish Journal of Music Research} 2013: 41–63.}

\cslbibitem{17}{Galdon, Fernando, and Ashley Hall. 2022. “(Un)Frayling Design Research in Design Education for the 21cth.” \textit{The Design Journal} 25 (6): 915–33. \url{https://doi.org/10.1080/14606925.2022.2112861}.}

\cslbibitem{18}{Gaver, William W. 1993. “What in the world do we hear? An ecological approach to auditory event perception.” Ecological Psychology.}

\cslbibitem{19}{Helmholtz, H. 1954. \textit{On the Sensations of Tone}. Dover Books on Music Series. Dover Publications, Incorporated.}

\cslbibitem{20}{Hirai, Yasushi, ed. 2023. \textit{Bergson’s Scientific Metaphysics : Matter and Memory Today.} Bloomsbury Academic.}

\cslbibitem{21}{Ingman, Benjamin C. 2022. “Artistic Sensibility Is Inherent to Research.” \textit{International Journal of Qualitative Methods} 21 (January): 160940692110692. \url{https://doi.org/10.1177/16094069211069267}.}

\cslbibitem{22}{Kelly, Michael R. 2010. \textit{Bergson and Phenomenology}. Palgrave Macmillan.}

\cslbibitem{23}{Kjørup, S. 2010. “Pleading for Plurality: Artistic and Other Kinds of Research.” In \textit{The Routledge Companion to Research in the Arts}, edited by M. Biggs and H. Karlsson, 24–43. Routledge.}

\cslbibitem{24}{Lawlor, Leonard. 2003. \textit{The Challenge of Bergsonism}. Continuum.}

\cslbibitem{25}{Nattiez, J-J. 1975. \textit{Fondements D’une Sèmiologie de La Musique}. Union Gènèrale d’Editions.}

\cslbibitem{26}{Oliveros, P. 2005. \textit{Deep Listening: A Composer’s Sound Practice}. iUniverse.}

\cslbibitem{27}{Patton, M.Q. 2002. \textit{Qualitative Research \& Evaluation Methods}. SAGE Publications.}

\cslbibitem{28}{Russell, B. 1981. \textit{Mysticism and Logic, and Other Essays}. Barnes \& Noble Books.}

\cslbibitem{29}{Schaeffer, Pierre. 1977. \textit{Traité Des Objets Musicaux:Essai Interdisciplines}. Éditions du Seuil.}

\cslbibitem{30}{Shklar, Judith. 1958. “Bergson and the Politics of Intuition.” \textit{The Review of Politics} 20 (4): 634–56. \url{https://doi.org/10.1017/s0034670500034264}.}

\cslbibitem{31}{Smalley, Denis. 1986. “Spectromorphotogy and Structuring Processes,” 61–93. Basingstoke, Macmillan Press.}

\cslbibitem{32}{Tenney, James. 2008. “On ’crystal Growth’ in Harmonic Space (1993 - 1998).” \textit{Contemporary Music Review} 27 (1): 47–56.}

\cslbibitem{33}{Thompson, Geoffrey. 2009. “Artistic Sensibility in the Studio and Gallery Model: Revisiting Process and Product.” \textit{Art Therapy} 26 (4): 159–66.}

\cslbibitem{34}{Tresch, John, and Emily I. Dolan. 2013. “Toward a New Organology: Instruments of Music and Science.” \textit{Osiris} 28 (1): 278–98. \url{https://doi.org/10.1086/671381}.}

\cslbibitem{35}{Tuin, Iris Van Der. 2011. “‘A Different Starting Point, a Different Metaphysics’: Reading Bergson and Barad Diffractively.” \textit{Hypatia} 26 (1): 22–42. \url{https://doi.org/10.1111/j.1527-2001.2010.01114.x}.}

\cslbibitem{36}{Valle, Andrea. 2018. “Sampcomp: Sample-Based Techniques for Algorithmic Composition.” \textit{Proceedings of the 22nd Cim}, 128–35.}

\end{cslbibliography}
\end{document}