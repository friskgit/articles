% Created 2019-05-27 Mon 19:55
% Intended LaTeX compiler: pdflatex
\documentclass[11pt]{article}
\usepackage[utf8]{inputenc}
\usepackage[T1]{fontenc}
\usepackage{graphicx}
\usepackage{grffile}
\usepackage{longtable}
\usepackage{wrapfig}
\usepackage{rotating}
\usepackage[normalem]{ulem}
\usepackage{amsmath}
\usepackage{textcomp}
\usepackage{amssymb}
\usepackage{capt-of}
\usepackage{hyperref}
\usepackage[lf]{ebgaramond}
\usepackage{sectsty}
\allsectionsfont{\sf}
\usepackage[style=authoryear-ibid,natbib=true,backend=biber,hyperref=false]{biblatex}
\bibliography{/Users/henrik_frisk/Dropbox/Documents/articles/biblio/bibliography.bib}
\author{Henrik Frisk}
\date{}
\title{Harriet Bloch's past and future: journey into the night}
\hypersetup{
 pdfauthor={Henrik Frisk},
 pdftitle={Harriet Bloch's past and future: journey into the night},
 pdfkeywords={},
 pdfsubject={},
 pdfcreator={Emacs 26.1 (Org mode 9.1.9)}, 
 pdflang={English}}
\begin{document}

\maketitle
\emph{Journey into the night} (1921) is a silent German drama film
written by Harriet Bloch (1881-1976).\footnote{\emph{Journey inte the night}
  was directed by F. W. Murnau.} Bloch was one of Denmark's most prolific
screenwriters. According to her own notes she has written about 150
scripts. The Danish Film Institute has 55 titles, of which only
fourteen on nitrate film. Merely two films have been digitized by
DFI. Bloch, independent throughout her career, is not widely known, a
fact commented on by German philologist Stephan Michael Schröder
\citep{Schroder2011}. We have a rich collection of her material in our
possession that includes sixteen manuscripts, Hörspeil and theatre
plays along with note books, sketches, letters, contracts, and a
collection of poems that we have only begun to research and that is
likely to have a number of different outcomes. The particular question
for this presentation, however, is concerned with how we may
understand Bloch's legacy by means of artistic production. By
investigating the gap where her work has been concealed we hope to
also reveal aspects that may have an influence on how we explore the
media landscape around us, and reassess "existing media-historical
narratives that are biased because their ideological and
historiographical presuppositions." \citep{Huhtamo2013}

In the duo Mongrel, our artistic method is one where improvisation
play a central role. It has grown out of our thinking about
contemporary media and our attempts to critically examine both our own
pro-technical approach, and the hypermedia landscape we act and live
in. Our process is slow and meticulous and the project that we have
now engaged in is likely to go on for several year. In other words,
what we are proposing is a work in progress. A preliminary goal is an
intermedia work that includes images from Bloch's oeuvre as well as
re-enactments of her texts, not intended as drama but rather with the
intention to place focus on the writing and the person that wrote. In this presentation
we will discuss the process and its implications.

\printbibliography

\section*{Biographies}
\label{sec:org6bf825c}
The duo Mongrel, consisting of Anders Elberling and Henrik Frisk have worked together for several years on numerous audio/visual projects. The overarching ambition with their work is to critically examine the nature of the relation between audio and video. Their works have been performed in Denmark, Sweden, Belgium, Germany and Vietnam.

\section*{Henrik Frisk}
\label{sec:org97172df}
Henrik Frisk (PhD) is an active performer (saxophones and laptop) of improvised and contemporary music and a composer of acoustic and computer music. With a special interest in interactivity, most of the projects he engages in explores interactivity in one way or another. Interaction was also the main topic for his artistic PhD dissertation `Improvisation, Computers, and Interaction'. Frisk is Professor at the Royal Academy of Music, Stockholm and an affiliate of Malmö Academy of Music, Lund University. Henrik has performed in many countries in Europe, North America and Asia including performances at prestigious festivals such as the Bell Atlantic Jazz Festival, NYC and the Montreux Jazz Festival, Switzerland. As a composer he has received commissions from many institutions, ensembles and musicians. He has made numerous recordings for American, Canadian, Swedish and Danish record labels and is currently a member of the collective Kopasetic Productions, an independent label owned and run by improvising musicians.

\section*{Anders Elberling}
\label{sec:org8956e2d}
Anders Elberling is a visual artist, photographer, innovator, instructor, and art director. Originally educated as a photographer (Paris 1985-89) Anders uses visual objects and sound objects as tools to create artistic installations. Due to dyslexia, Anders has developed an imaginative access to communication through the creation of narratives using picture and sound. An important condition for Anders’ artwork is the artistic expression that evolves in the interaction between different art forms. This interaction creates new entities that are larger than those which can be generated by the individual art form alone. Therefore, Anders typically works in projects together with other artists – for example composers, sound artists, electronic and video artists – with whom he generates micro communities for the handling and composition of digital material through artistic practice. His work is predominantly presented at exhibitions and performances, concerts and theatre productions in broadcast, publications and installations. 
\end{document}