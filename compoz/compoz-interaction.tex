% Created 2016-10-24 Mon 16:35
\documentclass[12pt]{article}
\usepackage[english]{babel}

%% Disable when not using html output
%\usepackage{tex4ht}
%\usepackage{pxfonts}
%%%%%%%%%%%%%%%%%%%%%%%%%%%%%%%%%%%%

\usepackage[T1]{fontenc}
\usepackage{url}
\usepackage[utf8]{inputenc}
\usepackage{enumitem}
\usepackage{csquotes}
\usepackage{fixltx2e}
\renewcommand{\encodingdefault}{T1}

%% Enable for graphics 
%\usepackage[pdftex]{graphicx}
%%%%%%%%%%%%%%%%%%%%%%%%%%%%%%%%%%%%

\usepackage{setspace}
\usepackage[style=authoryear,natbib=true,backend=biber,firstinits=true,hyperref=false]{biblatex}
\bibliography{./../biblio/bibliography}
%\bibliography{./compoz-interaction}

%% Enable when using PDF output
 \renewcommand{\rmdefault}{pad}
 \renewcommand{\sfdefault}{pfr}
%%%%%%%%%%%%%%%%%%%%%%%%%%%%%%%%%%%


\author{Henrik Frisk}
\title{Varför interaktiv?}

\begin{document}

\maketitle

\noindent
Interaktiv musik. Begreppet antyder att det utmärkande med denna typ av musik är att den är just interaktiv, men det antyder också att det finns musik som inte är interaktiv -- varför skulle den annars behöva särskiljas från musik? I beskrivningen av projektet Compoz nämns inte interaktiv musik som begrepp men det strävar efter att ``stimulera interaktivitet''\footcite{compoz16}. Interaktiviteten sker genom ett gränssnitt uppbyggt av en infraröd kamera som ``ser'' hur besökarna rör sig i rummet. Kamerans signaler inverkar på musiken och musiken frambringar upplevelsen av ett virtuellt rum, fysiskt avgränsat av sexton högtalare i en kvadratisk formation. När jag rörde mig i det rum som uppstår innanför högtalarna kunde jag påverka min egen upplevelse av rumslighet genom musiken som spelades upp och som delvis kontrollerades av mina rörelser. Denna känsla av deltagande hos lyssnaren/deltagaren kan ses som målet för interaktiv musik och konst.

För att kunna fastställa på vilket sätt interaktiv musik skiljer sig från annan, ej interaktiv, musik måste man dock bestämma vad 'interaktiv' betyder, och egentligen också vad 'musik' innebär. Detta är ämnen som har avhandlats av många och som jag diskuterat i flera sammanhang och som jag därför bara helt kort kommer att sammanfatta här. Mitt intresse här ligger istället på interaktivitet som en estetisk kategori. För att kunna initiera den diskussionen behöver vi ändå en åtminstone preliminär definition av vad interaktiv musik är i det här fallet och det gör vi genom att begränsa kategorin 'interaktiv musik' till musik likt den som komponerades och framfördes i projektet Compoz: musik där lyssnaren kan påverka musikens form (i vissa fall även dess innehåll), ofta i samverkan med andra lyssnare, på ett sätt som har en signifikant betydelse för hur musiken upplevs.

På ett sätt kan man säga att det som i allmänhet erbjuds genom det interaktiva gränssnittet är ett instrument på vilket man som lyssnare själv spelar musiken. Orden \emph{instrument} och \emph{spelar} måste dock ses som metaforer i det här sammanhanget snarare än referenser till absoluta kategorier. De musikinstrument vi har idag har i många fall utvecklats under århundraden eller årtusenden och att lära sig spela på dem till full perfektion är något som tar tusentals timmar i anspråk. Gränssnitt för interaktiv musik brukar i allmänhet sträva efter motsatsen. De ska vara lätta att ta till sig och genom att interagera med dem ska man på relativ kort tid förstå hur de fungerar. Återkopplingen mellan aktivitet och perception ska vara snabb eller omedelbar. Men här närmar vi oss också det helt specifika med interaktiv musik: lyssnaren är både åhörare och utövaren och det är genom lyssnandet som hon väljer hur interaktionen ska fortlöpa. Detta blir speciellt tydligt i Compoz eftersom det inte finns en visuell representation av vare sig musik (annat än högtalarna) eller gränssnitt. Det finns inget taktilt och endast statiska högtalare att se på.

För ett ögonblick vill jag stanna upp vid det virtuella och responsiviteten i interaktionen. för förväntan på realtid har blivit en del av utmaningen i att arbeta med interaktiva konstnärliga system. Vi har blivit vana vid att erbjudas ett omedelbart svar när vi interagerar med vår teknik. En stor del av våra sociala medier handlar om just detta, om att leverera oförmedlad respons på våra inlägg. I en interaktiv installation jag själv nyligen gjorde hade jag ett webbaserat gränssnitt där man som lyssnare kunde trycka på en högtalare i en bild och på så sätt starta ett ljud i installationen. Det var intressant att notera hur många förväntar sig en omedelbar återkoppling på sitt bidrag, sitt klickande i gränssnittet. I min installation fick man en visuell feedback i gränssnittet, men inte när man klickade utan när ljudet började. Med ett ljud med lång insvängning kunde det därför ta flera sekunder innan den visuella responsen kom, något flera uppfattade som att gränssnittet inte fungerade. I konstnärliga sammanhang är det ett problem när det är teknikinteraktionen som sätter ramarna. Att interagera med teknik handlar om kontroll, något som inte alltid är lika eftersträvansvärt i konsten. Det konstnärliga mötet behöver tid och kritisk reflektion, det behöver eftertanke och utrymme i både tid och rum.

Den franska filosofen och framtidsdystopikern Jean Baudrillard har kommenterat den höga upplösningen i modern realtidskommunikation. I en värld upphängd på det virtuella och det omedelbara ser han ett motsatsförhållande mellan definitionsnivån på mediet och den i meddelandet: ett högupplöst medium genererar ett lågupplöst budskap. I slutändan blir det ett etiskt dilemma för den högsta möjliga upplösningen av den andre (som i ögonblicklig interaktion) leder till lägsta möjliga upplösning av utbytet med den andre.\footnote{\fullcite[p. 30]{baudrillard96}} I en komplex installation som Compoz finns det flera aspekter på detta, men gemensamt för interaktiva konstinstallationer, vill jag hävda, är att det måste finnas utrymme för att bryta med realtidparadigmet. Annars kommer viktiga möjligheter gå förlorade. Utmaningen blir då att ändå lyckas förse deltagarna med den stimulans som ger de känslan av att det är värt insatsen att fortsätta interagera. Alternativt, övertyga deltagarna att det är meningsfullt att delta även om deltagandet inte leder till en omedelbar känsla av kontroll och återkoppling.

I artikeln \emph{Some ontological remarks about music composition processes} diskuterar tonsättaren Horacio Vaggione återkopplingen mellan lyssnandet och skapandet som en slags valideringsprocess som han definierar som en ``action-perception feedback loop''.\footnote{\fullcite{vaggione01}} Han skriver att lyssnandet är en central del av kompositionsprocessens olika faser. Om vi vidgar definitionen av 'skapandet' och inkluderar det estetiska upplevelsen som en slags skapande, så är det naturligtvis helt uppenbart att lyssnandet är centralt som Vaggione påpekar. Det är det sannolikt vare sig vi talar om ett inre lyssnande eller ett lyssnande till akustiska ljud. Det är också uppenbart att tidsdiskrepensen mellan \emph{action} och \emph{perception} måste finnas för att tillåta reflektionen att äga rum. Kanske kan man lite tillspetsat säga att en interaktion vars resultat är känt och i huvudsak leder till bekräftelse inte innehåller denna möjlighet för eftertanke och upptäckt. Finns utrymmet för reflektion, dock, blir lyssnandet en mental aktivitet, ett tänkande-genom-lyssnande, som på vissa sätt är väsenskilt från annan typ av tänkande. När vi drömmer oss bort genom vår favoritmusik är det inte nödvändigtvis ett rationellt eller logiskt tänkande utan ett tänkande som rör och påverkar andra delar av vårt medvetande -  en del av detta är estetiskt. När vi tänker-genom-lyssnandet, när vårt lyssnande är parat med en aktivitet så uppstår möjligheten för ytterligare en dimension. Detta rymmer också det estetiska, men har en annan epistemologisk potential än andra typer av lyssnande

I artikeln \emph{Negotiating the Musical Work} sätter gitarristen Stefan Östersjö och jag Vaggiones resonemang i ett hermeneutiskt perspektiv och pekar på hur 'tänkande' genom musikalisk aktivitet inte behöver språkliggöras genom aktion-perception loopen.\footnote{\fullcite{frisk-ost06-2}} Upplevelsen kan till exempel vara rent musikalisk eller estetisk. Dessutom pekar vi på hur lyssnandets tänkande och performativitetens tänkande är centrala aspekter av det vi kallar interpretation, eller i detta sammanhang mer allmänt som, tolkning eller meningskapande.

I en interaktiv installation som vill vara inbjudande ställs det krav på att gränssnittet är enkelt att ta sig till. Återkopplingen till deltagaren, som nämndes tidigare, är viktig men i bästa fall ska systemet vara självinstruerande, eller i annat fall ha en tydlig instruktion. Att ett instruments svårighetsgrad överhuvudtaget kan justeras och göras så enkelt att spela på beror mycket på musikteknikutvecklingen. Egentligen måste det påpekas att själva ordet 'instrument' här är missvisande då det inte är instrumentet självt som erbjuder olika nivåer av interaktion. Ta en synthesizer som exempel. En duktig musiker kan utveckla en unik röst genom övning och erfarenhet och lära sig att spela musik i en mängd stilar och traditioner. Samma synthesizer kan utnyttjas av en kompositör och utgöra gränssnitt till en komplex interaktiv komposition som kräver lite eller ingen erfarenhet av den som interagerar med den. Det är alltså inte det fysiska instrumentet, syntesizern, som skapar förutsättningarna för den interaktiva musiken relativa enkelhet, utan det som döljer sig bakom. 

Oftast så är det också så att ökat innehållsmässig flexibilitet leder till mer komplext gränssnitt, alternativt krav på högre kunskapskrav. Musikern som investerar tid och lär sig spela på sin synthesizer kan anpassa sin musik och estetik till många olika sammanhang, medan den som spelar på den interaktiva kompositionen inte sällan är begränsad till kompositörens eller programmerarens estetiska val. Ofta så finns en motsvarande relation som ger att låg ingångströskel ger låg möjlighet till variation (när man upptäckt hur instrumentet fungerar så riskerar man att tröttna) och omvänt (hög ingångströskel ger goda eller oändliga utvecklingsmöjligheter). Ett akustiskt piano kan ses som ett genialiskt undantag från detta förhållande i det att det har låg ingångströskel (ett barn kan spela på ett piano utan att förstå hur det fungerar) och oändlig utvecklingspotential (man blir aldrig fullärd). Genom att begränsa valen och flytta en del av musikskapandet från framförandefasen till kompositionsfasen kan vi påverka den upplevda svårigheten att spela på instrumentet/gränssnittet.\footnote{Per-Anders Nilsson använder begreppen design time och play time decisions. Genom att flytta valen från play-time till decision-time kan de tekniska kraven på framförandet flyttas från en nivå till en annan.} Samtidigt har vi som utövare dock förlorat en del av vår kontroll över kompositionen. Viktiga estetiska val har redan gjorts. 

Vad är då egentligen poängen med att låta en del av valen i musikskapandet ske i framförandet av en publik som i bästa fall har en tidigare relation till musik -- i sämsta fall ingen relation till den alls? Varför inte göra färdigt musiken, låta den vara fixerad i sin egen relativa tid och bara spela upp den? Frågan är felaktigt ställd och är jämförbar med att fråga sig varför ett visst stycke musik är skrivet för stråkkvartett och inte jazzkvintett. Elektroakustisk musik som spelas upp utan interaktivitet är en estetisk kategori som har lite att göra med den som bygger på interaktivitet likt den vi diskuterar här. Det interaktiva elementet i interaktiv musik är en del av musikens meningsskapande. Att förstå att man som lyssnare är medskapare av musiken är något som påverkar lyssnandet på ett helt fundamentalt plan. Det triggar igång en kroppslig dimension av det tidigare nämnda tänkande-genom-performativitet. Att jag är kroppsligt engagerad i en musik gör att jag hör och förstår den musiken annorlunda än om jag inte är det. Om jag endast lyssnar på musiken så är det tänkandet genom lyssnande som är den modell genom vilken jag skapar upplevelsen.

Alltså, såväl det passiva lyssnandet som det aktiva, participatoriska, är processer med vilka lyssnaren skapar upplevelsen, men den underliggande logiken mellan dessa två är väsensskild. Jag har själv experimenterat med detta och det participatoriska lyssnandet, det som är aktivt när jag deltar i framförandet, ger en annan bild av musiken som skapas än det passiva återlyssnandet till samma musik. Det är inte stor skillnad men tillräckligt stor för att det passiva lyssnandet kan ge en otillfredställande bild av en musik som i det aktiva lyssnandet upplevdes som bra. Att bedöma estetiken i interaktiv musik måste alltså ske genom att man själv är interaktiv. Interaktiviteten går inte att skilja från musiken utan är den process genom vilken vi kan tolka och förstå.

Compoz lägger till den kollaborativa aspekten till detta. Är jag ensam kan jag spela styckena själv, men den verkliga ingången till dessa verk är när jag kan interagera tillsammans med andra lyssnare. Det är här som det kroppsliga lyssnandet kommer till sin rätt. Jag kan röra mig med eller mot mina medlyssnare och antalet dimensioner i det interaktiva landskapet växer exponentiellt. Faran är uppenbar, jag tappar känslan för vad jag som individ bidrar med. Denna oro är bygger dock på ett missförstånd av interaktivitet i konst och är rotad i ett konsumistiskt förhållande till kultur. Naturligtvis vill jag bli tillfredställd och få ut något av min investerade tid men lika självklart som det kan tyckas lika viktigt är det att frigöra sig från den individuella förväntan på återkoppling. Det är interaktiviteten som skapar mötet men det är det kollektiva medskapandet som upprätthåller det och bara om jag ger mig hän kommer jag kunna förstå musiken. Det interaktiva verket är inte en proxy för mötet mellan människor som det kan vara i till exempel en orkesterkonster: individer i publiken upplever en närhet för det har delat en liknande upplevelse. Men det är inte musiken som extern faktor som är behållningen i Compoz utan musiken som facilitator. Det är lyssnandet till musiken genom interaktiviteten som skapar förutsättningen för den utvidgade närvaron tillsammans med andra.

Tillbaks till frågan om den interaktiva musikens identitet. Finns det egentligen någon musik som inte är interaktiv? Kanske kan man säga att den minst interaktiva musiken är den som spelas upp på en CD eller liknande. Det är bara få aspekter av den musiken som påverkas eller förändras av vad jag som lyssnare samtidigt gör. Men på ett filosofiskt plan är det omöjligt att påstå att inte lyssnaren såväl som musiken påverkas av lyssnandet. Man kan förvisso inte stänga öronen, som man kan med ögonen, men man kan välja att inte lyssna aktivt. Lyssnandet kan äga rum på så många olika nivåer och med så många olika intentioner, och alla dessa kommer att påverka nästa lyssnande på ett sätt som gör att jag faktiskt kan hävda att musiken, även om den är graverad som ettor och nollor på en CD, är annorlunda nästa gång jag lyssnar på den.

I Compoz är ettorna och nollorna graverade på en hårddisk. Inspelade ljud, samplingar, spelas upp i systemet, distribuerade i de åtta högtalarna. I några fall, när tiden mellan varje kopia av en sampling var för kort, blir det tydligt att det jag hör är en exakt repetition av något som tidigare har spelats. Den digitala repetitionen är skoningslös. Det borde vara omöjligt att skapa något som upplevs som en exakt kopia på något som spelades strax innan, men den digitala inskriptionen och precisionen döljer effektivt denna omöjlighet. Detta är ytterligare en av utmaningarna med interaktiv musik. En inspelning av ett gestaltat stycke musik lyssnar vi på just så, och då, som tidigare diskuterats, följer tänkandet genom lyssnandet på den gestaltningen. I interaktiv musik är en del av gestaltningen överlämnad till mig som deltagare och det är då min aktivitet som tillsammans med mitt lyssnande skapar gestaltningen som ger upphov till min upplevelse. Denna process kan jämföras med en slags komplex dramaturgi. Om jag helt och hållet hänger mig till lyssnandet och glömmer att jag är medskapare riskerar jag att upplevelsen paradoxalt nog förlorar sitt grepp om mig. Jag drömmer mig bort i ett tänkande-genom-lyssnande. Den interaktiva musiken kan aldrig helt och hållet försiggå i en annan verklighet utan som lyssnare och deltagare måste jag vara aktiv, och aktivt tro på mitt deltagande. Interaktionen och det resulterande tänkande-genom-lyssnande är på så sätt en slags Brechtiansk \emph{Verfremdungseffekt}. Kamerorna och mina publikkollegor är kritiska observatörer till dramat som pågår. Det är först då, i det skedet, som de digitala kopiorna upphör att vara exakta repetitioner och istället anknyter sig till gestaltningen.

---------

Ända sedan amatörmusicerandet dog ut - en process som började för över hundra år sedan\footnote{I det följande citatet pekar Adorno på hur kammarmusikens fall började med hemmamusicernadets uttåg: ``Chamber music remains possible, not as maintenance of a tradition that has long been moth-eaten, but only as an art for experts, something quite useless and lost that must be known to be useless if it is not to decay into home decoration.'' \fullcite[p.102-3]{adorno76}} - har Västerländsk konstmusik inte lyckats med att attrahera nya publikgrupper i någon större utsträckning. Konserthus fortsätter att i huvudsak vara en angelägenhet för priviligierade grupper i samhället, även om dessa har en annan sammansättning idag än för femtio eller hundra år sedan, och även om ambitionerna att förändra finns. Den experimentella och smala musiken, å sin sida, har även den fortsatt svårt att hävda sin plats i det nya medielandskapet, och också denna genre söker efter kontaktytor mot de publikgrupper som inte självmant söker upp dessa konsertscener. Till detta kan vi lägga det fetischistiska lyssnandet, som i sin mest extrema form utvecklar och förfinar ett musikintresse som knappt längre inkluderar just själva lyssnandet. Det är istället helt fokuserat på detaljerna och artefakterna -- skivomslag och partiturutgåvor, ljudkvalitet och inspelningsteknik -- och blir därigenom en konsumtion och en statusmarkör. Adorno beskriver en typisk medlem i denna grupp av musiklyssnare som undviker det 'vanliga' lyssnandet på grund av kraven detta ställer på honom:

\begin{quote}
He respects music as a cultural asset, often as something a man must know for the sake of his own social standing; this attitude runs the gamut from an earnest sense of obligation to vulgar snobbery. For the spontaneous and direct relation to music, the faculty of simultaneously experiencing and comprehending its structure, it substitutes hoarding as much musical information as possible, notably about biographical data and about the merits of interpreters, a subject for hours of inane discussion. It is not rare for this type to have an extensive knowledge of the literature, but of the sort that themes of famous, oft-repeated works of music will be hummed and instantly identified. The unfoldment of a composition does not matter. The structure of hearing is atomistic: the type lies in wait for specific elements, for supposedly beautiful melodies, for grandiose moments. On the whole, his relation to music has a fetishistic touch. The standard he consumes by is the prominence of the consumed. The joy of consumption, of that which--in his language--music "gives" to him, outweighs his enjoyment of the music itself as a work of art that makes demands on him. \citep[p.6-7]{adorno76}
\end{quote}

Jämfört med denna nidbild av ett musikintresse som har tappat fotfästet och spolat ut barnet med badvattnet framstår min, förmodligen bedrövliga, amatördans till Compoz som ett initierat och begåvat engagemang med en levande musik i utveckling.

Det är i detta sammanhang som Compoz blir extra intressant. För det första så är det interaktiva lyssnandet, som beskrivits ovan, väsenskilt från det fysiskt passiva lyssnandet. I det kroppsliga mötet med musiken uppstår en förståelsehorisont som är specifik för denna genre och som står i kontrast till det tekniskt distanserade musiklyssnandet. Den inkluderar inte bara den omedelbara tillfredställelsen av att vara med, av att deltaga, utan underbygger en estetisk dimension som kvalitativt påverkar hur musiken kan uppfattas. För det andra tillkommer den kollaborativa dimensionen där man som lyssnare tillåts agera, och därmed lyssna, tillsammans med andra. För det tredje är själva skapandeprocessen i Compoz från början kollaborativ och öppen. Den har skett tillsammans med människor som inte alltid har haft en enkel tillgång till nutida konst och musik. Denna aspekt kan man se som något som skapar förutsättningarna för den öppenhet man upplever som deltagare. Tack vare att den traditionella skapandeprocessen med upphovsmannen i centrum har brutits ner har materialet kunnat presenteras, interageras med och lyssnas till med en stor portion öppenhet. 

Men, egentligen vill jag vända på det och hävda att det är den kollaborativa metoden som lyssnandet här bygger på och den som skapandet har byggt på, som möjliggjordes av den estetiska potentialen i den interaktiva konsten. Det är den ändrade grundförutsättningen för det deltagande och kroppsliga lyssnandet som gör att den modernistiskt singulära skapandeprocessen redan från början kan brytas ned. Interaktionen som modell ställer krav på hela produktionskedjan och bejakar man den finns det fantastiska konstnärliga möjligheter.


%\printbibliography
\end{document}