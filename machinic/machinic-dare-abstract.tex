% Created 2017-02-04 Sat 21:17
\documentclass[11pt]{article}
\usepackage[utf8]{inputenc}
\usepackage[T1]{fontenc}
\usepackage{fixltx2e}
\usepackage[]{graphicx}
\usepackage{longtable}
\usepackage{float}
\usepackage{wrapfig}
\usepackage{rotating}
\usepackage[normalem]{ulem}
\usepackage{amsmath}
\usepackage{textcomp}
\usepackage{marvosym}
\usepackage{wasysym}
\usepackage{amssymb}
\usepackage[hidelinks]{hyperref}
\tolerance=1000
\usepackage[lf]{ebgaramond}

\usepackage[style=authoryear,natbib=true,backend=biber,firstinits=true,hyperref=false]{biblatex}
\bibliography{./../biblio/bibliography}

\author{Henrik Frisk and Anders Elberling}
\date{\today}
\title{Machinic propositions: artistic practice and deterritorialization}
\hypersetup{
  pdfkeywords={},
  pdfsubject={},
  pdfcreator={Emacs 24.5.1 (Org mode 8.2.5h)}}
\begin{document}

\maketitle

\begin{abstract}
  \emph{Machinic propositions} is a project started by the duo \emph{Mongrel} in
2015. It is simultaneously an artistic project and an attempt to critically examine Deleuze and
Guattari's theorems of deterritorialization as found in chapter seven
and ten of their seminal work \emph{A Thousand Plateaus}. The output has taken a few different shapes and
has used different kinds of media. Like much of our other works \emph{Machinic propositions} is
part of the attempt to counteract the predominance of one medium over
the other, in particular, video over audio. In this short paper we
discuss our artistic method in which narrativity and improvisation play central
roles. It has grown out of our thinking about contemporary media and
an attempt to critically examine both our own pro-technical
approach, and the hypermedia landscape we act and live in.

In this project, we have looked at the
relation between the two media as a system of
de/reterritorialization. Our practice, like many other artistic
practices, may be likened to a rhizome, a network of ideas
that in the beginning is spread out on a plane. Eventually, and partly
through a self-organizing process and conceptual development, a folding of this space is taking place. Nodes that in the beginning may
have been located far from each other may now be situated in
close proximity. Thereby, they become accessible nodes of interaction
in our practice.


\end{abstract}

% Emacs 24.5.1 (Org mode 8.2.5h)
\end{document}
