% Created 2016-10-24 Mon 16:35
\documentclass[11pt]{article}
\usepackage[utf8]{inputenc}
\usepackage[T1]{fontenc}
\usepackage{fixltx2e}
\usepackage{graphicx}
\usepackage{longtable}
\usepackage{float}
\usepackage{wrapfig}
\usepackage{rotating}
\usepackage[normalem]{ulem}
\usepackage{amsmath}
\usepackage{textcomp}
\usepackage{marvosym}
\usepackage{wasysym}
\usepackage{amssymb}
\usepackage{hyperref}
\tolerance=1000
\usepackage[lf]{ebgaramond}
\usepackage{sectsty}
\allsectionsfont{\sf}
\author{Henrik Frisk}
\date{2014-01-15 Wed}
\title{Hell is full of musical amateurs, but so is heaven}
\hypersetup{
  pdfkeywords={},
  pdfsubject={},
  pdfcreator={Emacs 24.5.1 (Org mode 8.2.5h)}}
\begin{document}

\maketitle
Det finns kanske inte någon konstform som är så upphängd kring det ekvilibristiska som musiken. Det går en rak linje från Beethovens Diabelli variationer till Yngwie Malmsteen: deras respektive bländande teknik \emph{är} i sig kvalitet. Det musikaliska framförandet ska samtidigt som det på ett plan är anonymt och avpersonifierat föra fram den unika individualitet som verket bär fram \emph{och} den fria konstnärlighet som exekutören porträtterar. Det är naturligtvis helt omöjligt. Det i sig är kanske \emph{en} anledning till den starka mytbildning som omringar fältet och som också odlar det metaforrika ordflödet som präglar t.ex. recensioner och programkommentarer. Mer om det senare.

Därför är amatörismen som estetisk kategori är så otroligt intressant, den är den ekvilibristiska musikens själva motpol. En förlösande process för mig i min utveckling som improvisatör var att sluta försöka spela "bra". 

\textbf{anekdot}
<ethersound>

Jag tror många kanske kan förstå eller relatera till det: Strävan efter det perfekta är inte alltid kreativt stimulerande.  Genom att bejaka det dåliga, fula, oprecisa kunde jag tränga igenom det konstnärliga överjaget. Det ständiga utvärderandet av resultatet tog fokus bort från processen och skapandet. Kreativiteten vill inte alltid vara bra, den vill också få vara ful. Här blir kanske amatörism fel ord, det rör sig snarare om något vi kan kalla konstruerad naivitet: det var med en medveten naivitet jag angrep mitt instrument.

Vad betyder det? Är det bra att det låter dåligt? Nej. Språket och vårt tänkande är ofta väldigt upphängt kring dikotomier av olika slag men det finns naturligtvis ingenting som säger att \emph{dåligt} är samma som frånvaron av \emph{bra}. Detta har inget med estetik eller postmodernism att göra, utan är snarare kopplat till det kognitiva. Jag kan uppfatta det konceptuella som bra och utförandet som dåligt, eller tvärtom. Eller en musik som kroppsligt bra men intellektuellt dålig. Vi hänvisar ofta till sammanhanget och säger att det har med kontexten och göra. Technomusik är bra i sin kontext. Jag tror att även det är en förenkling. Mänsklig perception är i grunden multi-modal och vi kan höra samma musik, inklusive vår egen bild av den den musiken samtidigt.

För mig öppnade det kontrollösa och fula upp för nya klanger och nya uttryck som i sig tränger in i det reflekterande medvetandet och sedan också blir intellektuellt. Dessa klanger hade jag kanske kunnat nå genom på andra vis också - eller inte. Poängen är den att strävandet efter perfektion inte alltid leder framåt. Det gör inte heller den konstruerade naiviteten.

Om perfektion är upphävande av motstånd genom övning är amatörismen att inte ta hänsyn till förväntant motstånd. Perfektionen får det svåra att se lätt ut men naivismen ignorerar det svåra i det svåra och riktar uppmärksamheten på det andra. Det är inte slumpen som skapar skönheten, det naiva kan vara helt och hållet planerat och samtidigt fullständigt oordnat. Det är oordningen som flytter perceptionens fokus från resultatet till processen.

Det "dåliga" kan vara en politisk akt eller det kan vara ett uttryck för frihet - i bägge fallen kan det såklart vara hämmande för den andre. Men det kan även perfektionen. Kultursidornas litteraturkritik fick ofta kritik för att man inte skrev om deckare och för att kulturkritiken kunde vara helt ogenomtränglig för gemene man. Recensioner av inspelningar av klassisk musik skrivs i ett språk och med en terminologi som även kan utesluta en person som mig: ibland förstår jag helt enekelt inte skillnaden mellan två tolkningar som för recensenten är skillnaden mellan himmel och helvete.

<portsmouth symphonia>

På tidigt sjuttiotal sammanställde den brittiske tonsättaren och improvisatören Gavin Bryars gruppen Portsmouth Symphonia som kom att bli amatörismens själva epicentrum och därigenom ett i mitt tycke 1900-talets mest briljanta konceptuella musikaliska konstverk. Den konstnärliga idéen var att spela klassiska repertoarverk med medlemmar som antingen inte kunde spela eller som inte spelade ett instrument som de behärskade. Detta var punk fem år innan punken föddes. Det var en dekonstruktion av motsatsparet bra/dålig musiker som en absolut kategori. Det var ett ifrågasättande av borgerlig högkultur och den utveckling som idag lett till att en inspelning av fem minuter orkestermusik kan innehålla över hundra editeringar. Det är långt mer än humor, även om det är det också. Den visar hur vi hör "verket" genom de många lager av "dålig" behandling det har fått samtidigt som vi hör den komiska misshandel som samma verk utsäts för. Portsmouth Symphonia visar också hur påtagligt central ekvilibrism och precision är i västerländsk konstmusik och tvingar oss att ställa oss frågorna: Vad är det vi hör? Och vad är det vi vill höra? 
% Emacs 24.5.1 (Org mode 8.2.5h)
\end{document}