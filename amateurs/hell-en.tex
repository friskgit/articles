% Created 2016-10-24 Mon 16:35
\documentclass[11pt]{article}
\usepackage[utf8]{inputenc}
\usepackage[T1]{fontenc}
\usepackage{fixltx2e}
\usepackage{graphicx}
\usepackage{longtable}
\usepackage{float}
\usepackage{wrapfig}
\usepackage{rotating}
\usepackage[normalem]{ulem}
\usepackage{amsmath}
\usepackage{textcomp}
\usepackage{marvosym}
\usepackage{wasysym}
\usepackage{amssymb}
\usepackage[hyperfootnotes=false]{hyperref}
\tolerance=1000
\usepackage[lf]{ebgaramond}
\usepackage{sectsty}
\allsectionsfont{\sf}

\usepackage[style=authoryear,natbib=true,backend=biber,firstinits=true,hyperref=false]{biblatex}
\bibliography{./../biblio/bibliography}

\author{Henrik Frisk}
\title{Hell is full of musical amateurs, but so is heaven}

\begin{document}

\maketitle
% This text is an attempt to look at the development of musical practice in Western classical music. I have no intention to give a full historical recollection. Rather, the method has been to look for sources that may help give light to phenomena such as the topic for the seminar in Copenhapen on September 17, 2016, my own experiences as a performer with questions concerning 

% This text is an attempt to unfold a commonly discussed phenomenon in Western classical music: the role of the composer and his or her relation to the musical work. This is obviously a huge topic and I make no claim that it will be fully or even satisfactorily explored. One point of departure is the performances during the symposium XXX in Copenhagen in September 17, 2016, and the discussions that followed. Another one is my own artistic practice as a composer and improvisor. By highlighting the development of the concept of the musical virtuoso, as well as of the musical amateur, the meaningfulness of perfection is questioned. 
% Roland Barthes summarizes these different kinds of listening in his seminal text \emph{Musica Practica} written at a time that had only begun to see the kind of complexity musical composition would develop. The text begins by a simple statement: ``There are two kinds of music (at least I have always thought so): the music one listens to, the music one plays'', and Barthes argues that these two kinds of music are entirely different arts. There is an element of nostalgia here. The daughter in the bourgoisie home is not likely to ever again entertain her parents and their friends in the aristocratic saloon, but is that really significant? That kind of music making was only reserved for a small number of people of the upper classes anyways, even though it ``lapsed into an insipid social rite with the coming of the democracy of the bourgeoisie''. 

%Musica practica - division of the act of music
% - division of musician
% - the amateur - Barthes
% - the embodied listener - Adorno

%adornBarthes
%Neuro

Today, when we think of musical performance in Western art music, it is easy to take for granted the division of labor between, for example, musician and composer. However, music has obviously been produced for many thousands of years without there being a need to compose and write it down before playing it. Most genres of music have sustained and developed without this split between creator and performer. In genres where improvisation play an important role the musician sometimes embodies both the creative act and the interpretative--simultaneously. In musics built on aural traditions the composed component is integrally bound to the musician. However, as advanced and standardized technologies for systematic notation of music were developed in Europe the role of the musician slowly began to evolve into two, often separate parts: one part primarily responsible for the construction of music (composer), and one part primarily responsible for the performance of it (musician). There is no doubt that composition and notation are extraordinarily efficient means to structure, communicate and preserve musical ideas and it is fair to assume that the development that led to the division of labor loosely sketched here participated in the advancement of Western music into new aesthetic areas.

Despite the advantages of the split between composer and performer, however, it has also come at a cost. What are the consequences and what effects has it had? The question concerning the composer as the romantic and original genius both questioned and promoted today, that Sanne Groth Krogh approached in her lecture in the seminar \emph{Komponister på scenen} (Composers on Stage) in Copenhagen in September 2016 is related. So is the question concerning artistic freedom that was also discussed in the seminar. But perhaps most intriguing to me is the question of the amateur, the non-professional musician, the music-doer, the antithesis of the brilliant expert performer. The amateur may be regarded an embodied music listener that may reach a sensibility towards the music without reach for the non-performing, disembodied listener. Amateurism also holds an emancipatory power: to be allowed to let go of the restrains of professionalism, so important for the social, economical and technological developments in music. This emancipation makes the move from result to process possible.


\vspace{0.4cm}
\begin{center}
  ***
\end{center}
\vspace{0.4cm}

If the split of the musician in two roles is one significant trait of Western art music, the importance of the musical amateur for its development is perhaps another. Based on the assumption that someone engaged in musical practice, regardless if this is on a professional level or as an amateur, listens differently to music than someone who does not, the musical amateur may have played an important role of the aesthetic development of classical music. Knowing how to physically play a piece of music in effect alters how you listen to it. But musical practice in general also appears to influence listening in general. In other words, within a genre, studying and learning how to play a piano sonata probably has influence on how an orchestra piece may be experienced when listened to. And even further, through my experience in the ensemble and research project \emph{The Six Tones}, this transformation of knowledge from one kind of musical practice (e.g. Vietnamese) can be transferred to knowledge about another kind of musical practice (e.g. Western European experimental music) \cite{frisk12-improv, frisk2013}

Roland Barthes points out in his seminal essay \emph{Musica Practica} that there ``are two kinds of music (at least I have always thought so): the music one listens to, the music one plays.'' \citep[p.149]{barthes77} The music one plays is an embodied activity as well as a spiritual. The listening takes place through the body, through memories of muscular activity and allows for a reading that is significantly different from listening passively to music. 
% The amateur musician was of course not only a musician but also a informed listener, but the heir to this kind of listener, according to Adorno, is the culture consumer.
Now, Barthes maintains that not only is there a difference between listening through playing, there is also a particular quality of the music performed by the amateur, one that the specialist cannot reach. His analysis goes beyond what at first seems like nostalgia for old times, and identifies a subsequent change that has also altered music aesthetically:

\begin{quote}
  The amateur, a role defined much more by a style than by a technical imperfection, is no longer anywhere to be found; the professionals, pure specialists whose training remains entirely esoteric for the public [\ldots] never offer that style of the perfect amateur [\ldots] touching off in us not satisfaction but desire, the desire to \emph{make} that music. In short, there was first the actor of music, then the interpreter (the grand Romantic voice), then finally the technician, who relieves the listener of all activity, even by procuration, and abolishes in the sphere of music the very notion of \emph{doing}. \citep[p.150]{barthes77}
\end{quote}

The amateur music life that Barthes is describing, however, has disappeared: the sons and daughters of the bourgeoisie homes are not likely to ever again entertain their parents and friends in the aristocratic salons playing Mozart and other composers. Also music sociologist Theodor Adorno remarks that the musical amateur has had a difficulties at the time of the dismantling of traditional European aristocracy: ``The amateur's best chance of survival may be where remnants of an aristocratic society have managed to hold out, as in Vienna.''  \citep[p.6]{adorno76} It should be stressed, however, that the present discussion is limited to the field of Western classical music. The amateur musician is obviously still common  in many, or even most, other styles of music.
% , but is that really significant? That kind of music making was only reserved for a small number of people of the upper classes anyways, even though Barthes describes how it ``lapsed into an insipid social rite with the coming of the democracy of the bourgeoisie''. 
% The amateur performer played an important part in the musical salons of bourgoise societies.

%HÄR SKA KOMMA IN LYSSNANDET SOM AKTIVITET, ecriture, in
Beyond what may then appear as a sentimental remembrance of a cultural activity long lost there is a more important aspect related to the disappearance of amateur musicians and the concurrent development of the new functions of the musician. The quote above by Barthes should be understood in the light of his literary theory of how it is not the origin of a work (literary or musical) that should be the decisive point, but the destination \citep[]{barthes77}. It is in the active listening that the work is understood, not in the preparatory phase of interpretation, or even in the act of composition. 
%For the text, the elements are not deciphered, they are to be seen in their full multiplicity and strung together as independent voices, as in a polyphony. 
The internal and coherent meaning of a text, according to Barthes, will surface in the discursive and communicative act of reading. This is allowed to happen through the special bond between the text and reader in a manner similar to how the relation between the music and the \emph{doer} of music develops through the physicality of playing. In other words, there is a special kind of understanding made possible by the physical resistance involved in playing an instrument not available in the ``liquid, effusive and 'lubricating' '' \citep[p.150]{barthes77} experience of disembodied listening to ``the technician, who relieves the listener of all activity'' (ibid).
%\footnote{There is some evidence to the fact that the physical experience of playing music effectively makes a difference for the listening. Neurological studies has shown that musician respond physically when listening to music. \citep{bangert2003,overy2009} This is consistent with my own experience. Not only do I respond physically to listening to the kinds of music that I play -- less so when listening to musics I have no playing experience of -- but listening to music that I enjoy also evokes a strong wish to participate and join in with the other musicians. Oddly, this happens even when I listen to recorded music and is probably related to the urge to move to music one enjoys.} 
The consequences and the underlying mechanisms concerning these topics are, I would argue, quite complex. We will return to the aspect of the physicality of playing and for now focus on the social concerns of the matter.

Adorno gives some support to the idea that Western classical music has been influenced in important ways by active domestic music-making, activities that are no longer taking place in the same manner. He notes that the decline of chamber music in general coincides with the decline in amateur music making, and further that

\begin{quote}
  chamber music remains possible, not as maintenance of a tradition
  that has long been moth-eaten, but only as an art for experts,
  something quite useless and lost that must be known to be useless if
  it is not to decay into home decoration.'' \citep[p.102-3]{adorno76}
\end{quote}
Undoubtedly a description strikingly similar to the critique offered by Barthes.

Jacques Attali points out, perhaps a bit pointed, that the commodification of music can be said to have begun with the copyright laws.\footnote{According to Attali copyright laws were originally not primarily there to protect the artist's rights. They were tools of capitalism against the feudalism that was ruling at the time. \citep[p.52]{attali85}}
But not only compositions had to be valorized, also performances. Musicians were paid, and compositions were commissioned also before the introduction of copyright laws, but the commodification of music as practice was greatly enhanced by it. As Jacques Attali puts it: ``In order for music to become institutionalized as a commodity, for it to acquire an autonomous status and monetary value, the labor of the creation and interpretation of music had to be assigned a value.'' \citep[p.51]{attali85} In other words, for the yet non-existing music industry to develop, all aspects of the creation and performance of music had to be professionalized. In his alternative overview of the music history (that also draws upon Barthes) Michael Chanan turns the focus to the great commercial exploitation of music listening as a factor that prompted the concept of the musical amateur to vanish. In this process the musical amateur was simply in the way:

\begin{quote}
In driving out the amateur, the whole vast modern commercial apparatus of music conspires to reduce the listener to the condition of compliant consumer, and thus to induce passive reception instead of active listening. The concert becomes parasitic upon the fame and success cultivated in the festival; the festival is the showcase which the impresario needs to capture for the promotion of the artist; the artist to catch the impresario's attention, must now win international music competitions. \citep[p.29]{chanan1994}
\end{quote}


Now, not only do we here witness the need for the field of music performance to become commercialized, we also see the rise of the star, the virtuoso, whose perfect renderings of perfect compositions in itself disallows the amateur performer. The musical equilibrist, the virtuoso delivering outstanding technical brilliance is a product made possible by the professionalization of the trade of musical performance as much as by division of labor. As performers do no longer have to take responsibility for the organization of the musical material and do not need to improvise, they are able to fully focus on refining their technical dexterity. These master performers have become exceptionally important figures for music---and the music industry---since the mid 19th century. The role transcends genre borders and there is a straight line from Beethoven's \emph{Diabelli Variations} (to which we will return soon), composer/performers such as Lizt, and someone like the Swedish Heavy Metal guitarist Ynqwie Malmsteen. The mastery of the staggering performance is a quality that stands by itself and feeds the myths surrounding this field. The equilibrist is the performer's equivalent of the romantic composer genius.

The ideal performance of a score by the virtuoso performer whose artistic goal it is to fully adhere to the wish of the composer (whether this wish is known from a first hand communication with the composer or if it is extracted from the sources such as the score, bibliographic references or the research of others is not really important here) will risk at resulting in an idealist reduction of the music onto a smooth, two-dimensional representation in the form of musical notation. This, claims British electronic music composer Trevor Wishart, is a consequence of a view on music where the score ``is seen as normative on the musical experience'' and where 

\begin{quote}
  ``the spatialisation of the time-experience which takes place when
  musical time is transferred to the flat surface of the score leads
  to the emergence of musical formalism and to a kind of musical
  composition which is entirely divorced from any relationship to
  intuitive gestural experience.'' \citep[p.35]{wis96}
\end{quote}
Wishart's mention of the ``gestural experience'' is obviously related to the muscle memory of the performer as an embodied listener discussed above. To use the score as a proxy in the communicative act of the musical performance is what Wishart is criticizing and as such it is clearly a reduction of music as a phenomenon. Intuitive gestural and embodied experiences takes time, space and money.

What several of these writers are pointing at is an increased distance between the music listener and the different sites of production of music. This alienation is manifold. It appeared at first between the composer and the composition, as a consequence of thenfact that the act of inscription had been detached from the performance. The score becomes a product and a work independent of its sonic trace. Wishart concludes that 

\begin{quote}
  ``with the increasing domination of notation, there has been a move
  towards Platonic idealism in our conception of what music is. In the
  most extreme cases, music is viewed as an essentially abstract
  phenomenon and the sound experience of essentially secondary
  importance.'' \citep[p.35]{wis96}
\end{quote}
It should come as no surprise that also music is affected by how it is represented. 

%If the written part of the expression is given precedence over the performed part, then perhaps we have to go with Barthes suggestion that we need to read the music rather than play it, in order to navigate its meaning. 
%I doubt it.

As a result of the economy of specialization and division of labor came the professionalization of the master musician, the technical wonder, exploited as an entertainer that in the end produced the passive and disembodied listener. The hierarchical nature of classical music is difficult to ignore here, and at worst, it leads to the listener being alienated from the music, and the composer from the musician, which in extension should render any musical activity useless. When listeners no longer can maintain a physical relation to the music, either because they cannot play it or because they do not understand it, only its social value is left. 
%As absurd as it may seem the most abstract of all artforms then becomes merely an object for consumption.

% If the performer is the body, the virtuoso composer is the mind, representing individuality as an aesthetic quality and freedom as a political category. The aim for individuality has led to a constant redefinition of the symbolic languages sometimes creating intimately connected composer-performer constellations. The efforts involved in both creating and reproducing the music sometimes become so great and so individually oriented that the possible generality of the text in standard notation is given up in favor of an elusive specificity. This may be seen as a counter movement to the commodification of music, and perhaps it is, but it certainly moved the amateur performer beyond the periphery of contemporary music.

Even if there would still have been bourgeoisie salons with plenty of opportunities for amateurs to meet and play, it would not have been contemporary music. The aesthetic development in Western classical music in the last century has made the musical dilettante an anachronism. Can we even imagine a musical amateur engaging in performing Boulez' \emph{Piano Sonata No. 2} or Ligeti's \emph{Etude} at a social gathering at home? With reference to Boucourechliev's often cited \emph{Essay sur Beethoven} Barthes points to how the great compser's \emph{Diabelli Variations}, completed in 1823, marks the end of understanding through hearing or playing: from here on Beethoven needed to be \emph{read}. The listener needs to participate in the making, ``with respect to this music one must put oneself in the position or, better, in the activity of an operator who knows how to displace, assemble, combine, fit together.''  \citep[p.153]{barthes77} We have held on to the general structure of Western classical music over time. We have the same concert houses, the same kinds of ensembles, the same social structures, conductors and soloists, virtuousos and eccentric composers. But the aesthetic and social underpinning has changed so much that the relations between the roles in the system are difficult to make sense of now. Boulez' music, like Beethoven's Diabelli variations, is music that requires an operator rather than a listener. The amateur musician was the operator at one point. Then the score, the text, the writing changed in a way that raised the need for a different kind of performer, a technician, and a different kind of listener, an assembler. Barthes postmodern tools have some flaws when transferred to the field of music, and they may not give us the full picture but they are helpful nonetheless.

The last piece of the puzzle, the final advancement that made the amateur musician obsolete all together was the introduction of the radio and later modern recording technology. The extreme amounts of available sources for listening and the added level of perfectionism that modern techniques for editing recordings allows for makes Barthes observation about a specialist's music that has become ``entirely esoteric for the public'' alarmingly accurate. Why bother to learn how to play the piano only to render poor versions of Schumann's piano pieces when I have access to 50 excellent quality recordings at the blink of an eye? This can be seen as another and consequential division of labor, and it offers a slightly altered view on the music recording industry. Freeing the music listener from the burdens of \emph{playing} music by offering perfect recordings, the listener can concentrate on \emph{consuming} more music. And it is more lucrative to sell interpretations of which there can be an infinite number, than to sell scores.

\vspace{0.4cm}
\begin{center}
  ***
\end{center}
\vspace{0.4cm}

In 1970 the British improviser and composer Gavin Bryars put together group of musicians associated with the Portsmouth School of Art in Great Britain. The orchestra, \emph{Portsmouth Sinfonia}, played pieces from the standard classical repertoire such as \emph{Williamm Tell Ouverture} by Rossini and \emph{Jupiter} from The Planets by Holst. The peculiar thing about this ensemble, however, that dissociates it from other orchestras is that the group was generally open to anyone, regardless of musical training. Musicians that were accomplished instrumentalists could join the orchestra but would then have to play an instrument they did not master or choose to play an instrument that was new to them.

\emph{Portsmouth Sinfonia} was the Western classical music's version of punk, but materialized several years before Sex Pistols arrived on the scene. As conceptual art \emph{Portsmouth Sinfonia} is exceptionally powerful. It effectively critiques the powers of exclusion that the world of classical music maintains. Everyone that listens to their performances understands that the music is not well performed in the traditional sense, there is very little nuance in the ``badness'' of the music. At the same time it is obvious to anyone who listens that only a genuinely enthusiastic performer would sustain playing this music in this manner, and this fact loads the performance with passion and spirit. Discussing quality in traditional orchestra performances can often be an activity closed to anyone except the most engaged listeners, critics and musicologists. Subtle changes of tempo, phrasing and agogics make a world of difference to the initiated and knowledgeable listener while these are qualities near impossible to grasp by the uninitiated.

For \emph{Portsmouth Sinfonia} the search for perfection and absolute precision is forsaken and replaced by sheer joy of playing. It is quite funny to listen to their rendering of \emph{An den schönen blauen Donau}, out of tune and out of time, but despite the layers of bad treatment that the score is subjected to the performance is still, in some sense,``true to the score'' insofar as there is no doubt that this is really Johan Strauss' II famous waltz. But this music is much more than humor. \emph{Portsmouth Sinfonia}'s version of \emph{Also Sprach Zarathustra} makes audible the physical effort involved in the performance in a way that a ``professional'' rendering of the piece disguises. The body of the performers is heard and the resistance in the performance is brought to the foreground; it adds a quality that a skillful performance would attempt to hide.

Before I was aware of Gavin Bryars' work I experimented with methods to suppress the deep-rooted search for perfection in my own playing \citep[][]{frisk2013}. Excellence and precision can curb even the most ardor performer at times and stall artistic development. By approaching my instrument as if I had no idea how to play it, with the devotion and eagerness of a child, I sought for new openings. Unless drugs or other substances are used, however, it is obviously impossible to undo thousands of hours of practice and reach a level of true naivety. Rather we are talking about methods to temporarily bypass the artistic super ego. Constant self-evaluation in search for an optimal result may take the focus away from the process, materiality and conceptualization of the artistic practice. Creativity is not always `good', it may be ugly, unpleasant, provocative and even repulsive, and a practice that denies these negative qualities will in the end also dispute the full creative potential.

%In \emph{Portsmouth Sinfonia} the attitude towards the music was guaranteed through the lack of expertise and that particular condition I cannot replicate unless I choose to play another instrument. But the attitude is possible to recreate and it can be perhaps be discussed as a dynamic between \emph{creation} and \emph{discovery}. Without knowledge one is discovering possibilities, with training one is creating them, but this is an epistemological question beyond the scope of this article. As an improviser it was rather my habitual musical and instrumental responses I felt an urge to break with and the method succeeded. I did break loose of some destructive ties and my discoveries made its way into my practice where they were refined and integrated into my vocabulary. The lack of control, and the freedom that followed, opened up for new modes of expressions and new sounds that eventually reached a level of conscious reflection that became intellectually oriented. These sounds were surely available to me through other processes as well and the method here should primarily be understood psychologically. The bottom line is that striving for perfection is not necessarily a way forward. It can hinder both the personal expression and exclude the listener from access.

What \emph{Portsmouth Sinfonia} shows to me is closely related to the way Barthes showed us how the text can be at the center, though avoiding the kind of musical logo-centrism that Wishart is warning us about. While what they do is music, that fact that the music is not ``well performed'' does not alter the conceptual value of the art, the text. As a matter of fact, had their version of \emph{An den schönen blauen Donau} been excellently performed the work would have been redundant and uninteresting. It is the way they operate the text (the text in this case would be the general concept, the score and the instructions) that makes it great.

This is similar to how many composers today are exploring the boundaries for their own practice. They enter the creative process and the performance from angles that are unexpected which allow them to read the practice, and read the work, in ways that are not accessible from the composers desk. They become amateurs in one sense while the extension of the practice is absolutely natural. Artistic practice is no longer one isolated activity that leads to a \emph{work}. It is a perpetual movement that only occasionally stops to deliver a temporary performance. The key to understanding one particular instance of the movement (which I have called a \emph{work-in-movement} \citep{frisk08}), however, is not the work as a \emph{work}, but rather the work-in-movement: it is the context that allows us to operate, displace, assemble, combine and fit together. What I would wish for, and what I try to do, is to go full circle: invite the audience to take part again, to join the performance.





% In the process of protecting the freedom of the author the amateur - the music-doer - needs to be neutralized because he or she destabilizes the central locality of the composer and the possibly infinite number of interpreters as a system of reference.
% The freedom of the composer relies on the 



% In the beginning notation allowed musicians to inscribe mnemonic traces of the music they played that would range from improvisations and known melodies to theoretically well structured compositions that eventually led to prescriptive notation and detailed symbolic representation of music. 

% The more involved the compositions became, the greater the need for a virtuouso performer. The master performer and intrepreter is needed to reinvoke the music from the symbols inscribed by the composer. In Western art music the dynamic between the composer, whose music requires ever more proficient interpreters and performers as we enter the 20th century, and the performer, whose unimaginable level of mastery keep us astonished, becomes a significant trait of classical music starting at the mid 19th century.

% The amateur musician was an important figure in the music up unitl the romantic era. 







 % All are parasitical upon the recording industry, which bitterly complains about the loss of profits from home recording, but is in turn a parasite upon radio and television, its operations heavily subsidized by the almost free publicity provided by these media.  \citep{chanan1994}








% The prize of this development is alienation understood as a rupture between the music and the listener, albeit with a slightly different impact than what it might be seen to have in the line of production of goods. 

% \begin{quote}
%   He respects music as a cultural asset, often as something a man must know for the sake of his own social standing; this attitude runs the gamut from an earnest sense of obligation to vulgar snobbery. For the spontaneous and direct relation to music, the faculty of simultaneously experiencing and comprehending its structure, it substitutes hoarding as much musical information as possible, notably about biographical data and about the merits of interpreters, a subject for hours of inane discussion. It is not rare for this type to have an extensive knowledge of the literature, but of the sort that themes of famous, oft-repeated works of music will be hummed and instantly identified. The unfoldment of a composition does not matter. The structure of hearing is atomistic: the type lies in wait for specific elements, for supposedly beautiful melodies, for grandiose moments. On the whole, his relation to music has a fetishistic touch. The standard he consumes by is the prominence of the consumed. The joy of consumption, of that which-in his language-music "gives" to him, outweighs his enjoyment of the music itself as a work of art that makes demands on him.  \citep[p.6-7]{adorno76}
% \end{quote}






% \begin{quote}
% The split in conception between what are seen as primary and secondary aspets of musical organisation and interpreation and the gradual devaluation of non-notable formations. This developmnet leads directly to the attitudes expressed by Boulez and to the intellectual devaluation forms of music where non-notable aspects of musical form have greater importance than in conventional classical music.\footnote{The primary and secondary aspects referred to above points to a distinction originally made by the french composer Pierre Boulez by which the primary are those that may be notated - primarily pitch and rhythm - and secondary are those that relate to color (timbre), expression and the like.}
% \end{quote}




% \begin{quote}
% For example, at one extreme, a professional musician listening to music which they know how to perform (e.g., a saxophonist listening to a saxophone piece they know well) is able to access precise information at all levels of the hierarchy, from imagined emotional intentions to specific finger movements and embouchure. At the other extreme, a musical novice listening to unfamilar music from an unknown sound source (e.g., someone who has no knowledge of the existence of saxophones) is not able to access precise information at any level, but may feel the beat, sub- vocalize, and interpret emotional intention accordingly (e.g., fast, loud, and high in pitch might be considered emotionally charged). Thus, the resonance or simula- tion mechanism implemented by the human MNS matching perceived and executed actions allows a lis- tener to reconstruct various elements of a piece of music in their own mind (bringing together auditory, motion, and emotion information), and the richness of that reconstruction depends on the individual’s musical experience.  \citep[p.493]{overy2009}
% \end{quote}



% Det "dåliga" kan vara en politisk akt eller det kan vara ett uttryck för frihet - i bägge fallen kan det såklart vara hämmande för den andre. Men det kan även perfektionen. Kultursidornas litteraturkritik fick ofta kritik för att man inte skrev om deckare och för att kulturkritiken kunde vara helt ogenomtränglig för gemene man. Recensioner av inspelningar av klassisk musik skrivs i ett språk och med en terminologi som även kan utesluta en person som mig: ibland förstår jag helt enekelt inte skillnaden mellan två tolkningar som för recensenten är skillnaden mellan himmel och helvete.




\printbibliography
\end{document}