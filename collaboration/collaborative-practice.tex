\documentclass[a4paper,hidelinks]{article}
\usepackage[english]{babel}
\usepackage[T1]{fontenc}
\usepackage[utf8]{inputenc}
\usepackage[authoryear,round]{natbib}

\makeatletter
\renewcommand\bibsection%
{
  \section*{Referenser
    \@mkboth{\MakeUppercase{\refname}}{\MakeUppercase{\refname}}}
}
\makeatother

\usepackage{graphicx}
\usepackage{hyperref}
\renewcommand{\rmdefault}{pad}

\usepackage{fancyhdr}
\pagestyle{fancy}

\lhead{\small{\textit{Henrik Frisk}}}
\chead{}
\rhead{}

\title{Artistic and creative collaborations - dismantling the creative genius}
\author{Henrik Frisk, PhD}
\date{}

\begin{document}
\selectlanguage{english}
\maketitle

\thispagestyle{empty}

%\section{Introduction}

\noindent
\begin{abstract}
  Much of my work is based on collaboration and interaction, and in
  this talk I will discuss a couple of different projects and how
  their collaborative aspects were carried out. A point of departure
  is that interaction of several forces will only properly function if
  neither of the collaborators is claiming authority.
\end{abstract}

As an improviser collaboration is part of ones everyday practice. To improvise with other musicians is to collaboratively take responsibility for the performance. The responsibilities can shift, one musicians can move forward and take the lead and then let someone else assume the responsibility. This is in my experience a self-organizing process that is rarely discussed, talked about nor negotiated. The importance and impact of individuality is a much more present factor of jazz aesthetics.

Very often we talk about ``feeling'' in music, in this context somewhat similar, but not identical, to intuition. Self organization in improvisation works through the agency of ``feeling''. It is my ``feeling'' that guides me when the structures of the in-time musical collaboration is unfolding. The ``feeling'' however, is also negotiated between the participants in the performance and is created and boosted through a feedback processes: We self organize based on feeling-as-intuition in order to reach feeling-as-property which feeds feeling-as-intuition, etc. Ingrid Monson points to how this feeling-as-property is something that is larger than the musicans, something that points to a higher awareness, so to speak.

In an important collection of texts discussing the epistemology of improvisation (The Other Side of Nowhere \citet[p. 17]{fischlin2004}), editors Daniel Fischlin and Ajay Heble argue that improvisation is an ongoing process of community building much more than an individual expression. (Fischlin and Heble, 2004, 17) In The Field of Musical Improvisation we write:

\begin{quote}
  Besides being a musical assemblage, the FMI can be regarded as a
  social assemblage playing a material role: it involves a set of
  human bodies properly oriented towards each other, mainly through
  nonlinguistic forms. (DeLanda, 2006, 12) The FMI enables an
  encounter between human beings that takes place in a physical and
  social setting through the medium of sound. (Small, 1998, 10) In the
  FMI musicians thus explore a set of relationships. Bodily posture,
  facial expression and limb movements provide a wide repertory of
  gestures and responses which give crucial information about the
  combination of social and musical relationships. To make music is to
  experience those relationships.
\end{quote}

However, there is an important distinction to be made between the realm of the social and that of the improvisational. The ``feeling'' is influenced by social interaction but not necessarily created through it. The ethical dimension of the meeting, face á face (Levinas), depends on mutual respect and understanding, but in music the negotiated feeling between players is in some respect apersonal. Its guiding principle is to create good music, not good relations. The other in this case is the music rather than the musician. Miles Davis was known for inducing tension between the musicians in his band. Two saxophonists that I know have told the same story independently of each other and the events are separated in time by at least 15 years. Miles would call another saxophone player to a gig without telling each of the two players about the other. Suddenly there would be two saxophonists and no one quite knew what to do. In the concert Miles would walk up to each and tell them to play better and show the other guy that he has nothing to do at the gig. This would go on for a succession of concerts until one of the two broke and quit, else Miles would eventually just stop calling one of them. 

Apart from being part of a power structure and rather twisted method it is a proof that the social and the musical are separated. One of these concerts where this happened are recorded and released and it is a brilliant performance with an amazing level of musical interaction.

\section{Extended collaboration}
\label{sec:extend-coll}

In my own work this collaborative aspect of improvisation has always fascinated me and one of the reasons I have always had difficulty in engaging in composition. Composing music takes place in solitude, and to some degree the process is to perfect an idea \emph{before} presenting it to the world similar to writing a book or painting a picture. Up until 2004 my method was to leave some of the work unfinished and left to the performers to negotiate in an attempt to bring in the aspect of integrated self-organization. In 2005, however, Stefan Östersjö and I started a project, one that is still going on, that opened up the collaborative possibilities in musical interactions in a quite radical way substantially different from my earlier methods.

At the time of launching this process we were not aware of how far it would take us, nor of the consequences. Initially the idea was to begin by collaboratively analyze an interactive process concerning the construction of a work for guitar and computer and use that data in our own project. However, the results we got were quite different to what we had expected and the negotiation appeared to be taking place at many levels. Without getting into the details here what this process led to was our rethinking of the nature of the musical work. It may appear surprising but there is still a fair number of academics that consider the score rather than its sonic realization as the musical work. The ways in which musical notation has developed the breach between the sound and its representation has increased in certain kinds of music during the 20th century. This process has further emphasized the division of labour between musician and composer that notation introduced in the first place, a dividing line further emphasized by Beethoven and seemingly irrevocably set in stone by modernist composers such as Stockhausen and Boulez.

With the piece \emph{Repetition repeats all other repetitions} for guitar and electronics we challenged some of the ideas surrounding the musical work focussing on the mobility of the work, its potential development and expanded field rather than the work as product.

\emph{Repetition Repeats all other Repetitions} is an open form composition for 10-stringed guitar and electronics. It was premiered in Beijing in 2006 and has been performed many times since then, in three separate versions. The piece emerged out of a collaboration between the composer Henrik Frisk and the guitarist Stefan Östersjö, an artistic research project in which interaction in the widest sense was allowed to play a major part already at the outset. In the preparatory phase, and through the first incarnations of the piece, the idea of a radically open work type, the work-in-movement, crystallised (Eco 1989). One of the conditions that allowed for the development of this openness was the disassembly of the hierarchies attached to the roles of composer and performer and one of its consequences was that intuition was allowed to play a great role in the work.

Though the score is quite detailed, the way the segments are combined is up to the performer. Henceforth three different versions have been produced: in the first two the structure was settled prior to the performance, and for the third version the choices were made interactively in real time. The performer is allowed to interrupt the segments at whatever place, leave it and go to another segment. Not only through in-perfromance interaction is the piece open, it also allows for the construction of other versions of the piece, other means of organisation of its materials.

\begin{itemize}
\item Division of labour
\item Ontology of the musical work
\item Work identity
\item Composition through improvisation
\item Auteur-ship
\item Collaborative practices
\item Performance practice
\item Musical interaction
\end{itemize}

\begin{itemize}
\item \emph{Work-in-movement}. This is a concept established by Umberto \citet{eco68} that I introduced as a work type encompassing radically open works. It requires different modes of representation, as the traditional musical score is too restrictive and is not able to communicate its most central aspects: the collaboration, negotiation and interaction in the conception, realization and documentation of the work. Work-in-movement does not necessarily distinguish between composition and improvisation, although for the latter, some kind of frame is needed for the concept to be meaningful. As a specification it is geared towards modes of interaction and openness involved in all phases of the work.

\item \emph{Interaction-as-difference}. I proposed that in human-computer interaction (HCI) the methodology of control (interaction-as-control) in certain cases should be abandoned in favour of a more dynamic and reciprocal mode of interaction, interaction-as-difference. This kind of interaction is an activity concerned with inducing differences \emph{that make a difference} \citep{bateson72:steps} and suggests parallelism rather than the typical click-and-response mode of HCI. In essence, the movement from \emph{control} to \emph{difference} is a result of rediscovering the power of improvisation as a method for organizing and constructing musical content. Interaction-as-difference is to be understood as an alteration of the more common paradigm of direct manipulation in HCI.
 
\item \emph{Giving up of the self}. I suggested the notion of \emph{giving up of the self} as the common denominator between the two previous concepts and as one of the important conditions for an improvisatory and self-organizing attitude towards musical practice that allows for interaction-as-difference. Only if one is able and willing to accept the loss of priority of interpretation, if one is willing to give up or disregard faithfulness to ideology or idiomatics, is the idea of interaction-as-difference conceivable. Hence, the giving up of the self is not to succumb to someone else but is rather contingent on the degree to which one is willing to engage in a dialogue on the creative process and to allow others to influence it. I.e., by giving up compositional control and replacing it with an interactive negotiation in the form of collaboration, the process realizes all of these three topics.
\end{itemize}

The iconified Western auteur mentioned above has been under attack at least since the sixties. The mythical creator behind non-negotiable works of art enjoys a natural freedom of expression and does not have to answer to criticism. My argument \citep{frisk08} is that the view of this Kantian genius is still very influential to how we teach and present art. For myself, merely realizing this was not enough. I needed to more profoundly understand what problems I had with this role in relation to my own artistic practice, and in what ways I could neutralize the expectation of being in control. The decisive moment occurred when I was working on the interactive sound installation \emph{etherSound}, where I was forced to accept that a large part of the compositional decisions had to be made by the users of the system rather than myself. In order to fulfil the idea behind the piece I had to come to terms with the fact that I was not able to restrain the input of the flow of users. I had to give up that which \citet{boulez64} refers to as ``the 'finished' aspect of the Occidental work, its closed cycle'' (p. 51) and approach the ``open work'' that \citet{eco68} discusses, to further attempt ``to reach that point where only language acts, `performs', and not `me'' \citep{barthes77}. Obviously, as an improviser the notion of the open work was familiar to me, but the dynamically open work that I was now approaching was not something I had experience with. Furthermore, the view of the authoritative creator is also something improvisers are often confronted with, and the view of musicians and composers as being absolutely clear about all details of their work is as prominent in improvised music as it is anywhere else. 

In Roland Barthes seminal essay \emph{The Death of the Author} we find many interesting ideas that parallel the concepts of embodiment, de-individualization and habit destruction as ``abrupt disappointment of expectations of meaning''\citep{barthes77}. The importance of individuality in many expressions and the ego-centered view on artistic production are in many respects related to the role and significance of the author, brilliantly interrogated by Barthes. The way in which the artistic 19th century genius has been shaped has created a mythology so powerful that it has had an impact on much of our understanding of \emph{any} artistic figure, authors, composers and musicians alike. The creative act is so strongly soldered to this romantic image that even the understanding of an improvising musician, whose creativity depends not on work creation, but on the real-time impulses in performance, which are very volatile by nature, is informed by it. In opposition this romantic view Barthes claims that:

\begin{quote}
Writing is the destruction of every voice, of every point of origin. Writing is that neutral, composite, oblique space where our subject slips away, the negative where all identity is lost, starting with the very identity of the body writing. \citep[p. 142]{barthes77}
\end{quote}

Furthermore, the preoccupation with the author is a consequence of the view of the work as emanating from its creator. Hence, the only relevant way to understand the work is through understanding the author's background, life, and context: ``The \emph{author} still reigns in histories of literature, biographies of writers, interviews, magazines, as in the very consciousness of men of letters anxious to unite their person and their work through diaries and memoires \citep[p. 143]{barthes77}.'' The reading in the larger sense of the word is the process of decoding the message, not in an act of critique but an act based on a reconstruction of the author, recombining the parts that he constitutes and, through this structure, being able to understand the true meaning of the work. As we know, this is in essence the focus of traditional musicology, to reveal the composer bit by bit and understand his work through the history of his life: Where did he live? Who was his maid? What did he eat? Where did he study? Though these questions may well be relevant for the study of our cultural and social history, the extreme focus on the individuality of the composer has had a strong influence on the interpretation and reading of his work at the expense of the position of the listener and that of the performer.

If we transfer Barthes's statement that ``a text's unity lies not in its origin but in its destination'' \citep[][p. 148]{barthes77} to the domain of music, the subjectivity of the performer would not be operative in the act of listening to an improvisation. Its unity is instead in the destination, the listener. Even today, almost a half century after the text was first published, this notion is still provocative.

\section{Interdisciplinary collaboration}

Releasing the artist from the chains of individuality allows for a different kind of interdisciplinary interaction. When issues of ownership and control are done away with a new possibility for meaningful interaction reveals itself. Rather than merely juxtaposing expressions on top of each other the collaboration can work on lower levels of organisation. This is obviously not the only way in which collaboration can take place, nor is it a guarantee that such collaborations will be successful. However, an open collaboration without a strong need for a clear output and defined framework can become a work-in-motion.

\subsection{Examples}

\subsubsection{Elberling/Frisk}

A video/sound duo working with installations, performances and videos. Tight conceptual collaboration with a method developed specifically for the context. The duo is credited for both the audio and the video and we work primarily with open source software. We work with concepts for weeks and even months before even touching the artistic production aparatus. Once we do we finnish the material in quite short time. 

\href{http://www.henrikfrisk.com/index.jsp?metaId=res\&id=music\&field=date\&query=2013-10-01\&show=-1}{Go to Hell}

\subsubsection{Frisk/Östersjö/Thanh Thuy/Ngo Tra My}

Together with the two Vietnamese master musicians Stefan and I have worked since 1995. It is a project peparting from political and postcolonial dimensions working with artistic as well as academic modes of output.

\href{http://www.henrikfrisk.com/index.jsp?metaId=res\&id=music\&field=date\&query=2013-05-15\&show=-1}{With only my hands}

\subsubsection{Isaac Julien}

This is a project different from the other collaborations described here. Julien being a celebrated video artist cannot afford to let go of the control of his aesthetic. Nevertheless, Stefan and I managed to carve out a space for ourselves in which we could build our own, somwhat independent world. This alternative space was eventually embraced by Julien too and the collaboration was succesful.

\href{http://www.henrikfrisk.com/index.jsp?metaId=res\&id=music\&field=date\&query=2011-09-11\&show=-1}{Better Life}

\subsubsection{The Field of Musical Improvisation}

\href{http://musicalimprovisation.free.fr/index.php}{The Field of Musical Improvisation} was a work we did for an online magazine \href{http://konturen.uoregon.edu/}{Konturen} (University of Oregon). It is an example of a more academically oriented collaboration with the ambition to integrate modes of expression into one unified format.

\section{Summary}

I believe that collaboration is one of the main methods for development of the arts. In many instances it is already there and operational but not acknowledged. In film the predominantly male directors are getting all the action while the creative process is distributed among a wide range of actors. Why is Bergman celebrated for his particular gaze when we know that Sven Nyqvist has had a huge impact on the composition? I believe there is much to be learned from looking more closely at the collaborative processes in action and abandoning the biographical and individually based analysises that, at worst, will reinforce some of the capitalist trends it has set out to challenge.

The arts has a wonderful potential to portray and explore an alternative ethics and a powerful means of organization that should be fully investigated. Artistic research is one of the great arenas for that investigation.

\bibliography{/home/henrikfr/shared-home/Documents/svn/admin/conf/biblio/bibliography} \bibliographystyle{plainnat}
\end{document}

%%% Local Variables: 
%%% mode: latex
%%% TeX-master: t
%%% End: 
