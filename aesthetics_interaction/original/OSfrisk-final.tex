% Created 2019-07-30 Tis 22:20
% Intended LaTeX compiler: pdflatex
\documentclass[11pt]{article}
\usepackage[utf8]{inputenc}
\usepackage[T1]{fontenc}
\usepackage{graphicx}
\usepackage{grffile}
\usepackage{longtable}
\usepackage{wrapfig}
\usepackage{rotating}
\usepackage[normalem]{ulem}
\usepackage{amsmath}
\usepackage{textcomp}
\usepackage{amssymb}
\usepackage{capt-of}
\usepackage[hidelinks]{hyperref}
\usepackage[lf]{ebgaramond}
\usepackage{sectsty}
\allsectionsfont{\sf}
\usepackage[style=authoryear-ibid,natbib=true,backend=biber,hyperref=false]{biblatex}
\bibliography{./OS_bibliography.bib}
\renewcommand*{\nameyeardelim}{\space}%
\renewcommand{\postnotedelim}{: }%
\author{Henrik Frisk}
\date{\today}
\title{Aesthetics, interaction and machine improvisation}
\hypersetup{
 pdfauthor={Henrik Frisk},
 pdftitle={Aesthetics, interaction and machine improvisation},
 pdfkeywords={},
 pdfsubject={},
 pdfcreator={Emacs 26.1 (Org mode 9.1.9)},
 pdflang={English}}
\begin{document}

\maketitle
% \section*{Info}
% \label{sec:org8cbdad1}
% Henrik Frisk, Professor of music, Royal College of Music, Stockholm \\
% Väderkvarnsgatan 38C, 753 29 Uppsala, Sweden \\
% henrik.frisk@kmh.se

\begin{abstract}
    Departing from the artistic research project Goodbye Intuition
    (GI) hosted by the Norwegian Academy of Music in Oslo, this
    article discusses the aesthetics of improvising with machines.
    Playing with a system such as the one described in this article,
    with limited intelligence and no real cognitive skills, will
    obviously reveal the weaknesses of the system, but it will also
    convey part of the preconditions and aesthetic frameworks that the
    human improviser brings to the table. If we want the autonomous
    system to have the same kind of freedom we commonly value in human
    players’ improvisational practice, are we prepared to accept that
    it may develop in a direction that departs from our original
    aesthetical ambitions? The analyses is based on some of the
    documented interplay between the musicians in a group in workshops
    and laboratories. The question of what constitutes an ethical
    relationship in this kind of improvisation is briefly
    discussed. The aspect of embodiment emerges as a central obstacle
    in the development of musical improvisation with machines.
\end{abstract}


\section*{Introduction}
\label{sec:org792560d}
The artistic research project \emph{Goodbye Intuition} (GI), hosted by
the Norwegian Academy of Music in Oslo, is a platform to explore the
meaning of intuition, musical structure and aesthetics by means of
playing with a newly developed improvisation machine, nick-named Kim
Auto (KA) \citep{grydeland2018}. By introducing this fifth member GI
attempts to challenge the roles and aesthetical values of the four
musicians in the group and explore what kinds of music may emerge from
the experiments. The core method is artistic and the members engage in
improvisations with the new machine member followed by discussions and
reflections, sometimes carried out in the context of open
laboratories. The labs are open and the audience is invited to listen
and participate in the discussions.

Whether KA is an instrument, a performer, a composer or a composition
is difficult to define, but the meaning and significance of these
different modalities of musical knowledge and communication has some
impact on the relations that are possible within GI. That there is a
dynamic variability concerning the possible roles of the computer in
musical applications in interactive music is confirmed by composer
Cort Lippe who states that:
\begin{quote}
    a composer can assign a variety of roles to a computer in an
    interactive music environment. The computer can be given the role
    of instrument, performer, conductor, and/or composer. These roles
    can exist simultaneously and/or change continually, and it is not
    necessary to conceive of this continuum
    horizontally. \citep[2]{Lippe2002}
\end{quote}

The basic outline of KA is that it collects material from whoever is
playing with it and constructs an archive of material that it uses
when it performs. It has four different personalities that may be
configured and that defines how it responds to input. These are used
to shape its output based on musical concepts such as high/low pitch,
and structural aspects such as dense or sparse. By interpolating
between the different personalities the responsiveness of the system
and its output may be varied.

The research questions in GI revolve around the notion of the
improviser's identity in the music and whether the presence of a
creative machine would alter the way they listen or play. Also
included in the inquiry is the more general aesthetic question of what
kind of music may emerge from these experiments. The question of
whether KA is a \emph{good} co-player in a musical, or artistic sense
is primarily investigated from the point of view of the practice of
playing with it. Improvising with KA brings forward questions relating
to the point of view of both what is expected from it and what is
expected from oneself. GI is not a music technology project and does
not primarily claim to be innovative on the level of the development
of KA,\footnote{It is NOTAM in Oslo, a centre for the development and
    innovative use of technology in music and the arts, that does the
    development of KA rather than the members of the group.} nor is
this article a discussion of the underlying technology, except very
briefly.

As was mentioned above, the project is set up around a series of
internal workshops in which people external to the group are invited
to discuss the processes. These include British musician and writer
David Toop and myself, and on one occasion the workshop was led by
American director and writer Annie Dorsen. We have been part of the
discussions and to some extent we also contribute artistically. Worth
noting is the fact that the methodology of the project does not list
concerts as the main form for output. Instead the work is presented in
a laboratory format with the purpose to engage the audience in a
discussion concerning the general goal of the project.

The focus for this article is to discuss some of the preliminary
results from my study on the interaction in \emph{Goodbye Intuition}
and the machine improviser KA with the attempt to identify the aspects
of the interplay that triggered discussions concerning aesthetical
judgements and the sensation of interactivity. Analysing some of the
discussions in a few of the laboratories and workshops performed in
the project, the music that came out of it, as well as the responses
from the participating musicians, sheds some light on the way that
value judgements are developed, and sometimes questioned in the
group. There is a particular interpretative space that is opened up
through the experience of playing with the seemingly
responsive\footnote{KA is responsive in the sense that it listens to a
    limited number of parameters.} machine co-player that is examined
within the project and also in this article.

To improvise with a computer running an adaptive software for musical
interaction raises questions that have relevance to both the designer
of the improvising system and the musicians playing with it. In these
cases it is difficult to avoid some level of anthropomorphism among
the users \citep{Blackwell2004,Young2009}, but it is also clear that
this effect contributes to the willingness to engage with the system
\citep{Nowak2003}, and in some cases the ability to learn from it
\citep{Schneider2018}. In GI the very fact that the machine is given a
personal name points to the willingness to attribute some aspect of
human traits to it. While this contributes to the sensation that the
machine is intelligent in a human sense (which it is not), it also
puts a limitation to what is possible from a system point of view. If
it proves to be possible to create a machine system that can improvise
interactively and creatively, why restrain the machine to behave like
a human? If not, by what standard do we allow ourselves to limit its
freedom? Although speculative and abstract in nature, this inquiry
leads to a number of questions concerning, for example, machine ethics
that I hope to continue to pursue in the near future, but which may
also be seen as a backdrop for the present study.

\section*{Method and background}
\label{sec:org820a2ae}
Since I have been an active part in many of the workshops and
laboratories performed by the group Goodbye Intuition from the
initiation of the project in 2017, I am obviously involved in, and
influenced by, any findings and discussions. In the workshop in
Stockholm in February 2019 I also played with the group, but in
general my role has been that of an external partner, or critical
friend.

Using a qualitative method of purposeful sampling I have selected a
few important points during the workshop in Stockholm, 17-19 February,
that will form the foundation for the discussion in this article. Due
to the nature of the process in the project's workshops and
laboratories, in which a shared experience, and to some extent shared
knowledge\footnote{It is only to a certain degree shared as the
    various kinds of experience (musical, interactional, intellectual,
    experiential) is distributed according to the various roles that
    we have in the project. Only to the extent to which we have had
    the chance to explore the knowledge acquired is it shared.} has
developed, this method appeared to be the most appropriate. As
described by \citep[46]{Patton2002}, purposeful sampling, or judgment
sampling, "focuses on selecting information-rich cases whose study
will illuminate the questions under study". Using my first hand
experience it was relatively easy to choose the relevant parts to
include in this study. In principle parts that were considered
particularly unsuccessful and those that were perceived as successful
were chosen.

In general the workshop in Stockholm in 2019 was an important event in
the project for many reasons. Essential topics were aired and
discussed, the software running KA had been improved and its modes of
interaction had improved, and the laboratory format had
matured.\footnote{A documentation of the laboratory in Stockholm can
  be seen here:
  \url{https://www.researchcatalogue.net/view/411228/431482 (accessed
    1 September 2019).}} There was also a quartet performance with KA
that came out both artistically and aesthetically convincing. With the
purpose to engage in a discussion on the impact of technology in the
form of artificial musicians on questions concerning musical ethics
and improvisation aesthetics it seemed natural to pick out a few of
the many information-rich contributions from this particular event.

That the field of music is also affected by the huge interest in
artificial intelligence is not surprising. The practice of
improvisation in music is sometimes seen as a model form for
organisation, or, as put by \citet[59]{Cook2017}, "[t]here is a long
standing tradition of seeing jazz, particularly free and avantgarde
jazz, as the expression of an ideal society". Given such an
assumption, musical improvisation may appear to be the perfect case
for evaluating the functionality of humanoid. Would it prove to be
possible to write a software that can improvise with musicians in a
manner that is indistinguishable from that of a human player, not only
is this a significant engineering task, but it may also be assumed
that this software can function in other social contexts. Music as a
test bed for intelligent technology. There are a number of criteria
that would need to be met and the computer system's ability to respond
to input is a challenge in deciding both what constitutes valid input
and what is a useful response.

One of the ways in which it has been theorised that a systems ability
to think, or its responsiveness to human interaction, may be measured
is the Turing test \citep{Turing1950}. There has been many ideas about
various designs of a Turing test for music, also those that evaluate
systems that improvise. In 1988 the topic was discussed in the
\emph{Computer Music Journal} \citep[7-9]{spiegel1988}. The piano
competition Rencon, a "forum for presenting and discussing the latest
research in automatic performance rendering" \citep[120]{Hiraga2004}
introduced a Turing test for evaluating methods for performance
expression on the Disklavier in 2004. \citeauthor{Roda2015} likewise
using the Disklavier, perform a Turing test in the context of a live
performance. A software developed by \citet{Pachet2003} and his team,
The Continuator \citep{Pachet2003,Pachet2003b}, is an interactive
system and a musical imitator that has been tested with a Turing test
design \citep{pachet2012}. However, as was noted by Laurie Spiegel in
a comment to the request for a musical Turing test in the
\emph{Computer Music Journal}, one may question "[w]hat purpose would
be satisfied by creating qualitative or quantitative metrics for
musical intelligence, given the lack of successful similar criteria
for natural musical intelligence, musicality, or even music per se"
\citep[9]{spiegel1988}. She thereby puts the focus on the real issue
concerning even a basic notion of artificial musical intelligence.

The question of the nature of an instrument or system such as KA may
appear to be a matter of primarily theoretical impact. As will be
demonstrated in the next section, however, it is nevertheless of some
importance. There has been no shortage of attempts to aim at a working
metaphor for the emerging field of intelligent instruments. As noted
by \citeauthor[5]{Bowers2005} "[s]ince 2001 the NIME series of
conferences has seen the presentation of a wealth of interface and
instrument design ideas" \Citep[p. 5]{Bowers2005}. Before that the
notion of the hyperinstrument was introduced by Tod Machover
\citep{Macover1989}.  In the previous section Cort Lippe pointed to
how the interactive system can take on any number of roles in a
musical context, and in her PhD thesis \citeauthor[17]{Fiebrink2011}
suggests that "when the computer takes the role of an instrument
within an interactive computer music context, the process of designing
how a performer will use the computer to play sound can be understood
as both composition and instrument building". Fiebrink refers to the
work by \citeauthor{Schnell2002} who used the term "composed
instruments" \citep{Schnell2002} to define the practice of designing
computer systems for music. While we have often continued to label
computer systems designed for music as \emph{instruments}, in some
cases these systems are as much a part of the score as the score is
(if one at all exists). This has influenced the way in which new
interfaces, and composed instruments, impact on the practice of
musicians and composers alike, as well as on the ontology of the
musical work \citep{frisk-ost06-2,frisk-ost06}. With a terminology in
part borrowed from game design composer and musician Per-Anders
\citeauthor{Nilsson2011} discusses software instrument building as
part of either "design time" or "play time": "Design time is outside
time activity, concerned with conception, representation, and
articulation of ideas and knowledge, whereas play time deals with
embodied knowledge, bodily activity, and interaction in real-time"
\citep[2]{Nilsson2011}.

As was mentioned in the introduction, in GI there is a strong aspect
of anthropomorphism that becomes obvious in a number of situations. KA
is often talked about as "trying to do" something, being in a
particular state of mind, or consciously going in a particular
direction. On the one hand this is obviously false, KA does not have a
consciousness in the way implied by this way of talking. It mainly
listens to a few parameters of its input and records sound (a more
detailed description is found below). On the other hand it says
something about its output that leads us to anthropomorphise. This can
be seen as a consequence of the close relationship between traditional
musical practice and interactive music practices, as is pointed to by
\citeauthor{Schnell2002}:

\begin{quote}
    Interpreting the attitude of the performer of a composed
    instrument with the help of categories from the traditional way
    music is created leads to various metaphors such as that of
    playing a musical instrument, conducting an orchestra, playing
    together (ensemble) with a machine, acting as a one-man
    band. \citep[140]{Schnell2002}
\end{quote}

The use of technology in art practices can have several different
objectives. From a general standpoint one may argue that art should
engage in available technologies for the simple reason that this
contributes to our understanding of its social and cultural
impact. Though this general notion is sometimes contested, most
famously by Heidegger who instead saw the opposite, that the
technology frames the human capacity
\citep{heidegger93},\footnote{Aden Evens has written an interesting
    commentary on it in the digital age \citep{evens05}.} there are
good reasons to evaluate uses of technology through artistic
practices. From the point of view of innovation it has been seen in
the past that artists' use of technology push the boundaries for what
is possible \citep[e.g. ][]{harris1999}. Although this has arguably
been true, the resources that the multinational technology and media
industry now are in control of makes it increasingly difficult for an
independent artist, or even a university, to produce artefacts that
may compete with the R\&D budgets of these companies, although the
artistic qualities in and of themselves may be uncontested.

\section*{Analysis and results}
\label{sec:org5fb345b}
\subsection*{Composition or improvisation}
\label{sec:orgc6d4b7f}
One of the topics raised early on in the workshop in Stockholm was the
notion of KA as a composition rather than a co-improviser, or an
instrument. This may seem odd at first, after all, GI is an
improvisation group, what purpose serves the notion of a composition
in this context? Despite the conceptual contradiction, this is related
to the notion of composed instruments, as was discussed above: some of
the performance instructions, the score, is encoded in the system. Or,
more plainly: "the \emph{work} is replaced by the \emph{interface}"
\citep[28]{frisk08phd}. As a composition it makes sense that it allows
a certain type of music, or response, but not other kinds.

It should be noted that constructing a system that plays back a
preconceived composition is obviously a more manageable task than the
attempt to create a system that is able to interact to unforeseen
musical events. This insight may have had something to do with why the
topic of KA as a compositional frame came up. However, there is a more
practical and pragmatic aspect to this as well. The goal of the
project is to challenge the roles and aesthetic values of the four
musicians in the group, and reflect on the topics of \emph{listening}
and \emph{playing}. But if the responses from the system are not
responsive enough, so to speak, if the co-player KA is not
\emph{listening} attentively enough, then it becomes difficult to make
meaningful commentaries about either the system, or of one's own
strategies interacting with the system (even though, as will be shown
in the next section, not listening may actually be a perfectly valid
strategy). If the perfect playing companion is not available within
KA, perhaps it is possible to instead look at it as a compositional
frame that affords certain musical behaviours?

To some extent KA is already a compositional frame as can be judged
from its design. A given archive with a specific setting renders a
music that is recognisable from performance to performance. Assuming
that the musician and the system share some notion of what is "good"
music and what is "less good" music it would furthermore be reasonable
to expect that it makes a judgement on what kind of material it should
archive and what material to avoid. In the version of the software
that we played with in the laboratory on 19 February, 2019, it was
however obvious that it did not do this when I played with it. Another
interpretation is that KA has a set of aesthetic values that are
different from mine. I was able to improvise with it and feel
relatively good about the interaction, but the way it collects
material allows it to pick up on a phrase that I considered a mistake,
or a badly shaped motive. Such material then quickly defines the
character of the music and becomes an obstacle that is difficult to
come by.\footnote{A discussion on this may be heard at 20'00''
    \citep{grydeland2019-1}}

One of the members of the group comments on the same fact, saying that
the system works quite well as long as one goes along with it. If you
resist it, however, and oppose its playing or the material it uses, it
is very slow to response.\footnote{This is discussed in
    \citet{grydeland2019-1} at 15'00''.} In effect, what is emerging
is a kind of composition, or a compositional frame. As long as one
stays within the boundaries of the composition and play correctly, the
musical result can be convincing. Perhaps this is more of a conceptual
change, a difference in attitude towards the task of playing with a
machine improviser, rather than a practical, musical change. The
performance following the discussion of KA as a compositional
framework rather than an improviser was in some ways quite
convincing. One of the performers remarked that it felt like the first
time they were able to construct musical structures \emph{with} the
machine without it being a struggle.
\subsection*{Not to listen}
\label{sec:org36873f8}
One matter that has come up on multiple occasions is the idea of
musical rudeness. Both in the sense that KA should be rude but also
that the improviser may be rude to KA. There is no doubt that this
freedom to shut another musician off, play louder than everyone else,
or, in the words of improviser Sten Sandell, to be allowed to change
direction at any time \citep[p. 33]{Sandell2013}, has been an
important aspect of some strands of free improvised music and
jazz. What triggered this discussion was a reflection written by
Morten Qvenild:

\begin{quote}
    My playing here is very rude, I am being empathic towards KimAuto
    here, thats for sure. I am playing with a machine, and this
    rudeness is possible without hurting someone. I think this
    assertiveness is a good musical option sometimes. The
    non-listening and the not taking care of the other. \citep[par
    12]{qvenild2019}
\end{quote}

To be rude, or choose to not listen to the other, may appear contrary
to the notion of music as an activity that creates and maintains
social networks \citep{monson98}, but may here be seen as a
possibility to engage in a musical ethics. While it is true that
listening and adopting are essential in improvisation as well as other
practices,\footnote{For an exploration of the philosophical impact of
    listening, see \citet{nancy2007}.} it is impossible to say what is
generally right in improvisation. Marcel Cobussen writes that:

\begin{quote}
    Reacting to the unfolding of the music, the musician and his
    instrument enter into a relation with already produced sounds,
    concretised musical ideas, present frames. All these cases contain
    an act of thinking during the act of doing. To listen to oneself
    and the other(s), to listen to the proceeding and developments of
    the music, to listen to the noises that direct the music to
    unknown areas.\footnote{For a discussion on the Self in
        improvisation, see \citet{frisk12-improv}}
    \citep[p. 33]{Cobussen2005}
\end{quote}

Interestingly enough, this could work as a characterisation of KA
whose archives may actually go beyond sounds recorded and develop into
a storage of \emph{thinking} through the act of doing. But where does
the rudeness come in? Well, in another text by Cobussen Keith Rowe's
concept of non-listening is discussed in a way that appears to be
approaching what Qvenild calls rude. Cobussen asks "how the concept of
non-listening, as suggested by improvising guitar player Keith Rowe,
can undermine, or, conversely, deepen an aural ethics"
\citep[p. 11]{cobussen2016}. And further:

\begin{quote} [N]on-listening is meant to prevent any form of
    interaction. It is meant to avoid relapsing into a musical
    performance which is built on previous explorations and
    discoveries; it is meant to avoid too many conventions, too many
    tricks that have already proven their success, it is meant to stay
    open to another otherness. Rowe opens an ethical space of
    creativity and change through resistance. His attitudes makes
    space for musical interactions that demand a response-ability that
    is not already prescribed, a praxis of risk for which there can be
    no rules, no codes, no principles and no
    guarantees. \citep[p. 87]{cobussen2016}
\end{quote}

The act of not listening is a tool for the improviser, and while a
similar activity could be problematic in social interaction, in
improvisation it instead appears to contribute to the
openness. Non-listening introduces resistance in the interplay whereby
the ethical capacity is increased rather than hindered. On member of
the group commented on how in improvisation there is sometimes a
regulatory force in the form of convention or aesthetics, and that
this force may block the potential for development in the
group. Another way to put it is to say that the capacity for being
rude and break with the convention is somehow interrupted. Playing
with a machine is deliberating according to the same testimony since
the machine does not care, so to speak. Although it is not always
obvious what may constitute a 'correct' intervention, this example
points to the importance of allowing rudeness in improvisation. In
some cases it may simply be the most ethical and respectful way
forward.

In music it is possible, and in principle correct, that the
aesthetical domain frames the ethical, but in reality it is clearly
more complex. For now, based on the theory presented and the
experience and discussions in GI, I believe it is possible to state
that what constitutes ethical behaviour in musical improvisation and
artistic practice extends what is generally seen as acceptable, or
good, in social interaction, and that there is an interrelation
between ethics and aesthetics.

\section*{Discussion}
\label{sec:org4fc2881}
Apart from it being an interesting frame for artistic work, my
interest in using digital technology in my artistic practice also
departs from the belief that art practices in general offer a context
in which experimentation and play is possible. That is, the aesthetics
of the practice offers a set of value judgements that an engineering
context may not provide, nor a traditional artistic practice. This,
however, is not a property of the field of digital art but rather a
function of it. In other words, only if the artist is consciously
working for it does this experimental opportunity present
itself. Furthermore, the framework of artistic practice does not by
itself guarantee sound values and reasonable ethics, it is merely a
\emph{possibility}, a potential, albeit an important one. Somewhat
beside the point, but worth mentioning is that this does not mean that
the art, nor the artist, is always ethically just, or morally
defensible. The emphasis here is on the practice and the way the
practice organises itself when the value system is primarily
aesthetical.

This, however, introduces a number of difficulties, the most important
for the current discussion is the impact that this may have on the
development of the technological systems. Although writing code may be
considered an art form in itself in practices such as live coding, the
role that the technology plays in works that are called digital art is
not always as obvious. An attempt at a definition is made by
\citeauthor{lopes2009} writing that "a item is a work of digital art
just in case (1) it's art (2) made by computer or (3) made for display
by computer (4) in a common digital code" \citep[p. 3]{lopes2009}. Sol
LeWitt's famous essay on conceptual art predates the digital
revolution, written long before digital technology became readily
available in the way it is now, states that: "The idea becomes a
machine that makes the art. This kind of art is not theoretical or
illustrative of theories; it is intuitive, it is involved with all
types of mental processes and it is purposeless" \citep{Lewitt1967}.

The question that remains unanswered is what the actual code of the
machine improviser constitutes a group such as GI. It was discussed
above that the ethics of improvisation is negotiated through the
aesthetics of the context, but what is it actually that guides the
aesthetics of KA? Judging from the discussions we have had in the
workshops and laboratories, it is clear there has been, and still is,
a serious attitude towards KA with a respect for its capabilities and
an understanding for its flaws. It is accepted and allowed to not
listen, or to enter into a musical context and change the course of
action in what may seem to be a disrespectful manner, not through
negotiation but by introducing change strong enough that it makes a
difference. In other words, though there was a discussion about the
way KA introduced music that not all members appreciated, it allowed
for a very different kind of musical relationship. The group accepts
KA as a fifth member and as a co-musician and in that sense it is a
\emph{machine that makes the art}, but what is the identity of this
machine, and who is in control of it? Is it the code, the hardware
running it, the sounds it produces, or is it maybe just an extension
of the programmer that created it?

One may imagine that in the near future there will be tools, like KA,
but more advanced, that exhibits some notion of what we can call
musical intelligence. Are we in that situation prepared to allow the
machine to develop its own aesthetics, as we would with a human
co-player? It is not difficult to imagine that such a machine can
develop extremely fast, much faster than a human player. In this case,
perhaps the machine eventually grows to be uninterested, not only in
the music that humans play, but even loses interest in playing with
humans at all. If the machine is still programmable by us we could of
course put some limitations on the way its knowledge about music
develops, thus avoiding this scenario. This kind of limitation could
also be part of the design of the system from the beginning, which may
appear to be a sensible thing to do, but it would nevertheless be to
narrow its freedom in a way that is not often done for human
musicians. If these systems in fact are intelligent in some manner,
what are the consequences if we introduce constraints on the machines
potential for development? Introducing a structural divide between one
class of performers (humans) and another (machines) may also have an
effect on the way our musical and aesthetical values develop. There
are (at least) two aspects of this reasoning.

First, the machine's capacity to learn and develop is obviously not
only dependent on its input. Until machines are capable of creating
machines without the interference of humans, they depend on designers
and programmers that create the systems. These also have values and,
as has been shown through studies such as by \citet{snow2018}, also
self-organising neural nets inherit the biases of the programmer that
created the system.\footnote{Twenty-eight members of the US congress
    where mistakingly matched against pictures of convicted criminals
    and a disproportionate number of these were people of colour.}
Following this, even if the software is training itself it does not
appear to be able to reach outside its own context, and perhaps it
should not try to.

Secondly, musical aesthetics is commonly developed in a wide field of
practices, not just music. As a musician my freedom in performance and
my aesthetics are shaped by a great number of
impressions. Consequently, if the goal is to allow the machine to
develop its own aesthetics its input should not be limited to sounds
in performance, but also include all the other contexts in which music
is negotiated. This point, however, reveals the most obvious
difference between KA and human musicians: KA has no real
physicality. The lack of presence is a great disadvantage for KA, as
is pointed to by one of the members of GI:
\begin{quote}
If you play with humans you have an idea of their aesthetics so that
part is kind of integrated, on beforehand. So maybe there is always
some kind of quick preparation, or plan in order to feed into the
total? \footnote{Lab \#5, part 2, 19 February, 2019.}
\end{quote}
The question is not only concerned with aesthetics, but also has a
bearing on the ethics of improvisation discussed
above. \citeauthor{benson03} makes the claim that the musical dialogue
can be said to be ethical in nature “since music making is something
we inevitably do with others (whether they are present or not)"
\citep{benson03}. To play with someone is to encounter the other, but
to play with a machine is not as simply analysed. What is KA in this
encounter? One of the members comment on this topic and asked: "How
can I, ethically, relate to and play with KA with the same kind of
trust and respect as when I play with humans?"
\citep[p. 4]{endresen2018} Here, the focus is not on the failure of KA
to be a good musician, but rather on the self.

The larger question that emerges from this is: Is it at all possible
to play, that is to fully play, without the feeling of a reciprocal
trust that, one may suspect, goes beyond musical experience? If the
co-musicians lacks a body and has no extension in the physical realm
and there is no sensation of the other, the situation becomes
radically different. The co-musician is a blank slate, not even
non-existent, but with negative extension. The machine is a void until
it starts playing and even then it is relatively difficult to
anticipate the output. But even if the difficult question of empathy
in music is put aside and disconnected from the question of physical
presence, the impact of embodiment in music cannot be disregarded
\citep{godoy2006,Leman2015},and embodiment has had a big impact in the
field of cognitive science. A slightly different question emerges:
What are the strategies that may be employed that substitute for the
lack of presence and lack of trust when playing with a machine? The
simple answer is that it is possible. There is a great deal of music
where one or more of the players is a machine, and GI's work with KA
proves that it is possible, but the challenge is changing as the
nature of the machine is changing.

Concerning the general situation of computer interaction, it appears
that we are quick to adopt. There are a number of tasks and activities
that belong to the social realm that we now carry out virtually such
as social media. Although the discussion concerning the precariousness
of committing your social life to social media pops up every so often
\citep[see, for example, ][]{lanier2018,lanier96} we appear to be able
to quite quickly adjust and be willing to substitute a physical
meeting with a chat on a media channel. But because it is possible,
and maybe even desirable in one context does not mean that it is so in
another. To play with someone is to encounter this other and the
ethics of this situation is greatly affected by the extent to which a
physical meeting takes place. Regardless if this is a person on knows
or not, a wide range of information is immediately gathered or
created, merely through the very first encounter. The body and
posture, the way the instrument is held, the facial expression and
many other things contribute to what one may expect this musician to
play. These impressions form the groundwork, or underpinning for what
will be played.

If music is to be regarded as a social activity that participates in
expanding the communicative possibilities both within the group of
musicians playing together and to some extent also among listeners,
then the very idea of playing with a machine does appear strange. The
communicative potential in an interactive music system such as KA is
relatively limited and different in nature from that of human--human
communication. In the case with GI, however, the output of the machine
is, after all, relatively well structured and possible to
anticipate. GI has proved, as many other projects have done before,
that it is possible to make meaningful music with a machine. But most
noteworthy are the challenges the practice has shed light on. There is
a need to better understand the possible modes of musical interaction
with the kind of machine that KA is. But comparing it to a real
musician makes little sense, because music in general, and
improvisation in particular, commonly relies on a myriad of other
parameters that are simply not available to KA. The most fascinating
results of this, however, are 1) the extent to which musicians adopt
and counteract the obvious shortcomings of the machine improviser and
the reflections this leads to, and 2) the impact a conceptual change
may have, such as thinking about KA as a composition rather than an
improviser. Playing with a machine makes the conceptual asymmetry
between the embodied musician and their instruments on the one hand,
and the abstract and disembodied computer on the other, come to the
fore. What we should do to overcome this difference, and why, will be
an important questions for the future.

\section*{Acknowledgments}
\label{sec:org9374948}
I wish to thank all the members of Goodbye Intuition: Ivar Grydeland,
Morten Qvenild, Sidsel Endresen and Andrea Neumann as well as David
Toop and Annie Dorsen, all of whom have contributed to the thoughts
presented in this article. Finally I wish to thank Stefan Östersjö for
reading the text and making valuable comments.

% \section*{Bibliography} 
\label{sec:orgce11cbb}
\printbibliography
\end{document}