\documentclass[12pt]{article}
\usepackage[english]{babel}

%% Disable when not using html output
%\usepackage{tex4ht}
%\usepackage{pxfonts}
%%%%%%%%%%%%%%%%%%%%%%%%%%%%%%%%%%%%

\usepackage[T1]{fontenc}
\usepackage{url}
\usepackage[utf8]{inputenc}
\usepackage{enumitem}
\usepackage{csquotes}
\usepackage{fixltx2e}
\renewcommand{\encodingdefault}{T1}

%% Enable for graphics 
%\usepackage[pdftex]{graphicx}
%%%%%%%%%%%%%%%%%%%%%%%%%%%%%%%%%%%%

\usepackage{setspace}
\usepackage[style=authoryear,natbib=true,backend=biber]{biblatex}
\bibliography{./../biblio/bibliography}

%% Enable when using PDF output
% \renewcommand{\rmdefault}{pad}
% \renewcommand{\sfdefault}{pfr}
%%%%%%%%%%%%%%%%%%%%%%%%%%%%%%%%%%%

\begin{document}
\title{A Marxist view on the theory/practice divide}
\author{Henrik Frisk}
\date{}

\maketitle

%\section*{Introduction bör ändras}

%% Quote the objekt och metod
\onehalfspacing
\noindent
I am first and foremost a practitioner, a musician. It is only along with my practice that I can make meaningful statements about it in my research. My credibility as an artistic researcher lies within my artistic production. Should I move too far away and loose touch with it, there is a great risk that what I have to contribute in terms of knowledge is diluted. This is true especially if what I come up with is theoretical knowledge. Practice and theory may be seen as two points on a continuum, and if either side of this range is too heavily emphasized, the nature of the information is altered. Not because the information is theoretical, or practical, but because information gathered through theoretical means is different from information collected with a strong relation to practice. As a practice based researcher I can contribute valid theory, and so can obviously a theoretically oriented researcher, but my point here is that the \emph{relation} between theory and practice matters. The general continuum between practical and abstract knowledge is an important factor in how we view knowledge at large.

Presenting the theory/practice divide as a continuum, however, is still a dual model that suggests that one can be chosen over the other. Instead the relation can be described as a mutual interdependence where theory is part of practice and vice versa: one can be said to exist \emph{along side} of the other. This is an important aspect of artistic research: it claims that the practice of creating the thing to be studied is also part of the attempt to understand it, thus opening up for the potential to bring together the two epistemological categories of \emph{knowledge how} with \emph{knowledge that}. 

As a consequence of the still unsettled relationship between theory and practice, artistic research has a tendency to lean against social sciences and philosophy instead of building its own theoretical frameworks. In the attempt to find theories that are applicable to the practice, and theoretical frames that allow for reflection upon it, these disciplines appear to be relevant. If this loaning of competence creates the kind of theoretical imbalance that was mentioned above the important equilibrium between theory and practice may be lost and the two kinds of knowledge may drift apart. The theory becomes a general theory with little, or even no relevance for the practice. This can still give rise to interesting work and engaging insights, and these insights may even have a greater impact compared to practical and specific knowledge thanks to the abstracted generality of the former. But the knowledge produced is different in kind to the knowledge embedded in the practice. As an artistic researcher I am an advocate for deconstructing the binary view on theory and practice, and artistic research is a field in which the discussion on these matters needs to be had. Both in order to avoid the slide from practice to theory, and also to allow a more general and better informed discussion on the relation. 

If there is a risk that practice based research will become theoretical, many other academic fields are theoretical in principle, although there are noteworthy exceptions and many variations of the theory/practice relation exists. Medicine is in some regards a field of research that has a strong relation to its practice, and there are subjects in natural science where the theory/practice division is not even meaningful to discuss in the same sense. In mathematics and theoretical physics, for instance, there is hardly any materiality in the thing studied, which is also the case in certain strands of theoretical philosophy and other disciplines in social sciences.\footnote{A philosophical subject such as Aesthetics, however, is an example of a discipline that is entirely separated from the world of the aesthetic objects studied.}

Very few forms of knowledge, however, are only theoretical, lacking all kinds of relation to practice. Likewise, it is very difficult to imagine a practice that is entirely void of theory. In artistic work and research it is the structures and relationships between theory and practice that is of interest to me. What is the purpose of dividing up the process of artistic creation in two distinct phases? That is a big epistemological question, the answer of which would require a much lengthier investigation than what I may offer here. Instead, I will make an attempt to contextualize this question politically. The premise that philosophy, and much of social sciences, are indeed concerned primarily with their academic activity whereas practice, be it artistic practice or some other embodied practice, is primarily concerned with material activity is interesting when a socioeconomic and historical perspective is applied.

---

It may be argued that the development of theoretical and academic knowledge in the Western world was only possible through division of labour. For example, Plato was able to lay the foundation of Western philosophy and science because he did not have to think about material issues. Put differently, thanks to the Greek slaves Plato and his peers could focus their thoughts on metaphysics and epistemology rather than food and housing.\footnote{As intriguing as the question may be, whether or not it is possible to be, for instance, a factory worker and a philosopher at the same time is not something I will discuss here.} Friedrich Engels brings this topic up in his book \emph{Anti-Dühring: Herr Eugen Dühring's Revolution in Science}, also reminding us of the impact the social structure of ancient Greece has had on Western civilization in general:
\begin{quote}
  It was slavery that first made possible the division of labour
  between agriculture and industry on a larger scale, and thereby also
  Hellenism, the flowering of the ancient world. Without slavery, no
  Greek state, no Greek art and science, without slavery, no Roman
  Empire. But without the basis laid by Hellenism and the Roman
  Empire, also no modern Europe. \footfullcite[p. 110]{engels47}
\end{quote}
Later in the same text Engels contextualizes the same relation in his own time and  points to how division of labour has given rise to the academic classes. Thanks to division of labour, thanks to dividing practically and materially engaged knowledge from the primarily theoretically framed knowledge of the academy, unprecedented advances in theory has been accomplished, and a special class developed, freed from the hassles that the working class was occupied with:
\begin{quote}
  So long as the really working population were so much occupied with
  their necessary labour that they had no time left for looking after
  the common affairs of society---the direction of labour, affairs of
  state, legal matters, art, science, etc.---so long was it necessary
  that there should constantly exist a special class, freed from
  actual labour, to manage these affairs \footnote{Ibid, p. 111}
\end{quote}

This could be seen in relation to Hegel's famous passage on the master-slave dialectic where the obvious relation between them eventually becomes reversed and the master ends up subordinate to the slave, \footfullcite[p. 166]{hegel08} and essentially what Marx inherited from Hegel and applied to the bourgeoisie and proletariat. The eventual reversal of the power relation is in principle the catalyst for the communist revolution according to Marxist theory. Not all processes of division of labour, however, are the outcome of disproportionate power relations, nor do they always result in destabilization. But I believe it is fair to say that division of labour has affected the way we think about also art and education.

There is no doubt that division of labour has had an impact on the ways in which Western art music has developed. The musical virtuoso delivering outstanding performances with technical brilliance was only made possible through division of labour that allowed soloists to focus wholly on their technical and artistic brilliance. This same division allowed modernist masterminds such as the German composer Karl-Heinz Stockhausen, whose theoretical and practical inventiveness created new standards for musical composition, but who was not a performer, unlike his 19th century colleagues. Perhaps one may go so far as to claim that the way the symphony orchestra was expanded during the 19th century was a result of increased specialization.\footnote{There were clearly other reasons for this too, the obvious one being the composer's demand for a bigger and more powerful sound.}

The globalized world economy is a meritocracy where ever increasing specialization is the rule, the field of the arts - at least as long as there is money involved – being no exception. In the experimental arts with only limited funding, however, the tendency is the opposite: more and more tasks are assigned to fewer people. Artistic director, performer, accountant and producer are roles often performed by one and the same person. This is logical. Division of labour is a process with which efficiency is achieved with the goal to economize. If there is no economic return in the endeavor, there is no point in reducing costs. Division of labour is a strategy that only makes sense in a capitalist system.
 
Universities and art schools are likewise no exceptions. Administrative tasks are divided up and assigned to more and more specialized employees and roles such as head of department and dean, for which teachers used to be recruited, are now appointed to external professionals with strong leadership skills. Information about the activities of the University is now handled through professional public relations offices rather than by the teachers themselves. 

It should be stressed, though, that the model of dividing up practical knowledge from theoretical reflection has given rise to a plethora of new knowledge in a great number of scientific subjects. In \emph{The wealth of nations} Adam Smith famously claims already in the first book that ``the greatest improvement in the productive powers of labour, and the greater part of the skill, dexterity, and judgment with which it is anywhere directed, or applied, seem to have been the effects of the division of labour.''\footfullcite[Ch. 1]{smith2000} Division of labour, he claims, appears to be a superior way to make any production more efficient.
According to Robert Alun Jones both Smith and Emile Durkheim gives some support to the idea that it is not only the productive powers that are affected by the positive forces of division of labour. Also academia and the arts are influenced by these concepts: ``And like Smith, Durkheim recognized that this extended beyond the economic world, embracing not only political, administrative, and judicial activities, but aesthetic and scientific activities as well. Even philosophy had been broken into a multitude of special disciplines, each of which had its own object, method, and ideas.''\footfullcite[p. 24]{jones1986} 

By describing the developments of academia and some art practices as results of division of labour using theories of economy and politics, is a way to understand the dynamics of what is sometimes described as production of knowledge. My purpose here is not necessarily to critique these structures, but to unwrap in order to understand them. My main ambition, as was described in the beginning, is to deconstruct the binary relation between theory and practice, and in order to achieve this it is important to understand why the dichotomy has come about to begin with. To develop a better understanding for the conditions of knowledge production in the study of music and other artistic practices, it is important to also understand the processes and structures involved in the shaping of the study. Ultimately, we may be able to contribute to changes that will improve our capacity to learn by broadening our area of focus rather than concentrate it.

In an early text, \emph{The German Ideology} Karl Marx and Friedrich Engels, whose ideas concerning the division of labour were inherited from Smith, accuses the group of philosophers that continued Hegel's legacy -- The Young Hegelians -- of not paying attention to the material dimension of what they are researching. Privileged and protected from the real world they have lost touch with the material world, Marx and Engels claim. They are trapped in a metaphysical investigation and overly concerned with a top down view, disregarding the social and political perspectives: ``It has not occurred to any one of these philosophers to inquire into the connection of German philosophy with German reality, the relation of their criticism to their own material surroundings.''\footfullcite[p. 41]{marx1970} 
Marx claims instead that the social structures and the contexts for existence should be ``continually evolving out of the life-process of definite individuals,'' and not as they are imagined by someone else, ``but as they really are; i.e. as they operate, produce materially.'' It is not the imagination of a system or an ideology that should construct knowledge, but the conditions of life of individuals, their ``definite material limits, presuppositions and conditions independent of their will.''\footfullcite[p. 46-7]{marx1970}

According to Marx knowledge about the conditions of life should be built from the bottom up:

\begin{quote}
  The production of ideas, of conceptions, of consciousness, is at
  first directly interwoven with the material activity and the
  material intercourse of men, the language of real life. Conceiving,
  thinking, the mental intercourse of men, appear at this stage as the
  direct efflux of their material behaviour.\footfullcite[p. 47]{marx1970}
\end{quote}

Spelled out differently, and adopted to the current discussion, what Marx is arguing for here is that knowledge is at first conceived in, and through, the practice. But he points also to the production of conceptions and of consciousness, which are necessary components for a reflection upon the material activity as. This is consistent with my experience that it is possible to conceive of theory as something that rises out of the practice: consciousness and concepts that come out of the practice are the beginnings of a theoretical abstraction.

One of the things that I find intriguing with artistic research is the way it allows for this combination of practical experience and expertise on the one hand, with theoretical precision and artistic rigor on the other. It is the practice that gives rise to the research. Thereby the deconstruction of the theory/practice divide has already begun. Theory without practice is impossible and the practice is bound to the material reality of the researcher. The artistic research process is a unified system of ideas whose structure is organized by the artistic practice first and foremost. I claim that it is possible to describe theory as something that occurs along with practice, rather than as a semi-disengaged element of the research.\footnote{I believe this goes for method development as well.} Through the practice theory may be constructed that can eventually be used to abstract away from the practice and make generalizations usable in other contexts. This method is radically different from what was described in the beginning of this paper, where artistic research instead lends the theoretical and academic competence from the outside, from other disciplines.

A strong connection between theory and practice such as the one proposed here is obviously not exclusive to artistic practice. I think Cecilia Hultberg's \emph{The printed score as a mediator of musical meaning} (2007) is an example of the impact a reference to practical and material experience may have. In the first sentence of the text Hultberg writes ``During my time as a music teacher in Germany and Sweden many students have asked me questions like `\ldots can you please show us how to play this' ''\footfullcite{hultberg2007} In the very beginning of the text she immediately contextualizes her study through her own practice as music teacher. This influences her study and influences the way I read it. It is a simple but significant example of how practice, reflection and theory can be brought together in one and the same articulation. It is possible to be an excellent teacher, musician and researcher at the same time. If these different capacities are used in parallel, referred to and explored in practice, there is a risk that the result may become too broad, lacking in specialization. This is far from the case with Hultberg's research. Her career shows us that increased specialization and academic precision is possible from a broadening of ones practices.

Division of labour is an efficient method for allowing professionals to focus on isolated aspects of a phenomenon, but the artistic and pedagogic practices in music are built from highly complex processes that involves a multitude of fields in interaction. In these cases theory may not give proper answers to the questions asked unless the material conditions for the practice are brought into the loop. We may make a comparison to the quite problematic but still informative anthropological terminology of \emph{emic} and \emph{etic}. These are used to denote the different kinds of reflection that goes on in a social or cultural context. The emic is the experience or perspective of the subject and the etic that of the scientist/researcher. When I argue for a practice oriented approach in research I am not merely pointing to the importance of observing the practice in an \emph{etic} approach. For the practice to play a part in the research it has to be part of the researchers experience, he or she should be able to produce both \emph{emic} and \emph{etic} accounts of it at the same time.

In the case of Hultberg's study mentioned above, what difference would it have made if the student's question had been collected from a book, or from the experience of another music teacher whose voice was not present in the text? What if Hultberg would not have had an expansive experience as teacher? On the surface, it would probably not have mattered much. After all, knowledge is most commonly communicated through an expert's account or observation of a practice. But in the unwrapping of the reasons behind the questions asked by the students in Hultberg's account, and in the design of the study, her first hand experience plays a potentially very important part. In the practice there is knowledge that is not accessible from an external, observing perspective. This knowledge is hidden in the doing, not only in the body, although the body plays an important role. If division of labour results in practitioners and those who think about practice we loose part of that knowledge and experience. I propose we start to think about practice and theory in non-hierarchical relations where one is not contained in the other, but where they may each give rise to one another. As a result we may conclude that division of labour does not always create higher quality and more efficiency. That is one of the many things that Cecilia Hultberg's work has taught me.

\printbibliography
\end{document}
