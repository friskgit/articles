\section{Discussion and reflection} 
As already explained, the main issue for \emph{etherSound} is to allow for unconditioned participation. I was more concerned with the collection of diverse input than I was with giving the contributor a sense of control or participation, and in the first versions of \emph{etherSound} the message-compositions were much less dependent on input than they are in the current version. My grounds for changing this, and in the current version letting the messages generate a musical event with a clear form, stem from the wish to retain a perceptible connection - even though this connection may only be dismantled by the \emph{change} in output - between input and output. 
The process of designing the analysis and synthesis programs described above is to a considerable extent tantamount with the process of composition in the traditional sense. In a sense, \emph{etherSound} is an algorithmic or ruled-based composition\footnote{French composer Michel Philippot and Italian composer Pietro Grossi were pioneers of algorithmic composition.} with stochastic elements, methods which have been explored by many composers for many years. In his book \textit{Formalized Music} composer Iannis \citeauthor{xenakis71} offers a thorough investigation of the concept of stochastic music which came about as a reaction to post-serialistic music:

\begin{squote}
 For if, thanks to complexity, the strict, deterministic causality which the neo-serialists postulated was lost, then it was necessary to replace it by a more general causality, by a probabilistic logic which would contain strict serial causality as a particular case.\footcite[8]{xenakis71}
\end{squote}

Whereas in the case of Xenakis, however, the results of the stochastic processes were strictly encoded into a score or a computer program, in \emph{etherSound} there is no score as such. The mapping between characters in the input, and synthesis and sequencing parameters used to produce the output, is fixed in the program but no sound will be produced unless someone takes the action to provide the system with input. John Cage's non-deterministic music based on chance operations is another example in which events in some regard external to the composer are allowed a great influence on the final result. Cage's aesthetics were a means to remove \textit{intention} from the artistic expression. On the face of it there may be a conceptual resemblance between \emph{etherSound} and the non-determinism of Cage. In \emph{etherSound}, however, it is precisely \textit{intention} that produces sound: the wish to participate is all that is needed. 

The choices that had to be made in the mapping of input to output in the program are the same kind of choices I make when I compose or improvise. I would call these compositional choices. They are based on my musical experience and what it is I want to achieve---for whatever reasons---at any particular moment. The process of making these choices in the context of developing and designing interactive systems is well described by Camurri and colleagues: ``The designer of the performance introduces sound and music knowledge into the system, along with the compositional goals [\ldots]''.\footcite[Camurri, Richetti, and Trocca, 1999, as cited in][373]{rowe01} This fact, that compositional choices were made in the course of constructing \emph{etherSound}, does not necessarily make it into a `composition'. Before we ponder more on the identity of \emph{etherSound}, however, what is the nature of its driving force, the interaction between the program, the \index{performer}performers and the participants? In what sense is \emph{etherSound} interactive? If the mapping is fixed in the program, what is the influence of the participant? 

\subsection{Interaction in \emph{etherSound}}
\label{sec:inter-emph}

\emph{etherSound} has been used in two different contexts: as a standalone sound installation that users can interact with but also in combination with one or several improvising musicians playing acoustical instruments. The discussion that follows will primarily deal with the latter situation, which resembles a traditional concert but one in which the audience, apart from listening, may also `play' the electronic instrument. As can be gathered from the description of the system given above, the sonic outcome of a received SMS is fairly strictly pre-conceived. On the individual level, only a limited amount of influence over detail is offered, and it is debatable whether \emph{etherSound} can be called an `instrument' at all. 
%%% Why would we like to call etherSound an instrument? %%%
This was, however, never the intention. It is the \emph{desire} to contribute to the whole that was intended to be the ruling factor, not the individuality of expression or the virtuosity of performance. But in that case, why not simply have a button that the users can press at will which generates a pseudo-random sequence of sonic events? Surely, this too would allow for unconditioned participation.
 
In the very first performance of \emph{etherSound}, on the day of the opening of the exhibition \emph{The Invisible Landscapes} in August 2003, and owing to a technical problem,\footnote{The problem was owed to an unknown inconsistency in how JSP (Java Server Pages), which I used on the server to parse the messages, handled HTTP/POST requests when the version of the HTTP differed between the caller and the receiver.} as an emergency solution I implemented a version which basically worked like a button. I was unable to parse the actual contents of the SMS messages sent to the system (I merely obtained a notification that a message had been received). Rather than cancel the performance I had the program read an arbitrary number of words from a text file on my hard drive and use that as a `fake' message. Still, in the information about the installation and in the program notes for the concert, all of which had been prepared well in advance, it was stated that the system responded to the contents of the message when composing its output. After the concert a few of the listeners/participants came up to me and told me how clear they thought the connections between the SMS contents and the sounds were. The expectancy of a correlation between input and output was so strong that, despite the fact that the actual mapping was completely random, the connection was created in the perception of the participant. This is not to say that `faking' interaction is a practicable solution, but merely that expectation, and hence information about (modes of) interaction, is an important factor. 

\emph{etherSound} is not interactive in the same way that, for example, a computer game is interactive. Once the SMS has been sent, there is no way for the participant to alter or influence the sound. There is correlation between input and output insofar as short messages produce short message-compositions and vice versa. After sending a few messages, or after listening to a series of message-compositions, the participant will know what to expect and the ease of use is perhaps the greatest advantage of \emph{etherSound}. Interaction in the context of computers and technology is more or less synonymous with \textit{control}, or with the ability to change the prerequisites during the course of action. As put by George \citeauthor{lewis00}: ``interactivity has gradually become a metonym for information retrieval rather than dialogue''.\footcite[36]{lewis00}
%%% What about 'dialogueâ' in etherSOund... Is there real dialogue? %%%
Computer programs that are not interactive perform a task based on the information given to them at the outset. In interactive computer programs the parameters can be changed dynamically. By this definition and if we restrict the time frame to one message-composition, \emph{etherSound} is not really interactive or only interactive in a very limited sense. It does not allow the user to dynamically control the musical contour of the message-composition. It is more of a stochastic jukebox whose `play' button works by means of sending an SMS. 

If, however, we expand our understanding of interaction and include readings that are more closely related to \index{interaction!social}social interaction, which is not about control, but about exchange, about giving and taking, and about growing and establishing identity,\footnote{The role of social interaction in human existence has a long philosophical history, in recent years kept alive by Hannah Arendt and J\"{u}rgen Habermas. This is discussed in more detail in Chapter \ref{sec:interraction-self}.} and we expand the time frame to include a series of message-compositions, we can come up with another analysis. If our general requirement for the definition of interaction is not limited to the subject's unbounded control over content (``information retrieval''), and the context-specific requisite for interaction is not restrained to the participant-\index{interaction!computer}computer interaction, but also includes participant-participant and participant-performer interaction: then \emph{etherSound} may well be said to manifest a form for dynamic interaction and the users who interact with it do indeed have influence.
%%% IS interaction the same as influence? %%%
In a recent performance (Copenhagen, August 2007) a participant sent a message that ended the concert. Whether that was intended or not is less important than the actual consequence. The participant introduces a change in the musical context and, though he or she does not control the outcome of this change, the participant still in effect has the power of influence (influence rather than control) through interaction. 

What then are the consequences regarding interaction that may be drawn from working with \emph{etherSound} in the context of performance? If we begin by thinking about this piece as an improvised live performance, on an individual level the system \emph{etherSound} provides the ability for any member of the audience (who by virtue of being a part of the audience is already interacting with the performance) interactively to introduce a change to the sonic environment at any point, albeit with a very limited control over the outcome. Now, from my elevated perspective as a musician with fifteen years of professional experience I am in no position to tell what this situation means to someone who has never before participated in an improvised musical event. It may be incredibly dull or it may be the most exciting sensation. For me, as an improviser, the interaction as it is taking place here supplies that which the computer does not (and never will be able to?) possess---the intention. The message-compositions are not dispersed randomly (as in a pseudo-random computer algorithm) but because someone wanted to participate. 
%%% To me this sounds a bit like our voting system. There is an intention on the part of the voter, there even might be some kind of influence. However, s/he has no idea at all what s/he will get after the elections %%%
When I was playing and I heard the sounds of a message-composition I felt honoured that someone had taken the time to participate, and it felt as though I had been given a gift.\footnote{The `gift' aspect is further discussed in \cite{frisk05} and \cite{yoshida06}.} Though the idea that the participation would supply me with a non-predictable, but not random, series of impulses, was part of the original conception of \emph{etherSound}, I had not anticipated the impact it would have on me. It is not easy to make general assumptions regarding interaction, but to me this shows the importance of moving beyond interaction as a deterministic mapping of stimulus to response.
% , to let both parties involved in the interaction create the object of interaction in order to intend it. %%% But is this determinism interaction? %%%

There are a number of different kinds of interaction going on in a performance of \emph{etherSound}, on many different levels. There is the low-level interaction between the participant and the computer mediated through the mobile phone as well as higher levels of interaction between groups of participants and groups of \index{performer}performers. To summarise, in order to appreciate the nature of the interactive potential for \emph{etherSound} (i) time needs to be considered, (ii) expectation is an important factor, and (iii) information \emph{about} the processes taking place as a result of interaction (`meta-information' or the `grammar' of interaction) is absolutely essential. 

%%% More general remarks after reading thus far: Go deeper into the topic of interaction. E.g. (1) What about the difference between musician-composer, participant-composer, participant-musician? (2) Is intention a prerequisite for interaction and is therefore no real interaction possible with a computer? %%%

\subsection{The work identity of \emph{etherSound}}
\label{sec:ident-emph}

The question of the work identity is not merely a theoretical issue in this context of purely scholarly import. If the intention of \emph{etherSound} was to create an open-ended platform for public participation with a focus on interaction and, in the end, the result has more in common with a composition for instruments and computer, not only did the intention fail (which may be perfectly all right), but my personal objective, to use interaction as a way to open up the creative process and give up compositional control, failed. The latter may also be all right, but if in the long run there is a continuous discrepancy between artistic intentions and practical results this is likely to create personal and artistic frustration. 

Given the different agents involved in the production of musical content in \emph{etherSound}, the most obvious perspective to adopt (given that we are talking about message-compositions) is that the participants are the composers and the computer with the improvising musicians are the \index{performer}performers. This would make the SMS the score(s). Musicologist Peter \citeauthor{kivy02} gives a definition of the musical score as ``a complex symbol system. From the performer's point of view it is a complex set of instructions for producing a performance of the musical work that it notates''.\footcite[204]{kivy02} Applying this definition to \emph{etherSound}, we may extract (at least) two other plausible explanations of its structure:
%
\begin{enumerate} 
\item If the participants (SMS senders and improvising musicians) are the \index{performer}performers, the instructions, the meta-information or the `grammar' of the interaction constitute \emph{a} score. 
\item If the computer is the performer (which would turn the participants into quasi conductors) the computer program, i.e. the code in which the mapping between input and output is defined, would constitute another \emph{version} of \emph{a} score. \end{enumerate}
%
According to the definition given, even if we regard the work from two or three different perspectives, there is a score. 
%%%The interesting point you’re coming at is that etherSound makes use of a singular score which seems like a kind of paradox %%%
If in fact there is a score of some sort in which the mapping is fixed and not subject to change through interaction, and if the process of building this mapping scheme is similar or even equal to compositional processes, in what sense does \emph{etherSound} differ from a composition? First of all, Kivy's definition is clearly by no means conclusive. Second, there is an important difference between a more traditional composition and a work such as \emph{etherSound} in the dimension of time, as the latter does not have a fixed beginning or an end. Last, and most important, between the two contrasting musical work concepts `closed' or `pre-conceived', and `open' or `free' there is a range of possibilities, and, as with so much other music and art, depending on \emph{when} and \emph{how} you look at the piece it will define itself at different points on the open-closed axis. \emph{etherSound} exists in this field ranging from the closed form of a message-composition to the openness of the large scale form of the improvisation. 

More than anything else \emph{etherSound} is an improvisation. The structure or the `language' for the improvisation may be different depending on your role, and the `score' (if it exists) is ``a recipe for possible music-making''.\footcite[Evan Parker, as cited in][81]{bailey92} The compositional choices discussed above are a part, for this piece a necessary part, of the structure that makes possible the different entry points. Systematic and pre-conceived construction in one phase of a musical project does not have to limit the performative freedom or result in a closed `work'. On the contrary (and perhaps in opposition to the romanticised view on improvisation), preparation for an improvised performance, even in free form jazz improvisation, is quite often highly structured and systematic. Improviser and jazz saxophonist Steve Lacy has the following recollection of the early years of Cecil Taylor's career:\footnote{It should be pointed out that I make no comparison between \emph{etherSound} as music and the music of Cecil Taylor.} ``And the results were as free as anything you could hear. But it was not done in a free way. It was built up very, very systematically [\ldots]''.\footcite[Cited in][81]{bailey92} Further, construction and pre-conception on the detailed level do not exclude that the whole is still open and self-generated: ``[E]ach of the numerous released recordings of, say, Coltrane's `Giant Steps,' regarded at the level of individual passages, is the result of careful preparation [\ldots]. At the same time, each improvisation, taken as a whole, maintains its character as unique and spontaneous''.\footcite[108]{lewis-1} In both the recordings of \emph{etherSound} (2003 and 2007) the common `language' of the performing musicians is their background as jazz improvisers.
%%% So, the interaction goes in fact one way: it’s just a matter of seducing the performer (you) %%%
The language of the participants is the text and logic of SMS messages and the intention to participate is what binds the two together. 

To conclude this discussion I would like to turn again to George \citeauthor{lewis-1}, who, I believe, captures the essence of how form and structure are developed in improvised music: 
\begin{squote} 
My own view is that in analyzing improvisative musical activity or behaviour in structural terms, questions relating to how, when, and why are critical. On the other hand, the question of whether structure exists in an improvisation---or for that matter, in any human activity---often begs the question in a manner that risks becoming not so much exegetic as pejorative. It should be axiomatic that, both in our musical and in our human, everyday-life improvisations, we interact with our environment, navigating through time, place, and situation, both creating and discovering form. On the face of it, this interactive, form-giving process appears to take root and flower freely, in many kinds of music, both with and without preexisting rules and regulations. \footcite[117]{lewis-1} 
\end{squote} 
From my personal thinking about the work identity of \emph{etherSound}, I have gone full circle. At the outset I thought of it as nothing but a framework for improvisation in which I could allow myself to experiment with using the computer in a way distinctly different from what I was used to. Then, for many reasons, one of which was the fact that it is precisely not \emph{axiomatic} that form may be created as well as constructed, I went into a phase of denial, in which I sought for a structure that would allow me to call \emph{etherSound} a `work', only in the end to arrive at the conclusion that what it is, first and foremost, is an improvisation. 

%%%
%%% General remark: With whom is the audience actually interacting? (1) With the musicians on stage; (2) With etherSound; (3) With a computer; (4) With their cell phones; (5) With music (sounds)????? 
%%%

\section{Summary}
\label{sec:endnote}

Whether or not the participants felt they had influence and whether this influence set creative energies in motion within the participant can only be proved, if at all, by performing empirical and statistical studies that are beyond my intentions and competence. What I can do, however, is reflect on my experiences of performing \emph{etherSound} on a number of occasions. From my experiences as a performer---from that perspective---I can say that the participants were truly interacting with the music and they had genuine influence on the development of the performance. As an improviser in the context of a performance, I experience no difference between an initiative taken by one of the other musicians or one introduced by a participant---they are of equal import. 
%%% Perhaps of equal import but at least the action-reaction among musicians continue and develop (deepen) %%%
Though I have programmed \emph{etherSound} myself, enough musically crucial aspects of the message-composition are unknown at the outset for any message-composition to hold potential for musical change. Obviously, the nature of the interface and the way the piece is programmed put great limitations on the creative possibilities of the participants, especially were they to work with it repeatedly. That, however, does not mean that the individual, single act of participating does not harbour creative and interactive potential. Just as I, when improvising, cannot be certain how a musical initiative taken by me will influence the development of the music, so the participant will not know either. 
%%% Is that really the same thing for you? %%%
Still---just to be absolutely clear about what it is we are talking about---sending an SMS to \emph{etherSound} during a performance is obviously nothing like, not even closely related to, improvising on an instrument one has learned and mastered. This is not, however, the point I am trying to make. The point I am making by this long detour of reflection is that perhaps---and my own experience seems to corroborate this---a tiny atom of that which constitutes the essence of, at its best, flowering, form-giving process of improvisation, to use the words of George Lewis, can be shared by those whose participation is restricted to a mobile phone.
%%% Is this your main conclusion? %%%

In musical improvisation the more I play with and get to know my co-improvisers, however, and now we are approaching the weak spot of \emph{etherSound} as a platform for interaction, the easier it will be to predict the result of musical actions taken. For the participants there are currently only very limited possibilities for this kind of development to take place. Developing the expert performance aspect of a work whose objective is related to a notion of `equality of participation'---that all, regardless of prior (musical) training, should be allowed an equal chance to participate---is not un-problematic. If this goal is to be adhered to care must be taken not to accomplish the expert performance aspect at the expense of un-initiated participation. Adding a second layer of interaction would be one way to allow the interested participant to acquire the skills to take part in a performance more actively and consciously. This second layer could be implemented by making a phone call and interacting with the message-composition---changing the volume, the timbre, the tempo, etc.---in \index{real-time}\index{time!real-time}real time, either by pressing digits on the phone or by voice control. 

In \emph{etherSound} in the performance context the audience is invited to take part in a group improvisation. Though the \emph{\index{interaction!as control}\index{interaction-as-control}interaction-as-control} aspect of the participation is very limited the interactive action influences the music in a similar way to \index{interaction!musical}musical interaction in the context of improvisation. 
%%% I think I don’t agree (see previous remark) %%%
To summarise my own experiences of working with this project over a number of years, I can say that the sensation of improvising in a context where the audience can give sonic input---input that becomes an important part of the performance---is very rewarding, to an extent that I did not anticipate at the time the project started. A challenge for future development of the concept, though the artistic implications of such development have to be carefully evaluated, would be to develop the participant control aspect of the interface without losing the collaborative focus.


%%% So, what did etherSound teach you about the relation computer-music-interaction? What can you conclude in more general terms that will benefit not only your own artistic work but also others? %%%

%%% Local Variables: 
%%% mode: latex
%%% TeX-master: "../ImprovisationComputersInteraction"
%%% End: 
