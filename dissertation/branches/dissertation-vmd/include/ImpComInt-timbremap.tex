\section{timebreMap}
\label{sec:timebremap}

\emph{timbreMap}

In this paper I intend to outline a proposed system for tracing relative change in timbre in an audio signal in real time. The output of this system may then be used to trigger musical events in a manner similar to how pitch tracking and score following is commonly used. The propsed system makes use of a self-organising feature map (commonly referred to as a Kohonan net) following a model for speaker independent word recognition \cite{huang92}.


The question as it is articulated here gives a somewhat false image of what `blend' or `common ground' comprise. 

where the `blend', to a high degree, is a property relative to the musicians playing and the context in which the music is taking place---the `blend' is not a predetermined factor. I cannot, before I play, tell someone else, human or machine, how I will articulate my tone. For two musicians to `blend' their sounds they simply have to play \emph{together} and in the action both will adjust their tone, intonation and articulation in a recursive fashion. It would be reasonable to maintain that it is primarily a question of getting the computer to `blend' with the human performer, as `blending' is a natural part of any (Western) musician's training. 

\emph{timbreMap} is a software development project that attempts to
allow for direct interaction with sound itself rather than with an
abstract classification of sound. This is the part of my doctoral
project that I was initially inclined to look at as the central goal. I
have already referred to my growing frustration with the way in which
the tools available to me for letting a computer interact with a
performer in real time did not satisfy my needs as a composer. One
example of a widely-used tool in electroacoustic music with live
instruments is something that is referred to as ``pitch-tracking''.
%%%%%%%%%%%%%%%%Used
What
this process attempts to achieve is the transformation of a (monophonic)
audio signal into discrete pitches. Aside from the fact that this is a
difficult task, the information gleaned by this system is only
useful if the pitch class representation is a meaningful and substantial
parameter in the intended totality of the musical output. In much of my
music it is not.
%%%%%%%%%%%%%%%%%%%%
\footnote{Please see
\url{http://www.henrikfrisk.com/index.jsp?metaId=music&id=music&about=1
&field=name&query=Insanity} for an example of an improvised piece in
which pitch has no significance from a structural point of view.}
With \emph{timbreMap} I have attempted to construct a system that uses
self-organizing feature maps and chained neural-networks to track the
relative change of timbre in an audio stream and make this information
available for interaction. As was mentioned above, 'connectionism' or
neuron-like computing is in itself a move away from the binary
representation of numbers, approaching what may be called an attempt at
modeling continuous processes. It is a special purpose machine that is
closely linked to the artistic enterprise that created the need for
it. In that sense, although in a more abstract way than the programming
of \emph{etherSound}, its code is part of the notation of the possible
pieces it may give rise to.

In the time that has passed since I started my doctoral project I have
come to realize that, although \emph{timbreMap} is a big part of the
totality of my project, both in terms of time invested and its
significance to the whole, its most intriguing aspect may be the mapping
of the output of the system and the musical stimuli to which it gives
rise. The mapping must be related to large-scale empirical studies, such
as those mentioned in this article, but also has to be tested in the
specific case-studies. From \emph{etherSound} I learned that successful
mapping involves a certain amount of pedagogy - knowledge creates
anticipation and expectation. The studies performed within the project
\emph{Negotiating the Musical Work} opened up the idea of the `creative
misunderstanding' and a semiological analysis of the communication
within an interactive system.

%%% Local Variables: 
%%% mode: latex
%%% TeX-master: "../ImprovisationComputersInteraction"
%%% End: 
