
\chapter*{Abstract}
\label{cha:abstract}
\addcontentsline{toc}{chapter}{Abstract}
\setcounter{page}{1}
\pagestyle{plain}

Interaction is an integral part of all music. Interaction is part of listening, of playing, of composing and even of thinking about music. In this thesis the multiplicity of modes in which one may engage interactively in, through and with music is the starting-point for rethinking \index{interaction!human-computer}\index{HCI}Human-Computer Interaction in general and Interactive Music in particular. I propose that in \index{interaction!human-computer}\index{HCI}Human-Computer interaction the methodology of control, \emph{\index{interaction!as control}interaction-as-control}, in certain cases should be given up in favour of a more dynamic and reciprocal mode of interaction, \emph{\index{interaction!as difference}\index{interaction-as-difference}interaction-as-difference}: Interaction as an activity concerned with inducing differences \emph{that make a difference}. \emph{\index{interaction!as difference}\index{interaction-as-difference}Interaction-as-difference} suggests a kind of parallelism rather than click-and-response. In essence, the movement from \emph{control} to \emph{difference} is a result of rediscovering the power of improvisation as a method for organising and constructing musical content. \emph{Control} and \emph{difference} are not mutually exclusive, however; they are not opposing concepts. \emph{Interaction-as-difference} is to be understood as a broadening of the more common paradigm of direct manipulation in \index{interaction!human-computer}\index{HCI}human-computer interaction. Improvisation is at the heart of all the sub-projects included in this thesis, and also, in fact, those that are not immediately related to music but geared more towards computation. Trusting the self-organising aspect of musical improvisation, and allowing it to diffuse into other areas of my practice, constitutes the pivotal change that has radically influenced my artistic practice. The \useGlosentry{glos:work_in_movement}{\emph{work-in-movement}}, a concept introduced by Umberto Eco, is (re-)introduced as a work type that encompasses radically open works. The \emph{work-in-movement}, presented and exemplified by a piece for guitar and computer, requires different modes of representation, as the traditional musical score is too restrictive and is not able to communicate that which is the most central aspect: the collaboration, negotiation and interaction. The Integra framework---comprising a database model and a corresponding XML representation---is proposed as a means to produce annotated scores that carry past performances and versions with it. The notion of the giving up of the \index{Self}Self is suggested as the common nominator, the prerequisite, for an improvisatory and self-organising attitude towards musical practice that allows for \emph{\index{interaction!as difference}\index{interaction-as-difference}interaction-as-difference}. Only if the \index{Self}Self is able and willing to accept the loss of priority of interpretation, willing to give up or disregard faithfulness to ideology or idiomatics, is difference conceivable. Only if one is willing to \emph{forget} is \emph{\index{interaction!as difference}\index{interaction-as-difference}interaction-as-difference} made possible. Among the artistic works that have been produced as part of this inquiry are some experimental tools in the form of computer \index{software}software to support the proposed concepts of interactivity and together with the musical works they make up both the object and the method in this PhD project. Contained within the thesis, these sub-projects (all of which are works-in-progress), are used to make inquiries into the larger question of the significance of interaction in the context of artistic practice involving computers.

\chapter*{Acknowledgements}
\label{cha:acknowledgments}
\addcontentsline{toc}{chapter}{Acknowledgments}

A great number of people have been of importance to this project and to my musical career in general, which provided the foundation for my PhD studies. First, I would like to thank all the musicians I have played with, for music is all about interaction and interaction is about meeting the other. Specifically: Peter Nilsson (to whom I owe a lot of knowledge), Anders Nilsson, Andreas Andersson, David Carlsson. Henrik Frendin, who commissioned and played \emph{Drive} and who endured my presence on a number of tours, Per Anders Nilsson who has inspired and contributed to my musical and technological development, Ngo Tra My and Ngyen Thanh Thuy who introduced me to another Other and another \index{Self}Self; and a very special acknowledgement should go to my colleague and co-musician Stefan \"{O}stersj\"{o}, without whom my thesis would have looked different in many respects (who also kept up with me on many travels).

Master musicians Richie Beirach and David Liebman, through their own musical mastery, inspired me to seek my own artistic paths. Bosse Bergkvist and Johannes Johansson should be recognised for having aroused my interest in \index{electro-acoustic music}electro-acoustic music as well as for giving me the opportunity to exercise it. Coincidentally they have both been present throughout my journey and Johannes played an important role through his early support of this project. Cort Lippe gave support and help and Kent Olofsson's enthusiasm, kindness and helpfulness should not be forgotten. I must thank the Integra team in general and Jamie Bullock in particular: the kind of collaboration we established I feel is rare and itself an example of \emph{\index{interaction!as difference}\index{interaction-as-difference}interaction-as-difference}. All my students over the years have been a continuous source of inspiration as have my colleagues at the Malm\"{o} Academy of Music. Peter Berry and the staff at the library should be thanked for their help and patience.

The joint seminars at the Malm\"{o} Art Academy led by Sarat Maharaj, where the concept of artistic research was discussed and interrogated, had a tremendous impact on how my studies and my project developed. My PhD colleagues at the Art Academy, Matts Leiderstam, Sopawan Bonnimitra, Anders Kreuger as well as Kent Sj\"{o}str\"{o}m and Erik Rynell at the Theatre Academy should be thanked for their feedback and a very special acknowledgement should go to Miya Yoshida as well as Gertrud Sandqvist and Sarat Maharaj. The staff at Malm\"{o} Museum are thanked for hosting \emph{etherSound}, in particular Marika Reutersw\"{a}rd.

Later, in our seminars at the Music Academy, apart from Stefan \"{O}stersj\"{o}, Hans Gefors has been a great inspiration (as when I studied composition with him much earlier). I am grateful to Prof. Hans Hellsten who, both when he participated in the seminars and when we worked together on other topics, showed ceaseless support. Trevor Wishart, Eric Clarke, Per Nilsson, Bengt Edlund and Simon Emmerson are among those who participated in the seminars and gave important input to my project. Prof. Greger Andersson and the department of musicology at Lund University have also been supportive and helpful in many different respects. The way H{\aa}kan Lundstr\"{o}m managed to keep up the seminars, head the entire faculty, and give individual support and guidance as well as handle the practicalities and bureaucracy of our PhD programme is nothing short of staggering.

I can safely assert that without Prof. Leif L\"{o}nnblad, as teacher, adviser and friend, and Prof. Miller Puckette, also a great source of inspiration, this project would have looked very different. Karsten Fundal has been my artistic guidance for many years, even prior to my PhD project, and has tirelessly kept asking all the difficult questions, a capacity shared by Dr Marcel Cobussen who gracefully handled the difficult task of picking up the project more than halfway through. The acuity of all four of my advisers has had a decisive impact on my work.

Finally I must not forget to thank my family, my siblings and my parents: my father, for getting me started, by insisting that I get a ``proper education'' (which,I hope, I have now finally acquired); my mother, for bringing me to the end, telling me to finish my PhD so she could be part of the ceremony before she passed away (which she hasn't); Karin and Sara, for being an inspiration; Thomas and Mikael, too; Lennart and Rose-Marie for all their help and support. My three sons, Arthur, Bruno and Castor, have all been born during this PhD project (I wonder if they will miss it?) and are part of it, as they are part of anything I do. Thank you, Lena, for putting up with me and for making all of it possible.

Finally, I wish to acknowledge Olle and Leo without whom none of this would have been written.

% \chapter*{Guide to the thesis}
% \label{cha:guide-this-document}
% \addcontentsline{toc}{chapter}{Guide to the thesis}

% This thesis is essentially divided up in two parts. This PDF document consisting primarily of text and some images, and an accompanying set of HTML documents consisting primarily of documentation of the artistic work in the form of video and audio recordings. Although most document viewers will work, for full compatibility the PDF is best read in Acrobat Reader\footnote{Available as a download free of charge from Adobe: \url{http://www.adobe.com/products/acrobat/readstep2.html}}. The HTML document is readable and viewable with most web browsers. The audio and video playback relies on the Flash player browser\footnote{Also available as a download free of charge at: \url{http://www.adobe.com/shockwave/download/download.cgi?P1_Prod_Version=ShockwaveFlash}} being installed and Javascript being enabled. Javascript is also needed for the IntegraBrowser demo. These components---Acrobat Reader, Flash player and javascript---are fairly standard on modern computers.

% \begin{wrapfigure}{r}{0.45\linewidth}
  \begin{minipage}[h]{\linewidth}
    \begin{flushright}
      \musicannot{Drive\\
        \emph{for Electric Viola Grande and computer}\\
        Composed \& premiered in 2002\\
        Commissioned by and dedicated to Henrik Frendin}
    \end{flushright}
  \end{minipage}
\end{wrapfigure}

%%% Local Variables: 
%%% mode: latex
%%% TeX-master: "../ImprovisationComputersInteraction"
%%% End: 

% The PDF is linked to the HTML document with \href{out/html/drive.html}{color coded} clickable links: \textcolor{blaa}{dark blue for links pointing to resources external to this PDF}, \textcolor{lila}{purple for internal links} and finally \textcolor{vin}{dark red for links to bibliographical data}. The pane to the right about the composition \emph{Drive} is an example of an annotated link, pointing to the documentation of Drive. When the \href{out/html/drive.html}{Listen} link is clicked on, the default web browser will start (on some systems you will get a warning message that the document is trying to connect to a remote location---this is normal and it is safe to allow it) and open a window with the requested node. No Internet access is required for this provided the media archive has also been downloaded. If you get a message that the file could not be found you probably need to download the HTML documents and the media files. These can be downloaded following the instructions found here: \url{http://www.henrikfrisk.com/diary/archives/2008/09/phd_dissertatio.php}. Another reason the web browser may fail to load the requested files is if this PDF document has been moved. For the hyperlinks to work, the PDF has to stay in its original directory. 

% From the web browser window other nodes may be accessed through the navigation provided in the web interface. To return to the PDF document, switch back to the PDF reader. It is possible to view also the PDF document from within the browser, in which case the PDF should be opened from the top \href{out/html/index.html}{HTML document}. Apart from the links that lead outside the PDF document it is also `locally' inter-linked. At the top of each page, in the header, links pointing to the \hyperref[sec:toc]{Contents}, the \hyperref[sec:biblio]{Bibliography}, and the \hyperref[sec:index]{Index} for convenience. If the interface of Acrobat lacks a 'back' button, the 'left arrow' takes you back to the to the previously read node. 

% The current document is divided in two sections: the opening five chapters discussing the main topic of \index{interaction!computer}computer interaction in music and improvisation, both from a theoretical and meta-reflective point of view (See \autoref{sec:introduction} and \autoref{sec:interraction-self}) from a perspective established in the artistic work (\autoref{sec:ethersound} and \autoref{cha:repe-main}). The concluding array of appendices are some of the papers already published and referenced in the first half of the text along with musical scores and documentation.

% To allow for a non-linear reading of the texts I have added a \hyperref[sec:gloss]{glossary} of some of the terms and acronyms used throughout. In most cases these terms are also defined within the document. I have however consciously tried to limit the use of acronyms. Citations are made using footnotes and a complete \hyperref[sec:biblio]{bibliography} of all works cited is provided. \emph{Ibidem} is used for repeated citations but with a hyperlink pointing to the full reference. Lookup of works cited may be done using the \hyperref[sec:index]{Index}, either by author or by title.

% Quotes of approximately 40 words or more are inset and put in a separate paragraph and the reference is given by a footnote immediately following the final period of the quote. I use American style ``double'' quotation marks for quotes and `single' quotation marks for inside quotes, except for longer indented quotations. These are typeset without surrounding quotation marks and any inside quotes are printed exactly as in the text cited. Commas and periods are put inside the closing quotation mark if they belong to the quotation, else outside. Footnote marks are consistently put after punctuation.
% \\[8cm]
% \begin{flushright}
%   \begin{minipage}[r]{0.7\linewidth}
%     This document, as well as most of the artistic contents, are     produced with open source software. The thesis is written on the     GNU Emacs text editor, typeset with \LaTeX, making use of Bib\TeX~     for references, and Auc\TeX~ for editing. Graphics are produced     with Inkscape SVG editor and images are edited with Gimp and     Imagemagick. Videos and screen casts are edited with Cinelerra and     ffmpeg. Musical notation is typeset using Lilypond.
%   \end{minipage}
% \end{flushright}
%%% Local Variables: 
%%% mode: latex
%%% TeX-master: "../ImprovisationComputersInteraction"
%%% End: 
