\section*{Abstract}
\label{cha:abstract}
\pagestyle{plain}

Interaction is an integral part of all music. Interaction is part of listening, of playing, of composing and even of thinking about music. In this thesis the multiplicity of modes in which one may engage interactively in, through and with music is the starting-point for rethinking \index{interaction!human-computer}\index{HCI}Human-Computer Interaction in general and Interactive Music in particular. I propose that in \index{interaction!human-computer}\index{HCI}Human-Computer interaction the methodology of control, \emph{\index{interaction!as control}interaction-as-control}, in certain cases should be given up in favour of a more dynamic and reciprocal mode of interaction, \emph{\index{interaction!as difference}\index{interaction-as-difference}interaction-as-difference}: Interaction as an activity concerned with inducing differences \emph{that make a difference}. \emph{\index{interaction!as difference}\index{interaction-as-difference}Interaction-as-difference} suggests a kind of parallelism rather than click-and-response. In essence, the movement from \emph{control} to \emph{difference} is a result of rediscovering the power of improvisation as a method for organising and constructing musical content. \emph{Control} and \emph{difference} are not mutually exclusive, however; they are not opposing concepts. \emph{Interaction-as-difference} is to be understood as a broadening of the more common paradigm of direct manipulation in \index{interaction!human-computer}\index{HCI}human-computer interaction. Improvisation is at the heart of all the sub-projects included in this thesis, and also, in fact, those that are not immediately related to music but geared more towards computation. 

Trusting the self-organising aspect of musical improvisation, and allowing it to diffuse into other areas of my practice, constitutes the pivotal change that has radically influenced my artistic practice. The \emph{work-in-movement}, a concept introduced by Umberto Eco, is (re-)introduced as a work type that encompasses radically open works. The \emph{work-in-movement}, presented and exemplified by a piece for guitar and computer, requires different modes of representation, as the traditional musical score is too restrictive and is not able to communicate that which is the most central aspect: the collaboration, negotiation and interaction. The Integra framework---comprising a database model and a corresponding XML representation---is proposed as a means to produce annotated scores that carry past performances and versions with it. The notion of the giving up of the \index{Self}Self is suggested as the common nominator, the prerequisite, for an improvisatory and self-organising attitude towards musical practice that allows for \emph{interaction-as-difference}. Only if the \index{Self}Self is able and willing to accept the loss of priority of interpretation, willing to give up or disregard faithfulness to ideology or idiomatics, is difference conceivable. Only if one is willing to \emph{forget} is \emph{\index{interaction!as difference}\index{interaction-as-difference}interaction-as-difference} made possible. Among the artistic works that have been produced as part of this inquiry are some experimental tools in the form of computer \index{software}software to support the proposed concepts of interactivity and together with the musical works they make up both the object and the method in this PhD project. Contained within the thesis, these sub-projects (all of which are works-in-progress), are used to make inquiries into the larger question of the significance of interaction in the context of artistic practice involving computers.

%%% Local Variables: 
%%% mode: latex
%%% TeX-master: "../Abstract-layout"
%%% End: 
