This text and the collection of published papers, together form the
written part of my PhD thesis with the working title \emph{Music,
Computers, and Interaction}. The other two major parts of this work
are the artistic output\footnote{An archive containing all the
  material may be downloaded from
  \url{http://henrikfrisk.homeunix.net:8800/svn/dissertation/FriskMusic.zip}. 
Use `guest' for login and password. Should you experience problems
with the download, send me an email at henrik.frisk@mhm.lu.se. The
archive is nearly 700MB.} 
and the computer programs. Though a paper on the software project
libIntegra is included the software projects \emph{timbreMap} and \emph{libIntegra}
are not themselves included in this version of the thesis.

The main purpose of the first chapter (Chapter \ref{cha:introduction},
\emph{Introduction}) of this text is to provide the reader with an
overview of the research project. The next chapter (Chapter
\ref{cha:interaction}, \emph{Music and Interaction}) will discuss
interaction from a general, as well as to music specific
perspectives. Only the first part of Chapter \ref{cha:interaction} is
included in this version of the text. The first section (Section
\ref{sec:interaction} looks a differences between human-computer
interaction and social interaction as well as musical interaction and
in the first subsection (\emph{Social interaction and the giving up of
  the self}) I posit myself in the context of interaction. This
version of the text ends rather abruptly after this subsection.

What I intend for the continuation is a discussion of the parties
involved in any interaction from a subjective perspective, as `the
self' and `the other'. The next section will be an overview of the
field of interactive music, and finally a section about my own
interactive music. The final chapter, also not included here, will be
a summary and an outlook.

This is not the final version of any of the parts, but a version
produced for my 75\% seminar, the primary purpose of which is to `try
out' the material in the `real world'. I envision the final result of
this thesis to be produced in a hypertext format: The medium should
allow for seamless transitions between documentation of artistic
output and text based content. A somewhat evolved version of the way
the artistic output is presented.  (All material---text, music, and
programs---on a DVD browsable from a standard WWW browser.)
\clearpage
\section{Terminology and acronyms}
\label{sec:terminology}

Below is a list of terms that are used in these texts and that may
require some extra explanation.  Either because I want to delineate
its meaning or to avoid misunderstandings due to ambivalent
interpretations. Also included here are also acronyms
that, for the most part are explained in the text the first time they
appear.

\begin{itemize} %%% [\bfseries\textendash]

\item \textbf{ANN} - Artificial Neural Network.

\item \textbf{Agent} - In the studies performed within the frame of
  \emph{Negotiating the Musical Work} (see Section
  \ref{sec:negot-music-work}) we use the concept of `agent' (not to be
  confused with the software based `intelligent agent'). Many
  different kinds of agents are involved in the production of musical
  content. ``We find that by using the concept of `agents' we bypass
  the otherwise problematic values traditionally assigned to''
  `composer' and `performer' \citetext{see also \citealp[p.
    35]{wis96}}.

\item \textbf{Electro-acoustic Music} (EAM) - According to the Oxford
  English Dictionary electro-acoustics are ``acoustics investigated by
  electrical methods''\footnote{``electro-acoustics'' The Oxford
    English Dictionary. 2nd ed. 1989. OED Online. Oxford University
    Press. 31 Oct. 2007
    <http://dictionary.oed.com.ludwig.lub.lu.se/cgi/entry/50073014>}.
  Electro-acoustic music is a broad term used to denote music produced
  by or with electrical methods. Today, since this is mainly achieved
  by the use of digital computers, in USA the use of the term
  \emph{Computer Music} is more common\footnote{The issue of
    terminology in the field of electro-acoustic music is complex and
    there is an apparent lack of standardized vocabulary. How to
    label the entire genre, let alone sub-genres and particular
    processes within the field of electro-acoustic music, has recently
    been debated during a conference organized by EMS (see
    \citealp{EMS06}, in particular \citealp{landy06,dack06,battier06} and
    in relation to translation see \citealp{fields06}}). Yet another
  term, more commonly used in the francophone countries is
  \emph{acousmatic}, and it is sometimes argued that \emph{acousmatic}
  refers to the genre and \emph{electro-acoustic} to the means of
  production (e.g. \textparagraph 9 ``Vous avez dit...''
  \citeyear{musique-recherche}). In this text I will consistently use
  electro-acoustic music, or the acronym \emph{EAM} to denote my own
  artistic work involving computers and music.
   
\item \textbf{Esthesic} - An analysis of the (inductive) esthesic
  ``grounds itself in perceptive introspection'' - that which is
  ``perceptively relevant'', that which one hears \cite[pp.
  140-3]{nattiez}. See also \emph{Poietic}

\item\textbf{GUI} or \textbf{UI} - Graphical User Interface or User
  Interface.

\item\textbf{HCI} - Human Computer Interaction.

\item \textbf{Intelligent agent} - An idea to improve HCI introduced in
  the mid 90's. An agent is a piece of software designed to collect
  and sort information and present it to the user. The idea is that
  the agent will learn what it is its user wants, or needs to know
  about. (The `intelligent agent' should not be confused with the
  `agent' as a factor in the production of musical content.)

\item \textbf{Musician} - I use `musician' in a very inclusive
  way in these texts. A composer, an improviser, a performer are all
  sub-categories to the general description `musician'.

\item\textbf{Pitch-tracking} (also pitch-to-MIDI, pitch detection) - To
  let a computer (or a special purpose device) analyze an audio signal
  in real-time and extract the most likely fundamental of the
  sound. See the discussion i Section \ref{sec:interaction}. 

\item \textbf{Poietic} - According to musical semiologists Jean-Jacques
  Nattiez and Jean Molino, the poietic phase of a musical work is the
  stage at which the musical material is constructed. According to
  \citet{nattiez}, articulating the poietic and esthesic level
  ``facilitates knowledge of all processes unleashed by the musical
  work'' \cite[pp. 92]{nattiez}. We reinterpret the terms `poietic'
  and `esthesic' in the paper \emph{Negotiating the Musical Work} (see
  Section \ref{sec:negot-music-work}).

\item\textbf{Production of musical content} - Any or all activities
  involved when producing music - its conception, performance, writing
  down (transcription) and listening.

\item \textbf{SOM} - Self Organizing Map. A type of ANN introduced by \citet{kohonen88}.

% \item\textbf{Telematics} - ``Telematics is a term used to designate
%   computer-mediated communications networking involving telephone,
%   cable, and satellite links between geographically dispersed
%   indivudals and institutions that are interfaced to data-processing
%   systems, remote semsing devices, and capacious data-storage banks''
%   \citep[p. 241]{Ascott}

\end{itemize}

\section{Typography}
\label{sec:typography}

I have tried to follow the APA referencing guide for citations as
consistently and truthfully as possible except for references to the
Oxford English Dictionary where I have used the rules depicted by OED.
When citing in text I have put the reference after the closing
quotation mark but before the period. Quotes of 40 words or more are
inset and put in a separate paragraph and the reference is given
enclosed in parenthesis after the final period.

I use American style ``double'' quotation marks for quotes and
`single' quotation marks for inside quotes, except for longer indented
quotations. These are typeset without surrounding quotation marks and
any inside quotes are printed exactly as in the text cited. Commas and
periods are put inside the closing quotation mark, but colon and
semi-colon outside. Footnote marks are put after punctuation.

%%% Local Variables: 
%%% mode: latex
%%% TeX-master: "../MusicComputersInteraction"
%%% End: 
