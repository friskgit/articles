\section{Interaction}
\label{sec:interaction}

The Oxford English Dictionary lists two meanings for the adjective
\emph{interactive}\footnote{``interactive, \textit{a}.'' The
  Oxford English Dictionary. 2nd ed. 1989. OED Online. Oxford
  University Press. 1 Nov. 2007.
  <http://dictionary.oed.com.ludwig.lub.lu.se/cgi/entry/50118746>}:
\begin{enumerate}
\item ``Reciprocally active; acting upon or influencing each other.''
\item ``Pertaining to or being a computer or other electronic device
  that allows a two-way flow of information between it and a user,
  responding immediately to the latter's input.''
\end{enumerate}

I will here attempt to unwrap and contextualize both of these
meanings, i.e. the more general concept as well as the specific
meaning relating to the use and control of electronic devices. There is a profound
difference between the two uses, the operative word being
\emph{immediately} in the second description. It introduces time as a
critical property of human-machine interaction. It is not difficult to
understand that the action initiated by the user and the humanly
perceptible response---the feedback---to the action needs to be
contiguous in time. 

\subsection{Interaction and time}
\label{sec:time-interaction}

Imagine a button to be pressed in order to introduce a change in an
interactive system. Further, say that the change itself may not be
immediately perceptible---it may be a signal to begin emptying a water
reservoir---so the user needs to be informed that the message
(``empty-reservoir-button pressed'') has been successfully received by
the system and that we implemented this by means of an audio signal
being emitted (a beep). Unless the audio signal is emitted
immediately after the user presses the button the sense of interaction
is distorted. In this context even short delays become
cumbersome---even if it is intended as a single user system.  Imagine
what would happen if 25 users are engaged in simultaneous interaction
with a system whose response is limited to a single audio signal and
the means of interaction is limited to a few buttons (such as a
stop signal frequently found in public transportation buses). The
synchronicity between the action and the response would be the only
way for the user to know that his or her action was the one the system
responded to. This to the point where the user should perceive the
sound as the result of the \emph{gesture} performed when pressing the
button; as if the button was not an electronic switch but rather a
mechanical device attached to a bell. 

The time aspect of a multi user interactive system is discussed more
thoroughly in relation to \emph{etherSound} where the latency imposed
by the mobile communication infrastructure had to be dealt with. In
the context of interactive electronic musical instruments, those of us
that have used computers and computer based synthesizers and
instruments for some years appreciate the nightmare like sensation of
a latency between a pressed keyboard key and the resulting sound of 20
ms or more. In their article on latency and computer operating systems
\citet{brandt98} writes about electronic instrument response times:

\begin{quotation}
  There do not seem to be published studies of tolerable delay, but
  our personal experience and actual measurements of commercial
  synthesizer delays indicate that 5 or maybe 10 ms is acceptable.
  This is comparable to common acoustic-transmission delays; sound
  travels at about 1 foot/ms.
\end{quotation}

Imprecise as this may be, it gives a hint of the sensitivity of human
audio perception. Any musician can learn how to play with a delay
exceeding 10 ms, at least if the latency is consistent\footnote{The church organ is an example of a mechanical
  instrument where it is necessary for musicians to adjust their
  musical timing according to the properties and delay of the
  particular instrument.} but a genuinely problematic situation occurs
when the latency does not behave linearly across the range of the
instrument. A typical example is the pitch-to-midi
converter.\footnote{A pitch-to-midi converter calculates an estimation
  of the fundamental frequency in an audio signal.} Depending on the
quality and the properties of the instantaneous audio signal that is
being analyzed the fundamental estimation may take one or several
buffers to output its result. Also, the range of the audio signal
affects the time it takes though, in general this is predictable.

The sensitivity of expectation does not seem to be limited to the
gesture/listening relation. Consider the annoying distortion of perception
that occurs when you hear an echo of your own voice when talking on
the phone---even more common now with the frequent use of VoIP. Or the
situation that occurred in the days of analog tape recorders in
which the record and playback devices were displaced by a few inches.
Listening to your own voice in headphones while simultaneously
recording it made it very difficult to talk. The recorded voice would
be delayed by perhaps 0.3 seconds, enough to create a breach between
perception and expectation so grave that speaking correctly became
impossible.

\subsection{Interaction and control}
\label{sec:control-interaction}

I will argue that interaction with technology is synonymous to a mode
of \emph{controlling} the technology. As a result of the usually
one-dimensional response of HCI---a note on the screen, a sound, a
change of direction---the `cleanliness' in time and space of this
response is of great import to the experience of the interaction. The
magnitude of possible ways in which human interaction may be carried
out on the other hand, makes the relation between time and that part of the
feedback---the response of an action---that constitutes its knowledge
bearing property less useful to discuss in terms of
milliseconds. The response to, or acknowledgment of, an action may be
a silent recognition---a body movement or facial expression---and the
actual response may come much later. Furthermore, human-human
interaction is less geared towards \emph{control} and perhaps more
focused on \emph{exchange}.

I have here put the focus on time, though this is obviously but one
conceptual difference between HCI and human-human interaction. One may
argue that the difference between interacting with a computer and
interacting with another human being is so immense that the discussion
of this difference is superfluous and uncalled for.  That the
prerequisite for human-human interaction is that both parties exhibit
some kind of sensible notion of intelligence which, by definition, the
computer will never (ever?) come close to. Therefore, HCI is, and has
to be, about control, about making technology useful through
interaction-as-control, and that this is a mode of interaction that is
of a different order compared to human-human interaction. There are
at least two sides to this issue:

The first belong to the general field of HCI where, as was briefly
discussed in Section \ref{sec:human-comp-inter}, there is a tendency
to limit the thinking about HCI to ``microlevel interactions between
programmers or users and computers. The broader social forces and
structures that constrain such interactions and are themselves
reproduced and molded by microlevel events are often left unexamined''
\citep[p. 325]{engestrom96}. Not only will this contribute ``to a
naive image of human-computer interaction as narrowly technical and as
a problem of cognitive optimization'' (\emph{ibid} p. 325), it will
also in effect risk at influencing the way we interact with other
humans. In a debate on intelligent agents computer scientist,
composer, visual artist, and author Jaron Lanier is concerned that
``people will gradually, and perhaps not even consciously, adjust
their lives to make agents appear to be smart. If an agent seems
smart, it might really mean that people have dumbed themselves down to
make their lives more easily representable by their agents' simple
database design'' \citep[\textparagraph
3]{lanier96}.\footnote{Throughout his writings, Lanier makes numerous
  accounts on the dangers of considering computers as posessing
  intelligence precisely for the reasons here mentioned. ``What starts
  as an epistemological argument quickly turns into a practical design
  argument. In the Turing test, we cannot tell whether people are
  making themselves stupid in order to make computers seem to be
  smart. Therefore the idea of machine intelligence makes it harder to
  design good machines'' \citep[\textparagraph 5]{lanier1000}. Though
  I sympathize with this and acknowledge the problem, I think Lanier
  employ a too narrow and binary reading of intelligence. The
  political as well as personal impact technology, and in
  particular information technology, has on our lives should not be
  understated, but neither should the enduringness of human
  intelligence.} Similarly, rather than making HCI more like
human-human interaction, there is a risk that we instead do it the
other way around: Assert properties of HCI on our human interaction.

The second aspect is closely related to the core of this research
project. If we differentiate HCI from human-human
interaction---understand them as two separate and only remotely
related modes of activity---how should we understand interactive music
or any other form of interaction with a computer within the spheres of
artistic practices? In Section \ref{sec:human-comp-inter} the idea of
the computer as merely a tool was questioned. This is of course
nothing new. In the mid 90's the notion of the `intelligent agent'
(which is what Jaron Lanier opposes against above) was seen as an alternative
to the tool as ``the prevailing metaphor for computers'' \citep[p.
67]{isbister95}. The personal computer could now easily communicate
with other computers, other users, keep track on things for its user,
perform many things simultaneously: ``Such an object seems inherently
different than a hammer or wrench---it has active qualities. It acts
on one's behalf---it is an agent'' (p. 68). Multimedia expert and
computer visionary Nicholas Negroponte envisioned that ``[w]hat we
today call `the agent-based interfaces' will emerge as the dominant
means by which computers and people talk with one another'' \citep[p.
102]{negroponte95}. In short and somewhat simplified: Rather than you
telling the computer what to do, it would anticipate what you wanted
to get done and ``emulate human action, assistance, and
communication'' \citep[p. 83]{isbister95}. As with so many other great
ideas, the prospect of intelligent agents has been depleted by
commercialism and, personally, I will not shed any tears if never
again I will receive an e-mail of `intelligently' selected shopping
suggestions.

Notwithstanding, the concept of `software agents' holds within it the
possibility of rethinking the idea of interaction with the computer.
As \citeauthor{isbister95} has it: ``Most forms of agent are all about
the user relinguishing (\emph{sic}) control of the computer for a
time'' (\emph{ibid}.). And to be willing to relinquish control is the
beginning of an understanding of HCI that also includes elements
usually seen to pertain to the domain of social interaction. To give
up personal control to a machine may be a frightening idea to many,
fueled by horrifying science fiction descriptions: ``the cataloging of
the individual, the processing of delocalized data, the anonymous
exercise of power, implacable techno-financial empires, [...]''
\citep[p.  117]{levy97}.\footnote{Though to me, judging from the
  popularity of online communitites such as Facebook, it seems like
  the individual of the 21st century is quite willing to allow for
  the cataloging of the identity.} But \citeauthor{levy97} reminds us that
``a virtual world of collective intelligence could just as easyily be
as replete with culture, beauty, intellect, and knowledge, as a Greek
temple [...]''  (p. 118):
\begin{quote}
  A site that harbors unimagined language galaxies, enables unknown
  social temporalities to blossom, reinvents the social bond, perfects
  democracy, and forges unknown paths of knowledge among men. But to
  do so we must full inhabit this site; it must be designated,
  recognized as a potential for beauty, thought, and new forms of
  social regulation. (\emph{ibid}.)
\end{quote}
And, to ``fully inhabit'' we must also invent new forms of
interaction.

The primary focus of the following sections is to take a
deeper look at the more general reading of interaction---``acting upon
or influencing each other''---in the context of human interaction in
the social and cultural dimensions.

\subsection{Social interaction and the giving up of the self}
\label{sec:social-interaction}

The request for responsiveness in HCI is indicative of the aspect of
control embedded in the definition: The machine should not act by
itself, it should without delay respond to our actions, to our
instructions, to how we want it to respond. In human-human interaction,
respectful of the other, a similar request for immediate response or
demand for control would be unthinkable.

Then, who is this `other'? What is the identity and location of this
`other' with whom social interaction takes place. As I mentioned briefly
in Section \ref{sec:research-question} my interest in human-human
interaction is not a goal in itself but a way to understand, inform
and try to develop musician-computer interaction in my own artistic
practice. I will here start from the specific context of my own experience
and then move to the more general idea of the `other'.

The `other' I am referring to is not only the `\emph{epistemological
  other}' of \citet{somers94}---a social construction created ``to
consolidate a cohesive self-identity and collective project''
\citetext{as cited in \citealp{lewis-1}}, though, whether I want it or
not, in a sense it is that too. The `other' is not a homogenic group that has
distinct properties that defines its `otherness'. The `other' is
`other' in relation to the `self', to \emph{me}, but not in order to
consolidate this `self', which also will not let itself be defined by
distinction.  There is no difference between the `otherness' of Ngyen
Thanh Thuy or Stefan \"{O}stersj\"{o}---the one is not more `other'
than the other---in the project The Six Tones (see Section
\ref{sec:six-tones-2006}). The `other' is the one or those I as a
musician am interacting with. It is my co-musicians with whom I am
trying to connect, whom I am trying to understand in order to
understand myself better. It is in the process of trying to
understand through interaction, that I, in a certain sense, need to give
up `the self'. Before moving on to the more general reading of the
`other' a few remarks should be made about these issues:
\begin{enumerate}
\item What I am describing here is my attempt to identify what I
  believe is going on when `things are working'. It is the ideal
  situation as I have experienced it. It is the sensation of wordless
  communication, of intuition and self organization. It is a sensation
  that is not tied to a particular idiom or style---it is not
  necessarily tied to music.
\item In no way am I able to reach this stage at all times. And, when
  unsuccessful, it is my experience that the `self' is exercising a
  wish to control the situation, though it is difficult to say if this
  precedes the failure (i.e. is a consequence of) or is an attempt to
  `fix' an error that has occurred due to other reasons. For example,
  it may be the mistake of trying to force idiom or style into a
  context that does not harmonize with that which is forced upon it.
\item I am using my artistic practice therapeutically and the idea of
  better understanding the `self' is an attempt to reach greater
  awareness of my responsibilities as a human being and as an artist.
  In particular it is a part of the process to reach self-awareness
  that I, as a white, European, male belong to a class that has
  exercised oppression and exploited women and more or less every
  other culture, religion or species that we have encountered in
  the last 2.500 years.
\end{enumerate}

% \subsection{The Other}
% \label{sec:other}

% \citet{derrida78-3} writes about the `Other' of Levinas thinking and
% how it presupposes ``[t]he infinity irreducible to the
% \textit{representation} of infinity, the infinity exceeding the
% ideation in which it is thought, thought of as more than I can think,
% as that which cannot be an object or a simple 'objective reality' of
% the idea'':

% \begin{quotation}
%   Den andre �r inte en varelse vi m�ter, som hotar oss eller vill
%   bem�ktiga sig oss. Att den andre st�ter bort v�r makt beror inte p�
%   en st�rre styrka. Det �r alteriteten som �r den andres hela styrka
%   \citep[p. 71]{levinas92}.
% \end{quotation}

% Levinas notion of the other is a face-to-face (face-�-face) encounter where the
% other is given precedence to my subjective egoism or egoistic wish to
% posses (the world) \citetext{as described by \citet[p.
% 96-7]{kemp91}}. This other cannot (and will not let itself) be
% restrained in the way that we expect to be able to control the technologies
% we interact with: ``Om man kunde �ga, gripa och k�nna till den andre,
% skulle han inte vara den andre. �gandet, vetandet och gripandet �r
% synonyma med makten''
% % If one could own, seize and know the other, would
% % he not be the other. To own and to seize is synonymous with power.
% \citep[p. 74]{levinas92}. This is what I mean with the fundamental and
% very important difference between the two aspects of interaction. Our
% relation to that with which we are interacting determines how the
% interaction will be carried out and what results we may expect, and
% when we may expect them.

% Although Levinas' first ideas of the face-to-face encounter in
% \citet{levinas69} may be criticized for not taking into account the
% absent other (see \citet[p. 102-15]{kemp91}), my main concern here is
% the physically present other. The other with which I can interact with
% directly.\footnote{How information technology, informatics, has
%   influenced social interaction is explored by \citet{kielser91} and
%   \citet{kiesler}}

% This idea of the infinitely/entirely other is in opposition to the idea
% of totality, to our wish to control nature and our environment. The
% naive desire to retain the bigger picture, the overview of the
% world. And the idea that creativity arises within the subject
% alone. It is not that the subject ceases to exist---on the contrary:
% \begin{quotation}
%   The alterity, the radical heterogeneity of the other, is possible only
%   if the other is other with respect to a term whose essence is to
%   remain at the point of departure, to serve as \textit{entry} into the relation,
%   to be the same not relatively but absolutely. \textit{A term can remain
%     absolutely at the point of departure of relationship only as
%     I}. \citep[p. 36]{levinas69}  
% \end{quotation}

% The giving up the self is to be understood as the giving up of the
% desire to control and opening yourself up to the `other'. This is what I
% desire and what I strive for in music and this is what I hear
% in the music I admire. To let creativity grow out of the meeting, the
% interaction with the `other'---may this be a co-musician or someone in
% the audience.

% \subsection{The Self}
% \label{sec:inter-prod}

% What then is the nature of this reading of `the self'---`the self'
% which at the same time must be abandoned in order to be
% re-constituted, and acknowledged in the interaction with the `other',
% by the `other'? This `self' is somewhat related to the Freudian notion
% of the superego. A culturally constructed and and socially taught
% `self' with a preconceived understanding of behaviour. My point here
% is that `the self' has to be re-constituted anew for each interactive
% context and that the result of this process is a stronger and yet more
% open ended, inclusive, sense of identity. Before looking at how these
% ideas relate to some of the more philosophical aspects of interaction
% and the self I will attempt to contextualize my understanding of `the
% self'.

% In her book Ingrid \citeauthor{monson96} discusses the conversation
% metaphor and its ``structural affinities with interactive
% improvisational process'' (chap. 3, p. 73). In an encounter with
% drummer Ralph Peterson they discuss a particular passage from a
% recording of his group. About a musical discourse between Peterson and
% the pianist, Geri Allen, he is quoted saying: 
% \begin{quotation}
%   ``[...] a lot of times when you get into a musical conversation one
%   person in the group will state an idea or the beginning of an idea
%   and another person will complete the idea or their interpretation of
%   the same idea, how they hear it'' (Peterson (1989b) in \citet[p. 78]{monson96}).
% \end{quotation}
% \citeauthor{monson96} analyzes the comment and states that ``[i]n
% associating the trading of musical ideas with conversation, Peterson
% stressed the interpersonal, face-to-face quality of improvisation''
% (\textit{ibid.}). And later, referring to the same passage: ``These
% moments of rhythmic interaction could also be seen as negotiations or
% struggles for control of musical space'' (p. 80). Conversation in this
% context must be regarded as truly, and only, a metaphor. Musical
% performance has very little to do with verbal discourse and ``nothing
% in common with a text (or its musical equivalent, the score) for it is
% music composed through face-to-face interaction''
% (\textit{ibid.}).\footnote{It should be noted that
%   \citeauthor{monson96} limits the quoted statement to relate to jazz
%   improvisation where I would argue that it holds true for all musical
%   performance.} We would have to move to the more abstract level of
% poetry as exercized by Ezra Pound or Ralph Waldo Emerson in order to
% make sense of a comparison with text, which we will do.

% To allow for someone to complete your idea, inserting their own
% interpretation of the same idea, without feeling the need to correct
% the `erroneous' reading is an aspect of giving up `the self'. And
% further that this can 

% The fact that, in a musical conversation it is perfectly valid to
% complete a co-musician's statement with a deeply personal
% interpretation of the same

% Habermas makes a distinction between instrumental or
% purposive-rational action (`Arbeit') and communicative action
% (`Interaktion') in an attempt to separate that which, according to
% Habermas, Hegel reduced to the general concept of \emph{Philosophie
%   des Geistes}. The instrumental action is based on technical rules

% \begin{quotation}
%   3:e paragrafen, s. 73
% \end{quotation}

% The natural sciences are a refined extrapolation of the instrumental
% action as it is manifested in the for human life so important
% work. ``Purposive-rational action is by its
% structure exercising control'' \citep[p. 63][(\textit{My
%   translation})]{habermas68}.

% \begin{quotation}
%   4:e paragrafen s. 73
% \end{quotation}

% Habermas is here outlining what is later to become the Theory of
% Communicative Action. Though he has been criticized for, and himself
% revised his thinking on, some of these matters \citetext{see
%   \citet{bertilsson83}}\footnote{For an overview of Critical Theory in
%   general and The Theory of Communicative Action in particular see
%   \citet{ericsson01}. An interesting criticism of the philosophical
%   aspect of work in relation to feminist theory is offered by
%   \citet{gurtler05}.} my main concern is the difference between labor
% and interaction. According to \citet{bertilsson83} for Habermas labor
% is a rule based and empirically founded activity and it fulfills the
% social needs for predictability and control. But it has been allowed
% to spread into domains where it should not be dominating at the
% expense of social interaction. ``When predictability and and control
% is allowed to spread at the expense of other domains of knowledge,
% this will happen in a social context where man is meeting the other,
% her neighbour, as an enemy and as object to dominate and rule'' (p.
% 16, my trans.).

% In other words, control belongs to the domain of instrumental action
% which is in every respect different from human interaction which can
% only happen truly if the subjects are mutually respectful of each other.

% \begin{quotation}
%   Medvetandet om mig sj�lv �r ett derivat av att tv� perspektiv
%   korsas. F�rst p� basis av �msesidigt erk�nnande, utvecklas det
%   sj�lvmedvetande, som m�ste bindas vid min spegling i ett annat
%   subjekts medvetande. \citep[p. 183]{habermas68}
% \end{quotation}


% \section{Interactive Music}
% \label{sec:interactive-music}

% What is \emph{interactive music}? Is there a music that is not
% interactive? that does not interact with its environment and its listeners
% in some way? Of the two definitions of the word interactive given to
% us by the Oxford Dictionary above, the second has become so common that,
% when the term \emph{interactive} is used, it is implied that one
% of the ``subjects'' in the interaction is a computer. This is true
% also for music---interactive music implies computer music. But in that
% case, what is the difference between interactive computer music and
% non-interactive computer music? Perhaps we could
% say that, in some sense electro-acoustic music recorded onto and
% played back from a tape, CD, hard drive or similar does not fulfill
% that which one would expect from something which is labeled
% \emph{Interactive}. However, today, most composers play back their
% ``tape'' pieces\footnote{Electro-acoustic music produced in a studio
%   and played back in concert are often referred to as tape music due
%   to the fact that historically these piece were not only stored on
%   tape but also produced by means of cutting and splicing pieces of
%   electro magnetic tape together. I will use the term} performing some
% kind of spatialization or live mixing. In other words, the composer,
% or someone else---an interpreter\footnote, is \emph{performing} the
% work. There is even a competition organized by the Belgian research
% center \emph{Musiques \& Recherches} in ``[l]'interpr�tation
% spatialis�e des {\oe}uvres acousmatiques'' \citetext{\textparagraph\ 1,
%   \citealp{concours-spat}}. This performance is by definition
% interacting with the audience, the performance space, and a number of
% other factors. Further, I would argue that even though there may not
% be any other activities involved than starting the tape---or whatever
% action is required to start the play back of the tape piece---is an
% \emph{interactive} action.

% Interactive music can adopt any one or several of different archetypes
% and paradigms and as we have seen the word \emph{interactive} has many
% meanings and may have many readings. Interactive electronic music
% could potentially be everything from playing back a compact disc on a
% home stereo to the most complex multi computer system imaginable
% controlling any number of real time synthesis engines. Further

% To understand the general term \emph{Interactive Music}
% or, more specifically, although this is in no way a qualified genre,
% \emph{Interactive Computer Music}, i.e. music that involves computers
% at some stage of its production and/or performance and that is
% interactive by nature, one need to have an overview of the history of
% electro acoustic music.

% In his keynote address at ICMC 2007 in Copenhagen John Chowning
% \citet{chowning07} presented notes he made during his early
% experiments with FM synthesis\footnote{Frequency Modulation synthesis,
%   see \citep{chowning73}}

% Robert \citet{rowe} offers a definition of Interactive Music
% Systems:

% \begin{quotation}
%   [...] interactive systems are not concerned with replacing human
%   players but with enriching the performance situations in which
%   humans work. The goal of incorporating human like intelligence grows
%   out of the desire to fashion computer performers able to play music
%   with humans , not for them. A program able to understand and play
%   along with other musicians ranging from the awkward neophyte to the
%   most accomplished professional should encourage more people to play
%   music, not discourage those who already do (\textit{ibid} p. 262)
% \end{quotation}

%%% Local Variables: 
%%% mode: latex
%%% TeX-master: "../MusicComputersInteraction"
%%% End: 
