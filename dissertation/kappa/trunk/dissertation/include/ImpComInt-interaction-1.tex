In the following text I discuss the concept of interaction and its relation to different technologies, the meaning of the different kinds of interaction and (musical) examples of their usage. As with some of the other texts I combine several different modes of reflection in what may best be described as a non-structured manner although it should be clear that the reasons for this organizations are more closely related to method than to lack of method. It is my hope that the reader, in spite of the non-structure, will be able (and willing) to navigate through these thoughts.

Since I first started using computers in my musical practice in 1994 I have been searching for ways in which to use them creatively while improvising as well as composing. In most of my compositions since then the \index{computer}computer is incorporated as a `player' along with acoustic instruments and in recent years I have also begun to perform \emph{on} the computer, as opposed to playing it through another (acoustic) instrument. In the beginning, however, my dominant interest lay in improvising with the computer, in playing saxophone and using that which I played to give the computer the impulse to play with me. Interacting with a computer in the context of (improvised) music raises inevitable questions regarding the interaction, its prerequisites and its needs, and it is not obvious how to pose the questions. The questions may look at the matter from the `outside'---e.g. analysing and evaluating that which I do or try to do, leaning towards established theories---and they may investigate the `object' from the `inside' by turning to the actual \emph{playing}. In this project I use both methods, which obviously have to overlap to a considerable degree. The questions will not always be the same but will vary with context and with time, just as the needs imposed on the interaction will not be the same for every situation. This research project is about raising issues that allow me and others to keep asking and reformulating these questions as new technology becomes available and new knowledge is produced. The somewhat ambiguous genre identifier \emph{interactive computer music} (is there a music which is \emph{not} interactive in some sense?) may easily encompass all of the contexts in which I am and have been active, as well as many others, depicting a music in which computers or other electronic devices or instruments are manipulated in \index{real-time}\index{time!real-time}real-time. The discussion below is an attempt to untangle the meaning and significance of \emph{interactive} and \emph{\index{real-time}\index{time!real-time}real-time} with regard to my own artistic practice and aesthetics.

The first years of playing, in every sense of the word, with the computer, despite the laptop's immaturity and the fact that any `direct manipulation' of audio was out of the question, were very creatively stimulating for me. But, after the initial fascination with the ostensibly interminable possibilities of the computer as `instrument' had worn off, several issues were called into question. What is interaction? What does it mean to interact with a computer? Should the interaction with the computer be different from interacting with a fellow \useGlosentry{glos:musician}{musician}? Is there any reason at all to `play' with a computer (rather than with a fellow musician)? Do I want to play `with' or `on' the computer? When I play, what do I express that has bearing on the current musical expression and, if successfully extracted, can that information be quantified and transformed into meaningful, machine-readable data? This last question may be said to summarise, albeit very briefly, one of the challenges of interactive \index{electro-acoustic music}electro-acoustic music: to develop \emph{interfaces} that may act as intermediaries between performer(s) and computer(s).\footnote{Gestural control, where ``gesture is used in a broad sense to mean any human action used to generate sounds'' \parencite[E. R. Miranda and M. M. Wanderley as quoted in][39]{jensenius08} is a large and very active field of research, which in a sense may be said to be closely related to (traditional) instrument design. The yearly NIME conference (see \url{http://www.nime.org/}) is devoted to the topic and \citeauthor{jensenius08}'s PhD dissertation referenced above discusses gestural control in (electro-acoustic) music. For a general overview of ``interface typologies'', see \cite[135-42]{emmerson07}.} In a situation with one \useGlosentry{glos:musician}{musician} and one computer involved in an interactive performance, \emph{if} it is of interest that the sounds produced `blend', that they may be perceived as having some attribute in common, though this attribute may not necessarily be known or even identifiable, what are the necessary prerequisites that would allow the sounds to `blend'? How can a `common ground', some sensation of unity, or ``\emph{fused sounds}'',\footcite[126]{emmerson07} be established between the actual and the virtual performer and how are the relevant parameters communicated between the different players? These were essentially the questions that led me into this project and, though still highly relevant, the issues of \emph{what} and \emph{how} to communicate with the machine have become subordinate to the question of the interaction itself. By that, I mean to say that a clouded concept of the intentions and purposes of the interaction, of the particular contextual meaning of interaction, is likely to result in an equally clouded interactive system. 

The subject of `blending'---perhaps a more appropriate term is `sonic interaction'---was briefly discussed in the \hyperref[sec:introduction]{Introduction} in the \hyperref[sec:timbremap]{section on \emph{timbreMap}} and the remainder of this section will deal with questions pertaining to the topic of musician-\index{interaction!computer}computer interaction. The question `What is interaction?' is clearly not specific to the field of music. Or, to put it more precisely and less axiomatically, though it is not particular to music, the way interaction is dealt with in the field of music is to a large extent informed by how it is dealt with in other realms, such as the social or the technological. \hypertarget{sec:target:interraction-self-1}{Inversely, any findings} within the field of music made with regard to interaction \emph{may} have general relevance, but may just as well be specific to music, interactive music, or even only the particular case. In other words, as with many concepts there is a terminological fluidity involved which necessitates a discussion involving several aspects of the subject matter. Below, I attempt such an undertaking, also using experiences of interaction accumulated from \emph{Drive}, \emph{etherSound}, \emph{timbreMap} and \emph{Repetition Repeats all other Repetitions}.

\section{Interactive Music}
\label{sec:interactive-music}

Before we discuss the implications of interaction independently of the field of music, the conception of interactive music itself, the focal point of this book, needs to be considered. Not only is the term by itself misleading and ambiguous, in practice, as a genre, it is used to denote so many different kinds of musical activities that the question of interaction loses its meaning.

What then is \emph{interactive music}? That question prompts a rhetorical defence. Is there a music that is not interactive? That does not interact with its environment and its listeners in some way? Depending on the context, commonly when the term \emph{interactive} is used, it is often implied that one of the `subjects' in the interaction is a computer or some other electronic device. This holds true also for music: interactive music implies computer music which is paradoxical because, if interactive music equates with interactive computer music, that implies that computer music, without the interactive prefix, is somehow devoid of interactive elements. At the same time, any human endeavour with a computer today is referred to as human-computer \emph{interaction}. That is, composing at the computer today, almost regardless of how the process is carried out, qualifies as \index{interaction!human-computer}\index{HCI}human-\index{interaction!computer}computer interaction; it is by definition an interaction with the computer. For this reason, strictly speaking, computer music, i.e. non-interactive computer music, can only be music created and performed by computers alone. That is obviously not the case. Not yet. But the question that begs an answer remains: what is it that is so specifically interactive about interactive music, and in what way does its interaction(s) deviate from the way that all music is interactive? On the one hand, the inquiry can be dismissed as uninteresting and irrelevant. If interactive music is a genre identifier, asking about the significance of `interaction' is as meaningful as questioning what is `dead' about `death metal'. On the other hand, one may (continue to) assume that the `interactive' part of the term refers to some aspect of the construction of the music it denotes. My primary concern here, however, is not for the \emph{genre}, but for the ways in which computers allow for interaction in the production of musical content.

Now, it is possible to assert that in some senses \index{electro-acoustic music}electro-acoustic music recorded onto, and played back from, a tape, CD, hard drive or similar does not fulfil the requirements of that which one would expect from something which is labelled \emph{Interactive}. That is to say that the \emph{performance} of the music is expected to allow for interaction in some way that the genre of \index{Fixed media music} fixed media music\footnote{\useGlosentry{glos:electro-mus}Electro-acoustic music produced in a studio and played back in concert is often referred to as tape music because historically these pieces were not only stored on tape but also produced by means of cutting and splicing pieces of electromagnetic tape together. As the storage medium is now more commonly CDs or DVDs the term `fixed media music' is often used.} does not fulfil. The phase of production is likely to be as interactive for that music as for any other pre-composed music and the listening phase, the listener's interaction with the music, should present no difference in kind. Be that as it may, with the delimitation of interactive activities, in addition to pressing the `play' button, which in every respect qualifies as a truly interactive action, different from but akin to musical performance,\footnote{Choosing the right moment for a performance to start is equally important in instrumental and electro-acoustic performances. The exact timing can mean the difference between a great and a not so great performance and for this reason, pressing the play button on the CD player to start the playback of a fixed media piece is part of its performance and, in some limited sense, part of its interpretation.} most composers play back their \useGlosentry{glos:fixed_media}{fixed media} pieces by performing some kind of spatialisation or live mixing. In other words, the composer, or someone else, an interpreter, is \emph{performing} the work. There is even a competition organised by the Belgian research centre \emph{Musiques \& Recherches} in ``[l]'interpr{\'e}tation spatialis{\'e}e des {\oe}uvres acousmatiques''.\footcite[\S~1]{concours-spat} Such a performance is by definition interacting with the audience, the performance space, and a number of other factors in \index{real-time}\index{time!real-time}real-time.

The truth of the matter is that interactive music can (and should) adopt any one or several of different archetypes and paradigms and, as will be shown, the word \emph{interactive} has many meanings and may have many readings. Interactive \index{electro-acoustic music}electro-acoustic music could potentially be everything from playing back a compact disc on a home stereo to the most complex multi-computer system imaginable controlling any number of real-time synthesis engines. The motivation for interactive music comes from the human fascination with, and dread of, technology. Attempting to create a machine that can play music with other humans is an existential quest that interrogates our abilities and know-how (\emph{Can the machine be constructed?}) and also evaluates the uniqueness of human intelligence, endeavour and creativity. What better way is there to examine these aspects of human practice than to test it on music (or chess)? But the greatest fear of all would be to succeed. Jean \citeauthor{baudrillard02} writes about world champion Garry Kasparov's chess game with IBM's Big Blue that ``[man] dreams with all his might of inventing a machine which is superior to himself, while at the same time he cannot conceive of not remaining master of his creations'',\footnote{The essay ``Deep Blue or the Computer's Melancholia'' was written in 1996 after Kasparov's winning over Deep Blue. According to the line of thought presented by \citeauthor{baudrillard02}, Kasparov's subsequent loss in 1997 is nothing short of a terrible defeat for mankind \parencites[160-1]{baudrillard02}. We may, however, turn to Hannah \citeauthor{arendt77} for consolation. In the context of chess playing machines, she points to the important difference between human \emph{intellect} (as needed for chess playing) and human \emph{reason}, and furthermore it is not really the machine, Big Blue, that won, but rather those who programmed it to win \parencite[269]{arendt77}.} about the feeling of frustration owing to the failure, to not being able to replicate intelligence, and the simultaneous comforting of the firm intelligent sovereignty of mankind. Robert \citeauthor{rowe}, at the very end of the first of his two books on the subject of interactive music, \citetitle{rowe}, discusses the frequently aired critique that music machines and computer performers are cold, inhumane, artificial sounding, unnaturally perfect, etc. (all of which are also barriers against the potential threat of the machine). He offers the following reassuring definition of interactive music systems:

\begin{squote} [they] are not concerned with replacing human players but with enriching the performance situations in which humans work. The goal of incorporating human like intelligence grows out of the desire to fashion computer performers able to play music with humans, not for them. A program able to understand and play along with other musicians ranging from the awkward neophyte to the most accomplished professional should encourage more people to play music, not discourage those who already do.\footcite[262]{rowe}
\end{squote}

These thoughts, that musical practice on all levels---from amateurs to professionals---may germinate thanks to the computer's increasingly great propensity to perform, improvise and compose are related to Trevor Wishart's statement that the computer is the piano of the twenty-first century.\footnote{At the opening round-table discussion at the Connect Festival 2006, Malm\"{o} Academy of Music, Sweden, British composer Trevor Wishart spoke of the computer as an instrument that has taken the role the piano had in the twentieth century as the most common musical instrument among middle class people.} If the amateur musician was forsaken by Beethoven and his music, which called for the ``new Romantic deity, the interpreter''\footcite[152]{barthesMus} in the nineteenth century, and the listeners were desolated by the ``double economy'' of the serialists and avant-garde composers,\footnote{The `double economy' refers to the duality particular to a modernist attitude where, while the `business' itself---the composing and staging of musical works---is extremely costly, the movement earns its authenticity from its lack of revenue and profit. One of the points of departure for \citeauthor{mcclary89} is composer Milton \citeauthor{babbitt58}'s \citeyear{babbitt58} article \citetitle{babbitt58} (see also the discussion in Section 3.2). It should be pointed out that it is not the avant garde in general that \citeauthor{mcclary94} is criticising here, but certain specific traits of the sub-culture. \parencite[][61]{mcclary89} For a response to one of the contributions to the heated debate that followed McClary's 1989 essay; see \cite{mcclary94}}\nocite{babbitt58} the thought that the amateur musician is reanimated by the \index{interaction!musical}musical interaction with the computer is riveting. The computer, originally in a sense, the ultimate tool of formalist modernism, now plays the part of a decentralising machine, initialising the revival of the musical amateur in a movement that subjugates the advantages of the mythical expert.

The critic will object to this positive description of the computer as a musical instrument and tool, with the potential to distribute and decentralise music and its constructions, stressing that the sense of `freedom' and `liberation' (from the structures of musical production) offered by technology is only a chimera, that the forces of power and production are also controlling these aspects. That is, if we feel free to create our own, private, soundtrack to our lives, it is because someone wants us to feel free, because there is a great profit to be made (for this someone), lurking behind the corner as a result of our imaginary freedom. Because, as \citeauthor{adorno97} write, a ``technological rationale is the rationale of domination itself'',\footcite[121]{adorno97} a machine such as a computer can never contribute to freedom in the social domain. At the same time, however, if the distinction between the spheres of ``the logic of the work and that of the social system'' has been wiped out, this ``is the result not of a law of movement in technology as such but of its function in today's economy''.\footcite[Without doubt the economy of 1944 is different from the economy of the early twenty-first century, but if it is true then I feel it is safe to assert that it still has some relevance today.][121]{adorno97} In other words, whatever suppressive tendencies technology may show are not necessarily properties of the technology itself, but emanate from economic power structures closely related to technology. Furthermore, though present day ``[a]utomobiles, bombs, and movies''\footcite[The original context for this list of technologies sheds further light on the technological rationale and reveals strong criticism: ``A technological rationale is the rationale of domination itself. It is the coercive nature of society alienated from itself. Automobiles, bombs, and movies keep the whole thing together until their leveling element shows its strength in the very wrong which it furthered''. ][121]{adorno97} to an increasing degree rely on \index{software}software to do their job, a multipurpose machine such as the computer does not do \emph{anything} without \index{software}software---yet the hardware by itself still obviously qualifies as technology. Hence it follows that there is a distinction to be made between technology as hardware and technology as \index{software}software. The former category is signified by physical objects with a clearly defined, and not easily modified, purpose (cars, proprietary \index{software}software, etc.), and the latter by amorphous tools whose shape and usage, in essence, are defined by interaction (open source \index{software}software, truly interactive and \index{cybernetics}cybernetic devices, etc.). Because the computer can host \index{software}software which may be altered, and because the paths of these alterations can be highly distributed---any group of people, with almost any geographical location, may contribute---do not the computer and present-day technology in fact \emph{resist} the single rationale of domination suggested by \citeauthor{adorno97}? Or, to put it another way, regardless of the \emph{\index{Heidegger, Martin!\index{Heidegger, Martin!enframing}enframing}enframing powers} of technology (\hyperlink{par:inter-defin:3}{see Section \ref*{sec:interaction-time:3}} for a discussion of Heidegger's concept of \index{Heidegger, Martin!enframing}enframing) and regardless of the powers of domination induced by economy, is it not true to say that \index{software}software that alters the technology may in fact make it (the hardware) useful in ways that are independent of the technological rationale? If the concept of this rationale is approached freely, that is, somewhat disengaged from the sociological thinking of \citeauthor{adorno97}, I believe it is in the interstices between the two categories of technology just outlined that the technological rationale and the rationale of domination are neutralised. In this small structural space or discontinuity the critics mentioned above may be proved wrong, and it is here that the powers of capitalism and consumerism in actuality \emph{cannot} control how and why the technology is used. Here is where the computer proves that it is indeed a potential vehicle for musical expression, equally stimulating of the amateur and the expert. Moreover, \index{software}software that allows for \index{interaction!musical}musical interaction with the computer, in whatever way that interaction manifests itself and apart from the ways it will promote musical activity, may contribute to a disruption of music as commodity, and move towards music as activity. While Robert \citeauthor{rowe01} makes ``no claims of special social or aesthetic virtue inherent to interactive music''\footcite[6]{rowe01} he also believes that ``if computers interact with people in a musically meaningful way, that experience will bolster and extend the musicianship already fostered by traditional forms of music education. Ultimately, the goal must be to enrich and expand human musical culture''.\footcite[5]{rowe01} If the computer can be made musically useful as an interactive player, I am confident that the ways in which this is achieved, the interactive methodology so to speak, may also be very useful outside the field of music.


%The human-computer relation and the different ways in which interaction may be carried out will be looked at further in the upcoming sections. Time will be given some focus in the upcoming sections---in a sense, the possiblity for real-time computation is the origin of the current notion of human-computer interaction---and I will here provide a very brief background to time as specific to interactive music.

To briefly summarise, interactive music can be many different things. It is interactive just as \emph{any} music or musical activity is interactive. If there are aspects of interactivity that are particular to interactive music in general, they are constituted by the mode of interaction made possible---or, which is sometimes the case, made \emph{impossible}---by the presence of computers and \useGlosentry{glos:electro-mus}{electro-acoustic} instruments. Whereas traditional musical practice may sometimes be assigned a romantically lofty position as an independent art form untouched by worldly reality (a picture obviously not true), computer music is conversely ``often accused of leading us to a day when machines will listen only to machines''\footcite[6]{rowe01} (obviously not true either). In between these opposites of musical mythologies, interactive music has an interesting multifaceted role to play, one which includes, I will argue, a strong social dimension related to the significance and meaning of technology in society and one that may teach us something about the role of technology in the twenty-first century. But this role, the role of technology, also influences the way we deal with music, particularly interactive music. It will be claimed below that \index{interaction!human-computer}\index{HCI}human-computer interaction, i.e. interaction with computers \emph{outside} the realm of music, is synonymous with a mode of \emph{control}---an attempt at curbing the powers of technology to make it useful. As a result, to reinforce the user's sense of being in control, a `cleanliness' in time and space in the intercommunication between those engaged in the interaction is aimed for. The (real-)time aspect is critical in this context because it is only when the control action, the human input, and the acknowledgement of the input, the response from the machine, are contiguous that the user will be able to connect the two. The magnitude of possible ways in which \index{interaction!social}social interaction may be carried out, on the other hand, makes the relation between the action and the feedback---the response---less useful to discuss in terms of milliseconds. In contrast with HCI, the response to, or acknowledgement of, an action may be a silent recognition---a body movement or facial expression---and the esteem given to the actual response is not necessarily measured according to its coinciding in time with the stimulus. \index{interaction!human-computer}\index{HCI}Human-computer interaction and \index{interaction!social}social interaction are two demarcation points that are used to formulate questions concerning the modes of interaction in interactive music.


\section{Multiple modes of interaction}
\label{sec:inter-defin}

In the previous section a perhaps unsuccessful attempt was made to uncover the identity of interactive music. But what is the identity of the word interactive? What are its meanings? If, to begin with, we turn to the Oxford English Dictionary's definition of the adjective, we find two distinct meanings:\footnote{``interactive, \textit{a}.'' The Oxford English Dictionary. 2nd ed. 1989. OED Online. Oxford University Press. 1 Nov. 2007. \url{http://dictionary.oed.com/cgi/entry/50118746}.}
\begin{enumerate}
\label{interaction:item:1}\item ``Reciprocally active; acting upon or influencing each other''.
\label{interaction:item:2}\item ``Pertaining to or being a computer or other electronic device that allows a two-way flow of information between it and a user, responding immediately to the latter's input''.
\end{enumerate}

A general meaning relates to interaction in the social realm, and a more specific, in fact very different, meaning is connected with the technological sphere. Though the second meaning, in most cases, is not at all social, it very often \emph{mediates} \index{interaction!social}social interaction (phone, e-mail, internet) and may also exercise social \emph{control}.\footnote{Although their book, as well as much of their research, focuses on how electronic communication increases participation, \emph{reduces} social control, and provides better conditions for group decisions in the corporate world, Lee Sproull and Sara Kiesler also discuss the new possibilities for exercising social control that come with electronic communication. \cite[See][Chap. 6]{kiesler91}.} The profound difference between the two accounts given by the Oxford Dictionary may be adduced from the verb `influence' of the first definition as compared with `respond' of the second definition. The first, seemingly, denotes a meeting on equal terms, where the influence is mutual; the participants in this kind of interaction are as equally likely to be influenced as they are to influence. The second definition, concerning interaction with electronic devices, is suggestive of an \emph{un}equal relation where only the actions of one of the participants are influential: it is demanded of the electronic device to ``respond \emph{immediately}'' to the user's input. Furthermore, the specifier ``immediately'' indicates that time is of some significance, whereas there is no mention of it in the first definition. These two, still rather unspecified and, unless their contextual and comparative meanings are carefully considered, equally problematic, uses of `interactive' may roughly be said to correspond to previously mentioned signposts of \index{interaction!social}`social interaction' and \index{interaction!human-computer}\index{HCI}`human-computer interaction'.\footnote{Conceivably there are types of interaction and interactive activities that do not fit in with social and computer interaction. \index{interaction!musical}`Musical interaction' could be said to cover aspects of both of these definitions.} I sense, perhaps erroneously, that there is a tendency to confusion, a dissolution of meaning, or uncertainty, between `social' and `computer' interaction.\footnote{In an interview with composer and sound(scape) artist David Dunn, interviewer Ren\'{e} van Peer feels it necessary for Dunn to designate what kind of interaction he is talking about: ``when you use the word `interaction,' you mean something different from what people make it mean in computer-related contexts''. Dunn replies that he thinks interaction is ``largely a misnomer as it's used in computer culture''. According to Dunn the entire concept of \index{interaction!computer}`computer interaction' is a dissolution of the meaning of social interaction. \cite[See][]{dunn99}.} This is not necessarily a bad thing; after all, the meanings of words and terminology are in a constant flux, especially in emerging fields of inquiry. Once we introduce \index{interaction!musical}`musical interaction', however, though likely to overlap, the three types of interaction, `social', `computer' and `musical', are largely asymmetric to each other and in \index{electro-acoustic music}electro-acoustic music, with a strong technological presence, the distinctions---let alone the definitions---are difficult to draw. Let me stress again that I am not interested in an instrumental delineation and separation of the kinds of interaction present in music. The activities in a performance of interactive music belong by necessity not to \emph{one} type of interaction but to multiple kinds simultaneously. Furthermore, we are exposed to different types of interaction in parallel every day in that much of our daily communication and \index{interaction!social}social interaction is mediated by technology; computer interaction mediates \index{interaction!social}social interaction. In order to understand better the possibilities and the limitations of performer-\index{interaction!computer}computer interaction, however, a prerequisite for designing (and using) interactive systems, it is necessary to be able to identify the different modes of interaction that may operate simultaneously.

\label{sec:inter-defin-2}
In his paper on the \index{ontology}ontology of interactive art, Dominic M. \citeauthor{lopes01} discusses ``hyperinstruments'' with reference to George Lewis's \emph{Voyager}\footcite{lewis92} and Todd Machover's \emph{Brain Opera},\footnote{\cite[See][]{paradiso99}. \cite[See also][360-2]{rowe01}.} among others, and states that these kinds of works ``are frequently touted as `interactive,' but this is true in only a trivial and uninteresting sense''. I believe that the argument behind his dismissal is rooted in confusion between different kinds of interactivity. According to \citeauthor{lopes01}, \emph{Voyager} is not interactive art, or at least not interesting interactive art, because playing an instrument, any instrument, is an interactive activity, and ``if any notion of interactivity is to be worth serious attention, it must be more refined a notion than the ordinary concept of interaction''.\footcite[67]{lopes01} Though equating musician-instrument interaction with musician-instrument-\index{interaction!computer}computer interaction (as is the set--up in \emph{Voyager}) may be theoretically useful, it does not tell me as a developer anything about how to develop better interactive systems (it cannot get any better than the trombone?) and it gives me as a listener an incorrect view of the computer (as already an integral part of the instrument and/or the performer?). Rather than refining the ``ordinary concept of interaction'' (whatever that is or may be), as suggested by \citeauthor{lopes01}, I argue that different interactive strategies can be deployed in parallel. When I listen to George Lewis and Roscoe Mitchell improvising together with/in \emph{Voyager}, that is what I hear: I hear that the interaction between the two musicians and the computer is of a different order than the interaction between Lewis and Mitchell. I see in this a multiplicity of interactive modes quite common in many kinds of music. \hypertarget{sec:inter-defin:monson}{Ingrid} \citeauthor{monson96}, to whom we will \hyperref[sec:mult-music-inter]{return later}, points to how, in a jazz rhythm section, the groove is (but) one `interactional text'\footnote{``Interactional text'' is a term \citeauthor{monson96} has borrowed from anthropologist and linguist Michael Silverstein, in this context denoting the ``interactive construction of the musical surface''.\cite[189]{monson96}.} and the relation between this and the soloist is another, dynamically linked to the former (and these are just two of many possible \index{interactional texts}\index{Monson, Ingrid!interactional texts}\index{interaction!interactional texts}interactional texts).\footcite[188]{monson96} It appears to me very difficult to state unambiguously that one of these `texts' is more ``refined'' than the other, which is the evaluation \citeauthor{lopes01} calls for. 

%GESTELL HINDRAR OSS FRÅN ATT SE TEKNINKENS POTENTIAL OCH TVINGAR OSS ATT ANVÄNDA OSS AV DE KOMMUNIKATIONSMODELLER SOM UTGÖR TEKNIKENS DELAR SNARARE ÄN ATT UTNYTTJA DEN FULLA POTENTIALEN.

\label{sec:inter-defin:par3}
\hypertarget{par:inter-defin:3}{As was} \hyperlink{sec:target:interraction-self-1}{suggested above}, with regard to interaction what may be useful in one field of inquiry is not necessarily applicable to another. The affinity between interaction as understood within the field of music and the arts, and interaction in the `extra-artistic' world is not necessarily conspicuous. Nevertheless, both music and the arts in general are often mentioned as sources of knowledge for the disentanglement of human-technology relations. For Martin \citeauthor{heidegger93}, art plays a central role in the unfolding of technology. In his seminal essay \citetitle{heidegger93}, he sees in the arts a possible alternative to the technological attitude, a counterbalance to the `\index{Heidegger, Martin!enframing}`enframing'' powers of technology and its predisposition for ceaseless quantitative categorisation; a reflection of the human will to control nature. Towards the end of the essay he concludes that ``essential reflection upon technology and decisive confrontation with it must happen in a realm that is, on the one hand, akin to the essence of technology and, on the other, fundamentally different from it. [\ldots] Such a realm is art''.\footcite[340]{heidegger93} Before continuing I should express a few reservations with respect to \citetitle{heidegger93}: (1) I have no intention of attempting to unravel all aspects of this wonderfully intricate text. In a sense there is enough material in it, and in \citeauthor{heidegger93}'s relation to technology, work and production at large, for a thesis entirely devoted to the subject.\footcite[An example of this is Zimmerman's well-informed book on Heidegger and modernity:][]{zimmerman90} Instead I will allow myself to use it, and relate to it somewhat more freely. (2) It should be noted that `technology', as the word is used by \citeauthor{heidegger93}, alludes to technology in a much wider sense than what I am concerned with here, but also, at the same time, has a meaning more contracted, excluding present-day distributed (information) technologies.\footcite[See][26-9]{kemp91} (The essay was written in 1954 when the computer primarily existed as an abstract thought.) (3) Because technology, according to \citeauthor{heidegger93}, ``is not equivalent to the essence of technology'',\footcite[311]{heidegger93} which is to say that that which is akin to its \emph{essence} although it may be, is not necessarily closely related to the \emph{representation} of the technological. In other words, what may first look like an acclamation of the \index{digital}digital arts (at the time non-existent) as a discipline that seemingly fulfils the aspects of closeness to and distance from technology is, unsurprisingly, in reality meant to be something much more abstract. (4) Finally, and most interestingly in the present context, when \citeauthor{heidegger93} points to the indispensable role art may play in the understanding and unfolding of technology, regardless of its dangers and frenziedness, he emphasises that, as a prerequisite, the ``reflection upon art [\ldots] does not shut its eyes to the constellation of truth, concerning which we are \emph{questioning}''.\footcite[340]{heidegger93} If for the sake of argument the difficulties in clearly separating \emph{technology} from \emph{music} and \emph{\index{interaction!human-computer}\index{HCI}human-computer interaction} from \emph{\index{interaction!musical}musical interaction} are dissassembled, then in a free interpretation (an improvisation?) of \citeauthor{heidegger93}, the quote above will serve as a reminder that, whatever fascination may be experienced with the technological aspects of interaction, when we deal with interactive music, \emph{music} and the way it is revealing itself should be a priority: only then will the right questions be asked.

%But also the field of science that, according to \citeauthor{heidegger93} is both contingent on, and assembler of, technology,\footcite[319-20]{heidegger93} 

Judging from the many examples of art referenced in discussions relating to computer usability, the call to relate to artistic practice and the field of art as a method for evaluating interactive systems and for developing and enhancing interactive models has survived. It is in fact used, albeit in a limited range, perhaps not to increase the creative potential of technology but at least to reduce its \index{Heidegger, Martin!enframing}`enframing' aspects, or, as \citeauthor{kirlik04} write, in order to avoid ``the ever-increasing use of proceduralization and automation in sociotechnical systems''.\footcite[629]{kirlik04} In their attempt to bring about ``[a] common framework for studying perception and performance in both human-technology interaction and music'' they conclude, with a reference to Leonard \citeauthor{meyer56},\footcite{meyer56} that ```deviations' from the written score and oral tradition'' are essential to the practice of music.\footnote{\citeauthor{kirlik04} (and \citeauthor{meyer56}) are only concerned with Western, score-based music, but the point they are making---that improvisation is an important aspect of musical practice---may be asserted for most musical practices. \cite[See also][]{benson03}.} Whereas in music, there is a large component of freedom and room for individual decisions, according to the experience of \citeauthor{kirlik04} ``design and training in many sociotechnical systems proceed [\ldots] as if `doing it by the book' or working `like a machine' were admirable qualities. Experienced human operators know otherwise, and in their better moments, so do engineers, researchers and practitioners in human-technology interaction.''\footcite[629]{kirlik04} A \useGlosentry[]{glos:sociotech}{sociotechnical} system and interactive \index{electro-acoustic music}electro-acoustic music share, at least to some limited extent, the interrelation between social/musical and technical aspects. Hence, both fields have in common the need to investigate the relation between the different modes of interaction operational within the system. But herein lies the danger of committing the mistake portrayed by so much science fiction literature: rather than attempting to match technology with humans, the human is mechanised to accommodate technology. Somewhat pointedly the phenomenon can be expressed like this: because humans are noisy and irrational they pose a problem to the machine, a problem whose solution can only be human re-factoring. This machine-centric view of interaction where ``the human is often reduced to being another system component with certain characteristics, such as limited attention span, faulty memory, etc.'',\footcite[Bannon, L. as quoted in][27]{kuutti96} may be necessary in the design of control systems for nuclear power plants (although I would like to think otherwise) but, unless explicitly desired, is likely to cause problems in a musical context. In a PhD dissertation on using the \emph{flow} heuristic when building GUIs for web-based applications, \citeauthor{thomassen03} also mentions music as a means``to fully research the applicability of the heuristic. The major disciplines are the field of social sciences such as psychology and cultural studies, but also the field of the arts in particular music and fine arts''.\footcite[239]{thomassen03}

All of these examples refer to music as a source containing possible clues to the development of interactive systems and their interfaces. The tendency can be summarised thus: because computers are `cold' and `insensitive' and music is `warm' and `emotional' a crossbreed should allow for a more dynamic and less strained \index{interaction!human-computer}\index{HCI}human-computer interaction that is not \emph{only} dictated by the social limitations of the technology. My point here is not to criticise these works but to attempt to understand the complex interrelations between different interactions. Technology as well as art is constructed by how we think about it and, perhaps, if we think of technology through art, as suggested by \citeauthor{heidegger93}, an alternate perspective may reveal itself. But we may equally well come up with an art that is more technological because, in the terminology of Heidegger, the \emph{\index{Heidegger, Martin!enframing}enframing} of technology is already a fact and as such it limits our possibilities to perceive the world. The ``essence of technology is in a lofty sense ambiguous'', however, which is to say that the very meaning of \index{Heidegger, Martin!enframing}enframing is ambiguous and such ``ambiguity points to the mystery of all revealing''.\footcite[338]{heidegger93} 

%I will return to \citetitle{heidegger93} and the notion of enframing later... 

If machine-like operation, repetition without difference, without change, are, as \citeauthor{kirlik04} write ``admirable qualities'' then not only the object for which we assert this judgement, but also our interactions with it, are likely to be affected. But, the solution is not, I believe, as simple as making the technology more `human' (which is not by no means a simple task), nor is it to reduce resistance at all costs, to make the interface to the technology `transparent' to the user (although this is exactly what I thought when I started this project). I acknowledge the need for ordinary computer users to have a UI that makes their interactions with the computer easy and effortless.\footnote{It should be noted that I do not think that this stage has been reached in the operating systems and programs offered by commercial companies today. In a large survey (6.000 participants) produced by one of the largest national trade unions for officials in Sweden (Sif), though 80\% felt the IT systems used were invaluable in their day-to-day work and in their customer relations, a stunning 50\% felt the software negatively influenced the way they worked, and 50\% felt the interfaces and help functions were defective (bad usability). \cite[See the (Swedish) report][]{sif07}.}  But those needs, and the thinking and research that have led to the solutions of the problems addressed by those needs, are not necessarily useful when we move from the domain of production and corporate efficiency to the abstract domain of artistic practice. That the latter domain has been used to inform or inspire the design of a UI, as \citeauthor{thomassen03} and Kirlik \& Maruyama suggest should be done, will in this regard not make a decisive difference. 

\hypertarget{sec:inter-defin:par4}{One may argue that the difference} between interacting with a computer and interacting with another human being is so immense that the discussion of this difference is superfluous and uncalled for, that the prerequisite for human-human interaction is that both parties exhibit some kind of sensible notion of intelligence which, by definition, the computer will never (ever?) come close to. Therefore, HCI is, and has to be, about control, about making technology useful through \index{interaction!as control}\index{interaction-as-control}`interaction-as-control', and this is a mode of interaction that is of a different order from human-human interaction. There are at least two sides to this issue: (1) The first belongs to the general field of HCI where there is a tendency to limit the thinking about HCI to ``microlevel interactions between programmers or users and computers. The broader social forces and structures that constrain such interactions and are themselves reproduced and molded by microlevel events are often left unexamined''. Not only will this contribute ``to a naive image of human-computer interaction as narrowly technical and as a problem of cognitive optimization'',\footcite[p. 325]{engestrom96} I believe it will also in effect be at risk of influencing the way we interact with other humans. In a debate on \useGlosentry{glos:agent}{intelligent agents}, computer scientist, composer, visual artist and author Jaron Lanier is concerned that
``People will gradually, and perhaps not even consciously, adjust their lives to make agents appear to be smart. If an agent seems smart, it might really mean that people have dumbed themselves down to make their lives more easily representable by their agents' simple database design''.\footnote{\cite[\textparagraph~3]{lanier96} Throughout his writings, Lanier makes numerous remarks on the dangers of considering computers as possessing intelligence precisely for the reasons here mentioned. ``What starts as an epistemological argument quickly turns into a practical design argument. In the Turing test, we cannot tell whether people are making themselves stupid in order to make computers seem to be smart. Therefore the idea of machine intelligence makes it harder to design good machines'' (\cite[\textparagraph~5]{lanier1000}). Though I sympathise with this and acknowledge the problem, I think Lanier employs a too narrow and binary reading of intelligence. The political as well as personal impact that technology, and in particular information technology, has on our lives should not be understated, but neither should the enduringness of human intelligence.} Similarly, rather than making HCI more like human-human interaction, there is a risk that we instead do it the other way around: assume properties of HCI in our human interaction. (2) The second aspect is closely related to the core of this project. If we differentiate HCI from human-human interaction---understand them as two distinct and only remotely related modes of activity---how should we understand interactive music, or any other form of interaction with a computer within the spheres of artistic practices? 

\label{sec:inter-defin:par5}
In the mid-1990s, the notion of the `\useGlosentry{glos:agent}{intelligent agent}' (which is what Jaron Lanier opposes above) was seen as an alternative to the tool as ``the prevailing metaphor for computers''.\footcite[67]{isbister95} The personal computer could now easily communicate with other computers, other users, keep track of things for its user, perform many things simultaneously: ``Such an object seems inherently different than a hammer or wrench---it has active qualities. It acts on one's behalf---it is an agent''.\footcite[p. 68]{isbister95} Multimedia expert and computer visionary Nicholas Negroponte envisioned that ``[w]hat we today call `the agent-based interfaces' will emerge as the dominant means by which computers and people talk with one another''.\footcite[p. 102]{negroponte95}  In short, and somewhat simplified, this means: rather than you telling the computer what to do, it would anticipate what you wanted to get done and ``emulate human action, assistance, and communication''.\footcite[83]{isbister95} As with so many other great ideas, the prospect of intelligent agents has been spoiled by commercialism and personally I will not shed any tears if I never receive another e-mail containing `intelligently' selected shopping suggestions.

\label{sec:inter-defin:par6}
\hypertarget{sec:target:inter-defin:par6}{Nevertheless}, the concept of `software \useGlosentry{glos:agent}{agents}' holds within it the possibility of rethinking the idea of interaction with the computer. As \citeauthor{isbister95} observe: ``Most forms of agent are all about the user relinguishing (\emph{sic}) control of the computer for a time''.\footcite{isbister95} To be willing to relinquish control is the beginning of an understanding of HCI that also includes elements usually seen to pertain to the domain of \index{interaction!social}social interaction. To give up personal control to a machine may be a frightening idea for many, fueled by horrifying science fiction descriptions: ``the cataloging of the individual, the processing of delocalised data, the anonymous exercise of power, implacable techno-financial empires [...]'',
\footcite[p. 117]{levy97} \footnote{Though to me, judging from the popularity of online communities such as Facebook, or Google for that matter, it seems like the individual of the twenty-first century is quite willing to allow the cataloguing of the identity.} but \citeauthor{levy97} reminds us that ``a virtual world of collective intelligence could just as easily be as replete with culture, beauty, intellect, and knowledge, as a Greek temple [\ldots]''\footcite[118]{levy97}:
\begin{squote}
A site that harbors unimagined language galaxies, enables unknown social temporalities to blossom, reinvents the social bond, perfects democracy, and forges unknown paths of knowledge among men. But to do so we must full inhabit this site; it must be designated, recognised as a potential for beauty, thought, and new forms of social regulation.\footcite[118]{levy97}
\end{squote}
To ``fully inhabit'' we need also to invent new modes of interaction.

\label{sec:inter-defin:par7}
As regards the two definitions of \emph{interaction} given by the Oxford English Dictionary I mapped two corresponding modes of interaction. I have already mentioned that these two modes, `social' and `computer' interaction, do not by any means form an exhaustive list. If anything they are dynamically defined coordinates on a plane of possible interactive strategies, a plane that can hold many other kinds of interactivities constantly redefining and repositioning each other. The relation between these two modes is not easily defined and it involves a tension that makes information about one of the modes not necessarily useful or valid in the other.

% It appears to me that the relation and distinction between these two modes are not always effortless, and information about one mode may or may not be valid in another. 

% It appears to me that the dynamics of these modes of interaction is very complex and non-linear and information about one mode may or may not be valid in another.

% relation 
% and distinction between these two modes are not always effortless, and information about one mode may or may not be valid in another.


\section{Computer interaction and time}
\label{sec:interaction-time}

One aspect of the significance of time in the discussion of \index{interaction!human-computer}\index{HCI}human-computer interaction is to be found in the history of computing and the fact that the notion of ``responding immediately'' that we find in the \hyperref[interaction:item:2]{OED's definition} of human-technology interaction is relatively new. That which we today take for granted when we work or play with our computers---to be able to interact with them in \index{real-time}\index{time!real-time}real-time is the prerequisite for `immediate response'---was obviously not always possible. Below I give a short account of part of the history of \index{interaction!human-computer}\index{HCI}human-computer interaction and some notes on one aspect of time in \index{interaction!musical}musical interaction with a computer.

\label{sec:interaction-time:2}
About 35 years ago, computing was primarily a non-interactive activity done `off-line'. Although some operations may still not be immediate on a modern computer system, at that time it was not unusual for the output from a system to arrive hours after the time at which input was given. This was particularly true of computationally intense tasks such as sound synthesis, and special patience was required of the early pioneers of computer music in the  1960s and early 1970s. Max Mathews, one of the groundbreakers of computer music, attests that ``[a] high-speed machine such as the I.B.M. 7090 [\ldots] can compute only about 5000 numbers per second when generating a reasonably complex sound''.\footcite[553]{mathews63} With a sampling rate of 20kHz (which is less than half the bandwidth of standard CD quality) a one-second-long five-note chord would take 20 seconds to compute and even longer, were the sound producing the chord to be `interesting': ``complexity of the instrument-unit [the part of the computer program that represents the `instrument' played] is paid for both in terms of the computer time and in terms of the number of parameters the composer must supply for each note. In general, the complicated instrument-units produce the most interesting sounds, and the composer must make his own compromise between interest, cost, and work''.\footcite[555]{mathews63} Even more frustrating, then, if the sound turned out to be useless and the process had to be restarted. In his keynote address at \useGlosentry{glos:icmc}{ICMC} 2007 in Copenhagen, John Chowning\footcite{chowning07} presented notes he had made during his early experiments with FM synthesis.\footcite[Frequency Modulation synthesis. See][]{chowning73} Because calculating the actual sound, i.e. the result of the FM-synthesis, was so time-consuming, and access to the computer `mainframe'\footnote{A mainframe computer is a large server-like central computer.} so limited, it was more economic for Chowning to estimate the effect of different FM parameter settings `by hand', i.e. with pen and paper, than to waste valuable computer access time with `experimental' attempts.\footnote{I find this example  particularly intriguing considering the enormous impact Chowning's research had on the electronic music scene. The commercial application of his work with FM-synthesis, as manifested in the Yamaha DX series synthesisers, gave a wide range of musicians and composers the chance to interact with \index{digital}digital sound synthesis in \index{real-time}\index{time!real-time}real-time.}

\label{sec:interaction-time:3}
Whereas today, most of the computer mediated work is done at the computer itself (I am not writing this manuscript by hand only to enter it into the computer at a later stage), before the advent of the personal computer this was not always the case. For Chowning and others, at this time the computer was but one step in the process. The preparatory work such as writing the code constituting the instructions for the computer, and pre-calculating parameters, was likely to be done at a location different from where the data was entered. Hence, not only was there a displacement in time but also in space. Furthermore, the physical computer, the mainframe, could well be dislocated from the terminal, or `teletype',\footnote{A `teletype' is a typewriter-type predecessor to the keyboard/screen combination, which was not widely used until the late 1970s.} where the data were entered. The following is an account of the multiplicity of processes involved by American (post)cyberpunk writer Neil Stephenson, who describes doing his homework during a high school computer programming class:
%
\begin{squote}
[M]y interaction with the computer was of an extremely formal nature, being sharply divided up into different phases, viz.: (1) sitting at home with paper and pencil, miles and miles from any computer, I would think very, very hard about what I wanted the computer to do, and translate my intentions into a computer language---a series of alphanumeric symbols on a page. (2) I would carry this across a sort of informational cordon sanitaire (three miles of snowdrifts) to school and type those letters into a machine---not a computer---which would convert the symbols into binary numbers and record them visibly on a tape. (3) Then, through the rubber-cup modem, I would cause those numbers to be sent to the university mainframe, which would (4) do arithmetic on them and send different numbers back to the teletype. (5) The teletype would convert these numbers back into letters and hammer them out on a page and (6) I, watching, would construe the letters as meaningful symbols.\footcite[chap. `Bit-Flinger', \S~4]{stephenson99}
\end{squote}
%
Despite the obvious disengagement involved in Stephenson's work with the computer (he was never in physical contact with the computer or mainframe), he relates to it as ``interaction'', almost as if he was actually referring to a kind of \index{interaction!social}social interaction, albeit a formal one.\footnote{It should, however, be pointed out that Stephenson's book \emph{\citefield{stephenson99}{title}} is a critique of the modern development of \index{GUI (Graphical User Interface)}Graphical User Interfaces showing how, according to him,  they detach the user from the computer.} Formal, because of the many different steps of preparation he had to go through, each one by itself absolutely essential to the success of the final result. Whatever Stephenson's sensation of interaction, however, by his account the computer he worked at cannot be said to have satisfied a notion of `immediate response'. The \index{real-time}\index{time!real-time}real-time interaction hinted at by this definition was not fully made possible until the early 1980s, incidentally coinciding with IBM's launch of the Personal Computer.\footnote{The IBM \index{PC}\index{Personal Computer}PC was soon to be followed by the Apple Macintosh.} Along with the \index{PC}\index{Personal Computer}PC came the first incarnations of the modern operating system as well as `new' input devices (mouse, display and keyboard). With these inventions the demarcation line between the computer and its user, both in time and in space, began to dissolve and it gave rise to a different idea of \index{interaction!human-computer}\index{HCI}human-computer interaction.

If the response time (the time that elapses, for example, from the point a key is pressed until the response is perceived by the initiator) of the systems with which \citeauthor{stephenson99} and \citeauthor{chowning07} interacted was measured in hours or days, the unit for measuring response times in present-day computers and operating systems is typically \emph{milliseconds}. The response time in an interactive system in music is commonly referred to as the \useGlosentry{glos:latency}{\emph{latency}} and its value and impact are highly dependent on the type of interaction. If the interaction is performative in the sense of playing an instrument, such as a synthesiser keyboard, the endurable delay time between a pressed key and a perceived sound is comparatively very low. Those of us who have used computers and computer-based synthesisers and instruments for some years appreciate the nightmare sensation of a latency of 20 ms or more. If, on the other hand, the interaction is of the kind where the system is analysing a performance in real time and `composes' an accompaniment or a counterpoint, to be played back at a later point in time, depending on the musical and aesthetic requirements, latency may not be a big issue, or at least one that can be handled and compensated for.

Time and music are intimately coupled together but it is difficult, if at all possible, to discern what `good' timing is, and equally difficult to determine the significance of `exactness' in time. (A more elaborate discussion on these issues is to be found in \hyperlink{sec:human-comp-inter:7}{Section \ref*{sec:time-interaction}} and \hyperlink{sec:target:time-inter-revis}{Section \ref*{sec:time-inter-revis}}.) Interestingly, it is difficult to find studies on latency and timing (in interactive music, that is), as observed by \citeauthor{brandt98} in their article on latency and computer operating systems.\footcite{brandt98} Concerning electronic instrument response times they write that:
\begin{squote}
There do not seem to be published studies of tolerable delay, but our personal experience and actual measurements of commercial synthesiser delays indicate that 5 or maybe 10 ms is acceptable. This is comparable to common acoustic-transmission delays; sound travels at about 1 foot/ms.
\end{squote}
Imprecise as this may be, it gives a hint of the sensitivity of human audio perception as well as a hint as to the biological reasons for this sensitivity: it is used to estimate the distance to sound sources. Any musician can learn how to play with a delay exceeding 10 ms, at least if the \useGlosentry{glos:latency}{\emph{latency}} is consistent,\footnote{The church organ is an example of a mechanical instrument where it is necessary for musicians to adjust their musical timing according to the properties and delay of the particular instrument.} but a genuinely problematic situation occurs when the latency does not behave linearly across the range of the instrument. A typical example is the pitch-to-MIDI converter.\footnote{A pitch-to-MIDI converter calculates an estimation of the fundamental frequency in an audio signal.} Depending on the quality and the properties of the instantaneous audio signal that is being analysed, the fundamental estimation may take one or several audio buffers to output its result. If the analysis buffer is $256$ samples and the sampling rate is $44100$, then the delay introduced for each buffer is $\sim5.805 ms$\footnote{$\frac{256}{44100}\approx0.005805$} Also, the pitch-range of the audio signal affects the time it takes to analyse its fundamental, even if this delay is generally predictable.

The sensitivity of expectation does not seem to be limited to the gesture/listening relation. Consider the annoying distortion of perception that occurs when an echo of one's own voice is heard when one is talking on the phone, even commoner now with the frequent use of VoIP, or the situation that occurred in the days of analogue tape recorders in which the record and playback tape heads were displaced by a few inches. Listening to the recorded version of one's own voice in headphones while simultaneously recording it made it very difficult to talk. The recorded voice would be delayed by perhaps 0.3 seconds, enough to create a breach between perception and expectation so grave that speaking correctly became impossible.

\index{interaction!human-computer}\index{HCI}Human-computer interaction for some years has predominantly equalled \index{real-time}\index{time!real-time}real-time interaction. From the definition of interaction \hyperref[interaction:item:2]{given above} one may presume that \index{real-time}\index{time!real-time}real-time is a prerequisite, that the response times necessarily must be low, in accordance with how, when I move my mouse, the pointer on the screen mimicks my movements; it is my body's extension in the \index{virtual}virtual, a blind man's stick, a prosthesis. The deception will only work if the interaction takes place in real or close to \index{real-time}\index{time!real-time}real-time. But consciousness and the powers of expectation in relation to sound also appear to employ a great sensibility also outside the field of \index{interaction!human-computer}\index{HCI}human-computer interaction, displaying a very low acceptance for \emph{\useGlosentry{glos:latency}{latency}}. Hence, two issues are distinguishable here: (1) the \index{real-time}\index{time!real-time}real-time interaction specific to \index{interaction!human-computer}\index{HCI}human-computer interaction and (2) the \index{real-time}\index{time!real-time}real-time of interactive music which by implication is related to \index{interaction!human-computer}\index{HCI}human-computer interaction but which is also closely linked with the apparent sensibility to latency of sound as independent of any technology involvement. These two threads will be further elaborated in the following sections, particularly in \hyperlink{sec:human-comp-inter:7}{Section \ref*{sec:time-interaction}} and \hyperlink{sec:target:time-inter-revis}{Section \ref*{sec:time-inter-revis}}. If this section introduced the topic of \emph{time} and computing, the next section will immerse itself 'in the concept of \emph{space} and computation; the intermediate space in which interaction is played out, the interface.

%%% Local Variables: 
%%% mode: latex
%%% TeX-master: "../ImprovisationComputersInteraction"
%%% End: 