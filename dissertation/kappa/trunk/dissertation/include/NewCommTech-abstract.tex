\begin{abstract}
  In this article we discuss the notion of `interaction' and
  `participation' and `the public' in artistic work, specifically
  within the context of the exhibition The Invisible Landscapes
  (curated by Miya Yoshida, Malm\"{o} Konstmuseum, 2003) and
  etherSound (created by Henrik Frisk), a sound installation displayed
  in that exhibition. In this work the audience is invited to
  participate in the creation of new sound events by sending text
  messages from their mobile phones. Thus, our discussion is focused
  on the space and the mode of participation opened up by new
  communication technology. Based on our experiences of that project,
  we introduce and explain what we believe are relations of creative
  production and a different kind of creativity that may emerge from
  active interaction. We also attempt to describe what we believe an
  implementation of active public participation can lead to.

  We are combining two modes of thinking in this article - one is
  inspired by discourse of the cultural theories and the other is the
  reflection upon our experience of the event. The latter is by
  definition rather subject-centered and expansive based on individual
  observation. We examine and analyze the phenomenon of
  `participation' whilst playing \emph{etherSound} as a process of creative
  production, and seek to reflect upon the power of the co-operative
  practice and its relation to participation and creativity.
\end{abstract}

%%% Local Variables: 
%%% mode: latex
%%% TeX-master: "../stim_create"
%%% End: 
