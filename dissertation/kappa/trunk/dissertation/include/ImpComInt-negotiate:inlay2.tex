\hspace{0.5\linewidth}
\begin{wrapfigure}{r}{0.5\linewidth}
  \begin{minipage}[H]{1\linewidth}
    \advance\rightskip-1cm
     \vspace{0.5cm}\label{fig:inlay2}
\musicannot{
Marcel Duchamp, introduced the notion of the ``personal 'art coefficient' '' in a discussion concerning the creative act. Specifically Duchamp was referring to the immanent processes, those in which the artist alone is involved and which ultimately lead to ``art in the raw state---\`{a} l'\'{e}tat brut'' and, in short, it constitutes the difference between the artistic intention and its realization. The gap goes unnoticed by the artist and it represents his (or her) inability ``to express fully his intention''\footcite{duchamps57} Duchamp concludes that ``the personal 'art coefficient' is like an arithmetical relation between the unexpressed but intended and the unintentionally expressed.''\footcite{duchamps57} It is primarily the description of the art coefficient as a \emph{relation} that I find interesting. What could the `interaction coefficient' constitute? The notion of the ``unintentionally expressed'' as having a relation to the ``unexpressed but intended'' could be directly transferred to interaction: In any interaction there are unintended outcomes (\emph{Why did the computer go down?} or \emph{Why did you sound so angry?}) as well as intended but tacit (\emph{How come the computer doesn't understand what I want to do?}). What can the interaction coefficient tell us about an interaction? What, if the interaction is part of an artistic context, what would the relation between the art coefficient and the interaction coefficient be? Would they be the same or different?
}
  \end{minipage}
\end{wrapfigure}
%%% Local Variables: 
%%% mode: latex
%%% TeX-master: "../ImprovisationComputersInteraction"
%%% End: 
