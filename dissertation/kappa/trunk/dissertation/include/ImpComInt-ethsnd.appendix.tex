\small
\begin{verbatim}
  sr        =           48000
  kr        =           4800
  ksmps     =           10
  nchnls    =           2

            maxalloc    10, 10
            maxalloc    11, 7
            maxalloc    1, 4
            maxalloc    2, 4
            maxalloc    3, 4
            maxalloc    4, 4
            maxalloc    5, 4
            maxalloc    6, 2

;;; G L O B A L   I N S T R U M E N T    1
;;; Global control instrument for random jitter

instr 1

  k50       randi                   .01, 20, .8135 
  k60       randi                   .01, 9, .5111
  k70       randi                   .01, .8, .6711
  gkjitter  = (k50 + k60 + k70) * p4

endin 

;;; G L O B A L   I N S T R U M E N T    2
;;; Global control instrument for phase envelope
;;; This value controls the position in the sample buffer to 
;;; read from
;;; Limit between 0. and 0.1.

instr 2
            ; **************************************
            ;; ARGUMENTS:
            ;; p2 -> start
            ;; P3 -> length
            ;; p4 -> init value
            ;; p5 -> bp1 time (fraction of p3)
            ;; p6 -> bp1 value
            ;; p7 -> bp2 time (fraction of p3)
            ;; p8 -> bp2 value
            ;; p9 -> bp3 time (fraction of p3)
            ;; p10 -> bp3 value
            ; **************************************
  gkphs1     linseg      p4, p3*p5, p6, p3*p7, p8, p3*p9, p10
  gkphs2    =           gkphs1+.0001

endin 

;;; G L O B A L   I N S T R U M E N T    3
;;; Global control instrument for duration envelope
;;; This value controls the duration of the grains -
;;; values should be between 1.0 and .02, but keep 
;;; low for safety.


instr 3
            ; **************************************
            ;; ARGUMENTS:
            ;; p2 -> start
            ;; p3 -> length
            ;; p4 -> init value
            ;; p5 -> bp1 time (fraction of p3)
            ;; p6 -> bp1 value
            ;; p7 -> bp2 time (fraction of p3)
            ;; p8 -> bp2 value
            ;; p9 -> bp3 time (fraction of p3)
            ;; p10 -> bp3 value
            ; **************************************
  gkdur     linseg      p4, p3*p5, p6, p3*p7, p8, p3*p9, p10
  gkdec     =           gkdur

endin 

;;; G L O B A L   I N S T R U M E N T    4
;;; Global control instrument for gliss envelope
;;; This value controls the gliss parameter. Formant rises as  
;;; parameter rises (between 0 and 5 ?)

instr 4
            ; **************************************
            ;; ARGUMENTS:
            ;; p2 -> start
            ;; p3 -> length
            ;; p4 -> init value
            ;; p5 -> bp1 time (fraction of p3)
            ;; p6 -> bp1 value
            ;; p7 -> bp2 time (fraction of p3)
            ;; p8 -> bp2 value
            ;; p9 -> bp3 time (fraction of p3)
            ;; p10 -> bp3 value
            ; **************************************
  gkgliss     linseg      p4, p3*p5, p6, p3*p7, p8, p3*p9, p10

endin 

;;; G L O B A L   I N S T R U M E N T    5
;;; Global control instrument for fundamental envelope
;;; This value controls the fundamental or density of 
;;; the granulation.

instr 5
            ; **************************************
        	;; ARGUMENTS:
            ;; p2 -> start
            ;; p3 -> length
            ;; p4 -> init value
            ;; p5 -> bp1 time (fraction of p3)
            ;; p6 -> bp1 value1
            ;; p7 -> bp1 value2
            ;; p8 -> bp2 time (fraction of p3)
            ;; p9 -> bp2 value1
            ;; p10 -> bp1 value2
            ;; p11 -> bp3 time (fraction of p3)
            ;; p12 -> bp3 value1
            ;; p13 -> bp1 value2
            ; **************************************
  kf1       linseg      p4, p3*p5, p6, p3*p8, p9, p3*p9, p12
  kf2       linseg      p4, p3*p5, p7, p3*p8, p10, p3*p9, p13
  gkfund1   =           kf1+gkjitter
  gkfund2   =           kf2+1+gkjitter

endin 

;;; G L O B A L   I N S T R U M E N T    6
;;; Global control instrument for formants in 
;;; Instrument 11 notes. Provides formant frequency,
;;; amplitude, bandwidth for a male voice.

instr 6
            ; **************************************
            ;; ARGUMENTS:
            ;; p2 -> start
            ;; p3 -> length
            ;; p4 -> starting formant (0=<p4<=5)
            ;; p5 -> ending formant (0=<p5<=5)
            ; **************************************
  kndx      linseg      p4, p3, p5

  gkf1      tablei      kndx, 34
  gkf2      tablei      kndx, 35
  gkf3      tablei      kndx, 36
  gkf4      tablei      kndx, 37
  gkf5      tablei      kndx, 38
        ;; Formant amplitude interpolation
  gkfamp1   tablei      kndx, 44
  gkfamp2   tablei      kndx, 45
  gkfamp3   tablei      kndx, 46
  gkfamp4   tablei      kndx, 47
  gkfamp5   tablei      kndx, 48
        ;; Formant bandwith interpolation
  gkbw1     tablei      kndx, 54
  gkbw2     tablei      kndx, 55
  gkbw3     tablei      kndx, 56
  gkbw4     tablei      kndx, 57
  gkbw5     tablei      kndx, 58

endin

;;; I N S T R U M E N T    10
;;; Talking...

instr 10
            ; **************************************
        	;; ARGUMENTS:
            ;; p2 -> start
            ;; p3 -> length
            ;; p4 -> amplitude
            ;; p5 -> sample
            ;; p6 -> bp1 time (fraction of p3)
            ;; p7 -> bp2 time (fraction of p3)
            ;; p8 -> bp2 value
            ;; p9 -> bp3 time (fraction of p3)
            ;; p10 -> bp3 value
     	    ;; p11 -> pan start position
    	    ;; p12 -> pan end position
	        ;; p13 -> distance start position (>=1)
    	    ;; p14 -> distance end position (>=1)
            ; **************************************
            ; **************************************
            ;; ADSR amplitude envelope for each note
            ; **************************************
  iamp      =           ampdb(p4)
  ibp1      =           p3*p6
  ibp2      =           p3*p7
  ibp3      =           p3*p9
  ibp4      =           p3-(ibp1+ibp2+ibp3)
  kampenv   expseg      .0000001, ibp1, 1., ibp2, p8, \
                        ibp3, p10, ibp4, .0000001
  kamp      =           kampenv*iamp

            ; **************************************
            ;; SOME OLD STUFF
            ; **************************************
  kbw       =           0.1                      ; BANDWITH
  ksw       =           .003                    ; SKIRTWIDTH
  koct1     =           gkjitter             ; OCTAVIATION
            ; **************************************
            ;; FOF opcodes
            ;; Use kphs to move in sample buffer and kgliss 
            ;; to change timbre. Nice to let kgliss down
            ;; and koct down at the same time.
            ;; For original pitch: 
            ;; xform = 1/(samplesinbuffer/samplingfrq)
            ; **************************************
  a1        fof2        kamp, gkfund1, 1.345825, koct1, kbw, \
                        ksw, gkdur, .003, 300, p5, 32, p3, \
                        gkphs1, gkgliss
  a2        fof2        kamp, gkfund2, 1.345825, koct1, kbw, \
                        ksw, gkdur, .003, 300, p5, 32, p3, \
                        gkphs2, gkgliss
  atalk	    =		a1+a2

            ; **************************************
            ;; SPATIALISATION
            ; **************************************
  kpanndx    linseg      p11, p3, p12
  kpan       tablei      kpanndx, 74
  kdist	     linseg	 p13, p3, p14
  apanl, apanr		 locsig	   atalk, kpan, kdist, .1
  gar1, gar2 locsend
            ; **************************************
            ;; Output
            ; **************************************
  kch1      =           1
  kch2      =           2
            outch       kch1, apanl, kch2, apanr

endin 

;;; I N S T R U M E N T    11
;;; Bells...

instr 11
            ; **************************************
            ;; ARGUMENTS:
            ;; p2 -> start
            ;; p3 -> length
            ;; p4 -> dB amplitude
            ;; p5 -> fundamental as pch
            ;; p6 -> modulator oscilliator function table
            ;; p7 -> index factor for FM
            ;; p8 -> envelope time1
            ;; p9 -> from octaviation... 
            ;; p10 -> ...to octaviation
            ; **************************************
            ;; P-FIELD INIT
            ; **************************************
  idur      =           p3          ; Duration of object.
  idbamp    =           p4          ; Amplitude
  icpspitch =           cpspch(p5)  ; Starting pitch
  index     =           p7
  iampvar   cauchy      100
  ispeed    =           index*.6667
  iattack   =           idur*p8
  kphs1     =           0.0
  kphs2     =           .1
   kgliss    =           .1
  kindex    =           (gkbw1 *.05)*index
            ; **************************************
            ;; FM MODULATION
            ; **************************************
  acar      =           gkf1
  kmodfr    =           gkf5
  kdev      =           kindex*ampdb(90+gkfamp1)
  amodsig   oscil       kdev, kmodfr, p6


            ; **************************************
            ;; SYNTHESIS
            ; **************************************
  ksw	    =		.003
  kdur	    =		.02
  kdec	    =		.007
  kgran     line        p9, idur, p10
  agate     expseg      .00001, iattack, 1, idur-iattack, .00001
  avib      oscil       3+iampvar, .7+iampvar, 2

  a3        fof2        ampdb(idbamp+gkfamp1), icpspitch+avib, \
                        gkf1+amodsig, kgran, gkbw1, ksw, kdur, \
                        kdur, 300, 30, 31, idur, kphs1, kgliss
  a4        fof2        ampdb(idbamp+gkfamp2), <as above>
  a5        fof2        ampdb(idbamp+gkfamp3), <as above>
  a6        fof2        ampdb(idbamp+gkfamp4), <as above>
  a7        fof2        ampdb(idbamp+gkfamp5), <as above>
            ; **************************************
            ;; SPATIALISATION
            ; **************************************
  anotes    =           (a3+a4+a5+a6+a7)*agate
  kdeg	    =		    (gkbw2-60)*6
  ap1, ap2  locsig	    anotes, kdeg, 1, .1
  gar1, gar2 locsend	

  kch3      =           1
  kch4      =           2
            outch       kch3, ap1, kch4, ap2

endin 

;;; I N S T R U M E N T    12
;;; Reverb

instr 12
            ; **************************************
            ;; ARGUMENTS:
            ;; p2 -> start
            ;; p3 -> length
            ; **************************************
  a1                 reverb2 gar1, 2.5, .5
  a2                 reverb2 gar2, 2.5, .5
  kch1      =           1
  kch2      =           2
            outch    kch1, a1, kch2, a2 
  ga1=0
  ga2=0
  ga3=0
  ga4=0

endin
\end{verbatim}
\normalsize
%%% Local Variables: 
%%% mode: latex
%%% TeX-master: "../etherSound"
%%% End: 