\section{Programme note}
\label{sec:drive-programme-note}

\begin{wrapfigure}{r}{0.45\linewidth}
  \begin{minipage}[h]{\linewidth}
    \begin{flushright}
      \musicannot{Drive\\
        \emph{for Electric Viola Grande and computer}\\
        Composed \& premiered in 2002\\
        Commissioned by and dedicated to Henrik Frendin}
    \end{flushright}
  \end{minipage}
\end{wrapfigure}

%%% Local Variables: 
%%% mode: latex
%%% TeX-master: "../ImprovisationComputersInteraction"
%%% End: 

The title, 'Drive', refers to the non-linear format of the score. The instructions in the score are laid out along a circle with reference to a map in which one can move from any point to any other point at any moment. Although the piece is composed as to be played clockwise from 12 to 12, the player may choose to take another path through the material. Taking another route will affect the computer part.

The harmony and the melodic fragments are built around a six note chord in which every note gravitates towards the center of the chord, a quarter of a semi tone raised B natural. This harmonic glissando is continuous throughout the piece and results in micro-tonal variations of the original harmony. It is heard as a harmonic shadow cast by the computer behind the viola. At five consecutive points in the score the harmony is 'sampled' and the harmonic overtones from these discrete chords gives the pitches for the four virtual resonant strings in the computer part. The third computer voice is a bass note, sampled from the very first note as played by the viola, D natural, and making a glissando up a perfect fourth.

These three elements are played back by the computer as a result of the incoming signal from the viola. The viola part has three distinct motifs that each relate to the three computer generated voices. The performer is allowed to use the notated material freely, change the order of the events if she so wishes. The C string of the viola is tuned down a minor seventh, probably only possible thanks to the special quality of the instrument 'Drive' was originally composed for. Like much of my music 'Drive' is an attempt at making possible the resolving of linear time and to start processes that can evolve their own space/time relations. 
%%% Local Variables: 
%%% mode: latex
%%% TeX-master: "../ImprovisationComputersInteraction.tex"
%%% End: 
