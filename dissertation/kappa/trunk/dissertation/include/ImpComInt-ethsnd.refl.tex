\begin{wrapfigure}{r}{0.4\linewidth}
  \begin{minipage}[h]{\linewidth}
    \begin{flushright}
      \musicannot{etherSound\\
        \emph{for improvising musicians, audience and mobile phones.}\\
        Composed \& premiered in 2003\\
        Commissioned by Miya Yoshida}
    \end{flushright}
  \end{minipage}
\end{wrapfigure}

%%% Local Variables: 
%%% mode: latex
%%% TeX-master: "../ImprovisationComputersInteraction"
%%% End: 

\emph{etherSound} was commissioned by curator Miya Yoshida for her project \emph{The Invisible Landscapes} and was realised for the first time in August 2003 at \emph{Malm\"{o} Art Museum} in the city of Malm\"{o}, Sweden. The curatorial concept for \emph{The Invisible Landscapes}  project was the use of cellular phones in the context of experiencing and creating artistic expressions. The principal idea behind \emph{etherSound} was an attempt at developing an instrument that can be played by anybody who knows about how to send an SMS (Short Messages Service) from their cellular phone. The focus of my artistic research project, of which \emph{etherSound} is a part, is the interaction between computers and musicians as well as non-musicians. \emph{etherSound} is an investigation of some of the aspects of interaction between the listener, the sounds created and, in the performance version of it, the musicians playing, and also of the formal and temporal distribution of the music that this interaction results in. 

Although interaction is an important aspect of any musical performance---as well as a part of any music listening activity, to music performed live or otherwise---opening up a musical work to anyone except trained musicians is not a trivial task; careful attention has to be paid to the purpose for doing so and to the intentions of the work. It is relevant to pose the question whether it is possible to reach a satisfactory result with almost no limitations on participation. 
%%%Is this satisfactory result technological or aesthetical? And how would you evaluate the result? %%%
Before these questions can be addressed, however, we need to delineate the reasons for wanting to allow public participation. 

Public participation has been explored in the visual arts for many years, for artistic as well as political reasons: ``All in all, the creative act is not performed by the artist alone; the spectator brings the work in contact with the external world by deciphering and interpreting its inner qualification and thus adds his contribution to the creative act'' \footcite{duchamps57}. The French curator and art critic Nicolas \citeauthor{bourriaud} speaks of ``\textit{relational} art'' (italics by the author) described as ``an art taking as its theoretical horizon the realm of human interactions and its social context, rather than the assertion of an independent and private symbolic space.''\footcite[13]{bourriaud}
If we look at it from a performing arts perspective, the audience visiting a performance can be said to participate in it - if only in a limited sense. As \citeauthor{adorno} points out, however, in the spheres of distribution and consumption of music, the musical work is objectified---reduced to a mere social commodity which severely limits the freedom of choice and influence of the listener\footcite[p. 211]{adorno}. In combination with the near monopoly of the multinational corporations of production and distribution of music, though the producers will always plead the public taste (``the manipulator's reference to the manipulated is empirically undeniable''\footcite[p. 212, my translation.]{adorno}), it is difficult to claim \index{audience participation}audience participation in a general sense. Furthermore, Western art music is to a considerable extent looked upon as a hierarchical process; a process that begins in the mind of the composer and ends at the level of the listener or even before that, at the level of interpretation. It is fair to assume that bringing in an uncontrollable agglomeration of participants influencing the distribution of musical events will disturb this order. 

In their article on the multi-participant environment \emph{The   Interactive Dance Club}, multi-media artists Ryan Ulyate and David Bianciardi define one of the design goals as wanting to ``deliver the euphoria of the artistic experience to `unskilled' participants''.\footcite[41]{ulyate} Rather than merely sharing the result with an audience, they attempt to unfold the creative process leading to the result and invite the audience to take part in this process: ``Instead of dancing to prerecorded music and images, members of the audience become participants. Within interactive zones located throughout the club, participants influence music, lighting, and projected imagery'' (p. 40). The activities normally engaged at a dance club are not only performed as a result of the music played, but also used to influence the music. Similar ideas are put forward by Todd Machover concerning his large scale, interactive work \emph{The Brain Opera}: ``\emph{The Brain Opera} is an attempt to bring expression and creativity of everyone, in public or at home, by combining an exceptionally large number of interactive modes into a single, coherent experience''.\footcite[Machover, 1996, as quoted in][p. 360]{rowe01} These ambitions point to one of the big challenges of building interactive environments: how to design musical interfaces that have a ``low entry fee, with no ceiling on virtuosity''.\footcites{wessel}{jorda02}[See also][]{rowe}{auracle} With the recent technological advances there are innumerable tools that can be used for collaborative efforts,\footcite{barbosa02} affordable devices that may easily be used as interfaces to computer mediated art works (game controllers, mobile telephones, GPS navigators, web-cams, etc.). Not only has this the potential of changing our perception of the arts, it can also help us understand this new technology and the impact it has on our lives.
%%%A more general question after reading this paragraph: Is there a political (ethical, social) agenda to audience participation? %%%

A project for which public participation is important must somehow deal with the aspect of access, and according to Pierre \footcite{bourdieu} there is an intimate association between social class, level of education and cultural interests that affects cultural consumption:
\begin{squote}
The experiences which the culturally most deprived may have of works of legitimate culture [\ldots] is only one form of a more fundamental and more ordinary experience, that of the division between practical, partial, tacit \textit{know-how} and theoretical, systematic, explicit \textit{knowledge} [\ldots], between science and techniques, theory and practice, `conception' and `execution', the `intellectual' or the `creator' (who gives his own name to an `original', `personal' work and so claims ownership) and the `manual' worker (the mere servant of an intention greater than himself, an executant dispossessed of the idea of his own practice) (p. 387, italics by the author). 
\end{squote}
Bourdieu is telling us that because ``ordinary workers'' are ``[l]acking the internalized cultural capital'' they lack access to ``legitimate culture''. Instead they are referred to `` `mass market' cultural products---music whose simple repetitive structures invite a passive, absent participation'' (p. 386). Perhaps \emph{active} and \emph{present} participation can counteract the effects of lack of cultural capital? Bourdieu couples the ordinary class border divisions with the experience the working class may have of legitimate culture. I would like to suggest, however, that the association between social and cultural class and consumer electronic devices like the mobile phone, and the behaviours associated with its use, are of a different nature from the association between class and cultural consumption. If entree to the art-work is mediated through an interface (in this case the mobile phone) for which access is not governed by the same rules as the conception of contemporary art this may help level the playing field. 
%%%For sure you’re able to make an audience participate but how satisfactory is that (for them, as an aesthetic result, as a political statement)? %%%
This is a motion that works externally, from the outside in, distorting the experience of division (owing to lack of cultural capital) between the un-initiated spectator and the work. But, as we will see, there is another equally important factor at play that works from within the work. Distributing the role of the `` `creator' (who gives his own name to an `original', `personal' work and so claims ownership)'' (p. 386) to several agents---anyone interacting with the work is in fact part of the creation of it---means the listener/performer and performer/composer dichotomies are blurred and thereby another opening is created that helps provide access to the art-work. 

Roy \citeauthor{Ascott}, in addressing the issue of `content' in art involving computers and telecommunications writes:
\begin{squote}   In telematic art, meaning is not something created by the artist,   distributed through the network, and \emph{received} by the   observer. Meaning is the product of interaction between the observer   and the system, the content of which is in a state of flux, of   endless change and transformation. (p. 241, italics by the author).
\end{squote}
As opposed to the classical notion of the educated `creator' who claims `authorship' (to use the language of Bourdieu), in collaborative, telematic art-works not only is meaning a consequence of interaction, the concept of `the work' is also greatly affected. The \index{ontology}ontology of the musical work or the `work concept' in music is a complex field which is dealt with in more detail in the essays entitled \emph{Negotiating the Musical Work}. \footcite{frisk-ost06,frisk-ost06-2} \citeauthor{Ascott}, however, is mainly concerned with the visual arts in which the question of the work concept is of a different order, and makes an interesting point when substituting \textit{interface} for \textit{art object}:
\begin{squote}
The culturally dominant objet d'art as the sole focus (the uncommon   carrier of uncommon content) is replaced by the interface. Instead   of the artwork as a window onto a composed, resolved, and ordered   reality, we have at the interface a doorway to undecidability, a   dataspace of semantic and material potentiality. The focus of the   aesthetic shifts from the observed object to participating subject,   from the analysis of observed systems to the (second-order)   cybernetics of observing systems: the canon of the immaterial and   participatory. Thus, at the interface to telematic systems, content   is created rather than received.
\footcite[242]{Ascott}
\end{squote} 
Transposed to the field of music then, the \emph{the work} is replaced by the \emph{interface}. Applied to \emph{etherSound}, the mobile phone as interface becomes the work and the number of participants and their contributions become the \index{ontology}ontology of whatever work we can speak of. The interface acts as the (only) way to navigate the space created by music.

Though less centred on public participation and more on improvisation, Guy E. \citeauthor{Garnett}, composer and computer scientist, touches on some of the same issues in his article on interactive computer music aesthetics considering the potential for change in ``unfixed works'' such as ``human improvisation with computer partner'': 
\begin{squote}   
Since the human performance is a variable one, by its nature, that  variability can become the focus of aesthetic issues, even simple  ontological issues. Because the performance changes from time to  time and from performer to performer, the notion of `the work'  becomes more and more clouded. The work, even from an objective  rather than an immanent point of view, becomes something open-ended.  Each performance becomes an `interpretation' of the possibilities  inherent in whatever was `composed.' However, each of these concepts  is highly problematic. This `interpretation' can have significant  consequences for the meaning - and therefore value - of a work in a  cultural context. Since the work is not fixed, it is open to new  interpretations, and therefore the possibility at least exists for  the growth of the work over time or across cultural boundaries. The  work can thus maintain a longer life and have a broader impact  culturally, because it is able to change to meet changing aesthetic  values. (p. 27) 
\end{squote} 
As is hinted at by Garnett himself, the idea of interpretation becomes troublesome in the context of improvised music and I will discuss the issue of the work identity in more detail in \hyperref[sec:ident-emph]{Section \ref*{sec:ident-emph}}. As far as \emph{etherSound} is concerned, it cannot be performed \emph{without} public participation. As music it holds no significant value unless there is a group of people interacting with it---its value is embedded in the interaction and in this way it differs from a written score or a pre-structured improvisation.\footnote{A written score of music may be said to have musical value in itself. Although I personally argue against it, it   may even be said to constitute the work. My argument here, shared by   \citeauthor{Ascott} and \citeauthor{Garnett}, is that an interactive   attitude towards music-making changes the conditions for how the   identity of the work may be established.}  Further, the understanding of it is not necessarily related to the contextualisation of the sounds produced within the history of (interactive) electronic music but may instead be regarded as one factor in the relation between the expectations of the subject interacting and the music produced.  Following these lines of thought, it may be concluded that the need for a thorough ontological understanding of the history of art or electronic music is not a prerequisite for understanding a collaborative, interactive work of music---anyone willing to interact and interested in making a contribution is equally well prepared to produce and interpret the `meaning' of \emph{etherSound} . This limits the advantage of the educated listener---``the dominating class'' \footcite{bourdieu}---and makes room for new interpretations of the term `understanding' in the arts. \footnote{This issue is also discussed in \cite{frisk05}.}. 

%%% Local Variables: 
%%% mode: latex
%%% TeX-master: "../ImprovisationComputersInteraction"
%%% End: 
