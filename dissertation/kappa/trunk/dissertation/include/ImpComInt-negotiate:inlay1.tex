\hspace{0.5\linewidth}
\begin{wrapfigure}{r}{0.5\linewidth}
  \begin{minipage}[H]{1\linewidth}
    \advance\rightskip-1cm
     \vspace{0.5cm}\label{fig:inlay1}
\musicannot{Loosely related to the poietic-esthesic oscillation, Karsten Fundahl, my artistic adviser and mentor, has often talked about the sensation of a fluctuation between left and right brain hemisphere while engaged in creative activity. The idea of a creative negotiation between agents that we discuss in the studies performed withing the framework of\emph{Negotiating the Musical Work}\footcite{frisk-ost06,frisk-ost06-2} is partly derived from the idea that impulses emanating from artistic intuition or sensibility are, in the creative process, constantly re-evaluated and reshaped by other agents such as, perhaps, skill, knowledge, habit, resistance, etc., within the individual. The idea implying that negotiation takes place even without inter-human interaction is neither new, nor radical. But it hints  at the complexity of human knowledge. There is obviously no such thing as an activity taking place \emph{only} in the right brain hemisphere: the experience of fluctuation is not likely to be specific to musical activity or artistic activities in general. But the trope of `internal fluctuation' involves far more than what the metaphor of the left vs right brain hemisphere is able to contain; it even  resists it by its very nature. Perhaps lies in it the notion of  intuition as method, intuition, second to memory, as multiplicity; as a virtual multiplicity that Deleuze is speaking of in his discussion of Bergson's thinking.\footcite[See][13-4, 115]{deleuze88} And pursuing the trail of a complex genesis of intuition suggested by Deleuze, we could compare the internal left - right brain vacillation to the ``inter-cerebral interval between intelligence itself and  society''.\footcite[109]{deleuze88} ``The little interval `between the pressure of society  and the Resistance of intelligence' defines a variability  appropriate to human societies.  Now, by means of this interval,  something extraordinary is produced or embodied: creative  emotion.''\footcite[111]{deleuze88} What does this tell us about the silent, internal  negotiations of improvisation and composition? Out of context the  notion of \emph{creative emotion} is of little help. In practice,  Are not emotions the sustenance of artistic and musical practice?}
  \end{minipage}
\end{wrapfigure}
% \footnotetext[999]{\cite{frisk-ost06,frisk-ost06-2}}
% \footnotetext[998]{\cite[See][13-4, 115]{deleuze88}}
% \footnotetext[997]{\cite[109]{deleuze88}}
% \footnotetext[996]{\cite[111]{deleuze88}}
%%% Local Variables: 
%%% mode: latex
%%% TeX-master: "../ImprovisationComputersInteraction"
%%% End: 

% LocalWords:  Karsten
