\section{Programme note}
\label{sec:dia-programme-note}


\begin{wrapfigure}{r}{0.5\linewidth}
  \begin{minipage}[h]{\linewidth}
    \begin{flushright}
      \musicannot{
        Repetition Repeats all other Repetitions, Symphonie Diagonale\\
        \emph{for ten-stringed guitar, computer and video projection}\\
        Composed \& premiered in 2007\\
        Prepared in collaboration with Stefan \"{O}stersj\"{o}}
    \end{flushright}
  \end{minipage}
\end{wrapfigure}

%%% Local Variables: 
%%% mode: latex
%%% TeX-master: "../ImprovisationComputersInteraction"
%%% End: 

This is a version of \emph{Repetition Repeats all other Repetitions} worked out by myself and Stefan \"{O}stersj\"{o} in January 2007. When \"{O}stersj\"{o} came across the classic dadaist film \emph{Symphonie Diagonale} he found that perhaps it could serve as an alternative source for inspiration in order to come up with a different way to interpret \emph{Repetition Repeats all other Repetitions}. Viking Eggeling's film became the method. When we started analyzing the film we soon realized that it, in its composition, had some striking similarities to \emph{Repetition...} in that it has three distinct set of materials that are combined and re-organized during the course of the work. Although we did strive for harmony between image and sound, we did not attempt to set music to the film; the two works have been aligned in time, still respectful of the integrity and identity of the classic work of art that the \emph{Symphonie Diagonale} has become.

Viking Eggeling (1880-1925) was an artist and film maker from Lund, Sweden. In Paris in the early 20's, he and Hans Richter started experimenting with the film medium which eventually led to Eggeling completing the \emph{Symphonie Diagonale} in 1924, just before he died in May 1925 at the age of 45. Although the \emph{Symphonie} was one of the few works he completed and that still remains---his first film, \emph{The Horizontal-Vertical Orchestra} has been lost---his reputation has grown. I wish to acknowledge Leif Ericsson and Bengt Rooke from the Swedish \href{http://www.rooke.se/text/viking.html}{Eggeling Society} for giving me access to a high quality copy of \emph{Symphonie Diagonale} to work with.

%%% Local Variables: 
%%% mode: latex
%%% TeX-master: "../ImprovisationComputersInteraction.tex"
%%% End: 
