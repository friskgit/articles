\section{Giving up The Self}
\label{sec:inter-prod}

The `Self' as psychological identity, as the way one understands oneself, has a history of multiplicity and is in not an unambiguous term. Sigmund \citeauthor{freud27}\footnote{In this context too, it is worth mentioning that the following text is not (in any way) intended as an exhaustive introduction to psycho-analysis or the philosophical history of identity and self. The references to \citeauthor{freud61} and \citeauthor{sartre43} are made to present differences with the double intention to establish a terminology and disrupt preconceived notions of the same.} isolated three interrelated aspects of the Self: the Id (\emph{das Es}), the Ego (\emph{das Ich}) and the Super-ego (\emph{das \"{U}ber-Ich}), and, although controversial and disputed, they have to a significant degree shaped our understanding of the psychic Self and provided us with a terminology which is today used in everyday language. If the id is situated in the unconscious, the ego---stretching from the unconscious, through preconscious and into consciousness---is the intermediary between the id and and world to the degree that the ego \emph{is a part of} the id. Furthermore, the consciousness is attached to the ego which controls the ``discharge of excitations into the external world; it [the ego] is the mental agency which supervises all its own constituent processes'',\footcite[8]{freud61} And it is the the ego that yields the repressions in an attempt to purge consiousness of `unwanted' trends in the mind and thus forming the unconscious ego. In an earlier essay, \citetitle{freud27}, \citeauthor{freud61} is suggesting a symmetry between conscious and unconscious on the one hand and ``the coherent \emph{ego} and the \emph{repressed}''\footcite[13]{freud27} on the other. Provided the Self is the location of the ego (which is not self-evident), based on this very brief epitome of \citeauthor{freud27}'s notion of the ego, for the moment ignoring its meaning and significance outside of the contest of psycho-analysis, the strata of the Self is already noticeable: it has conscious, preconsious and unconscious parts, and coherent and the repressed egos, the former loosely attached to consciousness, and whose role it is to mediate and control. But even if the ego is \emph{located} in the Self, the ego as such \emph{is} not the Self. If anything perhaps it is the combination of all the different parts of the psyche that together may form a kind of Self. 

\citeauthor{sartre43}, in his monumental work \citetitle{sartre43}, bring about a different view on the topic of the Self avoiding the conscious-unconscious dichotomy so central to psycho-analysis as well as the distinction between the `ego' and the `id', which ``cuts the psychic whole into two.''\footcite[][74]{sartre43}. Now, for \citeauthor{sartre43} the Ego, is \emph{not} as \emph{of the nature} of consciousness. The Ego is non-conscious (but far from unconscious), it is ``in-itself'',\footcite[127]{sartre43} which is to say, following \citeauthor{sartre43}, that it simply \emph{is}:\footcite[22]{sartre43} It is \emph{in-itself} and as such it overflows any knowledge we may have of it.\footcite[Sartre specifies the categories of Being, for-itself and in-itself where the for-itself is the nihilation of Being-in-itself. The central theme of Nothingsness is closely related to for-itself. Of the in-itself can only be said, according to Sartre, that it \emph{is}: ``being is opaque to itself precisely because it is filled with itself.''][21, 97, 650]{sartre43} To substantiate the transcendent in-itself nature of the Ego, \citeauthor{sartre43} points to differences between the `I' on the one hand and consciousness on the other. Any consciousness one may have of the `I' (self-consciousness?) does not exhaust or consume it, nor does consciousness constitute that which makes it come into existence. The `I' is existent regardless of any self-consciousness one may have of it and it does not depend on consciousness. So whereas the Freudian Ego is \emph{attached} to consciousness, here it appears to be detached and independent from it. Furthermore, the `I', according to \citeauthor{sartre43}, always precedes consciousness; it ``is always given as \emph{having} been there before consciousness''\footcite[\pno 127 (italics by the author)]{sartre43}. And just as consciousness does not bring forth the `I', the Ego is not what procreates personality in an otherwise impersonal consciousness. Rather, the disclosing interrelationship and signification between consciousness and the Ego is described such as it is ``consciousness in its fundamental selfness which [\ldots] allows the appearance of the Ego as the transcendent phenomenon of that selfness.''\footcite[127]{sartre43} The Ego as an in-itself, situated in the world. Through its reflection, its movement of reflection, the conscious makes itself \emph{personal} and the Ego, on its part, is the sign of this personality. The Ego as a reflector of consciousness? But where in this description is the Self? What is the relation between the Ego-consciousness association and the Self? For \citeauthor{sartre43}, though consciousness is not part of the self, the self defines the very \emph{being} of consiousness. Consciousness always and only refers to the thing, by itself it is not being and can not convey an absolute subjectivity.\footcite[638]{sartre43} The referring self, on the other hand, refers precisely to the subject. It is indicative of a relation between the subject and himself but the subject cannot \emph{be} self because, according to \citeauthor{sartre43}, coincidence with the self would cause the self to vanish. But also, as self is an indication of the subject himself, it refers to the subject, the subject cannot avoid to be itself. The presence of the self presupposes a distance, a way for the subject to not coincide with himself, and is described as ``being in a unstable equilibrium between identity as absolute cohesion without a trace of diversity and unity as a synthesis of a multiplicity.''\footcite[101]{sartre43} Presence to the self. Being as present to itself; a presence made possible due to a divergence between it and itself, because it is not wholly itself.\footcite[This difference, or non-coincidence, is what Sartre calls \emph{nothingness} precisely because it is nothing: ``But if we ask ourselves at this point \emph{what it is} which separates the subject from himslef, we are forced to admit it is \emph{nothing}'' He further refers to it as a fissure in consciousness: ``This fissure then is the pure negative.''][101-2]{sartre43} The self can only exist if it carries within it its own nothingness.

Is there a symmetry, albeit a subtle one, between \citeauthor{sartre43}'s concept of non-coincidence, \emph{Nothingness}, and \citeauthor{baudrillard02}'s as well as \citeauthor{levy97}'s critique of real-time? Real-time as the ``mad velocity [that] suppresses and destroys anything that attempts to grow slowly'',\footcite[180]{levy97} constructing the instantaneous reflections in the virtual of activities performed in the real. For \citeauthor{baudrillard02} both time and space collapses as we get `screened out'---threatened by interactivity. Distance, any kind of distance, is abolished and undecidability is all that remains as the ``vital tension is discharged.''\footcite[176]{baudrillard02} (That the interactivity \citeauthor{baudrillard02} refers to is somewhat different from interactivity in the sense it is used this thesis and in interactive music is discussed in \hyperlink{sec:target:human-comp-inter:par4}{section \ref*{sec:human-comp-inter}}.) Real-time as only simultaneity with no room for subjective proximity and interior distances.\footcite[182]{levy97} While \citeauthor{sartre43}'s description of presence as an immediate deterioration of coincidence, of a presence that supposes separation, ``supposes that an impalpable fissure has slipped into being'',\footcite[101]{sartre43} seems to be related, at least metaphorically to Baudriallard's expectations on social interaction in the sphere of exchange, it must be noted that the discontinuity between subject and self in \citeauthor{sartre43}'s model is nothing, it is precisely \emph{nothingness}. It is not a distance in time nor space, it is not, nor does it belong to a qualified reality.\footcite[See][101]{sartre43} For \citeauthor{baudrillard02:screened} on the other hand, the separation between given and returned in the social interaction is a distance in time, and for \citeauthor{levy97} the distance manifests itself as a difference in velocity of time, different intensities of real-time that will animate the collective intellect;\footcite[See][179]{levy97} a process of subjectification within the collective. However, and irrespective of these and other discontinuities, the common thread in all three cases is that non-coincidence is essential for any notion of the self. Purely individually, following \citeauthor{sartre43} the subject coinciding with the self causes the self to disappear. In the sphere of exchange, in the social interaction, immediacy, abolishing distance in time and in space, demolishes any real sense of the other as well as any real sense of the real.\footcites[See][]{baudrillard96:writing}[See also][]{baudrillard02:screened} In the sphere of the collective, the imagining community, continously flooding the quasi-punctual temporality of the interactive. It resists synchronization with clocks and calendars and may appear as displaced, interrupted, fragmented, but it all ``occurs within the obscure, invisible folds of the collective itself.''\footcite[126]{levy97} All three, from three diverse perspectives and with contrasting agendas, speak of the importance of difference, non-coincidence, in order to reveal the self, the other and the collective respectively. If it is possible to regard the consciousness-self articulation of \citeauthor{sartre43}, and the related unstable equilibrium between unity and multiplicity as an interaction the convergence between these three modes of thinking may perhaps appear more convincing. Three different kinds of interactions all sharing the dependence upon difference and non-coincidence. \emph{Interaction-as-difference}.



% \begin{squote}
% Additionally being does not refer to itself as self-consciousness   % does. It is this self. It is itself so completely that the perpetual   % reflection which constitutes the self is dissolved in an identity.   % That is why being is at bottom beyond the \emph{self} % [\ldots]\footcite[21]{sartre43}
% \end{squote}
% In consdering these questions another relation is revealed which is % that between Being and consciousness. 
% The relation between Being and consciousness it a difficult one 

This very brief introduction to some of the philosophy of the Self is only intended as a demarcation point for a, in all respects, rather different understanding of Self, subjectvity and identity. It was only intended to posit the terminology in a context and to show that, although the words may be stable, their meanings are filled with (necessary) contradictions. 

The self, whether \citeauthor{freud61}'s model of description is adopted or \citeauthor{sartre43}'s, is multiplicity. It is the transcendent self as well as the Superego. It is both culturally defined and internally constituted.
These two very briefly described points of view, although obviously related, reveal the non-static nature of the general investigation of the nature of the Self, which by any standard must be seen as a central topic in Western philsophy. However, my point here is obviously  not to present and compare philosophies of identity and the Self, but merely to adduce them in order to provide a framework for the continuing discussion. But they are also presented here to point to the multitude of possible understandings of the Self; the problem at issue is not the lack of a clearly defined and delineated impression of the Self. On the contrary, it is the increasingly narrow idea of the Self commonly portrayed and broadcast, dominating in consumerism, but also present in the for this thesis central field of human-computer interaction.\footcite[In a very readable essay, artist David Rokeby (whose installation \emph{Giver of Names} is mentioned in relation to the discussion on \hyperlink{sec:target:interaction-symbiosis}{man-computer symbiosis}.) points out the unfortunate and distorted mirror reflection the computer gives of its user: ``A standard GUI interface is a mirror that reflects back a severely misshapen human being with large hands, huge forefinger, one immense eye and moderate sized ears. The rest of the body is simply the location of backaches, neck strain, and repetitive stress injuries.'' The way we are being perceived as human beings will inevitably influence any view we may have of our Selves, as well as of any Other in relation to that Self.][]{rokeby98}

 \newpage

one of the reasons for bringing up the discussion of the elusive nature of the Self in this context

Perhaps the Self already is dissolved as is suggested by Kowalski

These two 

that allows for the Ego to reveal.

But if the Self is constituted by many different things, by many different logics, what about it is there to give up? In what ways does the Self \emph{get in the way} so to speak? If the act of giving up the Self is rooted in a wish to control what parts of the Self should be present and what parts should be abolished, isn't this in itself a control agency? And isn't the ambition then a contradiction in terms (to give up the Self to gain control over the Self)?

The loss of the self is not a novel thought. 

If the self is an enclosure that holds whithin it the different aspects of personality and identity (without confining ourselves to Freudian language these may be id, ego and superego, but also aspects usually thought as enclosed by those, such as control, fear, power, etc.) this enclosure has an aptitude to disclose or dissolve in music.

holds within it a duality. On the one hand there is the genuine, unmediated self, the individuality: our personality as 

, is often discussed in relation to music. The immersive qualities of music and sound in general amplifies the sensation of the dispersing of the self. The self further implies a duality

What then is the nature of this reading of `the self'---`the self' which at the same time must be abandoned in order to be re-constituted, and acknowledged in the interaction with the `other', by the `other'? This `self' is somewhat related to the Freudian notion of the superego. A culturally constructed and and socially taught `self' with a preconceived understanding of behaviour. My point here is that `the self' has to be re-constituted anew for each interactive context and that the result of this process is a stronger and yet more open ended, inclusive, sense of identity. Before looking at how these ideas relate to some of the more philosophical aspects of interaction and the self I will attempt to contextualize my understanding of `the self'.

The self is in the way for building larger systems. The self implies an enclosed system that interacts with its environment in a way . Giving up the self is believing that individuality is part of all levels of being. etc.


In her book Ingrid \citeauthor{monson96} discusses the conversation metaphor and its ``structural affinities with interactive improvisational process'' (chap. 3, p. 73). In an encounter with drummer Ralph Peterson they discuss a particular passage from a recording of his group. About a musical discourse between Peterson and the pianist, Geri Allen, he is quoted saying: 
\begin{quotation}
  ``[...] a lot of times when you get into a musical conversation one person in the group will state an idea or the beginning of an idea and another person will complete the idea or their interpretation of the same idea, how they hear it''\footcite[Ralph Peterson as cited in][78]{monson96}).
\end{quotation}
\citeauthor{monson96} analyzes the comment and states that ``[i]n associating the trading of musical ideas with conversation, Peterson stressed the interpersonal, face-to-face quality of improvisation'' (\textit{ibid.}). And later, referring to the same passage: ``These moments of rhythmic interaction could also be seen as negotiations or struggles for control of musical space'' (p. 80). Conversation in this context must be regarded as truly, and only, a metaphor. Musical performance has very little to do with verbal discourse and ``nothing in common with a text (or its musical equivalent, the score) for it is music composed through face-to-face interaction'' (\textit{ibid.}).\footnote{It should be noted that   \citeauthor{monson96} limits the quoted statement to relate to jazz   improvisation where I would argue that it holds true for all musical   performance.} We would have to move to the more abstract level of poetry as exercized by Ezra Pound or Ralph Waldo Emerson in order to make sense of a comparison with text, which we will do. 

To allow for someone to complete your idea, inserting their own interpretation of the same idea, without feeling the need to correct the `erroneous' reading is an aspect of giving up `the self'. And further that this can 

The fact that, in a musical conversation it is perfectly valid to complete a co-musician's statement with a deeply personal interpretation of the same

Habermas makes a distinction between instrumental or purposive-rational action (`Arbeit') and communicative action (`Interaktion') in an attempt to separate that which, according to Habermas, Hegel reduced to the general concept of \emph{Philosophie   des Geistes}. The instrumental action is based on technical rules 

\begin{quotation}
  3:e paragrafen, s. 73
\end{quotation}

The natural sciences are a refined extrapolation of the instrumental action as it is manifested in the for human life so important work. ``Purposive-rational action is by its structure exercising control'' \footcite[(\textit{My   translation})][63]{habermas68}. 

\begin{quotation}
  4:e paragrafen s. 73
\end{quotation}

Habermas is here outlining what is later to become the Theory of Communicative Action. Though he has been criticized for, and himself revised his thinking on, some of these matters\footcite[see]{bertilsson83}\footnote{For an overview of Critical Theory in general and The Theory of Communicative Action in particular see \cite{ericsson01}. An interesting criticism of the philosophical aspect of work in relation to feminist theory is offered by \cite{gurtler05}.} my main concern is the difference between labor and interaction. According to \citeauthor{bertilsson83} for Habermas labor is a rule based and empirically founded activity and it fulfills the social needs for predictability and control. But it has been allowed to spread into domains where it should not be dominating at the expense of social interaction. ``When predictability and and control is allowed to spread at the expense of other domains of knowledge, this will happen in a social context where man is meeting the other, her neighbour, as an enemy and as object to dominate and rule'' (p. 16, my trans.). 

In other words, control belongs to the domain of instrumental action which is in every respect different from human interaction which can only happen truly if the subjects are mutually respectful of each other. 
\begin{quotation}
  Medvetandet om mig själv är ett derivat av att två perspektiv   korsas. Först på basis av ömsesidigt erkännande, utvecklas det   självmedvetande, som måste bindas vid min spegling i ett annat   subjekts medvetande. \footcite[183]{habermas68} 
\end{quotation}

\section{Social interaction and the giving up of the self}
\label{sec:social-interaction}
 The request for responsiveness in HCI is indicative of the aspect of control embedded in the definition: The machine should not act by itself, it should without delay respond to our actions, to our instructions, to how we want it to respond. In human-human interaction, respectful of the other, a similar request for immediate response or demand for control would be unthinkable.

Then, who is this `other'? What is the identity and location of this `other' with whom social interaction takes place. As I mentioned briefly in Section \ref{sec:research-question} my interest in human-human interaction is not a goal in itself but a way to understand, inform and try to develop musician-computer interaction in my own artistic practice. I will here start from the specific context of my own experience and then move to the more general idea of the `other'.

The `other' I am referring to is not only the `\emph{epistemological other}' of \footcite{somers94}---a social construction created ``to consolidate a cohesive self-identity and collective project'' \footcite[As cited in][]{lewis-1}, though, whether I want it or not, in a sense it is that too. The `other' is not a homogenic group that has distinct properties that defines its `otherness'. The `other' is `other' in relation to the `self', to \emph{me}, but not in order to consolidate this `self', which also will not let itself be defined by distinction.  There is no difference between the `otherness' of Ngyen Thanh Thuy or Stefan \"{O}stersj\"{o}---the one is not more `other' than the other---in the project The Six Tones (see Section \ref{sec:negotiating-2}). The `other' is the one or those I as a musician am interacting with. It is my co-musicians with whom I am trying to connect, whom I am trying to understand in order to understand myself better. It is in the process of trying to understand through interaction, that I, in a certain sense, need to give up `the self'. Before moving on to the more general reading of the `other' a few remarks should be made about these issues:
\begin{enumerate}
\item What I am describing here is my attempt to identify what I believe is going on when `things are working'. It is the ideal situation as I have experienced it. It is the sensation of wordless communication, of intuition and self organization. It is a sensation that is not tied to a particular idiom or style---it is not necessarily tied to music.
%
\item In no way am I able to reach this stage at all times. And, when unsuccessful, it is my experience that the `self' is exercising a wish to control the situation, though it is difficult to say if this precedes the failure (i.e. is a consequence of) or is an attempt to `fix' an error that has occurred due to other reasons. For example, it may be the mistake of trying to force idiom or style into a context that does not harmonize with that which is forced upon it.
%
\item I am using my artistic practice therapeutically and the idea of better understanding the `self' is an attempt to reach greater awareness of my responsibilities as a human being and as an artist. In particular it is a part of the process to reach self-awareness that I, as a white, European, male belong to a class that has exercised oppression and exploited women and more or less every other culture, religion or species that we have encountered in the last 2.500 years.
\end{enumerate}

%%% Local Variables: 
%%% mode: latex
%%% TeX-master: "../ImprovisationComputersInteraction"
%%% End: 
