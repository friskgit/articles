\subsection{Social interaction and the giving up of the self}
\label{sec:social-interaction}
 The request for responsiveness in HCI is indicative of the aspect of control embedded in the definition: The machine should not act by itself, it should without delay respond to our actions, to our instructions, to how we want it to respond. In human-human interaction, respectful of the other, a similar request for immediate response or demand for control would be unthinkable.

Then, who is this `other'? What is the identity and location of this `other' with whom social interaction takes place. As I mentioned briefly in Section \ref{sec:research-question} my interest in human-human interaction is not a goal in itself but a way to understand, inform and try to develop musician-computer interaction in my own artistic practice. I will here start from the specific context of my own experience and then move to the more general idea of the `other'.

The `other' I am referring to is not only the `\emph{epistemological other}' of \footcite{somers94}---a social construction created ``to consolidate a cohesive self-identity and collective project'' \footcite[As cited in][]{lewis-1}, though, whether I want it or not, in a sense it is that too. The `other' is not a homogenic group that has distinct properties that defines its `otherness'. The `other' is `other' in relation to the `self', to \emph{me}, but not in order to consolidate this `self', which also will not let itself be defined by distinction.  There is no difference between the `otherness' of Ngyen Thanh Thuy or Stefan \"{O}stersj\"{o}---the one is not more `other' than the other---in the project The Six Tones (see Section \ref{sec:negotiating-2}). The `other' is the one or those I as a musician am interacting with. It is my co-musicians with whom I am trying to connect, whom I am trying to understand in order to understand myself better. It is in the process of trying to understand through interaction, that I, in a certain sense, need to give up `the self'. Before moving on to the more general reading of the `other' a few remarks should be made about these issues:
\begin{enumerate}
\item What I am describing here is my attempt to identify what I believe is going on when `things are working'. It is the ideal situation as I have experienced it. It is the sensation of wordless communication, of intuition and self organization. It is a sensation that is not tied to a particular idiom or style---it is not necessarily tied to music.
%
\item In no way am I able to reach this stage at all times. And, when unsuccessful, it is my experience that the `self' is exercising a wish to control the situation, though it is difficult to say if this precedes the failure (i.e. is a consequence of) or is an attempt to `fix' an error that has occurred due to other reasons. For example, it may be the mistake of trying to force idiom or style into a context that does not harmonize with that which is forced upon it.
%
\item I am using my artistic practice therapeutically and the idea of better understanding the `self' is an attempt to reach greater awareness of my responsibilities as a human being and as an artist. In particular it is a part of the process to reach self-awareness that I, as a white, European, male belong to a class that has exercised oppression and exploited women and more or less every other culture, religion or species that we have encountered in the last 2.500 years.
\end{enumerate}

%%% Local Variables: 
%%% mode: latex
%%% TeX-master: "../ImprovisationComputersInteraction"
%%% End: 