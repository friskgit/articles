% Created 2020-01-03 Fre 21:38
% Intended LaTeX compiler: pdflatex
\documentclass[11pt]{article}
\usepackage[utf8]{inputenc}
\usepackage[T1]{fontenc}
\usepackage{graphicx}
\usepackage{grffile}
\usepackage{longtable}
\usepackage{wrapfig}
\usepackage{rotating}
\usepackage[normalem]{ulem}
\usepackage{amsmath}
\usepackage{textcomp}
\usepackage{amssymb}
\usepackage{capt-of}
\usepackage{hyperref}
\author{Henrik Frisk}
\date{\today}
\title{}
\hypersetup{
 pdfauthor={Henrik Frisk},
 pdftitle={},
 pdfkeywords={},
 pdfsubject={},
 pdfcreator={Emacs 26.1 (Org mode 9.1.9)}, 
 pdflang={English}}
\begin{document}

\tableofcontents

\maketitle
\section{Introduktion}
\label{sec:orgdd1c696}
Konstnärlig forskning har haft en dynamisk och positiv utveckling de
senaste tjugo åren, men vägen är fortfarande ojämn och lite knagglig: På
ett bredare plan finns det relativt lite kunskap om vad det egentligen
är, även på vissa konstnärliga högskolor och likaså finns det
fortfarande frågor kring vad den kan bidra med. Man kan förvisso fråga
sig om kunskapen om till exempel den sociologiska eller antropologiska
forskningens metoder och resultat i allmänhet är större, eller om det
finns en bredare insikt i på vilket sätt forskning i, säg, fysik bidrar
till samhällets utveckling. Dessa exempel på forskningsdiscipliner är
dock uppenbart politiskt och kulturellt mer etablerade på ett sätt som
den konstnärliga forskningen inte är, varför okunskapen är ett större
problem i dessa fält än i andra.

Den konstnärliga forskningen i musik har en kanske längre historia än
andra konstnärliga fält, mycket tack vare de musikvetenskapliga
initiativ till en forskarutbildning med praktisk ingång som togs av
musikvetenskapen i Göteborg i slutet av 80-talet och etableringen av
musikpedagogik. Resultatet blev en mjuk övergång från dessa
vetenskapliga grenar av musikforskning till konstnärlig forskning i
musik. Detta har varit en fördel och delvis också en belastning. Det
senare då den konstnärliga forskningen i musik delvis har präglats på
detta vetenskapliga arv på ett sätt som till exempel den konstnärliga
forskningen i fri konst inte har gjort på konstvetenskap i samma
utsträckning. Fördelarna, å andra sidan, handlar om metodutvecklingen
och etablerade teorier att knyta an till.

När jag påbörjade min forskarutbildning på Musikhögskolan i Malmö vid
konstnärliga fakulteten, Lunds universitet, var det ett medvetet val att
vi doktorander på konst-, musik- och teaterhögskolan skulle närma oss
frågan om metod utan större inflytande från etablerade discipliner. På
Musikhögskolan innebar det att vi inte skulle ha gemensamma seminarier
med musikpedagogiken, som var den enda forskning som fanns innan den
konstnärliga forskningen etablerades i början av 00-talet. Istället hade
vi gemensamma seminarier med de andra konstnärliga doktoranderna. Den
mest aktiva gruppen bestod av oss tre från musikhögskolan och de tre
doktoranderna från konsthögskolan, där seminarierna också ägde rum.
Mycket tack vare Sarat Maharaj, Brittisk konsthistoriker och curator som
ledde seminarierna, och Gertrud Sandqvist, professor och då rektor för
konsthögskolan, fanns det trots denna relativa isolering ett stort
intresse av att vända sig mot discipliner som filosofi, sociologi,
psykologi och historia, vilket gjorde att diskussionerna redan i början
i stor utsträckning präglades av interdisciplinäritet. Ambitionen var
att de konstnärliga projekten skulle styra valet av metod och teori,
även om det i verkligheten nog snarare var en ganska dynamisk
interaktion mellan praktik och teori. Hade resultatet blivit ett annat
om vi från musikhögskolan istället hade haft vår seminarier med
musikpedagogiken? Definitivt, men det är naturligtvis omöjligt att
fastställa om det hade varit bättre eller sämre, men jag vågar påstå att
det hade varit ett snävare teoretiskt perspektiv

Avsaknaden av definierade teoribildningar och metoder är den
konstnärliga forskningens välsignelse och förbannelse. Enligt många har
utmaningen för forskningen, och det konstnärliga utvecklingsarbetet
innan dess, varit att metodologin har varit för otydlig och att
projekten inte har tagit sig ur den enskilda forskarens privata
sfär.\footnote{\footcite[s.105]{Dunin2007}.} Det problemet kommer att lösas så småningom, även om det kan
ses som att det tar lång tid. Fler kontaktytor mot såväl övriga
forskningsdiscipliner, resten av kulturlivet samt en bredare del av
samhället skulle kunna bidra till en mer stabil utveckling. Att bryta
den önskan till isolering som präglade den tidiga utvecklingen av fältet
behöver inte samtidigt betyda att konstnärlig forskning tappar sin
identitet som unikt forskningsfält.

Historiskt sett har dock en del energi ägnats åt att hävda att den
konstnärliga forskningen ska vara fri från den typ av band som annars
förväntas från forskning i andra discipliner. En långvarig diskussion i
det lilla kluster som jag beskrev ovan var om vi överhuvudtaget skulle
ha metod (ett mantra som återkom var "fuck method") och på ett
övergripande plan ville man till varje pris undvika en utdragen
metoddebatt i fältet. Någon större debatt har nog inte heller
förekommit, men det är tydligt att doktorander än idag tycker att
metodbegreppet i konstnärliga forskning är svårt att hantera. Det finns
ingen brist på metodologiska tillvägagångssätt i konstnärlig forskning i
Sverige, men däremot kan det var svårt att hitta en röd tråd vilket gör,
vill jag påstå, att fältet i sin helhet blir onödigt disparat. Lösningen
på detta är inte homogenitet, utan en för fältet övergripande diskussion
om metodologi. som samtidigt är djuplodande. Konstnärlig forskning kan
rymma en mängd olika perspektiv och förhållningssätt och det ska
förekomma stor rörligt inom hela spektrat av metodologiska möjligheter.

Anledningen till att vi befinner oss i den här situationen har mycket
att göra med det som är själva kärnan i utmaningen med konstnärliga
forskning, nämligen relationen mellan den konstnärliga praktiken och den
forskande aktiviteten, och det faktum att denna relation kan se ut och
utveckla sig på en rad olika vis. Om forskningsprojektet är helt
integrerat, det vill säga att forskningen helt utgår från den
konstnärliga praktiken, är de metodologiska behoven naturligtvis
annorlunda från ett projekt där en konstnärlig praktik utforskas med
skiftande perspektiv, eller ett där det snarare är resultatet än
praktiken som studeras. Med andra ord, var den forskande praktiken
befinner sig i relation till den konstnärliga är avgörande för valet av
metod. Därför är det svårt att se framför sig den typen av metodologisk
enhetlighet i konstnärlig forskning som man kan se i vissa andra fält,
men, gemensamt för all konstnärlig forskning borde vara den specifikt
konstnärliga metoden. Det är den som gör forskningen relevant som just
konstnärlig forskning då det är den som visar på att valen som har
gjorts har gjorts på konstnärlig grund.

Så länge den konstnärliga metoden är väl beskriven, stabil och hållbar,
så finns det god sannolikhet för att det finns tillgång på data som går
att utforska och en anledning till att dessa data är av intresse är det
faktum att de kommit till genom konstnärliga överväganden snarare än
vetenskapliga. Därmed har de potential att ge uttryck för konstnärlig
kunskap. I vår artikel \emph{Beyond Validity} \footnote{\footcite{frisk-ost13}.} diskuterar Stefan
Östersjö och jag bland annat just validitet i konstnärlig forskning och
pekar på behovet av att dekonstruera den positivistiska betydelsen av
detta begrepp. Det förutsatte dock också att vi tittade närmare på några
av de begrepp som färgat utvecklingen av konstnärlig forskning, och som
relaterar till diskussionen ovan. Henk Borgdorff är en person som har
haft en hel del inflytande över den tidiga utvecklingen av konstnärlig
forskning i Sverige och fortfarande så i Holland. Han var bland annat
medlem i den externa referensgruppen för nationella konstnärliga
forskarskolan, likaså aktiv på Göteborgs universitet i en period. I sin
bok \emph{Conflict of the Faculties} artikulerar han den ganska utbredda
hållningen att konstnärlig forskning inte söker efter att uttrycka
explicit kunskap utan snarare om att:

\begin{quote}
provide a specific articulation of the pre-reflective, non-conceptual
content of art. It thereby invites unfinished thinking. Hence, it is
not formal knowledge that is the subject matter of artistic research,
but thinking in, through, and with art. \footnote{\footcite[s.143]{borgdorff2012}.}
\end{quote}

Att konsten, och det konsten uttrycker, skulle vara icke-konceptuell,
det vill säga att vi erfar till exempel musik utan att den passerar ett
begreppsfilter, är en idé som har präglat den västerländska estetiken
sedan Kant. Är den inte konceptuell, eller inte låter sig
konceptualiseras, så kan resultatet av den inte heller kommuniceras
annat än möjligtvis metaforiskt, vilket i så fall skulle göra den till
ett extremt flyktigt objekt som endast svårligen låter sig studeras. Att
kunna fånga det som den konstnärliga processen uttrycker, vare sig det
är konstnären själv som upplever det i skapandet eller om det är
lyssnaren/publiken som erfar det i lyssnandet (om vi nu begränsar oss
till musik) måste ses som en central funktion för konstnärlig forskning.
Det är här mycket av den specifika kunskapen finns och den är ofta dold
eller vag även för den som erfar den och det rör sig sannolikt om olika
kategorier av kunskap i olika delar av konstnärlig praktik: även om de
sammanfaller till stor del så följer lyssnandet en annan logik än
skapandet.

Samtidigt vore det fel att påstå att konstnärligt uttryck inte kan vara
icke-konceptuellt. Ofta refererar vi till en stark upplevelse som något
som går rakt in utan att passera en analytisk eller konceptualiserande
fas. Att detta per automatik resulterar i \emph{ofärdigt} tänkande eller att
det per definition rör sig om ofärdigt tänkande som är för-reflektiv,
eller till naturen icke-konceptuell, är dock en onödig förenkling.
Istället kan vi se den konstnärliga kunskapsutvecklingen som något som
sker i olika strata av medvetande och färdighet, och i olika
konstnärliga projekt är de olika skikten mer eller mindre framträdande
och mer eller mindre möjliga att verbalisera. Som jag och Östersjö
konstaterar finns det ett behov av att bättre förstå detta fält:

\begin{quote}
[\ldots{}] we now have to examine more closely the notions of
pre-reflective and non-conceptual contents of art. Are they the same
or different things? Does not 'pre-reflective' indicate that there is
something unfinished in its trajectory? The category of non-conceptual
knowing seems to us to be distinct from the unfinished. This appears
to be a field not sufficiently discussed and theorized within artistic
research, and, still, it is the heart of the matter: in all artistic
production, knowledge is created and passed on in ways that are most
often distinct from the verbal domain. \footnote{\footcite[s.35]{frisk-ost13}.}
\end{quote}

Om konstnärlig kunskap är skapad och kommunicerad på sätt som är
väsensskilt från verbal kommunikation så är en del av utmaningen i
konstnärlig forskning att översätta denna så att den går att
begreppsliggöra på ett sätt som gör kunskapen i de studerade processerna
användbar. Här finns det dock metoder som kan användas, och för att bara
ta ett exempel i all korthet, är \emph{stimulated recall} en som har visat
sig relativt bra för att fånga kognitiva processer. I en översiktlig
artikel som är en metastudie på en rad studier av denna metod skriver
Lyle:

\begin{quote}
As this article will demonstrate, there is no doubt that there are
significant limitations in incorporating SR procedures into research
designs. Nevertheless, the method has considerable potential for
studies into cognitive strategies and other learning processes, and
also for teacher/educator behaviour, particularly complex, interactive
contexts characterised by novelty, uncertainty and non-deliberative
behaviour.\footnote{\footcite[s.861-2]{Lyle2003}.}
\end{quote}

Detta är inte rätt plats för en ingående beskrivning av denna ganska
ofta använda metod, men jag har själv god erfarenhet av den för just
detta ändamål. Ett annat tillvägagångssätt som använts i stor
utsträckning i konstnärlig forskning är filosofi. I den avhandling jag
kommer diskutera i detta kapitel, Åsa Stjernas , är det framför allt den
franske filosofen Gilles Deleuze begreppsapparat som hon vänder sig
till, och kanske framförallt de böcker han skrev tillsammans med Félix
Guattari.

Men här finns ytterligare en komplicerande faktor som jag kort nämnde
ovan. Konstnärlig forskning som i huvudsak ser till processen i
skapandet, hur den vecklar ut sig över tid och hur den förhåller sig
till politiska och sociala mönster, är inte nödvändigtvis samma sak som
konstnärlig forskning som i huvudsak ser på resultaten av det
konstnärliga arbetet. I många fall rör det sig om en kombination av båda
modellerna då det självklart är svårt att helt separera de från
varandra, men, för att begreppsliggöra en konstnärlig process kan det
behövas andra metoder än då målet är att förstå vari kunskapen i ett
konstnärligt resultat består. Oavsett vilken process man arbetar i
behövs det en metod för att komma åt den data och den kunskap som den
erbjuder. Att se på konstnärlig praktik som en multiplicitet av
potentiella uttryck och kunskapsformer som kan aktualiseras och
begreppsliggöras genom forskning betyder också att metoden inte bara är
viktig, utan att många olika typer av metoder ibland är nödvändiga. Det
är ofta bra att låta varje projekts individuella konstellation av
relationer styra vilken metod som är den riktiga. Dessa olika nivåer,
eller grader av konstnärlig (och vetenskaplig) kommunikation kan
möjliggöra att någon del av, eller hela erfarenheten, verbaliseras och
kommuniceras.

En viktig metod, som ibland glöms bort, är själva skrivandet. Även denna
har diskuterats en del inom konstnärlig forskning och konstnärliga
självständiga arbeten ("Ska det vara nödvändigt att skriva text eller
ska det konstnärliga arbetet få stå på egna ben?") Frågan om hur
skrivandet i konstnärlig forskning inverkar eller rentav försvårar
konstnärliga processer har diskuterats återkommande.\footnote{\footcite[Se till exempel Dieter Lesages översikt som
diskuterar detta:][]\{Lesage2009\}.} Den
diskussionen bygger på idén om att teori, praktik och metod är separata
och att för mycket av det ena kommer ha en negativ inverkan på det
andra. Istället, vilket jag argumenterar för i bidraget till antologin
om handledning i konstnärlig forskning \emph{Acts of Creation}, borde det
vara möjligt att:

\begin{quote}
[\ldots{}] reconsider the theory-practice, method-practice and
theory-method relationships beyond their most obvious appearances. If
we can reassess the dual nature of these relationships and begin to
see them as movements instead, continuities from practice to method to
theory and then back, from concept to abstraction to specificity, the
generalising and contextualising power of the theoretical approach may
be less of an obstacle to the practice-oriented artistic researcher
and doctoral candidate.\footnote{\footcite[s.120]{frisk2015}.}
\end{quote}

Här börjar nu själva kärnan av problemet träda fram, för hur kan
hållbara resultat utvinnas om koncept och kontext är i ständig rörelse
och hur kan metoden väljas om ingenting är konstant? Det finns flera
sätt att angripa dessa frågor på men först vill jag poängtera att detta
är en av de inte tillräckligt diskuterade problem i konstnärlig
forskning idag. Det ligger nära till hands att jämföra med annan
forskning, men det kan lika gärna skapa ytterligare förvirring. Det är
utan tvivel så att det stora fältet av vetenskaplig forskning också är
otroligt disparat och i ständig rörelse, och synen på den ideala, rigida
vetenskapliga metoden är överdrivet enhetlig och förenklad. Fördelen för
den traditionella forskningen är dock dess långa historia och kulturella
och sociala acceptans. Samtidigt räcker det inte för den konstnärliga
forskningen att, som ofta sker, återkommande hänvisa till att det finns
experimentella kvalitativa metoder som används inom vetenskapen också
(vilket det finns), eller att det vetenskapliga forskningsfältet är
minst lika heterogent som det konstnärliga (vilket det är). Metoderna i
konstnärliga forskning, vare sig de avviker från vetenskapen eller inte,
måste kunna förklaras och redovisas och metodutvecklingen inom fältet
behöver få ett större fokus.

All konstnärlig praktik befinner sig inte heller i ett ständigt flux.
Det finns flera exempel på praktiker som är konceptuellt stabila, som
till tidig musik. Men i allmänhet finns det mycket rörelse och variation
och det är detta som gör konstnärlig forskning såväl svår att förklara
och definiera, men samtidigt också så otroligt användbar: generella
förklaringsmodeller håller inte. Detta har konsten i allmänhet gemensamt
med de kulturella och sociala forskningsdisciplinerna. Detta är en
egenskap som ska underhållas och inte undertryckas. Resultatet även av
en dynamisk konstnärlig praktik kan analyseras genom såväl kvantitativa
som kvalitativa metoder och forma modeller för framtida forskning. Dessa
modeller kan ge stabila data även om de för sig själva inte
representerar beständiga resultat. Om praktiken som genererade dem
inkluderas i beskrivningen är det större sannolikhet för att de
genererar kunskap som är användbar och därför är det inte i reduktionen,
eller den deduktiva metoden, som resultaten ska sökas i första hand,
utan i komplexiteten och instabiliteten, eller kort, i relationen till
kaos. Det är detta som Deleuze och Guattari i sitt sista samarbete pekar
på som skillnaden mellan filosofi och vetenskap:

\begin{quote}
The object of science is not concepts but rather functions that are
presented as propositions in discursive systems. The elements of
functions are called \emph{functives}. A scientific notion is defined not
by concepts but by functions or propositions. This is a very complex
idea with many aspects as can be seen already from the use to which it
is put buy mathematics and biology respectively. Nevertheless, it is
this idea of the function which enables the sciences to reflect and
communicate. Science does not need philosophy for these tasks. On the
other hand, when an object---a geometrical space, for example
is---scientifically constructed by functions its philosophical
concept, which is by no means given in the function, must still be
discovered. Furthermore, a concept may take as its components the
functives of any possible function without thereby having the least
scientific value, but with the aim of marking the differences in kind
between concepts and functions.

Under these conditions, the first difference between science and
philosophy is their respective attitudes toward chaos. \footnote{\footcite[s.117-8]{deleuze1994}.}
\end{quote}

De fortsätter med att beskriva den vetenskapliga metoden som en process
som saktar ner tiden, som tar ett komplext fenomen och reducerar det
till en enda funktion. Reduktion och representation vilket genererar
tillstånd, funktioner och referentiella propositioner. \footnote{\footcite[s.197]{deleuze1994}.} Även om
vetenskapen inte behöver filosofin, eller konsten, kompletterar de tre
perspektiven varandra och i strävar Deleuze och Guattari efter att
placera det filosofiska tänkandet i relation till vetenskapen och
konsten, och samtidigt undvika att skapa hierarkier dem emellan (jag
kommer återkomma till detta i slutet av kapitlet). Tillblivelse och
transformation framför varande. Begrepp, inte som former som fylls med
innehåll utan som uppstår i relation till varandra i ständig rörelse.
Vad betyder detta för diskussionenen om den konstnärliga forskningens
betydelse och förmåga att skapa resultat relevanta för en värld i kaos?
Olika konstnärliga projekt har olika mål, men hur vet vi i bedömningen
av konstnärlig forskning vad som är avsikten med ett givet projekt, när
det är själva verket och analysen som ska bedömas när dessa kommuniceras
i delvis olikt kodade språk?

Deleuze och Guattaris uppdelning i de tre tankeaktiviteterna konst,
vetenskap och filosofin för det med sig att vissa frågeställningar
faktiskt blir enklare i och med en initial särdelning. Även om ingen av
de tre formerna egentligen är beroende av varandra för sin egen existens
så erbjuder denna beskrivning av dem en potentiell renodling och en
möjlig interaktion med ett gemensamt mål: att bemöta kaoset. Dock finns
det, åtminstone på ytan, ett visst fokus på vad konsten (som konst) i
sig självt uttrycker, respektive vad filosofin respektive vetenskapen
vill uttrycka. Och om tänkande genom konsten inte i huvudsak är kunskap,
och inte heller bygger stabila begrepp, utan är baserad på tidlöshet,
sensationer och känslor, eller affekter, hur användbar är detta
resonemang för konstnärlig forskning:

\begin{quote}
What defines thought in its three great forms---art, science and
philosophy---is always confronting chaos, layng out a plane, throwing
a plane over chaos. [\ldots{}] Art wants to create the finite that restores
the infinite: it lays out the plane of composition that, in turn,
through the action of aesthetic figures, bears monuments for composite
sensations.\footnote{\footcite[s.197]{deleuze1994}.}
\end{quote}

Konsten är inte en syntes av filosofin och vetenskapen och den ena
formen för tänkandet är inte överlägsen någon av de andra, men
möjligheten för dem att bli sammanflätade eller korsa varandra utan att
det för den sakens skull uppstår en sammanblandning dem emellan finns
hela tiden. Jag återkommer till detta i relation till Åsa Stjernas
arbete \emph{Before Sound} i nästa stycke.

Det är ingen tvekan om att det finns många och stora likheter mellan
konstnärlig forskning och vetenskaplig forskning. Men det är heller
ingen tvekan om att det finns stora skillnader. Den kanske främsta
anledningen till att det kan vara svårt att samarbeta över
disciplingränser är samtidigt det som ger konstnärlig forskning sin
särprägel, nämligen den konstnärliga metoden och den konstnärliga
grunden. Man kan tycka att det borde vara enkelt eftersom det
vetenskapliga och det konstnärliga kompletterar varandra så väl och här
ligger en viktig insikt: det konstnärliga kan aldrig bli en vetenskap,
det är bara i synen på konstnärlig kunskap som något annat än
vetenskaplig kunskap som den har relevans. Om vi definierar det
konstnärliga som allt det som vi inte säkert kan säga någonting om med
vetenskapliga metoder, och resten, det vill säga, det som vi kan säga
någonting om utifrån vad vi vet om dess fysiska egenskaper, så blir det
kanske tydligt att det kan vara svårt att samarbeta över gränserna.
Samtidigt öppnar det också upp för helt nya typer av samarbeten i synen
på de som två interagerande kunskapspotential. Baksidan med denna bild
att den i alltför stor utsträckning lutar sig mot en romantisk och
transcendental syn på konsten som riskerar att skapa större problem än
lösningarna den erbjuder. Eller som Eisner skriver:

\begin{quote}
"For me, the defining feature that allows us to talk collectively
about the arts is that art forms share the common mission of achieving
expressiveness through the ways in which form has been crafted or
shape. The arts historically have addressed the task of evoking
emotion. We sometimes speak of the arts as resources that can take us
on a ride. The arts, as I have indicated elsewhere, provide a natural
high. They can also provide a natural low. The range of emotional
responses is enormous. These emotional consequences in relation to a
referent color the referent by virtue of the character of the emotion
that the artistically crafted form possesses." \footnote{\footcite{Eisner2008}.}
\end{quote}

Detta poetiska resonemang är svårt att argumentera mot och det ger en
möjlig bild av konstnärligt uttryck och dess betydelse, men det ger inte
en komplett bild heller. Den, vill jag hävda, kommer Deleuze och
Guattari närmare i som jag kommer återkomma till och denna korta
introduktion ger ett grundläggande underlag till Åsa Stjernas \emph{Before
Sound}.

\section{\emph{Before Sound}}
\label{sec:org050dcf4}
Åsa Stjernas avhandling \emph{Before Sound: Transversal Processes in
Site-Specific Sonic Practice} "utforskar konstnärlig transformation i
platsspecifik sonisk praktik" \footnote{\footcite[s.291]{Stjerna2018}.}. Denna praktik är multidisciplinär
och transformativ och äger rum i det offentliga rummet. Detta skapar i
sin tur sammanhang och situationer som kräver förhandling och
omförhandling i den konstnärlig praktiken och den visar därmed på
komplexiteten i all mänsklig aktivitet. Fyra konstnärliga processer
beskrivs detaljerat och de processer som bidrar till det som rör
avhandlingens centrala frågeställning: själva omförhandlingen och
transformationen av platsen. Detta diskuteras framförallt utifrån Gilles
Deleuze och Félix Guattaris teori. Som framgår av undertiteln är
begreppet transversal centralt i undersökningen. Transversalt ska här
ses som "en förståelse av konstnärlig produktion som etablerandet av
relationer mellan komponenter i ömsesidig kontinuerlig process av
tillblivelse". \footnote{\footcite[s.293]{Stjerna2018}.} Eller, i Stjernas egen engelska formulering:

\begin{quote}
"Transversal" here refers to an understanding of artistic production
as the creation of affective, immanent relations between components in
mutual continuous processes of becoming. These relations span be-
tween material and discursive, and human and non-human, compo- nents.
In exploring the acts performed by the artist within transversal
processes, the aim is to develop explorative approaches and concepts
that might contribute to a more complex understanding of the pro-
cesses at work in site-specific sonic practice.
\end{quote}

Stjerna strävar efter att undvika de traditionellt starka dikotomier som
präglar förståelsen av konstverket och som kan representeras av begrepp
som konstnär-publik, skapande-lyssnande, innanför-utanför eller
kropp-själ. Istället för dessa ontologiskt manifesterade relationer
lyfter hon fram en "förståelse av konstnärlig praktik som befattandet
med affektiva, \emph{immanenta}, kraftrelationer i vilka varje komponent har
agens", det vill säga kapaciteten att både påverka och påverkas.\footnote{\footcite[s.293]{Stjerna2018}.}
Detta är i mångt och mycket i överensstämmelse med Deleuze filosofiska
perspektiv och delvis kan avhandlingen därigenom ses som en
instantiering, eller ett utforskande, av det teoretiska ramverk som
läggs fram.\footnote{\footcite[Se
t.ex. Kapitel 8, avsnitt \emph\{Machinic Interferences in the Oslo
    Opera House as a Smooth and Striated Space.\} ][
s.240-5]\{Stjerna2018\}.}

Utifrån denna ambition är tre forskningsfrågor formulerade:

\begin{quote}


\begin{itemize}
\item På vilket sätt kan jag som konstnär utveckla utforskande
tillvägagångssätt som understödjer en transversal skapandeprocess?

\item Vilka begrepp behöver jag som konstnär kunna artikulera för att
kunna synliggöra och förstå nyanserna av en sådan transversal
process?

\item Vilka konsekvenser har dessa utforskande tillvägagångssätt och
begrepp i förståelsen av den platsspecifika soniska
praktiken? \footnote{\footcite[s.294 (s.27)]{Stjerna2018}.}
\end{itemize}
\end{quote}

Som metoder, här kallade utforskande tillvägagångssätt, pekar Stjerna på
i huvudsak tre strategier: "att kartlägga de affektiva linjerna, att
skapa nya sammankopplingar samt att bli icke-autonom" \footnote{\footcite[s.294]{Stjerna2018}.}. Även
dessa tre är även kopplade till Deleuze och Guattari på olika nivåer,
kanske framförallt metoden att skapa nya sammankopplingar, som i sig kan
ses som en del av den transversala ambitionen. Som en central tes som
lyfts fram på flera platser i avhandlingen, och som är bakgrunden till
titeln, är att ljud är något som ska ses som en effekt av affektiva,
transversala och immanenta processer som samtliga äger rum \emph{före ljudet}
hörs, uppstår eller uppfattas. Etablerandet av nya kopplingar lyfter
fram den konstnärliga processen som både transformativ och
transversal \footnote{\footcite[s.119]{Stjerna2018}.} och dessa kopplingar är helt centrala för de olika
aspekter, eller modaliteter, som Stjerna lyfter fram för sin egen
praktik: sonifiering, teknologi och på-plats installation. Men det är
också genom transversala processer som nya kopplingar skapas mellan
heterogena objekt, vilket är den viktig del av den konstnärliga
praktiken, som ljudet uppstår.

I avhandlingen diskuteras som sagt fyra konstnärliga projekt med
titlarna: \emph{Currents} (2011), An \emph{Excursion to Nairobi} (2013), \emph{The
Well} (2014) och \emph{Sky Brought Down} (2017) och av dessa kommer jag
framförallt fokusera på den första, \emph{Currents}. \emph{Currents} var ett
egeninitierat projekt som utvecklade sig till en beställning av
Ultima-festivalen, en nutida musikfestival i Oslo, avsedd för foajén på
nya operahuset precis invid vattnet i Oslo. Utgångspunkten för verket är
data från mätningar gjorda av Nordatlantiska strömmen. Dessa har gjorts
utanför Färöarna och i samverkan med ett forskningsprojekt som studerar
issmältningen på norra halvklotet till följd av den globala
uppvärmningen. Stjerna tillgång till datan och ambitionen var att
utforska i vilken utsträckning ljud som konstnärligt material kan
mediera frågor av stor politisk relevans och hur dessa frågor kan ge
upphov till förkroppsligade upplevelser i ett publikt sammanhang.

\subsection{Currents}
\label{sec:orge891943}
Anledningen till att jag väljer just detta projekt är på grund av de
frågor som det ställer kring relationen mellan representation och
uttryck och hur de politiska och sociala aspekterna naturligt hamnar i
fokus. Dessutom ligger flera av frågeställningarna nära de jag själv har
arbetat med och känns därför intressant för mig att studera just detta
projekt. Projektet är samtidigt en tydlig illustration av hur en
transversal process kan se ut. De olika komponenterna i de olika faserna
av projektet, såsom forskningsprojektet som höll i datainsamlingen,
processen att extrahera relevanta delar av den vetenskapliga datan,
operahuset som social och politisk plats och plats för själva
renderingen av verket, den konstnärliga utvecklingen av mjuk- och
hårdvara, och Stjerna själv som konstnär \footnote{\footcite[s.133]{Stjerna2018}.} är var och för sig
självständiga agenter och \emph{maskiner}. Utmaningen i avhandlingen är att
artikulera hur dessa kopplingar etableras och består samt vad de gör för
processen. Intressant hade också varit att se hur agensen förändrades
genom processen och om den förändringen är beständig.

En annan aspekt av \emph{Currents} som Stjerna diskuterar i kapitel fem av
avhandlingen är frågan om konstnären som subjekt och dennes begränsade
oberoende eller autonomi, eller snarare, Stjernas önskan att uppnå
icke-autonomi:

\begin{quote}
The second issue that I address in this chapter concerns the artist
subject's minimized autonomy. Through Currents I show the ways in
which the artist subject is implicated in a continuous, affective
relation with a myriad of agencies, all with their specific capacity
to affect the process. This implies a model of daily practice that
exceeds the established idea of an active, sovereign artist-subject
exercising their power to produce a work, which is subsequently placed
in a space. On the contrary, the account given here emphasizes an
ethological process in which I as artist stand in continuous encounter
with the different agencies spanning from the scientific data to the
Oslo Opera, the programmer, as well as the software, which all had a
direct effect on the artistic process of developing \emph{Currents}.\footnote{\footcite[s.133]{Stjerna2018}.}
\end{quote}

Det är klart att den väv av relationer som skissas här ovan också gör
det tydligt att det blir svårt för konstnären att hävda ensidig
bestämmanderätt, och att denne snarare ska ses som en agent av många.
Men valet att uttrycka det som en arbetsmodell som i sig går bortom det
suveräna subjektet är intressant och talande här. Det handlar inte om
att konstnären här blir berövad möjligheten att vara autonom, utan att
hon istället erbjuds ett mer omfattande ramverk som skapar möjligheter
som går bortom det autonoma subjektets möjligheter. Den autonoma
konstnären ses ofta som ett ideal, tätt kopplat till den fria
konstnären, ett begrepp som likväl kan, och bör, ifrågasättas.\footnote{\footcite[Se till exempel ][ s.21]{peters09}.}
Utöver den möjlighet till transversalitet som Stjerna betonar som lyfts
fram genom det icke-autonoma, finns det både etiska och rent
konstnärliga skäl att ifrågasätta frihetsbegreppet. Men, det är viktigt
att de tre centrala delarna här: platsen, verket och konstnärssubjektet,
ses som delar av en helhet, snarare än åtskilda:

\begin{quote}
As I have suggested it through my descriptions, the initial
explorative process, the establishment of spatial perception, the
development of sonic strategies and technology, and the construction
process on site, all emerge as the result of complex, machinic
interconnections that span transversally between the site, the
artwork, and the artist-subject. In this, I advocate a move beyond the
traditional separations that establish these as three distinct
entities.\footnote{\footcite[s.92]{Stjerna2018}.}
\end{quote}

Frågan om sonifiering som metod och hur den kan förstås som en
transversal process diskuteras likaledes. Sonifiering är såväl en
vetenskaplig som konstnärlig metod och Stjerna gör en genomgång av olika
aspekter av de olika förhållningssätten. Som vetenskaplig metod handlar
sonifiering om att skapa en korrespondens mellan den data som ska
sonifieras och det ljudande resultatet så att den förra blir en
representation av den tidigare. Syftet med sonifiering i dessa
sammanhang är att erbjuda en klanglig ingång till en datamängd. Det kan
till exempel handla om att titta på extremt komplex och stratifierad
data som inte är möjlig att läsa, men möjliga att höra. Detta är en typ
av representation som varken går ihop med den teori som detta arbete
vilar på eller nödvändigtvis är särskilt intressant rent konstnärligt,
vilket Stjerna lyfter fram. I \emph{Currents} har sonifiering en betydligt
bredare politisk betydelse och som konstnärlig metod är den tätt
sammanvävd med det teoretiska ramverket i avhandlingen:

\begin{quote}
From this outset, sonification emerges as a practice of transforming
affective forces between bodies----material, social, discursive----in
which the practice of transformation should be considered to
constitute a creative, machinic process that generates new expressions
of becoming. This process is always situated and dependent on its
specific conditions, meaning that the sonification process of Currents
must be understood in relation not only to the scientific data but to
all the other assemblages involved, including the Oslo Opera's unique
ecology.\footnote{\footcite[s.157]{Stjerna2018}.}
\end{quote}

På ett processplan är det inte svårt att se att denna
sonifieringsprocess i allt väsentligt måste ses i relation till det
transversala flöde som alla andra delar av installationen vilar på.
Samtidigt är det en utmaning att se hur detta kan erfaras i
ljudinstallationen. Datan från havsströmmarna är uppdelade i två olika
delar, varav en läses i realtid medan de andra har renderats i
icke-realtid, vilket kanske understryker betydelsen av att se \emph{Currents}
som spatial snarare än temporal. Detta öppnar upp för en annan fråga som
hade varit intressant att gräva djupare i, nämligen relationen just
mellan det spatiala och det temporala, inte minst utifrån Deleuze. Jag
har i huvudsak förstått Deleuze immanensplan som ett alternativ till den
västerländska filosofins upptagenhet med det spatiala, till förmån för
Bergsons temporala tänkande. Detta ställer en intressant fråga om
relationen mellan det temporala och det spatiala i \emph{Before Sound}.

\subsection{Före \emph{Before Sound}}
\label{sec:orgc6e08ea}
Utifrån citatet ur där Deleuze och Guattari diskuterar vad konst som en
form för tänkande är eller kan vara, tecknar de en relativt traditionell
syn på konst, i bemärkelsen att det framför allt är konstverkets
potential som de diskuterar. Kommentarer som "the act of painting that
appears as a painting"\footnote{\footcite[s.197]{deleuze1994}.} antyder ändå att process och resultat
överlappar, men vad betyder det för den typ av transversal och radikal
praktik som Åsa Stjerna bedriver? Är deras resonemang fortfarande
relevanta för den hybrid som konstnärlig forskning utgör eller är det i
huvudsak en något gammaldags syn på estetik och konstnärlig kunskap? Vad
är förhållandet mellan filosofin som begreppsskapande aktivitet och
konsten som estetisk aktivitet?

Att döma av det stora intresse som Deleuze och Guattari har ägnats av
konstnärliga forskare är det ingen tvekan om att deras tänkande har, och
har haft, en hel del relevans för konstfältet. I början av 2000-talet
var det kanske framförallt bildkonsten som vurmade för Deleuze som ett
tag sågs som konstens teoretiska galjonsfigur. Efterhand har man dock
inom musiken också öppnat ögonen för dessa filosofer. I december 2019
hölls den tredje internationella konferensen om Deleuze, \emph{Dare}, vid
konstnärlig forskning på Orpheus Institutet i Ghent, Belgien, och
alldeles nyligen släppte även Leuven University Press den andra delen i
serien \emph{Deleuze and Artistic Research}\footnote{\footcite{deAssis2017}\footnote\{I denna volym
    har jag tillsammans med Anders Elberling bidragit med ett kapitel
    med viss bäring på denna diskussion. Det är en text om vårt
    videoverk \emph{Machinic Proposisitions} som är uppbyggt utifrån
    några av de theorem för deterrioralisering som Deleuze och
    Guattari beskriver i \emph{Mille Plateaux}\nocite{deleuze80} och vilket på det viset är ett
    annat exempel på hur filosofi kan diskuteras genom konstnärlig
    praktik. I \emph{Machinic Propositions} är vi dock noga med att
    påpeka att syftet inte är att skapa filosofi utan att använda
    konstnärlig praktik för att förstå och kritiskt granska
    filosofi. \fullcite[s.121-9]{frisk2017e}\} (den första hette
\emph{The Dark Precursor}.\footcite{deAssis2017}) Dessa är imponerande
samlingar, inte bara sett till innehållet men också till omfånget och
med ett huvudsakligt, dock inte exklusivt, fokus på musik. I Sverige
avslutades förra året det av Vetenskapsrådet(VR) finansierade
projektet \citetitle{hultqvist2019} lett av Anders Hultqvist
(2015-2018) på Göteborgs universitet som också utgick från Deleuze
teoribildning\footcite{hultqvist2019}, samt Klas Nevrins VR-projekt
\emph{Musik i Oordning} med hemvist på Kungliga Musikhögskolan, som
avslutades i december 2018, likaså med tydligt avstamp i Deleuze
tänkande. Nästan parallellt med Åsa Stjerna gjorde tonsättaren Fredrik
Hedelin sin avhandling på Luleå Universitet i vilken han utför en
analys av tre av sina egna solokonserter i vilken begreppet ritornell
blir en kontaktyta mellan musik och Deleuze
filosofi.\footcite{Hedelin2017} I ett projekt tillsammans med Marcel
Cobussen och Bart Weijland gjorde vi ett försök att diskutera
\emph{The Field of Musical Improvisation} (FMI) utifrån
komplexitetsteori och Deleuze och Guattari på ett sätt som relaterar
till den begreppsapparat som Stjerna använder:

\begin{quote}
    Assemblages are wholes whose properties emerge from the
    interactions between parts, for example, interpersonal networks
    such as can be found in the FMI. In other words, assemblages
    cannot be defined by nor do they consist of the properties of
    their constituting parts. Think ecology instead of
    reductionism.\footcite{frisk-cobussen09}
\end{quote}

Det som Stjerna refererar till som de transversala processerna och som
hon använder för att visa på hur den konstnärliga praktiken är
rhizomatisk till sin natur finner en hel del resonans i annan
konstnärlig forskning, även den som inte explicit hänvisar till
Deleuze. Ambitionen att bryta ner de politiska och sociala hierarkier
som i århundraden präglat musiklivet och som fått fart i och med
institutionaliseringen och till viss del professionaliseringen av
musiklivet\footcite{frisk2016b} har diskuterats i många projekt och är
något jag själv har intresserat mig mycket för. I en rad studier från
2006 tittade vi på hur agensen mellan tonsättare, musiker och
teknologi påverkade och till och med hade en avgörande betydelse på de
konstnärliga tillvägagångssätten.\footnote\{Se bland annat
    \fullcite{frisk-ost06} och \fullcite{frisk-ost06-2}.\} En stor del
av mitt fokus i min avhandling från 2008\footcite{frisk08phd} tar sin
avstamp i erfarenheterna från ljudinstallationen \emph{etherSound} och
idéen om ickekontroll och perpetuell rörelse vilket på vissa sätt
ligger nära Stjernas icke-autonoma upphovsperson och transversala
praktik. Inget av dessa exempel tar avstamp i Deleuze utan i min
avhandling var det istället en annan postmodernist, dystopikern Jean
Baudrillard, som utgjorde en del av teorin. I de tidigare studierna
från 2006 var utgångspunkten snarare strukturalism och semiotik, även
om vi relativt snabbt sträckte oss förbi denna startpunkt. Men det
finns även reflektioner och associationer till angränsande fält som
kanske vid första anblick inte är relaterade. Doktoranden och
konsertpianisten Franciska Skoogh pekar på hur MPA (Music Performance
Anxiety) ofta ses som ett individuellt problem, något musikern själv
får kompensera för, när det i själva verket finns en rad strukturella
faktorer som påverkar situationen som konservatorietraditionen, som de
kommersiella aktörerna och den extrema kommodifieringen av
musiken.\footcite{frisk2019} Samtidigt är det viktigt att påminna sig
om att \emph{Before Sound} inte framförallt är ett musikprojekt utan
snarare ett konstprojekt och därför har en delvis annan ingång.
\% \ldots{} till hit.} (den första hette \emph{The
Dark Precursor}.\footnote{\footcite{deAssis2017}.}) Dessa är imponerande samlingar, inte bara sett
till innehållet men också till omfånget och med ett huvudsakligt, dock
inte exklusivt, fokus på musik. I Sverige avslutades förra året det av
Vetenskapsrådet(VR) finansierade projektet lett av Anders Hultqvist
(2015-2018) på Göteborgs universitet som också utgick från Deleuze
teoribildning\footnote{\footcite{hultqvist2019}.}, samt Klas Nevrins VR-projekt \emph{Musik i Oordning}
med hemvist på Kungliga Musikhögskolan, som avslutades i december 2018,
likaså med tydligt avstamp i Deleuze tänkande. Nästan parallellt med Åsa
Stjerna gjorde tonsättaren Fredrik Hedelin sin avhandling på Luleå
Universitet i vilken han utför en analys av tre av sina egna
solokonserter i vilken begreppet ritornell blir en kontaktyta mellan
musik och Deleuze filosofi.\footnote{\footcite{Hedelin2017}.} I ett projekt tillsammans med Marcel
Cobussen och Bart Weijland gjorde vi ett försök att diskutera \emph{The Field
of Musical Improvisation} (FMI) utifrån komplexitetsteori och Deleuze
och Guattari på ett sätt som relaterar till den begreppsapparat som
Stjerna använder:

\begin{quote}
Assemblages are wholes whose properties emerge from the interactions
between parts, for example, interpersonal networks such as can be
found in the FMI. In other words, assemblages cannot be defined by nor
do they consist of the properties of their constituting parts. Think
ecology instead of reductionism.\footnote{\footcite{frisk-cobussen09}.}
\end{quote}

Det som Stjerna refererar till som de transversala processerna och som
hon använder för att visa på hur den konstnärliga praktiken är
rhizomatisk till sin natur finner en hel del resonans i annan
konstnärlig forskning, även den som inte explicit hänvisar till Deleuze.
Ambitionen att bryta ner de politiska och sociala hierarkier som i
århundraden präglat musiklivet och som fått fart i och med
institutionaliseringen och till viss del professionaliseringen av
musiklivet\footnote{\footcite{frisk2016b}.} har diskuterats i många projekt och är något jag själv
har intresserat mig mycket för. I en rad studier från 2006 tittade vi på
hur agensen mellan tonsättare, musiker och teknologi påverkade och till
och med hade en avgörande betydelse på de konstnärliga
tillvägagångssätten. En stor del av mitt fokus i min avhandling från
2008\footnote{\footcite{frisk08phd}.} tar sin avstamp i erfarenheterna från ljudinstallationen
\emph{etherSound} och idéen om ickekontroll och perpetuell rörelse vilket på
vissa sätt ligger nära Stjernas icke-autonoma upphovsperson och
transversala praktik. Inget av dessa exempel tar avstamp i Deleuze utan
i min avhandling var det istället en annan postmodernist, dystopikern
Jean Baudrillard, som utgjorde en del av teorin. I de tidigare studierna
från 2006 var utgångspunkten snarare strukturalism och semiotik, även om
vi relativt snabbt sträckte oss förbi denna startpunkt. Men det finns
även reflektioner och associationer till angränsande fält som kanske vid
första anblick inte är relaterade. Doktoranden och konsertpianisten
Franciska Skoogh pekar på hur MPA (Music Performance Anxiety) ofta ses
som ett individuellt problem, något musikern själv får kompensera för,
när det i själva verket finns en rad strukturella faktorer som påverkar
situationen som konservatorietraditionen, som de kommersiella aktörerna
och den extrema kommodifieringen av musiken.\footnote{\footcite{frisk2019}.} Samtidigt är det
viktigt att påminna sig om att \emph{Before Sound} inte framförallt är ett
musikprojekt utan snarare ett konstprojekt och därför har en delvis
annan ingång.

\section{Metoden och resultat}
\label{sec:orgda3e826}
Som tidigare nämnts kallar Stjerna sina metoder för \emph{explorative
approaches}. Givet hur nära kopplade de är till teorin är det sannolikt
en klok strategi, även om det också kan skapa förvirring. De tre
övergripande utforskande tillvägagångssätten \emph{att kartlägga de affektiva
linjerna}, \emph{att skapa nya sammankopplingar} samt \emph{att bli icke-autonom}
(som diskuterades en del i förra avsnittet). Denna nära koppling mellan
teori och metod gör det på ett sätt enklare att diskutera utmaningarna i
projektet. Om vi fortsätter utifrån linjen att Deleuze och Guattaris
filosofi framförallt sysslar med begreppskapande (skapande av koncept),
är det inte långsökt att tänka sig att deras verktyg användas för att
skapa nödvändiga begrepp för att förstå och analysera den konstnärliga
praktiken. Frågan är om resultatet av den processen först och främst är
kunskap om konst eller om det är filosofi, men oavsett svaret så kan
resultatet ha validitet som forskning. En annan utmaning till följd av
denna användning av metod är att det av uppenbara skäl helt enkelt kan
uppstå en sammanblandning mellan teori, metod och resultat med utfallet
att den konceptuella stabiliteten blir lidande. Men samtidigt, som också
diskuterades ovan, är detta i realiteten ett sätt att hantera metod och
teori i konstnärlig forskning. Jag ska återkomma till det.

I Stjernas fall uppstår inte dessa problem, delvis tack vare att hon
handskas varsamt och ytterst konsekvent med begreppen. Dessutom ska man
inte glömma att teorin i \emph{Before Sound} i stor utsträckning relateras
till den kontextualisering och positionering som görs i kapitel två:
\emph{Contextualisation--Artistic Strategies within the Field}. Denna typ av
situering är något som ibland förbises i konstnärliga avhandlingar
vilket kan få till följd att de konstnärliga resultaten blir svåra att
bedöma: Om det är oklart inom vilket konstnärligt fält
konstnären/forskaren rör sig kan det vara svårt att korrelera de
konstnärliga resultaten till konstfältet i stort. Samma problem uppstår
om fältet som beskrivs är alltför disparat eller alltomfattande, eller
alltför smalt. Eller, om fältet som beskrivs är ett annat än ämnet för
avhandlingen eller forskningen. Det finns flera anledningar till att det
tagit lång tid för konstnärlig forskning att hitta rätt på detta område
men några uppenbara skäl är:

\begin{enumerate}
\item Bristen på etablerade modeller för att referera till alla typer av
konstverk och svårigheten beskriva dem på ett sätt som bidrar till
att skapa en bild av ett fält är påtaglig. Ett sätt runt detta är att
göra som Stjerna gör och referera till en (vetenskaplig) studie av
verken i fråga.\footnote{\footcite[Se
t.ex. referenser till Cox och LaBelle: ][s.45]\{Stjerna2018\}.} Detta är i grund och botten ett beprövat sätt
att såväl inkludera den större diskursen om ett verks validitet för
ett specifikt sammanhang och kan skapa trovärdighet i argumentationen
när möjligheten att lyssna och uppleva, om än genom en dokumentation,
också erbjuds. En eventuell baksida med denna metod är att
argumentationen kan göra sig beroende av en annan disciplin, som
filosofi, eller musikvetenskap i det här fallet, eller rent av
resultera i en i huvudsak vetenskaplig undersökning. Detta är i sig
inte ett problem, jag menar att konstnärlig forskning är i grunden
tvärvetenskaplig, men det är samtidigt viktigt att det
tvärvetenskapliga hanteras på ett sätt så att det understödjer, inte
raserar, den konstnärliga undersökningen.

\item Det är tydligt att man i konstnärliga avhandlingar har lättare att
referera till etablerade verk och studier av verk. Men utan en intern
diskussion, och utan att varje avhandling också relaterar till det
fält som är i dess omedelbara närhet, såsom samtida konstnärlig
forskning eller konstnärlig praktik, riskerar studiens validitet att
begränsas. Stjerna går elegant runt detta genom att med omsorg
beskriva verktygen, tillvägagånssätten och det teoretiska
perspektivet hon använder vilket skapar en tydlighet. Men jag skulle
vilja gå så långt som att säga att den största utmaningen vi har för
konstnärlig forskning idag är just bristen av ett befintligt
forskningsfält där den konstnärliga undersökningen står i centrum.
Detta kan byggas genom att doktorander samarbetar och refererar till
varandras arbeten, gärna kritiskt men alltid noggrant, och genom att
större forskningmiljöer skapas. Ett nytt samarbete mellan Kungliga
Musikhögskolan och Musikhögskolan i Piteå, Luleå Tekniska
Universitet, samlar mer än tio doktorander, en PostDoc och flera
seniora forskare i vad vi kallar lab tre gånger per år. Det är en
början men vi behöver fler samarbeten där utgångspunkten är
diversitet, samarbete och kritisk granskning. 2018 skrev vi i
programförklaringen till samarbetet mellan LTU och KMH att vi vill
ytterligare arbeta för att skapa:

\begin{quote}
[\ldots{}] former för kunskapsbyggande i konstnärlig forskning som är
baserat på konstnärlig kunskap. Även teoretiskt drivna resonemang
kan här ta sin utgångspunkt i den konstnärliga praktiken varför
själva praktiken är central i seminarier och labsessioner. De
gemensamma seminarier som vi planerar i Piteå och Stockholm ska ta
formen av ett laboratorium som alla gemensamt ska bidra till den
fortsatta utvecklingen av detta format. En viktig förutsättning är
att skapa ett arbetsklimat som bygger på förtroende av samma slag
som uppstår i konstnärliga samarbeten, och som därigenom skapar
förutsättningar för en kritisk dialog som kan gå på djupet in i
konstnärliga processer.\footnote{\footcite{frisk2018:irl}.}
\end{quote}

Idén är att sätta den konstnärliga praktiken i centrum i en skyddad
seminariemiljö som erbjuder möjligheten att experimentera. I detta
initiativ hoppas vi att det ska bli möjligt att i större utsträckning
begreppsliggöra praktiken och undersökningen vilket samtidigt kan
underlätta att arbetena också diskuteras, kritsikt granskas av
jämlikar, samt att resultaten, såväl som metod och teori, delas.

\item Även det större fältet av konstnärlig praktik, utanför konstnärlig
forskning, måste vara möjligt att referera till. Musikhögskolan i
Piteå och konstnärliga fakulteten vid Göteborgs universitet har
etablerade metoder för att registrera konstnärliga verk vilket gör
det enkelt att hänvisa till dessa, och KMH och Musikhögskolan i
Ingesund, Karlstads universitet, har påbörjat ett arbete som
förhoppningsvis leder till att vi bidrar till ett fält som inkluderar
akademien och skapar förutsättningar för forskningsanknuten
undervisning. Om möjligheten för att publicera konstnärliga arbeten i
forskningsdatabaser erbjuds kommer det uppstå en ackumulativ effekt
som kan få mycket stor betydelse för hur konst som kunskapsfält kan
utvecklas.

\item Utan tydliga begrepp för konstnärlig forskning och konstnärlig
praktik kan det vara svårt att på ett öppet sätt beskriva det
konstnärliga fält som är det centrala för studien, utan att samtidigt
bli låst av det. Att använda filosofi, som tidigare diskuterats,
eller andra discipliner är här en möjlighet som samtidigt kan bidra
till en breddad förståelse för detta forskningsfält. Dock är det
nödvändigt att även begreppen växer fram i gemensamma miljöer där
resultat och diskussioner delas.
\end{enumerate}

Det ackumulativa kunskapsbyggandet som kan bli resultatet av en stabil
forskningsmiljö där forskare på olika sätt bygger vidare på varandras
arbeten på ett genomskinligt och konsekvent sätt är en förutsättning för
att konstnärlig forskning ska få respekt och förtroende som ett
självständigt kunskapsfält, men också för att den interna
kunskapsutvecklingen ska ta fart. Detta sker naturligtvis redan idag i
viss utsträckning, men här finns ett stort utrymme för utveckling. I ett
nytt fält kan det vara naturligt att man framför allt vänder sig till
teoribildning utanför sitt eget fält, men nu är det viktigt att i ännu
större utsträckning rikta blicken också mot annan konstnärlig forskning
för att utvärdera och bygga vidare på dess teori, metod och resultat.
Jag menar att konstnärlig forskning inte bara är tvärvetenskaplig utan
också är multidisciplinär till sin natur, det vill säga att den i vissa
fall rent av är beroende av andra discipliner än det rent konstnärliga
för att kunskapen ska kunna kommuniceras såväl i som utanför dess egen
domän. Symptomatiskt beskriver även Stjerna arbete med
ljudinstallationer som en multidisciplinär praktik, och själva
avhandlingen i sin helthet kan även den ses som multidisciplinär. Det är
dock viktigt att förstå den politisk dimension som det tvärdisciplinära
pekar mot. Samverkan är ett ledord för samtliga universitet och
högskolor idag och tvärvetenskap har i vissa fall blivit ett neoliberalt
självändamål. Detta gagnar inte alltid utvecklingen av ett
forskningsfält där behoven också behöver komma inifrån, snarare än att
de läggs på utifrån.

I Stjernas avhandling harmonierar det sätt hon bygger upp det teoretiska
och metodologiska ramverket i stor utsträckning med hur jag i tidigare
nämnda bokkapitlet föreslår att den den gängse uppfattningen av
relationen mellan teori, metod och praktik behöver omformuleras. Även om
Stjerna inte beskriver det explicit är det min uppfattning att hon
bygger upp definitionen av begreppet "sound art", och dess ontologiska
underbyggnad, genom att korrelera sin egen erfarenhet som praktiker med
en filosofisk och musikvetenskaplig genomgång av hur begreppet har
etablerats.\footnote{\footcite[Se kapitel
tre]\{Stjerna2018\}.} Hon ger följande beskrivning av praktiken:

\begin{quote}
To engage in sound installation as a site-specific practice is thus to
position oneself, as an artist, as a node in the heterogenic field of
what often is referred to as "sound art" respectively "sound art in
public space." It is to understand that sound installation, in all its
specificity emanates from a variety of different practices and
traditions, which together generate a spatially explorative,
multi-disciplinary practice.\footnote{\footcite[s.42]{Stjerna2018}.}
\end{quote}

Även med en rudimentär förståelse av Deleuze och Guattaris filosofi är
det redan i detta citat möjligt att se hur valet av teori är
välmotiverad. Det multidisciplinära angreppssättet förutsätter att de
olika delarna i undersökningen är sammankopplade och hur de kommunicerar
med varandra, vilket är själva kärnan i hur begreppet transversalitet
ska förstås. I en miljö som är genuint multidisciplinär är det
nödvändigt att ha en metod som tillåter obruten kommunikation mellan de
olika delarna av projektet och begreppet transversalitet användes från
början av Guattari i ett liknande syftet, som en kritik mot den
dualistiska synen på relationen mellan analytiker och analysand:

\begin{quote}
The concept of transversality emerges in part out of Guattari's
prolonged critique of the 'personological' understanding of language
at work within psychoanalysis, and, specifically, within Lacanian
versions of analysis. While not initially conceptualized in terms of
enunciation, transversality---in Guattari's early writings
institutional transference (later reframed as 'group transversality')
--- aims to capture the unconscious as an investment of the broader
elements and processes within the specific social setting of the
hospital, a pattern of investment that would come to light only with
the greatest difficulty in the dyadic enunciative setting of the
analyst's consulting room.\footnote{\footcite[s.234]{Goffey2015}.}
\end{quote}

Själva begreppet bär alltså redan från början med sig det som Stjerna
beskriver som ett resultat: en sammanvävd transversal process som
omformulerar hierarkier till kontinuerliga och i vissa fall spatiala
system. Här ingår relationen mellan konstnärssubjektet och publiken, som
visserligen har utsatts för kritik sedan 1960-talet.\footnote{\footcite[s.48]{Stjerna2018}.} I \emph{Before
Sound} är fokus processerna i skapandet och produktionen och publiken
framträder inte direkt som en agent som diskuteras. Det transversala
utspelar sig därför primärt mellan platsen, konstverket och
konstnärssubjektet:

\begin{quote}
In this doctoral research, the concept of assemblage has enabled me to
articulate a mode of artistic practice in which site-specific sonic
conditions and production operate as immanent, inter-relational,
machinic, and transversal processes. I acknowledge the importance of
this way of thinking in the subtitle of the thesis, "Transversal
Processes in Site-Specific Sonic Practice," and its influence can be
seen in the previous chapter's presentation of the field. As I have
suggested through my descriptions, the initial explorative process,
the establishment of spatial perception, the development of sonic
strategies and technology, and the construction process on site, all
emerge as the result of complex, machinic interconnections that span
transversally between "the site," "the artwork," and the
"artist-subject." In this, I advocate a move beyond the traditional
separations that establish these as three distinct entities.\footnote{\footcite[s.92]{Stjerna2018}.}
\end{quote}

Kanske kan man därigenom dra slutsatsen att den transversala processen
är både metod och resultat? Åtminstone är det som metod transversalitet
beskrivs i citatet ovan och som sådan borde den vara intressant även i
annan konstnärlig forskning. Det undermedvetna, som Guattari diskuterar
i citatet ovan och som jag återkommer till längre fram i kapitlet, har
en del med den konstnärliga upplevelsen att göra, och om transversalitet
kan bidra med att begreppsliggöra det som sker även i den konstnärlig
processen så skulle en del vara vunnet. Och, som sagt, \emph{Before Sound}
visar att det kan vara möjligt.

Att begreppet härstammar från Guattaris önskan att fånga förståelsen av
det undermedvetna är inte oväsentligt, inte heller dennes brott med med
den Lacanska traditionen för psykoanalys. Som tidigare nämnts så bygger
mycket av den filosofi som Deleuze och Guattari utvecklade tillsammans
på en kritisk granskning av bland annat den Freudianska teoribildningen.
Lite förenklat kan vi knyta Guattaris ambition att bryta med den binära
analytisk modell som analytiker/analysand innebär till Deleuze kritik av
det transcendentala tänkandet som har varit så central för Europeisk
filosofi. Genom Guattaris analytiska grupptransversalitet kunde man
sannolikt närma sig själva terapisessionen mer som en process av
tillblivelse genom det undermedvetna (nu som en produktiv kraft),
snarare än som en aspekt av det medvetna (eller omvänt). Detta påminner
samtidigt om de försök att dekonstruera de binära eller dyadiska
relationerna mellan olika agenter i den konstnärliga produktionen som så
många konstnärliga forskare, inklusive Åsa Stjerna, har varit upptagna
med. Frågan är hur vi från denna förståelse av relationerna mellan
aktiva och sammanvävda komponenter i det konstnärliga arbetet kan ta oss
mot en insikt i vad dessa relationer säger om den konstnärliga
praktiken.

För att förstå varför det undermedvetna är relevant i det här
sammanhanget vill jag ta upp en text av Gregory Bateson som jag har
använt flera gånger tidigare. Bateson var en brittisk antropolog,
sociolog, filosof och cybernetiker som har en nära relation till
Deleuze. Det finns ett fåtal referenser till Bateson hos Deleuze, bland
annat två i \emph{Mille Plateaux},\footnote{\footcite{deleuze80}.} men det finns de som menar att
inflytandet från Bateson egentligen var betydligt större än vad som
framgår av referenserna.\footnote{\footcite[Se
t.ex. ][ Begrepp som \emph{rhizome}, \emph{double bind}, och
\emph{schizoanalysis} som alla var viktiga för Deleuze och Guattari
diskuterades långt tidigare av Bateson, även om just \emph\{double
    bind\} introducerades av Nietsche.]\{Shaw2015\}.} Det som gör Bateson intressant i den
specifika diskussionen om konstnärlig kunskapsbildning är dock hans syn
på just det undermedvetna och hur olika typer av information och
upplevelser kodas i hjärnan, såväl som i kroppen.

I Freudiansk teori delar man upp mental aktivitet i primära och
sekundära processer. De primära är icke-verbala och drömlika, och
föregriper de sekundära som är det reflekterande och medvetna jagets
uttryck. Konst är generellt "an exercise in communicating about the
species of unconsciousness [\ldots{}] a play behaviour whose function is
[\ldots{}] to practice and make more perfect communication of this
kind."\footnote{\footcite[s.137]{bateson72}.} Nu kan det framstå som att vi har återinfört en separation
mellan det inre, de primära processerna, och det yttre, de sedundära.
Sannolikt så var det bland annat denna uppdelning som Guattari ville
komma åt när han försökte tänka om terapisituationen. Men dessa två
kategorier av processer behöver inte vara väsensskilda utan kan snarare
ses som två möjliga, och i vissa fall parallella, sätt att koda kunskap
och erfarenhet (eller affekter). Då är inte frågan hur vi ställer de mot
varandra, utan hur vi kommunicerar mellan, eller inom dem. Bateson
skriver:

\begin{quote}
[The] algorithms of the heart, or, as they say, of the unconscious,
are, however, coded and organized in a manner totally different from
the algorithms of language. And since a great deal of conscious
thought is structured in terms of the logics of language, the
algorithms of the unconscious are double inaccessible. It is not only
that the conscious mind has poor access to this material, but also
that when such access is achieved. \emph{e.g.}, in dreams, art, poetry,
religion, intoxication, and the like, there is still a formidable
problem of translation.\footnote{\footcite[s.139]{bateson72}.}
\end{quote}

I inledningen pekade jag i all korthet på hur konstnärlig kunskap ofta
inte utan vidare låter sig beskrivas verbalt. I ljuset av detta kan det
vara tilltalande att se en översättning från det omedvetna till det
medvetna som lösningen, men det finns flera saker som behöver lyftas för
att vi på ett hållbart sätt ska kunna ta ställning till problemet om
"översättning". För det första har vi frågan om begreppsliggörandet av
den konstnärliga praktiken, det vill säga processen av att skapa begrepp
som gör det möjligt att artikulera en kunskap. I \emph{Before Sound} gör Åsa
Stjerna detta, bland annat genom att använda Deleuze begreppsapparat.
Detta leder henne också till vissa specifika resultat som i sin tur
relaterar till bland annat det platsspecifika, men också till hur själva
praktiken ter sig. Men om man tänker sig att man är i behov av att
formulera egna begrepp så kan man hamna inför Batesons utmaning: Hur det
är möjligt att omformulera eller översätta en konstnärlig strategi till
en verbal utan att den samtidigt förlorar mening eller fastnar i en
meningslös rad av metaforer eller representationer?

Samtidigt finns det inomkonstnärliga begrepp som inte behöver en
översättning för att fungera. För detta är den kontextualiseringen av
projektet som diskuterades ovan viktig för att begreppen som används ska
få pregnans och tillåter att diskursen inom fältet blir användbar. I
musik kan till exempel musikteoretiska begrepp nyttjas i detta syfte,
förutsatt att dessa relateras till praktiken på ett användbart sätt.
Både Deleuze och Guattari i och Bateson i diskuterar dock framförallt
det som den konstnärliga upplevelsen ger upphov till, rent kognitivt,
snarare än den kreativa processen i sig. Mycket konstnärlig forskning,
så även Åsa Stjernas avhandling, beskäftigar sig framförallt med hur
processen att \emph{göra} konst fungerar och i den undersökningen så kan
resultatet, förutom att det är konst, vara ett sätt att validera
utforskandet av processen (vilket dock fortfarande gör det angeläget att
kunna diskutera resultatet).

Det är inte säkert att den konstnärliga processen som leder fram till
ett konstnärligt resultat är enkelt jämförbar med upplevelsen av att
erfara resultatet. I vissa fall kan det vara så men i andra fall, som i
\emph{Before Sound}, ligger delar av processen närmare ställningstaganden som
har med praktiska omständigheter att göra; Hur fungerar en sladd? Hur
kan en högtalare installeras? etc. Är även dessa kodade i det
"omedvetnas algoritmer", för att använda Batesons terminologi, eller är
de del av en process som egentligen ligger närmare andra
forskningsdiscipliner än vad man kanske först vill tro? Jag menar att
det kan vara så i vissa fall, men att vi samtidigt inte får glömma att
den centrala aspekten av konstnärlig forskning att det är en konstnärlig
sensibilitet som ligger bakom valen som görs i det konstnärliga arbetet.
Förmågan att föreställa sig det konstnärliga resultatet som just konst,
även i arbetet med tekniska installationer är avgörande och
signifikativt för konstnärligt arbete. Av den anledningen är det svårt
att helt komma bort från det affektiva eller det undermedvetnas logik
när vi vill beskriva den konstnärliga forskningsprocessen.

Frågan är om det egentligen handlar mindre om en översättning och mer om
att förstå hur vi kan förhålla oss till olika former för mänsklig
kommunikation. Guattari ville komma runt det han benämnde det
"personologiska" språkbruket i pyskoterapin, och Bateson pekar på att
konsten har en kommunikationskapacitet som gör den mer lik till exempel
andliga upplevelser, berusning och drömmar. Utmaningen är inte att det
finns olika logiska typer av medvetande och kunskap, utan hur vi kan
förstå dem genom en kommunikativ helhet. Klart är i alla fall att de
transversala processerna kan spela en stor roll här, men kvar är frågan
om dessa i sig kan ge oss mer stringent formulerad konstnärlig forskning
där det blir lättare för andra forskare att förstå och relatera till
resultat, process och metod.

Att det som jag här förenklat benämner det inre och det yttre inte är
två åtskilda paradigm för förståelse, kunskap och kommunikation kan inte
nog poängteras. Konst har länge präglats av idén om det "rena" och
"inre" inre uttrycket som ofta ställs i kontrast till det yttre
"befläckade". I denna modell är det inre idealistiskt, ärligt och
transcendentalt, och det yttre kan vara kommersiellt, beräknande och
materialistiskt. Även om det är förhållandevis lätt att ta avstånd från
dessa grovt tillyxade kategorier har de haft ett stort inflytande över
hur konst- och musikvärlden har utvecklat sig. Denna utveckling
uppmuntrar sökandet efter idéen om det rena uttrycket, det som passerar
förbi medvetandet, förbi det självmedvetna jaget. Ibland är strävan
efter originaliteten själva källan till sökandet efter det av
medvetandet obesudlade uttrycket, färgat av upplysningens bild av
identiteten; om varje individ är unik och oberoende borde också det
genuint personliga \emph{uttrycket} vara originellt. Saxofonisten Ornette
Coleman, till exempel, talar om strävan efter ett så spontant skapande
som möjligt och om en kreativitet utan minne.\footnote{\footcite[s.117]{litzweiler92}.} Han talar om hur
hans spel innan han nådde framgångar var mera ärligt än det sedan blev
och valde att börja spela trumpet och violin (som han var nybörjare på)
för att kunna spela och samtidigt slippa onödig kunskap.\footnote{\footcite[Intervju med
Ornette Coleman i][s.33]\{taylor77\}.}
Naturligtvis var Coleman lika originell efter sitt genombrott som före,
och retoriken här speglar till stor del den sociala och politiska
tidsandan som rådde, men kan ändå sägas peka på kraften i bilden av
personlig originalitet. På skivan \emph{The empty foxhole} från 1966\footnote{\footcite{coleman66}.}
spelar Coleman tillsammans med sin tioåriga son Denardo Coleman på
trummor och beskriver sin tillfredställelse över att spela med någon som
inte behövde bry sig om kritiker eller konsertarrangörer, utan som kunde
spela och vara fri.\footnote{\footcite[s.121]{litzweiler92}.} Detta hör Coleman när han lyssnar på Denardo
men också för att han har förmågan att lyssna på sig själv; han kan
konstatera att sonen besitter en egenskap han själv har förlorat. Det
yttre lyssnandet, att lyssna på den andre, kompletteras av det inre
lyssnande. Och, jämfört med att lyssna på den andre så är det i vissa
fall betydligt svårare att lyssna på sig själv. Som konstnärlig metod är
det utvidgade lyssnandet central i musikalisk konstnärlig praktik och
det är möjligt att vidareutveckla praktiken enbart genom lyssnande.
Colemans önskan att släppa taget om det invanda och inlärda kan ses som
ett försök att etablera nya transversala relationer mellan den medvetna
och språkligt kodade viljan att förnya, och den konstnärligt kodade
kunskapen om hur detta ska, eller skulle kunna, gestaltas. Utan att göra
en djupare analys av metoderna (lyssnandet), är risken överhängande att
landa i en syn på en relation mellan det trascendentala inre och det
fysiska yttre som i grunden är hierarkisk: det inre, transcendentala är
att föredra framför det yttre. Den bilden har inte frigjort sig från det
politiska bagage den bär med sig och gör det därför inte lättare att
beskriva vad konstnärlig kunskap är.

En anledning till att vi har en tendens att se konstnärlig praktik som
en individuellt artikulerad form för kunskap är att vi i huvudsak ser
konstnärlig verksamhet som en individuellt situerad praktik. Det är den
i vissa fall, och utan tvekan är detta den romantiska bilden av konsten
som något som kretsar kring ett solipsistiskt geni. Stjerna tar spjärn
mot denna bild när hon diskuterar sin metod:

\begin{quote}
established traditions in contemporary art practice still harbour
segments of binaries that separate an autonomous active (white, male)
subject and a (passive) urban text. Rejecting this traditional view,
in proposing that we become non-autonomous, I advocate that we view
the artist-subject's agency in artistic production as
transversal.\footnote{\footcite[s.119-20]{Stjerna2018}.}
\end{quote}

Det finns många anledningar, även utanför argumenten som förs fram i
detta kapitel, att motverka denna normativitet och jag tror att
konstnärlig forskning är ett utomordentligt väl anpassat sätt att göra
det på, åtminstone på det konstnärliga fältet.

Slutsatsen som kan dras av detta resonemang är att det är lätt att hamna
i en dubbelt problematisk situation när konstnärlig forskning
diskuteras: Först är det nödvändigt att beskriva konstnärlig praktik som
något som är byggt på kunskap och erfarenhet och som inte är internt,
mystiskt, hemligt eller underligt. Att skapa trovärdighet kring detta
argument är svårt på grund av den sociala och politiska starka
föreställningen om konstens, respektive (den vetenskapliga) forskningens
funktion i samhället. Först när det är möjligt att i någon mening
dekonstruera denna bild är det möjligt att påvisa att det i konstnärlig
praktik finns kunskap som har ett allmängiltigt värde och att detta kan
diskuteras, kritiseras och kommuniceras som forskning, och i interaktion
med andra forskningsmiljöer. Om inte den första delen av argumentet
finner trovärdighet, kommer inte den andra delen av det göra det heller.
Konstens roll som kunskapsform i ett större perspektiv, bortom enskilda
projekt, kommer jag att diskutera i nästa avsnitt.

\section{Metodologisk dynamik}
\label{sec:orgeb5425f}
I en avhandling som är gränserna mellan praktik, metod och teori som
tidigare nämnts inte tydligt artikulerade utan ständigt rörliga.
Terminologin hämtad från Deleuze och Guattari är till exempel i vissa
fall såväl metod, teori och i vissa fall tydligt relaterade till
resultatet, och det är inte alltid från början självklart vad som är
avsikten (detta är dock inte en svaghet i Stjernas fall). Om vi tittar
på den första forskningsfrågan, /På vilket sätt kan jag som konstnär
utveckla utforskande tillvägagångssätt som understödjer en transversal
skapandeprocess?/, så är det klart att den transversala
skapandeprocessen är ett mål i sig, men lika tydligt är det i kapitel
fem att Stjerna genom sin praktik redan i början av projektet har
etablerat transversala kopplingar \footnote{\footcite[s.145]{Stjerna2018}.}. Det transversala är alltså i
vissa fall både metod och resultat. Stjerna pekar på att det finns ett
stort behov för nya konceptuella verktyg och tillvägagångssätt som kan
gå bortom representation och transcendenta hierarkier, och framför allt,
teorier som kan synliggöra hur transformation etableras utifrån en
gedigen förståelse för konstnärliga processer.\footnote{\footcite[s.85]{Stjerna2018}.} Det är i princip
detta avhandlingen sedan visar på genom en syn på själva
ljudinstallationen som en transformativ praktik som löper mellan ett
flertal konstnärliga strategier som var och en är en transversal
process.

Det är ännu svårt att se konstnärlig forskning som en disciplin som
genererar tydliga resultat utifrån väl beprövade metoder. Istället finns
det en rad möjliga artikulationer av kunskapsutveckling som alla pekar
på att det i den konstnärliga praktiken finns en epistemologisk
potential, ett möjligt kunskapssystem som, om den begreppsliggörs, kan
ha stort inflytande på en rad olika fält. I sammanhanget kan det påpekas
att denna tro på konsten som kunskapsbärande är delvis överensstämmande
med Deleuze och Guattaris resonemang i \emph{What is Philosophy?}, där de som
tidigare nämnts definierar tre kreativa metoder för tänkande:
\emph{filosofi}, \emph{vetenskap} och \emph{konst}. Egentligen kan man gå ännu längre
och säga att det i princip överensstämmer med Deleuzes hela filosofiska
gärning att se konst som en form för kunskap eller som en form för
tänkande. Men om konst är kunskap, eller bär på en kunskapsbärande
potential, vari består den? Deleuze och Guattari skriver vidare:

\begin{quote}
What about the creator? It is independent of the creator through the
self positing of the created, which is preserved in itself. What is
preserved---this thing for the work of art---\emph{is a bloc of sensations,
that is to say, a compound of percepts and affects}. [\ldots{}] The artist
creates blocs of percepts and affects, but the ony law of creation is
that the compound must stand up on its own. [[][s.164,
kursivering av författaren.]]
\end{quote}

Att konstverket i sig självt, och oberoende av upphovspersonen, har en
potential är tydligt, likaså att det kan och bör frigöra sig själv, men
i konstnärlig forskning är det ofta praktiken -- och vad för slags
kunskap den kan bära eller föra med sig -- snarare än resultatet, som
står i centrum. Det finns många exempel på hur denna kunskapsutveckling
kan se ut men här ska jag ge tre korta exempel som är i större eller
mindre grad hämtade från musikfältet.

\begin{enumerate}
\item Utöver att studera den konstnärliga praktiken som en kunskap i och
för sig självt, är det möjligt att se den som ett sätt att förstå
annan kunskap såsom teknik eller filosofi. I dessa fall kan man se
den konstnärliga praktiken som en testbädd för ett konceptuellt
ramverk som är omfattar andra discipliner än bara konsten. Det finns
en aspekt av detta i \emph{Before Sound} där de filosofiska koncepten
prövas mot en existerande praktik och sättet som praktiken utvecklar
sig kan då ses som ett utforskande av filosofin. Begreppet
\emph{immanensplanet} är ett exempel på ett koncept som är viktigt för
Stjerna -- och helt centralt för Deleuze filosofi -- och som kan ses
få, om inte en förklaring så en praktisk applikation, genom Stjernas
konstnärliga praktik. I teknologisammanhang kan man föreställa sig
att en teknik, säg ett programmeringsgränssnitt eller en specifik
hårdvara, utforskas i en konstnärlig praktik. Den konstnärliga
metoden, som i detta fall ska ses som en experimentell praktik där
koncept prövas och utvärderas baserat på hur bra eller dåligt de
interagerar med den konstnärliga ambitionen, används för att validera
teknologin. En undersökning på denna nivå kan mycket väl leda fram
till att tekniken som studerats bedöms annorlunda än den hade gjort i
ett rent tekniskt sammanhang: en teknologi som i allt väsentligt ses
som funktionell och stabil från ett tekniskt synsätt kan framstå som
mindre användbar genom en konstnärlig undersökning. Det konstnärliga
ramverket behöver i sig inte vara experimentellt utan kan mycket väl
följa en befintlig tradition helt idiomatiskt, det är metoden som är
experimentell: kritiskt evaluering genom konstnärlig metod. Detta
angrepssätt skulle utan tvekan även generera insikter om den
konstnärliga praktiken. Skulle studien även utforskas med en
filosofisk begreppsapparat som Stjerna gör i \emph{Before Sound} så skulle
de tre perspektiven konst, vetenskap och filosofi samverka och
komplettera varandra på ett sätt som påminner om hur Deleuze och
Guattari föreslår i .\footnote{\footcite{deleuze1994}.}

Insikten om att den konstnärliga sensibiliteten kan behövas i större
utsträckning än vad vetenskapstraditionen kanske fram till nu har
velat göra gällande kommer dock inte bara från konstnärlig forskning
eller filosofin. Det samarbete som Kungliga Musikhögskolan och
Kungliga Tekniska Högskolan initierade 2016 byggde till exempel på
KTH:s insikt att en framtida ingenjör behöver en kompetens som går
bortom den rent vetenskapliga kompetensen och speciellt intressant
för KTH är det konstnärliga perspektivet. Det är likaledes
motivationen bakom ett nyligen uppstartat tvärvetenskapligt centrum
NAVET på KTH där KMH, Stockholms konstnärliga högskola samt Konstfack
är partner.

\item Med en användbar metod kan även själva praktiken ses som en
kunskapsgenererande fas. Här ingår de numera ganska vanliga studierna
i interpretation eller alternativa speltekniker i nutida musik. En
frågeställning utforskas genom praktiken och om experimentet faller
väl ut så är det ett bevis på att praktiken är användbar även för
andra som söker svar på liknande problem och är således en
undersökning som genererar kunskap i det specifika fältet. Även denna
typ av undersökningar kan dock sträcka sig bortom den konstnärliga
sfären i vilket fall valideringen kan ske åt två håll. En studie i
gruppimprovisation kan till exempel studera hur specifika typer av
musikalisk interaktion kan ge gynnsamma resultat givet en viss
problemformulering. Samma metod kan sedan prövas i andra interaktiva
situationer, som social interaktion, och om den visar sig
framgångsrik även där så går resultaten att återföra till det
musikaliska sammanhanget och det uppstår en
kunskapsåterkoppling.\footnote{\footcite[Se t.ex. projektet ICASP
som jobbade enligt denna modell:][]\{lewis09\}.} Även denna typ av undersökning finns
representerad i \emph{Before Sound}, kanske framförallt i relation till
den andra forskningsfrågan. I exemplet med \emph{Currents} är
beskrivningen av arbetsmetoderna en kommunikation av en process som
inte bara gestaltade dataströmmarna och förhöll sig till de
uppställda metoderna, utan som också skapade en modell för ett
konstnärligt tillvägagångsätt som har både politiska och konstnärliga
implikationer.

\item En tredje variant är att se det resulterande konstverket som en
kunskapskälla, frigjord från upphovspersonen på det sätt som Deleuze
och Guattari framhåller ovan. Denna strategi har uppenbara nack-
eller fördelar (beroende på hur man ser det). Om det ska stå för sig
själv ("stand up on its own"\footnote{\footcite{deleuze1994}.}) så måste det, åtminstone i
musik, förlita sig på ickekonceptualiserade kommunikationsformer, det
vill säga dokumentation av verket för att inte bli extremt begränsat.
Tidigt i konstnärlig forskning var detta normen. Det skulle vara
konsten, i och för sig själv, som utgjorde slutresultatet i
forskningsarbetet och därmed utgöra det kunskapsbärande elementet i
konstnärlig forskning. I praktiken var det endast ett fåtal
avhandlingar som egentligen fullföljde den principen, men fortfarande
är diskussionen om balansen mellan det som lite slarvigt kallas för
"det skrivna" och "det gestaltade" aktuell. Det finns naturligtvis
flera anledningar till detta, men en relevant punkt som förtjänar att
framhållas är att det hela tiden finns en risk att en konstnärlig
avhandling är en avhandling med en omfattning som motsvarar en
monografi inom humaniora (där detta är normen), men som även
innehåller ett konstnärligt arbete som är lika omfångsrikt. Helt
enkelt en dubbel avhandling. Det konstnärliga resultatet i sig måste
ha en framträdande position i en konstnärlig avhandling då det utgör
själva objektet och även det man skulle kunna kalla empirin, men det
behövs en mer initierad diskussion om hur detta ska representeras i
avhandlingen. Men om vi accepterar en representation av det i form av
en dokumentation, vad är det som säger att en inspelning är bättre än
en beskrivande text? Det kan naturligtvis finnas många fall där det
är det (de flesta), men den poäng jag försöker göra här är att
förutsättningarna för vad som är en relevant dokumentation och/eller
diskussion av ett konstnärligt resultat inte kan avgöras på generell
nivå utan måste göras utifrån de specifika förutsättningarna som
råder i projektet. Därför kan i vissa fall konstverket i sig självt
vara det slutgiltiga resultatet, men som sådant behöver det i regel
vila på någon form för dokumentation och denna kan vara multimodal.
\end{enumerate}

I \emph{Before Sound} pekar de tre forskningsfrågorna mot processen snarare
än resultatet varför jag menar att Stjernas avhandling är ett bra
exempel på hur avgränsningar har gjorts utifrån innehållet i
avhandlingen. Den begränsade dokumentationen kan helt enkelt inte ses
som ett problem eftersom det inte är upplevelsen av verken som är den
centrala diskussionen. Dessutom är det platsspecifika en helt central
parameter i Stjernas praktik såsom den presenteras i avhandlingen,
vilket gör en eventuell dokumentation ännu mindre relevant. Även om
detta är en aspekt som ytterligare kan diskuteras vill jag här först
kommentera hur det för avhandlingen viktiga begreppet \emph{platsspecifik} är
sammanvävt med den teoretiska ingången i avhandlingen. Centralt för
Deleuze och Guattaris filosofi i \emph{Capitalism and Schizophrenia} och
\emph{What is Philosophy?} är som sagt immanensplanet, som i sin tur har sitt
ursprung hos Spinozas panteism,\footnote{\footcite[s.93]{Stjerna2018}.} eller tanken på att allt är en
substans snarare än ordnat i en hierarkisk och dualistisk struktur.
Givet att Stjerna utgår från detta immanensplan när hon beskriver den
konstanta rörelse i tillblivelse som de transversala processerna är
sammanvävda i, är idén om det platsspecifika och odokumenterbara
installationen helt konsekvent.

Stjerna beskriver sin praktik som multidisciplinär\footnote{\footcite[s.42]{Stjerna2018}.} och även om
det inte är nödvändigt att en undersökning som denna samtidigt är
interdisciplinär betyder det att flera olika kunskapsfält samsas sida
vid sida i forskningen. Detta kan lätt bli en utmaning och även om
Stjerna elegant navigerar runt behovet att beskriva den filosofi som hon
utgår från och lyckas dra rimliga gränser för vad som inkluderas och vad
som exkluderas, är det tveklöst en avhandling i minst två discipliner:
filosofi och konst. Detta är inte, vill jag betona, ett problem, snarare
är det sannolikt en nödvändighet för att komma åt de verkligt
intressanta perspektiven, och här behövs det goda exempel som kan föra
fältet framåt. Men risken finns att det blir en dubbel avhandlingar som
diskuterades ovan, det vill säga omfattande avhandlingar som egentligen
avhandlar två distinkta ämnen. Men här finns även risken för
avhandlingar som helt undviker det konstnärliga perspektivet och
fokuserar på ett angränsande ämne eller avhandlingar som inte tydligt
nog relaterar till det angränsande ämnet, vilket kan leda till att
argumentationen i sin helhet faller. Samtliga dessa faror går att
undvika med rätt handledning, men problemet med att hitta rätt
handledare är uppenbart när det handlar om ämnen som eventuellt inte
finns representerade på fakulteten.

Detta är en av anledningarna till varför bihandledaren ofta fyller en så
viktig roll i konstnärlig forskning på ett sätt som den inte alltid gör
i vetenskaplig forskning. Bihandledaren kan vara den personen som
garanterar att ett angränsande ämnen får tillräckligt stor roll och
genomlysning i arbetets helhet och kan i vissa fall, eller i vissa
perioder av arbetet, framstå som projektets huvudhandledare. Men det
finns ytterligare en anledning till bihandledarens betydelse som har att
göra med den tidigare nämnda svårigheten att hitta rätt
handledarkompetens på fakulteten eller på högskolan. Ännu är endast ett
fåtal lärare på de konstnärliga lärosätena disputerade (även om
variationen här är stor mellan olika lärosätena). Då det ofta finns ett
behov av att huvudhandledaren och doktoranden är på samma institution så
kan en lösning vara att huvudhandledaren blir mer av en
institutionshandledare och att den huvudsakliga handledningen sköts av
bihandledarna. Även om detta inte behöver vara problematiskt i sig kan
det leda till obalans i hur forskningsfältet utvecklar sig i relation
till andra. Handledarkompetens och seminarieverksamahet är uppenbart
centrala delar av en forskningsmiljö.

\section{Den konstnärliga kunskapens dynamik}
\label{sec:org541db05}
Utvecklingen av konstnärlig forskning i Sverige kan ses utifrån minst
tre delvis överlappande processer. Den ena rör den utbildningspolitiska
aspekten av konstnärlig utbildning i Bologna-modellen, men började ännu
tidigare än så, i Sverige med högskolereformen 1977.\footnote{\footcite[Se t.ex. ][]{Lilja2015}.} Den pekade
på att alla högskolor skulle bygga på utbildning som är baserad på
forskning varför även de konstnärliga utbildningarna nu skulle bedriva
forskning. Eftersom dessa bedriver konstnärlig undervisning eller
undervisning med konstnärliga metoder så måste de även bedriva
konstnärlig forskning -- detta kallades dock för konstnärligt
utvecklingsarbete snarare än forskning. I grunden ligger jämställandet
av konstnärlig och vetenskaplig forskning som nu ses som två uttryck för
kunskapsproduktion.

Den andra processen är mer svårfångad men handlar om hur konst- och
kulturlivet i samhället har utvecklat sig under de senaste decennierna.
Det fält inom vilket konstnärliga uttryck diskuteras och kommuniceras
har för vissa uttryck, som musik, förändrats i mycket stor grad.
Dagspressens recensionsverksamhet, Public Service funktion och det
offentligas stöd till musiklivet har förändrats radikalt vilket har
skapat nya behov för ytor att diskutera och experimentera med
konstnärliga uttryck. Här har den konstnärliga forskningen börjat fylla
ett stort hål.

En tredje process rör en mer filosofiskt orienterad epistemologisk fråga
om vad kunskap kan ses vara, och hur den kan kommuniceras. En vanlig,
initial, invändning mot konstnärlig forskning, som diskuterades ovan, är
att något som i allt väsentligt är beroende av sinnesintryck, som
upplevelsen av konstnärligt uttryck kan sägas vara, inte kan utgöra
grunden för forskningsmässig kunskap. Även om denna invändning vilar på
en missuppfattning av såväl forskningsmässig kunskap som konstnärlig
kunskap så rör den vid en viktig grundförutsättning för all
kunskapsutveckling, nämligen att det finns grundläggande förutsättningar
som det går att enas omkring. Utan dessa blir det omöjligt att etablera
ett nytt forskningsfält.

\subsection{Konstnärlig forskning som kunskap i praktiken}
\label{sec:orgbe16191}
Det är en utmaning att argumentera för konstnärlig kunskap i en samtid
som samtidigt präglas av en övertro på det vetenskapliga
kunskapssystemet, en missriktad postmodernistisk avart där varken rätt
eller fel existerar och en hyperkapitalism som inte ser sina gränser.
Machiarini-fallet på KI\footnote{\footcite{macchiarini2019}.} är ett exempel på vad som händer när
dessa tre samverkar och havererar. Jag ska försöka peka på några områden
där jag ser att konstnärlig forskning kan komma att få stor betydelse om
kunskapen hanteras på ett adekvat sätt, men jag argumenterar framförallt
utifrån ett musikperspektiv och bilden kan vara väsentligt annorlunda
inom andra konstnärliga fält.

Det digitala har idag helt genomsyrat såväl produktion, distribution som
konsumtion av musik samtidigt som det digitala endast i liten
utsträckning är en teknik som präglas av specifikt konstnärlig
utveckling. Det betyder att produktionsverktyg, som mjukvara för
inspelning och redigering, samt program för uppspelning visserligen för
det allra mesta är anpassade för förutsättningarna för dessa
verksamheter, men det finns förhållandevis liten kunskap om de faktiska
konstnärliga processer som ligger bakom användandet. Etableringen av
artificiell intelligens visar ytterligare på behovet av nya metoder för
att bättre förstå vidden av förändringen som vi är inne i. Även om
internet inte är nytt var det inte många som för tio år sedan hade trott
att vi idag, genom våra telefoner, trådlöst skulle få kontakt med en
till synes oändlig samling musik, bara för att ge ett exempel. Följden
av detta är att en majoritet av människor i västvärlden, själva designar
sin egen ljudmiljö och i någon mening, om man vill se det positivt, har
tagit kontroll över sitt eget musiklyssnande. Många av de verktyg som
har möjliggjort denna transformation är produkter av en ingenjörskonst
på mycket hög nivå. Spotify, till exempel tog utgångspunkt i teknik som
vuxit fram i en dunkel periferi under Pirate Bay, och gjorde delningen
legal och legitim. De fick ett tekniskt försprång och blev snart ensamma
herre på täppan. Detta är dock en utveckling som har skett helt och
hållet på kommersiella grunder och helt utan konstnärliga ambitioner.
Själva förutsättningen för Spotifys framgång var att man sänkte
ersättningen till de artister som spelades vilket kan ses som en
makrovariant av att såga av den gren man sitter på. Trots att detta
företag inte på något sätt kan ses vara en del av det konstnärliga
musiklivet där musik produceras talar man om Spotify som del av det
svenska musikundret.\footnote{\footcite{konig2018}.}

Det här är ett uttryck för problemet som musik lider av: musiken som
konstform har aldrig gjort upp med musik \emph{som produkt} utan de båda,
musiken och produkten, är i allt väsentligt sammanvävda. Musiken är helt
och fullt kommodifierad och därför kan "det svenska musikundret"
samtidigt inkludera musikteknikföretag och Ann-Sofie von Otter. Nu finns
det naturligtvis extremt starka kommersiella krafter i operabranschen,
men skillnaden är den att von Otter hade aldrig kunnat ta sig till den
position hon har utan att hon hade otvetydiga konstnärliga kvaliteter,
medan teknikföretag som Spotify kan utvecklas helt utan dessa att
konstnärliga överväganden styr utvecklingen.

Här finns en viktig plats för den konstnärliga forskningen och här kan
dess resultat utvärderas på en marknad långt utanför akademiens
skyddande väggar. I min avhandling från 2008 pekade jag på behovet att
inkludera en konstnärlig dimension när modeller för interaktivitet
skapas.\footnote{\footcite{frisk08phd}.} Den bakomliggande idén var den att en teknologiskt
orienterad interaktivitet inte bara riskerar att bli en grov förenkling
av vad vi förväntar oss av en interaktiv upplevelse (välj en färdig
spellista istället för att skapa din egen). Resultatet kan dessutom bli
att användarens förväntningarna på tekniken sänks. Genom att istället se
på utvecklingen av interaktivitet utifrån vad konstnärer, musiker och
tonsättare vill få ut av den skapas en miljö där helt andra beslut tas
och de interaktiva möjligheterna expanderas snarare än begränsas.

Utifrån det komplexa och svårgreppbara fältet av artificiell intelligens
(AI) finns flera möjligheter för konstnärlig forskning att spela en
roll. Liksom i exemplet ovan, beträffande interaktivitet, är AI ett
område där antaganden om mänsklig interaktion spelar roll, men även
antaganden om vad mänsklig erfarenhet utgör. Dessutom ryms otroligt
svåra etiska frågor. Alla dessa kan angripas genom konstnärliga
experiment där det mest omedelbara är att genom experiment utnyttja en
teknologi för konstnärlig produktion. Ett sådant projekt kan säga något
både om teknologins stabilitet och om hur konstnärlig aktivitet fungerar
i interaktion med teknik.

Genom metoder lånade från postkolonialismen har gruppen The Six Tones
under en rad år arbetat med interkulturella möten. Med det övergripande
målet att överbrygga kulturella avstånd har vi metodiskt arbetat
konstnärligt med konserter, skivproduktion och forskning. Målet i varje
projekt har varit att komma fram till ett övertygande konstnärligt
resultat har vi inte strävat efter enkel kunskapsöverföring utan efter
ömsesidigt lärande, differens och lyssnande. Genom en rad turnéer och
skivinspelningar har vi haft möjlighet att utvärdera praktiken i ett
flertal sammanhang. Under 2020 kommer vi att arbeta med migrerade
musiker i Sverige för att bland annat försöka förstå hur och varför
Sverige konsekvent har uteslutit några dessa musikers kulturtraditioner.
Några av de stora grupper som har kommit till Sverige, som de från Iran
och Irak och forna Jugoslavien på 70- och 80-talen, är inte i allmänhet
representerade med sina musiktraditioner på konserthus, scener eller i
kulturpolitiska sammanhang. Konsekvensen är att en del av dessa musiker
befinner sig i ett konstnärligt ingenmansland, långt från sitt hemlands
levande traditioner, men samtidigt utan möjlighet att etablera sig
konstnärligt i sitt nya land. I projektet hoppas vi bättre förstå dessa
musikers traditioner, men också bygga upp en kunskap som kan bidra till
en förändring. Att arbeta med musik på liknande sätt, med väl utarbetade
metoder i ett forskningssammanhang är ett sätt att utnyttja konstnärlig
kunskap inom ett fält där Sverige i dagsläget har svårt att politiskt,
etiskt och kulturellt hitta rätt.

\section{Diskussion}
\label{sec:org3173211}
Jag vill än en gång återkomma till den interrelation mellan konst,
vetenskap och filosofi som tre former för tänkande som Deleuze och
Guattari propagerar för i , och som jag i någon mening anser präglar
undersökningen i \emph{Before Sound}. Det finns en viktig poäng i deras
resonemang som finner oväntad resonans i den svenska
högskoleförordningen och dess unika uppdelning mellan konstnärlig och
vetenskaplig grund och där dessa två kunskapsformer betraktas som
likvärdiga. Att se det konstnärliga och det vetenskapliga som två
väsenskilda men samtidigt kompletterande former för tänkande får
självklart konsekvenser för hur vi ska, eller kan se, på konstnärlig
forskning och valideringen av resultaten. Konsekvenser som kanske inte
helt utnyttjats i konstnärlig forskning. Det finns en tendens att se
forskning som forskning vilket är adekvat i många fall, men inte alla.
Även om vi till exempel kan peka på en konkret nytta med konstnärlig
forskning, en nytta som i någon mening är relaterad till nyttoaspekten
av en del vetenskaplig forskning, är det inte säkert att detta ska vara
modellen för all konstnärlig forskning. Istället behövs en mer aktiv
diskussion om vilken eller vilka modeller som ska användas för
kvalitetsbedömning av konstnärlig forskning: Vilken roll har den
konstnärliga praktiken och utifrån vilka kriterier ska den bedömas? I
vilka forum ska kvalitet i konstnärlig forskning diskuteras? Hur bedöms
metod och teori i konstnärlig forskning? Hur bedöms \emph{impact} i
konstnärlig forskning? Ytterligare en fråga som skulle behöva diskuteras
mer för att hantera relationen mellan konstnärlig forskning och
konstnärlig praktik utan forskning, är hur kunskapsöverföringen ska ske.
Med andra ord, hur kommer resultat av konstnärlig forskning hela det
konstnärliga fältet till del?

Eftersom de flesta länder saknar den uppdelning mellan vetenskaplig och
konstnärlig grund, och eftersom det vetenskapliga forskningsfältet är så
dominant, finns det alltid en tendens att konstnärlig forskning glider
mot det vetenskapliga och blir en variant av vetenskaplig forskning,
snarare än ett unikt och självständigt sätt att resonera på. Men vi kan
inte mäta den konstnärliga forskningens värde utifrån vad den kan
åstadkomma \emph{som vetenskap}, helt enkelt för att det inte finns någon
större poäng med det. Hybridkonstellationer kan skapas, likt de jag i
alla korthet presenterar ovan, och dessa kan genom sitt tvärdisciplinära
angreppssätt bli ytterst effektiva sätt att utöka kunskapen till
angränsade fält. Men detta är, och ska vara ett komplement till en
konstnärlig forskning som i huvudsak befinner sig i det konstnärliga
fältet. Men oavsett hur mycket eller litet forskningen relaterar till
olika kunskapsområden som den rör sig omkring måste den ske i en aktiv
samverkan med annan forskning i samma fält och i aktiv samverkan med sin
omvärld, \emph{utan} att förlora de specifika bedömningsgrunder som gäller
för det konstnärliga fältet.

Ett möjligt sätt att behandla den konstnärliga forskningen utifrån
Deleuze och Guattaris modell i skulle vara att se den som ett kontaktnät
mellan alla tre discipliner. Ungefär som en transversal process som
möjliggör kontakter från konsten till filosofin och vetenskapen. På ett
plan är det det som Stjerna gör i \emph{Before Sound}. Det skulle samtidigt
möjliggöra att konsten som en form för tänkande är delvis oberoende av
den konstnärliga forskningen som här istället blir en \emph{möjligthet} för
kontakt mellan konsten, vetenskapen och filosofin.

För att ta ett exempel på hur svårt det kan vara att leva upp till de
vetenskapligt definierade kraven på \emph{impact} räcker det med att
återvända till relationen mellan filosofi och konst och betrakta de
tydliga paralleller mellan dessa två kunskapsområden. Båda har i
allmänhet setts som viktiga delar av (den västerländska)
kunskapsutvecklingen och båda har i någon mening förlorat sin
särställning och betydelse i samtiden. Dessutom så delar de båda ofta en
empiri som utgår från ett individuellt perspektiv. Det räcker med att
ytligt betrakta filosofin, som är en av våra äldsta forskningspraktiker,
för att snabbt konstatera att den har haft svårt att hävda sin
särställning. Lika otvetydigt är det att filosofi som konceptskapande
kraft har lagt grunderna till själva fundamentet för vårt samhälle idag.
Konsten har haft en lika uppenbar funktion för vår kultur och har lika
svårt att mäta sig som filosofin har haft.

För att illustrera vikten av att hålla fast i idén att konsten även kan
vara en distinkt kunskapskälla i sig själv lånar jag filosofen Frank
Jacksons berömda kunskapsargument. Det är ett tankexperiment vars syfte
det är att argumentera mot fysikalismen genom att visa att det måste
finns icke-fysiska aspekter av medvetandet. I grova drag går det ut på
att beskriva en kvinna som växer upp helt avskärmad från omvärlden, i en
helt svartvit miljö. Hon får aldrig uppleva färger, men hon lär sig
allt, precis allt, som finns att veta om dem. Hon lär sig även om det
neurofysiologiska och vad som händer när olika vågländer av ljus träffar
näthinnan. Hon kan föreställa sig hur det fysiskt är att se färg, men
har aldrig upplevt det. Vad händer när hon får komma ut ur sitt rum och
får uppleva färger på riktigt? Lär hon sig då något nytt? Finns det
något utöver det fysiska att lära sig som vi bara kommer åt genom
upplevelsen?\footnote{\footcite[s.130]{Jackson1982}.} Jackson menar att det gör det och att medvetandet är
mer än det neurologiska. Jag drar en parallell till idén om det
konstnärliga som en kunskapsform som går utöver det som vetenskapen kan
lära oss som jag menar såväl Bateson som Deleuze ger visst stöd: Om
vedertagna vetenskapliga definitioner och allmän empiri ger oss de fakta
som vi behöver för att förstå en sida av vår tillvaro så konsten kan
säga någonting om är allt det andra, det vi inte omedelbart kan se.
Vetenskapen kan bara ge oss en del av sanningen och det finns
kompletterande fält av kunskap, grundade i etik och konst, och bygger
liksom vetenskapen på empiri. Dessa är nödvändiga för att vi ska kunna
skapa oss en så komplett bild som möjligt av tillvaron. Det finns det
goda chanser att konstnärlig forskning kan bidra till att vi får en
bättre förståelse och kunskap om dessa andra dimensioner. Exempel på
sådan kunskap, varav några redan har presenterats, kan vara forskning på
förkroppsligad kunskap, kommunikativa strategier och interaktion. Alla
dessa tre har varit fokus för flera konstnärliga projekt och i alla
dessa tre områden har det varit tydligt att den konstnärliga
sensibiliteten och undersökningen har kunna öppna upp för en förståelse
som endast svårligen hade kunna angripas på annat vis.

Samtidigt ska möjligheterna utnyttjas att inom ramarna för de
konstnärliga högskolornas forskningsverksamhet att utnyttja högskolan
som en plats för konstnärlig produktion. I takt med att utrymmet för den
experimentella konsten minskar i samhället så kan denna verksamhet i
större utsträckning äga rum på högskolorna: Konstnärlig forskning som en
ny arena för experimentell konst. Det är inte frågan om en akademisering
av konsten utan snarare det motsatta: låt konsten påverka akademien.
Konsten är i detta fall experimentell i den bemärkelsen att den utmanar
de gängse metoderna för såväl konst som forskning även om inte
resultatet i sig behöver vara experimentellt rent stilistiskt.
Högskolans egna scener som ett öppet fönster mot omvärlden och ett
utomordentligt bra sätt att bedriva samverkan på. Det kan vara
Barockmusik eller elektroakustisk musik, eller någon annan genre.

Jag tror att frågan om vad konstnärlig forskning är, eller vilken nytta
den har, är felställd. Frågan vi borde ställa oss är snarare i linje
med: Hur kommer ett samhälle se ut som inte har aktivt arbetar för att
utveckla möjligheterna till att ta del av, och utveckla, alla
kunskapsformer som påverkar den mänskliga tillvaron, inklusive
konstnärlig sensibilitet och filosofiskt resonerande? Sett till den
frågan blir det klart att detta måste vara en fråga som diskuteras på
bred front i samhället. Det är ingen tvekan om att vi lever i ett
vetenskapssamhälle först och främst och att det i sin tur skapat ett
tekniksamhälle. Om det finns övertygande argument för att vetenskapen
inte behöver konsten och filosofin så är jag beredd att tänka om. Tills
dess menar jag att allt pekar på att vi behöver tillgång till alla typer
av tänkande vi kan komma åt för att lösa de utmaningar vi står inför i
världen idag och att just tänkande genom konstnärlig praktik är ett
mycket användbart verktyg. Det visar inte minst Åsa Stjerna med .


\section*{Metoden och resultat}
\label{sec:orgc2861a5}
Som tidigare nämnts kallar Stjerna sina metoder för \emph\{explorative
    approaches\}. Givet hur nära kopplade de är till teorin är det
sannolikt en klok strategi, även om det också kan skapa förvirring. De
tre övergripande utforskande tillvägagångssätten \emph\{att kartlägga
    de affektiva linjerna\}, \emph{att skapa nya sammankopplingar} samt
\emph{att bli icke-autonom} (som diskuterades en del i förra
avsnittet). Denna nära koppling mellan teori och metod gör det på ett
sätt enklare att diskutera utmaningarna i projektet. Om vi fortsätter
utifrån linjen att Deleuze och Guattaris filosofi framförallt sysslar
med begreppskapande (skapande av koncept), är det inte långsökt att
tänka sig att deras verktyg användas för att skapa nödvändiga begrepp
för att förstå och analysera den konstnärliga praktiken. Frågan är om
resultatet av den processen först och främst är kunskap om konst eller
om det är filosofi, men oavsett svaret så kan resultatet ha validitet
som forskning. En annan utmaning till följd av denna användning av
metod är att det av uppenbara skäl helt enkelt kan uppstå en
sammanblandning mellan teori, metod och resultat med utfallet att den
konceptuella stabiliteten blir lidande. Men samtidigt, som också
diskuterades ovan, är detta i realiteten ett sätt att hantera metod
och teori i konstnärlig forskning. Jag ska återkomma till
det. \%\% Kolla att jag återkommer

I Stjernas fall uppstår inte dessa problem, delvis tack vare att hon
handskas varsamt och ytterst konsekvent med begreppen. Dessutom ska
man inte glömma att teorin i \emph{Before Sound} i stor utsträckning
relateras till den kontextualisering och positionering som görs i
kapitel två: \emph\{Contextualisation--Artistic Strategies within the
    Field\}. Denna typ av situering är något som ibland förbises i
konstnärliga avhandlingar vilket kan få till följd att de konstnärliga
resultaten blir svåra att bedöma: Om det är oklart inom vilket
konstnärligt fält konstnären/forskaren rör sig kan det vara svårt att
korrelera de konstnärliga resultaten till konstfältet i stort. Samma
problem uppstår om fältet som beskrivs är alltför disparat eller
alltomfattande, eller alltför smalt. Eller, om fältet som beskrivs är
ett annat än ämnet för avhandlingen eller forskningen. Det finns flera
anledningar till att det tagit lång tid för konstnärlig forskning att
hitta rätt på detta område men några uppenbara skäl är:

\begin{enumerate}
  \item Bristen på etablerade modeller för att referera till alla
    typer av konstverk och svårigheten beskriva dem på ett sätt som
    bidrar till att skapa en bild av ett fält är påtaglig. Ett sätt
    runt detta är att göra som Stjerna gör och referera till en
    (vetenskaplig) studie av verken i fråga.\footcite[Se
    t.ex. referenser till Cox och LaBelle: ][s.45]{Stjerna2018} Detta
    är i grund och botten ett beprövat sätt att såväl inkludera den
    större diskursen om ett verks validitet för ett specifikt
    sammanhang och kan skapa trovärdighet i argumentationen när
    möjligheten att lyssna och uppleva, om än genom en dokumentation,
    också erbjuds. En eventuell baksida med denna metod är att
    argumentationen kan göra sig beroende av en annan disciplin, som
    filosofi, eller musikvetenskap i det här fallet, eller rent av
    resultera i en i huvudsak vetenskaplig undersökning. Detta är i
    sig inte ett problem, jag menar att konstnärlig forskning är i
    grunden tvärvetenskaplig, men det är samtidigt viktigt att det
    tvärvetenskapliga hanteras på ett sätt så att det understödjer, inte
    raserar, den konstnärliga undersökningen.

  \item Det är tydligt att man i konstnärliga avhandlingar har lättare
    att referera till etablerade verk och studier av verk. Men utan en
    intern diskussion, och utan att varje avhandling också relaterar
    till det fält som är i dess omedelbara närhet, såsom samtida
    konstnärlig forskning eller konstnärlig praktik, riskerar studiens
    validitet att begränsas. Stjerna går elegant runt detta genom att med omsorg
    beskriva verktygen, tillvägagånssätten och det teoretiska
    perspektivet hon använder vilket skapar en tydlighet. Men jag
    skulle vilja gå så långt som att säga att den största utmaningen
    vi har för konstnärlig forskning idag är just bristen av ett
    befintligt forskningsfält där den konstnärliga undersökningen står
    i centrum. Detta kan byggas genom att doktorander samarbetar och
    refererar till varandras arbeten, gärna kritiskt men alltid
    noggrant, och genom att större forskningmiljöer skapas. Ett nytt
    samarbete mellan Kungliga Musikhögskolan och Musikhögskolan i
    Piteå, Luleå Tekniska Universitet, samlar mer än tio doktorander,
    en PostDoc och flera seniora forskare i vad vi kallar lab tre
    gånger per år. Det är en början men vi behöver fler samarbeten där
    utgångspunkten är diversitet, samarbete och kritisk
    granskning. 2018 skrev vi i programförklaringen till samarbetet
    mellan LTU och KMH att vi vill ytterligare arbeta för att skapa:
    \begin{quote} [\ldots] former för kunskapsbyggande i konstnärlig
        forskning som är baserat på konstnärlig kunskap. Även
        teoretiskt drivna resonemang kan här ta sin utgångspunkt i den
        konstnärliga praktiken varför själva praktiken är central i
        seminarier och labsessioner. De gemensamma seminarier som vi
        planerar i Piteå och Stockholm ska ta formen av ett
        laboratorium som alla gemensamt ska bidra till den fortsatta
        utvecklingen av detta format. En viktig förutsättning är att
        skapa ett arbetsklimat som bygger på förtroende av samma slag
        som uppstår i konstnärliga samarbeten, och som därigenom
        skapar förutsättningar för en kritisk dialog som kan gå på
        djupet in i konstnärliga processer.\footcite{frisk2018:irl}
    \end{quote}
    Idén är att sätta den konstnärliga praktiken i centrum i en
    skyddad seminariemiljö som erbjuder möjligheten att
    experimentera. I detta initiativ hoppas vi att det ska bli möjligt
    att i större utsträckning begreppsliggöra praktiken och
    undersökningen vilket samtidigt kan underlätta att arbetena också
    diskuteras, kritsikt granskas av jämlikar, samt att resultaten,
    såväl som metod och teori, delas.

  \item Även det större fältet av konstnärlig praktik, utanför
    konstnärlig forskning, måste vara möjligt att referera
    till. Musikhögskolan i Piteå och konstnärliga fakulteten vid
    Göteborgs universitet har etablerade metoder för att registrera
    konstnärliga verk vilket gör det enkelt att hänvisa till dessa,
    och KMH och Musikhögskolan i Ingesund, Karlstads universitet, har
    påbörjat ett arbete som förhoppningsvis leder till att vi bidrar
    till ett fält som inkluderar akademien och skapar förutsättningar
    för forskningsanknuten undervisning. Om möjligheten för att
    publicera konstnärliga arbeten i forskningsdatabaser erbjuds
    kommer det uppstå en ackumulativ effekt som kan få mycket stor
    betydelse för hur konst som kunskapsfält kan utvecklas.

  \item Utan tydliga begrepp för konstnärlig forskning och konstnärlig
    praktik kan det vara svårt att på ett öppet sätt beskriva det
    konstnärliga fält som är det centrala för studien, utan att
    samtidigt bli låst av det. Att använda filosofi, som tidigare
    diskuterats, eller andra discipliner är här en möjlighet som
    samtidigt kan bidra till en breddad förståelse för detta
    forskningsfält. Dock är det nödvändigt att även begreppen växer
    fram i gemensamma miljöer där resultat och diskussioner delas.
\end{enumerate}

Det ackumulativa kunskapsbyggandet som kan bli resultatet av en stabil
forskningsmiljö där forskare på olika sätt bygger vidare på varandras
arbeten på ett genomskinligt och konsekvent sätt är en förutsättning
för att konstnärlig forskning ska få respekt och förtroende som ett
självständigt kunskapsfält, men också för att den interna
kunskapsutvecklingen ska ta fart. Detta sker naturligtvis redan idag i
viss utsträckning, men här finns ett stort utrymme för utveckling. I
ett nytt fält kan det vara naturligt att man framför allt vänder sig
till teoribildning utanför sitt eget fält, men nu är det viktigt att i
ännu större utsträckning rikta blicken också mot annan konstnärlig
forskning för att utvärdera och bygga vidare på dess teori, metod och
resultat.\footnote\{Det är intressant att notera att detta behov finns
    nu, trots de isoleringstendenser jag beskrev i början av detta
    kapitel.\} Jag menar att konstnärlig forskning inte bara är
tvärvetenskaplig utan också är multidisciplinär till sin natur, det
vill säga att den i vissa fall rent av är beroende av andra
discipliner än det rent konstnärliga för att kunskapen ska kunna
kommuniceras såväl i som utanför dess egen domän.\footnote\{För
    konstnärlig forskning som är helt inriktad på att skapa ny kunskap
    inom sitt eget fält, som till exempel projekt som studerar hur man
    bäst preparerar ett piano för ett givet verk eller liknande, har
    eventuellt det multidisciplinära perspektivet inte samma
    betydelse.\}  Symptomatiskt beskriver även Stjerna arbete med
ljudinstallationer som en multidisciplinär praktik, och själva
avhandlingen i sin helthet kan även den ses som multidisciplinär. Det
är dock viktigt att förstå den politisk dimension som det
tvärdisciplinära pekar mot. Samverkan är ett ledord för samtliga
universitet och högskolor idag och tvärvetenskap har i vissa fall
blivit ett neoliberalt självändamål. Detta gagnar inte alltid
utvecklingen av ett forskningsfält där behoven också behöver
komma inifrån, snarare än att de läggs på utifrån.

I Stjernas avhandling harmonierar det sätt hon bygger upp det
teoretiska och metodologiska ramverket i stor utsträckning med hur jag
i tidigare nämnda bokkapitlet \citetitle{frisk2015} föreslår att den den gängse
uppfattningen av relationen mellan teori, metod och praktik behöver
omformuleras. Även om Stjerna inte beskriver det explicit är det min
uppfattning att hon bygger upp definitionen av begreppet ``sound
art'', och dess ontologiska underbyggnad, genom att korrelera sin egen
erfarenhet som praktiker med en filosofisk och musikvetenskaplig
genomgång av hur begreppet har etablerats.\footcite[Se kapitel
tre]\{Stjerna2018\} Hon ger följande beskrivning av praktiken:

\begin{quote}
    To engage in sound installation as a site-specific practice is
    thus to position oneself, as an artist, as a node in the
    heterogenic field of what often is referred to as “sound art”
    respectively “sound art in public space.” It is to understand that
    sound installation, in all its specificity emanates from a variety
    of different practices and traditions, which together generate a
    spatially explorative, multi-disciplinary
    practice.\footcite[s.42]{Stjerna2018}
\end{quote}

Även med en rudimentär förståelse av Deleuze och Guattaris filosofi är
det redan i detta citat möjligt att se hur valet av teori är
välmotiverad. Det multidisciplinära angreppssättet förutsätter att de
olika delarna i undersökningen är sammankopplade och hur de
kommunicerar med varandra, vilket är själva kärnan i hur begreppet
transversalitet ska förstås. I en miljö som är genuint
multidisciplinär är det nödvändigt att ha en metod som tillåter
obruten kommunikation mellan de olika delarna av projektet och
begreppet transversalitet användes från början av Guattari i ett
liknande syftet, som en kritik mot den dualistiska synen på relationen
mellan analytiker och analysand:

\begin{quote}
    The concept of transversality emerges in part out of Guattari’s
    prolonged critique of the ‘personological’ understanding of
    language at work within psychoanalysis, and, specifically, within
    Lacanian versions of analysis. While not initially conceptualized
    in terms of enunciation, transversality—in Guattari’s early
    writings institutional transference (later reframed as ‘group
    transversality’) — aims to capture the unconscious as an
    investment of the broader elements and processes within the
    specific social setting of the hospital, a pattern of investment
    that would come to light only with the greatest difficulty in the
    dyadic enunciative setting of the analyst’s consulting
    room.\footcite[s.234]{Goffey2015}
\end{quote}

Själva begreppet bär alltså redan från början med sig det som Stjerna
beskriver som ett resultat: en sammanvävd transversal process som
omformulerar hierarkier till kontinuerliga och i vissa fall spatiala
system. Här ingår relationen mellan konstnärssubjektet och publiken,
som visserligen har utsatts för kritik sedan
1960-talet.\footcite[s.48]{Stjerna2018} I \emph{Before Sound} är fokus
processerna i skapandet och produktionen och publiken framträder inte
direkt som en agent som diskuteras. Det transversala utspelar sig
därför primärt mellan platsen, konstverket och konstnärssubjektet:

\begin{quote}
    In this doctoral research, the concept of assemblage has enabled
    me to articulate a mode of artistic practice in which
    site-specific sonic conditions and production operate as immanent,
    inter-relational, machinic, and transversal processes. I
    acknowledge the importance of this way of thinking in the subtitle
    of the thesis, “Transversal Processes in Site-Specific Sonic
    Practice,” and its influence can be seen in the previous chapter’s
    presentation of the field. As I have suggested through my
    descriptions, the initial explorative process, the establishment
    of spatial perception, the development of sonic strategies and
    technology, and the construction process on site, all emerge as
    the result of complex, machinic interconnections that span
    transversally between “the site,” “the artwork,” and the
    “artist-subject.” In this, I advocate a move beyond the
    traditional separations that establish these as three distinct
    entities.\footcite[s.92]{Stjerna2018}
\end{quote}

Kanske kan man därigenom dra slutsatsen att den transversala processen
är både metod och resultat? Åtminstone är det som metod
transversalitet beskrivs i citatet ovan och som sådan borde den vara
intressant även i annan konstnärlig forskning. Det undermedvetna, som Guattari
diskuterar i citatet ovan och som jag återkommer till längre fram i kapitlet, har en del med den konstnärliga upplevelsen
att göra, och om transversalitet kan bidra med att begreppsliggöra det
som sker även i den konstnärlig processen så skulle en del vara
vunnet. Och, som sagt, \emph{Before Sound} visar att det kan vara
möjligt.

Att begreppet härstammar från Guattaris önskan att fånga förståelsen
av det undermedvetna är inte oväsentligt, inte heller dennes brott med
med den Lacanska traditionen för psykoanalys. Som tidigare nämnts så
bygger mycket av den filosofi som Deleuze och Guattari utvecklade
tillsammans på en kritisk granskning av bland annat den Freudianska
teoribildningen. Lite förenklat kan vi knyta Guattaris ambition att
bryta med den binära analytisk modell som analytiker/analysand
innebär till Deleuze kritik av det transcendentala tänkandet som har
varit så central för Europeisk filosofi. Genom Guattaris analytiska
grupptransversalitet kunde man sannolikt närma sig själva
terapisessionen mer som en process av tillblivelse genom det
undermedvetna (nu som en produktiv kraft), snarare än som en aspekt av
det medvetna (eller omvänt). Detta påminner samtidigt om de försök att
dekonstruera de binära eller dyadiska relationerna mellan olika
agenter i den konstnärliga produktionen som så många konstnärliga
forskare, inklusive Åsa Stjerna, har varit upptagna med. Frågan är hur
vi från denna förståelse av relationerna mellan aktiva och sammanvävda
komponenter i det konstnärliga arbetet kan ta oss mot en insikt i vad
dessa relationer säger om den konstnärliga praktiken.

För att förstå varför det undermedvetna är relevant i det här
sammanhanget vill jag ta upp en text av Gregory Bateson som jag har
använt flera gånger tidigare. Bateson var en brittisk antropolog,
sociolog, filosof och cybernetiker som har en nära relation till
Deleuze. Det finns ett fåtal referenser till Bateson hos Deleuze,
bland annat två i \emph{Mille Plateaux},\footcite{deleuze80} men det
finns de som menar att inflytandet från Bateson egentligen var
betydligt större än vad som framgår av referenserna.\footcite[Se
t.ex. ][ Begrepp som \emph{rhizome}, \emph{double bind}, och
\emph{schizoanalysis} som alla var viktiga för Deleuze och Guattari
diskuterades långt tidigare av Bateson, även om just \emph\{double
    bind\} introducerades av Nietsche.]\{Shaw2015\} Det som gör Bateson
intressant i den specifika diskussionen om konstnärlig
kunskapsbildning är dock hans syn på just det undermedvetna och hur
olika typer av information och upplevelser kodas i hjärnan, såväl som
i kroppen.

I Freudiansk teori delar man upp mental aktivitet i primära och
sekundära processer. De primära är icke-verbala och drömlika, och
föregriper de sekundära som är det reflekterande och medvetna jagets
uttryck. Konst är generellt ``an exercise in communicating about the
species of unconsciousness [\ldots] a play behaviour whose function is
[\ldots] to practice and make more perfect communication of this
kind.''\footcite[s.137]{bateson72} Nu kan det framstå som att vi har
återinfört en separation mellan det inre, de primära processerna, och
det yttre, de sedundära. Sannolikt så var det bland annat denna
uppdelning som Guattari ville komma åt när han försökte tänka om
terapisituationen. Men dessa två kategorier av processer behöver inte
vara väsensskilda utan kan snarare ses som två möjliga, och i vissa
fall parallella, sätt att koda kunskap och erfarenhet (eller
affekter). Då är inte frågan hur vi ställer de mot varandra, utan hur
vi kommunicerar mellan, eller inom dem. Bateson skriver:

\begin{quote} [The] algorithms of the heart, or, as they say, of the
    unconscious, are, however, coded and organized in a manner totally
    different from the algorithms of language. And since a great deal
    of conscious thought is structured in terms of the logics of
    language, the algorithms of the unconscious are double
    inaccessible. It is not only that the conscious mind has poor
    access to this material, but also that when such access is
    achieved. \emph{e.g.}, in dreams, art, poetry, religion,
    intoxication, and the like, there is still a formidable problem of
    translation.\footcite[s.139]{bateson72}
\end{quote}

I inledningen pekade jag i all korthet på hur konstnärlig kunskap ofta
inte utan vidare låter sig beskrivas verbalt. I ljuset av detta kan
det vara tilltalande att se en översättning från det omedvetna till
det medvetna som lösningen, men det finns flera saker som behöver
lyftas för att vi på ett hållbart sätt ska kunna ta ställning till
problemet om ``översättning''. För det första har vi frågan om
begreppsliggörandet av den konstnärliga praktiken, det vill säga
processen av att skapa begrepp som gör det möjligt att artikulera en
kunskap. I \emph{Before Sound} gör Åsa Stjerna detta, bland annat
genom att använda Deleuze begreppsapparat. Detta leder henne också
till vissa specifika resultat som i sin tur relaterar till bland annat det
platsspecifika, men också till hur själva praktiken ter sig. Men om
man tänker sig att man är i behov av att formulera egna begrepp så kan
man hamna inför Batesons utmaning: Hur det är möjligt att
omformulera eller översätta en konstnärlig strategi till en verbal
utan att den samtidigt förlorar mening eller fastnar i en meningslös
rad av metaforer eller representationer?

Samtidigt finns det inomkonstnärliga begrepp som inte behöver en
översättning för att fungera. För detta är den kontextualiseringen av
projektet som diskuterades ovan viktig för att begreppen som används
ska få pregnans och tillåter att diskursen inom fältet blir
användbar. I musik kan till exempel musikteoretiska begrepp nyttjas i
detta syfte, förutsatt att dessa relateras till praktiken på ett
användbart sätt. Både Deleuze och Guattari i \citetitle{deleuze1994}
och Bateson i \citetitle{bateson72} diskuterar dock framförallt det
som den konstnärliga upplevelsen ger upphov till, rent kognitivt,
snarare än den kreativa processen i sig. Mycket konstnärlig forskning,
så även Åsa Stjernas avhandling, beskäftigar sig framförallt med hur
processen att \emph{göra} konst fungerar och i den undersökningen så
kan resultatet, förutom att det är konst, vara ett sätt att validera
utforskandet av processen (vilket dock fortfarande gör det
angeläget att kunna diskutera resultatet).

Det är inte säkert att den konstnärliga processen som leder fram till
ett konstnärligt resultat är enkelt jämförbar med upplevelsen av att
erfara resultatet. I vissa fall kan det vara så men i andra fall, som
i \emph{Before Sound}, ligger delar av processen närmare
ställningstaganden som har med praktiska omständigheter att göra; Hur
fungerar en sladd? Hur kan en högtalare installeras? etc. Är även
dessa kodade i det ``omedvetnas algoritmer'', för att använda Batesons
terminologi, eller är de del av en process som egentligen ligger
närmare andra forskningsdiscipliner än vad man kanske först vill tro?
Jag menar att det kan vara så i vissa fall, men att vi samtidigt inte
får glömma att den centrala aspekten av konstnärlig forskning att det
är en konstnärlig sensibilitet som ligger bakom valen som görs i det
konstnärliga arbetet. Förmågan att föreställa sig det konstnärliga
resultatet som just konst, även i arbetet med tekniska installationer
är avgörande och signifikativt för konstnärligt arbete. Av den
anledningen är det svårt att helt komma bort från det affektiva eller
det undermedvetnas logik när vi vill beskriva den konstnärliga
forskningsprocessen.

Frågan är om det egentligen handlar mindre om en översättning och mer
om att förstå hur vi kan förhålla oss till olika former för mänsklig
kommunikation. Guattari ville komma runt det han benämnde det
"personologiska" språkbruket i pyskoterapin, och Bateson pekar på att
konsten har en kommunikationskapacitet som gör den mer lik till
exempel andliga upplevelser, berusning och drömmar. Utmaningen är inte
att det finns olika logiska typer av medvetande och kunskap, utan hur
vi kan förstå dem genom en kommunikativ helhet. Klart är i
alla fall att de transversala processerna kan spela en stor roll
här, men kvar är frågan om dessa i sig kan ge oss mer stringent formulerad
konstnärlig forskning där det blir lättare för andra forskare att
förstå och relatera till resultat, process och metod.

Att det som jag här förenklat benämner det inre och det yttre inte är
två åtskilda paradigm för förståelse, kunskap och kommunikation kan
inte nog poängteras. Konst har länge präglats av idén om det ``rena''
och ``inre'' inre uttrycket som ofta ställs i kontrast till det yttre
``befläckade''. I denna modell är det inre idealistiskt, ärligt och
transcendentalt, och det yttre kan vara kommersiellt, beräknande och
materialistiskt. Även om det är förhållandevis lätt att ta avstånd
från dessa grovt tillyxade kategorier har de haft ett stort inflytande
över hur konst- och musikvärlden har utvecklat sig. Denna utveckling
uppmuntrar sökandet efter idéen om det rena uttrycket, det som
passerar förbi medvetandet, förbi det självmedvetna jaget. Ibland är
strävan efter originaliteten själva källan till sökandet efter det av
medvetandet obesudlade uttrycket, färgat av upplysningens bild av
identiteten; om varje individ är unik och oberoende borde också det
genuint personliga \emph{uttrycket} vara originellt. Saxofonisten
Ornette Coleman, till exempel, talar om strävan efter ett så spontant
skapande som möjligt och om en kreativitet utan
minne.\footcite[s.117]{litzweiler92} Han talar om hur hans spel innan
han nådde framgångar var mera ärligt än det sedan blev och valde att
börja spela trumpet och violin (som han var nybörjare på) för att
kunna spela och samtidigt slippa onödig kunskap.\footcite[Intervju med
Ornette Coleman i][s.33]\{taylor77\} Naturligtvis var Coleman lika
originell efter sitt genombrott som före, och retoriken här speglar
till stor del den sociala och politiska tidsandan som rådde, men kan
ändå sägas peka på kraften i bilden av personlig
originalitet.\footnote\{För övrigt är Coleman's uttalande om att spela
    utan minne påfallande likt Marcel Duchamps tal om traditionens
    fängelse och att glömma med handen: ``I unlearned to draw. The
    point was to forget \emph\{with my
	hand\}.''\fullcite[s.29]{tomkins65}\} På skivan \emph\{The empty
    foxhole\} från 1966\footcite{coleman66} spelar Coleman tillsammans
med sin tioåriga son Denardo Coleman på trummor och beskriver sin
tillfredställelse över att spela med någon som inte behövde bry sig om
kritiker eller konsertarrangörer, utan som kunde spela och vara
fri.\footcite[s.121]{litzweiler92} Detta hör Coleman när han lyssnar
på Denardo men också för att han har förmågan att lyssna på sig själv;
han kan konstatera att sonen besitter en egenskap han själv har
förlorat. Det yttre lyssnandet, att lyssna på den andre, kompletteras
av det inre lyssnande. Och, jämfört med att lyssna på den andre så är
det i vissa fall betydligt svårare att lyssna på sig själv. Som
konstnärlig metod är det utvidgade lyssnandet central i musikalisk
konstnärlig praktik och det är möjligt att vidareutveckla praktiken
enbart genom lyssnande. Colemans önskan att släppa taget om det
invanda och inlärda kan ses som ett försök att etablera nya
transversala relationer mellan den medvetna och språkligt kodade
viljan att förnya, och den konstnärligt kodade kunskapen om hur detta
ska, eller skulle kunna, gestaltas. Utan att göra en djupare analys av
metoderna (lyssnandet), är risken överhängande att landa i en syn på
en relation mellan det trascendentala inre och det fysiska yttre som i
grunden är hierarkisk: det inre, transcendentala är att föredra
framför det yttre. Den bilden har inte frigjort sig från det politiska
bagage den bär med sig och gör det därför inte lättare att beskriva
vad konstnärlig kunskap är.

En anledning till att vi har en tendens att se konstnärlig praktik som
en individuellt artikulerad form för kunskap är att vi i huvudsak ser
konstnärlig verksamhet som en individuellt situerad praktik. Det är
den i vissa fall, och utan tvekan är detta den romantiska
bilden av konsten som något som kretsar kring ett solipsistiskt
geni. Stjerna tar spjärn mot denna bild när hon diskuterar sin metod:
\begin{quote}
    established traditions in contemporary art practice still harbour
    segments of binaries that separate an autonomous active (white,
    male) subject and a (passive) urban text. Rejecting this
    traditional view, in proposing that we become non-autonomous, I
    advocate that we view the artist-subject’s agency in artistic
    production as transversal.\footcite[s.119-20]{Stjerna2018}
\end{quote}
Det finns många anledningar, även utanför argumenten som förs fram i
detta kapitel, att motverka denna normativitet och jag tror att
konstnärlig forskning är ett utomordentligt väl anpassat sätt att göra
det på, åtminstone på det konstnärliga fältet.

Slutsatsen som kan dras av detta resonemang är att det är lätt att
hamna i en dubbelt problematisk situation när konstnärlig forskning
diskuteras: Först är det nödvändigt att beskriva konstnärlig praktik
som något som är byggt på kunskap och erfarenhet och som inte är
internt, mystiskt, hemligt eller underligt. Att skapa trovärdighet
kring detta argument är svårt på grund av den sociala och politiska
starka föreställningen om konstens, respektive (den vetenskapliga)
forskningens funktion i samhället. Först när det är möjligt att i
någon mening dekonstruera denna bild är det möjligt att påvisa att det
i konstnärlig praktik finns kunskap som har ett allmängiltigt värde
och att detta kan diskuteras, kritiseras och kommuniceras som
forskning, och i interaktion med andra forskningsmiljöer. Om inte den
första delen av argumentet finner trovärdighet, kommer inte den andra
delen av det göra det heller. Konstens roll som kunskapsform i ett
större perspektiv, bortom enskilda projekt, kommer jag att diskutera i
nästa avsnitt.

\section*{Metodologisk dynamik}
\label{sec:org463364d}
I en avhandling som \citetitle{Stjerna2018} är gränserna mellan
praktik, metod och teori som tidigare nämnts inte tydligt artikulerade
utan ständigt rörliga. Terminologin hämtad från Deleuze och Guattari
är till exempel i vissa fall såväl metod, teori och i vissa fall
tydligt relaterade till resultatet, och det är inte alltid från början
självklart vad som är avsikten (detta är dock inte en svaghet i
Stjernas fall). Om vi tittar på den första forskningsfrågan, \emph\{På
    vilket sätt kan jag som konstnär utveckla utforskande
    tillvägagångssätt som understödjer en transversal
    skapandeprocess?\}, så är det klart att den transversala
skapandeprocessen är ett mål i sig, men lika tydligt är det i kapitel
fem att Stjerna genom sin praktik redan i början av projektet
har etablerat transversala kopplingar \footcite[s.145]{Stjerna2018}. Det
transversala är alltså i vissa fall både metod och resultat. Stjerna
pekar på att det finns ett stort behov för nya konceptuella verktyg
och tillvägagångssätt som kan gå bortom representation och
transcendenta hierarkier, och framför allt, teorier som kan synliggöra
hur transformation etableras utifrån en gedigen förståelse för
konstnärliga processer.\footcite[s.85]{Stjerna2018} Det är i princip
detta avhandlingen sedan visar på genom en syn på själva
ljudinstallationen som en transformativ praktik som löper mellan ett
flertal konstnärliga strategier som var och en är en transversal
process.


Det är ännu svårt att se konstnärlig forskning som en disciplin som
genererar tydliga resultat utifrån väl beprövade metoder. Istället
finns det en rad möjliga artikulationer av kunskapsutveckling som alla
pekar på att det i den konstnärliga praktiken finns en
epistemologisk potential, ett möjligt kunskapssystem som, om den
begreppsliggörs, kan ha stort inflytande på en rad olika fält. I
sammanhanget kan det påpekas att denna tro på konsten som
kunskapsbärande är delvis överensstämmande med Deleuze och Guattaris
resonemang i \emph{What is Philosophy?}, där de som tidigare nämnts
definierar tre kreativa metoder för tänkande: \emph{filosofi}, \emph{vetenskap} och
\emph{konst}. Egentligen kan man gå ännu längre och säga att det i
princip överensstämmer med Deleuzes hela filosofiska gärning att se
konst som en form för kunskap eller som en form för tänkande. Men om
konst är kunskap, eller bär på en kunskapsbärande potential, vari
består den? Deleuze och Guattari skriver vidare:
\begin{quote}
    What about the creator? It is independent of the creator through
    the self positing of the created, which is preserved in
    itself. What is preserved---this thing for the work of
    art---\emph{is a bloc of sensations, that is to say, a compound of
        percepts and affects}. [\ldots] The artist creates blocs of
    percepts and affects, but the ony law of creation is that the
    compound must stand up on its
    own. [[\footcite{deleuze1994}][][s.164, kursivering av författaren.]]
\end{quote}
Att konstverket i sig självt, och oberoende av upphovspersonen, har en
potential är tydligt, likaså att det kan och bör frigöra sig själv,
men i konstnärlig forskning är det ofta praktiken -- och vad för slags
kunskap den kan bära eller föra med sig -- snarare än resultatet, som
står i centrum. Det finns många exempel på hur denna
kunskapsutveckling kan se ut men här ska jag ge tre korta exempel som
är i större eller mindre grad hämtade från musikfältet.
\% Om filosofi är en konceptskapande process är konst en teckenskapande process (och vetenskap en funktionsskapande process).

\% \subsection*{1}
\% \label{sec:1}

\begin{enumerate}
% Flytta?
% Exempel 1
  \item Utöver att studera den konstnärliga praktiken som en kunskap i
    och för sig självt, är det möjligt att se den som ett sätt att
    förstå annan kunskap såsom teknik eller filosofi. I dessa fall kan
    man se den konstnärliga praktiken som en testbädd för ett
    konceptuellt ramverk som är omfattar andra discipliner än bara
    konsten. Det finns en aspekt av detta i \emph{Before Sound} där de
    filosofiska koncepten prövas mot en existerande praktik och sättet
    som praktiken utvecklar sig kan då ses som ett utforskande av
    filosofin. Begreppet \emph{immanensplanet} är ett exempel på ett
    koncept som är viktigt för Stjerna -- och helt centralt för
    Deleuze filosofi -- och som kan ses få, om inte en förklaring så
    en praktisk applikation, genom Stjernas konstnärliga praktik. I
    teknologisammanhang kan man föreställa sig att en teknik, säg ett
    programmeringsgränssnitt eller en specifik hårdvara, utforskas i
    en konstnärlig praktik. Den konstnärliga metoden, som i detta fall
    ska ses som en experimentell praktik där koncept prövas och
    utvärderas baserat på hur bra eller dåligt de interagerar med den
    konstnärliga ambitionen, används för att validera teknologin. En
    undersökning på denna nivå kan mycket väl leda fram till att
    tekniken som studerats bedöms annorlunda än den hade gjort i ett
    rent tekniskt sammanhang: en teknologi som i allt väsentligt ses
    som funktionell och stabil från ett tekniskt synsätt kan framstå
    som mindre användbar genom en konstnärlig undersökning. Det
    konstnärliga ramverket behöver i sig inte vara experimentellt utan
    kan mycket väl följa en befintlig tradition helt idiomatiskt, det
    är metoden som är experimentell: kritiskt evaluering genom
    konstnärlig metod. Detta angrepssätt skulle utan tvekan även
    generera insikter om den konstnärliga praktiken. Skulle studien
    även utforskas med en filosofisk begreppsapparat som Stjerna gör i
    \emph{Before Sound} så skulle de tre perspektiven konst, vetenskap
    och filosofi samverka och komplettera varandra på ett sätt som
    påminner om hur Deleuze och Guattari föreslår i
    \citetitle{deleuze1994}.\footcite{deleuze1994}

    Insikten om att den konstnärliga sensibiliteten kan behövas i
    större utsträckning än vad vetenskapstraditionen kanske fram till
    nu har velat göra gällande kommer dock inte bara från konstnärlig
    forskning eller filosofin. Det samarbete som Kungliga
    Musikhögskolan och Kungliga Tekniska Högskolan initierade 2016
    byggde till exempel på KTH:s insikt att en framtida ingenjör
    behöver en kompetens som går bortom den rent vetenskapliga
    kompetensen och speciellt intressant för KTH är det konstnärliga
    perspektivet. Det är likaledes motivationen bakom ett nyligen
    uppstartat tvärvetenskapligt centrum NAVET på KTH där KMH,
    Stockholms konstnärliga högskola samt Konstfack är partner.

% \subsection*{2}
% \label{sec:2}
  \item Med en användbar metod kan även själva praktiken ses som en
    kunskapsgenererande fas. Här ingår de numera ganska vanliga
    studierna i interpretation eller alternativa speltekniker i nutida
    musik. En frågeställning utforskas genom praktiken och om
    experimentet faller väl ut så är det ett bevis på att praktiken är
    användbar även för andra som söker svar på liknande problem och är
    således en undersökning som genererar kunskap i det specifika
    fältet. Även denna typ av undersökningar kan dock sträcka sig
    bortom den konstnärliga sfären i vilket fall valideringen kan ske
    åt två håll. En studie i gruppimprovisation kan till exempel
    studera hur specifika typer av musikalisk interaktion kan ge
    gynnsamma resultat givet en viss problemformulering. Samma metod
    kan sedan prövas i andra interaktiva situationer, som social
    interaktion, och om den visar sig framgångsrik även där så går
    resultaten att återföra till det musikaliska sammanhanget och det
    uppstår en kunskapsåterkoppling.\footcite[Se t.ex. projektet ICASP
    som jobbade enligt denna modell:][]{lewis09} Även denna typ av
    undersökning finns representerad i \emph{Before Sound}, kanske
    framförallt i relation till den andra forskningsfrågan. I exemplet
    med \emph{Currents} är beskrivningen av arbetsmetoderna en
    kommunikation av en process som inte bara gestaltade
    dataströmmarna och förhöll sig till de uppställda metoderna, utan
    som också skapade en modell för ett konstnärligt tillvägagångsätt
    som har både politiska och konstnärliga implikationer.

% \subsection*{3}
% \label{sec:3}


% Exempel 3
  \item En tredje variant är att se det resulterande konstverket som
    en kunskapskälla, frigjord från upphovspersonen på det sätt som
    Deleuze och Guattari framhåller ovan. Denna strategi har uppenbara
    nack- eller fördelar (beroende på hur man ser det). Om det ska stå
    för sig själv ("stand up on its own"\footcite{deleuze1994}) så
    måste det, åtminstone i musik, förlita sig på
    ickekonceptualiserade kommunikationsformer, det vill säga
    dokumentation av verket för att inte bli extremt begränsat. Tidigt
    i konstnärlig forskning var detta normen. Det skulle vara konsten,
    i och för sig själv, som utgjorde slutresultatet i
    forskningsarbetet och därmed utgöra det kunskapsbärande elementet
    i konstnärlig forskning. I praktiken var det endast ett fåtal
    avhandlingar som egentligen fullföljde den principen, men
    fortfarande är diskussionen om balansen mellan det som lite
    slarvigt kallas för "det skrivna" och "det gestaltade"
    aktuell.\footnote{I sig utgör det ytterligare en dikotomi som det
        finns all anledning att tänka om. När är en text \emph{bara}
        en text och när är något gestaltat \emph{uteslutande} konst?}
    Det finns naturligtvis flera anledningar till detta, men en
    relevant punkt som förtjänar att framhållas är att det hela tiden
    finns en risk att en konstnärlig avhandling är en avhandling med
    en omfattning som motsvarar en monografi inom humaniora (där detta
    är normen), men som även innehåller ett konstnärligt arbete som är
    lika omfångsrikt. Helt enkelt en dubbel avhandling. Det
    konstnärliga resultatet i sig måste ha en framträdande position i
    en konstnärlig avhandling då det utgör själva objektet och även
    det man skulle kunna kalla empirin, men det behövs en mer
    initierad diskussion om hur detta ska representeras i
    avhandlingen.\footnote{Detta är ett problem som vi har haft i
        Sverige inom konstnärlig forskning: undersökningarna och
        förväntningarna på resultaten bli orimligt höga vilket leder
        till avhandlingar som är enormt omfattande. Nyligen har jag
        varit opponent i såväl Estland som Nederländerna där
        textdelens omfattning på avhandlingen är ner mot en fjärdedel
        av vad svenska avhandlingar tenderar bli.} Men om vi
    accepterar en representation av det i form av en dokumentation,
    vad är det som säger att en inspelning är bättre än en beskrivande
    text? Det kan naturligtvis finnas många fall där det är det (de
    flesta), men den poäng jag försöker göra här är att
    förutsättningarna för vad som är en relevant dokumentation
    och/eller diskussion av ett konstnärligt resultat inte kan avgöras
    på generell nivå utan måste göras utifrån de specifika
    förutsättningarna som råder i projektet. Därför kan i vissa fall
    konstverket i sig självt vara det slutgiltiga resultatet, men som
    sådant behöver det i regel vila på någon form för dokumentation
    och denna kan vara multimodal.
\end{enumerate}

I \emph{Before Sound} pekar de tre forskningsfrågorna mot processen
snarare än resultatet varför jag menar att Stjernas avhandling är ett
bra exempel på hur avgränsningar har gjorts utifrån innehållet i
avhandlingen. Den begränsade dokumentationen kan helt enkelt inte ses
som ett problem eftersom det inte är upplevelsen av verken som är den
centrala diskussionen. Dessutom är det platsspecifika en helt central
parameter i Stjernas praktik såsom den presenteras i avhandlingen,
vilket gör en eventuell dokumentation ännu mindre relevant. Även om
detta är en aspekt som ytterligare kan diskuteras vill jag här först
kommentera hur det för avhandlingen viktiga begreppet
\emph{platsspecifik} är sammanvävt med den teoretiska ingången i
avhandlingen. Centralt för Deleuze och Guattaris filosofi i
\emph{Capitalism and Schizophrenia} och \emph{What is Philosophy?} är
som sagt immanensplanet, som i sin tur har sitt ursprung hos Spinozas
panteism,\footcite[s.93]{Stjerna2018} eller tanken på att allt är en
substans snarare än ordnat i en hierarkisk och dualistisk
struktur.\footnote\{Detta är inte på något sätt ett försök att ge en
    fullödig utsaga om vad detta begrepp är för Deleuze. Det är
    centralt för hela hans arbete och har vida konsekvenser.\} Givet
att Stjerna utgår från detta immanensplan när hon beskriver den
konstanta rörelse i tillblivelse som de transversala processerna är
sammanvävda i, är idén om det platsspecifika och odokumenterbara
installationen helt konsekvent.\footnote\{Även om själva begreppet
    platsspecifik skulle kunna ses som problematisk i sammanhanget,
    men det är delvis en annan diskussion.\}

Stjerna beskriver sin praktik som
multidisciplinär\footcite[s.42]{Stjerna2018} och även om det inte är
nödvändigt att en undersökning som denna samtidigt är
interdisciplinär\footnote\{En praktik som kombinerar många olika
    discipliner men som i grunden vilar på konstnärliga
    frågeställningar kan naturligtvis undersökas utifrån en
    konstnärlig grund och med konstnärliga metoder och därmed undvika
    att forskningen av den anledningen bli interdisciplinär.\} betyder det att flera olika
kunskapsfält samsas sida vid sida i forskningen. Detta kan lätt bli en
utmaning och även om Stjerna elegant navigerar runt behovet att
beskriva den filosofi som hon utgår från och lyckas dra rimliga
gränser för vad som inkluderas och vad som exkluderas, är det tveklöst
en avhandling i minst två discipliner: filosofi och konst. Detta är
inte, vill jag betona, ett problem, snarare är det sannolikt en nödvändighet för
att komma åt de verkligt intressanta perspektiven, och här behövs det
goda exempel som kan föra fältet framåt. Men risken finns att det blir
en dubbel avhandlingar som diskuterades ovan, det vill säga omfattande avhandlingar som
egentligen avhandlar två distinkta ämnen. Men här finns även risken
för avhandlingar som helt undviker
det konstnärliga perspektivet och fokuserar på ett angränsande ämne
eller avhandlingar som inte tydligt nog relaterar till det
angränsande ämnet, vilket kan leda till att argumentationen i
sin helhet faller. Samtliga dessa faror går att undvika
med rätt handledning, men problemet med att hitta rätt handledare är
uppenbart när det handlar om ämnen som eventuellt inte finns
representerade på fakulteten.

Detta är en av anledningarna till varför bihandledaren ofta fyller en
så viktig roll i konstnärlig forskning på ett sätt som den inte alltid
gör i vetenskaplig forskning.\footnote\{I KMH:s samarbete med KTH har
    jag blivit varse detta. Bihandledare är inte arvoderade i deras
    system och ses inte som tillnärmesivis lika viktiga i
    avhandlingsarbetet som huvudhandledaren.\} Bihandledaren kan vara
den personen som garanterar att ett angränsande ämnen får tillräckligt
stor roll och genomlysning i arbetets helhet och kan i vissa fall,
eller i vissa perioder av arbetet, framstå som projektets
huvudhandledare. Men det finns ytterligare en anledning till
bihandledarens betydelse som har att göra med den tidigare nämnda
svårigheten att hitta rätt handledarkompetens på fakulteten eller på
högskolan. Ännu är endast ett fåtal lärare på de konstnärliga
lärosätena disputerade (även om variationen här är stor mellan olika
lärosätena). Då det ofta finns ett behov av att huvudhandledaren och
doktoranden är på samma institution så kan en lösning vara att
huvudhandledaren blir mer av en institutionshandledare och att den
huvudsakliga handledningen sköts av bihandledarna. Även om detta inte
behöver vara problematiskt i sig kan det leda till obalans i hur
forskningsfältet utvecklar sig i relation till
andra. Handledarkompetens och seminarieverksamahet är uppenbart
centrala delar av en forskningsmiljö.

\section*{Den konstnärliga kunskapens dynamik}
\label{sec:orgeef416e}
Utvecklingen av konstnärlig forskning i Sverige kan ses utifrån minst
tre delvis överlappande processer. Den ena rör den
utbildningspolitiska aspekten av konstnärlig utbildning i
Bologna-modellen, men började ännu tidigare än så, i Sverige med
högskolereformen 1977.\footcite[Se t.ex. ][]{Lilja2015} Den pekade på att
alla högskolor skulle bygga på utbildning som är baserad på forskning
varför även de konstnärliga utbildningarna nu skulle bedriva
forskning.  Eftersom dessa bedriver konstnärlig undervisning eller
undervisning med konstnärliga metoder så måste de även bedriva
konstnärlig forskning -- detta kallades dock för konstnärligt
utvecklingsarbete snarare än forskning. I grunden ligger jämställandet
av konstnärlig och vetenskaplig forskning som nu ses som två uttryck
för kunskapsproduktion.\footnote\{Detta är givetvis en kraftigt
    förenklad bild.\}

Den andra processen är mer svårfångad men handlar om hur konst- och
kulturlivet i samhället har utvecklat sig under de senaste
decennierna. Det fält inom vilket konstnärliga uttryck diskuteras och
kommuniceras har för vissa uttryck, som musik, förändrats i mycket
stor grad. Dagspressens recensionsverksamhet, Public Service funktion
och det offentligas stöd till musiklivet har förändrats radikalt vilket har
skapat nya behov för ytor att diskutera och experimentera med
konstnärliga uttryck. Här har den konstnärliga forskningen börjat
fylla ett stort hål.

En tredje process rör en mer filosofiskt orienterad epistemologisk
fråga om vad kunskap kan ses vara, och hur den kan kommuniceras. En
vanlig, initial, invändning mot konstnärlig forskning, som diskuterades
ovan, är att något som i allt väsentligt är beroende av sinnesintryck,
som upplevelsen av konstnärligt uttryck kan sägas vara, inte kan
utgöra grunden för forskningsmässig kunskap. Även om denna invändning
vilar på en missuppfattning av såväl forskningsmässig kunskap som
konstnärlig kunskap så rör den vid en viktig grundförutsättning för
all kunskapsutveckling, nämligen att det finns grundläggande
förutsättningar som det går att enas omkring. Utan dessa blir det
omöjligt att etablera ett nytt forskningsfält.


\subsection*{Konstnärlig forskning som kunskap i praktiken}
\label{sec:praktiska-exempel}
Det är en utmaning att argumentera för konstnärlig kunskap i en samtid
som samtidigt präglas av en övertro på det vetenskapliga
kunskapssystemet, en missriktad postmodernistisk avart\footnote\{Jag
    menar att postmodernismen egentligen har mycket mer att lära oss
    och att det inte är den i sig som är problemet, men detta är en
    diskussion som går bortom ramarna för detta kapitel.\} där varken
rätt eller fel existerar och en hyperkapitalism som inte ser sina
gränser. Machiarini-fallet på KI\footcite{macchiarini2019} är ett
exempel på vad som händer när dessa tre samverkar och havererar. Jag
ska försöka peka på några områden där jag ser att konstnärlig
forskning kan komma att få stor betydelse om kunskapen hanteras på
ett adekvat sätt, men
jag argumenterar framförallt utifrån ett musikperspektiv och bilden
kan vara väsentligt annorlunda inom andra konstnärliga fält.

Det digitala har idag helt genomsyrat såväl produktion, distribution
som konsumtion av musik samtidigt som det digitala endast i liten
utsträckning är en teknik som präglas av specifikt konstnärlig
utveckling. Det betyder att produktionsverktyg, som mjukvara för
inspelning och redigering, samt program för uppspelning visserligen
för det allra mesta är anpassade för förutsättningarna för dessa
verksamheter, men det finns förhållandevis liten kunskap om de
faktiska konstnärliga processer som ligger bakom
användandet. Etableringen av artificiell intelligens visar ytterligare
på behovet av nya metoder för att bättre förstå vidden av förändringen
som vi är inne i. Även om internet inte är nytt var det inte många som
för tio år sedan hade trott att vi idag, genom våra telefoner,
trådlöst skulle få kontakt med en till synes oändlig samling musik,
bara för att ge ett exempel. Följden av detta är att en majoritet av
människor i västvärlden, själva designar sin egen ljudmiljö och i
någon mening, om man vill se det positivt, har tagit kontroll över
sitt eget musiklyssnande. Många av de verktyg som har möjliggjort
denna transformation är produkter av en ingenjörskonst på mycket hög
nivå. Spotify, till exempel tog utgångspunkt i teknik som vuxit fram i
en dunkel periferi under Pirate Bay, och gjorde delningen legal och
legitim. De fick ett tekniskt försprång och blev snart ensamma herre
på täppan. Detta är dock en utveckling som har skett helt och hållet
på kommersiella grunder och helt utan konstnärliga ambitioner. Själva
förutsättningen för Spotifys framgång var att man sänkte ersättningen
till de artister som spelades vilket kan ses som en makrovariant av
att såga av den gren man sitter på. Trots att detta företag inte på
något sätt kan ses vara en del av det konstnärliga musiklivet där musik
produceras talar man om Spotify som del av det svenska
musikundret.\footcite{konig2018}

Det här är ett uttryck för problemet som musik lider av: musiken som
konstform har aldrig gjort upp med musik \emph{som produkt} utan de
båda, musiken och produkten, är i allt väsentligt sammanvävda. Musiken
är helt och fullt kommodifierad och därför kan "det svenska
musikundret" samtidigt inkludera musikteknikföretag och Ann-Sofie von
Otter. Nu finns det naturligtvis extremt starka kommersiella krafter i
operabranschen, men skillnaden är den att von Otter hade aldrig kunnat
ta sig till den position hon har utan att hon hade otvetydiga
konstnärliga kvaliteter, medan teknikföretag som
Spotify kan utvecklas helt utan dessa att konstnärliga överväganden
styr utvecklingen.

Här finns en viktig plats för den konstnärliga forskningen och här kan
dess resultat utvärderas på en marknad långt utanför akademiens
skyddande väggar. I min avhandling från 2008 pekade jag på behovet att
inkludera en konstnärlig dimension när modeller för interaktivitet
skapas.\footcite{frisk08phd} Den bakomliggande idén var den att en
teknologiskt orienterad interaktivitet inte bara riskerar att bli en
grov förenkling av vad vi förväntar oss av en interaktiv
upplevelse (välj en färdig spellista istället för att skapa din egen). Resultatet kan dessutom bli att användarens
förväntningarna på tekniken sänks. Genom att istället se på
utvecklingen av interaktivitet utifrån vad konstnärer, musiker och
tonsättare vill få ut av den skapas en miljö där helt andra beslut tas
och de interaktiva möjligheterna expanderas snarare än begränsas.

Utifrån det komplexa och svårgreppbara fältet av artificiell
intelligens (AI) finns flera möjligheter för konstnärlig forskning att
spela en roll. Liksom i exemplet ovan, beträffande interaktivitet, är AI
ett område där antaganden om mänsklig interaktion spelar roll, men
även antaganden om vad mänsklig erfarenhet utgör. Dessutom ryms
otroligt svåra etiska frågor. Alla dessa kan angripas genom
konstnärliga experiment där det mest omedelbara är att genom
experiment utnyttja en teknologi för konstnärlig produktion. Ett
sådant projekt kan säga något både om teknologins stabilitet och
om hur konstnärlig aktivitet fungerar i interaktion med
teknik.\footnote\{I ett pågående projekt på Musikhögskolan i Oslo med
    titeln Goodbye Intuition under ledning av Ivar Grydeland jobbar vi
    med dessa och liknande frågor. En ej ännu publicerad artikel i
    Organised Sound beskriver detta: \emph\{Aesthetics, interaction and
	machine improvisation\} (Frisk 2019)\}

Genom metoder lånade från postkolonialismen har gruppen The Six Tones
under en rad år arbetat med interkulturella möten.\footnote\{Förutom
    mig själv består The Six Tones av Nguyen Thanh Thuy, Ngo Tra My
    och Stefan Östersjö. Se www.thesixtones.org för mer information.\}
Med det övergripande målet att överbrygga kulturella avstånd har vi
metodiskt arbetat konstnärligt med konserter, skivproduktion och
forskning. Målet i varje projekt har varit att komma fram till ett
övertygande konstnärligt resultat har vi inte strävat efter enkel
kunskapsöverföring utan efter ömsesidigt lärande, differens och
lyssnande. Genom en rad turnéer och skivinspelningar har vi haft
möjlighet att utvärdera praktiken i ett flertal sammanhang. Under 2020
kommer vi att arbeta med migrerade musiker i Sverige för att bland
annat försöka förstå hur och varför Sverige konsekvent har uteslutit
några dessa musikers kulturtraditioner. Några av de stora grupper som
har kommit till Sverige, som de från Iran och Irak och forna
Jugoslavien på 70- och 80-talen, är inte i allmänhet representerade
med sina musiktraditioner på konserthus, scener eller i
kulturpolitiska sammanhang. Konsekvensen är att en del av dessa
musiker befinner sig i ett konstnärligt ingenmansland, långt från sitt
hemlands levande traditioner, men samtidigt utan möjlighet att
etablera sig konstnärligt i sitt nya land. I projektet hoppas vi
bättre förstå dessa musikers traditioner, men också bygga upp en
kunskap som kan bidra till en förändring. Att arbeta med musik på liknande
sätt, med väl utarbetade metoder i ett forskningssammanhang är ett
sätt att utnyttja konstnärlig kunskap inom ett fält där Sverige i
dagsläget har svårt att politiskt, etiskt och kulturellt hitta rätt.


\section*{Diskussion}
\label{sec:diskussion}

Jag vill än en gång återkomma till den interrelation mellan konst,
vetenskap och filosofi som tre former för tänkande som Deleuze och
Guattari propagerar för i \citetitle{deleuze1994}, och som jag i någon
mening anser präglar undersökningen i \emph{Before Sound}. Det finns
en viktig poäng i deras resonemang som finner oväntad resonans i den
svenska högskoleförordningen och dess unika uppdelning mellan
konstnärlig och vetenskaplig grund och där dessa två kunskapsformer
betraktas som likvärdiga.
\% Filosofin ingår visserligen i det vetenskapliga i den svenska
\% modellen, till skllnad från hur Deleuze och Guattari ser den som
\% särskild.
Att se det konstnärliga och det vetenskapliga som två väsenskilda men
samtidigt kompletterande former för tänkande får självklart
konsekvenser för hur vi ska, eller kan se, på konstnärlig forskning
och valideringen av resultaten. Konsekvenser som kanske inte helt
utnyttjats i konstnärlig forskning. Det finns en tendens att se
forskning som forskning vilket är adekvat i många fall, men inte
alla. Även om vi till exempel kan peka på en konkret nytta med
konstnärlig forskning, en nytta som i någon mening är relaterad till
nyttoaspekten av en del vetenskaplig forskning, är det inte säkert att
detta ska vara modellen för all konstnärlig forskning. Istället behövs
en mer aktiv diskussion om vilken eller vilka modeller som ska
användas för kvalitetsbedömning av konstnärlig forskning: Vilken roll
har den konstnärliga praktiken och utifrån vilka kriterier ska den
bedömas? I vilka forum ska kvalitet i konstnärlig forskning
diskuteras? Hur bedöms metod och teori i konstnärlig forskning? Hur
bedöms \emph{impact} i konstnärlig
forskning?  Ytterligare en fråga som skulle behöva diskuteras mer för att hantera
relationen mellan konstnärlig forskning och konstnärlig praktik utan
forskning, är hur kunskapsöverföringen ska ske. Med andra ord, hur
kommer resultat av konstnärlig forskning hela det konstnärliga fältet
till del?
\%Samhällsnyttan är ett begrepp som är i konstant rörelse
\% och nära kopplat till de politiska och ekonomiska sfärerna genom vilka
\% definitionen av vad som är nyttigt bestäms i stor utsträckning. Det går därför inte att fastslå nyttigheten i en
\% forskningspraktik utan att lyfta blicken och se bredare på frågan.

Eftersom de flesta länder saknar den uppdelning mellan vetenskaplig
och konstnärlig grund, och eftersom det vetenskapliga forskningsfältet
är så dominant, finns det alltid en tendens att konstnärlig forskning
glider mot det vetenskapliga och blir en variant av vetenskaplig
forskning, snarare än ett unikt och självständigt sätt att resonera på. Men vi kan inte
mäta den konstnärliga forskningens värde utifrån vad den kan
åstadkomma \emph{som vetenskap}, helt enkelt för att det inte finns
någon större poäng med det. Hybridkonstellationer kan skapas, likt de
jag i alla korthet presenterar ovan, och dessa kan genom sitt
tvärdisciplinära angreppssätt bli ytterst effektiva sätt att utöka
kunskapen till angränsade fält. Men detta är, och ska vara ett
komplement till en konstnärlig forskning som i huvudsak befinner sig i
det konstnärliga fältet. Men oavsett hur mycket eller litet forskningen relaterar till
olika kunskapsområden som den rör sig omkring måste den ske i en aktiv
samverkan med annan forskning i samma fält och i aktiv samverkan med
sin omvärld, \emph{utan} att förlora de specifika bedömningsgrunder som
gäller för det konstnärliga fältet.

\% Flytta?
Ett möjligt sätt att behandla den konstnärliga forskningen utifrån
Deleuze och Guattaris modell i \citetitle{deleuze1994} skulle vara att
se den som ett kontaktnät mellan alla tre discipliner. Ungefär som en
transversal process som möjliggör kontakter från konsten till
filosofin och vetenskapen. På ett plan är det det som Stjerna gör i
\emph{Before Sound}. Det skulle samtidigt möjliggöra att konsten som
en form för tänkande är delvis oberoende av den konstnärliga
forskningen som här istället blir en \emph{möjligthet} för kontakt
mellan konsten, vetenskapen och filosofin.

För att ta ett exempel på hur svårt det kan vara att leva upp till de
vetenskapligt definierade kraven på \emph{impact} räcker det med att
återvända till relationen mellan filosofi och konst och betrakta de
tydliga paralleller mellan dessa två kunskapsområden. Båda har i
allmänhet setts som viktiga delar av (den västerländska)
kunskapsutvecklingen och båda har i någon mening förlorat sin
särställning och betydelse i samtiden. Dessutom så delar de båda ofta
en empiri som utgår från ett individuellt perspektiv. Det räcker med
att ytligt betrakta filosofin, som är en av våra äldsta
forskningspraktiker, för att snabbt konstatera att den har haft svårt
att hävda sin särställning. Lika otvetydigt är det att filosofi som
konceptskapande kraft har lagt grunderna till själva fundamentet för
vårt samhälle idag. Konsten har haft en lika uppenbar funktion för vår
kultur och har lika svårt att mäta sig som filosofin har haft.  \% varför vi behöver
\% konstnärlig forskning, snarare än bara konstnärlig praktik, men på den
\% frågan finns det ganska enkla och raka svar: för det första
\% behöver vi både konstnärlig forskning och praktik, tillsammans och
\% oberoende av varandra. För det andra är det genom forskningen som vi
\% har möjligheten att komma åt den verkligt intressanta kunskapen i
\% konstnärlig praktik som tidigare inte funnits tillgänglig.

För att illustrera vikten av att hålla fast i idén att konsten även
kan vara en distinkt kunskapskälla i sig själv lånar jag filosofen
Frank Jacksons berömda kunskapsargument. Det är ett tankexperiment
vars syfte det är att argumentera mot fysikalismen genom att visa att
det måste finns icke-fysiska aspekter av medvetandet. I grova drag går
det ut på att beskriva en kvinna som växer upp helt avskärmad från
omvärlden, i en helt svartvit miljö. Hon får aldrig uppleva färger,
men hon lär sig allt, precis allt, som finns att veta om dem. Hon lär
sig även om det neurofysiologiska och vad som händer när olika
vågländer av ljus träffar näthinnan. Hon kan föreställa sig hur det
fysiskt är att se färg, men har aldrig upplevt det. Vad händer när hon
får komma ut ur sitt rum och får uppleva färger på riktigt?  Lär hon
sig då något nytt? Finns det något utöver det fysiska att lära sig som
vi bara kommer åt genom upplevelsen?\footcite[s.130]{Jackson1982}
Jackson menar att det gör det och att medvetandet är mer än det
neurologiska. Jag drar en parallell till idén om det konstnärliga som
en kunskapsform som går utöver det som vetenskapen kan lära oss som
jag menar såväl Bateson som Deleuze ger visst stöd: Om vedertagna
vetenskapliga definitioner och allmän empiri ger oss de fakta som vi
behöver för att förstå en sida av vår tillvaro så konsten kan säga
någonting om är allt det andra, det vi inte omedelbart kan
se. Vetenskapen kan bara ge oss en del av sanningen och det finns
kompletterande fält av kunskap, grundade i etik och konst, och bygger
liksom vetenskapen på empiri. Dessa är nödvändiga för att vi ska kunna
skapa oss en så komplett bild som möjligt av tillvaron. Det finns det
goda chanser att konstnärlig forskning kan bidra till att vi får en
bättre förståelse och kunskap om dessa andra dimensioner. Exempel på
sådan kunskap, varav några redan har presenterats, kan vara forskning
på förkroppsligad kunskap, kommunikativa strategier och
interaktion. Alla dessa tre har varit fokus för flera konstnärliga
projekt och i alla dessa tre områden har det varit tydligt att den
konstnärliga sensibiliteten och undersökningen har kunna öppna upp för
en förståelse som endast svårligen hade kunna angripas på annat vis.

Samtidigt ska möjligheterna utnyttjas att inom ramarna för de
konstnärliga högskolornas forskningsverksamhet att utnyttja högskolan
som en plats för konstnärlig produktion. I takt med att utrymmet för
den experimentella konsten minskar i samhället så kan denna verksamhet
i större utsträckning äga rum på högskolorna: Konstnärlig forskning
som en ny arena för experimentell konst. Det är inte frågan om en
akademisering av konsten utan snarare det motsatta: låt konsten
påverka akademien. Konsten är i detta fall experimentell i den
bemärkelsen att den utmanar de gängse metoderna för såväl konst som
forskning även om inte resultatet i sig behöver vara experimentellt
rent stilistiskt. Högskolans egna scener som ett öppet fönster mot
omvärlden och ett utomordentligt bra sätt att bedriva samverkan på.
Det kan vara Barockmusik eller elektroakustisk musik, eller någon
annan genre.

Jag tror att frågan om vad konstnärlig forskning är, eller vilken
nytta den har, är felställd. Frågan vi borde ställa oss är snarare i
linje med: Hur kommer ett samhälle se ut som inte har aktivt arbetar
för att utveckla möjligheterna till att ta del av, och utveckla, alla
kunskapsformer som påverkar den mänskliga tillvaron, inklusive
konstnärlig sensibilitet och filosofiskt resonerande?  Sett till den
frågan blir det klart att detta måste vara en fråga som diskuteras på
bred front i samhället. Det är ingen tvekan om att vi lever i ett
vetenskapssamhälle först och främst och att det i sin tur skapat ett
tekniksamhälle. Om det finns övertygande argument för att vetenskapen
inte behöver konsten och filosofin så är jag beredd att tänka
om. Tills dess menar jag att allt pekar på att vi behöver tillgång
till alla typer av tänkande vi kan komma åt för att lösa de utmaningar
vi står inför i världen idag och att just tänkande genom konstnärlig
praktik är ett mycket användbart verktyg. Det visar inte minst Åsa
Stjerna med \citetitle{Stjerna2018}.

\printendnotes
\printbibliography
\end{document}
.
\end{document}